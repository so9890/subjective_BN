\section{Introduction}

\paragraph{Motivation/ Real life setting}
Due to climate change and vital threats to biodiversity we need to reduce the consumption of resources. % why this?/ Resource consumption= to emissions??? 
Macroeconomic research has largely focused on a green recomposition of consumption. However, doubt has been raised if recomposition alone is sufficient to fight climate change. An alternative and most likely complementary approach that has been proposed is a reduction of economic output. %consumption levels. 
In light of the uncertainty about the success of a pure recomposition approach, it is time to ask: what are the effects of a reduction in consumption? 

\paragraph{Hypotheses/ why a macro model?}
More specifically, I am interested in the political conflicts which may arise from a reduction in economic output. For this purpose, there is heterogeneity on the household side. 
Households differ with respect to the sectors they are employed in and the ownership of capital. As a result, household income varies which again determines the emission-quality of the households consumption bundle.  




\paragraph{This paper/ results}
I use a general equilibrium framework in which the level of economic productivity is determined by demand. 

The focus rests on general equilibrium effects on the environmental externality and on the social acceptance of 


\paragraph{How to achieve a reduction in economic output?}
An essential contribution of the paper is to compare the effects of different policy measures to achieve a sufficient reduction in emissions. 