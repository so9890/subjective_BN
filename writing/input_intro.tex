\section{Introduction}

\paragraph{Motivation/ Real life setting}
Due to climate change and vital threats to biodiversity we need to reduce the consumption of resources. % why this?/ Resource consumption= to emissions??? 
Macroeconomic research has largely focused on a green \textit{recomposition} of consumption. However, doubt has been raised if recomposition alone suffices to fight climate change. An alternative and most likely complementary approach that has been proposed is a \textit{reduction} of economic output \citep[e.g.][]{Dasgupta2021, GoughCANGREEN, Naqvi2017FiftyPollutants}. %consumption levels. 
In light of the uncertainty about the success of a pure recomposition policy and the urgency to act, it is time to ask: what are the effects of a reduction in economic output? 

\paragraph{Hypotheses/ why a macro model?}
More specifically, I am interested in two aspects. First, the paper serves to learn about the political conflicts which may arise from a reduction in economic output. For this purpose, there is heterogeneity on the household side. 
Households differ with respect to the sectors they are employed in and the ownership of bonds. As a result, household income varies and will be affected differently by a reduction in output. 

Assume the reduction in output is implemented by a reduction in demand. First, due to the excess supply firms lay off workers. This might lower labour income affecting poor households more severely thereby increasing inequality and social tension.   

On the other hand, I want to learn about the indirect effects a reduction in output may have on the externality. 

To this end, I integrate  endogenous, directed technolgical innovation and monopolistic competition (COULD BE INTRINSIC TO ENDOG GROWTH! YES IT IS \ar build on \cite{Acemoglu2012TheChange}) into the model.


In order to satisfy basic needs, these households may increase their sustainable demand at the expense of sustainable consumption. This channel mitigates the effect of a reduction in demand on the externality. 
Second, endogenous and directed innovation might be negatively affected by the reduction in demand as it becomes less profitable. Again, this might slow down the recomposition of output. However, this channel might lower returns to investment which again only affects income of the rich. 
It is hence a priori unclear what effect a reduction in demand has on inequality and the externality calling for an analysis in a general equilibrium model. 

\paragraph{Qualitative versus quantitative model?}
A quantitative model would be needed if I was interested in numbers which I would want to compare to the data and take more seriously. 
I think I am more interested in understanding the conditions (parameter values etc.) under which a reduction is beneficial to the poor or to the rich. And also under what circumstances a reduction in output might be impeded by other mechanisms in lowering resource usage (endogenous innovation, general equilibrium effects through prices (and \textit{subjective} basic needs)). Finally, accompanying policies to support a reduction in economic output should be studied. 


\paragraph{This paper/ results}
I use a general equilibrium framework in which the level of economic productivity is determined by demand. In this setting I study the effects of two disting policy interventions: a reduction in working time, and second, a reduction in demand (be it intrinsically or a policy intervention). 

The focus rests on general equilibrium effects on the environmental externality and on the social acceptance of alternative policies. 

\paragraph{Potential consequences of the paper and how it could lead to learn about the ability of the market to comply with climate goals}

\textit{I might find, that first, under the assumption that growth is too slow, and a reduction in consumption could become necessary/ or as a consequence (externality in production function) might impede technological change so much that the market solution is not feasible \ar more subsidies are needed}
\paragraph{Data}
Two aspects are crucial for the mechanisms scrutinised in this paper. (1) First, the distribution of labour across the green and non-green sector. How are income and employment related? \textit{What data is available here? Is it sufficient to draw from the existing literature here, e.g. }
(2) Second, the consumption of environmentally friendly goods is non-homothetic. However, not only the share of a certain products is relevant, say the budget share of energy, but also the quality: sustainable versus non-sustainable energy sources.
Most empirical studies on resource consumption only capture differences in the type of products consumed due to data availability. \citep{Sager2019IncomeCurves}

\paragraph{How to achieve a reduction in economic output?}
An essential contribution of the paper is to compare the effects of different policy measures to achieve a sufficient reduction in emissions. 

I analyse, first, an intrinsic reduction in demand, second, a policy-induced reduction in demand, and third, a policy-induced reduction in working time.

\paragraph{Literature}
The paper relates to several strands of literature. First, to papers discussing policies in a climate-externality context. 
Second, macro models with demand-determined output level.
Third, political economy. 
Fourth, degrowth. 
Fifth, endogeneous innovation. 