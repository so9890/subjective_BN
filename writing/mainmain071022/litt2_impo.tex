\paragraph{Literature}

%The paper relates to several strands of literature. Second, it connects to the literature connecting environmental and fiscal policies and how to recycle environmental tax revenues. Third, as the paper combines environmental and fiscal policies it naturally connects to the public finance literature. Finally, the results speak to the literature discussing inefficiently high production.
 
%\begin{itemize}
%	\item How to use environmental tax revenues \citep{Fried2018TheGenerations}
%	\item Optimal environmental policy \ar focuses on environmental taxes
%	\item weak double dividend
%	\item to be incorporated: \tr{\cite{Metcalf2003EnvironmentalPollution} why does he find that the optimal pigou tax equals first best when gov spending is satisfied with tax revenues? }
%	\\
%	Williams III: Welfare improvement with xxx \citep{Parry1999WhenMarkets} \tr{is this weak or strong dd?}
%\end{itemize}

%---------------------------------------
%.. optimal environmental policy
%---------------------------------------
The paper relates to three strands of literature. 
Firstly, the paper speaks to the literature on environmental policy in endogenous growth models.
In general, these papers focus on environmental taxation and analyze settings with inelastic labor supply so that there is no role for labor income taxes in stabilizing or reducing production.

 \cite{Golosov2014OptimalEquilibrium} investigate the optimal carbon tax in a dynamic stochastic general equilibrium model.  
\cite{Acemoglu2012TheChange} discuss the optimal environmental policy in a tractable model of directed technical change.
They highlight the need for green research subsidies to foster green innovation in combination with carbon taxes. Research subsidies serve to correct for the dynamic spillovers of green innovation not internalized by the research sector. They discuss that under a second-best policy, when no subsidies are available, the carbon tax needs to be higher to redirect research. The present paper's exercise 
directly connects to this case. In my framework, there is no research subsidy, and the government takes into account the effect of the carbon tax on the direction of research. In this scenario, I highlight the need for an additional policy measure\textemdash labor income taxes\textemdash to correct the level of economic activity when research subsidies are lacking. 
\cite{Fried2018ClimateAnalysis} extends the framework of the aforementioned paper to a quantitative model. % mainly by introducing cross-sectoral knowledge spillovers and diminishing returns to research. 
She finds that a constant emission limit can be met at a lower carbon tax when growth is endogenous. % To this discussion, I add the adverse effect of knowledge spillovers which make a continuous increase in carbon taxes necessary. The result emerges in my set-up as I allow for a dynamic policy in contrast to \cite{Fried2018ClimateAnalysis}. 


% OVERVIEW LITERATURE
% Acemoglu 2016 have lump-sum transfers and taxes
% Acemoglu Aghion 2012: lump-sum transfers, no optimal policy
%Golosov: hightlight the need of lump-sum transfers! but exogenous labor supply
% Therefore, the main finding of the present paper, the necessity of reductive policy measures to implement the efficient allocation, complements this literature. 
%Especially, when environmental tax revenues are not redistributed lump-sum in these papers, a variable labor supply would give an argument for labor income taxation. 


%%\paragraph{Endogenous growth, elastic labor supply and optimal environmental policy}
%Staying within the field of endogenous growth, secondly, the paper connects to work examining the interaction of directed technical change and skill heterogeneity. \cite{Acemoglu2002DirectedChange} develops a theory to explain the positive correlation of skill supply and the skill premium: the higher supply of skilled labor raises incentives to innovate in the skill sector. \cite{Loebbing2019NationalChange} introduces fiscal policy into the model to investigate how the equalizing effect of redistribution is amplified through directed technical change. %Similar to my paper, a higher tax progressivity changes the relative supply of skills. As low-skill labor is in relative higher supply, low-skill-specific innovation depresses the wage distribution thereby contributing to equity. While the channels are comparable,
%% I evaluate the effect of income tax progressivity and endogenous growth on emissions. 
%%\cite{Hemous2021DirectedEconomics} provide an overview of models of directed technical change in environmental economics. They argue that a rise in the skill ratio directs innovation towards skill-intense technology when the high- and the low-skill output good are sufficient substitutes. Furthermore, when the two input goods are substitutes, the more advanced sector attracts more innovation. 
%My paper contributes to these two branches by integrating endogenous and heterogeneous skill supply in an environmental model of directed technical change. These ingredients enable me to analyze labor income taxes through the lens of environmental policies.  
%
%content...


%\cite{Oueslati2002EnvironmentalSupply} studies the optimal environmental policy with elastic labor supply and endogenous growth. Yet, he allows for lump-sum transfers of environmental revenues. \textit{He should find something on reduction of hours}: No: capital is the only polluting factor, and labor is the clean factor of production.


%%%--------------------------------------------------------------------
% How to recycle environmental tax revenues: weak double dividend
%%%-------------------------------------------------------------------- 
\
% A big literature has examined potential benefits arising from corrective tax revenues to ameliorate fiscal distortions. The double-dividend literature is concerned with fiscal advantages arising from environmental tax revenues. My results speak directly to the weak double dividen hypothesis which 
%My paper most closely relates to the literature on the weak double-dividend
%\paragraph{Recycling environmental tax revenues}
%\tr{Read:\cite{Freire-Gonzalez2018EnvironmentalReview} yet on strong dd, I assume}
Secondly, this paper is not the first to integrate distortionary fiscal policies into the analysis of environmental policies. However, the role of fiscal policies has in general been passive, and the focus rested on the impact of preexisting fiscal distortions on the environmental policy. One example is the literature focusing on potential double dividends of environmental taxes by generating government revenues \citep[][]{Goulder1995EnvironmentalGuide, Bovenberg2002EnvironmentalRegulation}. 
%Another focal point of the discussion on environmental policy is the redistribution of environmental tax revenues. This literature generally assumes labor supply to be elastic and incorporates fiscal policies. 
%The dominant focus of this literature are fiscal advantages from environmental policies. One aspect is the so-called weak double dividend of environmental taxes : given an exogenous government funding constraint it is cost saving to recycle environmental tax revenues to lower distortionary labor income taxes as opposed to higher lump-sum transfers.
 %The latter decreases labor supply through an income effect thereby lowering the tax base of the labor income tax. %Consequently, it becomes more expensive for the government to generate revenues.
%Therefore, this literature advocates recycling environmental tax revenues through a reduction of distortionary fiscal taxes as opposed to lump-sum rebates.
%In relation to this literature, my paper stresses the role for the environmental policy of lump-sum transfers in reducing economic activity. When lump-sum transfers deviate from their first-best level, for example, because the environmental tax additionally focuses on research, then a role for distortionary labor taxes emerges. Interestingly, the motive follows from the environmental externality per se. This channel contrasts the idea to exclusively use environmental tax revenues to lower distortionary income taxes. Some reduction in economic activity might be optimal. 
The present paper closely relates to \cite{Barrage2019OptimalPolicy} who examines the role of fiscal distortions on the environmental policy in a quantitative framework. She also optimizes jointly over fiscal and environmental policy instruments, but her focus rests on the deviation of the optimal environmental tax from the social costs of carbon.


%% my contribution : 1) lower bound on dist income tax; 2) motivation and role of income taxes
%One of this paper's contributions is to discuss the existence of an upper bound on the reduction of distortionary income taxes: the weak double-dividend literature rests on the assumption that no lump-sum transfers and taxes are available. When this is the case, this paper shows that absent a government funding constraint, distortionary taxes should be set to reduce labor supply: some reduction in hours is in fact efficient from an environmental perspective. However, a shrinking labor supply is generally perceived as an inefficiency in this literature and environmental taxes on its own as efficient. The importance of lump-sum transfers to implement the efficient allocation receives less attention.  %To the best of my knowledge, the papers theoretically discussing the weak double-dividend \citep{LansBovenberg1996OptimalAnalyses, Goulder1995EnvironmentalGuide} do not formally derive the result; a possible explanation for why the lower bound on distortionary tax reduction remained unnoticed. 
%% The primary distinction of this paper and the weak double-dividend literature is the motive for income taxation. In this literature an exogenous funding constraint motivates the use of distortionary income taxes. In contrast, my model rationalizes a progressive income tax absent an exogenous revenue constraint arising in an otherwise equal set-up from the environmental externality per se as the environmental tax on its own does not establish the efficient allocation. 
%
%
%content...



 %This becomes clear when environmental tax revenues suffice to meet the government's funding constraint, then labor supply would be inefficiently high when the labor income tax is unused. 
%In contrast to the present paper, the double-dividend literature focuses on non-environmental cost advantages of environmental taxation either via interactions with other taxes and their bases or via their revenues. However, it remains unmentioned that under the assumption of elastic labor supply, which the literature necessarily assumes, the environmental tax alone is not efficient.


%----------------------------------------
%---- optimal revenue recycling 
%---- empirical and quantitative-------
%----------------------------------------
%The question of how to use environmental tax revenues has seen a surge in interest recently and diverged from the prominence of fiscal advantages. 
%My contribution to this debate is to point to lump-sum redistribution to constitute an integral part of an efficient pollution mitigation. % together with corrective taxes.
% %They do not constitute a free lunch if efficiency is the goal.\footnote{ Often, environmental taxes alone seem to be perceived as being able to implement the efficient allocation: \cite{LansBovenberg1999GreenGuide} writes "\textit{Environmental taxes are  generally  an  efficient  instrument  for  protecting  the  environment.}" thereby neglecting the role environmental tax revenue redistribution. Or "\textit{Establishing a price on carbon [...] is well understood to be the most efficient approach for reducing greenhouse gas emissions.}" \citep{Fried2018TheGenerations}. } 
%The advantage of different recycling means is often assessed using inter- and within generational equity or political feasibility as value measures \citep{Carattini2018, Goulder2019IncomeGroups, VANDERPLOEG2022103966, Kotlikoff2021MakingWin, Carbone2013DeficitImpacts}. Building on the weak double-dividend literature, \cite{Fried2018TheGenerations} compare distinct recycling scenarios investigating the impact on inequality in an overlapping generations model. 
%% They find lump-sum transfers to be preferred by the  living generation.
%  \cite{VANDERPLOEG2022103966} find an equity advantage of lump-sum transfers using German data. % The authors suggest the government to split environmental tax revenues to both lower preexisting tax distortions and as lump-sum transfers. %The present paper employs the first-best allocation as a benchmark to assess distinct recycling methods.
%
%content...


%\paragraph{Public finance}
Thirdly, the paper contributes to the public finance literature.
An equity-efficiency trade-off is central to this literature.  The benefits of distortive labor taxes arise from redistribution and generating government revenues. 
%With concave utility specifications full redistribution is efficient. However, the optimal tax system does not feature full redistribution when labor supply is endogenous. Instead, redistribution is traded off against aggregate output as individuals reduce their labor supply and skill investment in response to labor income taxation 
\citep{Heathcote2017OptimalFramework, Conesa2009TaxingAll, Domeij2004OnTaxes}.
To this literature, I add another motive for the use of distortionary fiscal policies: adjusting the level of economic activity as part of the optimal environmental policy. 
%One closely related work is \cite{Loebbing2019NationalChange} who studies optimal income taxation in a model of directed technical change. The redistributive effect of tax progressivity is amplified through a compression of the wage rate distribution \textit{to be continued}

	\paragraph{Outline}
The remainder of the paper is structured as follows. Section \ref{sec:mod_an} presents the core model and the analytical results. In Section \ref{sec:model2}, I extend and calibrate the model to a quantitative framework.  Results are discussed in Section \ref{sec:res}. Section \ref{sec:con3} concludes.