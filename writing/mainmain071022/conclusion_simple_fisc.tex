\section{Conclusion}\label{sec:con}
%Some scholars argue that  reductive policies are necessary to handle environmental limits \citep{Schor2005SustainableReductionb, VanVuuren2018AlternativeTechnologies, Bertram2018TargetedScenarios}, and the question has been raised whether consumption is too high \citep{Arrow2004AreMuch}. On the other hand, the focus of environmental policy discussions in economics rests on corrective environmental taxation. 

The latest IPCC report \citep{IPCC2022} stresses the need to transition to a net-zero emission limit. The economics literature has largely focused on carbon taxes to reduce emissions. However, labor income taxes may contribute to lowering emissions through changing the level of production. I ask what is the optimal fiscal mix of taxes on carbon and labor income in a transition toward net-zero emissions?

%In the analytical part of the paper, I show in a simple model that the optimal environmental policy features a reductive policy element: lump-sum transfers of environmental tax revenues. When lump-sum transfers are not available, progressive labor income taxes act as a complement to the environmental tax. The intention is to mitigate distortions on the labor market arising from the lack of lump-sum rebates. %The model does not feature inequality.
% Quantitative results
% baseline model
%When environmental tax revenues are not redistributed lump sum, labor supply is inefficiently high. Then, income taxes serve to diminish hours worked closer to the efficient level. The result prevails absent income inequality.

 I build an endogenous growth model in which an emission limit renders fossil energy socially costly. I find that the optimal policy always chooses a combination of carbon and labor income taxes. Once the net-zero emission limit binds, the optimal policy taxes carbon extensively. A subsidy on labor serves to stabilize production.
 The rationale is a distortion in the labor market as the carbon tax is chosen to also discourage fossil research. This reduces the wage rate.  A subsidy on labor mitigates this side effect of the carbon tax.
 
  In the short run, the policy implications differ.  
  The carbon tax should rise less. This allows the economy to benefit from fossil research. However, due to the lower carbon tax the wage rate does not fully capture the social costs of work through emissions. Labor taxes then optimally reduce economic activity and emissions.
  
 Higher fossil growth in early periods is optimal only since knowledge spills to the green sector in the future. Then, a more equal distribution of researchers across sectors allows to circumvent decreasing returns to research. Absent knowledge spillovers, green research has to be fostered from the beginning. The optimal policy, therefore, sets a higher carbon tax, and labor is subsidized throughout. Overall, however, knowledge spillovers to the fossil sector make a more aggressive environmental policy necessary. 
 
% The paper intends to highlight an alternative motive for labor income taxation: stimulating or curbing the level of economic activity. It, therefore, abstracts from inequality. Nevertheless, it would be interesting to study the same mechanism in a more comprehensive model which accounts for income inequality. This would help to relate the importance  of the environmental and the equity motive for optimal income taxation.  
 
\clearpage
% extensions
%In an extension, I plan to give the government the opportunity to limit working hours. The literature advocating a reduction in consumption levels \citep[e.g.,][]{Schor2005SustainableReductionb} proposes a restriction of hours worked as policy instrument to lower the consumption of resources. Even though discussed in the literature, there is evidence for political difficulties in reducing working hours. In 2020, the French Citizens' Convention on Climate voted against reducing working hours to handle climate change. Potentially, ignorance of economic consequences is an explanation. %The extension would serve to better understand these consequences.

%Thirdly, it would be inter´ndant. % Then again, subsidies may boost labor demand aggravating the inefficiency in working hours. 
%Secondly, the model abstracts from income inequality and government funding constraints which constitute traditional motives for income taxation.  Integrating these aspects into the model 
