\section{Quantitative results}\label{sec:res}

This section presents and discusses the quantitative results. 
In Section \ref{subsec:exp}, I use the model to learn how a constant carbon tax affects the economy and how it interacts with a tax on labor income. Section \ref{subsec:meetlim} calculates how high a carbon tax is necessary to meet the emission limit. Most importantly, I find that an increasing carbon tax is necessary to counter market forces directing production and research towards the fossil sector. 
Section \ref{subsec:mr} goes one step further asking  how the government can optimally satisfy the emission limit using carbon and labor income taxes. Results show that a combination of the two instruments is optimal throughout. 

%I focus on analyzing the mechanisms and welfare benefits from integrating the income tax scheme into the environmental policy. I also discuss the costs of not using lump-sum transfers.

\subsection{Results }
This section depicts the environmental tax which is required to meet the emission limits and the evolution of key variables under distinct policy regimes. The income tax scheme is kept fixed at its calibrated levels: $\tau_{\iota t}=0.181$, $\lambda=0.43$.
I consider the following regimes: first, environmental tax revenues are redistributed lump sum to households. Second, the government runs a consolidated budget and environmental tax revenues are redistributed via the income tax scheme; that is, $\lambda$ adjusts. Third, environmental tax revenues are recycled as subsidies to the green sector. Fourth, the government consumes environmental tax revenues. 



\subsection{Optimal policy}\label{subsec:mr}


This section seeks to answer the question how a benevolent planner optimally attains the emission limit. After showing the results in Section \ref{sec:optres}, I discuss the intention behind the optimal policy in Section \ref{subsec:dis}. The optimal policy consists of a combination of labor income and carbon taxes to implement the emission limit. The reason is that when the carbon tax is used to target the direction of research, this causes distortions in the labor market. The labor tax boosts or curbs labor supply to counter these distortions. 
%This section depicts results on the optimal policy followed by the implied allocation in the benchmark model where environmental tax revenues are redistributed via the income tax scheme. 

\subsubsection{Results}\label{sec:optres}
Figure \ref{fig:optPol} depicts the optimal policy.
To meet the emission limits suggested by the IPCC, the optimal policy is to tax labor until 2044; see Panel (a). The labor tax becomes a subsidy from 2045 onward, $\tau_{\iota t}<0$. 
\vspace{3mm}
\begin{figure}[h!!]
	\centering
	\caption{Optimal policy }\label{fig:optPol}
	\begin{subfigure}{0.4\textwidth}
		\caption{Average marginal income tax rate }
		%	\captionsetup{width=.45\linewidth}
		\includegraphics[width=1\textwidth]{../../codding_model/own_basedOnFried/optimalPol_010922_revision/figures/all_13Sept22_Tplus30/dTaulAv_OPT_T_NoTaus_COMPtaul_regime4_spillover0_knspil0_noskill0_sep0_xgrowth0_PV1_etaa0.79_lgd0.png}
	\end{subfigure}
\begin{minipage}[]{0.1\textwidth}
	\
\end{minipage}
	\begin{subfigure}{0.4\textwidth}
		\caption{Tax per ton of carbon in 2022 US\$ }
		%	\captionsetup{width=.45\linewidth}
		\includegraphics[width=1\textwidth]{../../codding_model/own_basedOnFried/optimalPol_010922_revision/figures/all_13Sept22_Tplus30/Single_periods12_OPT_T_NoTaus_Tauf_regime4_spillover0_knspil0_noskill0_sep0_xgrowth0_extern0_PV1_sizeequ0_GOV0_etaa0.79.png}
	\end{subfigure}
%\floatfoot{Notes: \footnotesize{The x-axis indicates the first year of the 5 year period to which the variable value corresponds. }}
\end{figure} 
Consider Panel (b). The optimal carbon tax increases over time and jumps to a higher level when the net-zero emission limit is introduced in 2050.
In 2020, the carbon tax equals US\$987 and rises steadily to US\$1,325 in 2045.  As the emission limit declines to net-zero in 2050, the tax rapidly surges to US\$2,833 and gradually increases afterwards reaching US\$3,278 in 2075. 
\paragraph{Efficient and optimal allocation}\label{subsec:notaul}

Figure \ref{fig:optAll_percLf_dyn} depicts the adjustments of key variables under the first-best (efficient) and the second-best (optimal) policy relative to the laissez-faire allocation.\footnote{\ I formulate the social planner's problem in Appendix \ref{app:sp_prob}.  Figure \ref{fig:LF} in Appendix \ref{app:quant_res_opt} shows the laissez-faire, the efficient, and the optimal allocation in levels.} 
The efficient allocation can be perceived as the allocation the Ramsey planner seeks to implement. However, she may not be able to achieve the efficient allocation due to the reliance on a limited number of tax instruments.

\begin{figure}[h!!!]
	\centering
	\caption{Efficient and optimal allocation in percentage deviation from laissez-faire	}\label{fig:optAll_percLf_dyn}
	\begin{subfigure}[]{0.4\textwidth}
		\caption{Consumption}
		%	\captionsetup{width=.45\linewidth}
		\includegraphics[width=1\textwidth]{../../codding_model/own_basedOnFried/optimalPol_010922_revision/figures/all_13Sept22_Tplus30/C_PercentageLFDyn_Target_regime4_knspil0_spillover0_noskill0_sep0_xgrowth0_PV1_etaa0.79_lgd1.png}
	\end{subfigure}
\begin{minipage}[]{0.1\textwidth}
	\ 
\end{minipage}
	\begin{subfigure}[]{0.4\textwidth}
		\caption{High-skill hours worked}
		%	\captionsetup{width=.45\linewidth}
		\includegraphics[width=1\textwidth]{../../codding_model/own_basedOnFried/optimalPol_010922_revision/figures/all_13Sept22_Tplus30/hh_PercentageLFDyn_Target_regime4_knspil0_spillover0_noskill0_sep0_xgrowth0_PV1_etaa0.79_lgd0.png}
	\end{subfigure}

\vspace{3mm}
	\begin{subfigure}[]{0.4\textwidth}
		\caption{Low-skill hours worked}
		%	\captionsetup{width=.45\linewidth}
		\includegraphics[width=1\textwidth]{../../codding_model/own_basedOnFried/optimalPol_010922_revision/figures/all_13Sept22_Tplus30/hl_PercentageLFDyn_Target_regime4_knspil0_spillover0_noskill0_sep0_xgrowth0_PV1_etaa0.79_lgd0.png}
	\end{subfigure}
\begin{minipage}[]{0.1\textwidth}
	\ 
\end{minipage}
\begin{subfigure}[]{0.4\textwidth}
\caption{Green-to-fossil output}
%	\captionsetup{width=.45\linewidth}
\includegraphics[width=1\textwidth]{../../codding_model/own_basedOnFried/optimalPol_010922_revision/figures/all_13Sept22_Tplus30/GFF_PercentageLFDyn_Target_regime4_knspil0_spillover0_noskill0_sep0_xgrowth0_PV1_etaa0.79_lgd0.png}
\end{subfigure}

\vspace{3mm}
%	\begin{subfigure}[]{0.4\textwidth}
%		\caption{Non-energy scientists}
%		%	\captionsetup{width=.45\linewidth}
%		\includegraphics[width=1\textwidth]{../../codding_model/own_basedOnFried/optimalPol_010922_revision/figures/all_13Sept22_Tplus30/sn_PercentageLFDyn_Target_regime4_knspil0_spillover0_noskill0_sep0_xgrowth0_PV1_etaa0.79_lgd0.png}
%	\end{subfigure}
	\begin{subfigure}[]{0.4\textwidth}
		\caption{Scientists}
		%	\captionsetup{width=.45\linewidth}
		\includegraphics[width=1\textwidth]{../../codding_model/own_basedOnFried/optimalPol_010922_revision/figures/all_13Sept22_Tplus30/S_PercentageLFDyn_Target_regime4_knspil0_spillover0_noskill0_sep0_xgrowth0_PV1_etaa0.79_lgd0.png}
	\end{subfigure}
\begin{minipage}[]{0.1\textwidth}
\ 
\end{minipage}
\begin{subfigure}[]{0.4\textwidth}
\caption{Non-energy scientists}
%	\captionsetup{width=.45\linewidth}
\includegraphics[width=1\textwidth]{../../codding_model/own_basedOnFried/optimalPol_010922_revision/figures/all_13Sept22_Tplus30/snS_PercentageLFDyn_Target_regime4_knspil0_spillover0_noskill0_sep0_xgrowth0_PV1_etaa0.79_lgd0.png}
\end{subfigure}

%	\begin{subfigure}[]{0.4\textwidth}
%		\caption{Energy share in GDP}
%		%	\captionsetup{width=.45\linewidth}
%		\includegraphics[width=1\textwidth]{../../codding_model/own_basedOnFried/optimalPol_010922_revision/figures/all_13Sept22_Tplus30/EY_PercentageLFDyn_Target_regime4_knspil0_spillover0_noskill0_sep0_xgrowth0_PV1_etaa0.79_lgd0.png}
%	\end{subfigure}
	\floatfoot{Notes: \footnotesize{ The figure shows the percentage deviation of the allocation resulting under the integrated policy, that is, the progressivity of the income tax can be chosen (the black solid graph), the optimal allocation in the separate regime when no progressive income tax is available (the gray dashed graph), and the efficient allocation (the orange dotted graph), relative to the laissez-faire allocation. 
	}}
\end{figure} 
%
%\clearpage
%%
% Labor supply
The social planner attains the emission limit while increasing consumption and decreasing labor supply; Panels (a) to (c) in Figure \ref{fig:optAll_percLf_dyn}. This allocation is achieved by more research in all sectors (Panel (e)). Furthermore, the social planner increases the share of energy scientists of which a larger number is allocated to the fossil sector. 
Under the optimal policy, in contrast, consumption reduces relative to the laissez-faire allocation since the government lacks sufficient instruments to raise research while meeting the emission limit. The number of scientists reduces relative to the laissez-faire allocation, and the number fossil scientists remains low; consider Figure \ref{fig:LF}.

The social planner can sustain  high growth rates and simultaneously meet the emission limit by choosing a lower energy share to GDP and a higher ratio of green-to-fossil energy. 
In the competitive economy, a rise in fossil research has to be regulated via demand. Hence, a trade-off between research and emission mitigation occurs, and implementing the emission limit is costly in terms of R\&D investment and growth. 

%%%%%%%%%%%%%%%%%%%%%%%%%%%%%%%%%%%%%%%%%%%%%%%%%%%%%%%%%%%%%%%%%%%%%%%%%%%%%%%%%%%%
%% DISCUSSION 
%%%%%%%%%%%%%%%%%%%%%%%%%%%%%%%%%%%%%%%%%%%%%%%%%%%%%%%%%%%%%%%%%%%%%%%%%%%%%%%%%%%%

\subsubsection{Discussion}\label{subsec:dis}

 What explains the optimal policy? To answer this question, I look at how the optimal allocation with labor income tax differs from the optimal allocation when no income tax is available. 
 The subsequent section studies a counterfactual experiment where only the optimal carbon tax is implemented. Comparing the resulting allocation to the optimal allocation sheds light on the specific role of the income tax. 


	
\paragraph{Comparison to separate policy regime}

Figure \ref{fig:opt_TLs} shows percentage deviations of the allocation under the policy regime with income tax, henceforth \textit{integrated} regime, relative to a \textit{separate} regime where no labor income tax can be chosen; i.e., $\tau_{\iota t}=0$.  
Panel (a) depicts the carbon tax which is smaller under the integrated policy regime up to 2045. Afterwards, the carbon tax is higher than in the separate regime. The higher carbon tax falls together with a subsidy on labor.
Labor income taxes and carbon taxes act as substitutes. 

%The gain of the integrated policy is higher non-energy productivity  during initial periods until 2060  and higher green growth during the net-zero emission periods (Panels (c) to (e)).

Utility gains of the use of an income tax occur during the initial periods up to 2035 (Panel (b)). 
The household profits from more leisure while technology growth from more research in the fossil and non-energy sector keep consumption high; see Panels (c) and (d). 
To achieve the higher technology levels, the planner accepts less green growth and a smaller green-to-fossil ratio today. % which result in lower utility levels tomorrow. %
The costs of the integrated policy is borne in future periods (Panel (b)). 
\begin{figure}[h!!!]
	\centering
	\caption{Deviation from optimal policy with only a carbon tax}\label{fig:opt_TLs}
	\begin{subfigure}{0.4\textwidth}
		\caption{ Carbon tax}
		%	\captionsetup{width=.45\linewidth}
		\includegraphics[width=1\textwidth]{../../codding_model/own_basedOnFried/optimalPol_010922_revision/figures/all_13Sept22_Tplus30/Tauf_OPT_T_NoTaus_COMPtaulPer_regime4_spillover0_knspil0_noskill0_sep0_xgrowth0_PV1_etaa0.79.png}
	\end{subfigure}
\begin{minipage}[]{0.1\textwidth}
\
\end{minipage}
	\begin{subfigure}{0.4\textwidth}
	\caption{Period utility}
	%	\captionsetup{width=.45\linewidth}
	\includegraphics[width=1\textwidth]{../../codding_model/own_basedOnFried/optimalPol_010922_revision/figures/all_13Sept22_Tplus30/SWF_OPT_T_NoTaus_COMPtaulPer_regime4_spillover0_knspil0_noskill0_sep0_xgrowth0_PV1_etaa0.79.png}
	\end{subfigure}

\vspace{3mm}
	\begin{subfigure}{0.4\textwidth}
		\caption{Fossil scientists}
		%	\captionsetup{width=.45\linewidth}
		\includegraphics[width=1\textwidth]{../../codding_model/own_basedOnFried/optimalPol_010922_revision/figures/all_13Sept22_Tplus30/sff_OPT_T_NoTaus_COMPtaulPer_regime4_spillover0_knspil0_noskill0_sep0_xgrowth0_PV1_etaa0.79.png}
	\end{subfigure}
\begin{minipage}[]{0.1\textwidth}
	\
\end{minipage}
	\begin{subfigure}{0.4\textwidth}
		\caption{Non-energy scientists}
		%	\captionsetup{width=.45\linewidth}
		\includegraphics[width=1\textwidth]{../../codding_model/own_basedOnFried/optimalPol_010922_revision/figures/all_13Sept22_Tplus30/sn_OPT_T_NoTaus_COMPtaulPer_regime4_spillover0_knspil0_noskill0_sep0_xgrowth0_PV1_etaa0.79.png}
	\end{subfigure}

\vspace{3mm}
	\begin{subfigure}{0.4\textwidth}
	\caption{Green scientists}
	%	\captionsetup{width=.45\linewidth}
	\includegraphics[width=1\textwidth]{../../codding_model/own_basedOnFried/optimalPol_010922_revision/figures/all_13Sept22_Tplus30/sg_OPT_T_NoTaus_COMPtaulPer_regime4_spillover0_knspil0_noskill0_sep0_xgrowth0_PV1_etaa0.79.png}
\end{subfigure}
\begin{minipage}[]{0.1\textwidth}
	\
\end{minipage}
\begin{subfigure}{0.4\textwidth}
\caption{Green-to-fossil output}
%	\captionsetup{width=.45\linewidth}
\includegraphics[width=1\textwidth]{../../codding_model/own_basedOnFried/optimalPol_010922_revision/figures/all_13Sept22_Tplus30/GFF_OPT_T_NoTaus_COMPtaulPer_regime4_spillover0_knspil0_noskill0_sep0_xgrowth0_PV1_etaa0.79.png}
\end{subfigure}
	\floatfoot{Notes: \footnotesize{ Graphs show the percentage deviations of the variable under the integrated policy regime where the planner can choose income tax progressivity and the separate regime where the income tax scheme is non-distortive, $\tau_{\iota t}=0$. }}
\end{figure} 

\paragraph{Knowledge spillovers}
 Knowledge spillovers are an essential model ingredient explaining the result. Absent knowledge spillovers, the optimal income tax scheme is regressive throughout which boosts labor supply in general and raises the high-to-low skill ratio supplied.\footnote{\ Consider Figure \ref{fig:opt_TLs_noKN} in Appendix \ref{app:TLS} which shows the results in a model without knowledge spillovers.} Green growth is higher, and fossil and non-energy growth decline in all periods relative to the separate policy regime.\footnote{\  The optimal income tax is not regressive due to the compositional effect but rather due to the level effect on economic activity. With homogeneous skills, there is no compositional effect of income tax progressivity. Nevertheless, the optimal policy features regressive income taxes to boost labor supply in combination with higher carbon taxes. This policy allows to redirect research and production towards the green sector while the regressive income tax keeps working hours high. Figure \ref{fig:opt_TLs_noknow_homoskill} in Appendix \ref{app:TLS} shows the results. 
} This policy achieves a stronger composition towards green production and growth through the higher carbon tax, while the regressive income tax mitigates the reductive effect of the higher carbon tax on consumption. 

Thus, knowledge spillovers render it advantageous to initially boost technology growth in fossil energy and non-energy instead of pushing towards green growth directly.
The gains from knowledge spillovers arise from decreasing returns to research within sectors. A smoother distribution of scientists across sectors makes them more productive. Knowledge spillovers to the green sector from conventional ones then mitigate the costs of the net-zero emission limit. 


In line with this argument, the social planner lowers fossil research relative to the laissez-faire allocation in all periods; compare Figure \ref{fig:optAll_percLf_dyn_nokn}. Instead, non-energy and green research increase more. As a result of decreasing returns to research, the rise in consumption attained under the efficient relative to the laissez-faire allocation is muted. In addition, efficient hours worked increase over time.  Hence, absent knowledge spillovers, the emission limit implies such a reduction in consumption, that the income effect dominates the substitution effect on labor, and work effort has to raise to raise consumption.


\begin{figure}[h!!!]
	\centering
	\caption{Deviation from optimal policy to counterfactual without labor income tax}\label{fig:opt_Count}
	\begin{subfigure}{0.4\textwidth}
		\caption{Fossil production}
		%	\captionsetup{width=.45\linewidth}
		\includegraphics[width=1\textwidth]{../../codding_model/own_basedOnFried/optimalPol_010922_revision/figures/all_13Sept22_Tplus30/CountTAUFPerDif_Opt_target_F_nsk0_xgr0_knspil0_regime4_spillover0_sep0_extern0_PV1_etaa0.79.png}
	\end{subfigure}	
\begin{minipage}[]{0.1\textwidth}
\
\end{minipage}
	\begin{subfigure}{0.4\textwidth}
		\caption{Period utility}
		%	\captionsetup{width=.45\linewidth}
		\includegraphics[width=1\textwidth]{../../codding_model/own_basedOnFried/optimalPol_010922_revision/figures/all_13Sept22_Tplus30/CountTAUFPerDif_Opt_target_SWF_nsk0_xgr0_knspil0_regime4_spillover0_sep0_extern0_PV1_etaa0.79.png}
	\end{subfigure}

\vspace{3mm}	
	\begin{subfigure}{0.4\textwidth}
		\caption{High-skill hours worked}
		%	\captionsetup{width=.45\linewidth}
		\includegraphics[width=1\textwidth]{../../codding_model/own_basedOnFried/optimalPol_010922_revision/figures/all_13Sept22_Tplus30/CountTAUFPerDif_Opt_target_hh_nsk0_xgr0_knspil0_regime4_spillover0_sep0_extern0_PV1_etaa0.79.png}
	\end{subfigure}
\begin{minipage}[]{0.1\textwidth}
\
\end{minipage}	
	\begin{subfigure}{0.4\textwidth}
		\caption{Low-skill hours worked}
		%	\captionsetup{width=.45\linewidth}
		\includegraphics[width=1\textwidth]{../../codding_model/own_basedOnFried/optimalPol_010922_revision/figures/all_13Sept22_Tplus30/CountTAUFPerDif_Opt_target_hl_nsk0_xgr0_knspil0_regime4_spillover0_sep0_extern0_PV1_etaa0.79.png}
	\end{subfigure}
%	\begin{subfigure}{0.4\textwidth}
%		\caption{Fossil scientists}
%		%	\captionsetup{width=.45\linewidth}
%		\includegraphics[width=1\textwidth]{../../codding_model/own_basedOnFried/optimalPol_010922_revision/figures/all_13Sept22_Tplus30/CountTAUFPerDif_Opt_target_sff_nsk0_xgr0_knspil0_regime4_spillover0_sep0_extern0_PV1_etaa0.79.png}
%	\end{subfigure}
%	\begin{subfigure}{0.4\textwidth}
%		\caption{Green scientists}
%		%	\captionsetup{width=.45\linewidth}
%		\includegraphics[width=1\textwidth]{../../codding_model/own_basedOnFried/optimalPol_010922_revision/figures/all_13Sept22_Tplus30/CountTAUFPerDif_Opt_target_sg_nsk0_xgr0_knspil0_regime4_spillover0_sep0_extern0_PV1_etaa0.79.png}
%	\end{subfigure}
%	\begin{subfigure}{0.4\textwidth}
%		\caption{Non-energy scientists}
%		%	\captionsetup{width=.45\linewidth}
%		\includegraphics[width=1\textwidth]{../../codding_model/own_basedOnFried/optimalPol_010922_revision/figures/all_13Sept22_Tplus30/CountTAUFPerDif_Opt_target_sn_nsk0_xgr0_knspil0_regime4_spillover0_sep0_extern0_PV1_etaa0.79.png}
%	\end{subfigure}
%	\begin{subfigure}{0.4\textwidth}
%	\caption{Green-to-fossil output}
%	%	\captionsetup{width=.45\linewidth}
%	\includegraphics[width=1\textwidth]{../../codding_model/own_basedOnFried/optimalPol_010922_revision/figures/all_13Sept22_Tplus30/CountTAUFPerDif_Opt_target_GFF_nsk0_xgr0_knspil0_regime4_spillover0_sep0_extern0_PV1_etaa0.79.png}
%\end{subfigure}
\end{figure}
\paragraph{Counterfactual policy: only optimal carbon tax}
%\tr{\ar Contribution of labor tax to allocation}
This section focuses on the contribution of the labor income tax to the optimal allocation.
I will argue that the rationale for using the labor income tax stems 
from labor market distortions caused by using the carbon tax to target the allocation of research. 
Figure \ref{fig:opt_Count} presents percentage deviations under the integrated optimal policy relative to the separate regime. The difference can be perceived as the effect of the labor income tax, assuming the carbon tax is implemented first.\footnote{\ Hence, interactions between the carbon and the labor tax are assigned to the effect of the labor tax.}

To mitigate the reduction in fossil research in early years, the Ramsey planner sets a lower carbon tax. However, this keeps the wage rate high encouraging work effort. Yet, this would lead to too high emissions; especially in light of the bigger fossil share in production. The progressive income tax counters this effect by reducing returns to labor. Compare Panels (c) and (d). Then, the labor income tax contributes to lowering fossil production. 

The higher carbon tax under the net-zero emission limit, in contrast, implies too low a wage rate discouraging labor supply. The labor subsidy ensures that labor supply rises. It, thereby, raises fossil production (Panel (a)). Yet, the emission limit remains satisfied due to the higher carbon tax. 

The labor income tax is mainly targeted at the level of labor and not research activity since also in the model version with homogeneous skills, the labor income tax is used to subsidize labor. However, there is no effect of the labor income tax on research in equilibrium. A higher demand for research is absorbed by a higher wage rate for scientists. Compare Figure \ref{fig:opt_Count_homskill} in Appendix \ref{app:quant_res_opt}. 

% The tax on labor raises period utility in early years and decreases it during the net-zero emission period (Panel (b)). The rise in utility stems from more leisure. In contrast to the overall effect of the integrated policy, the labor income tax raises green growth in initial periods. This follows from a reduction in non-energy research since the energy good becomes more expensive. More scientists are allocated to the energy sector of which a higher number is directed to fossil. The regressive tax implies a reduction in period utility and a rise in fossil production.
 
 
 
 
% However, when skills are homogeneous, the labor income tax has no effect on the allocation of research. : the amount of fossil scientists is largely unaffected by the income tax. Nevertheless, the labor income tax is used to subsidize labor. Therefore, I argue, that the labor income tax serves to correct for distortions in the labor market, as the carbon tax is set higher to lower fossil  research activity. The higher carbon tax distorts the labor market since the wage rate declines more relative to a benchmark where the carbon tax was set to internalize the social cost of carbon. This would be the optimal level of the carbon tax if research subsidies were available. In this second-best setting, however, the wage rate is lower and labor supply is inefficiently low. The 
 
% In a nutshell, the labor income tax, first, serves to implement the emission limit by reducing economic activity (the green-to-fossil energy ratio even declines (Panel (h))). It becomes necessary to reduce work effort since the smaller carbon tax induces a smaller green-to-fossil energy ratio. In addition, the wage rate reduces less so that the carbon tax implies a smaller reduction in labor supply. Second, the regressive income tax, in contrast, serves to increase labor supply replicating the efficient allocation. Furthermore, non-energy research and the green-to-fossil ratio rise since high-to-low skill labor supply increases. 
 
   
\clearpage