\section{Quantitative results}\label{sec:res}

This section presents and discusses the quantitative results. 
In Section \ref{subsec:exp}, I use the model to learn how a constant carbon tax affects the economy and how it interacts with a tax on labor income. Section \ref{subsec:meetlim} calculates how high a carbon tax is necessary to meet the emission limit. I find that an increasing carbon tax is necessary to counter market forces directing production and research towards the fossil sector. 
Section \ref{subsec:mr} goes one step further asking  how the government can optimally satisfy the emission limit using carbon and labor income taxes. Results show that a combination of the two instruments is optimal throughout. 

%I focus on analyzing the mechanisms and welfare benefits from integrating the income tax scheme into the environmental policy. I also discuss the costs of not using lump-sum transfers.

\subsection{Results }
This section depicts the environmental tax which is required to meet the emission limits and the evolution of key variables under distinct policy regimes. The income tax scheme is kept fixed at its calibrated levels: $\tau_{\iota t}=0.181$, $\lambda=0.43$.
I consider the following regimes: first, environmental tax revenues are redistributed lump sum to households. Second, the government runs a consolidated budget and environmental tax revenues are redistributed via the income tax scheme; that is, $\lambda$ adjusts. Third, environmental tax revenues are recycled as subsidies to the green sector. Fourth, the government consumes environmental tax revenues. 



\subsection{Optimal policy}\label{subsec:mr}


This section seeks to answer the question how a benevolent planner optimally attains the emission limit. After showing the results in Section \ref{sec:optres}, I discuss the intention behind the optimal policy in Section \ref{subsec:dis}. The optimal policy consists of a combination of labor income and carbon taxes to implement the emission limit. The reason is that when the carbon tax is used to target the direction of research, distortions in the labor market occur. The labor tax boosts or curbs labor supply to mitigate these distortions. 
%This section depicts results on the optimal policy followed by the implied allocation in the benchmark model where environmental tax revenues are redistributed via the income tax scheme. 

\subsubsection{Results}\label{sec:optres}
Figure \ref{fig:optPol} depicts the optimal policy.
To meet the emission limit suggested by the IPCC, the optimal policy taxes labor until 2044 (Panel (a)). The labor tax  turns into a subsidy from 2045 onward, $\tau_{\iota t}<0$. 
\begin{figure}[h!!]
	\centering
	\caption{Optimal policy }\label{fig:optPol}
	\begin{subfigure}{0.4\textwidth}
		\caption{Average marginal income tax rate }
		%	\captionsetup{width=.45\linewidth}
		\includegraphics[width=1\textwidth]{../../codding_model/own_basedOnFried/optimalPol_010922_revision/figures/all_13Sept22_Tplus30/dTaulAv_OPT_T_NoTaus_COMPtaul_regime4_spillover0_knspil0_noskill0_sep0_xgrowth0_PV1_etaa0.79_lgd0.png}
	\end{subfigure}
\begin{minipage}[]{0.1\textwidth}
	\
\end{minipage}
	\begin{subfigure}{0.4\textwidth}
		\caption{Tax per ton of carbon in 2022 US\$ }
		%	\captionsetup{width=.45\linewidth}
		\includegraphics[width=1\textwidth]{../../codding_model/own_basedOnFried/optimalPol_010922_revision/figures/all_13Sept22_Tplus30/Single_periods12_OPT_T_NoTaus_Tauf_regime4_spillover0_knspil0_noskill0_sep0_xgrowth0_extern0_PV1_sizeequ0_GOV0_etaa0.79.png}
	\end{subfigure}
%\floatfoot{Notes: \footnotesize{The x-axis indicates the first year of the 5 year period to which the variable value corresponds. }}
\end{figure} 
Consider now Panel (b). The optimal carbon tax increases over time and jumps to a higher level when the net-zero emission limit is introduced in 2050.
In 2020, the carbon tax equals US\$987 and rises steadily to US\$1,325 in the 2045-2049 period.  As the emission limit declines to net-zero, the tax rapidly surges to US\$2,833 and gradually increases afterwards reaching US\$3,278 in 2070-2074. 
%\paragraph{Efficient and optimal allocation}\label{subsec:notaul}

Figure \ref{fig:optAll_percLf_dyn} depicts the adjustments of key variables under the first-best (efficient) and the second-best (optimal) policy relative to the laissez-faire allocation.\footnote{\ I formulate the social planner's problem in Appendix \ref{app:sp_prob}. }  Dashed graphs show the efficient and solid graphs the optimal allocation.
The efficient allocation can be perceived as the allocation the Ramsey planner seeks to implement. However, she may not be able to achieve the efficient allocation due to the reliance on a limited number of tax instruments.

\begin{figure}[h!!!]
	\centering \caption{Efficient and optimal allocation relative to laissez-faire}
\label{fig:optAll_percLf_dyn}
	\begin{subfigure}[]{1\textwidth}	
		\centering\footnotesize{\textbf{In percentage deviation from laissez-faire}}\\ \vspace{2mm}
	\begin{subfigure}[]{0.4\textwidth}
		\caption{Consumption}
		%	\captionsetup{width=.45\linewidth}
		\includegraphics[width=1\textwidth]{../../codding_model/own_basedOnFried/optimalPol_010922_revision/figures/all_13Sept22_Tplus30/C_PercentageLFDyn_Target_regime4_knspil0_spillover0_noskill0_sep0_xgrowth0_PV1_etaa0.79_lgd1.png}
	\end{subfigure}
\begin{minipage}[]{0.1\textwidth}
	\ 
\end{minipage}
	\begin{subfigure}[]{0.4\textwidth}
				\caption{Average hours worked }
		%	\captionsetup{width=.45\linewidth}
		\includegraphics[width=1\textwidth]{../../codding_model/own_basedOnFried/optimalPol_010922_revision/figures/all_13Sept22_Tplus30/Hagg_PercentageLFDyn_Target_regime4_knspil0_spillover0_noskill0_sep0_xgrowth0_PV1_etaa0.79_lgd0.png}
	\end{subfigure}

\vspace{3mm}
\begin{subfigure}[]{0.4\textwidth}
				\caption{Green-to-fossil energy ratio}
	%	\captionsetup{width=.45\linewidth}
	\includegraphics[width=1\textwidth]{../../codding_model/own_basedOnFried/optimalPol_010922_revision/figures/all_13Sept22_Tplus30/GFF_PercentageLFDyn_Target_regime4_knspil0_spillover0_noskill0_sep0_xgrowth0_PV1_etaa0.79_lgd0.png}
\end{subfigure}
\begin{minipage}[]{0.1\textwidth}
	\ 
\end{minipage}
\begin{subfigure}[]{0.4\textwidth}
			\caption{ Non-energy scientists share}
	%	\captionsetup{width=.45\linewidth}
	\includegraphics[width=1\textwidth]{../../codding_model/own_basedOnFried/optimalPol_010922_revision/figures/all_13Sept22_Tplus30/snS_PercentageLFDyn_Target_regime4_knspil0_spillover0_noskill0_sep0_xgrowth0_PV1_etaa0.79_lgd0.png}
\end{subfigure}
\end{subfigure}

\vspace{3mm}
\begin{subfigure}[]{1\textwidth}
	\centering	\footnotesize{{\textbf{In levels}}}\\ \vspace{2mm}
\begin{subfigure}[]{0.4\textwidth}
				\caption{Fossil scientists}
	%	\captionsetup{width=.45\linewidth}
	\includegraphics[width=1\textwidth]{../../codding_model/own_basedOnFried/optimalPol_010922_revision/figures/all_13Sept22_Tplus30/sff_CompEffOPT_T_NoTaus_regime4_opteff_knspil0_spillover0_noskill0_sep0_xgrowth0_countec0_PV1_etaa0.79_lgd1_lff1.png}
\end{subfigure}
\begin{minipage}[]{0.1\textwidth}
\ 
\end{minipage}
\begin{subfigure}[]{0.4\textwidth}
			\caption{Green-to-fossil scientists}
%	\captionsetup{width=.45\linewidth}
\includegraphics[width=1\textwidth]{../../codding_model/own_basedOnFried/optimalPol_010922_revision/figures/all_13Sept22_Tplus30/sgsff_CompEff_Target_onlyeff_reg4_spillover0_knspil0_noskill0_sep0_xgrowth0_countec0_PV1_etaa0.79_lgd1.png}
\end{subfigure}
\end{subfigure}
	\floatfoot{Notes: \footnotesize{ The figure shows the percentage deviation of the allocation resulting under the combined policy, that is, the progressivity of the income tax can be chosen (the black solid graph), the optimal allocation in the carbon-tax-only regime when no progressive income tax is available (the gray dashed graph), and the efficient allocation (the orange dotted graph), relative to the laissez-faire allocation. 
	}}
\end{figure} 
%
%\clearpage
%%
% Labor supply
The social planner attains the emission limit while increasing consumption and decreasing labor supply relative to laissez-faire (Panels (a) and (b) in Figure \ref{fig:optAll_percLf_dyn}). This allocation is achieved by a higher research effort on aggregate: average hours of scientists roughly double in all periods. The social planner increases the share of non-energy scientists (Panel (d)). Within the energy sector, the efficient allocation is characterized by a higher level of fossil scientists as compared to green scientists (Panel (e) and (f)). As the emission limit becomes stricter, the ratio of green-to-fossil scientists increases.\footnote{\ Figure \ref{fig:LF} in Appendix \ref{app:quant_res_opt}  shows the laissez-faire, the efficient, and the optimal allocation in levels.} 
The social planner can sustain high growth rates -especially in the fossil sector- and simultaneously meet the emission limit by choosing a lower energy share to GDP and a higher ratio of green-to-fossil energy (Panel (c)). 


Under the optimal policy, in contrast, consumption reduces relative to the laissez-faire allocation. Average hours of scientists fall slightly by approximately 0.1\%. The optimal policy implements a lower share of energy research, and the number of fossil scientists remains close to zero (Panels (d) and (e)).
In the competitive economy, a rise in fossil research has to be induced via demand. A higher demand for fossil, however, would conflict with the emission target. A trade-off between research and emission mitigation exists.  In fact, the optimal allocation falls short of the green-to-fossil energy ratio (Panel (c)) and research effort. Thus, implementing the emission target is costly in terms of R\&D investment and growth.

%%%%%%%%%%%%%%%%%%%%%%%%%%%%%%%%%%%%%%%%%%%%%%%%%%%%%%%%%%%%%%%%%%%%%%%%%%%%%%%%%%%%
%% DISCUSSION 
%%%%%%%%%%%%%%%%%%%%%%%%%%%%%%%%%%%%%%%%%%%%%%%%%%%%%%%%%%%%%%%%%%%%%%%%%%%%%%%%%%%%

\subsubsection{Discussion}\label{subsec:dis}

 What explains the optimal policy?  To answer this question, I, first, consider the social planner allocation without knowledge spillovers. Knowing the reasons behind the efficient allocation enables us to better understand the use of policy instruments. Second, I look at how the optimal allocation with labor income tax differs from the optimal allocation when no income tax is available. 
Finally, I conduct a counterfactual experiment where only the optimal carbon tax is implemented. Comparing the resulting allocation to the optimal allocation sheds light on the specific role of the income tax. 

\paragraph{Efficient and optimal allocation without knowledge spillovers}

Absent knowledge spillovers, the social planner raises research efforts  by a factor of 3.5 compared to 2 in the benchmark model, and hours worked increase more over time. Nevertheless, consumption growth rises less than in the model with knowledge spillovers. The reason is that meeting the emission limit is only maintainable under a less productive allocation of researchers across sectors. 

Figure \ref{fig:optAll_percLf_dyn_noKN} depicts the efficient allocation of scientists across energy sectors. When knowledge cannot spill from conventional to the green sector, meeting the emission limit requires a strong reduction in fossil research (Panel (a)). Almost all energy research happens in the green sector (Panel (b)). Due to decreasing returns to scale, this extreme allocation reduces overall research productivity and, hence, consumption growth.
 
The rise in hours over time under the social planner reflects the slow down in consumption. Consumption becomes so valuable, that hours have to rise. 
Hence, knowledge spillovers mitigate the costs of implementing the emission limit since they allow to mitigate decreasing returns to research. 

\begin{figure}[h!!!]
	\centering 	\caption{Efficient and optimal allocation without knowledge spillovers}\label{fig:optAll_percLf_dyn_noKN}
		\begin{subfigure}[]{1\textwidth}	
		\centering\footnotesize{\textbf{In percentage deviation from laissez-faire}}\\ \vspace{2mm}
		\begin{subfigure}[]{0.4\textwidth}
			\caption{Consumption}
			%	\captionsetup{width=.45\linewidth}
			\includegraphics[width=1\textwidth]{../../codding_model/own_basedOnFried/optimalPol_010922_revision/figures/all_13Sept22_Tplus30/C_PercentageLFDyn_Target_regime4_knspil1_spillover0_noskill0_sep0_xgrowth0_PV1_etaa0.79_lgd1.png}
		\end{subfigure}
		\begin{minipage}[]{0.1\textwidth}
			\ 
		\end{minipage}
		\begin{subfigure}[]{0.4\textwidth}
			\caption{Average hours worked }
			%	\captionsetup{width=.45\linewidth}
			\includegraphics[width=1\textwidth]{../../codding_model/own_basedOnFried/optimalPol_010922_revision/figures/all_13Sept22_Tplus30/Hagg_PercentageLFDyn_Target_regime4_knspil1_spillover0_noskill0_sep0_xgrowth0_PV1_etaa0.79_lgd0.png}
		\end{subfigure}
	\end{subfigure}
	
	\vspace{3mm}
			\begin{subfigure}[]{1\textwidth}	
		\centering\footnotesize{\textbf{In levels}}\\ \vspace{2mm}
\begin{subfigure}[]{0.4\textwidth}
	\caption{Fossil scientists}
	%	\captionsetup{width=.45\linewidth}
	\includegraphics[width=1\textwidth]{../../codding_model/own_basedOnFried/optimalPol_010922_revision/figures/all_13Sept22_Tplus30/sff_CompEffOPT_T_NoTaus_regime4_opteff_knspil1_spillover0_noskill0_sep0_xgrowth0_countec0_PV1_etaa0.79_lgd1_lff1.png}
\end{subfigure}
	\begin{minipage}[]{0.1\textwidth}
		\ 
	\end{minipage}
	\begin{subfigure}[]{0.4\textwidth}
		\caption{Green-to-fossil scientists}
		%	\captionsetup{width=.45\linewidth}
		\includegraphics[width=1\textwidth]{../../codding_model/own_basedOnFried/optimalPol_010922_revision/figures/all_13Sept22_Tplus30/sgsff_CompEffOPT_T_NoTaus_regime4_opteff_knspil1_spillover0_noskill0_sep0_xgrowth0_countec0_PV1_etaa0.79_lgd0_lff1.png}
	\end{subfigure}	
\end{subfigure}
\end{figure} 


\paragraph{Comparison to carbon-tax-only policy regime}
How is the income tax used to achieve a more efficient allocation?
This section analysis the benefits of the policy regime with income tax, the \textit{combined} regime, as opposed to a \textit{carbon-tax-only} regime where $\tau_{\iota t}=0$.  
Panel (a)  in Figure \ref{fig:efftaul} present percentage deviations of variables under the optimal combined policy relative to the carbon-tax-only regime. In the period from 2020 to 2044 the carbon tax is lower when a labor income tax can be used (Panel (ai)). In the exact same period, the labor income tax is used to tax labor. From 2045 onward, the carbon tax exceeds its counterpart when no labor tax is available. Now, the government subsidizes labor. Thus, labor income taxes and carbon taxes act as substitutes.

By setting a lower carbon tax and taxing labor, the government achieves a smoother distribution of energy scientists in the periods before the net-zero emission limit: the share of green-to-fossil scientists decreases (Panel (aii)). 
% the fossil and non-energy sector keep consumption high; see Panels (c) and (d). 
%To achieve the higher technology levels, the planner accepts a smaller green-to-fossil ratio today. % which result in lower utility levels tomorrow. %
Hence, in terms of the allocation of research, the optimal policy comes closer to the efficient allocation. Yet, to do so, it forfeits an advantageous green-to-fossil energy ratio (both, a smaller carbon tax and a tax on labor contribute to the adverse energy ratio). This observation highlights the trade-off of more fossil research at lower fossil demand. 

When the net-zero emission limit binds, the carbon tax under the combined policy is higher than in the carbon-tax-only scenario.
Now, the government accepts an adverse green-to-fossil research ratio. The  stricter emission limit leaves no room to profit from fossil growth. Instead, growth has to be reduced more in favor of a higher green energy share. 

In the next two paragraphs, I decompose the effect of the combined policy regime relative to the carbon-tax-only regime into (i) the effect of the adjustment of the carbon tax, and (ii) the adjustment in the labor income tax. Under the assumption that the government first adjusts the carbon tax and then implements the labor tax, this decomposes the  aggregate effect into the effect of the policy regime as a whole.  

\paragraph{Effect of adjustment in carbon tax}
The lower carbon tax almost fully accounts for the deviation of the ratio of green to fossil research. (Panel (bi)). However, it also induces a rise in average hours worked prior to 2045. The higher wage rate makes households willing to work more hours. Especially in light of an adverse green-to-fossil production ratio, more labor supply raises emissions. The allocation with the lower carbon tax alone would violate the emission limit. 
Average hours are close to the level under the carbon-tax-only allocation under the net-zero emission limit. Yet, they are lower due to a reduction in the wage rate induced by the higher carbon tax. As shown in the analytical section, a smaller share of fossil production is concomitant with a lower marginal product of labor as more labor is allocated in the green sector thereby reducing the aggregate wage rate. 

\paragraph{Effect of labor income tax}

Panel (c) in Figure \ref{fig:efftaul} compares the allocation under the combined policy to the one where only the carbon tax is set to its optimal value and the labor income tax is kept at zero. The difference is, thus, explained by income taxation. 

The labor income tax contributes to the growth-enhancing green-to-fossil ratio of scientists. Albeit minimally (Panel (ci)). 

Instead, the labor income tax becomes more important when looking at average hours worked (Panel (cii)). It reduces labor supply countering the stimulating effect of the lower carbon tax before the net-zero emission. Households supply more labor because the carbon tax is used to allocate research, a distortion in the labor market occurs: households do not internalize the negative effect of their work effort on emissions. 

The higher carbon tax under the net-zero emission limit, in contrast, implies too low a wage rate discouraging labor supply. The labor subsidy ensures that labor supply rises. It, thereby, raises fossil production. Yet, the emission limit remains satisfied due to the higher carbon tax.  \textit{Emissions are below their optimal social level}

Indeed, the progressive income tax fosters fossil growth in early periods through its effect on the skill ratio supplied. However, even in a model with homogeneous skills, the labor income tax is part of the optimal policy. In this counterfactual model, the labor income tax does not affect research effort. The reason is that a higher demand for research is absorbed by a higher wage rate for scientists. Compare Figure \ref{fig:opt_Count_homskill} in Appendix \ref{app:quant_res_opt}.  Hence, the intention behind using the income tax is to correct for distortions on the labor market. These distortions occur through "over-" or "undertaxing" carbon once the carbon tax also targets research activity. 


\begin{figure}[h!!!]
	\centering
	\caption{Decomposition effect of combined policy}\label{fig:efftaul}
	\begin{subfigure}{1\textwidth}
		\caption{\textbf{Deviation of combined policy from carbon-tax-only policy in percent}}
		\vspace{3mm}
	\begin{subfigure}{0.4\textwidth}
		\centering{(i) Carbon tax}
		%	\captionsetup{width=.45\linewidth}
		\includegraphics[width=1\textwidth]{../../codding_model/own_basedOnFried/optimalPol_010922_revision/figures/all_13Sept22_Tplus30/Tauf_OPT_T_NoTaus_COMPtaulPer_regime4_spillover0_knspil0_noskill0_sep0_xgrowth0_PV1_etaa0.79.png}
	\end{subfigure}
\begin{minipage}[]{0.1\textwidth}
\
\end{minipage}
\begin{subfigure}{0.4\textwidth}
	\centering{(ii) Green-to-fossil scientists}
	%	\captionsetup{width=.45\linewidth}
	\includegraphics[width=1\textwidth]{../../codding_model/own_basedOnFried/optimalPol_010922_revision/figures/all_13Sept22_Tplus30/sgsff_OPT_T_NoTaus_COMPtaulPer_regime4_spillover0_knspil0_noskill0_sep0_xgrowth0_PV1_etaa0.79.png}
\end{subfigure}
\end{subfigure}

\vspace{3mm}
	\begin{subfigure}{1\textwidth}
	\caption{\textbf{Deviation only optimal carbon tax from carbon-tax-only policy in percent}}
	\vspace{3mm}
	\begin{subfigure}{0.4\textwidth}
		\centering{(i) Green-to-fossil scientists}
		%	\captionsetup{width=.45\linewidth}
		\includegraphics[width=1\textwidth]{../../codding_model/own_basedOnFried/optimalPol_010922_revision/figures/all_13Sept22_Tplus30/CountTAUF_CTOPer_Opt_target_sgsff_nsk0_xgr0_knspil0_regime4_spillover0_sep0_extern0_PV1_etaa0.79.png}
	\end{subfigure}
	\begin{minipage}[]{0.1\textwidth}
		\
	\end{minipage}
	\begin{subfigure}{0.4\textwidth}
		\centering{(ii) Average hours worked}
		%	\captionsetup{width=.45\linewidth}
		\includegraphics[width=1\textwidth]{../../codding_model/own_basedOnFried/optimalPol_010922_revision/figures/all_13Sept22_Tplus30/CountTAUF_CTOPer_Opt_target_Hagg_nsk0_xgr0_knspil0_regime4_spillover0_sep0_extern0_PV1_etaa0.79.png}
	\end{subfigure}
\end{subfigure}

\vspace{3mm}
\begin{subfigure}{1\textwidth}
	\caption{\textbf{Deviation of combined policy from only optimal carbon tax in percent}}\label{fig:opt_Count}
			\vspace{3mm}
	\begin{subfigure}{0.4\textwidth}
		\centering{(i) Green-to-fossil scientists}
		%	\captionsetup{width=.45\linewidth}
		\includegraphics[width=1\textwidth]{../../codding_model/own_basedOnFried/optimalPol_010922_revision/figures/all_13Sept22_Tplus30/CountTAUFPerDif_Opt_target_sgsff_nsk0_xgr0_knspil0_regime4_spillover0_sep0_extern0_PV1_etaa0.79.png}
	\end{subfigure}	
	\begin{minipage}[]{0.1\textwidth}
		\
	\end{minipage}	
	\begin{subfigure}{0.4\textwidth}
		\centering{(ii) Average hours worked}
		%	\captionsetup{width=.45\linewidth}
		\includegraphics[width=1\textwidth]{../../codding_model/own_basedOnFried/optimalPol_010922_revision/figures/all_13Sept22_Tplus30/CountTAUFPerDif_Opt_target_Hagg_nsk0_xgr0_knspil0_regime4_spillover0_sep0_extern0_PV1_etaa0.79.png}
	\end{subfigure}
\end{subfigure}

	\floatfoot{Notes: \footnotesize{ Graphs show the percentage deviations of the variable under the combined policy regime where the planner can choose income tax progressivity and the carbon-tax-only regime where the income tax scheme is non-distortive, $\tau_{\iota t}=0$. }}
\end{figure} 

\begin{comment}

%\paragraph{Knowledge spillovers}
Knowledge spillovers are a crucial determinant of the optimal policy. As we have seen previously, knowledge spillovers change the allocation of research in the efficient allocation. To implement a lower share of fossil research right from the beginning, the planner chooses a more aggressive carbon tax and subsidizes labor throughout.\footnote{\ Consider Figure \ref{fig:opt_TLs_noKN} in Appendix \ref{app:TLS} which shows the results in a model without knowledge spillovers.} Green growth is higher, and fossil and non-energy growth decline in all periods relative to the carbon-tax-only policy regime.\footnote{\  The optimal income tax is not regressive due to the compositional effect but rather due to the level effect on economic activity. With homogeneous skills, there is no compositional effect of income tax progressivity. Nevertheless, the optimal policy features regressive income taxes to boost labor supply in combination with higher carbon taxes. This policy allows to redirect research and production towards the green sector while the regressive income tax keeps working hours high. Figure \ref{fig:opt_TLs_noknow_homoskill} in Appendix \ref{app:TLS} shows the results. 
} 
Thus, knowledge spillovers render it advantageous to initially boost technology growth in fossil energy and non-energy instead of pushing towards green growth directly.


content...
\end{comment}





% The tax on labor raises period utility in early years and decreases it during the net-zero emission period (Panel (b)). The rise in utility stems from more leisure. In contrast to the overall effect of the combined policy, the labor income tax raises green growth in initial periods. This follows from a reduction in non-energy research since the energy good becomes more expensive. More scientists are allocated to the energy sector of which a higher number is directed to fossil. The regressive tax implies a reduction in period utility and a rise in fossil production.
 
 
 
 
% However, when skills are homogeneous, the labor income tax has no effect on the allocation of research. : the amount of fossil scientists is largely unaffected by the income tax. Nevertheless, the labor income tax is used to subsidize labor. Therefore, I argue, that the labor income tax serves to correct for distortions in the labor market, as the carbon tax is set higher to lower fossil  research activity. The higher carbon tax distorts the labor market since the wage rate declines more relative to a benchmark where the carbon tax was set to internalize the social cost of carbon. This would be the optimal level of the carbon tax if research subsidies were available. In this second-best setting, however, the wage rate is lower and labor supply is inefficiently low. The 
 
% In a nutshell, the labor income tax, first, serves to implement the emission limit by reducing economic activity (the green-to-fossil energy ratio even declines (Panel (h))). It becomes necessary to reduce work effort since the smaller carbon tax induces a smaller green-to-fossil energy ratio. In addition, the wage rate reduces less so that the carbon tax implies a smaller reduction in labor supply. Second, the regressive income tax, in contrast, serves to increase labor supply replicating the efficient allocation. Furthermore, non-energy research and the green-to-fossil ratio rise since high-to-low skill labor supply increases. 
 
\clearpage