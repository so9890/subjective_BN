\subsection{Policy experiments}\label{subsec:exp}


We are now equipped to study how a carbon tax  equal to the social costs of carbon of 185\$, as found by a joint research effort of the \textit{Resources for the Future} (RFF), an independent research institution, and the University of Berkeley \citep{RFF} affects the economy. In particular, I am interested in whether a progressive income tax scheme shapes the effectiveness of a carbon tax (sections \ref{subsec:eff_cc}). 
 Comparing the level of emissions resulting under the constant carbon tax to the emission limits following the IPCC reveals that a stronger reduction in emissions is required. Therefore, I calculate a necessary environmental tax to satisfy the US emission limit in section  \ref{subsec:meetlim}. A progressive income tax lowers the required environmental tax by between 10 and 7\%. Furthermore, the allocation is characterized by higher technology growth in the fossil and non-energy sector.
 


\subsubsection{Effect of a constant carbon tax and the role of income tax progressivity}\label{subsec:eff_cc}
% \begin{itemize}
% 	\item  emissions continue to rise due to knowledge spillovers \checkmark
% 	\item  effect on research (non-energy research) \checkmark
% 	\item taul has level effect on emissions, interaction: compositional effect but small (but could become relevant through disadvantageous effect of tauf on non-energy research) \checkmark 	
% \end{itemize}

\begin{figure}[h!!]
	\centering
	\caption{Effect of a constant carbon tax equal to 185\$ per ton of carbon }\label{fig:Leveltauf_nsk0_xgr0_know}		
\begin{minipage}[]{0.32\textwidth}
	\centering{\footnotesize{(a) Net emissions}}
	%	\captionsetup{width=.45\linewidth}
	\includegraphics[width=1\textwidth]{../../codding_model/own_basedOnFried/optimalPol_010922_revision/figures/all_13Sept22/CompTauf_bytaul_Reg5_Emnet_spillover0_nsk0_xgr0_knspil0_sep0_LFlimit0_emsbase0_countec0_GovRev0_etaa0.79_lgd1.png}
\end{minipage}	
\begin{minipage}[]{0.32\textwidth}
\centering{\footnotesize{(b) Fossil }}
%	\captionsetup{width=.45\linewidth}
\includegraphics[width=1\textwidth]{../../codding_model/own_basedOnFried/optimalPol_010922_revision/figures/all_13Sept22/PerdifNoTauf_regime0_CompTaul_F_spillover0_nsk0_xgr0_knspil0_sep0_LFlimit0_emsbase0_countec0_GovRev0_etaa0.79_lgd1.png}
\end{minipage}
\begin{minipage}[]{0.32\textwidth}
\centering{\footnotesize{(c) Energy share to GDP}}
%	\captionsetup{width=.45\linewidth}
\includegraphics[width=1\textwidth]{../../codding_model/own_basedOnFried/optimalPol_010922_revision/figures/all_13Sept22/PerdifNoTauf_regime0_CompTaul_EY_spillover0_nsk0_xgr0_knspil0_sep0_LFlimit0_emsbase0_countec0_GovRev0_etaa0.79_lgd1.png}
\end{minipage}

\floatfoot{Notes:{ Panel (a) shows levels of net emissions under a constant carbon tax equal to 185\$ for a world with progressive income taxation at $\tau_l=0.181$, the solid graph, and without progressive income taxation, $\tau_l=0$. The thin dotted graph shows the emission limit suggested by the IPCC. Panels (b) and (c) show the percentage difference between the business-as-usual (BAU) policy without carbon tax and with carbon tax in the economy with and without progressive income tax by the solid and dash graphs, respectively. }}
\end{figure} 

\paragraph{Static effect of a carbon tax}
% 1) reallocation of demand by energy producers and by final good producers. 
 A carbon tax equal to 185\$ diminishes emissions by around 42.4\% initially relative to the business-as-usual policy without carbon tax.
 Panel (a) in figure \ref{fig:Leveltauf_nsk0_xgr0_know} shows how net emissions evolve under the carbon tax in a world with and without progressive income tax. In this and all following figures, the x-axis indicates the first year of the 5 year period to which the variable value corresponds.  Importantly, net emissions exceed the emission limit  derived previously; see the dotted line. Panels (b) and (c) show the deviation from the BAU allocation for fossil energy and the energy share in final production.
 
  As energy producers face a higher price for fossil energy, they lower demand for fossil and rise demand for green energy. Fossil production falls  (panel (b)), and green production rises.\footnote{\  I show research related and other variables in figure \ref{fig:Leveltauf_nsk0_xgr0_add} in appendix \ref{app:polexp_cc}. 
  }  
 The tax on fossil goods also increases the price for energy goods relative to non-energy goods on impact. Final good producers recompose their inputs towards non-energy goods. But, the recomposition is limited as energy and non-energy goods are complements. The energy share of GDP declines (panel (c)).  
 
 % 2) effect on research
 The shift in demand by energy and final good producers induces a reallocation of research. 
 As the market size of the green sector increases, profitability of research in the green sector rises. In contrast, research in the fossil sector falls. The ratio of green to fossil research increases; this makes the green good even cheaper contributing to an increase in the green-to-fossil energy ratio.
 
 As a side effect of the fossil tax, research in the non-energy sector declines. 
 The reason is that knowledge spillovers from the non-energy sector boost research profitability in the energy sector so that the aggregate effect is a decline in non-energy research. The theory on directed technical change suggests that a price effect directs research to the energy sector since energy and non-energy goods are complements.  However, absent knowledge spillovers, the share of non-energy research rises; consider panel (a) in figure \ref{fig:Leveltauf_nsk0_xgr0_noknow} which shows the effect of the constant carbon tax in a model without knowledge spillovers. The rise in non-energy scientists without knowledge spillovers - contrary to the theory - results from heterogeneous labor shares across sectors. The reduction in labor demand from the energy sector fails to lower non-energy production costs, and the price of non-energy goods remains higher. I discuss this finding in more detail in the following footnote.\footnote{\ The carbon tax raises the price of energy goods. As energy and non-energy goods are complements, demand for the non-energy good reduces, too. The price of non-energy goods falls. For the direction of research the relative change in revenues is pivotal. 
 It turns out that in the model absent knowledge spillovers, revenues in the energy sector fall more, so that non-energy research increases. However, knowledge spillovers from the non-energy sector, which is the biggest sector and has therefore a higher share in aggregate technology, rise profitability of the energy sector so much, that in equilibrium, non-energy research declines in response to the carbon tax. 
 
 The result of the theoretic literature \citep[e.g.][]{Hemous2021DirectedEconomics} on directed technical change is reestablished in the model without knowledge spillovers when labor shares are homogeneous. Then, a supply-side effect makes non-energy goods even cheaper: Lower labor demand from the fossil sector reduces production costs of the non-energy sector. Lower production costs amplify the reduction in the price of non-energy goods induced by lower demand.  As a result, revenues in the non-energy sector fall more and non-energy research declines -in line with the theory- even absent knowledge spillovers; see panel (c) in figure \ref{fig:Leveltauf_nsk0_xgr0_noknow} which shows the effects in a model without knowledge spillovers and with equal labor shares.  } 
 This mechanism counteracts the intention to lower energy consumption. %\footnote{\ In line with the theory on directed technical change, a price effect dominates the direction of research when goods are complements \citep{Hemous2021DirectedEconomics}.} 
 
 %3) Supply side effect
 The effectiveness of the carbon tax is amplified by a supply-side mechanism. 
 The lower demand for fossil reduces the demand for labor from the fossil sector. As a result, more labor is available for green energy production, and green supply increases. The price of green goods may fall despite higher demand. This supply-side channel facilitates a transition to green energy; however, it is muted due to a low labor share in the green sector and heterogeneity in labor input goods.

% EFFECT OF TAUL: income tax progressivity does not change growth in the non-energy sector,\tr{ since the level effect is absorbed by price changes} CHECK THIS. A compositional effect arises when skills are heterogeneous. In equilibrium, high-skill workers work more hours due to a higher wage rate. Therefore, the marginal gains from leisure are higher for this type. When the tax scheme becomes more progressive, high-skill workers are more responsive and curb their labor supply more.\footnote{\ They have to be compensated by a higher wage rate for a further reduction of leisure. The additional unit of leisure is more valuable to them.}  A more progressive tax scheme makes green labor more expensive and green production reduces more than fossil production. More research in the fossil sector amplifies this compositional effect. 

\paragraph{Dynamics}

%%%%%%%%%%%%%%%%%%%%%%%%%%%%%%%%%%%%%
%% knowledge spillovers
%%%%%%%%%%%%%%%%%%%%%%%%%%%%%%%%%%%%%
Over time the effectiveness of the carbon tax to lower fossil production declines from 37\% to 35.5\%. There are two model features explaining the declining effectiveness of carbon taxes: knowledge spillovers and heterogeneous labor shares. 

Consider first the effect of knowledge spillovers.
Initially, the fossil tax reduces research in the fossil sector, however, as green technology advances, knowledge spillovers from the green sector make fossil research more profitable again, and demand for fossil scientists resurges.
It is not only that a constant amount of researchers becomes more productive but also a change in the equilibrium level of fossil researchers which intensifies the effect of knowledge spillovers. This mechanism explains the quick rise in emissions under a constant fossil tax.\footnote{\ Figure \ref{fig:Leveltauf_nsk0_xgr0_noknow} shows the effect of a constant carbon tax in a model without knowledge spillovers, $\phi=0$. The rise in net emissions over time is muted and a constant carbon tax becomes more effective over time in reducing fossil production.  Absent knowledge spillovers more research is allocated to the green and non-energy sector.} 

Therefore, when knowledge spillovers are strong, reducing emissions to a net-zero emission limit requires a continues intensification of environmental intervention. In its extreme, growth might have to stop eventually in order to prevent the fossil sector from growing too much.

%%%%%%%%%%%%%%%%%%%%%%%%%%%%%%%%%%%%%%%%%%%
% \paragraph{Role of heterogeneous labor shares}
%%%%%%%%%%%%%%%%%%%%%%%%%%%%%%%%%%%%%%%%%%%%%%
Second, the green sector has the smallest labor share. Labor is more important in the fossil and most important in the non-energy sector. This heterogeneity lowers the effectiveness of the fossil tax. 
Consider a situation in which the green and the fossil sector both use the same share of labor. Then, a reduction in demand for labor in the fossil sector eases labor costs for the green sector. When, however, the green sector only uses a small share of labor, the higher labor supply does not lower green production costs as much, and the green good remains more expensive. The share of green energy and labor rises less. Panel (b)  in figure \ref{fig:Leveltauf_nsk0_xgr0_noknow} displays the behavior of key variables in a model with homogeneous labor shares across sectors, i.e., $\alpha_g=\alpha_f=\alpha_n$. 

When labor shares are equal, the effectiveness of the carbon tax to lower fossil production increases over time. 
Absent a carbon tax, a smaller labor share in the green sector raises the share of labor allocated to the fossil sector over time. As the fossil sector becomes more productive, the marginal product of labor in the fossil sector increases more and labor transitions to the fossil sector. The carbon tax counters this mechanism. %\textbf{Is it the BAU allocation that changes or the policy one?} % (Note that consequently the economy is not on a bgp in BAU)

%Endogenous growth intensifies this adverse effect of heterogeneous capital shares: as the market size of green goods is depressed, the fossil tax does not boost green research as much as with equal capital shares. 
%In sum, heterogeneous labor shares imply an increasing path of emissions over time under the fossil tax. 
%This feature of the economy also calls for a carbon tax increasing over time.


Absent knowledge spillovers and with equal labor shares, the constant carbon tax suffices to lower emissions over time. Then, endogenous growth directs research away from the fossil sector so that emissions fall over time under a constant carbon tax. Panel (c) in figure \ref{fig:Leveltauf_nsk0_xgr0_noknow} shows the result in this model variant. 
This finding replicates the result in \cite{Acemoglu2012TheChange} who abstract from knowledge spillovers and heterogeneity in labor shares and conclude that when dirty and clean goods are sufficient substitutes, a temporary intervention suffices to prevent too high pollution. In the present model, knowledge spillovers and heterogeneous capital shares 
call for an ever increasing carbon tax even though the green and the energy good are sufficiently substitutable. Panel (c) in figure \ref{fig:Leveltauf_nsk0_xgr0_noknow} visualizes this result: Without knowledge spillovers and with equal labor shares, the path of emissions is declining over time and the effectiveness of the carbon tax to lower fossil production rises. 

\paragraph{Effect of progressive income tax}
 %\textit{Annotation (SEP=0):  a progressive income tax raises research in fossil and in green \ar in energy goods, and lowers research in non-energy (the progressive tax makes energy goods more expensive which are more skill intense, being complements, research for energy becomes more profitable) But the effect ist small. This shift in innovation counters a reduction in the energy to output share (energy is less labor intense making labor more expensive it goes to the non-energy sector plus a reduction in skills; Q is this reduction also present in model with equal cap shares?); Furthermore, a prgressive income tax lowers the green to fossil ratio, intensified by response in research.  }
 
 The presence of a progressive income tax lowers the level of emissions; compare the solid and the dashed graph in panel (a) in figure \ref{fig:Leveltauf_nsk0_xgr0_know}. As labor supply reduces, output shrinks, and emissions fall. 
 However, there are compositional effects of a progressive income which (i) affect the economic structure, and (ii) interact with the effectiveness of the carbon tax. I will explain each statement in turn. 
 
% \paragraph{Compositional effect of progressive income tax}
 % compositional effect
 A progressive income tax lowers the green-to-fossil output ratio but diminishes the energy-to-GDP ratio. Figure \ref{fig:Efftaul_nsk0_xgr0_know} displays the effect of a progressive income tax in presence of a constant carbon tax. 
 The compositional effect of a progressive income tax originates from the asymmetric reaction of high- and low-skill workers, panel (c). High skill workers reduce their labor supply more, as the after-tax wage declines in response to a higher tax progressivity. Since they profit from less leisure initially, they require a higher wage rate to be compensated for the marginal unit of labor due to the concavity of the utility function in leisure. %\tr{The effect of a marginal increase in income tax progressivity intensifies with the level of pre-tax income.}
 Green production is skill biased as opposed to fossil production.
 As a  consequence, green production becomes more expensive, while fossil production becomes cheaper. Non-energy goods, which are less skill-intense than energy goods, become cheaper too. Energy producers substitute fossil for green energy, and final good producers turn to non-energy goods. The former effect raises emissions; the latter diminishes emissions.
 
 % research
 
 Research responds to the change in demand and in prices. First, non-energy research becomes less profitable, since the price of non-energy goods falls, while the amount of non-energy goods does not rise sufficiently due to energy and non-energy goods being complements. As a result, research is directed towards the energy sector. The share of non-energy researchers reduces, albeit minimally; see panel (e) in figure \ref{fig:Efftaul_nsk0_xgr0_know_app}.\footnote{\ The effect of the progressive income tax also prevails with heterogeneous labor shares and absent knowledge spillovers. Thus, the price-effect dominates the direction of research.} Second, focusing solely on the allocation of researchers between the fossil and green sector, the relatively higher supply of low-skill labor rises the market size of the fossil good. Since intermediate energy goods are sufficiently substitutable, the market size effect dominates the price effect and research shifts from green to fossil. Nevertheless, due to the reallocation of research to energy goods in general, the green sector, too, sees a rise in researchers. Overall, the share of green to fossil research rises. The reallocation of research towards fossil goods intensifies the fall in green-to-fossil energy (panel (a)). The reallocation of research towards the energy sector mitigates the decline in the energy share (panel (b)) . 
 
 I find that there is no level effect of the progressive income tax on innovation in equilibrium. To differentiate the compositional from the level effect, I consider the model with homogeneous skills. With only one skill type, there is no compositional effect of the labor tax. In this model, the equilibrium amount of scientists remains unchanged by a progressive income tax. Indeed, demand for innovation reduces in response to a progressive income tax since less labor is available to work with technology. However, at the same time, scientists are willing to accept a lower wage rate, since consumption of the household reduces. In equilibrium, the reduction in demand is absorbed by a change in the wage rate and the level of aggregate research remains unchanged. 
% Overall, the compositional effect of the progressive tax is small and the level effect dominates implying a reduction in net-emissions. 
 \begin{figure}[h!!]
 	\centering
 	\caption{Effect of a progressive income tax, $\tau_{\iota t}=0.181$, with a constant carbon tax}		\label{fig:Efftaul_nsk0_xgr0_know}		
 	\begin{minipage}[]{0.32\textwidth}
 		\centering{\footnotesize{(a) Green-to-fossil output}}
 		%	\captionsetup{width=.45\linewidth}
 		\includegraphics[width=1\textwidth]{../../codding_model/own_basedOnFried/optimalPol_010922_revision/figures/all_13Sept22/CompTaufPER_bytaul_Reg5_GFF_spillover0_nsk0_xgr0_knspil0_sep0_LFlimit0_emsbase0_countec0_GovRev0_etaa0.79_lgd0.png}
 	\end{minipage}
 	\begin{minipage}[]{0.32\textwidth}
 		\centering{\footnotesize{(b) Energy to GDP}}
 		%	\captionsetup{width=.45\linewidth}
 		\includegraphics[width=1\textwidth]{../../codding_model/own_basedOnFried/optimalPol_010922_revision/figures/all_13Sept22/CompTaufPER_bytaul_Reg5_EY_spillover0_nsk0_xgr0_knspil0_sep0_LFlimit0_emsbase0_countec0_GovRev0_etaa0.79_lgd0.png}
 	\end{minipage}
 	\begin{minipage}[]{0.32\textwidth}
 		\centering{\footnotesize{(c) High-to- low-skill ratio}}
 		%	\captionsetup{width=.45\linewidth}
 		\includegraphics[width=1\textwidth]{../../codding_model/own_basedOnFried/optimalPol_010922_revision/figures/all_13Sept22/CompTaufPER_bytaul_Reg5_hhhl_spillover0_nsk0_xgr0_knspil0_sep0_LFlimit0_emsbase0_countec0_GovRev0_etaa0.79_lgd0.png}
 	\end{minipage}
 \floatfoot{Notes: \footnotesize{All panels show the percentage difference in the allocation with and without progressive income tax scheme. The carbon tax is set to introduce a constant tax of 185\$ per ton of CO$_2$. }}
 \end{figure}
 
  
\paragraph{Interaction of progressive income tax and carbon tax}
The compositional effect of a income tax progressivity interacts with the impact of the carbon tax, as I will elaborate on now.
  Quantitatively, the effect of the carbon tax seems by and large unaffected by the value of income tax progressivity.
 % The stronger reduction of net emissions under a scenario with progressive income tax follows mechanically from a lower status-quo level of gros emissions due to the constant size of sinks.\footnote{\ A smaller initial level of fossil production inflates the percentage change in net emissions. }
 % interaction
 Yet, there is a smaller reduction in fossil energy visible in panel (b).
This discrepancy emerges from the effect of a carbon tax on skill supply which is affected by the income tax. 

    The carbon tax changes the skill premium since demand for green-specific high-skill labor increases. High-skill hours in equilibrium rise, while hours of low-skilled workers reduce.  A progressive income tax mutes the effect of changes in the wage rate on labor supply. The reason is that the elasticity of after-tax labor income with respect to pre-tax labor income reduces with a higher progressivity. This diminishes the supply response in the skill ratio to the carbon tax, and production costs of the green good remain higher. Therefore, the reduction in fossil production is muted in presence of a progressive income tax scheme. 
    
    % dynamic interaction
    Figure \ref{fig:Efftaul_nsk0_xgr0_know} reveals some dynamics in the effect of a progressive income tax. 
    Dynamics arise from a continuous rise in the high-to-low skill ratio. This rise occurs absent the carbon tax and is driven by the economy transitioning to a higher green energy share under the status-quo policy; compare panels (g) and (h) in figure \ref{fig:Leveltauf_nsk0_xgr0_add}. 
     A wider gap between hours worked by high- and low-skill workers intensifies the effect of income tax progressivity on the skill ratio. Higher hours of high-skill workers make them more responsive to the same rise in income tax progressivity while low-skill workers are relatively less responsive. 
    
   % The carbon tax induces a transition to green energy until growth in the green sector does no longer change the marginal productivity of labor across sectors and the economy is on a balanced growth path. 
    
    
    
  %  Research responds to the smaller supply of the skill ratio, and the rise in green research is muted while it is higher in the fossil sector. Furthermore, a smaller reduction in low-skill supply makes non-energy goods, which are more low-skill intense than energy goods, cheaper. The lower price of non-energy goods and its complementarity to energy goods (so that demand for energy remains high) explains a stronger reduction in non-energy research when the income tax scheme is progressive.\footnote{\ The price for non-energy research also reduces more in equilibrium. The reduction in non-energy research is demand-side driven.} Energy research reduces less under a progressive income tax scheme. Yet, the effect is so small that there is no visible heterogeneity in the effect of the carbon tax on the energy share in final good production (panel (c)). Consequently, the presence of a progressive income tax scheme counters the intention to lower fossil and energy production when skills are heterogeneous.
   
Overall, the smaller level in production induced by the tax on labor helps to reduce net emissions below the level without income tax despite the adverse compositional effects; see panel (a) in figure \ref{fig:Leveltauf_nsk0_xgr0_know}.
 
% \begin{comment}
% \paragraph{Role of heterogeneous skills}
% When skills are heterogeneous, emissions diminish because availability of the labor input good reduces. The effect of the fossil tax is similar to the model with homogeneous skills, yet, the fossil tax now affects labor.\footnote{\ With homogeneous skills, the equilibrium level of labor is unaffected by the fossil tax due to the log-utility of consumption implying that income and substitution effects of the wage rate cancel.}
% With two types of skills and heterogeneous high-skill share, the fossil tax has a diminishing effect on low-skill hours in equilibrium and raises high-skill hours. The reason is that a higher demand for green produce translates into a higher demand for the green-specific labor good which is high-skill biased. 
% 
% The effect on working hours through the fossil tax is affected by the presence of a progressive income tax scheme. The higher income tax progressivity, the smaller the effect of a change in the wage rate on labor supply, since the elasticity of post-tax labor income with respect to pre-tax labor income falls. 
% This asymmetry has an effect on research. A smaller reduction in low-skill supply makes non-energy goods, which are more low-skill intense than energy goods, cheaper. The lower price of non-energy goods and its complementarity to energy goods (so that demand for energy remains high) explains a stronger reduction in non-energy research when the income tax scheme is progressive.\footnote{\ The price for non-energy research also reduces more in equilibrium. The reduction in non-energy research is demand-side driven.} Energy research reduces less under a progressive income tax scheme. In addition, since a relatively higher low-skill supply boosts the market share of fossil producers, fossil research diminishes less than green research in presence of a progressive income tax. 
% Consequently, the presence of a progressive income tax scheme counters the intention to lower fossil and energy production when skills are heterogeneous. Nevertheless, is numerically negligible. 
%  
%  content...
%  \end{comment}
  
  
 \subsubsection{Meeting the emission limit}\label{subsec:meetlim}
 
 The previous section made apparent that, first, the carbon tax suggested by the RFF does not cause emissions in line with the IPCC's limit.  Second, it showed that model dynamics call for a dynamic carbon tax. In this section, I calculate the necessary carbon tax to meet the emission limit derived in section \ref{sec:calib}. I compare the resulting tax and allocations for the policy regimes with and without progressive income tax.\footnote{\ For results under other policy regimes, with lump-sum transfers and green subsidies see appendix \ref{app:pol_regimes}. }
 

 \begin{figure}[h!!]
 	\centering
 	\caption{Necessary carbon tax with and without progressive income tax  }\label{fig:Limit_nsk0_xgr0_know}		
 	\begin{minipage}[]{0.32\textwidth}
 		\centering{\footnotesize{(a) Tax per ton of carbon US\$}}
 		%	\captionsetup{width=.45\linewidth}
 		\includegraphics[width=1\textwidth]{../../codding_model/own_basedOnFried/optimalPol_010922_revision/figures/all_13Sept22/CompTauf_bytaul_Reg5_Tauf_spillover0_nsk0_xgr0_knspil0_sep0_LFlimit1_emsbase0_countec0_GovRev0_etaa0.79_lgd1.png}
 	\end{minipage}		\begin{minipage}[]{0.32\textwidth}
 	\centering{\footnotesize{(b) Deviation in carbon tax}}
 	%	\captionsetup{width=.45\linewidth}
 	\includegraphics[width=1\textwidth]{../../codding_model/own_basedOnFried/optimalPol_010922_revision/figures/all_13Sept22/CompTaufPER_bytaul_Reg5_Tauf_spillover0_nsk0_xgr0_knspil0_sep0_LFlimit1_emsbase0_countec0_GovRev0_etaa0.79_lgd0.png} \end{minipage}		
%\begin{minipage}[]{0.32\textwidth}
%	\centering{\footnotesize{(c) Fossil research }}
%	%	\captionsetup{width=.45\linewidth}
%	\includegraphics[width=1\textwidth]{../../codding_model/own_basedOnFried/optimalPol_010922_revision/figures/all_13Sept22/CompTauf_bytaul_Reg5_sff_spillover0_nsk0_xgr0_knspil0_sep0_LFlimit1_emsbase0_countec0_GovRev0_etaa0.79_lgd1.png}
%\end{minipage}	
\begin{minipage}[]{0.32\textwidth}
\centering{\footnotesize{(c) Green-to-fossil energy ratio}}
%	\captionsetup{width=.45\linewidth}
\includegraphics[width=1\textwidth]{../../codding_model/own_basedOnFried/optimalPol_010922_revision/figures/all_13Sept22/CompTauf_bytaul_Reg5_GFF_spillover0_nsk0_xgr0_knspil0_sep0_LFlimit1_emsbase0_countec0_GovRev0_etaa0.79_lgd0.png}
\end{minipage}	
\begin{minipage}[]{0.32\textwidth}
\centering{\footnotesize{(d) Fossil growth}}
%	\captionsetup{width=.45\linewidth}
\includegraphics[width=1\textwidth]{../../codding_model/own_basedOnFried/optimalPol_010922_revision/figures/all_13Sept22/CompTauf_bytaul_Reg5_gAf_spillover0_nsk0_xgr0_knspil0_sep0_LFlimit1_emsbase0_countec0_GovRev0_etaa0.79_lgd0.png}
\end{minipage}			
\begin{minipage}[]{0.32\textwidth}
\centering{\footnotesize{(e) Non-energy growth}}
%	\captionsetup{width=.45\linewidth}
\includegraphics[width=1\textwidth]{../../codding_model/own_basedOnFried/optimalPol_010922_revision/figures/all_13Sept22/CompTauf_bytaul_Reg5_gAn_spillover0_nsk0_xgr0_knspil0_sep0_LFlimit1_emsbase0_countec0_GovRev0_etaa0.79_lgd0.png}
\end{minipage}			
\begin{minipage}[]{0.32\textwidth}
\centering{\footnotesize{(f) Green growth}}
%	\captionsetup{width=.45\linewidth}
\includegraphics[width=1\textwidth]{../../codding_model/own_basedOnFried/optimalPol_010922_revision/figures/all_13Sept22/CompTauf_bytaul_Reg5_EY_spillover0_nsk0_xgr0_knspil0_sep0_LFlimit1_emsbase0_countec0_GovRev0_etaa0.79_lgd0.png}
\end{minipage}			%\begin{minipage}[]{0.32\textwidth}
%\centering{\footnotesize{(b) Deviation in carbon tax}}
%%	\captionsetup{width=.45\linewidth}
%\includegraphics[width=1\textwidth]{../../codding_model/own_basedOnFried/optimalPol_010922_revision/figures/all_13Sept22/CompTaufPER_bytaul_Equlab_Reg5_tauf_spillover0_nsk1_xgr1_knspil1_sep0_LFlimit1_emsbase0_countec0_GovRev0_etaa0.79_lgd0.png} \end{minipage}		
 	\floatfoot{Notes:{  Except for panel (b) all graphs show variables in levels when the emission limit has to be met (i) in a scenario without income taxation, $\tau_{\iota}=0$, the gray dashed graph, and (ii) in a scenario with a progressive income tax, $\tau_\iota=0.181$, the solid graph. Panel (b) depicts the deviation in the required carbon tax in the scenario with progressive income tax relative to the carbon tax necessary in the world without progressive income tax. }}
 \end{figure} 
 
 % necessary carbon tax
 %     0.9371    1.0025    1.0678    1.1334    1.2000    1.2674    2.7431    2.8254    2.9098    2.9961    3.0841    3.1735
 
Figure \ref{fig:Limit_nsk0_xgr0_know} shows the necessary carbon tax and allocations by progressivity scenario. 
The necessary carbon tax ranges from US\$937.1 in 2022 prices in 2020 to around US\$3,173.5\$ in 2022 prices in 2075. It is lower when labor is taxed (panel (a)). The deviation of the carbon tax reaches -10\% in initial periods but diminishes over time to approximately -7.25\% (panel (b)).
The smaller deviation under the net-zero emission limit results primarily from the tightness of the emission limit itself.\footnote{\ Abstracting from all model features discussed earlier, the carbon tax approaches the one without progressive income tax, as well, as the emission limit gets tighter. I show the deviation in emission limits in figure \ref{fig:zeromod_tauf} in appendix \ref{app:eff_feat_exp}. Absent heterogeneous skills, tax progressivity has no adverse compositional effect which should be countered by the carbon tax and which might be changing over time. }

%On top, adverse compositional effects of the progressive income tax aggravate. The reason is as follows. The tighter emission limit makes a higher carbon tax necessary. This again aggravates the compositional effect of the labor tax:  as the carbon tax increases demand for high-skill the difference in leisure across skill types broadens. Then high-skill labor is more responsive to a change in the tax progressivity.\footnote{\ For a more in depth discussion see section \ref{subsec:eff_cc} above.}


%The compositional effect of progressive income taxes, intensifies over time but only minimally;% see figure \ref{fig:Efftaul_nsk0_xgr0_know_app} in appendix \ref{app:neccab}, where I contrast the economy with a constant carbon tax but with and without a progressive income tax. 
%Over time, the progressive income tax lowers the green-to-fossil energy ratio more. As discussed earlier, the compositional effect of the progressive income tax intensifies

%The reason is that the constant carbon tax entails a transition to green production so that demand for high-skill labor increases. The higher hours of high-skill labor prior to setting a progressive tax intensifies the impact of the progressive tax, because the difference in leisure across labor type broadens. This more adverse effect of a progressive income tax, again, contributes to the smaller reduction in carbon taxes. %Rather, over time, the economy can profit less from the reductive effect of progressive income taxes to meet emission limits. 

% compositional effect of tauf
\begin{comment}
Panel (c) depicts the ratio of green-to-fossil output, and (d) shows the energy share of GDP. The joint policy with progressive income tax implies a smaller green-to-fossil ratio and a higher energy share.  
The difference in ratios is mainly driven by the gap in carbon taxes and the progressive income tax adds to this effect by diminishing the ratio of high- to low-skilled labor. 

content...
\end{comment}
%\tr{Add below earlier to effect of carbon tax}
%A carbon tax makes green goods relatively cheaper and energy goods more expensive.  Endogenous growth contributes to the rise in green energy share. A market size effect directs research to the green sector. In contrast, a smaller carbon tax results in a higher share of non-energy scientists due to the complementarity of energy and non-energy goods. Nevertheless, in equilibrium, the energy to GDP ratio falls with the carbon tax.\\ 
%\tr{-----------until here ------------------}

%First, as explained earlier, a carbon tax induces more green and less non-energy research (the latter being due to the higher price of energy goods when carbon is taxed). 
 
%This contributes to a smaller green energy share. Furthermore, a progressive income tax implies a smaller energy share. The reason is that non-energy goods are less skill intense than energy goods. As the ratio of high-to-low skill falls, non-energy goods become cheaper. This effect counteracts the use of fossil energy. Yet, the effect of income tax progressivity is small. 
%\textbf{check if this result remains with equal capital shares: it does: but the energy share is increasing with equal labor shares since productivity rise in non-energy does not redirect labor. Hence, a reduction in high-skill supply has the same qualitative effect on the energy ratio. But it seems to be smaller... } 

 % effect on growth
I want to use this section to highlight the variation in how policies affect growth rates.
Under the joint policy, the economy sees higher technology growth in the fossil and the non-energy sector and smaller ones in the green sector (panels (d) to (f)). 
%Again, changes in growth rates are mainly induced by the smaller carbon tax as the effect of progressive income taxes is small. % Qualitatively, the progressive income tax counters higher non-energy research by making non-energy goods less expensive. On the other hand, it contributes to the reduction in the green-to-fossil research ratio through a market size mechanism.  In sum, the higher amount of energy scientists outweighs the recomposition towards fossil research so that green research increases in presence of a progressive income tax. Again, effects are small. 
%To summarize, a progressive income tax has an adverse effect on the green energy share directly and indirectly through its impact on the effectiveness of the carbon tax.
%Then again,
This result suggests that under a progressive income tax the emission limit can be reached at higher technology growth rates.
Knowledge spillovers emerge as an important driver of the effect of the joint policy on growth rates as opposed to a carbon-tax-only policy. 
%In the following, I will briefly point to the role of model features in shaping the necessary carbon tax. 
% \begin{itemize}
% 	\item dynamics: I expect rising over time due to capital ratio heterogeneity and knowledge spillovers
% 	\item role of taul as a level effect versus compositional effect
% 	\item advantages under policy tuple progressive income tax and lower fossil tax as opposed to only fossil tax?
% \end{itemize}

%\paragraph{The role of knowledge spillovers}
Figure \ref{fig:Limit_nsk0_xgr0_know_Devs} shows the necessary carbon tax absent knowledge spillovers and compares variables in the benchmark model to the model without knowledge spillovers. 
Knowledge spillovers render a higher fossil tax in all periods and a more aggressive environmental policy over time necessary (panel (a)).
This underlines the previous observation that emissions rise faster especially in future periods when knowledge spills from the green to the fossil sector. 

%On the other hand, knowledge spillovers mitigate the adverse compositional effect of progressive income taxes on the green-to-fossil energy share.
%COMMENT: THIS IS TRUE BUT THERE IS NO EVIDENCE FOR AN IMPACT ON TAUF

 %\textbf{comparison effect taul with and without knowledge spillovers}%As a result, the required fossil tax deviates more from the fossil tax when no progressive income tax is present. 
 

The combination of a progressive income tax and a lower carbon tax leads to a smaller decline in green growth in future periods when knowledge spills over from the fossil sector (panel (e)). In addition, fossil growth rises less (panel (d)). Fossil grows more slowly as it is hampered by knowledge spillovers as leading sector. As a result, the green-to-fossil energy ratio falls less in the model with knowledge spillovers when a progressive income tax is present. Knowledge spillovers, therefore, lower the costs, i.e. an adverse green-to-fossil ratio, of a combined environmental policy consisting of a progressive income tax and a carbon tax. The gains are higher growth rates in the fossil and non-energy sector.
The smaller reduction in the green-to-fossil ratio induced by knowledge spillovers, allows the planner to choose a lower carbon tax in future periods, compare panel (b).
 %the required carbon tax can deviate more from the necessary carbon tax without progressive income tax then absent knowledge spillovers. %In other words, due to the better green-to-fossil energy ratio resulting from knowledge spillovers, a lower carbon tax can be set in future periods. 
%In the initial periods, however, the carbon tax in the benchmark model can deviate less from its counterpart since \tr{maybe because the reductive effect of the progressive tax is smaller. I thought knowledge spillovers mitigate the compositional effect of labor taxes.}
%The energy-to-output ratio exceeds its counterpart under the scenario with only a carbon tax approximately by the same percentage in early periods; later, however, it is higher due to the lower carbon tax.  \footnote{\ It is not that knowledge spillovers make a fossil tax more costly in terms of technology growth in fossil or non-energy. In fact, growth in these two sectors is closer to the allocation with a combined policy.   Non-energy growth is higher because non-energy research rises more as the non-energy good is relatively more expensive absent knowledge spillovers. Fossil growth faster as it is hampered by knowledge spillovers as leading sector.}

This section preempts some of the optimal policy results which I will discuss in section \ref{sec:res}: Progressive income taxes are optimally used to substitute for carbon taxes due to knowledge spillovers. This policy achieves higher growth rates in the fossil and non-energy sector at the cost of slower green growth.
 But only when the costs of lower green growth in the future is mitigated by knowledge spillovers, the Ramsey planner finds such a policy optimal. 

I discuss the effects of heterogeneous skills, heterogeneous labor shares, and endogenous growth in appendix \ref{app:eff_feat_exp}. 
 \begin{figure}[h!!]
	\centering
	\caption{Comparison in deviations with and without knowledge spillovers}\label{fig:Limit_nsk0_xgr0_know_Devs}
		\begin{minipage}[]{0.32\textwidth}
		\centering{\footnotesize{(a) Tax per ton of carbon in US\$}}
		%	\captionsetup{width=.45\linewidth}
		\includegraphics[width=1\textwidth]{../../codding_model/own_basedOnFried/optimalPol_010922_revision/figures/all_13Sept22/CompTauf_bytaul_Reg5_Tauf_spillover0_nsk0_xgr0_knspil1_sep0_LFlimit1_emsbase0_countec0_GovRev0_etaa0.79_lgd1.png}
	\end{minipage}	
\begin{minipage}[]{0.32\textwidth}
		\centering{\footnotesize{(b) Deviation in carbon tax}}
		%	\captionsetup{width=.45\linewidth}
		\includegraphics[width=1\textwidth]{../../codding_model/own_basedOnFried/optimalPol_010922_revision/figures/all_13Sept22/CompTaufPER_bytaul_KN_Reg5_Tauf_spillover0_nsk0_xgr0_knspil0_sep0_LFlimit1_emsbase0_countec0_GovRev0_etaa0.79_lgd1.png} 
	\end{minipage}	
	\begin{minipage}[]{0.32\textwidth}
		\centering{\footnotesize{(c) Green-to-fossil energy ratio}}
		%	\captionsetup{width=.45\linewidth}
		\includegraphics[width=1\textwidth]{../../codding_model/own_basedOnFried/optimalPol_010922_revision/figures/all_13Sept22/CompTaufPER_bytaul_KN_Reg5_GFF_spillover0_nsk0_xgr0_knspil0_sep0_LFlimit1_emsbase0_countec0_GovRev0_etaa0.79_lgd0.png}
	 \end{minipage}	
	\begin{minipage}[]{0.32\textwidth}
		\centering{\footnotesize{(d) Fossil growth}}
		%	\captionsetup{width=.45\linewidth}
		\includegraphics[width=1\textwidth]{../../codding_model/own_basedOnFried/optimalPol_010922_revision/figures/all_13Sept22/CompTaufPER_bytaul_KN_Reg5_gAf_spillover0_nsk0_xgr0_knspil0_sep0_LFlimit1_emsbase0_countec0_GovRev0_etaa0.79_lgd0.png} 
	\end{minipage}	
	\begin{minipage}[]{0.32\textwidth}
		\centering{\footnotesize{(e) Non-energy growth}}
		%	\captionsetup{width=.45\linewidth}
		\includegraphics[width=1\textwidth]{../../codding_model/own_basedOnFried/optimalPol_010922_revision/figures/all_13Sept22/CompTaufPER_bytaul_KN_Reg5_gAn_spillover0_nsk0_xgr0_knspil0_sep0_LFlimit1_emsbase0_countec0_GovRev0_etaa0.79_lgd0.png} \end{minipage}	
	\begin{minipage}[]{0.32\textwidth}
		\centering{\footnotesize{(f) Green growth}}
		%	\captionsetup{width=.45\linewidth}
		\includegraphics[width=1\textwidth]{../../codding_model/own_basedOnFried/optimalPol_010922_revision/figures/all_13Sept22/CompTaufPER_bytaul_KN_Reg5_gAg_spillover0_nsk0_xgr0_knspil0_sep0_LFlimit1_emsbase0_countec0_GovRev0_etaa0.79_lgd0.png} 
	\end{minipage}	
\floatfoot{Notes: \footnotesize{The panel (a) shows the carbon tax when there are no knowledge spillovers. Panels (b) to (f) show the percentage difference in allocations in the model with to the model without progressive income tax. The solid graph refers to the benchmark model, while the gray dashed graph represents deviations in the model without knowledge spillovers, $\phi=0$. }}	
\end{figure}
\clearpage

%Absent knowledge spillovers, growth in the non-energy sector is indeed higher than absent a progressive tax, but it becomes smaller when a more aggressive fossil tax is required. Then, the reductive effect of the progressive income tax does not suffice to diminish the fossil tax in a way that non-energy research increases. Then, the progressive income tax in addition to the fossil tax both lower non-energy research and growth. With knowledge spillovers, non-energy research and growth are higher; a positive spillover effect of fossil growth. The green sector, too, profits from knowledge spillovers from the fossil sector. Green growth eventually is higher under the policy with progressive 

%However, overall, the benefits of a lower fossil tax in presence of a higher tax progressivity come at the cost of lower consumption and a smaller green-to-fossil ratio and a higher energy share. Yet, the latter two are acceptable due to a smaller level of production. 
