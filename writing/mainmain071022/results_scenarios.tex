\subsection{A constant carbon tax}\label{subsec:exp}


We are now equipped to study how a carbon tax  equal to US\$185 (in 2020 prices) affects the economy. The value was calculated as reflecting the social costs of carbon by a joint research effort of the \textit{Resources for the Future} (RFF), an independent research institution, and the University of Berkeley \citep{Rennert2022ComprehensiveCO2}. 
%Comparing the level of emissions resulting under the constant carbon tax to the emission limit based on the IPCC report, reveals that a stronger reduction in emissions is required. Therefore, I calculate a necessary environmental tax to satisfy the US emission limit in section  \ref{subsec:meetlim}. A progressive income tax lowers the required environmental tax by between 10 and 7\%. Furthermore, the allocation wit is characterized by higher technology growth in the fossil and non-energy sector.
% 


%\subsubsection{}\label{subsec:eff_cc}
% \begin{itemize}
% 	\item  emissions continue to rise due to knowledge spillovers \checkmark
% 	\item  effect on research (non-energy research) \checkmark
% 	\item taul has level effect on emissions, interaction: compositional effect but small (but could become relevant through disadvantageous effect of tauf on non-energy research) \checkmark 	
% \end{itemize}

\begin{figure}[h!!]
	\centering
	\caption{Effect of a constant carbon tax equal to 185\$ per ton of carbon }\label{fig:Leveltauf_nsk0_xgr0_know}		
\begin{subfigure}[]{0.4\textwidth}
	\caption{Net emissions}
	%	\captionsetup{width=.45\linewidth}
	\includegraphics[width=1\textwidth]{../../codding_model/own_basedOnFried/optimalPol_010922_revision/figures/all_13Sept22/CompTauf_bytaul_Reg5_Emnet_spillover0_nsk0_xgr0_knspil0_sep0_LFlimit0_emsbase0_countec0_GovRev0_etaa0.79_lgd1.png}
\end{subfigure}	
 \begin{minipage}[]{0.1\textwidth}
	\
\end{minipage}
\begin{subfigure}[]{0.4\textwidth}
\caption{Fossil production}
%	\captionsetup{width=.45\linewidth}
\includegraphics[width=1\textwidth]{../../codding_model/own_basedOnFried/optimalPol_010922_revision/figures/all_13Sept22/PerdifNoTauf_regime5_CompTaul_F_spillover0_nsk0_xgr0_knspil0_sep0_LFlimit0_emsbase0_countec0_GovRev0_etaa0.79_lgd1.png}
\end{subfigure}

\vspace{3mm}
\begin{subfigure}[]{0.4\textwidth}
	\caption{Green production }
	%	\captionsetup{width=.45\linewidth}
	\includegraphics[width=1\textwidth]{../../codding_model/own_basedOnFried/optimalPol_010922_revision/figures/all_13Sept22/PerdifNoTauf_regime5_CompTaul_G_spillover0_nsk0_xgr0_knspil0_sep0_LFlimit0_emsbase0_countec0_GovRev0_etaa0.79_lgd0.png}
\end{subfigure}
 \begin{minipage}[]{0.1\textwidth}
	\
\end{minipage}
\begin{subfigure}[]{0.4\textwidth}
\caption{Energy share to GDP}
%	\captionsetup{width=.45\linewidth}
\includegraphics[width=1\textwidth]{../../codding_model/own_basedOnFried/optimalPol_010922_revision/figures/all_13Sept22/PerdifNoTauf_regime5_CompTaul_EY_spillover0_nsk0_xgr0_knspil0_sep0_LFlimit0_emsbase0_countec0_GovRev0_etaa0.79_lgd0.png}
\end{subfigure}

\floatfoot{Notes:{ Panel (a) shows levels of net emissions under a constant carbon tax equal to 185\$ for a world with progressive income taxation at $\tau_l=0.181$, the solid graph, and without progressive income taxation, $\tau_l=0$. The thin dotted graph shows the emission limit suggested by the IPCC. Panels (b) to (d) show the percentage deviations from the business-as-usual policy in the economy (i) with and (ii) without income tax by the solid and dashed graphs, respectively. }}
\end{figure} 

\paragraph{Static effect of a carbon tax}
% 1) reallocation of demand by energy producers and by final good producers.
 Figure \ref{fig:Leveltauf_nsk0_xgr0_know} shows the effect of a constant carbon tax in a world with and without labor income tax represented by the solid and the dashed graphs, respectively. In this and all following figures, the x-axis indicates the first year of the 5 year period to which the variable value corresponds.\footnote{\ Figure \ref{fig:Leveltauf_nsk0_xgr0_add} in Appendix \ref{app:polexp_cc} shows research related and other variables. }  
 
 A carbon tax equal to 185\$ diminishes emissions by around 42.4\% initially relative to the business-as-usual (BAU) policy without carbon tax.
 Panel (a) depicts how net emissions evolve under the carbon tax. The level of emissions is smaller in presence of a labor income tax with $\tau_{\iota}=0.181$. Importantly, net emissions exceed the emission limit  derived previously; see the dotted line in Panel (a). Panels (b) to (d) show the percentage deviation from the BAU allocation for fossil energy and the energy share in final production. 
 
A carbon tax operates as follows. As energy producers face a higher price for fossil energy, they lower demand for fossil and rise demand for green energy. Fossil production falls, and green production rises (Panels (b) and (c)).
 The tax on fossil goods also increases the price for energy goods relative to non-energy goods on impact. Final good producers recompose their inputs towards non-energy goods. But, the recomposition is limited as energy and non-energy goods are complements. The energy share to GDP declines (Panel (d)).  
 
 % 2) effect on research
 The shift in demand by energy and final good producers induces a reallocation of research. 
 As the market size of the green sector increases, profitability of research in the green sector rises. In contrast, research in the fossil sector falls. The ratio of green to fossil research increases; this makes the green good even cheaper contributing to an increase in the green-to-fossil energy ratio. Non-energy research falls because the energy good gets more expensive and knowledge spillovers to the energy sector direct research away from the non-energy sector\footnote{\ I examine the effect of a carbon tax on non-energy research in Appendix \ref{app:polexp_cc}. Contrary to theory, the price effect does not dominate and directs research to the more expensive good. The reason are heterogeneous labor shares. In fact, knowledge spillovers dominate the direction of research across energy and non-energy goods.}. 
 
 
 
 %3) Supply side effect
% The effectiveness of the carbon tax is amplified by a supply-side mechanism. The lower demand for fossil reduces the demand for labor from the fossil sector. As a result, more labor is available for green energy production, and green supply increases. The price of green goods may fall despite higher demand. This supply-side channel facilitates a transition to green energy; however, it is muted due to a low labor share in the green sector and heterogeneity in labor input goods.

% EFFECT OF TAUL: income tax progressivity does not change growth in the non-energy sector,\tr{ since the level effect is absorbed by price changes} CHECK THIS. A compositional effect arises when skills are heterogeneous. In equilibrium, high-skill workers work more hours due to a higher wage rate. Therefore, the marginal gains from leisure are higher for this type. When the tax scheme becomes more progressive, high-skill workers are more responsive and curb their labor supply more.\footnote{\ They have to be compensated by a higher wage rate for a further reduction of leisure. The additional unit of leisure is more valuable to them.}  A more progressive tax scheme makes green labor more expensive and green production reduces more than fossil production. More research in the fossil sector amplifies this compositional effect. 

\paragraph{Dynamics}

%%%%%%%%%%%%%%%%%%%%%%%%%%%%%%%%%%%%%
%% knowledge spillovers
%%%%%%%%%%%%%%%%%%%%%%%%%%%%%%%%%%%%%
Over time, the effectiveness of the carbon tax to lower fossil production declines from 38\% to 37\%. There are two model features explaining the declining effectiveness of carbon taxes: knowledge spillovers and heterogeneous labor shares. Hence, to meet a net-zero emission limit, a continuous intervention is necessary. 

This finding is in contrast to the result by \cite{Acemoglu2012TheChange} who abstract from knowledge spillovers and heterogeneity in labor shares. They conclude that when dirty and clean goods are sufficient substitutes, a temporary intervention suffices to prevent too high pollution. In contrast, in the present model, knowledge spillovers and heterogeneous labor shares 
call for an ever increasing carbon tax even though the green and the energy good are sufficiently substitutable. I now examine both mechanisms in more detail.

Consider, first, the effect of knowledge spillovers.
Initially, the fossil tax reduces research in the fossil sector, however, as green technology advances, knowledge spillovers from the green sector make fossil research more profitable again, and demand for fossil scientists resurges.
It is not only that a constant amount of researchers becomes more productive but also a change in the equilibrium level of fossil researchers which intensifies the effect of knowledge spillovers. This mechanism explains the quick rise in emissions under a constant fossil tax.\footnote{\ Figure \ref{fig:Leveltauf_nsk0_xgr0_noknow} shows the effect of a constant carbon tax in a model without knowledge spillovers, $\phi=0$. The rise in net emissions over time is muted and a constant carbon tax becomes more effective over time in reducing fossil production.  Absent knowledge spillovers more research is allocated to the green and non-energy sector.} 

Therefore, when knowledge spillovers are strong, reducing emissions to a net-zero emission limit requires a continues intensification of environmental intervention. In its extreme, growth might have to stop eventually in order to prevent the fossil sector from growing too much.

%%%%%%%%%%%%%%%%%%%%%%%%%%%%%%%%%%%%%%%%%%%
% \paragraph{Role of heterogeneous labor shares}
%%%%%%%%%%%%%%%%%%%%%%%%%%%%%%%%%%%%%%%%%%%%%%
Second, the green sector has the smallest labor share. Labor is more important in the fossil and most important in the non-energy sector. This heterogeneity lowers the effectiveness of the fossil tax through a supply-side mechanism. 
Consider a situation in which the green and the fossil sector both use the same share of labor. A reduction in demand for labor in the fossil sector eases labor costs for the green sector. When, however, the green sector only uses a small share of labor, the higher labor supply does not lower green production costs as much, and the green good remains more expensive. The share of green energy and labor rises less. Panel (b)  in Figure \ref{fig:Leveltauf_nsk0_xgr0_noknow} displays the behavior of key variables in a model with homogeneous labor shares across sectors.\footnote{\ In this counterfactual calibration, I set capital shares equal across sectors to the average in the baseline calibration: $\alpha_g=\alpha_f=\alpha_n=0.66$. }

When labor shares are equal, the effectiveness of the carbon tax to lower fossil production increases over time. 
Absent a carbon tax, a smaller labor share in the green sector raises the share of labor allocated to the fossil sector over time. As the fossil sector becomes more productive, the marginal product of labor in the fossil sector increases more and labor transitions to the fossil sector. The carbon tax counters this mechanism. %\textbf{Is it the BAU allocation that changes or the policy one?} % (Note that consequently the economy is not on a bgp in BAU)

%Endogenous growth intensifies this adverse effect of heterogeneous capital shares: as the market size of green goods is depressed, the fossil tax does not boost green research as much as with equal capital shares. 
%In sum, heterogeneous labor shares imply an increasing path of emissions over time under the fossil tax. 
%This feature of the economy also calls for a carbon tax increasing over time.


Absent knowledge spillovers and with equal labor shares, a constant carbon tax suffices to lower emissions over time. Then, endogenous growth directs research away from the fossil sector so that emissions fall over time. Panel (c) in Figure \ref{fig:Leveltauf_nsk0_xgr0_noknow} shows the result in this model variation. This finding replicates the result in \cite{Acemoglu2012TheChange}.


\paragraph{Effect of the income tax}
 %\textit{Annotation (SEP=0):  a progressive income tax raises research in fossil and in green \ar in energy goods, and lowers research in non-energy (the progressive tax makes energy goods more expensive which are more skill intense, being complements, research for energy becomes more profitable) But the effect ist small. This shift in innovation counters a reduction in the energy to output share (energy is less labor intense making labor more expensive it goes to the non-energy sector plus a reduction in skills; Q is this reduction also present in model with equal cap shares?); Furthermore, a prgressive income tax lowers the green to fossil ratio, intensified by response in research.  }
 
 The presence of a progressive income tax lowers the level of emissions; compare the solid and the dashed graph in Panel (a) in Figure \ref{fig:Leveltauf_nsk0_xgr0_know}. As labor supply reduces, output shrinks, and emissions fall. 
 However, there are compositional effects of a progressive income which (i) affect the economic structure, and (ii) interact with the effectiveness of the carbon tax. Yet, interaction effects are small. I will explain each statement in turn. 
 
% \paragraph{Compositional effect of progressive income tax}
 % compositional effect
 A progressive income tax lowers the green-to-fossil output ratio but diminishes the energy-to-GDP ratio. %\footnote{\ Figure \ref{fig:Efftaul_nsk0_xgr0_know} in Appendix \ref{app:polexp_cc} displays the effect of a progressive income tax in presence of a constant carbon tax. }
 The compositional effect of a progressive income tax originates from the asymmetric reaction of high- and low-skill workers. High-skill workers reduce their labor supply more, as the after-tax wage declines in response to a higher tax progressivity. Since they work more prior to a rise in income tax progressivity, they require a higher wage rate to be compensated for the marginal unit of labor due to concavity of the utility function in leisure. Second, a more progressive tax reduces after-tax labor income more the higher pre-tax income. %\tr{The effect of a marginal increase in income tax progressivity intensifies with the level of pre-tax income.}
 Green production is skill biased as opposed to fossil production.
 As a  consequence, green production becomes more expensive, while fossil production becomes cheaper. The price of non-energy goods, which are less skill-intense than energy goods, falls, too. Therefore, energy producers substitute fossil for green energy, and final good producers turn to non-energy goods. The former effect raises emissions; the latter diminishes emissions.
 
 % research
 
 Research responds to the change in demand and in prices. First, non-energy research is less profitable due to its smaller price, while the amount of non-energy goods does not rise sufficiently due to complementarity of energy and non-energy goods. As a result, research is directed towards the energy sector. The share of non-energy researchers reduces, albeit minimally. %; see Panel \eqref{figpan:nonre} in Figure \ref{fig:Efftaul_nsk0_xgr0_know}. %\footnote{\ The effect of the progressive income tax also prevails with heterogeneous labor shares and absent knowledge spillovers. The price-effect dominates the direction of research.} 
 Second, focusing solely on the allocation of researchers between the fossil and green sector, the relatively higher supply of low-skill labor raises the market size of the fossil good. Since intermediate energy goods are sufficient substitutes, the market size effect dominates the price effect and research shifts from green to fossil. Nevertheless, due to the reallocation of research to energy goods in general, the green sector, too, sees a rise in researchers. Overall, the share of green-to-fossil research drops. The reallocation of research towards fossil goods intensifies the fall in green-to-fossil energy . The shift in research to the energy sector mitigates the decline in the energy share. 
 
 Despite these compositional effects of the labor tax on research activity, there 
 is no level effect of the progressive income tax on innovation in equilibrium. To differentiate the compositional from the level effect, I consider the model with homogeneous skills. With only one skill type, there is no compositional effect of the labor tax. In this model, the equilibrium amount of scientists remains unchanged by a progressive income tax. Indeed, demand for innovation reduces in response to a progressive income tax since less labor is available to work with technology. However, at the same time, scientists are willing to accept a lower wage rate, since consumption of the household reduces. In equilibrium, the reduction in demand is absorbed by a change in the wage rate and the level of aggregate research remains unchanged. 
 This is an important finding as it allows to differentiate the role of labor income taxation in the optimal policy.  
 
 
  
\paragraph{Interaction of income and carbon taxes}
The compositional effect of a tax on labor interacts with the impact of the carbon tax.
  Quantitatively, the effect of the carbon tax seems by and large unaffected by the value of income tax progressivity.
 Yet, there is a smaller reduction in fossil energy visible in Panel (b) in Figure \ref{fig:Leveltauf_nsk0_xgr0_know}.
This discrepancy emerges from the effect of a carbon tax on skill supply which is affected by the income tax. 

    The carbon tax changes the skill premium since demand for green-specific high-skill labor increases. High-skill hours in equilibrium rise, while hours of low-skilled workers reduce.  A progressive income tax mutes the effect of changes in the wage rate on labor supply. The reason is that the elasticity of after-tax labor income with respect to pre-tax labor income reduces with a higher progressivity. This diminishes the supply response in the skill ratio to the carbon tax, and production costs of the green good remain higher. Therefore, the reduction in fossil production is muted in presence of a progressive income tax scheme. 
   
%Overall, the smaller level in production induced by the tax on labor helps to reduce net emissions below the level without income tax despite the adverse compositional effects; see Panel (a) in Figure \ref{fig:Leveltauf_nsk0_xgr0_know}.
 
% \begin{comment}
% \paragraph{Role of heterogeneous skills}
% When skills are heterogeneous, emissions diminish because availability of the labor input good reduces. The effect of the fossil tax is similar to the model with homogeneous skills, yet, the fossil tax now affects labor.\footnote{\ With homogeneous skills, the equilibrium level of labor is unaffected by the fossil tax due to the log-utility of consumption implying that income and substitution effects of the wage rate cancel.}
% With two types of skills and heterogeneous high-skill share, the fossil tax has a diminishing effect on low-skill hours in equilibrium and raises high-skill hours. The reason is that a higher demand for green produce translates into a higher demand for the green-specific labor good which is high-skill biased. 
% 
% The effect on working hours through the fossil tax is affected by the presence of a progressive income tax scheme. The higher income tax progressivity, the smaller the effect of a change in the wage rate on labor supply, since the elasticity of post-tax labor income with respect to pre-tax labor income falls. 
% This asymmetry has an effect on research. A smaller reduction in low-skill supply makes non-energy goods, which are more low-skill intense than energy goods, cheaper. The lower price of non-energy goods and its complementarity to energy goods (so that demand for energy remains high) explains a stronger reduction in non-energy research when the income tax scheme is progressive.\footnote{\ The price for non-energy research also reduces more in equilibrium. The reduction in non-energy research is demand-side driven.} Energy research reduces less under a progressive income tax scheme. In addition, since a relatively higher low-skill supply boosts the market share of fossil producers, fossil research diminishes less than green research in presence of a progressive income tax. 
% Consequently, the presence of a progressive income tax scheme counters the intention to lower fossil and energy production when skills are heterogeneous. Nevertheless, is numerically negligible. 
%  
%  content...
%  \end{comment}
  
  
\subsection{Meeting the emission limit}\label{subsec:meetlim}
 
 The previous section made apparent that, first, the carbon tax suggested by the RFF does not cause emissions in line with the IPCC's emission limit.  Second, it showed that model dynamics call for an increasing carbon tax. In this section, I calculate the necessary carbon tax to meet the emission limit. I compare the resulting tax and allocations for the policy regimes with labor income tax to a carbon-tax-only policy.
 

 \begin{figure}[h!!]
 	\centering
 	\caption{Necessary carbon tax with and without progressive income tax  }\label{fig:Limit_nsk0_xgr0_know}		
 	\begin{subfigure}[]{0.4\textwidth}
 		\caption{Tax per ton of carbon in 2022 US\$}
 		%	\captionsetup{width=.45\linewidth}
 		\includegraphics[width=1\textwidth]{../../codding_model/own_basedOnFried/optimalPol_010922_revision/figures/all_13Sept22/CompTauf_bytaul_Reg5_Tauf_spillover0_nsk0_xgr0_knspil0_sep0_LFlimit1_emsbase0_countec0_GovRev0_etaa0.79_lgd1.png}
 	\end{subfigure}	
 \begin{minipage}[]{0.1\textwidth}
\
 \end{minipage}
 	\begin{subfigure}[]{0.4\textwidth}
 	\caption{Percentage deviation of carbon tax}
 	%	\captionsetup{width=.45\linewidth}
 	\includegraphics[width=1\textwidth]{../../codding_model/own_basedOnFried/optimalPol_010922_revision/figures/all_13Sept22/CompTaufPER_bytaul_Reg5_Tauf_spillover0_nsk0_xgr0_knspil0_sep0_LFlimit1_emsbase0_countec0_GovRev0_etaa0.79_lgd0.png} \end{subfigure}		
	\vspace{2mm}
	
	 \begin{minipage}[]{0.04\textwidth}
		\
	\end{minipage}
\begin{subfigure}[]{0.4\textwidth}
	\caption{Green growth}
	%	\captionsetup{width=.45\linewidth}
	\includegraphics[width=1\textwidth]{../../codding_model/own_basedOnFried/optimalPol_010922_revision/figures/all_13Sept22/CompTauf_bytaul_Reg5_gAg_spillover0_nsk0_xgr0_knspil0_sep0_LFlimit1_emsbase0_countec0_GovRev0_etaa0.79_lgd0.png}
\end{subfigure}	 
\begin{minipage}[]{0.1\textwidth}
\
\end{minipage}
\begin{subfigure}[]{0.4\textwidth}
\caption{ Non-energy growth}
%	\captionsetup{width=.45\linewidth}
\includegraphics[width=1\textwidth]{../../codding_model/own_basedOnFried/optimalPol_010922_revision/figures/all_13Sept22/CompTauf_bytaul_Reg5_gAn_spillover0_nsk0_xgr0_knspil0_sep0_LFlimit1_emsbase0_countec0_GovRev0_etaa0.79_lgd0.png}
\end{subfigure}				%\begin{minipage}[]{0.32\textwidth}
%\centering{\footnotesize{(b) Deviation in carbon tax}}
%%	\captionsetup{width=.45\linewidth}
%\includegraphics[width=1\textwidth]{../../codding_model/own_basedOnFried/optimalPol_010922_revision/figures/all_13Sept22/CompTaufPER_bytaul_Equlab_Reg5_tauf_spillover0_nsk1_xgr1_knspil1_sep0_LFlimit1_emsbase0_countec0_GovRev0_etaa0.79_lgd0.png} \end{minipage}		
 	\floatfoot{Notes:{  Except for Panel (b) all graphs show variables in levels when the emission limit has to be met (i) in a scenario without income tax, $\tau_{\iota}=0$, dashed graphs, and (ii) in a scenario with labor income tax, $\tau_\iota=0.181$, the solid graphs. Panel (b) depicts the deviation in the required carbon tax in the scenario with  income tax relative to the scenario without income tax. }}
 \end{figure} 
 
 % necessary carbon tax
 %     0.9371    1.0025    1.0678    1.1334    1.2000    1.2674    2.7431    2.8254    2.9098    2.9961    3.0841    3.1735
 
Figure \ref{fig:Limit_nsk0_xgr0_know} shows the results.
Consider Panel (a). The necessary carbon tax ranges from US\$937  in the 2020-2024 period to around US\$3,174 in the 2075-2080 period (both in 2022 prices). It is lower when labor is taxed. The deviation of the carbon tax reaches -10\% in initial periods but diminishes over time to approximately -7.25\% (Panel (b)).\footnote{\ The smaller deviation under the net-zero emission limit results primarily from the tightness of the emission limit itself. Abstracting from all model features discussed earlier, the carbon tax approaches the one without progressive income tax, as well, as the emission limit gets tighter. Absent heterogeneous skills, tax progressivity has no adverse compositional effect which should be countered by the carbon tax and which might be changing over time. }

The smaller carbon tax results in a lower green-to-fossil output ratio, a higher energy share to GDP, and slower green growth (Panel (c)). Yet, the reduction in economic activity induced by the labor income tax ensures that the emission limit is satisfied. On the upside, the combined policy enables more growth in the non-energy sector (Panel (d)). Growth in the fossil sector is close to zero in both scenarios but slightly higher under the combined policy (not shown).  %This exercise highlights the effect of a carbon tax on research activity. 
% \begin{figure}[h!!]
%	\centering
%	\caption{Necessary carbon tax with and without progressive income tax  no knowledge spillovers}\label{fig:Limit_nsk0_xgr0_noknow}		
%	\begin{subfigure}[]{0.32\textwidth}
%		\centering{\footnotesize{(a) Tax per ton of carbon US\$}}
%		%	\captionsetup{width=.45\linewidth}
%		\includegraphics[width=1\textwidth]{../../codding_model/own_basedOnFried/optimalPol_010922_revision/figures/all_13Sept22/CompTauf_bytaul_Reg5_Tauf_spillover0_nsk0_xgr0_knspil1_sep0_LFlimit1_emsbase0_countec0_GovRev0_etaa0.79_lgd1.png}
%	\end{subfigure}	
%	\begin{subfigure}[]{0.32\textwidth}
%		\centering{\footnotesize{(b) Deviation in carbon tax}}
%		%	\captionsetup{width=.45\linewidth}
%		\includegraphics[width=1\textwidth]{../../codding_model/own_basedOnFried/optimalPol_010922_revision/figures/all_13Sept22/CompTaufPER_bytaul_Reg5_Tauf_spillover0_nsk0_xgr0_knspil1_sep0_LFlimit1_emsbase0_countec0_GovRev0_etaa0.79_lgd0.png} \end{subfigure}		
%	%\begin{minipage}[]{0.32\textwidth}
%	%	\centering{\footnotesize{(c) Fossil research }}
%	%	%	\captionsetup{width=.45\linewidth}
%	%	\includegraphics[width=1\textwidth]{../../codding_model/own_basedOnFried/optimalPol_010922_revision/figures/all_13Sept22/CompTauf_bytaul_Reg5_sff_spillover0_nsk0_xgr0_knspil0_sep0_LFlimit1_emsbase0_countec0_GovRev0_etaa0.79_lgd1.png}
%	%\end{minipage}	
%	\begin{subfigure}[]{0.32\textwidth}
%		\centering{\footnotesize{(c) Green-to-fossil energy ratio}}
%		%	\captionsetup{width=.45\linewidth}
%		\includegraphics[width=1\textwidth]{../../codding_model/own_basedOnFried/optimalPol_010922_revision/figures/all_13Sept22/CompTauf_bytaul_Reg5_GFF_spillover0_nsk0_xgr0_knspil1_sep0_LFlimit1_emsbase0_countec0_GovRev0_etaa0.79_lgd0.png}
%	\end{subfigure}	
%	\begin{subfigure}[]{0.32\textwidth}
%		\centering{\footnotesize{(d) Fossil growth}}
%		%	\captionsetup{width=.45\linewidth}
%		\includegraphics[width=1\textwidth]{../../codding_model/own_basedOnFried/optimalPol_010922_revision/figures/all_13Sept22/CompTauf_bytaul_Reg5_gAf_spillover0_nsk0_xgr0_knspil1_sep0_LFlimit1_emsbase0_countec0_GovRev0_etaa0.79_lgd0.png}
%	\end{subfigure}			
%	\begin{subfigure}[]{0.32\textwidth}
%		\centering{\footnotesize{(e) Non-energy growth}}
%		%	\captionsetup{width=.45\linewidth}
%		\includegraphics[width=1\textwidth]{../../codding_model/own_basedOnFried/optimalPol_010922_revision/figures/all_13Sept22/CompTauf_bytaul_Reg5_gAn_spillover0_nsk0_xgr0_knspil1_sep0_LFlimit1_emsbase0_countec0_GovRev0_etaa0.79_lgd0.png}
%	\end{subfigure}			
%	\begin{subfigure}[]{0.32\textwidth}
%		\centering{\footnotesize{(f) Green growth}}
%		%	\captionsetup{width=.45\linewidth}
%		\includegraphics[width=1\textwidth]{../../codding_model/own_basedOnFried/optimalPol_010922_revision/figures/all_13Sept22/CompTauf_bytaul_Reg5_EY_spillover0_nsk0_xgr0_knspil1_sep0_LFlimit1_emsbase0_countec0_GovRev0_etaa0.79_lgd0.png}
%	\end{subfigure}		
%	\floatfoot{Notes:{  Except for Panel (b) all graphs show variables in levels when the emission limit has to be met (i) in a scenario without income taxation, $\tau_{\iota}=0$, the gray dashed graph, and (ii) in a scenario with a progressive income tax, $\tau_\iota=0.181$, the solid graph. Panel (b) depicts the deviation in the required carbon tax in the scenario with progressive income tax relative to the carbon tax necessary in the world without progressive income tax. }}
%\end{figure} 
%On top, adverse compositional effects of the progressive income tax aggravate. The reason is as follows. The tighter emission limit makes a higher carbon tax necessary. This again aggravates the compositional effect of the labor tax:  as the carbon tax increases demand for high-skill the difference in leisure across skill types broadens. Then high-skill labor is more responsive to a change in the tax progressivity.\footnote{\ For a more in depth discussion see Section \ref{subsec:eff_cc} above.}


%The compositional effect of progressive income taxes, intensifies over time but only minimally;% see Figure \ref{fig:Efftaul_nsk0_xgr0_know_app} in appendix \ref{app:neccab}, where I contrast the economy with a constant carbon tax but with and without a progressive income tax. 
%Over time, the progressive income tax lowers the green-to-fossil energy ratio more. As discussed earlier, the compositional effect of the progressive income tax intensifies

%The reason is that the constant carbon tax entails a transition to green production so that demand for high-skill labor increases. The higher hours of high-skill labor prior to setting a progressive tax intensifies the impact of the progressive tax, because the difference in leisure across labor type broadens. This more adverse effect of a progressive income tax, again, contributes to the smaller reduction in carbon taxes. %Rather, over time, the economy can profit less from the reductive effect of progressive income taxes to meet emission limits. 

% compositional effect of tauf
\begin{comment}
Panel (c) depicts the ratio of green-to-fossil output, and (d) shows the energy share of GDP. The joint policy with progressive income tax implies a smaller green-to-fossil ratio and a higher energy share.  
The difference in ratios is mainly driven by the gap in carbon taxes and the progressive income tax adds to this effect by diminishing the ratio of high- to low-skilled labor. 

content...
\end{comment}
%\tr{Add below earlier to effect of carbon tax}
%A carbon tax makes green goods relatively cheaper and energy goods more expensive.  Endogenous growth contributes to the rise in green energy share. A market size effect directs research to the green sector. In contrast, a smaller carbon tax results in a higher share of non-energy scientists due to the complementarity of energy and non-energy goods. Nevertheless, in equilibrium, the energy to GDP ratio falls with the carbon tax.\\ 
%\tr{-----------until here ------------------}

%First, as explained earlier, a carbon tax induces more green and less non-energy research (the latter being due to the higher price of energy goods when carbon is taxed). 
 
%This contributes to a smaller green energy share. Furthermore, a progressive income tax implies a smaller energy share. The reason is that non-energy goods are less skill intense than energy goods. As the ratio of high-to-low skill falls, non-energy goods become cheaper. This effect counteracts the use of fossil energy. Yet, the effect of income tax progressivity is small. 
%\textbf{check if this result remains with equal capital shares: it does: but the energy share is increasing with equal labor shares since productivity rise in non-energy does not redirect labor. Hence, a reduction in high-skill supply has the same qualitative effect on the energy ratio. But it seems to be smaller... } 

 % effect on growth
%I want to use this section to highlight the variation in how policies affect growth rates.
%Under the joint policy, the economy sees higher technology growth in the fossil and the non-energy sector and smaller ones in the green sector (Panels (d) to (f)). 
%%Again, changes in growth rates are mainly induced by the smaller carbon tax as the effect of progressive income taxes is small. % Qualitatively, the progressive income tax counters higher non-energy research by making non-energy goods less expensive. On the other hand, it contributes to the reduction in the green-to-fossil research ratio through a market size mechanism.  In sum, the higher amount of energy scientists outweighs the recomposition towards fossil research so that green research increases in presence of a progressive income tax. Again, effects are small. 
%%To summarize, a progressive income tax has an adverse effect on the green energy share directly and indirectly through its impact on the effectiveness of the carbon tax.
%%Then again,
%This result suggests that under a progressive income tax the emission limit can be reached at higher technology growth rates.
%Knowledge spillovers emerge as an important driver of the effect of the joint policy on growth rates as opposed to a carbon-tax-only policy. 
%In the following, I will briefly point to the role of model features in shaping the necessary carbon tax. 
% \begin{itemize}
% 	\item dynamics: I expect rising over time due to capital ratio heterogeneity and knowledge spillovers
% 	\item role of taul as a level effect versus compositional effect
% 	\item advantages under policy tuple progressive income tax and lower fossil tax as opposed to only fossil tax?
% \end{itemize}

%\paragraph{Knowledge spillovers}
Knowledge spillovers are a key driver of the effects of a combined policy and the level of the necessary carbon tax. Figure \ref{fig:Limit_nsk0_xgr0_know_Devs} shows the necessary carbon tax absent knowledge spillovers and compares variables in the benchmark model to the model without knowledge spillovers. 
Knowledge spillovers render a higher fossil tax in all periods necessary, and the gap widens over time (Panel (a)).
This underlines the previous observation that emissions rise faster especially in future periods when knowledge spills from the green to the fossil sector. 

%On the other hand, knowledge spillovers mitigate the adverse compositional effect of progressive income taxes on the green-to-fossil energy share.
%COMMENT: THIS IS TRUE BUT THERE IS NO EVIDENCE FOR AN IMPACT ON TAUF

 %\textbf{comparison effect taul with and without knowledge spillovers}%As a result, the required fossil tax deviates more from the fossil tax when no progressive income tax is present. 
 
Then again, knowledge spillovers imply a smaller decline in the green-to-fossil output ratio under the combined relative to the carbon-tax-only policy (Panel (b)). The reason is that, on the one hand, knowledge spillovers to the green sector boost green growth. Especially in future periods green growth reduces less under the combined policy (Panel (c)). On the other hand, knowledge spillovers mitigate fossil growth as it is the leading sector (Panel (d)). %The non-energy sector grows more in the model without knowledge spillovers.\ar not a reason not to grow more.
 Knowledge spillovers, therefore, lower the costs, i.e., an adverse green-to-fossil ratio, of a combined environmental policy consisting of an income tax and a carbon tax. The gains are higher growth rates in conventional sectors.\footnote{\ 
 	I discuss the effects of heterogeneous skills, heterogeneous labor shares, and endogenous growth in Appendix \ref{app:neccab}. }
 
This section preempts some of the optimal policy results which I investigate in the next section: a tax on labor income is optimally used to substitute for carbon taxes  before the emission limit binds in 2050. This policy achieves higher growth rates in the fossil and non-energy sector at the cost of slower green growth.
But, only if the future reduction in green growth is mitigated by knowledge spillovers, the Ramsey planner finds such a policy optimal. 
 \begin{figure}[h!!]
	\centering
	\caption{Percentage deviations from carbon-tax-only economy with and without knowledge spillovers}\label{fig:Limit_nsk0_xgr0_know_Devs}
		\begin{subfigure}[]{0.4\textwidth}
		\caption{Tax per ton of carbon in 2022 US\$}
		%	\captionsetup{width=.45\linewidth}
		\includegraphics[width=1\textwidth]{../../codding_model/own_basedOnFried/optimalPol_010922_revision/figures/all_13Sept22/CompTauf_bytaul_KN_Reg5_Tauf_spillover0_nsk0_xgr0_knspil0_sep0_LFlimit1_emsbase0_countec0_GovRev0_etaa0.79_lgd1.png}
	\end{subfigure}	
 \begin{minipage}[]{0.1\textwidth}
	\
\end{minipage}
\begin{subfigure}[]{0.4\textwidth}
	\caption{Green-to-fossil energy ratio}
	%	\captionsetup{width=.45\linewidth}
	\includegraphics[width=1\textwidth]{../../codding_model/own_basedOnFried/optimalPol_010922_revision/figures/all_13Sept22/CompTaufPER_bytaul_KN_Reg5_GFF_spillover0_nsk0_xgr0_knspil0_sep0_LFlimit1_emsbase0_countec0_GovRev0_etaa0.79_lgd0.png}
\end{subfigure}	
	\vspace{2mm}

\begin{minipage}[]{0.04\textwidth}
	\
\end{minipage}
%\begin{minipage}[]{0.32\textwidth}
%		\centering{\footnotesize{(b) Deviation in carbon tax}}
%		%	\captionsetup{width=.45\linewidth}
%		\includegraphics[width=1\textwidth]{../../codding_model/own_basedOnFried/optimalPol_010922_revision/figures/all_13Sept22/CompTaufPER_bytaul_KN_Reg5_Tauf_spillover0_nsk0_xgr0_knspil0_sep0_LFlimit1_emsbase0_countec0_GovRev0_etaa0.79_lgd0.png} 
%	\end{minipage}
 \begin{subfigure}[]{0.4\textwidth}
 	\caption{Green growth}
 	%	\captionsetup{width=.45\linewidth}
 	\includegraphics[width=1\textwidth]{../../codding_model/own_basedOnFried/optimalPol_010922_revision/figures/all_13Sept22/CompTaufPER_bytaul_KN_Reg5_gAg_spillover0_nsk0_xgr0_knspil0_sep0_LFlimit1_emsbase0_countec0_GovRev0_etaa0.79_lgd0.png} 
 \end{subfigure}	
 \begin{minipage}[]{0.1\textwidth}
	\
\end{minipage}
	\begin{subfigure}[]{0.4\textwidth}
		\caption{Fossil growth}
		%	\captionsetup{width=.45\linewidth}
		\includegraphics[width=1\textwidth]{../../codding_model/own_basedOnFried/optimalPol_010922_revision/figures/all_13Sept22/CompTaufPER_bytaul_KN_Reg5_gAf_spillover0_nsk0_xgr0_knspil0_sep0_LFlimit1_emsbase0_countec0_GovRev0_etaa0.79_lgd0.png} 
	\end{subfigure}	
\floatfoot{Notes: \footnotesize{Panel (a) shows the carbon tax when there are no knowledge spillovers. Panels (b) to (d) show the percentage deviation of the model with income tax to the one without income tax. The solid graphs refer to the benchmark model. Dashed graphs represent deviations in the model without knowledge spillovers, $\phi=0$. }}	
\end{figure}
\clearpage

%Absent knowledge spillovers, growth in the non-energy sector is indeed higher than absent a progressive tax, but it becomes smaller when a more aggressive fossil tax is required. Then, the reductive effect of the progressive income tax does not suffice to diminish the fossil tax in a way that non-energy research increases. Then, the progressive income tax in addition to the fossil tax both lower non-energy research and growth. With knowledge spillovers, non-energy research and growth are higher; a positive spillover effect of fossil growth. The green sector, too, profits from knowledge spillovers from the fossil sector. Green growth eventually is higher under the policy with progressive 

%However, overall, the benefits of a lower fossil tax in presence of a higher tax progressivity come at the cost of lower consumption and a smaller green-to-fossil ratio and a higher energy share. Yet, the latter two are acceptable due to a smaller level of production. 
