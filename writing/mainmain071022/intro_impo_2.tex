\clearpage
\section{Introduction}

% average marginal tax rate    0.3128    0.2340    0.1582    0.0834    0.0080   -0.0697   -0.0607   -0.0904   -0.1128   -0.1305   -0.1446   -0.1562

%policy experiment

% TAUF     0.8892    0.9516    1.0138    1.0765    1.1402    1.2047    2.6180    2.6986    2.7810    2.8651    2.9507    3.0376 *1e3

% optimal policy
% 1.0e+03 * 0.9873    1.0543    1.1211    1.1883    1.2566    1.3257    2.8333 2.9184    3.0057    3.0949    3.1858    3.2781
The Intergovernmental Panel on Climate Change highlights the need to transition to net-zero emissions by 2050 \citep{IPCC2022}. The literature has by and large focused on carbon taxes to mitigate climate change. Carbon taxes direct demand towards less polluting goods. When knowledge accumulation is endogenous, the carbon tax also shifts research across sectors \citep{Acemoglu2012TheChange}. A labor income tax, in contrast, affects the level of emissions by diminishing work effort. When a labor income tax is available, the carbon tax can be targeted more towards directing research. 
What is the optimal fiscal mix to implement the emission limit?

I answer this question in a model of directed technical change. The government chooses labor income taxes and a carbon tax. Carbon tax revenues are rebated lump sum. I find that the optimal policy mix consists of a carbon tax in combination with a tax on labor in early years. This policy enables the economy to profit from growth in the fossil sector while the tax on labor reduces economic activity to meet the emission limit. When the net-zero emission limit binds, the economy has to reduce fossil research more. The optimal carbon tax rises. A subsidy on labor stabilizes output. 

The role of labor income taxes arises from the contracting effect of lump-sum rebates. When, for instance, the carbon tax is set lower to mitigate the decline in fossil research, lump-sum transfers fall, which again stimulates economic activity. As a result, production and emissions increase. To counter this effect, the government taxes labor. Absent a labor income tax, a reduction in the carbon tax to foster fossil research would conflict with the emission limit.

The paper separates into two parts. First, I propose a simple model without growth to highlight the role of lump-sum rebates of carbon tax revenues. There are two intermediate sectors of production, one of which induces a negative environmental externality. The environmental externality is the only distortion motivating government action. A Ramsey planner maximizes welfare of a representative agent using an environmental and a labor income tax. I consider the extreme case where no lump-sum transfers are feasible and the government consumes environmental tax revenues. In this case, it is optimal to tax labor to contract economic activity. When lump-sum rebates are available, however, there is no role for labor income taxes in the exogenous growth model. 

Second, I introduce endogenous growth into the model. The resulting model builds on \cite{Fried2018ClimateAnalysis}. The model features three sectors producing intermediate goods:  a non-energy good, fossil energy, and green energy. The latter two are used to produce energy. Energy and non-energy goods are combined to form the final consumption good. The fossil sector exerts emissions. Within each intermediate sector of production, imperfectly monopolistic machine producers demand research to increase productivity of their machines and augment profits.

The evolution of research features a \textit{building-on-the-shoulders-of-giants} mechanism: last period's technology level positively affects today's level. This introduces dynamics and an inefficiency into the model because machine producers do not internalize future profits in today's investment decision. Knowledge spillovers across sectors are another source of inefficiency. To allow the planner to reduce growth in light of an exogenous emission limit, I endogenize the supply of scientists contrary to the literature \citep{Acemoglu2012TheChange, Fried2018ClimateAnalysis}. 

The economy is populated by a representative household. The household provides workers and scientists. There are two types of workers: high and low skill. The representative household structure ensures perfect risk sharing across worker types so that no equity motive drives the optimal policy. 
Skill heterogeneity is introduced to capture variation in the use of skill in green and fossil energy production. \cite{Consoli2016DoCapital} find a skill bias in the green sector. 

The government cares about fossil production due to an exogenous constraint on emissions. This approach has the advantage of reducing modeling and parametric uncertainty. The uncertainty surrounds (i) the relation of greenhouse gas emissions and climate warming and (ii) the damage resulting from climate change. I use estimated emission limits stemming from meta studies of integrated assessment models provided by the \cite{IPCC2022}. Since these limits are designed to meet climate targets, they are highly politically relevant.  
The emission limit for the US used in the analysis foresees an approximate 85\% reduction in net emissions relative to 2019 levels starting from 2020. In 2050, the emission limit further reduces to net zero. 

%//Quantitative Exercises//
I perform two quantitative exercises. First, I calculate the necessary carbon tax to meet the emission limit. The labor income tax is fixed. Second, in an optimal policy analysis, the planner sets the carbon tax and chooses whether to tax or subsidize labor.


%//Findings/Mechanism//

In the first exercise, I find that the necessary carbon tax equals US\$889 (in 2022 prices) per ton of carbon in the 2020-2024 period. It increases to US\$3,0376 in the 2070-2074 period. The first value is approximately 4.8 times higher than the social costs of carbon recently calculated by the Resources for the Future institute and the University of Berkeley: US\$185 \citep{RFF}. Using a non-distortive income tax instead of an income tax scheme resembling the progressivity of the US tax scheme \citep[taken from][]{Heathcote2017OptimalFramework},  the required carbon tax increases by between 7\% and 10\%. 


In the optimal policy analysis, I find that the emission limit is best implemented by a combination of carbon and labor income taxes. 
Before the net-zero emission limit binds, that is during the years from 2020 to 2040, a positive tax on labor optimally accompanies a carbon tax. 
The carbon tax amounts to US\$987 in the 2020-2024 period. It increases steadily to US\$1,326 in 2025-2049. The average (income-weighted) marginal tax rate equals 0.41\% in the 2020-2024 period and decreases to 0.01\% in the 2040-2044 period.


The motive for labor taxation follows from the effect of carbon taxes on the direction of research. In the initial years from 2020 to 2044, the government substitutes some of the carbon tax with a tax on labor. The smaller carbon tax allows for more research in the fossil sector which is especially beneficial for consumption growth as (i) the fossil sector is technologically more advanced\footnote{\citep{Acemoglu2012TheChange} show that in this case the periods necessary for the economy to grow at the same rate as without policy intervention are increasing in the productivity gap. Simply because more growth in the green sector is required  to catch up with the technology level in the fossil sector.}, and (ii) decreasing returns to research make a more equal allocation of scientists across sectors more productive. 
 
The optimal policy mix changes sharply when the net-zero emission limit has to be implemented. Now, the carbon tax rises steeply to discourage research and production in the fossil sector. In order to stabilize output, the government subsidizes labor. The optimal tax on carbon jumps to US\$2,833 in 2050 and steadily increases to US\$3,186 in the 2070-2074 period. The labor income tax becomes negative already in 2045-2049: the average marginal tax rate is -0.10\%. It declines to -0.19\% in 2070-2074.
The more stringent emission limit makes it infeasible to benefit from more research in the fossil sector as in earlier periods. Instead, fossil research has to be reduced intensely to meet the emission limit. 

%Overall, this policy sees smaller growth rates in green technology in early years and profits from more green growth once the net-zero emission limit binds. 


% motive for labor taxation. 
The optimality of labor income taxation follows from the contracting effect of lump-sum transferring carbon tax revenues. When the carbon tax not only targets energy producers’ demand for fossil energy, but also research, the size of lump-sum transfers changes. A smaller carbon tax not only mutes reductive pressure on fossil research but also lowers lump-sum rebates thereby stimulating economic activity. Yet, the rise in output would violate the emission limit. A tax on labor serves to counter the rise in labor supply. In contrast, when a higher carbon tax is optimal to diminish fossil research, lump-sum transfers increase, and labor supply reduces. To stabilize output, it becomes optimal to subsidize labor. 

Knowledge spillovers crucially affect the optimal policy.  On the one hand, knowledge spillovers to the fossil sector decrease the effectiveness of the carbon tax. Growth in other sectors drags growth in the fossil sector. On the other hand, knowledge spillovers from the fossil to other sectors mitigate the costs of more fossil growth by fostering green technology. 
Overall, I find that the optimal carbon tax to meet the emission limit increases with knowledge spillovers.
 
Then again, knowledge spillovers enable the economy to profit from fossil growth in early periods as green growth follows suit.  As discussed above, more fossil growth today is implemented by a smaller carbon tax combined with a tax on labor. Then, knowledge spillovers ensure that meeting the net-zero emission limit is not too costly in later periods. 
In contrast, absent knowledge spillovers, the optimal policy consists of a carbon tax and a subsidy on labor for all periods. In this case, meeting the emission limit in later years would become too costly if fossil growth today was higher. Therefore, more research should be allocated to the green sector already in initial periods. The labor income tax serves to mitigate the contractionary effect of higher lump-sum rebates. 

\begin{comment}
The less sector-specific knowledge is, the more should the government tax carbon to prevent the fossil sector from growing too much through spillovers from other sectors. The optimal carbon tax increases over time since knowledge spillovers from other sectors drag research to the fossil sector countering the effect of the carbon tax to make fossil energy more expensive. 
On the other hand, knowledge spillovers from the fossil sector allow the economy to profit from fossil research in early years while meeting the net-zero emission limit later on. Then, the green sector grows fast enough to mitigate the costs of meeting the net-zero emission limit in later years albeit more research is allocated to the fossil sector. 

Importance of fossil research 
Research in the fossil sector is especially important to consumption growth for two reasons. First, the fossil sector is the more productive one compared to green energy. \citep{Acemoglu2012TheChange} show that in this case the periods necessary for the economy to grow at the same rate as without policy intervention are increasing in the productivity gap. Simply because more growth in the green sector is required  to catch up with the technology level in the fossil sector. Second, the model features decreasing returns to research within a sector. When all scientists were allocated to one sector, it would be growth enhancing to reallocate some scientists to other sectors to equalize the returns to research. To meet the emission limit, the economy has to bear these costs. 

\end{comment}

\paragraph{Literature}

The paper relates to several strands of literature. 
First, to the literature on environmental policies. Second, it connects to the literature on how to recycle environmental tax revenues. Third, as the paper combines environmental and fiscal policies it naturally connects to the public finance literature. Finally, the results speak to the literature discussing overconsumption  which may be social preferences or environmental constraints. 

 
\begin{itemize}
	\item How to use environmental tax revenues \citep{Fried2018TheGenerations}
	\item Optimal environmental policy \ar focuses on environmental taxes
	\item weak do
\end{itemize}


In general, macro papers on (optimal) environmental policy focus on environmental taxation and analyze settings with inelastic labor supply \citep{Golosov2014OptimalEquilibrium, Acemoglu2012TheChang, Fried2018ClimateAnalysis, Acemoglu2016TransitionTechnology}. The mentioned papers assume lump-sum transfers of environmental tax revenues, hence endogenizing labor supply would not affect the optimal policy; yet, the role of lump-sum transfers changes to a reductive policy measure.
% 
% Acemoglu 2016 have lump-sum transfers and taxes
% Acemoglu Aghion 2012: lump-sum transfers, no optimal policy
%Golosov: hightlight the need of lump-sum transfers! but exogenous labor supply
% Therefore, the main finding of the present paper, the necessity of reductive policy measures to implement the efficient allocation, complements this literature. 
%Especially, when environmental tax revenues are not redistributed lump-sum in these papers, a variable labor supply would give an argument for labor income taxation. 

%
%Furthermore, I argue that the Pigou principle does generally not apply when no lump-sum transfers are available.  

Another focus of the literature on environmental policy is the redistribution of environmental tax revenues. In contrast to the previous strand of literature, this one generally assumes labor supply to be elastic. 
The dominant focus of this literature is the weak double dividend of environmental taxes \citep{LansBovenberg1994EnvironmentalTaxation, LansBovenberg1996OptimalAnalyses, Bovenberg2002EnvironmentalRegulation,  Barrage2019OptimalPolicy}: given an exogenous government funding constraint it is optimal to recycle environmental tax revenues to lower distortionary labor income taxes as opposed to higher lump-sum transfers. The latter decreases labor supply through an income effect thereby decreasing the tax base of the labor income tax. Consequently, it becomes more expensive for the government to generate revenues.
The weak double-dividend literature rests on the assumption that no lump-sum transfers and taxes are available. Yet, when this is the case, this paper shows that absent a government funding constraint, distortionary taxes should be set to reduce labor supply: some reduction in hours is in fact efficient. Hence, this paper provides an upper bound on the reduction of distortionary income taxes. This becomes clear when environmental tax revenues suffice to meet the government's funding constraint, then labor supply would be inefficiently high when the labor income tax is unused. 
The analytical literature on optimal environmental policy treats the labor income tax as pre-existing \citep{LansBovenberg1994EnvironmentalTaxation, LansBovenberg1996OptimalAnalyses} in the settings when no lump-sum transfers are possible.
 

Building on the weak double-dividend literature, \cite{Fried2018TheGenerations} compare distinct scenarios of how to recycle environmental tax revenues and investigate the impact on inter- and intragenerational inequality in an overlapping generations model. Lump-sum transfers are preferred by the  living generation. 
In this literature, the advantage of different recycling means is often evaluated with respect to equity or political feasibility \cite{Carattini2018, VANDERPLOEG2022103966}. The present paper employs an efficient allocation as a benchmark to assess lump-sum transfers, government consumption, and redistribution via the income tax scheme. 

\cite{VANDERPLOEG2022103966} do comment on efficiency costs. 
What they do: 
structural estimate effects of carbon tax and lump-sum redistribution on households across income distribution; "better off" measured by utility: to capture both: consumption and labor supply!
The term efficiency is synonym to labor supply redcutions. Abstracting from the possibility that labor supply reductions may be advantageous;\tr{ do they assume an endogenous funding condition?} The politically most feasible recycling is through lower income tax rates, which boosts labor supply and economic activity, but hurts the poor. If equity is relevant, the government should do split env tax revenues: some recycling as carbon dividend, the other to lower income taxes.

\tr{1) What if there is a pre-existing income tax but not gov funding constraint? Do BLÖDsinnn; maybe not since could still find an increase in labor income tax if optimal level is above initial level;\\
 2) look at a policy where income taxes are used to fund government spending \ar then this would generate more gov. revenues }

Why do bov and de Mooji not find that the reduction in labor tax revenues depends on the level of the income tax? \ar because its a non-formal argument

equity: horowitz 2017: equity gains from lump-sum redistribution

Metcalf 2007/ 2008
Kotlikoff making carbon pricing a generational win win

\cite{LansBovenberg1996OptimalAnalyses}: they focus on how the optimal environmental policy deviates from the Pigou principle due to pre-existing distortionary taxes; 

\textbf{Goulder1995} defines the weak double dividend as: recycling revenues to lower pre existing tax distortions: one achieves \textbf{cost savings} relative to the case where the tax revenues are returned to taxpayers in lump-sum fashion; 
\textbf{It seems like the weak double dividend is defined in terms of costs}; costs (i.e. non-environmental costs) are the reduction in price times quantity relative to the equilibrium without tax; p.18: \textbf{the double dividend literature focuses on non-environmental costs of the environmental policy. And earlier works focus on consumption and output as only measure of gains/ missing gains from leisure. } 
Linking paper to double dividend literature means to compare efficiency gains to costs. Leisure versus consumption \ar but I show analytically that consumption costs are lower. 

\textcolor{blue}{so fare havent seen a paper which analytically shows the weak double dividend holds \ar could be that they indeed miss the lower bound, size dependency of advantage to reduce labor taxes}

\textit{To be fair, the double dividend literature focuses on cost advantages by using environmental tax revenues to substitute for distortionary labor income taxes. However, it remains unmentioned that under the assumption of elastic labor supply, which the literature necessarily assumes, the environmental tax alone is not efficient. }

\textbf{Bovenber 1998}: "\textit{environmental taxes are  generally  an  efficient  instrument  for  protecting  the  environment}"
\tr{This statement is wrong! they are only efficient in combination with reductive measures \ar there is no double-dividend as - to efficiently reduce emissions - environmental tax revenues have to be redistributed lump-sum}
\textcolor{blue}{the question arises what is better from an equity perspective: lump-sum transfers or additional progressive taxes?} Evaluate by looking at high and low skill wages. 
\\
Environmental tax revenues are not a free lunch. By not redistributing them lump-sum,there are efficiency costs because work effort would be too high.




\textit{Citation in Fried: Pigou 1920, Dales 1968, Montgomery 1972, Baumol and Oates 1988 }: "\textit{Establishin a price on carbon [...] is well understood to be the most efficient approach for reducing greenhouse gas emissions.}"

\paragraph{Reduction policies}

\paragraph{Outline}
The remainder of the paper is structured as follows. The next section \ref{sec:mod_an} presents a tractable model which is used to derive the analytical results in section \ref{sec:theory}. In section \ref{sec:model}, I extend the model to a quantitative framework and calibrate it. I present and discuss the quantitative results in section \ref{sec:res}. Section \ref{sec:con} concludes.