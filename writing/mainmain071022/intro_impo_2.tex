
\section{Introduction}

% average marginal tax rate    0.3128    0.2340    0.1582    0.0834    0.0080   -0.0697   -0.0607   -0.0904   -0.1128   -0.1305   -0.1446   -0.1562

%policy experiment

% TAUF     0.8892    0.9516    1.0138    1.0765    1.1402    1.2047    2.6180    2.6986    2.7810    2.8651    2.9507    3.0376 *1e3

% optimal policy
% 1.0e+03 * 0.9873    1.0543    1.1211    1.1883    1.2566    1.3257    2.8333 2.9184    3.0057    3.0949    3.1858    3.2781
To meet climate targets, the Intergovernmental Panel on Climate Change highlights the need to transition to net-zero emissions by 2050 \citep{IPCC2022}.  The literature to date has largely focused on carbon taxes. Such taxes direct production and R\&D towards activities that pollute less.\footnote{ See, for instance, \cite{Acemoglu2012TheChange}, \cite{Golosov2014OptimalEquilibrium}, or \cite{Fried2018ClimateAnalysis}.} However, other fiscal instruments also affect carbon emissions. Labor taxes, for example, curb the level of production generally thereby diminishing emissions. Conversely, they may stabilize production when carbon taxes are high. Thus, labor income taxes may play a role in optimally meeting emission goals. The question is what fiscal policy mix best supports the transition to net-zero emissions.

To analyze the optimal fiscal mix, I set up a model of directed technical change. The government chooses the dynamic path of labor income taxes and carbon taxes. In so doing, it anticipates that net emissions are limited in the short run and have to be zero  at some point in the future. Calibrating the model to the US economy, I can characterize the optimal fiscal mix during the transition toward net-zero emissions and thereafter. My main finding is that when the net-zero emission limit is binding, the optimal policy is to tax carbon heavily and to subsidize labor to stabilize production. During the transition phase, the optimal carbon tax is lower. A tax on labor helps reduce emissions by minimizing overall production. This mix allows for a higher productivity of research while meeting emission targets.

The key to the optimal mix of carbon and labor taxes lies in the endogenous allocation of research activity. A knowledge advantage in the fossil sector means that a rapid shift to green research is costly. Cross-sectoral knowledge spillovers from research in the fossil to the green industry enable a steady transition: they ensure that fossil knowledge generated in the short run profits green growth in the long run. Setting a smaller carbon tax, the government  engineers a smoother transition of research activity. A tax on labor, in turn, reduces production in the short run to satisfy emission targets. In the long run, only green production and research is feasible. Relying on labor income taxes to reduce emissions by lowering overall production becomes too costly. Therefore, the optimal policy sets an excessively high carbon tax to meet emission limits. The higher carbon tax directs research to the green sector which again stimulates green innovation in the future. Yet, this comes at the expense of lower consumption. Therefore, a labor subsidy optimally mutes the disruption in output.

More in detail, the modeling follows \cite{Fried2018ClimateAnalysis}. A final consumption good is produced from energy and non-energy goods. The energy good, in turn, is composed of green and fossil energy. The fossil sector exerts emissions. Imperfectly monopolistic producers of machinery invest in research to increase the productivity of their machines. Machines are used in the intermediate sectors: non-energy, fossil, and green energy.  The model builds on the directed technical change framework developed in \cite{Acemoglu2012TheChange}, where innovation profits from past technology levels within a sector (\textit{within-sector knowledge spillovers}). In addition to their model, returns to research decrease in the number of scientists employed within a sector, and some knowledge spills across sectors (\textit{cross-sectoral knowledge spillovers}).

Relative to the study by \cite{Fried2018ClimateAnalysis}, the current paper adds the consideration of an optimal policy. The optimal policy accounts for a gradually declining  net emission limit that eventually turns to zero in 2050. Where the literature  has mainly focused on carbon taxes \citep{Fried2018ClimateAnalysis, Barrage2019OptimalPolicy}, the planner in my model chooses (i) a sales tax per unit of fossil energy (“the carbon tax”) and (ii) a tax on labor income. The role for the labor income tax does not arise from equity considerations. Rather, the labor tax is an environmental policy instrument next to carbon taxes because it affects the level of production.  I solve for the optimal path of carbon and labor taxes using the numerical approach by  \cite{Jones1993OptimalGrowth} and \cite{Barrage2019OptimalPolicy}.

To highlight the importance of endogenous growth for the optimal fiscal mix, I start with a simple pencil-and-paper setting. This has the externality from fossil energy but does away with endogenous growth. In this static setting, optimal fiscal policy only resorts to a Pigouvian carbon tax and lump-sum rebates. The carbon tax adjusts the share of fossil energy in production. Labor taxes do not play a role. The reason for this is that when the carbon tax is set to the Pigouvian level, the wage rate reduces exactly by the social costs from an additional hour worked. These costs arise from more pollution as economic activity rises. Hence, households internalize the externality of work in their labor supply decision.

Then, I turn to the dynamic model with endogenous growth. I first calibrate the model to the US economy in the period from 2015 to 2019. Then, I feed an exogenous emission limit into the model that is based on the path of the global target provided by the \cite{IPCC2022}. The emission limit for the US used in the analysis stipulates a reduction by 84.5\% in net emissions  in 2020 relative to 2019-levels. The value increases to 85.6\% in 2045.\footnote{ These targets are more than twice as big than the reduction prescribed by the Biden administration amounting to 38\% relative to 2019-emissions. Source:  \href{https://www.whitehouse.gov/briefing-room/statements-releases/2021/04/22/fact-sheet-president-biden-sets-2030-greenhouse-gas-pollution-reduction-target-aimed-at-creating-good-paying-union-jobs-and-securing-u-s-leadership-on-clean-energy-technologies/}{https://www.whitehouse.gov/briefing-room/statements-releases/2021/04/22/}, retrieved 14 September 2022.} In 2050, the emission limit further reduces to net zero.
In this setting, I perform two quantitative exercises. First, I calculate carbon taxes that would be necessary to meet the emission limit while keeping the labor income tax fixed. Second,  the planner optimally sets the carbon and the labor tax to maximize welfare.

As for the first exercise, I find that the necessary carbon tax is US\$889 (in 2022 prices) per ton of carbon in the 2020-2024 period. The tax increases to US\$2,951 in the 2070-2074 period. % The first value is approximately 4.4 times higher than the social costs of carbon recently calculated by the \textit{Resources for the Future} institute and the University of Berkeley: US\$203.5 \citep{Rennert2022ComprehensiveCO2}.\footnote{  For comparability, I transformed the \textit{Resources for the Future} (RFF) estimate of US\$185 in 2020 prices to 2022 prices. The RFF measure aims to capture the damage associated with a marginal increase in CO$_2$. It is not intended to implement a certain level of emissions.} % Abstracting from modeling and parameter uncertainty, the discrepancy suggests that focusing on the social costs of carbon alone would fail to achieve the emission limit designed to meet the Paris Agreement. } 
The surge in the necessary tax on carbon over time follows from the tighter emission limit and market forces which direct inputs to the fossil sector, such as cross-sectoral knowledge spillovers. I compare the necessary carbon tax in the calibrated model\textemdash which has a distortive income tax\textemdash to the necessary carbon tax absent labor income taxes. Without income taxation, the required carbon tax is  7\% to 10\% higher.

As for the second exercise, results show that the emission limit is best implemented by a combination of carbon and labor income taxes. Before the net-zero emission limit binds, a positive tax on labor optimally accompanies a carbon tax. The carbon tax amounts to US\$987 in the 2020-2024 period. It increases steadily to US\$1,257 in 2040-2044. The average (income-weighted) marginal income tax rate equals 0.31\% in the 2020-2024 period and decreases to 0.01\% in the 2040-2044 period.

The role for labor taxation follows from the use of carbon taxes to direct research across sectors. This constitutes a second target for the carbon tax next to adjusting the share of fossil energy in production. The government chooses a lower carbon tax to allow for more research in the fossil sector. Knowledge accumulation in this sector in the past renders fossil researchers particularly productive. Furthermore, decreasing returns to research make a more equal allocation of scientists more productive. Technological advances in the fossil sector, in turn, benefit green growth in future periods because of cross-sectoral knowledge spillovers. However, as the carbon tax targets the direction of research, it deviates from the Pigouvian rate and the wage rate does no longer fully capture the effect of hours on the externality. Labor supply is inefficiently high.
To mitigate the higher level of emissions, the government taxes labor income. 

The optimal policy mix changes sharply once the net-zero emission limit has to be implemented. Now, the carbon tax rises steeply to direct research to the green sector. The optimal tax on carbon jumps to US\$2,833 per ton in 2050 and gradually increases to US\$3,186 in the 2070-2074 period. In addition, the government subsidizes labor. The labor income tax becomes negative already in 2045-2049: the average marginal tax rate is -0.07\%. It declines to -0.16\% in 2070-2074.

The more stringent emission limit makes it too expensive for the economy to take advantage of more fossil research which would mean using a labor income tax to lower emissions. In this case, encouraging green research becomes optimal in order to benefit from dynamic spillovers within this sector. To do so, the government raises the carbon tax. However, this diminishes the wage rate, and labor supply declines. The subsidy on labor helps to raise labor supply.


Knowledge spillovers are instrumental for the economy to profit from a smooth allocation of researcher. With knowledge spillovers, fossil research in early periods boosts green growth tomorrow. A more productive green sector renders a transition to net-zero emissions in future periods less costly. When knowledge is sector specific, however, the optimal policy should allocate more research to the green sector early on. This fosters green innovation tomorrow as scientists can build on green technology advances today. In this setting, the optimal carbon tax is higher to bolster green research, and the labor income tax subsidizes labor throughout time. While knowledge spillovers to the green sector allow the economy to profit from fossil research, overall, though, knowledge spillovers make a more aggressive environmental policy necessary. The reason is that knowledge also spills to the fossil sector thereby raising the share of fossil energy in production.  % this mechanism regards the adjustment in the fossil share of production! Knowledge spillovers from fossil to green change the social value of fossil research. 


