\clearpage
\section{Introduction}

% average marginal tax rate    0.3128    0.2340    0.1582    0.0834    0.0080   -0.0697   -0.0607   -0.0904   -0.1128   -0.1305   -0.1446   -0.1562

%policy experiment

% TAUF     0.8892    0.9516    1.0138    1.0765    1.1402    1.2047    2.6180    2.6986    2.7810    2.8651    2.9507    3.0376 *1e3

% optimal policy
% 1.0e+03 * 0.9873    1.0543    1.1211    1.1883    1.2566    1.3257    2.8333 2.9184    3.0057    3.0949    3.1858    3.2781
The Intergovernmental Panel on Climate Change highlights the need to transition to net-zero emissions by 2050 \citep{IPCC2022}. The question is what fiscal policy mix best supports this goal. The literature to date has largely focused on carbon taxes alone. Such taxes direct production and R\&D towards activities that pollute less \citep{Acemoglu2012TheChange}. However, other fiscal instruments also affect carbon emissions. Labor taxes, for example, curb the level of production generally. Conversely, labor subsidies may boost production. Such taxes may, thus, be an important part of the fiscal mix in achieving the  net-zero emission limit.

To analyze the optimal fiscal mix, I set up a model of directed technical change \citep{Acemoglu2002DirectedChange, Acemoglu2012TheChange}. The government chooses the dynamic path of labor income taxes and carbon taxes. Doing so, it anticipates that the net-zero emission limit will bind at some point in the future. Calibrating the model to the US economy, I can characterize the optimal fiscal mix during this transition and in the long term. The main finding is that when the net-zero emission limit binds, the optimal policy is to tax carbon heavily and to subsidize labor. During the transition phase, the optimal carbon tax is lower and a tax on labor reduces production. 

The key to the optimal mix of carbon and labor taxes are knowledge spillovers between research in conventional carbon-based and green industries. Decreasing returns to scale in research activity in each industry mean that a rapid shift in the composition of activity is socially costly. The labor tax allows the government to reduce emissions in the short run by  lowering production while–at the same time–engineering a smoother transition of economic activity across industries. In the long-run, only green production is feasible. A high carbon tax ensures this. Labor subsidies, in turn, serve to keep production and research activity high.

More in detail, the modeling follows \cite{Fried2018ClimateAnalysis}. A final consumption good is produced from energy and non-energy goods. The energy good, in turn, is composed of green and fossil energy. The fossil sector exerts emissions. Imperfectly monopolistic producers of machinery invest in research to increase the productivity of their machines. Machines are used in the intermediate sectors: non-energy, fossil, and green energy. Returns to research are decreasing in the number of researchers employed within a sector, and some knowledge spills over within sectors and across sectors.

Relative to the study by \cite{Fried2018ClimateAnalysis}, the current paper adds the consideration of an optimal policy. The optimal policy accounts for a gradually declining  net emission limit that eventually turns to zero in 2050. Where the literature  has mainly focused on carbon taxes \citep{Acemoglu2012TheChange, Fried2018ClimateAnalysis}, the planner chooses (i) a sales tax per unit of fossil energy (“the carbon tax”) and (ii) a tax on labor income in my model. The role for the labor income tax does not arise from equity considerations. Rather, the labor tax is an important instrument next to carbon taxes because it affects research activity and production.  I solve for the optimal path of carbon and labor taxes using the numerical approach by  \citep{Jones1993OptimalGrowth, Barrage2019OptimalPolicy}.

To highlight the importance of endogenous growth for the optimal fiscal mix, I start with a simple pencil-and-paper setting. This has the externality from fossil energy use but does away with endogenous growth. In this static setting, optimal fiscal policy only resorts to a Pigouvian carbon tax and lump-sum rebates. Labor taxes do not play a role.

Then, I turn to the dynamic setting with endogenous growth. I, first, calibrate the model to the US economy in the period from 2015 to 2019. Then, I feed into the model an exogenous emission limit based on the path of the global limit provided by the \cite{IPCC2022}. The emission limit for the US used in the analysis foresees a reduction by 84.5\% in net emissions relative to 2019-levels in 2020. The value increases to 85.6\% in 2045. In 2050, the emission limit further reduces to net zero. In this setting, I perform two quantitative exercises. First, I calculate the necessary carbon tax to meet the emission limit while keeping the labor income tax fixed. Second,  the planner optimally sets the carbon and the labor tax to maximize welfare.

As for the first exercise, I find that the necessary carbon tax equals US\$889 (in 2022 prices) per ton of carbon in the 2020-2024 period. The tax increases to US\$3,038 in the 2070-2074 period. The first value is approximately 4.4 times higher than the social costs of carbon recently calculated by the Resources for the Future institute and the University of Berkeley: US\$203.5 \citep{RFF}.\footnote{\  For comparability to my results, I transformed the RFF estimate of US\$185 in 2020 prices to 2022 prices. The RFF measure aims to capture the damage associated with a marginal increase in CO$_2$. It is not intended to implement a certain level of emissions. Abstracting from modeling and parameter uncertainty, the discrepancy suggests that focusing on the social costs of carbon alone would fail to implement the emission limits designed to meet the Paris Agreement. } 
The increase in the necessary tax on carbon follows from the tighter emission limit and market forces which direct inputs to the fossil sector, such as knowledge spillovers. I compare the necessary carbon tax in the calibrated model - which has a distortive income tax - to the necessary carbon tax absent labor income tax. Without income taxation, the required carbon tax is  7\% to 10\% higher.

As for the second exercise, I find that the emission limit is best implemented by a combination of carbon and labor income taxes. Before the net-zero emission limit binds, a positive tax on labor optimally accompanies a carbon tax. The carbon tax amounts to US\$987 in the 2020-2024 period. It increases steadily to US\$1,326 in 2045-2049. The average (income-weighted) marginal income tax rate equals 0.31\% in the 2020-2024 period and decreases to 0.01\% in the 2040-2044 period.

The motive for labor taxation follows from the effect of carbon taxes on the direction of research. In the initial years from 2020 to 2044, the government substitutes some of the carbon tax with a tax on labor. The smaller carbon tax allows for more research in the fossil sector. Decreasing returns to research make a more equal allocation of scientists across sectors more productive. Technological advances in the fossil sector, in turn, benefit green growth in future periods because of knowledge spillovers.

The optimal policy mix changes sharply once the net-zero emission limit has to be implemented. Now, the carbon tax rises steeply to discourage further research and production in the fossil sector. The optimal tax on carbon jumps to US\$2,833 per ton in 2050 and gradually increases to US\$3,186 in the 2070-2074 period. In addition, the government subsidizes labor. The subsidy stabilizes production and incentivizes R\&D investment as more workers stand ready to use machines. The labor income tax becomes negative already in 2045-2049: the average marginal tax rate is -0.07\%. It declines to -0.16\% in 2070-2074. The more stringent emission limit makes it infeasible to continue to benefit from more research in the fossil sector. Instead, fossil research has to be reduced intensely to meet the net-zero emission limit.

Knowledge spillovers crucially affect the optimal policy. On the one hand, spillovers to the fossil sector decrease the effectiveness of the carbon tax because green growth makes research in the fossil sector more productive. The optimal carbon tax to meet the emission limit, therefore, increases when knowledge spills across sectors relative to a benchmark without knowledge spillovers. Furthermore, the optimal carbon tax rises over time to counter growing productivity in the fossil sector. 

On the other hand, knowledge spillovers to the green sector enable the economy to profit from fossil growth in early periods as green growth follows suit. A more productive green sector renders a transition to net-zero emissions in future periods less costly. As discussed above, more fossil growth today is implemented by a smaller carbon tax combined with a tax on labor. In contrast, absent knowledge spillovers, the optimal policy consists of a carbon tax and a subsidy on labor in all periods. In this case, meeting the emission limit in later years would become too costly if fossil growth today was higher. Therefore, more research should be allocated to the green sector already in initial periods. 

\paragraph{Literature}

The paper relates to several strands of literature. 
First, to the literature on environmental policies. Second, it connects to the literature on how to recycle environmental tax revenues. Third, as the paper combines environmental and fiscal policies it naturally connects to the public finance literature. Finally, the results speak to the literature discussing overconsumption  which may be social preferences or environmental constraints. 

 
\begin{itemize}
	\item How to use environmental tax revenues \citep{Fried2018TheGenerations}
	\item Optimal environmental policy \ar focuses on environmental taxes
	\item weak do
\end{itemize}


In general, macro papers on (optimal) environmental policy focus on environmental taxation and analyze settings with inelastic labor supply \citep{Golosov2014OptimalEquilibrium, Acemoglu2012TheChang, Fried2018ClimateAnalysis, Acemoglu2016TransitionTechnology}. The mentioned papers assume lump-sum transfers of environmental tax revenues, hence endogenizing labor supply would not affect the optimal policy; yet, the role of lump-sum transfers changes to a reductive policy measure.
% 
% Acemoglu 2016 have lump-sum transfers and taxes
% Acemoglu Aghion 2012: lump-sum transfers, no optimal policy
%Golosov: hightlight the need of lump-sum transfers! but exogenous labor supply
% Therefore, the main finding of the present paper, the necessity of reductive policy measures to implement the efficient allocation, complements this literature. 
%Especially, when environmental tax revenues are not redistributed lump-sum in these papers, a variable labor supply would give an argument for labor income taxation. 

%
%Furthermore, I argue that the Pigou principle does generally not apply when no lump-sum transfers are available.  

Another focus of the literature on environmental policy is the redistribution of environmental tax revenues. In contrast to the previous strand of literature, this one generally assumes labor supply to be elastic. 
The dominant focus of this literature is the weak double dividend of environmental taxes \citep{LansBovenberg1994EnvironmentalTaxation, LansBovenberg1996OptimalAnalyses, Bovenberg2002EnvironmentalRegulation,  Barrage2019OptimalPolicy}: given an exogenous government funding constraint it is optimal to recycle environmental tax revenues to lower distortionary labor income taxes as opposed to higher lump-sum transfers. The latter decreases labor supply through an income effect thereby decreasing the tax base of the labor income tax. Consequently, it becomes more expensive for the government to generate revenues.
The weak double-dividend literature rests on the assumption that no lump-sum transfers and taxes are available. Yet, when this is the case, this paper shows that absent a government funding constraint, distortionary taxes should be set to reduce labor supply: some reduction in hours is in fact efficient. Hence, this paper provides an upper bound on the reduction of distortionary income taxes. This becomes clear when environmental tax revenues suffice to meet the government's funding constraint, then labor supply would be inefficiently high when the labor income tax is unused. 
The analytical literature on optimal environmental policy treats the labor income tax as pre-existing \citep{LansBovenberg1994EnvironmentalTaxation, LansBovenberg1996OptimalAnalyses} in the settings when no lump-sum transfers are possible.
 

Building on the weak double-dividend literature, \cite{Fried2018TheGenerations} compare distinct scenarios of how to recycle environmental tax revenues and investigate the impact on inter- and intragenerational inequality in an overlapping generations model. Lump-sum transfers are preferred by the  living generation. 
In this literature, the advantage of different recycling means is often evaluated with respect to equity or political feasibility \cite{Carattini2018, VANDERPLOEG2022103966}. The present paper employs an efficient allocation as a benchmark to assess lump-sum transfers, government consumption, and redistribution via the income tax scheme. 

\cite{VANDERPLOEG2022103966} do comment on efficiency costs. 
What they do: 
structural estimate effects of carbon tax and lump-sum redistribution on households across income distribution; "better off" measured by utility: to capture both: consumption and labor supply!
The term efficiency is synonym to labor supply redcutions. Abstracting from the possibility that labor supply reductions may be advantageous;\tr{ do they assume an endogenous funding condition?} The politically most feasible recycling is through lower income tax rates, which boosts labor supply and economic activity, but hurts the poor. If equity is relevant, the government should do split env tax revenues: some recycling as carbon dividend, the other to lower income taxes.

\tr{1) What if there is a pre-existing income tax but not gov funding constraint? Do BLÖDsinnn; maybe not since could still find an increase in labor income tax if optimal level is above initial level;\\
 2) look at a policy where income taxes are used to fund government spending \ar then this would generate more gov. revenues }

Why do bov and de Mooji not find that the reduction in labor tax revenues depends on the level of the income tax? \ar because its a non-formal argument

equity: horowitz 2017: equity gains from lump-sum redistribution

Metcalf 2007/ 2008
Kotlikoff making carbon pricing a generational win win

\cite{LansBovenberg1996OptimalAnalyses}: they focus on how the optimal environmental policy deviates from the Pigou principle due to pre-existing distortionary taxes; 

\textbf{Goulder1995} defines the weak double dividend as: recycling revenues to lower pre existing tax distortions: one achieves \textbf{cost savings} relative to the case where the tax revenues are returned to taxpayers in lump-sum fashion; 
\textbf{It seems like the weak double dividend is defined in terms of costs}; costs (i.e. non-environmental costs) are the reduction in price times quantity relative to the equilibrium without tax; p.18: \textbf{the double dividend literature focuses on non-environmental costs of the environmental policy. And earlier works focus on consumption and output as only measure of gains/ missing gains from leisure. } 
Linking paper to double dividend literature means to compare efficiency gains to costs. Leisure versus consumption \ar but I show analytically that consumption costs are lower. 

\textcolor{blue}{so fare havent seen a paper which analytically shows the weak double dividend holds \ar could be that they indeed miss the lower bound, size dependency of advantage to reduce labor taxes}

\textit{To be fair, the double dividend literature focuses on cost advantages by using environmental tax revenues to substitute for distortionary labor income taxes. However, it remains unmentioned that under the assumption of elastic labor supply, which the literature necessarily assumes, the environmental tax alone is not efficient. }

\textbf{Bovenber 1998}: "\textit{environmental taxes are  generally  an  efficient  instrument  for  protecting  the  environment}"
\tr{This statement is wrong! they are only efficient in combination with reductive measures \ar there is no double-dividend as - to efficiently reduce emissions - environmental tax revenues have to be redistributed lump-sum}
\textcolor{blue}{the question arises what is better from an equity perspective: lump-sum transfers or additional progressive taxes?} Evaluate by looking at high and low skill wages. 
\\
Environmental tax revenues are not a free lunch. By not redistributing them lump-sum,there are efficiency costs because work effort would be too high.




\textit{Citation in Fried: Pigou 1920, Dales 1968, Montgomery 1972, Baumol and Oates 1988 }: "\textit{Establishin a price on carbon [...] is well understood to be the most efficient approach for reducing greenhouse gas emissions.}"

\paragraph{Reduction policies}

\paragraph{Outline}
The remainder of the paper is structured as follows. Section \ref{sec:mod_an} presents a tractable model and the analytical result. In Section \ref{sec:model}, I extend and calibrate the model to a quantitative framework. I present and discuss the quantitative results in Section \ref{sec:res}. Section \ref{sec:con} concludes.