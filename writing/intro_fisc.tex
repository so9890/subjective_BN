
\section{Introduction}
The public finance literature has focused the discussion of optimal labour income taxation and the degree of progressivity on the equity-efficiency trade-off.  The benefits of labour taxes arise from redistribution. With concave utility specifications full redistribution is efficient. However, the optimal tax system does not feature full redistribution when labour supply is endogenous. Instead, the degree of redistribution is traded-off against aggregate output as individuals reduce their labour supply in response to the labour income tax. 

\paragraph{An environmental perspective}
 Adding environmental externalities to the classical public finance framework may change the light with which efficiency costs are perceived. Instead of merely reducing welfare, direct benefits through output reduction arise by lowering the externality. 

In general, the presence of corrective tax instruments, which direct production to a non-polluting alternative, however, distortionary labour taxes are not employed as environmental policy instruments under standard preferences.\footnote{ Compare my paper on Sustainable production and demand where there is no limit/ no disaster.} This finding depends on how the externality is modeled. Indeed, some (non-economist) scholars argue for the necessity to reduce current consumption levels in developed economies in the light of threats to planetary boundaries. This can be understood by the pressing time frame to correct for climate externalities in particular and serious threats to other planetary boundaries \citep{Rockstrom2009AHumanity}\footnote{\ In their article, \cite{Rockstrom2009AHumanity} argue that several planetary boundaries have already been surpassed. Uncertainties about their interaction are another argument in favour of more cautious environmental policies. } such as biodiversity and nitrogen xxx.
Taken this broader view on environmental pollution through economic activity, it becomes more questionable, if a perfectly clean technology exists or can ever be innovated.  

In sum, the need to reduce environmental externalities rather quickly and the potential non-existence of perfectly green technologies could render classical fiscal tax instruments environmental policy tools through the efficiency channel. 

\textit{So far: efficiency-equity channel; plus on efficiency side}
\paragraph{Trade-offs: environmental externality}
While having an advantageous direct effect on the externality, counteracting indirect effects arise in a general equilibrium framework. Proponents of a reduction policy especially focus on consumption by the rich which consume a higher amount of natural resources.\footnote{\ Note Sonja: Abstracting from inequality, would it still be best to reduce consumption by the rich when the poor have a higher marginal propensity to consume dirty? \textit{Could be an important aspect in the model}.}
This concern could add to the benefits of tax progressivity.
However, targeting these households in particular for environmental reasons, in contrast, will lower high skill labour supply and investment into it. Yet, these skills are especially used in green sectors of the economy \citep{Consoli2016DoCapital}. As a result, dirty production becomes relatively cheaper and the dirty share of production rises. 

The literature on optimal environmental policy knows the 
Furthermore, the direction of technological change might be drawn towards the more polluting good in response to a higher progressivity. The reduction in the cleaner sector's labour input good diminishes the relative market size of this sector. The market effect shifts innovation towards the bigger dirty sector. On the other hand, the higher price of the clean sector makes innovation in the clean sector more profitable, the price effect. 
One contribution of the paper is to discuss optimal fiscal policy in an endogenous growth setting. 

Given these counteracting forces, this paper seeks to answer whether a reduction policy is indeed optimal for environmental reasons. 
\tr{Why not use one corrective tax in each sector? Political argument? Changing already established policy measures less difficult than introducing ne taxes?}
\\

\noindent\rule[1ex]{\textwidth}{1pt}
\paragraph{The below is an addendum in case I want to add capital taxation somewhere...} \tr{below relatively close to \cite{Conesa2009TaxingAll}} 
Another prominent debate in the public finance literature centres on the optimal size of the capital tax. The optimality of zero capital taxation has been argued for in the seminal work by Judd (1985) and affirmed by others.  However, as discussed by \cite{Conesa2009TaxingAll}, the optimality of a zero capital tax hinges upon the endogeneity of labour supply. The capital tax rises when allowing for life-cycle aspects. In this setting, it is optimal to tax capital heavily and to accept a reduction in consumption at a modest reduction of hours worked for the benefit of better insurance and redistribution. Taking environmental externalities into account, the optimal capital tax may even be higher due to its advantageous effect on the externality.

\cite{Domeij2004OnTaxes} discuss a reduction of capital taxes in a model with idiosyncratic productivity shocks which imply  wealth inequality. Their focus rests on the distributional effect which arise in the transition from the benchmark to a lower capital tax. They find huge negative distributional effects over the transition since a linear labour tax is less progressive.  (\textit{check this}) The distributional effects outweigh the benefits of a higher capital stock and output. 
\\

\noindent\rule[1ex]{\textwidth}{1pt}

\paragraph{Empirical motivation}
The key channel in the paper which induces adverse consequences of a reduction policy on the externality hinges on the skill-bias in the cleaner sector and that consumption-rich households are high skill households. I document a positive correlation between skills and sectors and between income and skills using xxx data.

\paragraph{Exercise}
First, I study the effects of labour-tax progressivity  on the environmental externality in a tractable general equilibrium model with directed technical change. The model builds on \cite{Heathcote2017OptimalFramework} and \cite{Acemoglu2012TheChange}. 
In contrast to \cite{Heathcote2017OptimalFramework}, I abstract from idiosyncratic risk and the life cycle component to focus on the medium to long run and the environment. 
The planner still faces the trade-off between equity and skill accumulation. 
In this part of the paper, I focus on conditions on the processes shaping the relation of production and the environment so that fiscal policies are employed for environmental reasons although corrective taxes are available.  

Second, in a quantitative analysis, I examine the optimal policy and transitions to the new steady state from a realistically calibrated current state of the economy. Two possible setups seem interesting. (1) The government maximises a \textit{(possibly pareto-weighted, utilitarian)} social welfare function. (2, \textit{if I only focus on carbon}) The government takes the climate target of the Paris agreement as constraint while maximising a social welfare function.  
To scrutinise the contribution of directed technological change and skill heterogeneity, I rerun the analysis in versions of the model where either or both channels are shut down. In order to understand why the optimal policy in the benchmark model differs from the simplified versions, I introduce the optimal policy in e.g. the version without skill heterogeneity into the full model. The difference between the two model is driven by inequality. 

\paragraph{Sensitivity}
First, I adjust the model for different specifications of the skill accumulation process which is essential in shaping the adverse effects of reduction policies on the environment. Second, I change the specification of utility functions which incorporates the arguments put forward by advocates of reduction policies such as: habits, social preferences which even absent an environmental externality imply overconsumption, and social effects of leisure \textit{(or one of these)}.  These extensions all add to the benefits of reducing consumption. Third, more evolved modelling of the innovation process as in \cite{Fried2018ClimateAnalysis} is introduced \textit{(add this to the main quantitative part)}. 

\paragraph{Literature}
\textbf{Empirics: Skill and Green production}