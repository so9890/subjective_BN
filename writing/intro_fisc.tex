
\section{Introduction}
The public finance literature has focused the discussion of optimal labour income taxation and the degree of progressivity on the equity-efficiency trade-off.  The benefits of labour taxes arise from redistribution. With concave utility specifications full redistribution is efficient. However, the optimal tax system does not feature full redistribution when labour supply is endogenous. Instead, the degree of redistribution is traded-off against aggregate output as individuals reduce their labour supply in response to the labour income tax. 

\paragraph{An environmental perspective}
 Adding environmental externalities to the classical public finance framework may change the light with which efficiency costs are perceived. Instead of merely reducing welfare, direct benefits through output reduction arise by lowering the externality. 

In general, the presence of corrective tax instruments, which direct production to a non-polluting alternative, however, distortionary labour taxes are not employed as environmental policy instruments under standard preferences.\footnote{ Compare my paper on Sustainable production and demand where there is no limit/ no disaster.} This finding depends on how the externality is modeled. Indeed, some (non-economist) scholars argue for the necessity to reduce current consumption levels in developed economies in the light of threats to planetary boundaries. This can be understood by the pressing time frame to correct for climate externalities in particular and serious threats to other planetary boundaries \citep{Rockstrom2009AHumanity}\footnote{\ In their article, \cite{Rockstrom2009AHumanity} argue that several planetary boundaries have already been surpassed. Uncertainties about their interaction are another argument in favour of more cautious environmental policies. } such as biodiversity and nitrogen xxx.
Taken this broader view on environmental pollution through economic activity, it becomes more questionable, if a perfectly clean technology exists or can ever be innovated.  

In sum, the need to reduce environmental externalities rather quickly and the potential non-existence of perfectly green technologies could make it optimal to use classical fiscal tax instruments for environmental concerns through the efficiency channel. 

\textit{So far: efficiency-equity channel; plus on efficiency side}
\paragraph{Trade-offs: environmental externality}
While having an advantageous direct effect on the externality, counteracting indirect effects arise in a general equilibrium framework. Proponents of a reduction policy especially focus on consumption by the rich which consume a higher amount of natural resources.\footnote{\ Note Sonja: Abstracting from inequality, would it still be best to reduce consumption by the rich when the poor have a higher marginal propensity to consume dirty? \textit{Could be an important aspect in the model}.}
This concern could add to the benefits of tax progressivity.
However, targeting these households in particular for environmental reasons, in contrast, will lower high skill labour supply and investment into it. Yet, these skills are especially used in green sectors of the economy \citep{Consoli2016DoCapital}. As a result, dirty production becomes relatively cheaper and the dirty share of production rises. 

The literature on optimal environmental policy knows the 
Furthermore, the direction of technological change might be drawn towards the more polluting good in response to a higher progressivity. The reduction in the cleaner sector's labour input good diminishes the relative market size of this sector. The market effect shifts innovation towards the bigger dirty sector. On the other hand, the higher price of the clean sector makes innovation in the clean sector more profitable, the price effect. 
One contribution of the paper is to discuss optimal fiscal policy in an endogenous growth setting. 

Given these counteracting forces, this paper seeks to answer whether a reduction policy is indeed optimal for environmental reasons. 
\tr{Why not use one corrective tax in each sector? Political argument? Changing already established policy measures less difficult than introducing ne taxes?}
\\

\noindent\rule[1ex]{\textwidth}{1pt}
\paragraph{The below is an addendum in case I want to add capital taxation somewhere...} \tr{below relatively close to \cite{Conesa2009TaxingAll}} 
Another prominent debate in the public finance literature centres on the optimal size of the capital tax. The optimality of zero capital taxation has been argued for in the seminal work by Judd (1985) and affirmed by others.  However, as discussed by \cite{Conesa2009TaxingAll}, the optimality of a zero capital tax hinges upon the endogeneity of labour supply. The capital tax rises when allowing for life-cycle aspects. In this setting, it is optimal to tax capital heavily and to accept a reduction in consumption at a modest reduction of hours worked for the benefit of better insurance and redistribution. Taking environmental externalities into account, the optimal capital tax may even be higher due to its advantageous effect on the externality.

\cite{Domeij2004OnTaxes} discuss a reduction of capital taxes in a model with idiosyncratic productivity shocks which imply  wealth inequality. Their focus rests on the distributional effect which arise in the transition from the benchmark to a lower capital tax. They find huge negative distributional effects over the transition since a linear labour tax is less progressive.  (\textit{check this}) The distributional effects outweigh the benefits of a higher capital stock and output. 
\\

\noindent\rule[1ex]{\textwidth}{1pt}

\paragraph{Empirical motivation}
The key channel inducing adverse consequences of a reduction policy on the externality hinges on the skill-bias in the cleaner sector. % and that consumption-rich households are high skill households. \ar that they are richer is an equilibrium effect 
\cite{Consoli2016DoCapital} find that within broader occupational group greener occupations are skill-biased: on average, the degree of non-routine tasks is higher as are formal education, work experience, and on-the-job training for the US. I link this evidence to the type of skills employed in the dirty and the cleaner sector in the model. %\ar might be able to abstract from CRRA utility function and to focus on the skill accumulation dimension...
% \cite{Bowen2018CharacterisingComposition} estimate a relatively low difference in skills between jobs and argue that this gap can be closed quickly by on-the-job trainings. 


\paragraph{Exercise}
In the first part of the paper, I study the effects of labour-tax progressivity  on the environmental externality in a tractable general equilibrium model with directed technical change. The model builds on \cite{Heathcote2017OptimalFramework} and \cite{Acemoglu2012TheChange}. 
In contrast to \cite{Heathcote2017OptimalFramework}, I abstract from idiosyncratic risk and the life cycle component to focus on the medium to long run and the environment. 
The planner still faces the trade-off between equity and skill accumulation. 
In this part of the paper, I focus on conditions on the processes shaping the relation of production and the environment so that fiscal policies are employed for environmental reasons although corrective taxes are available.  

Second, in a quantitative analysis, I examine the optimal policy and transitions to the new steady state from a realistically calibrated current state of the economy and a quantitative model\footnote{\ For example, as in  \cite{Fried2018ClimateAnalysis} I add important features to the innovation process and a third neutral sector of production.}. Two possible setups seem interesting. (1) The government maximises a \textit{(possibly pareto-weighted, utilitarian)} social welfare function. This version allows to account for other planetary boundaries potentially interacted with the dominant climate boundary and carbon emissions. (2) The government takes the climate target of the Paris agreement as constraint when maximising a social welfare function.  The advantage of this approach is that it suffers less from potential misspesifications with regard to the interaction of emissions and the climate boundary of the environment. Furthermore, it is closer related to the current political debate. 
(\textit{\textbf{Note:}}(3) A reduction policy can also become optimal when climate policies as required to meet the targets are socially unfeasible. For example, poor households consume a higher share of polluting goods, reducing overall consumption through labour taxes, instead, could be more favourable in terms of inequality. )

I perform two quantitative experiments to scrutinise the contribution of directed technological change and skill heterogeneity on the optimal policy. First, I rerun the analysis in versions of the model where either or both channels are shut down.  Second, I introduce the optimal policy of the amended model, e.g. the  version without skill heterogeneity, into the model with skill heterogeneity to learn how skill heterogeneity shapes the effect of the optimal policy on the externality.

\paragraph{Sensitivity}
First, I adjust the model for different specifications of the skill accumulation process which is essential in shaping the adverse effects of reduction policies on the environment.\footnote{\ For example, skill can be modelled as generating intergenerational spill-overs within households as in \cite{Borissov2019CarbonDevelopment}.} Second, I change the specification of utility functions which incorporates the arguments put forward by advocates of reduction policies such as: habits, social preferences which even absent an environmental externality imply overconsumption, and social effects of leisure \textit{(or one of these)}.  These extensions all add to the benefits of reducing consumption. Third, instead of fiscal policies the government has a policy tool to lower the maximum hours worked per worker and period. Such a policy is closer to what proponents of a reduction policy suggest. % Third, more evolved modelling of the innovation process as in \cite{Fried2018ClimateAnalysis} is introduced \textit{(add this to the main quantitative part)}. 

\paragraph{Literature}
\begin{itemize}
\item Public finance literature on optimal labour tax (progressivity) or capital tax (advantage of labour tax progressivity: there is a tractable model already available; capital tax with the zero capital tax finding seems interesting and important though...But requires a different model with capital) \citep{Heathcote2017OptimalFramework, Conesa2009TaxingAll, Domeij2004OnTaxes}
\item Optimal environmental policy with and without directed technical change \citep{Acemoglu2012TheChange, Acemoglu2016TransitionTechnology, Fried2018ClimateAnalysis, Barrage2019OptimalPolicy, Golosov2014OptimalEquilibrium, Hassler2016EnvironmentalMacroeconomics}
\item potentially: limits to growth \citep{Stokey1998AreGrowth, Jones2016LifeGrowth, Arrow2004AreMuch}
\item building on the following literature: \begin{itemize}
\item skills and green production: empirical \citep{Consoli2016DoCapital, Bowen2018CharacterisingComposition, Borissov2019CarbonDevelopment}; growth model with skill accumulation \citep{Borissov2019CarbonDevelopment}
\item literature on the environment and limits to growth \citep{Dasgupta2021, Rockstrom2009AHumanity, Brock2005ChapterEmpirics, Arrow2004AreMuch, Cohen2019AnnualSubstitutable}
\item literature on reduction policies \citep{Schor2005SustainableReduction, Pullinger2014WorkingDesign}
\end{itemize}
\end{itemize}
