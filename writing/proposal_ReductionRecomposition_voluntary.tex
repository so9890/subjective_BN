\documentclass[12pt]{article}
\usepackage[utf8]{inputenc}
\usepackage{xcolor}
\usepackage{graphicx}
\usepackage{listings}
\usepackage{epstopdf}
\usepackage{etoc}
\usepackage{pdfpages}
\usepackage[capposition=top]{floatrow}
\usepackage{pdflscape} % landsacpe package
% set font to times
%\usepackage{mathptmx} % times!!! 
%\usepackage[T1]{fontenc}
\usepackage{amsmath}
\usepackage{soul}
\usepackage[left=2.5cm, right=2.5cm, top=2.5cm, bottom =2.5cm]{geometry}
\usepackage{natbib}
%\usepackage[natbibapa]{apacite}
%\usepackage{apacite}
%\bibliographystyle{apacite}
\bibliographystyle{apa}
%\renewcommand{\footnotesize}{\fontsize{10pt}{11pt}\selectfont}
\usepackage[onehalfspacing]{setspace}
\usepackage{listings}
\renewcommand{\figurename}{\textbf{Figure}}
\renewcommand{\hat}{\widehat}
\usepackage[bf]{caption}
\usepackage{tikz}
%\begin{comment}
%\usepackage[headsepline,footsepline]{scrlayer-scrpage} % has to come before package!!! otherwise option clash
%\usepackage{scrlayer-scrpage}
%\pagestyle{scrheadings} % kopfzeile/ fußzeile
%\clearpairofpagestyles
%\ohead{}
%\ihead{\textit{Redistribution, Demand and  Sustainable Production}}
%\cfoot{\thepage}
%\pagestyle{plain} % comment this one to have header
%\end{comment}
\usepackage{comment}
\usepackage{siunitx}
\usepackage{textcomp}
\definecolor{sonja}{cmyk}{0.9,0,0.3,0}
%\definecolor{purple}{model}{color-spec}
\usepackage{amssymb}
\newcommand{\ar}{$\Rightarrow$ \ }
\newcommand{\frp}[2]{\frac{\partial{#1}}{\partial{#2}}}
\newcommand{\tr}[1]{\textcolor{red}{#1}}
\newcommand{\vlt}[1]{\textcolor{violet}{#1}}
\newcommand{\bl}[1]{\textcolor{blue}{#1}}
\newcommand{\sn}[1]{\textcolor{sonja}{#1}}
%%% TIKZS
\usepackage{tikz}
\usetikzlibrary{mindmap,trees}
\usetikzlibrary{backgrounds}
\tikzstyle{every edge}=  [fill=orange]  
\usetikzlibrary{tikzmark}
\usetikzlibrary{decorations.markings}
\usepackage{tikz-cd}
\usetikzlibrary{arrows,calc,fit}
\tikzset{mainbox/.style={draw=sonja, text=black, fill=white, ellipse, rounded corners, thick, node distance=5em, text width=8em, text centered, minimum height=3.5em}}
\tikzset{mainboxbig/.style={draw=sonja, text=black, fill=white, ellipse, rounded corners, thick, node distance=5em, text width=13em, text centered, minimum height=3.5em}}
\tikzset{dummybox/.style={draw=none, text=black , rectangle, rounded corners, thick, node distance=4em, text width=20em, text centered, minimum height=3.5em}}
\tikzset{box/.style={draw , rectangle, rounded corners, thick, node distance=7em, text width=8em, text centered, minimum height=3.5em}}
\tikzset{container/.style={draw, rectangle, dashed, inner sep=2em}}
\tikzset{line/.style={draw, very thick, -latex'}}
\tikzset{    pil/.style={
		->,
		thick,
		shorten <=2pt,
		shorten >=2pt,}}

% other stuff
\newcommand{\innermid}{\nonscript\;\delimsize\vert\nonscript\;}
\newcommand{\activatebar}{%
	\begingroup\lccode`\~=`\|
	\lowercase{\endgroup\let~}\innermid 
	\mathcode`|=\string"8000
}
%\usepackage{biblatex}
%\addbibresource{bib_mt.bib}
\usepackage{ulem}
\title{Growth, the Environment, and Inequality\\ \small{ Voluntary reduction in an endogenous growth model with inequality}}
\date{Sonja Dobkowitz\\ Bonn Graduate School of Economics\\ %University of Bonn\\
	\vspace{1mm}
	%Preliminary and incomplete\\
	First version: December 25, 2021\\
	This version: \today }
\usepackage{graphicx,caption}
\usepackage{hyperref}
\usepackage{minitoc}
\setcounter{secttocdepth}{5}
\usetikzlibrary{shapes.geometric}

% for tabular

%\usepackage{array}
\usepackage{makecell}
\usepackage{multirow}
\usepackage{bigdelim}

\renewenvironment{abstract}
{\small
	\list{}{
		\setlength{\leftmargin}{0.025\textwidth}%
		\setlength{\rightmargin}{\leftmargin}%
	}%
	\item\relax}
{\endlist}
\begin{document}
	%	\includepdf[pages=-]{../titlepage.pdf}
	\maketitle

\section{Notes}
\tr{Career shifters: from highly skilled voluntarily choose a simpler work; documentation: \url{https://www.arte.tv/de/videos/050584-000-A/wachstum-was-nun/} }

\section{Introduction 2.0}
What is the role of individual environmental concerns for economic behaviour and the economy?
We document a role for environmental concerns in how households make economic decisions. 
More precisely, a rise in environmental concerns explains a decrease in consumption of durable goods and working hours. These findings have macroeconomic relevance. 
We first show a negative correlation in consumption and working hours and environmental concerns after controlling for other relevant demographic and economic factors. 

We then show that environmental concerns are relevant for consumption and labour decision by use of a structural model. \textit{the first preference version would be matched to the data, without environmental concerns, the second one with a role for environmental concerns would equally be matched but the error is smaller? How to evaluate better fit? \cite{Bartling2015DoResponsibility} also talk about what preferences match the behaviour...}

In the second part of the paper we qualitatively study the effects of such a voluntary reduction on the macroeconomy and study policy implications that best accompany such a change. 

	
\section{Introduction}

Using a \textit{representative} panel data set for the Netherlands, this paper documents a rising share of households which voluntarily reduce their consumption and labour supply. These households find additional consumption of furniture or clothes unnecessary suggesting the existence of a satiation point for certain goods; compare panels (a) and (b) in figure \ref{fig:evolution_notNecessary}. Simultaneously, there is a positive trend (which started later though than the consumption trend) in the share of panelists which voluntarily works less hours than full time; compare pane (c) in figure \ref{fig:evolution_notNecessary}. These panelists choose to work less in order to increase leisure or to take it easier \textit{(check if this does not coincide with changing hours of partner! Could only look at those where partner did not change)}. This indicates that the marginal utility of leisure  has risen above that of consumption. 

\paragraph{Document relation to environmental concerns} 
In a next step, we show that both the share of panelists which don't think it is necessary to buy new products and the share which voluntarily works part time is positively correlated with environmentally friendly attitudes (\textit{I am willing to change, moral duty, simpler lives should be lived}) and behaviour \textit{(buy second hand regularly, dont need to own things, no need for new products)}.

Figures \ref{fig:evolution_notNecessary_bygroup:furniture} to \ref{fig:evolution_wtr_willingtochange} show the evolution of the share of voluntary reductionists sorted by environmental attitudes. 
Environmental attitudes are clearly positively correlated with part-time work for the full time span. Yet, the \textbf{rise} in part-time for leisure seems not to be driven by environmental concerns. Rather, by those households who care less about the environment. 
when it comes to consumption the relation is less clear in levels . However, the increase in the share of panelists which find new furniture or clothes unnecessary may be driven by environmentally friendly households. 

Figures \ref{fig:behaviour_opinion} to \ref{fig:behaviour_opinion:moral} show the distribution of whether a panelist acts or is willing to act conditional on environmental attitudes (cross-section). Indeed, expressing environmentally friendly attitudes is positively correlated with environmentally friendly behaviour, such as buying second hand and recycled products. Also, the desire to own things seems less pronounced.
\tr{1) Need to control for other demographics, especially income, family size etc...}
 \tr{2) I am looking at levels in the graphs sorted by group... look at growth rates to visualise how increase relates to environmental attitudes.} \textbf{\tr{Key question: Does environmental attitude explain $\underline{increase}$ in voluntary reduction? Why now? Why the rising trend?} }

\paragraph{How to continue: option a)}
What consequences does this change in behaviour have on the macro economy and the environmental externality in particular? What is the optimal policy to accompany this behavioural change? (\cite{Hou2020FeelingsIntentions} link psychology to early replacement of durables..)
To answer this question, I use a dynamic general equilibrium model with heterogeneous preferences. The model is informed by the economic characteristics of these panelists: education, skills, sectors of work, and income. 
% What is driving the voluntary reduction? How are reduction in consumption and in labour supply related on the individual level?  What comes first?

\ar \textbf{To Do: Empirical}:
\begin{enumerate}
\item who are the households which reduce consumption and labour supply \ar averages by group; pooled logit/probit; use a measure of hours reduction and interaction\\ 
%\tr{CONTINUE WITH PYTHON SCRIPT 5}
Considerations:
(i) look at part time workers; and those who become part time workers during panel; (ii) look at hours worked over time; how does this behave? How do they move for panelists which reduce at some point; i.e. which transition? 
\item what are the dynamics? \ar logit/ probit with lagged variables of labour supply/ consumption indicators; for hours worked OLS
\item  relate this to green production \ar use information on green skills from \cite{Consoli2016DoCapital} \checkmark
\end{enumerate}

\ar \textbf{To do: Model}
\begin{itemize}
\item demand-determined models as in \cite{Michaillat2015AggregateUnemployment} or (more stylised, less quantitative) \cite{Auerbach2021InequalityEconomy}
\item policy implications: logic of demand boosting policies does not work for voluntary reductionists; these households also reduce the capacity of the state to redistribute, their skills might be missing in a transition to sustainable production (however, no evidence of a correlation with skill type.)
\end{itemize}

\paragraph{How to continue: option b)}
 Relation to happiness. Exploit timing in happiness and work: Is it that happier households reduce their work or do they become more happy once work is reduced? Or no difference at all? Look at households which transition. 
 Have to get an idea of the literature: has it already been done? Psychology papers on character traits which are positively correlated with environmentally friendly behaviour and less materialistic values: \cite{Brown2005AreLifestyle,Heikkinen2015DegrowthConsumers}; Are there any policy recommendations which could be derived?


\section{To do}
\begin{itemize}
	\item representativeness of dataset and environmental panel?
	\item show distribution of answers \checkmark
	\item show correlation with 
	\begin{itemize}
		\item skills \ar need to sort skills into green and non-green specific (11/01)\ar looks like there is no different distribution of attitudes towards the environment by skills! \checkmark
		\item income
		\item consumption/labour supply over time \checkmark
	\end{itemize}
\end{itemize}

\section{Data sources}
\begin{itemize}
\item 
Questions on reduction:
\url{https://www.dataarchive.lissdata.nl/study_units/view/1045}
\item Questions on moral consumer behaviour! 
\url{https://www.dataarchive.lissdata.nl/study_units/view/420} \tr{Not used so far, more about animal well being}
\item liss all waves on consumption and time use, on work, and on income
\end{itemize}


\section{Descriptive Statistics}

\begin{figure}[h!!]
	\centering	
	\caption{Opinions on behaviour}\label{fig:opinions}	
	\begin{minipage}[h!!]{0.32\textwidth}  
		%	\captionsetup{width=.45\linewidth}
		\centering\footnotesize{(a) I am willing to change my lifestyle to help the environment}
		\includegraphics[width=1\textwidth]{../codding_data/results/liss/qk20a175title0.png}
	\end{minipage}
	\begin{minipage}[h!!]{0.32\textwidth}
		%	\captionsetup{width=.45\linewidth}
		\centering\footnotesize{(b) We all need to live simpler lives}
		\includegraphics[width=1\textwidth]{../codding_data/results/liss/qk20a181title0.png}
	\end{minipage}
	\begin{minipage}[h!!]{0.32\textwidth}  
		%	\captionsetup{width=.45\linewidth}
		\centering\footnotesize{(c) It is a moral duty to care for nature and the environment}
		\includegraphics[width=1\textwidth]{../codding_data/results/liss/qk20a183title0.png}
	\end{minipage}	
%\begin{minipage}[h!!]{0.32\textwidth}  
%%	\captionsetup{width=.45\linewidth}
%\centering\footnotesize{(d) I do not want the government's policy to tackle environmental problems to cost me anything extra}
%\includegraphics[width=1\textwidth]{../codding_data/results/liss/qk20a178title0.png}
%\end{minipage}		
%\begin{minipage}[h!!]{0.32\textwidth}  
%%	\captionsetup{width=.45\linewidth}
%\centering\footnotesize{(e) Protecting the environment must not stand in the way of economic progress}
%\includegraphics[width=1\textwidth]{../codding_data/results/liss/qk20a180title0.png}
%\end{minipage}	
	\floatfoot{Notes: Looks like a gap between opinions on normative duties and the willingness to act.
	}
\end{figure}

\begin{figure}[h!!]
	\centering	
	\caption{Reported behaviour/ intentions}\label{fig:behaviour}	
	\begin{minipage}[h!!]{0.32\textwidth}  
		%	\captionsetup{width=.45\linewidth}
		\centering\footnotesize{(a) I regularly buy second-hand products}
		\includegraphics[width=1\textwidth]{../codding_data/results/liss/qk20a135title0.png}
	\end{minipage}
	\begin{minipage}[h!!]{0.32\textwidth}
		%	\captionsetup{width=.45\linewidth}
		\centering\footnotesize{(b) I prefer to own things}
		\includegraphics[width=1\textwidth]{../codding_data/results/liss/qk20a144title0.png}
	\end{minipage}
	\begin{minipage}[h!!]{0.32\textwidth}  
		%	\captionsetup{width=.45\linewidth}
		\centering\footnotesize{(c) I would be open to long-term leasing of products}
		\includegraphics[width=1\textwidth]{../codding_data/results/liss/qk20a141title0.png}
	\end{minipage}	
	\begin{minipage}[h!!]{0.32\textwidth}  
	%	\captionsetup{width=.45\linewidth}
	\centering\footnotesize{(d) I am open to buying products made from used parts or materials}
	\includegraphics[width=1\textwidth]{../codding_data/results/liss/qk20a147title0.png}
\end{minipage}
	\begin{minipage}[h!!]{0.32\textwidth}  
	%	\captionsetup{width=.45\linewidth}
	\centering\footnotesize{(e) I prefer new products}
	\includegraphics[width=1\textwidth]{../codding_data/results/liss/qk20a148title0.png}
\end{minipage}
	%%	\floatfoot{Notes: \tiny{ 
	%In appendix \ref{app:att}, more details on data manipulations are provided.
	%	}}
\end{figure}

\newpage
\begin{figure}[h!!]
	\centering	
	\caption{Joint distribution of opinion and behaviour: Willing to change lifestyle}\label{fig:behaviour_opinion_joint}	
	\begin{minipage}[h!!]{0.32\textwidth}  
		%	\captionsetup{width=.45\linewidth}
		\centering\footnotesize{(a) Buy second-hand products regularly}
		\includegraphics[width=1\textwidth]{../codding_data/results/liss/joint_heatmap175_135labels0.png}
	\end{minipage}
	\begin{minipage}[h!!]{0.32\textwidth}
		%	\captionsetup{width=.45\linewidth}
		\centering\footnotesize{(b) I prefer new products}
		\includegraphics[width=1\textwidth]{../codding_data/results/liss/joint_heatmap175_148labels0.png}
	\end{minipage}
	\begin{minipage}[h!!]{0.32\textwidth}  
		%	\captionsetup{width=.45\linewidth}
		\centering\footnotesize{(c)I am open to long-term leasing}
		\includegraphics[width=1\textwidth]{../codding_data/results/liss/joint_heatmap175_141labels0.png}
	\end{minipage}
	\floatfoot{Notes: \tiny{ 
		Rows refer to the willingness to change one's lifestyle for the environment; columns refer to the variable in the title of the panle (a) to (c). It seems like that being willing to change one's lifestyle and acting accordingly are the most frequent choices.	}}
\end{figure}

\newpage
\section{Model}
Test the following thesis: Already today some people want to reduce their consumption of resources due to environmental reasons. Reduction can be by buying second-hand or recycled products or leasing durables. I also look at the variables for whether a panelist think its necessary to replace furniture, buy new clothes, or to reduce working time for leisure (these latter variables are available over time). 

\subsection{Discrete choice: binary outcome variable}

marginal effects of regressor on dependent variable; two last models below are not linear, cannot use OLS! Use Maximum likelihood. \\
What determines the probability to buy second hand, buy recycled products, of long-run leasing
\paragraph{Linear probability model}
\paragraph{Logit Model}
\paragraph{Probit Model}

\subsection{Is there a voluntary reduction in cosnumption/labour supply due to environmental concerns?}

More precisely, I want to see, if environmental concerns explain part of the rise in panelists which think buying new items is not necessary, or which reduce their working time. 

\textbf{First:} plot average consumption by groups over time: (a) willing to change vs. (b) not willing to change; 
Look at levels first.\\ If don't find anything, look at subgroups which also buy second hand...

\textbf{Second:} look at residuals of regression consumption in t (levels) as in approach 1 below. 

\paragraph{Approach 1}
\begin{itemize}
	\item outcome variables
	\begin{itemize}

\item changes in consumption (annual) over time \ar but very coarse... changes in expenditures could be driven by increase in quantity or change in composition of products bought! Not good!
\item time spent working (income divided by wage)/ also as direct variable
\item survey questions allow to differentiate quantity from quality 
\item in income dataset: \textit{ci21n382-ci21n354} \ar Do you buy new clothes/ furniture regularly \ar no bcs not necessary \ar some sort of voluntary reduction; time series dimension! \ar changes over time?
\item in working time dataset: variables
\begin{itemize}
\item cw21n526 - cw21n510 For what reaosns did you work parttime ($<$36 hours)
\begin{itemize}
\item \textbf{cw21n399}: I wanted to take it a bit easier
\item \textbf{cw21n400}: because I want to have more leisure time
\end{itemize}
\item cw21n145: How many hours would you like to work in total? \ar should be below or equal to actual hours when consider person as voluntary reductionist
\item cw21n291 - cw21n306: If you were to stop working before the old pension age, for what reason would that be?
\begin{itemize}
\item \textbf{cw21n292}: I believe I have worked long enough
\item \textbf{cw21n294}: I would like to do something different
\item \textbf{cw21n295}: I want to take it a bit easier
\end{itemize}
\item \textbf{cw21n001}: does have paid work
\end{itemize}
\item[\ar] combine affirmative answers to (cw21n399 + cw21n400) (\ar already reduced), (cw21n292+cw21n294+cw21n295) (\ar willing to reduce)
\item construct time series from income/working data set questions
\begin{itemize}
\item consumption data works well! 
\item working hours: \\ what is the control group: employed or unemployed\ar employed
\end{itemize}
	\end{itemize}
\item Controls
\begin{itemize}
\item control for income changes \tr{these can be a choice, only account for exogenous changes in income! \ar could use narrative on reason for changes in income, job change... }, 
\item family changes
\item  politics, time FE
\end{itemize}	

\item first regress consumption changes on controls \ar residuals: changes in consumption unexplained by controls
\item[\ar] plot residuals: if negative then there are reductions in consumption which are unexplained by controls
\item regress residuals on opinions interacted with taking action etc \ar Does this explain reduction?
\item[\ar] If so, then there are households which reduce their consumption due to environment (arguably controls sufficient to argue for causality)
\end{itemize}

Steps to be taken
\begin{enumerate}
\item download and merge consumption data and income data\ar over waves and environment dataset
\item run 
\end{enumerate}

\paragraph{Approach 2}
\begin{itemize}
\item policy effects
\item heterogeneous effects of policy \ar interaction with willingness to change
\end{itemize}
\textbf{The policy intervention as in \cite{Pullinger2014WorkingDesign}} \ar Pullinger argues that policies have not been taken up well. Furthermore, the introduction of some policies to facilitate reductions in working hours over the life cycle have been introduced in 2006. The panel starts in 2008. 


\section{Results}

Subsection \ref{subsec:heatmaps} shows conditional distributions  of environmentally firendly behaviour and environmental concerns. Subsection \ref{subsec:trends} shows trend over time of variables which are meant to capture the share of panelists which voluntarily reduce consumption and work for the full panel and by environmental attitudes. 

\subsection{Correlations: Opinions and action in 2020}\label{subsec:heatmaps}
Those who are willing to change their lifestyle, figure \ref{fig:behaviour_opinion}, also have a higher chance to act accordingly. That is, to buy second-hand products regularly, to not prefer new products, and to be open for long-term leasing. 
Out of the three variables which are meant to capture a panelists concerns about the environment ((a) willing to change one's lifestyle for the environment (variables qk20a175) figure \ref{fig:behaviour_opinion}, (b) We all have to simpler lives to protect the environment (variables qk20a181), figure \ref{fig:behaviour_opinion:simpler}, (c) it is a moral duty to care for nature and the environment (variables qk20a183), figure \ref{fig:behaviour_opinion:moral})
 the willingness to change one's lifestyle seems most related to actions. In general, opinions which concern society in general or normative judgments are less correlated with specific actions than statements that concern the panelist herself. 


\begin{figure}[h!!]
	\centering	
	\caption{Correlation Opinion and behaviour: Willing to change lifestyle}\label{fig:behaviour_opinion}	
	\begin{minipage}[h!!]{0.32\textwidth}  
		%	\captionsetup{width=.45\linewidth}
		\centering\footnotesize{(a) Buy second-hand products regularly}
		\includegraphics[width=1\textwidth]{../codding_data/results/liss/conditional_heatmap175_135labels0.png}
	\end{minipage}
	\begin{minipage}[h!!]{0.32\textwidth}
		%	\captionsetup{width=.45\linewidth}
		\centering\footnotesize{(b) I prefer new products}
		\includegraphics[width=1\textwidth]{../codding_data/results/liss/conditional_heatmap175_148labels0.png}
	\end{minipage}
	\begin{minipage}[h!!]{0.32\textwidth}  
		%	\captionsetup{width=.45\linewidth}
		\centering\footnotesize{(c)I am open to long-term leasing}
		\includegraphics[width=1\textwidth]{../codding_data/results/liss/conditional_heatmap175_141labels0.png}
	\end{minipage}
	\floatfoot{Notes: Rows refer to the willingness to change one's lifestyle; columns represent the conditional distribution of the respective variable in the title (a) to (c) for a given category of the willingness to change.}
\end{figure}

\begin{figure}[h!!]
	\centering	
	\caption{Correlation Opinion and behaviour: We all have to live simpler lives}\label{fig:behaviour_opinion:simpler}	
	\begin{minipage}[h!!]{0.32\textwidth}  
		%	\captionsetup{width=.45\linewidth}
		\centering\footnotesize{(a) Buy second-hand products regularly}
		\includegraphics[width=1\textwidth]{../codding_data/results/liss/conditional_heatmap181_135labels0.png}
	\end{minipage}
	\begin{minipage}[h!!]{0.32\textwidth}
		%	\captionsetup{width=.45\linewidth}
		\centering\footnotesize{(b) I prefer new products}
		\includegraphics[width=1\textwidth]{../codding_data/results/liss/conditional_heatmap181_148labels0.png}
	\end{minipage}
	\begin{minipage}[h!!]{0.32\textwidth}  
		%	\captionsetup{width=.45\linewidth}
		\centering\footnotesize{(c)I am open to long-term leasing}
		\includegraphics[width=1\textwidth]{../codding_data/results/liss/conditional_heatmap181_141labels0.png}
	\end{minipage}
	\floatfoot{Notes: { 
			As in figure \ref{fig:behaviour_opinion}.}}
\end{figure}

\begin{figure}[h!!]
	\centering	
	\caption{Correlation Opinion and behaviour: It is a moral duty to care for nature and the environment}\label{fig:behaviour_opinion:moral}	
	\begin{minipage}[h!!]{0.32\textwidth}  
		%	\captionsetup{width=.45\linewidth}
		\centering\footnotesize{(a) Buy second-hand products regularly}
		\includegraphics[width=1\textwidth]{../codding_data/results/liss/conditional_heatmap183_135labels0.png}
	\end{minipage}
	\begin{minipage}[h!!]{0.32\textwidth}
		%	\captionsetup{width=.45\linewidth}
		\centering\footnotesize{(b) I prefer new products}
		\includegraphics[width=1\textwidth]{../codding_data/results/liss/conditional_heatmap183_148labels0.png}
	\end{minipage}
	\begin{minipage}[h!!]{0.32\textwidth}  
		%	\captionsetup{width=.45\linewidth}
		\centering\footnotesize{(c) I am open to long-term leasing}
		\includegraphics[width=1\textwidth]{../codding_data/results/liss/conditional_heatmap183_141labels0.png}
	\end{minipage}
	\floatfoot{Notes: { 
			As in figure \ref{fig:behaviour_opinion}.}}
\end{figure}
\newpage
\subsection{Trends over time}\label{subsec:trends}

This subsection documents a rise in the share of panelists which report some voluntary reduction in consumption/ labour supply. Figure \ref{fig:evolution_notNecessary} shows the full panel.  


\begin{figure}[h!!]
	\centering	
	\caption{Indicators of voluntary reduction over time}\label{fig:evolution_notNecessary}	
	\begin{minipage}[h!!]{0.32\textwidth}  
		%	\captionsetup{width=.45\linewidth}
		\centering\footnotesize{(a) Percentage which does not find it necessary to replace worn furniture}
		\includegraphics[width=1\textwidth]{../codding_data/results/liss/total_share_notnecessary_ci307.png}
	\end{minipage}
	\begin{minipage}[h!!]{0.32\textwidth}
		%	\captionsetup{width=.45\linewidth}
		\centering\footnotesize{(b) Percentage which does not buy new clothes regularly because deemed unnecessary}
		\includegraphics[width=1\textwidth]{../codding_data/results/liss/total_share_notnecessary_ci306.png}
	\end{minipage}
	\begin{minipage}[h!!]{0.32\textwidth}  
	%	\captionsetup{width=.45\linewidth}
	\centering\footnotesize{(c) Percentage working part time voluntarily\\ }
	\includegraphics[width=1\textwidth]{../codding_data/results/liss/total_share_voluntary_work_reduction_actual.png}
\end{minipage}
%\begin{minipage}[h!!]{0.4\textwidth}
%	%	\captionsetup{width=.45\linewidth}
%	\centering\footnotesize{(d) Share would reduce work for voluntary reason}
%	\includegraphics[width=1\textwidth]{../codding_data/results/liss/total_share_voluntary_work_reduction_willing.png}
%\end{minipage}
	\floatfoot{Notes: { The evolution of the share of panelists who think that it is not necessary to replace old furniture (panel (a), variable ci307) or to buy new clothes (panel (b), variable ci306) over time (Not yet matched with environmental dataset information). Panel size increases over waves: $\approx$700 to $\approx$1500. Panel (c) shows panelists which work part time (less than 36 hours) either because they \textit{want to take it easier} (minority) or because they want to have more leisure time. (variables cw399, cw400). (Where I dropped the year-specific part of variable names.) }}
\end{figure}

As regards figure \ref{fig:evolution_notNecessary}, the following questions arise: 
(1) who are these people who think it is not necessary to buy more clothes or to replace worn furniture?
(2) What explains the rise in the share? \ar probit/ logit over time or pooled (pooled over time)
(3) Do working hours follow consumption choices? 
(4) What explains the negative trend in part time before 2015? 

\subsubsection{Consumption graphs by group}
Figures
\ref{fig:evolution_notNecessary_bygroup:furniture} to \ref{fig:evolution_notNecessary_bygroup:clothes} depict statements on consumption behaviour over time sorted by environmental concerns. Especially panelists which buy second hand regularly, think that we have to live simpler lives, or do not prefer new products seem to drive the increase in the share which finds new clothes unnecessary. Less clear findings for other environmental indicators and the necessity to replace furniture. 

\begin{figure}[h!!]
	\centering	
	\caption{Share of panelists which don't find it necessary to replace worn furniture }\label{fig:evolution_notNecessary_bygroup:furniture}	
	\begin{minipage}[h!!]{0.32\textwidth}  
		%	\captionsetup{width=.45\linewidth}
		\centering\footnotesize{(a) Willingness to change one's lifestyle}
		\includegraphics[width=1\textwidth]{../codding_data/results/liss/broad_groups_notnecessaryqk20a175_ci307.png}
	\end{minipage}
	\begin{minipage}[h!!]{0.32\textwidth}  
	%	\captionsetup{width=.45\linewidth}
	\centering\footnotesize{(b) buys second hand regularly}
	\includegraphics[width=1\textwidth]{../codding_data/results/liss/broad_groups_notnecessaryqk20a135_ci307.png}
\end{minipage}
\begin{minipage}[h!!]{0.32\textwidth}  
	%	\captionsetup{width=.45\linewidth}
	\centering\footnotesize{(c) Prefers new products}
	\includegraphics[width=1\textwidth]{../codding_data/results/liss/broad_groups_notnecessaryqk20a148_ci307.png}
\end{minipage}
\begin{minipage}[h!!]{0.32\textwidth}  
%	\captionsetup{width=.45\linewidth}
\centering\footnotesize{(d) prefers to own}
\includegraphics[width=1\textwidth]{../codding_data/results/liss/broad_groups_notnecessaryqk20a144_ci307.png}
\end{minipage}
\begin{minipage}[h!!]{0.32\textwidth}  
%	\captionsetup{width=.45\linewidth}
\centering\footnotesize{(e) We have to live simpler lives}
\includegraphics[width=1\textwidth]{../codding_data/results/liss/broad_groups_notnecessaryqk20a181_ci307.png}
\end{minipage}
\begin{minipage}[h!!]{0.32\textwidth}  
	%	\captionsetup{width=.45\linewidth}
	\centering\footnotesize{(f) Moral duty}
	\includegraphics[width=1\textwidth]{../codding_data/results/liss/broad_groups_notnecessaryqk20a183_ci307.png}
\end{minipage}
	\floatfoot{Notes: Higher variance in the group which disagrees; could be due to smaller size. Clearer positive correlation between environmental concerns and thinking that replacement is unnecessary in levels. The rise over time, however, suggests to be similar. When grouping by whether a panelist buys second hand products regularly there is a clear higher share that thinks new furniture is not necessary. Could be that income plays a determining role here.}
\end{figure}

\begin{figure}[h!!]
	\centering	
	\caption{{Share of panelists which don't find it necessary to buy new clothes }}\label{fig:evolution_notNecessary_bygroup:clothes}	
\begin{minipage}[h!!]{0.32\textwidth}
	%	\captionsetup{width=.45\linewidth}
	\centering\footnotesize{(a) Willingness to change one's lifestyle}
	\includegraphics[width=1\textwidth]{../codding_data/results/liss/broad_groups_notnecessaryqk20a175_ci306.png}
\end{minipage}
	\begin{minipage}[h!!]{0.32\textwidth}
		%	\captionsetup{width=.45\linewidth}
		\centering\footnotesize{(b) Buys second hand regularly}
		\includegraphics[width=1\textwidth]{../codding_data/results/liss/broad_groups_notnecessaryqk20a135_ci306.png}
	\end{minipage}
	\begin{minipage}[h!!]{0.32\textwidth}
	%	\captionsetup{width=.45\linewidth}
	\centering\footnotesize{(c) Prefers new products}
	\includegraphics[width=1\textwidth]{../codding_data/results/liss/broad_groups_notnecessaryqk20a148_ci306.png}
\end{minipage}
	\begin{minipage}[h!!]{0.32\textwidth}
	%	\captionsetup{width=.45\linewidth}
	\centering\footnotesize{(d) prefers to own}
	\includegraphics[width=1\textwidth]{../codding_data/results/liss/broad_groups_notnecessaryqk20a144_ci306.png}
\end{minipage}
\begin{minipage}[h!!]{0.32\textwidth}
%	\captionsetup{width=.45\linewidth}
\centering\footnotesize{(e) We have to live simpler lives}
\includegraphics[width=1\textwidth]{../codding_data/results/liss/broad_groups_notnecessaryqk20a181_ci306.png}
\end{minipage}
	\begin{minipage}[h!!]{0.32\textwidth}
	%	\captionsetup{width=.45\linewidth}
	\centering\footnotesize{(f) Moral duty}
	\includegraphics[width=1\textwidth]{../codding_data/results/liss/broad_groups_notnecessaryqk20a183_ci306.png}
\end{minipage}
	\floatfoot{Notes: Grouping by second hand shoppers, the recent rise in the share seems to stem from households which also buy second hand regularly.  Whether a panelist prefers to buy new products is also indicative of buying new clothes regularly. Potentially steeper increase in group of environmentally friendly in panels (b), (e), and (c). }
\end{figure}


\newpage
\subsubsection{Voluntary low hours worked}
This subsection shows the share of part-time working panelists (figure \ref{fig:evolution_notNecessary} panel (c)) by environmental concerns. 

\begin{figure}[h!!]
	\centering	
	\caption{Voluntary low hours }\label{fig:evolution_wtr_willingtochange}	
	\begin{minipage}[h!!]{0.32\textwidth}  
		%	\captionsetup{width=.45\linewidth}
		\centering\footnotesize{(a)Willing to change lifestyle for environmental reasons}
		\includegraphics[width=1\textwidth]{../codding_data/results/liss/broad_groups_work_redcuctionqk20a175_actual.png}
	\end{minipage}
	\begin{minipage}[h!!]{0.32\textwidth}  
	%	\captionsetup{width=.45\linewidth}
	\centering\footnotesize{(b) Moral duty}
	\includegraphics[width=1\textwidth]{../codding_data/results/liss/broad_groups_work_redcuctionqk20a183_actual.png}
\end{minipage}
	\begin{minipage}[h!!]{0.32\textwidth}  
	%	\captionsetup{width=.45\linewidth}
	\centering\footnotesize{(c) Simpler lives}
	\includegraphics[width=1\textwidth]{../codding_data/results/liss/broad_groups_work_redcuctionqk20a181_actual.png}
\end{minipage}
	\begin{minipage}[h!!]{0.32\textwidth}  
	%	\captionsetup{width=.45\linewidth}
	\centering\footnotesize{(d) second hand regularly}
	\includegraphics[width=1\textwidth]{../codding_data/results/liss/broad_groups_work_redcuctionqk20a135_actual.png}
\end{minipage}
		\begin{minipage}[h!!]{0.32\textwidth}  
		%	\captionsetup{width=.45\linewidth}
		\centering\footnotesize{(e) Want to own things}
		\includegraphics[width=1\textwidth]{../codding_data/results/liss/broad_groups_work_redcuctionqk20a144_actual.png}
	\end{minipage}
	\begin{minipage}[h!!]{0.32\textwidth}  
	%	\captionsetup{width=.45\linewidth}
	\centering\footnotesize{(f) Prefers new products}
	\includegraphics[width=1\textwidth]{../codding_data/results/liss/broad_groups_work_redcuctionqk20a148_actual.png}
\end{minipage}
%	\floatfoot{Notes:  }
\end{figure}




All in all, working low hours seems to be positively correlated with environmental concerns. Those who have environmentally friendly attitudes, want to change or actually do change their behaviour also work less hours roughly in all years considered. 

However, the rise in the share which works part time in recent years seems uncorrelated with environmental concerns. 
For example, in panels (a), (b), (e), (f) the rise is more pronounced by the group of less environmentally friendly panelists. An exceptionThose who think we should live simpler lives (panel (c)) tend to work more in part time for voluntary reasons. The rise in recent years seems also to be driven by this group. Why is that? Could also be due to some joint factor; rr having more time = change in values = simpler is ok (habits) \ar reverse causality. In panel (d) there seems to be an equal increase across groups. 

\tr{Show growth rates!}

%\subsection{Does a reduction in consumption cause a reduction in working hours?}
%Regression!; Is there information on hours worked?
%Exploit time dimension \ar yesterday's consumption informative on today's hours worked?
%-------------------------------------
\clearpage
\bibliography{../../bib_2_0}
\addcontentsline{toc}{section}{References}
\end{document}