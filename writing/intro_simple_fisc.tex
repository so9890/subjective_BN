\section{Intro}
% this intro refers to the following setup:
% The government can only use labour taxes and corrective taxes are not available
% e.g what can the government achieve with common 

An equity-efficiency trade-off is central to the discussion of optimal labour income taxation and tax progressivity in the public finance literature.  The benefits of labour taxes and progressivity arise from redistribution. With concave utility specifications full redistribution is efficient. However, the optimal tax system does not feature full redistribution when labour supply is endogenous. Instead, redistribution is traded off against aggregate output as individuals reduce their labour supply and skill investment in response to labour income taxation. 

Adding environmental externalities to the classical public finance framework changes the perception of efficiency costs. Instead of merely reducing welfare, direct benefits through a reduction of the externality arise by lowering output. 
In theory, corrective, environmental taxes can establish the efficient allocation in a representative agent economy. Such a tax instrument is optimally set to the social cost of an externality. Originators then internalise these costs in addition to their private ones. However, governments face political difficulties in implementing such policy instruments.\footnote{\ Compare for instance the Yellow Vest movement in France in 2018.} Therefore, this paper shifts the focus to already established and widely used fiscal policy instruments. 


\paragraph{Outline}
The paper is structured as follows. In the next section I define a simple model.% to analyse the role of fiscal taxes on the environment. 
Section \ref{sec:} discusses theoretical optimal policy results. Section \ref{sec:calib} discusses plausibility of parameter values. In section \ref{sec:simul}, I show dynamics of the economy under the laissez-faire and the optimal policy regimes. 


