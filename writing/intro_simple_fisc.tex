\section{Intro}
% this intro refers to the following setup:
% The government can only use labour taxes and corrective taxes are not available
% e.g what can the government achieve with common 

% Structure Intro
% 1. Motivation: (2) setting real world, (3) Why is the question interesting? (Tradeoff)

% 2. What I do: Contribution and main finding

% 3. Model (several layers)

% 4. Calibration

% 5. Main quantitative experiment and results
% 
% To do: 
%\tr{ (i) Connect paragraphs,
% (ii) guide reader, 
% (iii) make smooth }

\textcolor{violet}{Still to do:
\begin{itemize}
	\item \textbf{Why is the economy not on a bgp right from the initial period?}
	\item \textbf{Derive optimal policy analytically to use as initial value in code!}
	\item investigate how to calibrate regeneration and emission rate in model:  \cite{Hassler2016EnvironmentalMacroeconomics}, \cite{Rogelj2018MitigationDevelopment.} \checkmark
	\item interpretation of results and how relates to other models
	\item possibilities to model technical change: substitutability of goods, growth in sector, innovation on substitutability versus consumption growth
\end{itemize}
}
\paragraph{Classical use of fiscal instruments}
An equity-efficiency trade-off is central to the discussion of optimal labour income taxation and tax progressivity in the public finance literature.  The benefits of labour taxes and progressivity arise from redistribution and from generating government revenues. 
With concave utility specifications full redistribution is efficient. However, the optimal tax system does not feature full redistribution when labour supply is endogenous. Instead, redistribution is traded off against aggregate output as individuals reduce their labour supply and skill investment in response to labour income taxation. 

\paragraph{Environmental Externality}
Adding environmental externalities to the classical public finance framework changes the perception of efficiency costs. Instead of merely reducing welfare, direct benefits through a reduction of the externality arise by lowering output. 
In theory, corrective, environmental taxes can establish the efficient allocation in a representative agent economy. Such a tax instrument is optimally set to the social cost of an externality. Originators then internalise these costs in addition to their private ones. However, governments face political difficulties in implementing such policy instruments.\footnote{\ Compare, for instance, the Yellow Vest movement in France in 2018.} On the other hand, scientific research has emphasised the urgency to act. Therefore, this paper shifts the focus of optimal environmental policy to already established and widely used fiscal tax instruments. What can be achieved in terms of climate targets and what are the costs?

\paragraph{Trade-off/ Mechanisms}
 Consumption reduction in affluent countries has also been promoted by some  scholars \citep{Schor2005SustainableReduction, Pullinger2014WorkingDesign, Arrow2004AreMuch}. But, the general equilibrium effects are less well understood.
While having an advantageous direct effect on the externality, counteracting indirect effects may arise in a general equilibrium framework. Proponents of a reduction policy especially focus on consumption by the rich which consume a higher amount of natural resources. %\footnote{\ Note Sonja: Abstracting from inequality, would it still be best to reduce consumption by the rich when the poor have a higher marginal propensity to consume dirty? \textit{Could be an important aspect in the model}. Not in the baseline, look at it in an extension...}
This concern could add to the benefits of tax progressivity.
In contrast, targeting rich households in particular for environmental reasons will lower the supply of high skilled labour.\footnote{\ The relation of labour tax progressivity and skill investment has been studied by \cite{Heathcote2017OptimalFramework}.} Yet, these skills are essentially important in greener sectors of the economy \citep{Consoli2016DoCapital}. As a result, dirty production becomes relatively cheaper and the dirty share of production rises. 
% I want to add endogenous innovation later

\paragraph{Model}
% this version: With Rep agent
I build a tractable model which incorporates the key aspects sketched above: there are two sectors one of which emits pollutants, the dirty sector. Both, clean and dirty goods are necessary inputs to the final consumption good. Sectors produce with a sector-specific labour input good. The labour input good in the clean sector contains a higher share of high-skilled labour. 
The economy behaves as if there was a representative household. The household provides high and low-skilled labour. The former exerts a higher utility cost for the household. This generates a wage premium for high-skilled labour. 

The government maximises social welfare from a Ramsey planner's perspective. However, it is constrained by an exogenous limit on emissions. The advantage of this approach is that it suffers less from  model misspesifications due to  uncertainties about how emissions affect the environment. Furthermore, it is closely related to the current political debate.\footnote{\ Compare appendix section \ref{app:emission_climate_targets} for a more in depth discussion of this aspect. } 

\paragraph{Calibration}
I inform the exogenous emission limit by the  targets proposed in the 2018 report of the Intergovernmental Panel on Climate Change (IPCC)\footnote{\ A body of the United Nations established to assess the science related to climate change.},  \cite{Rogelj2018MitigationDevelopment.}. These targets are designed for states to comply with the Paris Agreement: net greenhouse-gas emissions in 2030 shall equal 25-30 GtCO2e per year and zero in 2050 (p.95 in \cite{Rogelj2018MitigationDevelopment.}). Indded,  the agreement  foresees a tight time frame for emission reductions: climate neutrality should be achieved by mid-century.\footnote{\ Under this treaty, states have agreed on limiting temperature rise to well below 2°C, preferably 1.5°C and to achieve climate neutrality by mid-century \url{https://unfccc.int/process-and-meetings/the-paris-agreement/the-paris-agreement}. }

Another important calibration choice is the substitutability of clean and dirty production in the final consumption good. 
\tr{Read PAPER}

\paragraph{Exercise}
The paper is divided into two parts: an analytical part where I derive propositions concerning the optimal policy and a quantitative part which discusses the optimal policy and transitions. 

\paragraph{Outline}
The paper is structured as follows. In the next section, I define a simple model. % to analyse the role of fiscal taxes on the environment. 
Section \ref{sec:theory} discusses theoretical optimal policy results. Section \ref{sec:calib} argues for the plausibility of chosen parameter values. In section \ref{sec:simul}, I show dynamics of the economy under the laissez-faire and the optimal policy regimes. 

