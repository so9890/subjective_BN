\paragraph{Literature}

\paragraph{Optimal environmental policy}
\textbf{Main claim: focus on environmental taxes and recomposition}
\begin{itemize}
	\item with exogenous growth
	\item with endogenous growth
\end{itemize}

In general, papers on optimal environmental policy focus on the optimal environmental tax and analyze settings with inelastic labor supply \citep{Golosov2014OptimalEquilibrium, Acemoglu2012TheChang, Fried2018ClimateAnalysis}. Therefore, the main finding of the present paper, the necessity of reductive policy measures to implement the efficient allocation, complement this literature. Furthermore, I argue that the Pigou principle does generally not apply when no lump-sum transfers are available.  


\paragraph{Recycling of environmental tax revenues}
\textbf{Main claim: they overlook that when lump-sum transfers are not available, then labor supply is inefficiently high, and that env. tax exceeds (?) scc in optimal policy}
\begin{itemize}
	\item in general: \cite{Fried2018TheGenerations}
	\item double dividend literature
\end{itemize}
These findings have important consequences for the literature on the so-called double dividend of environmental policy and the question how to recycle environmental tax revenues. While this literature argues for the recycling of environmental tax revenues to lower pre-existing tax distortions, my paper constitutes an argument for a lower bound on distortionary income taxes: some reduction of labor supply is in fact efficient. 
Furthermore, when revenues are not redistributed lump-sum, the Pigouvian tax does not implement the efficient allocation. 

\paragraph{Environmental protection and inequality}
\ar 1) Inequality and environment as competing goods.

In the literature discussing environmental policies in an unequal framework, a competition between equity and environmental good provision have been discussed. 
This trade-off can be separated into (i) the competition for public funds \citep{LansBovenberg1996OptimalAnalyses, Jacobs2019RedistributionCurves} and (ii) effects of either environmental policies on equity or equalizing policies on environmental quality \citep{Jacobs2019RedistributionCurves, Sager2019IncomeCurves, Dobkowitz2022}. 

Since hours are inefficiently high, equalizing policies become part of the optimal environmental policy. As a byproduct, the distribution of income becomes more equal.


\ar 2) Inequality to shape effects of environmental policies and effect of fiscal policy on environment due to heterogeneity

 Furthermore, the differentiation of skills and the skill-bias of the green sector in the paper give rise to a new channel through which labor taxation affects environmental protection. The literature has primarily focused on a demand channel arising from non-linear Engel curves through which inequality and redistribution shape the degree of dirty production in the economy.   

%\textbf{Non-linear Engel Curves: redistribution}
\cite{Jacobs2019RedistributionCurves} the motive to redistribute and to provide an environmental good compete for government resources due to a negative effect of environmental taxes on the wage rate. Even with lump-sum transfers, the optimal environmental tax does not follow the Pigou principle when the government seeks to enhance equity.

\cite{Sager2019IncomeCurves} argues empirically, that redistribution to poorer households may result in a higher demand for polluting goods. 
\paragraph{Pubic Finance literature}
\textbf{I add: a new perspective on labor income taxes as a tool to lower inefficiently high hours worked. }

 \cite{Heathcote2017OptimalFramework}, \cite{Loebbing2019NationalChange}

\paragraph{Reductive policies in the literature}
The finding relates to the literature discussing rationales for the usage of reductive policy measures. These arise from o
Negative externalities of consumption and hours worked such as
 envy \cite{Alvarez-Cuadrado2007EnvyHours}, habits \cite{Ravn2006DeepHabits} \tr{Check if this is on inefficiency} or a positive externality of leisure \cite{Alesina2005WorkDifferent}. The present paper relates to this literature by identifying an externality of work which emerges from the existence of an environmental externality of production. In addition, once an environmental tax recomposes production towards a cleaner alternative, the wage rate understates the marginal product of labor. \tr{This needs to be made clearer.}
\begin{itemize}
	\item literature which \cite{Alvarez-Cuadrado2007EnvyHours}
\end{itemize}