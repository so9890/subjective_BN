\paragraph{Literature}

\paragraph{Optimal environmental policy}
\textbf{Main claim: focus on environmental taxes and recomposition}
\begin{itemize}
	\item with exogenous growth
	\item with endogenous growth
\end{itemize}

In general, papers on optimal environmental policy focus on the optimal environmental tax and analyze settings with inelastic labor supply \citep{Golosov2014OptimalEquilibrium, Acemoglu2012TheChang, Fried2018ClimateAnalysis}. Therefore, the main finding of the present paper, the necessity of reductive policy measures to implement the efficient allocation, complement this literature. Furthermore, I argue that the Pigou principle does generally not apply when no lump-sum transfers are available.  

\paragraph{The Pigou principle }


\paragraph{Recycling of environmental tax revenues}
\textbf{Main claim: they overlook that when lump-sum transfers are not available, then labor supply is inefficiently high \ar a lower bound on the optimal distortionary labor tax: even if env. tax revenues would suffice to cover exogenous funding constraint, the labor tax should be progressive, and that env. tax exceeds (? as in table \ref{tab:lin_nolst}) scc in optimal policy}
\begin{itemize}
	\item in general: \cite{Fried2018TheGenerations}
	\item double dividend literature
	\begin{itemize}
		\item \cite{LansBovenberg1994EnvironmentalTaxation}: government spending enters utility of the household \ar when public spending rises, the household feels richer \ar reduces labor supply
		\item there are no lump-sum transfers
		\item defined "first best" as: no need to generate public funds through distortionary labor taxes or $\tau_l=0$ no other pre tax distortion
		\item note that dirty and clean consumption are not necessarily perfect substitutes, as this depends on the utility function; they make no assumption; \ar labor does have a negative externality if production is never perfectly clean
		\item " Changes in employment would not affect welfare; in the absence of distortionary labor taxation, the social opportunity costs of additional employment exactly offset the social benefits."\ar this finding follows from implicitly assuming lump-sum transfers to redistribute environmental tax revenues! So that the resource constraint holds without government consumption! 
		\item when $\tau_D=SCC$ then a marginal rise in $\tau_D$ lowers welfare if labor supply decreases due to the reduction in labor tax revenues
		\item I can show that a fall in $\tau_D$ (erosion of the tax base) increases utility as it decreases government spending through the tax base if environmental tax revenues are not fully redistributed lump-sum 
		\item question: they discuss effect of a change in dirt tax but in a model derived
	\end{itemize} \citep{Barrage2019OptimalPolicy}
\end{itemize}
These findings have important consequences for the literature on the so-called double dividend of environmental policy. 
\cite{LansBovenberg1994EnvironmentalTaxation} have shown that environmental tax revenues do not constitute a double dividend in that its revenues could be used to lower distortions of fund-raising taxes. 
They argue, instead, that environmental tax revenues exert efficiency costs, too, by lowering the gains from work. When the substitution effect dominates, labor supply reduces and the tax base for the income tax diminishes. This mechanism makes it more difficult for the government to raise funds. They show further that environmental taxes exacerbate instead of alleviate pre-existing distortions \tr{WHY?}.  However, \cite{LansBovenberg1994EnvironmentalTaxation} show that it is indeed optimal to recycle environmental tax revenues to lower distortionary labor income taxes as opposed to higher lump-sum transfers. The latter in contrast to the former usage decreases labor supply even further through an income effect. This is the so-called \textit{weak} double dividend of environmental taxes. \tr{Check this understanding in \cite{Jacobs2019RedistributionCurves}.} 

Due to the labor-reducing effect of environmental taxes, the also answer the question if
in a distortionary fiscal setting, the Pigou principle still holds. In their suggested setting, the optimal environmental tax   falls short to capture the social cost of the externality. 
The government forgoes an efficient reduction of the externality in order to prevent a erosion of the tax base of fiscal policies. 
This result has gained a lot of attention in the literature \citep{LansBovenberg1996OptimalAnalyses, Barrage2019OptimalPolicy} \tr{What is the difference between Bov/Goulder 96 and BOv/deMooji 94?}

\cite{LansBovenberg1996OptimalAnalyses} show that this is not the case: when no lump-sum transfers are available and the government seeks to raise funds, the optimal environmental tax

 While this literature argues for the recycling of environmental tax revenues to lower pre-existing tax distortions, my paper constitutes an argument for a lower bound on distortionary income taxes: some reduction of labor supply is in fact efficient. 
Furthermore, when revenues are not redistributed lump-sum, the Pigouvian tax does not implement the efficient allocation. 

\paragraph{Comment: Understanding \cite{LansBovenberg1994EnvironmentalTaxation}}

The analysis in \cite{LansBovenberg1994EnvironmentalTaxation} rests on the premise that government revenues cannot be raised through lump-sum transfers. However, when there is no prerequisite to raise government funds, then environmental tax revenues are redistributed through lump-sum transfers. The second claim becomes obvious from the resource constraint, equation (1) in their paper. 

As a result, the costs of labor effort and the benefits cancel when the environmental tax is set to the social cost of pollution. However, when no lump-sum transfers are available for redistribution, then the social gains and the social costs of labor supply are not equal when the environmental tax is set to the social costs of pollution. Then, labor supply is inefficiently high: the costs of labor are not compensated for by enough through more consumption. In what way the optimal environmental tax deviates from the social cost of the externality depends on (i) how government spending affects household utility and (ii) if dirt taxes exacerbate the inefficiency of labor supply. 

To see this take the total differential of the resource constraint in \cite{LansBovenberg1994EnvironmentalTaxation} equation (1) under the assumption that all environmental tax revenues are consumed by the government: $G=\tau_D D$.

\begin{align}
hNdL = dC+dD+\tau_D dD +D d\tau_d.
\end{align}
 Then the total effect of a revenue neutral 


\paragraph{Environmental protection and inequality}
\ar 1) Inequality and environment as competing goods.

In the literature discussing environmental policies in an unequal framework, a competition between equity and environmental good provision have been discussed. 
This trade-off can be separated into (i) the competition for public funds \citep{LansBovenberg1996OptimalAnalyses, Jacobs2019RedistributionCurves} and (ii) effects of either environmental policies on equity or equalizing policies on environmental quality \citep{Jacobs2019RedistributionCurves, Sager2019IncomeCurves, Dobkowitz2022}. 

Since hours are inefficiently high, equalizing policies become part of the optimal environmental policy. As a byproduct, the distribution of income becomes more equal.


\ar 2) Inequality to shape effects of environmental policies and effect of fiscal policy on environment due to heterogeneity

 Furthermore, the differentiation of skills and the skill-bias of the green sector in the paper give rise to a new channel through which labor taxation affects environmental protection. The literature has primarily focused on a demand channel arising from non-linear Engel curves through which inequality and redistribution shape the degree of dirty production in the economy.   

%\textbf{Non-linear Engel Curves: redistribution}
\cite{Jacobs2019RedistributionCurves} the motive to redistribute and to provide an environmental good compete for government resources due to a negative effect of environmental taxes on the wage rate. Even with lump-sum transfers, the optimal environmental tax does not follow the Pigou principle when the government seeks to enhance equity.

\cite{Sager2019IncomeCurves} argues empirically, that redistribution to poorer households may result in a higher demand for polluting goods. 
\paragraph{Pubic Finance literature}
\textbf{I add: a new perspective on labor income taxes as a tool to lower inefficiently high hours worked. }

 \cite{Heathcote2017OptimalFramework}, \cite{Loebbing2019NationalChange}

\paragraph{Reductive policies in the literature}
The finding relates to the literature discussing rationales for the usage of reductive policy measures. These arise from o
Negative externalities of consumption and hours worked such as
 envy \cite{Alvarez-Cuadrado2007EnvyHours}, habits \cite{Ravn2006DeepHabits} \tr{Check if this is on inefficiency} or a positive externality of leisure \cite{Alesina2005WorkDifferent}. The present paper relates to this literature by identifying an externality of work which emerges from the existence of an environmental externality of production. In addition, once an environmental tax recomposes production towards a cleaner alternative, the wage rate understates the marginal product of labor. \tr{This needs to be made clearer.}
\begin{itemize}
	\item literature which \cite{Alvarez-Cuadrado2007EnvyHours}
\end{itemize}