The latest assessment report of the Intergovernmental Panel on Climate Change (IPCC) \citep{IPCC2022} highlights the urgency to reduce greenhouse gas emissions.%relative to the previous report from 2018 \citep{Rogelj2018MitigationDevelopment.}.
\footnote{ \  The report stresses the decreasing likelihood of meeting the Paris Agreement and limiting climate warming to 1.5°. The Paris Agreement of 2015 formulates clear political goals to mitigate climate change. Under this treaty, states have agreed on a legally binding maximum increase in temperature to well below 2°C, preferably 1.5° over pre-industrial levels, and the global community seeks to be climate-neutral in 2050  (see: \url{https://unfccc.int/process-and-meetings/the-paris-agreement/the-paris-agreement}). 
}
On the other hand, scholars have pointed to reductive policy measures to handle environmental limits \citep{Arrow2004AreMuch, Schor2005SustainableReduction, Dasgupta2021}. A reduction in work effort and consumption mitigates pollution by diminishing economic activity. Such a reduction could be achieved by using distortionary fiscal policy tools.
However, the economic literature on environmental policy has focused on  recomposing policies: environmental taxes. %\citep{Fried2018ClimateAnalysis}. 
Given the exigency to act, this paper addresses the question whether fiscal policies can help meet climate targets.

I show analytically that, indeed, once 
labor supply is elastic, reductive policy measures optimally complement the environmental tax. 
It is established in the literature that absent any other distortion, an environmental tax equal to the social cost of the externality implements the efficient allocation. 
%Environmental taxes are perceived as a cost-effective way to reduce emissions. 
I argue that this result crucially depends on the use of lump-sum transfers to redistribute environmental tax revenues. Transfers reduce labor supply through an income effect. %Thus, indeed there is a role for reductive policy measures. 
%\textcolor{blue}{This is interesting independent of whether they are feasible or not. Could relate to the fact that there is a discussion how to use revenues. Yet, one might argue that we are always in a setting with distortionary labor income taxes; so that recycling lump-sum is never needed; numbers on size of expected revenues and government spending}
When environmental tax revenues are not redistributed lump-sum, environmental taxes are optimally combined with progressive labor income taxes. The use of income taxes as a reductive policy measure is not directly targeted at the externality: the motive for labor taxation emerges from a distortion in labor markets due to lower income. Hence,  % to lower inefficiently high hours worked. 
% I show that redistributing environmental tax revenues through an income tax scheme allows to implement the efficient allocation. The optimal income tax scheme is progressive.
the optimal environmental policy equalizes the distribution of income as  a side effect. The theoretic analysis forms the first part of the paper.

In the second part,  I study a quantitative model where environmental tax revenues are consumed by the government. Current debates on how to optimally use environmental tax revenues in politics \citep{Baker2017TheDividends} and academia \citep[e.g.][]{Fried2018TheGenerations, Carattini2018} motivate to focus on this policy regime. 
In particular, I scrutinize whether progressive income taxes remain optimal in a model with endogenous growth and heterogeneous skills.
In this setting, opposing mechanisms shape the optimal income tax progressivity. On the one hand, a skill bias documented for the green sector \citep{Consoli2016DoCapital} in combination with a relatively more elastic high-skill labor supply causes a higher tax progressivity to recompose the economic structure towards dirty production. On the other hand, a higher labor share in the fossil sector implies a recomposition of the economy towards green production.
 %In the spirit of \cite{Acemoglu2002DirectedChange}, directed technical change may intensify or mitigate these channels thr recomposition. %Second, an overall reduction in labor supply curbing production may lower general research incentives.
 In addition, reductive policies could substitute for corrective taxes  which are especially costly due to knowledge spillovers when the fossil sector has the leading technology. % % more low skill supply, more fossil innovation, more fossil production, and higher low income \ar reduction in the wage premium! 
 The model suggests that the optimal income tax scheme is progressive. The benefits of labor taxation emerge from more leisure and gains from knowledge spillover when the fossil tax is partly substituted by a progressive income tax. Endogenous growth mitigates the recomposition towards the less labor intense green sector, and  price adjustments diminish a reinforcement of the skill-recomposition channel through endogenous growth. 
% labor income taxes are used to substitute for environmental taxes to realize the gains from knowledge spillovers. 
 
%\textit{I quantify the welfare gains of setting progressive income taxes to equal yyy in consumption equivalent measure. TO BE DONE  }

% relation to literature
Before dwelling on the analyses, I dedicate this and the following two paragraphs to highlighting the paper's contribution to the environmental policy discussion. 
First, the results are relevant for the political and academic debate on how  to recycle environmental tax revenues. The paper points to the importance of lump-sum transfers within the optimal environmental policy as a reductive policy tool; an aspect which appears overlooked in the discussion.%\footnote{\ POLICY debate; \cite{Fried2018TheGenerations}}
When thinking about how to recycle environmental tax revenues other than by lump-sum transfers,  one should take into account alternative reductive tools such as progressive labor income taxes. 
If the reductive part of the environmental policy is neglected, environmental taxes have to be higher to meet emission limits, as I demonstrate in the quantitative exercise.

Second, the results contribute to the academic debate on the so-called \textit{weak double-dividend} \citep[for example:][]{LansBovenberg1994EnvironmentalTaxation, LansBovenberg1996OptimalAnalyses}. The hypothesis posits that recycling environmental tax revenues to reduce preexisting tax distortions is advantageous to recycling  revenues as lump-sum transfers. The rationale is that transfers decrease labor supply thereby diminishing the tax base of the income tax. %A conflict between generating government funds and environmental protection arises. 
The findings in the present paper suggest a lower bound on the reduction in distortionary income taxes: when environmental tax revenues are not redistributed lump sum, some reduction in labor supply via distortionary income taxes is in fact efficient from an environmental policy perspective. %In other words, even if environmental tax revenues suffice to satisfy a government revenue requirement, there is a motive for progressive income taxation. 

Third, the findings are especially interesting as the provision of the environmental public good and equity have been perceived as competing targets in the literature. First, when the poor consume more of the polluting good, a corrective tax is regressive \citep{ Fried2018TheGenerations, Sager2019IncomeCurves}. % \textit{Metcalf 2007, Hassett 2009 as  in Fried 2018}. 
Second and more indirectly, a fossil tax exerts efficiency costs by lowering labor efforts\footnote{\ The reduction in hours worked is per se not inefficient. The reduction in dirty production reduces the marginal product of labor, so that the disutility of labor is not compensated enough. However, when the government seeks to tax labor income using distortionary policy tools, the reduced labor supply diminishes the tax base of the labor tax making it more costly to redistribute.} which again raises the cost for the government to redistribute \citep{Dobkowitz2022}. 
In contrast to this literature, the present paper provides an argument for progressive income taxes under perfect income-risk sharing. This suggests a double dividend of redistribution. %: equity on the one hand and efficiency gains from less labor as part of the environmental policy.
