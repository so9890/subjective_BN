
\section{Going forward: 12 August 22}
\begin{itemize}
	\item check CEV calculation and report results
	\item results without PV to understand dynamics!
	\item include results in other policy regimes\\
	Finally, in section \ref{subsec:comp_lumpsum}, I turn to analyze the optimal allocation under the alternative policy regimes: redistribution of environmental tax revenues via (1) lump-sum transfers and (2) the income tax scheme. 
	%
	\item \tr{integrated policy has an advantage even absent externality when there is endogenous growth!}
	\item why is there the strong deviation in the green to fossil energy ratio in the non-skill version with endogenous growth \ar has to stem from the higher environmental tax? \checkmark 
	\item \tr{Idea: could be that endogenous growth intensifies shift in skill ratio \ar if so, would expect a smaller reaction of skill supply in xgr model (but then the policy is differen!) Would need experiment with same policy; counterfactual}
	\item also why is there a higher environmental tax in the non-skill model? Why is it needed? \checkmark
\end{itemize}
\textbf{extensions:}
\begin{itemize}
	\item other derivation of emission target not reduction by 50\% but weighted by what the country contributes, for instance
	
	\item which  regime is best for equity measured in terms of utility, wages?lump-sum transfers or additional progressive taxes? \ar Evaluate by looking at high and low skill wages. 
	\item think about \textbf{involuntary unemployment} \ar shouldn't there be gains from an overall reduction in labor supply in terms of involuntary unemployment? As households want to work less?
	\item empirical studies motivated through this paper's results?
	\item  What if there is a pre-existing income tax but not gov funding constraint? could still find an increase in labor income tax if optimal level is above initial level
	\item  look at a policy where income taxes are used to fund government spending \ar then this would generate more gov. revenues
	\item commment on whether there is a double dividend from using income taxes
	\item add inequality and heterogenous consumption bundles to model and ask about changes in the distribution of income
\end{itemize} 
\subsection{Sensitivity}
I will now briefly discuss sensitivity analyses to the quantitative exercise. 
\subsubsection{Wage elasticity of labor}

Recent papers have examined the wage elasticity of labor. \cite{Boppart2019LaborPerspectiveb} present evidence that hours worked per worker have been falling steadily over time 

\subsubsection{Research subsidy}
but finding should be similar to version without endogenous growth
\subsubsection{Changing emission limits calculation}
The \textit{equal-per-capita} approach is favorable for population high countries like the US. Therefore, in the sensitivity analysis, I rerun the model where US emission limits follow from \textit{Equal cumulative per capita} approach, where countries with historically  high emissions per capita mitigate more.
% \textit{constant-emission-ratio} approach which is achieved by all countries reducing emissions by 50\% in the 2030s relative to 2019. This principle limits US emissions to 2.309Gt in the 2030s. THIS APPROACH IS EVEN MORE FAVOURABLE  TO THE US BECAUSE CURRENTLY THEY EMIT A HIGHER SHARE PER CAPITA THAN THE REST OF THE WORLD. 

\subsubsection{Technology gap}