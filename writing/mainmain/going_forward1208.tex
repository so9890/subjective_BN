
\section{Going forward: 12 August 22}

Importantly, the equity and the environmental targets of government intervention are perceived as competing goals as both tax instruments exert efficiency costs through a reduction in labor supply.
I argue in this paper that what is perceived as an efficiency cost -  the reduction in labor supply - is part of the optimal environmental policy. Hence, income taxation has a double dividend: an environmental and an equity one.   




\begin{itemize}
	\item to think about: why does the ramsey allocation not achieve the same growth rates as the efficient one? would need additional instruments.
	\item !!! why does the green-to-fossil ratio increase when the labor income tax is progressive in the model with exogenous growth?
	\item compare environmental tax in model without labor income tax and gov consumption to model with lump sum redistribution and without labor income tax \ar how does the environmental tax change? Is it higher?
	\item why this downward sloping tax progressivity? The social planner also chooses an increasing labor supply; could depends on the marginal utility of consumption, but consumption increases. Why does work become more valuable again? 
	Bcs due to rising technology labor becomes more productive. 
	\item analytically derive mechanisms driving direction of innovation in model
	\item check CEV calculation and report results
	\item results without PV to understand dynamics!
	\item include results in other policy regimes\\
	Finally, in section \ref{subsec:comp_lumpsum}, I turn to analyze the optimal allocation under the alternative policy regimes: redistribution of environmental tax revenues via (1) lump-sum transfers and (2) the income tax scheme. 
	%
	\item \tr{integrated policy has an advantage even absent externality when there is endogenous growth!}
	\item why is there the strong deviation in the green to fossil energy ratio in the non-skill version with endogenous growth \ar has to stem from the higher environmental tax? \checkmark 
	\item \tr{Idea: could be that endogenous growth intensifies shift in skill ratio \ar if so, would expect a smaller reaction of skill supply in xgr model (but then the policy is differen!) Would need experiment with same policy; counterfactual}
	\item also why is there a higher environmental tax in the non-skill model? Why is it needed? \checkmark
\end{itemize}
\textbf{extensions:}
\begin{itemize}
	\item other derivation of emission target not reduction by 50\% but weighted by what the country contributes, for instance
	
	\item which  regime is best for equity measured in terms of utility, wages?lump-sum transfers or additional progressive taxes? \ar Evaluate by looking at high and low skill wages. 
	\item think about \textbf{involuntary unemployment} \ar shouldn't there be gains from an overall reduction in labor supply in terms of involuntary unemployment? As households want to work less?
	\item empirical studies motivated through this paper's results?
	\item  What if there is a pre-existing income tax but not gov funding constraint? could still find an increase in labor income tax if optimal level is above initial level
	\item  look at a policy where income taxes are used to fund government spending \ar then this would generate more gov. revenues
	\item commment on whether there is a double dividend from using income taxes
	\item add inequality and heterogenous consumption bundles to model and ask about changes in the distribution of income
\end{itemize} 

\subsection{Literature}

\paragraph{endogenous growth and distortionary taxes}
Fullerton and Kim 2008
\cite{Bovenberg1997EnvironmentalGrowth}
Hettich 1998
Lighart and van der Ploeg 1994
\cite{Loebbing2019NationalChange}


\paragraph{reduction policies and their optimality}
\begin{itemize}
	\item due to social preferences (envy, keeping up with the Joneses, habits) \ar read Layard 2006 on Happiness
	\item due to environmental limits
	\item do not include LIMITS to GROWTH literature (this seems to be a different question )
\end{itemize} 

\paragraph{Pigou and optimal tax deviation}

%%%------------------------------------------------
% Deviation from the pigou principle: The effect of fiscal distortions on the optimal environmental policy
%%%--------------------------------------------------------------------

Another realm the literature which combines fiscal and environmental policy considers is the deviation of the environmental tax from the Pigou principle. In the classic setup, the government faces an exogenous funding condition. This motivates using labor income taxes to generate funds. 
The environmental tax reduces the wage rate - an efficient decline from an environmental perspective - depressing the tax base of the income tax if the uncompensated wage elasticity of labor is positive.  This is why the two motives of government intervention, the environmental externality and generating revenues, compete.\footnote{\ Importantly, using environmental tax revenues to fund the government is more costly as opposed to labor income taxes due to an additional distortion generated from envrionmental taxes: environmental taxes distort commodity in addition to reducing labor supply. This argument rejects the strong double dividend hypothesis and was brought forward by \cite{LansBovenberg1994EnvironmentalTaxation}. } 
In order to satisfy the funding constraint, the optimal environmental tax falls short of the social costs of the externality; the Pigou principle is violates \citep{LansBovenberg1996OptimalAnalyses}. \cite{Barrage2019OptimalPolicy} studies the role of fiscal distortions for the optimal environmental policy in a dynamic setting with climate cycle. 
The results presented in this paper connect to the considerations on whether and why the environmental tax deviates from the social cost of the externality. While it is the second target, i.e., to generate funds, which rationalizes a lower environmental tax, the reason of the deviation in my paper emerges from the non-redistribution of environmental tax revenues. This reduces consumption and thereby utility so that the government seeks to reduce environmental tax revenues.

\subsection{Sensitivity}
I will now briefly discuss sensitivity analyses to the quantitative exercise. 
\subsubsection{Wage elasticity of labor}

Recent papers have examined the wage elasticity of labor. \cite{Boppart2019LaborPerspectiveb} present evidence that hours worked per worker have been falling steadily over time 

\subsubsection{Research subsidy}
but finding should be similar to version without endogenous growth
\subsubsection{Changing emission limits calculation}
The \textit{equal-per-capita} approach is favorable for population high countries like the US. Therefore, in the sensitivity analysis, I rerun the model where US emission limits follow from \textit{Equal cumulative per capita} approach, where countries with historically  high emissions per capita mitigate more.
% \textit{constant-emission-ratio} approach which is achieved by all countries reducing emissions by 50\% in the 2030s relative to 2019. This principle limits US emissions to 2.309Gt in the 2030s. THIS APPROACH IS EVEN MORE FAVOURABLE  TO THE US BECAUSE CURRENTLY THEY EMIT A HIGHER SHARE PER CAPITA THAN THE REST OF THE WORLD. 

\subsubsection{Technology gap}

\paragraph{Sensitivity}
\begin{itemize}
	\item utility specification (Building on Bick can think of European version when substitution effect is stronger)
	
	Since the target of the labor tax in the environmental setting presented here is to align hours worked with their efficient level, results are sensitive to the elasticity of labor with respect to after-tax wages. 
	Quantitative finding to be shaped by income and substitution effect!
	Literature on how households react to changes in income \cite{Bick2018HowImplications} and \cite{Boppart2019LaborPerspectiveb}
	
	\item spillovers across scientists: with positive spillovers potentially no growth \ar then connects to degrowth!
\end{itemize}