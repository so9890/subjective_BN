
\section{Model}
This section presents the model building on \cite{Fried2018ClimateAnalysis}. 
The demand side behaves as if there was a representative family. The representative household provides two skills: high and low which are used in different shares for the production of the neutral, fossil and green intermediate good.
Endogenous growths is modeled in form of directed technical change resulting from research. The government seeks to maximise utility of the representative household under the constraint of meeting an exogenous emission target. Emissions arise from the usage of fossil energy in the final good production. 
I study the model for a fixed amount of periods as I do not want to make any assumption on steady growth due to the absolute constraint on fossil production. 

\paragraph{Households}
% the rep agent
 The household chooses hours of high- and low-skilled workers and average consumption taking prices as given. The share of worker types is fixed with a lower share of high-skilled workers, $z_h$, resulting in a skill premium. The household's problem reads

\begin{align}
%U=
\underset{\{\{C_{t}\}_{t=0}^{T}, \{h_{lt}\}_{t=0}^{T}, \{h_{ht}\}_{t=0}^{T}\}}{max}&
\sum_{t=0}^{T}\beta^t u(C_{t}, h_{lt}, h_{ht})\\
%U_{s}=\underset{\{c_{st}\}_{t=0}^{\infty}, \{h_{st}\}_{t=0}^{\infty}}{max}&\sum_{t=0}^{\infty}\beta^t u_s(c_{st}, h_{st}; S_t)\\
s.t.& \ \ p_{t}C_{t}\leq% (1-\tau_{lt})(h_{ht}w_{ht}+h_{lt}w_{lt})+T_t\\ 
z_h\lambda_t \left(h_{ht}w_{ht}\right)^{1-\tau_{lt}}+(1-z_h)\lambda_t\left(h_{lt}w_{lt}\right)^{1-\tau_{lt}}+T_t\\
\ & h_{ht}\leq \bar{H}_t\\
\ & h_{lt}\leq \bar{H}_t
\end{align}
The government levies a tax on skill-level income. The tax schedule is characterised by  a scaling factor, $\lambda$, and a parameter determining the progressivity of the tax schedule, $\tau_{lt}$, \citep[compare, e.g.,][]{Heathcote2017OptimalFramework}. 

\paragraph{Production}
Production separates into final good production, energy production, intermediate good production, and the production of machines and the intermediate labour input good. 

The final good producing sector is perfectly competitive combining the neutral and energy intermediate input goods according to:
\begin{align}
Y_t=\left[\delta_y^\frac{1}{\varepsilon_y}E_{t}^{\frac{\varepsilon_y-1}{\varepsilon_Y}}+(1-\delta_y)^\frac{1}{\varepsilon_y}N_{t}^{\frac{\varepsilon_y-1}{\varepsilon_y}}\right]^\frac{\varepsilon_y}{\varepsilon_y-1}
\end{align} 
I take the composite good as the numeraire and define its price as $p_t=\left[\delta_y^?p_{Et}^{1-\varepsilon_y}+(1-\delta_y)p_{Nt}^{1-\varepsilon}\right]^{\frac{1}{1-\varepsilon}}$.

Energy producers perfectly competitively combine fossil and green energy to a composite energy good:
\begin{align}
E_t=\left[F_t^\frac{\varepsilon_e-1}{\varepsilon_e}+G_t^\frac{\varepsilon_e-1}{\varepsilon_e}\right]^\frac{\varepsilon_e}{\varepsilon_e-1}.
\end{align}
The price of Energy is determined as  $p_{Et}= \left[p_{Ft}^{1-\varepsilon_e}+p_{Gt}^{1-\varepsilon_e}\right]^\frac{1}{{1-\varepsilon_e}}$.

Intermediate goods, $F_t, G_t$, and $N_t$ are again produced by competitive sectors by use of a sector-specific labour input good and machines. The production function in sector $J$ reads
\begin{align}
&J_{t}= L_{jt}^{1-\alpha_j}\int_{0}^{1}A_{jit}^{1-\alpha_j}x_{jit}^{\alpha_j} di.
\end{align}
$A_{jit}$ determines the productivity of machine i in sector j at time t: $x_{jit}$.

The labour input good is produced by a perfectly competitive and sector-specific labour industry according to: 
\begin{align}
L_{jt}=h_{hjt}^{\theta_j}h_{ljt}^{1-\theta_j}.
\end{align}
This additional intermediate sector allows to capture differences in skills by sector, especially in the fossil and green sector.

Machine producers are imperfect monopolist and produce sector $J$ specific machines. Machine producers maximise their profits by chosing the price at which to sell their machines to intermediate good producers. Machine producers also decide on the amount of scientists to employ. Growth is machine specific. As the productivity of machine $ij$ increases, demand for this machine increases too. Irrespective of the sector, the production of one machine costs one unit of the final output good.
Each period, machine producers solve
\begin{align}
\underset{p_{xjit}, s_{jit}}{\max}p_{xjit}(1-\zeta_{jt})x_{jit}-x_{jit}-w_{sjt}s_{jit}.
\end{align}
taking as given the law of motion of technology and internalising demand for machines as a function of technology and the price of machines. 
The government subsidises machine production by $\zeta_{jt}$ which is set to implement the socially optimal level of machines and correct for the monopolistic structure of machine markets. 

\paragraph{Research and technology}
Technological growth is endogenously determined by research and spillover effects. The marginal product of research determines the amount of researchers employed and hence growth. 
The law of motion of technology is defined as
\begin{align}
A_{jit}=A_{jit-1}\left(1+\gamma\left(\frac{s_{jit}}{\rho_j}\right)^\eta\left(\frac{A_{t-1}}{A_{jt-1}}\right)^\phi\right)
\end{align}
I make the following definitions
\begin{align}
A_{jt}=\int_{0}^{1}A_{jit}di\ \forall j,\\
A_{t}=\frac{\rho_fA_{ft}+\rho_gA_{gt}+\rho_n A_{nt}}{\rho_f+\rho_g+\rho_n}.
\end{align}
The private benefits of research for machine producers diverges from the social benefits as they do not observe the effect of today's research on tomorrow's productivity due to the positive spillovers for all research sectors captured by the term $\left(\frac{A_{t-1}}{A_{jt-1}}\right)^\phi$. 

The supply of scientists is endogenous in my model. With this choice I depart from the standard assumption of a fixed supply of scientists in the literature on directed technical change \cite{Acemoglu2012TheChange, Fried2018ClimateAnalysis}. 

Scientists form another sector in the economy, they are risk neutral. Modelling the supply of researchers flexibly gives more freedom for the planner to choose lower growth levels: no a-priori fixed amount of research has to be employed. Furthermore, I do not assume free movement of scientists which simplifies the numeric calculation of equilibria in a model without balanced growth. 
Scientists in each sector behave according to 
\begin{align}
\underset{s_{jt}}{\max}\ \ & w_{jst}s_{jt}-\chi_s \frac{s_{jt}^{1+\sigma_s}}{1+\sigma_s}\\
s.t. \ \ & s_{jt}\leq \bar{H}.
\end{align}

I assume that all income of a scientist is confiscated by the government which simplifies notation. The assumption that scientists are risk neutral, introduces an additional externality as scientists do not internalise the social value of their research on society which is shaped by the shadow value of income. The advantage of this specification is that it prevents income tax parameters to affect the supply of scientists allowing to focus on the supply of hours by workers and consumption as the channels through which income taxes affect emissions. 
\begin{comment}
\paragraph{Impossibility of reaching target in laissez-faire with exogenous growth}
\tr{Note that this is wrong! There is an option for the gov to affect inflation which then redirects demand.}
Note that with exogenous growth in each sector there is no possibility for the government to stop emissions from growing, since production of the dirty good is essential for the consumption good (no perfect substitution: $\varepsilon<\infty$). To meet the emission target, the government either needs to affect the growth rate in the economy; i.e., $\upsilon_j$ is a choice variable, or work and consumption need to be set to zero; or the emission target has to be defined in relative terms. The latter possibility contradicts the Paris Agreement which is concerned with absolute emissions.  
I therefore assume, that the government can change the growth rate.

The government chooses the growth rate in each sector, taking into account that research is constrained by an exogenous  amount of scientists
\begin{align}
\upsilon_{ct}+\upsilon_{dt}\leq\Upsilon
\end{align}
\end{comment} 
  
\paragraph{Markets}
In equilibrium markets have to clear. I explicitly model markets for skill and the final consumption good
\begin{align}
z_h h_{ht}&=h_{hft}+h_{hgt}+h_{hnt}\\
(1-z_h) h_{lt}&=h_{lft}+h_{lgt}+h_{lnt}\\
C_t&=Y_t-\int_{0}^{1}x_{fit}+x_{git}+x_{nit}di
\end{align}
\paragraph{Government}

The government maximises social welfare but is constrained by meeting emission targets in line with the Paris Agreement. Social welfare is defined as the equally weighted sum of utilities in the economy. The government can use income taxes and corrective taxes levied on fossil sales to maximise social welfare. The planner solves:

\begin{align*}
\underset{\{\tau_{ft}\}_{t=0}^{T},\{\tau_{lt}\}_{t=0}^{T}}{max}&\sum_{t=0}^{T}\beta^t\left( u(C_{t}, h_{ht}, h_{lt})-\chi_s\frac{s_{ft}^{1-\sigma_s}}{{1-\sigma_s}}-\chi_s\frac{s_{gt}^{1-\sigma_s}}{{1-\sigma_s}}-\chi_s\frac{s_{nt}^{1-\sigma_s}}{{1-\sigma_s}}\right) \\
s.t.\ %& (1)\  \tau_{lt}(h_{ht}w_{ht}+h_{lt}w_{lt})=T_t\  \forall \ t\geq 0\\
& (1)\ \omega F_{t} -\delta \leq \Omega_t \  \forall \ t\geq 0 \\ %\hspace{3mm} \text{(emission target)}\\
& (2)\ z_h\left(w_{ht}h_{ht}-\lambda_t \left(w_{ht}h_{ht}\right)^{1-\tau_{lt}}\right)+(1-z_h)\left(w_{lt}h_{lt}-\lambda_t\left(w_{lt}h_{lt}\right)^{1-\tau_{lt}}\right) +\tau_{ft}p_{ft}F_{t} \\ 
& +\sum_{j\in\{f,g,n\}}\left(\int_{0}^{I}\pi_{jit}di-\zeta_{jt}\int_{0}^{I}p_{jixt}x_{jit}di+w_{sjt}s_{jt}\right)= T_t\hspace{3mm}\\
%& (3)\ \upsilon_{ct}+\upsilon_{dt}\leq\Upsilon\  \forall \ t\geq 0\\
& (3)\ \text{behaviour of firms and households}\\
& (4)\ \text{feasibility}
\end{align*}

$\Omega_t$ are flow emissions in period $t$.  The parameter $\delta$ captures the capacity of the environment to reduce emitted CO2 through sinks, such as forests and moors.  For simplicity, I assume that the sink capacity is constant. The parameter $\omega$ determines greenhouse-gas emission in CO2 equivalents caused by the fossil sector which is assumed to be the sole source of emissions in the model. % \tr{Read up in \cite{Hassler2016EnvironmentalMacroeconomics} what possibilities there are in the literature}
%Hence, under the emission target it has to hold that $Y_{nt}=\frac{\delta+E_t}{\kappa}$ assuming that the analysis starts in 2020.
The government generates revenues from taxing labour income and redistributes to run a balanced budget. 


