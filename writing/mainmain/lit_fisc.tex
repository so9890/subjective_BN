\paragraph{Literature}

The paper is related to 3 strands of literature. 

First, to the public finance literature.  \cite{Heathcote2017OptimalFramework} study optimal labour tax progressivity on 



Second, to the literature on optimal environmental policy. 

Third, to the literature on directed technical change. 
\textbf{HEMOUS and Olsen} discuss an endogenous growth model with heterogeneous labour input:
\begin{itemize}
	\item the wage premium is not constant on a BGP which they specify as stable if innovation occurs in both sectors
	\item hence: a non stable BGP is one where innovation does not occur in both sectors at some point
	\item need to allow for this option< when solving the model
	\item quality ladder model: each scientist after having chosen a sector of production, there is no congestion (each scientist works on one machine) legitimate due to within-sector spillovers
	\item on a BGP with equal growth the wage premium may grow (Result in \cite{Acemoglu2002DirectedChange}) 
	\item in \cite{Acemoglu2012TheChange} 
	\begin{itemize}
		\item as emissions are proportional to dirty output implicit assumption of a Leontief production function if there was energy (and also this as only source of emissions); 
		 \item endogenous labour (no sector-specific labour supply) \ar the more productive sector attracts more labour (the MPL is higher at an equal ratio so that more labour ends up in the more productive sector to have equal wages)
		 \item for substitutes innovation might be stuck in the more advanced market as the price effect (which directs innovation to the less productive market) is muted
		 \item[\ar] in \cite{Fried2018ClimateAnalysis} fossil and green energy are substitutes \ar stuck in fossil innovation; but non-energy goods and energy are complements \ar price effect strong; equalising effect
		 \item in LF stuck with dirty innovation, government can redirect innovation towards the clean sector until it catches up \ar policy intervention needer for a " sufficient amount of time" \ar might be missing in today's world
		 \item with DTC need postponing intervention problematic!
		 \item subsidies and corrective taxes needed to implement first best 
	\end{itemize}
\item \cite{Acemoglu2016TransitionTechnology} incremental (sector-specific) and radical (building on the leading technology irrespective of sector) innovation \ar cross-sectoral spillovers which were absent in \cite{Acemoglu2012TheChange}
\end{itemize}

%Finally, the paper is meant to add to the discussion on reduction versus recomposition policies as tools to reduce human impact on the environment. 