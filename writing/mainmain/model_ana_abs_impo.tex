\section{Core Model}\label{sec:mod_an}
%\tr{if $tau_f$ was to replicate share of marginal products, then $H^*\geq H_{FB}$; i.e. absent lump-sum transfers (sub-optimal setting)}
%\textbf{Points to be made}
%\begin{enumerate}
%\item the efficient allocation consists of both a recomposing and a scaling element \ar discuss social planner allocation \checkmark
%\item if $tau_f$ was to replicate share of marginal products, then $H^*\geq H_{FB}$; i.e. absent lump-sum transfers (sub-optimal setting) \checkmark
%\item lump-sum transfers implement the efficient reduction in hours worked and ensure the efficient amount of consumption \checkmark
%\item absent lump-sum transfers and income, households work too much  \checkmark; This follows from setting $\tau_\iota>0$.
%\item redistribution through the income tax scheme establishes the efficient allocation if the income tax is progressive \textit{Intuition: in contrast to lump-sum transfers, transferring environmental tax revenues through the income tax scheme constitutes a positive multiplication of labor income \ar this increases labor efforts. The progressive tax counters this tendency.} \checkmark
%\end{enumerate}

This section develops a general model which is at the core of the analytical and quantitative results. 
The model presented in this section is designed as simple as possible to derive  the main theoretical results and to provide intuition. In section \ref{sec:model}, the model is extended to the quantitative framework, notably by adding endogenous growth and skill heterogeneity. %investigate the inefficiency arising in hours worked when an environmental externality has to be taken care of.

In the model, the household sector can be described by a representative household. The household faces a consumption and labor supply decision. The final consumption good is a composite of a dirty and a clean good. Labor is the only input to production. For simplicity, the clean sector does not induce any externality; yet, whenever intermediate goods are no perfect substitutes, final good production is never perfectly clean. The core model abstracts from endogenous growth; the model becomes dynamic an innovation decision is added in the quantitative model (section \ref{sec:model}).

\paragraph{Representative household}
The representative household maximizes its life-time utility. Throughout the paper, household's decisions are static corresponding to maximizing the period utility
\begin{align}
U(C,H; F)
\end{align} 

The household derives utility from consumption, $C$, but experiences disutility from hours spent working, $H$. An externality from dirty (or fossil) production, $F$, decreases household utility and is taken as given by the household.
I assume additive separability of consumption, hours, and the externality. I further assume that utility of consumption is increasing and concave. As regards hours and the externality, utilty is decreasing and convex.
Utility maximization is subject to a period budget constraint
\begin{align}
	 C= \lambda(wH)^{1-\tau_{\iota}}+T_{ls}. \label{eq:hhbudget}
\end{align}

The varibale $w$ indicates the wage rate, and $T_{ls}$ denotes lump-sum transfers from the government.
The planner levies income taxes on labor income using a non-linear tax scheme common in the public finance literature \citep{Heathcote2017OptimalFramework, Benabou2002TaxEfficiency}. The tax scheme is
characterized by (i) a scaling factor, $\lambda$, which determines the level of average tax revenues in the economy and (ii) a measure of the tax progressivity denoted by $\tau_{\iota}$. 
\cite{Heathcote2017OptimalFramework} show that whenever $\tau_{\iota}>0$ the tax scheme is progressive since the marginal tax rate exceeds the average tax rate irrespective of  pre-tax labor income. Hence, average tax payments increase with labor income. An alternative intuition is that when $\tau_{\iota}>0$, the elasticity of post- to pre-tax labor income is smaller unity for all levels of pre-tax labor income.  %\footnote{\ I show that the result is equivalent with a linear tax rate in the appendix.} 

\paragraph{Production}
All sectors of production are perfectly competitive. The final consumption good, $Y$, is a composite of the dirty, $F$, and the clean intermediate good, $G$. 
Intermediate goods are produced from labor, $L_j$, and technology, $A_j$, where $j\in \{f,g\}$ indicates the dirty and the clean sector. A capital letter indicates intermediate, $\{G,F\}$ and final good, $Y$, production functions: 
\begin{align}
Y=F_Y(F, G), \hspace{5mm} F=F_F(A_f, L_f),\hspace{5mm} G=F_G(A_g, L_g) \label{eq:prod}
\end{align}

\paragraph{Government}
The government raises income taxes from households and levies ad-valorem sales taxes, $\tau_f$, on dirty producers' revenues $p_FF$, where $p_F$ denotes the price of the dirty good paid by final good producers. Revenues from the income tax and the environmental tax are treated separately by the government. Income tax revenues are fully redistributed through the income tax schedule while environmental tax revenues are either lump-sum redistributed to households, $T_{ls}$, consumed by the government, $Gov$, or redistributed through the income tax scheme, $T_\iota$:
\begin{align}
\tau_{f}p_FF=T_{ls}+T_\iota+Gov, \hspace{7mm}
0={w h}-\lambda(w h)^{1-\tau_{\iota}}+T_\iota. \label{eq:gov_but}
\end{align}
%Environmental tax revenues are either transferred lump-sum, fully consumed by the government, or transferred through the income tax schedule.

\paragraph{Markets}
The market for labor and the final good both clear: 
\begin{align}
H=L_f+L_g,\ \hspace{5mm} Y=C+Gov. \label{eq:market_clear}
\end{align}
 The final good, $Y$, is the numéraire. In this simple model, labor moves freely between the clean and dirty sector. 
%I summarize the equations determining the competitive equilibrium in appendix section \ref{app:model}.
\subsection{Competitive equilibrium}
In a competitive equilibrium, household behavior is described by the budget constraint, equation \ref{eq:hhbudget}, and labor supply which follows from the household first order conditions and substituion of $\lambda$ from the government's budget on the income tax:
\begin{align}
-U_H=U_C(1-\tau_{\iota})w\left(1+\frac{T_\iota}{wH}\right). \label{eq:hsup}
\end{align}
Firms choose the quantity of input goods to maximize their profits taking prices as given. The following equations describe this behavior in equilibrium:
\begin{align}
p_G=\frac{\partial Y}{\partial G}, \hspace{5mm}
p_F = \frac{\partial Y}{\partial F}, \hspace{5mm}
w= p_F(1-\tau_f)\frac{\partial F}{\partial L_f}=p_G\frac{\partial G}{\partial L_g}\label{eq:profmax}
\end{align}

The competitive equilibrium is defined as a set of variables so that households and firms behave optimally; i.e. equations \ref{eq:hhbudget}, \ref{eq:hsup} and \ref{eq:profmax} hold. Production happens according to \ref{eq:prod}.  Equilibrium prices and the wage rate adjust to clear markets, equations \ref{eq:market_clear}. Finally, the government's budgets are satisfied \ref{eq:gov_but}. Policy variables $\tau_f$ and $\tau_\iota$ are taken as given. 

\section{Theoretic results}
This section derives and discusses the main theoretical results. Subsection \ref{subsec:sp} defines the efficient allocation. It constitutes a benchmark for the optimal allocation a discussion of which follows suit in subsection \ref{subsec:decen_ec}. 
\subsection{Social planner}\label{subsec:sp}
Let the share of dirty to total labor be denoted by $s=L_f/H$. The social planner's problem reads
\begin{align}
\underset{s, H}{\max}\ & U(C,H; F)\\ s.t\ \ & C=Y.
\end{align}
The first order conditions are given by
\begin{align}
wrt.\ s:\hspace{4mm} & U_C\left(\frac{dY}{dF}\frac{dF}{ds}+\frac{dY}{dG}\frac{dG}{ds}\right)=-U_F\frac{dF}{ds}, \label{eq:fbs}
\\
wrt.\ H:\hspace{4mm} & U_C\frac{dY}{dH}+U_F\frac{dF}{dH}=-U_H\label{eq:fbh}. 
\end{align}
These equations determine the efficient or first-best allocation. 
Absent an externality, $U_F=0$, the efficient distribution of labor across sectors equalizes the marginal product of labor across sectors; compare equation \ref{eq:fbs}. Efficient hours balance the marginal utility gain from consumption and the disutility from working formalized by equation \ref{eq:fbh}. 

When there is an externality, the social planner adjusts the allocation by two modulations: (i) a recomposing and (ii) a scaling one. 
The recomposition is determined by equation \ref{eq:fbs}.
The negative externality of dirty production  lowers the efficient share of labor allocated to  the dirty sector so that the marginal product of labor in the dirty sector exceeds the marginal product in the clean sector; note that $U_F<0$ by assumption so that the right-hand side is positive and that $\frac{dG}{ds}<0$.
\begin{comment}
The equation 
\begin{align}
\frac{-U_F}{U_C \frac{dY}{dF}}=1+\frac{\frac{dY}{dG}\frac{dG}{ds}}{\frac{dY}{dF}\frac{dF}{ds}}.
\end{align}
The term on the left-hand side is the social cost of the externality: it measures what the representative household is willing to pay for a further reduction in dirty production. 
\end{comment}

The scaling effect is summarized by equation \ref{eq:fbh}.
There are two reasons for why the efficient level of hours worked changes. 
First, the recomposition of labor input towards the  clean sector reduces the marginal product of labor, $\frac{dY}{dH}$, and the marginal increase in consumption for an additional hour worked declines.  This effect has two opposing impacts on efficient labor supply. On the one hand, there is a substitution effect: as leisure becomes less costly, the efficient amount of hours reduces (note that the right-hand side of equation \ref{eq:fbh} is increasing in $H$). On the other hand, the economy becomes poorer and more work effort might become efficient. This is equivalent to an income effect captured by the term $U_C$. 
%In total, which effect dominates depends on the curvature of the utility from consumption, $\theta$. With $\theta>1$ the  lower marginal product of labor decreases the efficient amount of hours worked. 
Second, the social planner reduces hours worked due to their negative exeternality through dirty production. This effect is introduced by the term $U_F\frac{dF}{dH}<0$. 

Depending on the importance of the income effect, efficient hours worked may be higher or lower than  absent an externality. %\footnote{ \ I discuss in the appendix conditions on parameter values when assuming functional forms of the model.}
I will show in the following, that irrespective of whether the social planner de- or increases hours, the decentralized economy always features higher hours when environmental taxes are the only instrument. That is, environmental tax revenues are not redistributed. 

\begin{comment}
\hrule
One can show that the total effect of a drop in the dirty labor share on hours worked is positive, i.e. $\frac{dh_{FB}}{ds}>0$, if $\theta<\frac{\varepsilon}{\varepsilon-s}$. If the income effect dominates, the social planner increases hours worked as the economy becomes less productive. 
Under the value for $\theta$ suggested by \cite{Boppart2019LaborPerspectiveb}, the efficient scale effect is to increase hours worked. When, however, the substitution effect outweighs or dominates the income effect - as commonly assumed in the public finance literature \citep{Heathcote2017OptimalFramework, LansBovenberg1994EnvironmentalTaxation, LansBovenberg1996OptimalAnalyses} \tr{CHECK this}!.
Nevertheless, the level of hours worked exceeds the efficient level irrespective of $\theta$ when no lump-sum transfers are available. 
When the efficient level of hours increases, though, the dirty labor share reduces even more to outweigh the increase in the externality.

content...
\end{comment}
\subsection{Decentralized economy}\label{subsec:decen_ec}

In today's market economies, a planner to allocate hours worked and consumption does not exist. Instead, governments can revert to tax and transfer instruments to correct for market distortions, such as an environmental externality. The question arises if the efficient allocation can be decentralized by the use of taxes and transfers in a competitive economy. And if so, how? %For now, I assume that the income tax is not available and $\tau_{\iota}=0$, $\lambda=0$.

I show in this section that redistribution of environmental tax revenues are essential to implement the first-best allocation in the competitive equilibrium. Only in combination with lump-sum transfers of  environmental tax revenues does an environmental tax suffice to implement the efficient allocation. %Then the environmental tax equals the social cost of the externality as shown by \textit{PIGOU}. 
When lump-sum transfers are not available, hours worked exceed their efficient level, and a role for income taxes to lower hours worked arises. I consider two cases.
In the first case, environmental tax revenues are consumed by the government. Under the assumption of non-increasing returns to scale, the optimal policy consists of (i) a progressive labor income tax scheme and (ii) an environmental tax which deviates from the social cost of the externality. The logic is that labor taxes help to align hours worked closer to the efficient allocation. 
Nevertheless, the efficient allocation is not feasible under this policy set-up without redistributing environmental tax revenues.

In the second scenario, therefore, I point to an option to implement the efficient allocation even if lump-sum transfers are not available: redistributing environmental tax revenues through the income tax scheme.
 I show that, again, the optimal tax scheme is progressive. 
As a consequence, the considered optimal environmental policies which establish the efficient allocation feature - as a side effect - a more equal distribution of income, through either lump-sum transfers or a progressive tax scheme.% \tr{Not sure though if this holds true in progressive scheme as lambda multiplies labor income}).

%\begin{enumerate}
%\item lump-sum transfers important for Pigou tax to implement efficient allocation: Proposition \ref{prop:1}
%\item when transfers are not redistributed: infeasibility of efficient allocation,  role for labor tax, and violation of Pigou principle \ref{prop:2}.
%\item redistribution through income tax scheme with progressive income tax restores efficient allocation \ref{prop:3}
%\end{enumerate}

\subsubsection{Government problem}\label{subsec:Rams}
The government is characterized by a Ramsey planner: it seeks to maximize utility of the representative household but can only revert to tax instruments and transfers to implement the welfare-maximizing allocation. The behavior of firms and households constrain the government's optimization problem. 
The Ramsey problem is defined as
\begin{align}
\underset{s, H}{\max}\ & U(C,H; F)\\ s.t\ \ & (1)\  C=Y-Gov\\ & (2) \ \text{behavior of firms and households}.
\end{align}
The first order conditions differ from the social planner's ones through the derivatives on government revenues, $Gov$:
\begin{align}
wrt.\ s:\hspace{4mm} & U_C\left(\frac{\partial Y}{\partial F}\frac{\partial F}{\partial s}+\frac{\partial Y}{\partial G}\frac{\partial G}{\partial s}-\frac{\partial Gov}{ \partial s}\right)=-U_F\frac{\partial F}{\partial s}, \label{eq:sbs}
\\
wrt.\ H:\hspace{4mm} & U_C\left(\frac{\partial Y}{\partial H}-\frac{\partial Gov}{\partial H}\right)+U_F\frac{\partial F}{\partial H}=-U_H\label{eq:sbh}. 
\end{align}

%-- paragraph to show that with Gov=0 and lump-sum transfers, the efficient allocation is implemented
When environmental tax revenues are fully redistributed lump-sum, i.e. $Gov=0$, $T_\iota=0$, then an environmental tax equal to the marginal social cost of dirty production\footnote{\ I define and derive the social cost of dirty production in appendix section \ref{app:derivations}.} suffices to implement the efficient allocation. An observation is known as the \textit{Pigou principle} in the literature. 
To see this, note that equation \ref{eq:sbs} ensures that the social planner's first order condition, equation \ref{eq:fbs}, is satisfied. 
Rewriting equation \ref{eq:fbs} reveals that the Pigou principle is satisfied: %\footnote{\ I derive the social cost of pollution as the price the representative household is willing to pay for a marginal reduction in dirty production. The derivation is exponded in appendix section \ref{sec:mod_an}. 
%	To be precise, social cost of pollution refers to the marginal cost evaluated at the resulting equilibrium allocation.}: The Pigou principle. 
\begin{align}
\underbrace{\frac{-U_F}{U_C\frac{\partial Y}{\partial F}}}_{\text{marginal social cost of dirty production}}=1+\frac{\frac{\partial Y}{\partial G}\frac{\partial G}{\partial s}}{\frac{\partial Y}{\partial F}\frac{\partial F}{\partial s}}=\tau^*_f.
\end{align}
Where the second equality follows from substituting intermediate firms' profit maximization conditions from equations \ref{eq:profmax}. 

I show in appendix section \ref{app:incometax0} that setting the environmental tax to the social cost of dirty production, implies that the second first order condition of the Ramsey planner is satisfied without use of the income tax instrument. %at $\tau_\iota=0$ when all environmental tax revenues are redistributed lump-sum: $T_{ls}=\tau_{f}p_FF$ (and $Gov=0$ and $T_\iota=0$).

In this paragraph, I briefly discuss the mechanism of the dirt tax.
As discussed previously, absent an externality of production it is efficient to balance marginal products of labor across sectors. However, when there is an externality, the social planner lowers the dirty share of labor which results in a higher marginal product of labor in the dirty sector. To sustain this gap between marginal products in the efficient allocation, the government has to introduce a dirt tax so that market forces do not direct labor towards the sector with the higher marginal product. In other words, the dirt tax is set so that wage rates equalize despite heterogeneous marginal products of labor. As a result of this intervention, the equilibrium wage rate is below the marginal product of labor in the dirty sector.\footnote{\ I formally discuss this statement in appendix section \ref{app:wageMPL}. The proof is insightful on the type of inefficiency resulting from environmental taxation. } These are efficiency costs associated with a use of the environmental tax. They are the source of the competition between environmental good provision and raising government funds or equity alluded to in the literature \citep{LansBovenberg1994EnvironmentalTaxation}.  

\subsubsection{Optimal environmental policy without lump-sum transfers}

Even though this distortion in the wage rate affects the supply of labor potentially downwards, it alone is insufficient to reduce hours to their efficient level when lump-sum transfers are lacking. %This section highlights the importance of lump-sum transfers for efficiency of the Pigou principle. 
%Lump-sum transfers serve to reduce hours worked through an income effect.
This is the first main result summarized in proposition \ref{prop:1} below. 

\begin{prop}\label{prop:1}\textbf{Importance of lump-sum redistribution of environmental tax revenues}
Absent lump-sum transfers and when the  wage rate is non-increasing in equilibrium hours, 
implementing the efficient share of dirty labor, $s^*=s_{FB}$, via an environmental tax results in inefficiently high hours worked. Lump-sum transfers would serve as a means to lower hours worked via an income channel. %The efficient allocation is infeasible.=> this statement would need to look at the optimal policy, this here is not a statement on the optimal policy
\end{prop}

% The logic is as follows:
%$s^*=s_{FB}$ when tauf implements the efficient allocation. 
% yet, implementing s-efficient without lump-sum transfers results in inefficiently high hours.


The proof of proposition \ref{prop:1}, depicted in appendix section \ref{app:derivations}, is informative on the mechanism: The conclusion that $H^*>H_{FB}$ follows from consumption in the competitive equilibrium being lower than in the first-best allocation. Lump-sum transfers, thus, imply lower hours worked in the competitive equilibrium through an income effect.

% violation Pigou principle
Choosing the environmental tax to implement the efficient share of dirty production while hours are inefficiently high, most likely  violates the Pigou principle; the environmental tax does not equal the social cost of pollution. One reason is that a higher labor supply increases dirty production above the efficient level; the social cost of pollution increase when the disutility of pollution is convex. Another reason is that household consumption deviates from the efficient level of consumption, if it is below, then the marginal utility of consumption increases diminishing the willingness to pay for a reduction in the externality. 

% discussion: importance to reduce hours worked in context of exogenous emssion limit
When labor supply is endogenous, lump-sum transfers gain in importance for the optimal allocation. When labor supply is fixed, the non-redistribution of environmental tax revenues results in inefficiently low consumption; when labor supply is elastic, however, the lower consumption results in too high hours worked. This is especially important as it aggravates the externality by increasing economic production. When there is an absolute limit on the externality - as is the case for greenhouse-gas emissions today- the scale effect could make a stricter environmental tax necessary. %\textit{could be} optimal absent means to reduce hours worked. 

% transition to proposition 2: optimal policy
Proposition \ref{prop:1} rationalizes the use of distortionary income taxes when lump-sum transfers are not available as a tool to lower the supply of labor. 
I will show in the next section that, indeed, the optimal environmental policy consists of both a progressive income tax scheme and an environmental tax when lump-sum transfers are not available. However, the efficient allocation is infeasible as private consumption is inefficiently low when environmental tax revenues are consumed by the government. Proposition \ref{prop:2} highlights these result.

\begin{prop}\label{prop:2}\textbf{Optimal environmental policy without redistribution of environmental tax revenues}
When environmental tax revenues are not redistributed but instead consumed by the government, i.e., $Gov=\tau_fp_FF$ and $T_{ls}=T_\iota=0$, then (i) the Pigou principle is violated, and (ii) a motive for labor taxation arises: The optimal income tax scheme is progressive. % if the aggregate production function features decreasing or constant returns to scale. 
The efficient allocation is infeasible.  
\end{prop}

\paragraph{Optimal environmental tax}
One can show that under the optimal policy the environmental tax equals
	\begin{align}
\tau_{f}^*= SCC+\frac{\partial Gov}{\partial s}\frac{1}{\frac{\partial Y}{\partial F}\frac{\partial F}{\partial s}}.
\end{align}
\begin{comment}
content...

This can be further simplified:
\begin{align}
\tau_f^* = 1-\frac{SCC}{\frp{w}{s}}w. 
\end{align}
A condition for $\tau_f$ to exceed the social cost of the externality reads
\begin{align}
SCC<\frac{1}{1+\frac{w}{\frp{w}{s}}}
\end{align}
\end{comment}
Hence, if government revenues increase with the share of dirty labor, then the optimal environmental tax exceeds the social cost of pollution in equilibrium. 	

With the environmental tax the government intents to reduce the share of dirty labor in equilibrium to lower the externality. 
When environmental tax revenues are consumed by the government and reduce private consumption, there is another mechanism of the environmental tax which the government takes into account. In general, it is welfare increasing to raise private consumption, or, equivalently, to lower government consumption. Since a higher environmental tax implies a lower dirty labor share, a positive relation of dirty labor and government consumption adds to the welfare enhancing effect of the environmental tax. If, in contrast, a lower dirty labor share raises government revenues, the environmental tax has an additional negative effect on social welfare: the Ramsey planner chooses a lower environmental tax. 

\tr{What mechanisms make government revenues rise with s, which reduce it?}

\paragraph{Optimal labor income tax}
The optimal labor income tax progressivity parameter is given by 
\begin{align}\label{eq:nls_taulopt}
\tau_\iota^*=\frac{1}{w}\frp{w}{s}_{F=\bar{F}}.
\end{align}
%\tr{Checked!!}
In words, the optimal income tax progressivity parameter equals the semi-elasticity of the wage rate in response to a decrease in the green labor share keeping dirty production unchanged. 
Since the environmental tax serves to sustain a gap between marginal products of labor, thereby  reducing the wage rate, a lower labor share in clean production increases the wage rate. As a result, the labor income tax scheme is progressive: $\tau_\iota>0$. For a proof and the derivation of the optimal income tax progressivity see appendix section \ref{app:subsub_nltaul}.

Intuitively, the optimal labor income tax scheme is progressive to lower work effort closer to the efficient level. As argued in proposition \ref{prop:1}, hours worked are inefficiently high absent lump-sum transfers, when the wage rate is a decreasing function of equilibrium hours. \tr{One can show that this condition is equivalent to the one implying optimality of progressive income taxes.  \textbf{to be done}}
% In the same section of the appendix, I prove that $\tau_\iota^*>0$ if clean production and aggregate production feature constant or decreasing returns to scale and at least one produces with decreasing returns to scale. An assumption satisfied under Cobb-Douglas or constant elasticity of substitution production functions when goods are substitutes. When goods are complements, then...  
 
% \tr{Only direct effect, not a general equilibrium result}
% \begin{corollary}
% Absent income taxes, hours worked are inefficiently high when production features decreasing or constant returns to scale. 
% \end{corollary}
% As a corollary of proposition \ref{prop:2}, it follows that absent income taxes, hours worked are inefficiently high. This follows directly from the houeshold's first order condition:  a positive value of $\tau_\iota$ ireduces the right-hand side. Since the marginal utility is decreasing in hours the left-hand side is a positive function of labor supply. Hence, as the right-hand side decreases, hours diminish.   
% 
 


 

%\begin{proof}\textit{With an environmental tax alone, hours worked are inefficiently high } \tr{waiting; to be done}
%\end{proof}
% \tr{Possible interpretation of regressive income taxes: –  then it is more important to increase consumption! }
 

\subsubsection{Optimal policy with combined environmental and fiscal policy}

In this subsection, I propose a policy set-up which allows to establish the efficient allocation even absent lump-sum transfers. In this setting, the government redistributes environmental tax revenues through the income tax scheme:  
\begin{align}
Gov= wh-\lambda (wH)^{1-\tau_\iota}+\tau_f p_FF.
\end{align}
The budget is balanced and $Gov = 0$ which determines $\lambda=\frac{w_h + \tau_f p_F F}{(wh)^{1-\tau_{\iota}}}$. 
Under this policy, the Ramsey planner can replicate the efficient allocation. 
The efficiency result is summarized in proposition \ref{prop:3}

\begin{prop}\label{prop:3}
	If lump-sum transfers are not available, the government can implement the efficient allocation by  transferring environmental tax revenues through the income tax scheme. The optimal tax scheme is progressive and the optimal environmental tax equals the social cost of pollution. Recomposing and reductive policies are complements in the optimal environmental policy.
\end{prop}

	That the optimal environmental tax satisfies the Pigou principle follows straight from the Ramsey planner's first order conditions. Since $Gov=0$ when environmental tax revenues are redistributed through the income tax scheme, the Ramsey planner's first order condition with respect to the dirty labor share ensures that the optimal environmental tax equals the social cost of pollution; compare equation \ref{eq:pigou}. 
	
	The proofs that the optimal income tax scheme is progressive and that the optimal allocation is efficient is sketched in appendix section \ref{app:derivations}. The proof also derives the optimal income tax as
\begin{align}
\tau^*_\iota=1-\frac{wH}{Y}=\frac{\tau_f^*p_FF}{Y}.
\end{align}
This simple relation reveals that the optimal income tax progressivity is positively related to environmental tax revenues.  The higher the optimal environmental tax, which equals the social cost of pollution in this setting, the higher the optimal income tax progressivity. Hence, when environmental tax revenues are on the upward sloping part of the Laffer curve, it holds that  the more the Ramsey planner recomposes production, the more intense the reduction policy has to be. Reductive and recomposing policies complement each other. 
 
\begin{comment}
\begin{prop}
Effect of using progressive income scheme on inequality (maybe as opposed to lump-sum transfers)
\end{prop}

content...
\end{comment}

\subsubsection{Discussion in relation to the literature}
These findings relate to the literature on a double dividend of environmental taxation discussed and partly rejected in the seminal paper by \cite{LansBovenberg1994EnvironmentalTaxation}.\footnote{ \ The double dividend of environmental policies refers to the idea that the revenues of environmental taxation can serve to improve on other policy targets such as equity or lowering distortionary fiscal policies. \cite{LansBovenberg1994EnvironmentalTaxation} argue that there is no double dividend because environmental taxes exert efficiency costs which outweigh the gains from lower distortionary income taxes. } The authors, inter alia, argue that recycling environmental tax revenues to lower distortionary fiscal policies has an advantage above recycling environmental tax revenues as lump-sum transfer. The latter would reduce labor supply even more thereby further narrowing the tax base of income taxes.  This result is referred to as the \textit{weak double dividend} hypothesis. The present paper demonstrates that there exists a lower bound up to where a reduction of labor reducing policies is beneficial. 
%Nevertheless, the \textit{weak double dividend} result states that the  use of environmental tax revenues to lower fiscal policies is advantageous about recycling revenues as lump-sum transfers, because lump-sum transfers would lower the tax base of the income tax even more.
%To implement the efficient allocation, there is no choice of how to use environmental tax revenues and that they are to be perceived as a means to lower hours worked absent other motives for government intervention. To reconcile these two findings, think of the result presented herein as a lower bound on the optimality to diminish the reduction in hours worked through cutting distortionary taxes. 
Some reduction in hours worked is efficient.  
Although this trade-off between the environmental advantage of lower work effort versus a lower income tax base has implicitly been studied in the double-dividend literature, the efficiency of lump-sum transfers or progressive income taxes has gone unnoticed. 
% Absent other motives of government intervention, such as an exogenous funding condition or inequality, lump-sum transfers should 

%\tr{Caution: in double dividend literature there is a second motive for government intervention... then there are gains, but only up to a certain point}

This section's findings also pose a warning to the literature discussing how to use environmental tax revenues such as \cite{Fried2018TheGenerations}. My results stress that there is no free choice in how to use environmental tax revenues but to redistribute if the government wants to implement the efficient allocation. 