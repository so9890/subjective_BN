\clearpage
\appendix
\section{Appendix}
\subsection{Why output (growth) reduction might be optimal}
It is a vibrant debate whether technological process will result in a production technology that is perfectly clean in that it does not exert any environmental externality. 
\begin{itemize}
	%\item \underline{Extensions to technology in \cite{Acemoglu2012TheChange} }
	%\begin{itemize}
	\item \underline{externality of ``clean'' sector} \citep[see also][]{Dasgupta2021, Brock2005ChapterEmpirics}
	\begin{itemize}
		\item[-] renewable/ non-fossil fuels \ar externalities in production process are present e.g. production of solar panels uses toxic inputs \citep{Yue2014DomesticAnalysis}; non-fossil fuel nitrogen generation (e.g., biomass burning to clear land) important ($\approx$ 50\%) \citep{Song2021ImportantEmissions}; low but chronical levels of nitrogen cause species extinctions \citep{Clark2008LossGrasslands}
		\item[-] waste (after use) \ar depends on recycling technology %\ar recycling system for solar panels not profitable enough today
		%	\item[-] substitutability of nature in production (input sources eg. fossil vs. non-fossil fuels)
		%\end{itemize}
		%\item Irreversibilities already before thresholds are hit (e.g. species extinction)
		
	\end{itemize}
	%\item greenhouse gases: Carbon dioxide $CO_2$ (vast majority), Nitrous oxide $N_2O$, methane $CH_4$
	%\item stock of nature globally determined
	\item \underline{parallel positive trend in demand} (population growth, rebound effect) that outperforms increase in clean technology growth \small{(no long-run issue if perfectly clean technology exists)}
	\item \normalsize{\underline{objective function}:} \cite{Arrow2004AreMuch}(Journal of Economic Perspectives) \ar using a sustainability measure they provide evidence that consumption is too high (= not leaving enough natural resources for future generations)
	\item \underline{risk, ambiguity}
	\item if have to meet climate target in short run, might need to lower production to do so; or it might be better in terms of inequality?
\end{itemize}

\subsection{Greenhouse-gas emissions and the Paris Agreement}

Two alternatives exist to specifiy the relation between the environment and production: (i) a broad approach considering natures use as a sink and as a resource, and all relevant pollutants. 
In order to determine \textit{relevant}, I refer to the planetary boundaries discussed in \cite{Rockstrom2009AHumanity}. (ii) a more specific approach that focuses on greenhouse gas emissions in particular which allows to draw on emission goals specified by country. Paris agreement goal: `'\textit{Climate neutral world by the mid-century}'' (source \url{https://unfccc.int/process-and-meetings/the-paris-agreement/the-paris-agreement}). In 2020 countries had to submit plansfor a \textit{long-term low ghg emissions} (LT-LEDS) where long term means mid-century (I assume). In the EU member states have submitted \textit{integrated national-energy and climate plans} (NECPS) (source \url{https://ec.europa.eu/info/energy-climate-change-environment/implementation-eu-countries/energy-and-climate-governance-and-reporting/national-long-term-strategies_en}). According to this source, emissions occur in the following fields: 
\textit{emission reductions and enhancements of removals in individual sectors, including \textbf{electricity, industry, transport, the heating and cooling and buildings sector (residential and tertiary), agriculture, waste and land use, land-use change and forestry (LULUCF)}}; the website also contains documents on country specific plans and actions

\subsection{Modelling choice: external emission target}\label{app:emission_climate_targets}
There is a multitude of uncertainties shaping the relation of production, on the one hand, and nature and climate warming, on the other hand. These uncertainties relate to (i) the technological possibilities to reduce emissions in the future and (ii) the relation of emissions and the climate. 

In the Paris Agreement clear political goals have been formulated in 2015. Under this treaty, states have agreed on a legally binding maximum increase in temperature to well below 2°C, preferably 1.5° over pre-industrial levels, and the global community seeks to be climate-neutral in 2050  (compare:\\ \url{https://unfccc.int/process-and-meetings/the-paris-agreement/the-paris-agreement}). 

\paragraph{Uncertainty 1): Emissions $\rightarrow$ temperature}
Carbon dioxide has been the focus of the literature integrating climate change and economic models \citep[such as,][]{Golosov2014OptimalEquilibrium,Barrage2019OptimalPolicy}. 
The (geo-)physical mechanisms which determine the interrelation between carbon emissions and temperature changes are highly uncertain and complex. For example, (1) there is no good understanding of the relation of CO2 and the climate as the temperature rises to certain limits, (2)  feedback of the Earth system, such as permafrost thawing, has to be taken into account, as well as (3) interactions of carbon with non-CO2 emissions, (\citep[][p.96, 2nd paragraph]{Rogelj2018MitigationDevelopment.}).  Uncertainty also surrounds the regeneration rate of the environment \citep{Acemoglu2012TheChange} and irreversibilities (might CITE Hassler handbook chapter here). In a quantitative study on optimal environmental policies, hence, a lot of assumptions and simplifications have to be made. 

In chapter 2 of the
\textit{IPCC Special Report} \citep{Rogelj2018MitigationDevelopment.}, scientists quantify emission pathways to meet the 1.5°C goal of the Paris Agreement by carefully taking uncertainties and the complex geophysical processes into account. I use these limits on emissions as constraints to the government's objective function. This approach is clearly policy relevant, while at the same time reduces the need to make (geophysical) assumptions. Furthermore, it allows me to take other important non-CO2 emissions into account, too. (\textit{Look at the discussion of integrated assessment models in \cite{Hassler2016EnvironmentalMacroeconomics} for the advantages of integrating a simplified carbon cycle into macro models (\ar dynamics) })
%\tr{\ar In a nutshell, I take from these reports the emission reduction pathways. I do have to make assumptions on the possibilities of technological innovations. No carbon cycle needed but less assumptions have to be made.}

\paragraph{Uncertainty 2: Technological progress $\rightarrow$ emissions}
An important modelling uncertainty remains: what degree of emission reduction can be achieved by technological progress in the specified time frame? I use different specifications of technological possibilities: (i) a scenario where technological progress is sufficient to reduce emissions to zero until 2050 at current consumption levels per capita, (ii) and one where innovation steps are insufficient.
Also look at different modelling approaches to technological change: (i) sector-specific innovations, (ii) porgress on the substitutability of clean and dirty input goods.

\paragraph{Uncertainty 2: regeneration rate of nature}

\paragraph{Uncertainty 2: Substitutability of natural capital in welfare}
Related to technological possibilities is the substitutability of clean and dirty production. 

A key reference is \cite{Cohen2019AnnualSubstitutable}. They argue for a limit of substitutability of \textit{natural capital} (defined as the stock of renewable and nonrenewable resources including minerals, soils, plants, animals, water, air, and energy). The role of natural capital for welfare: resources for production, absorption of waste, basic-life support, direct conrtibution to human welfare \ar No perfect substitutability: Humans cannot live without natural resources. 

BUT: my model so far is about greenhouse-gas emissions only.

Definition \textit{sustainability} common in economics: maintaining a non-decreasing level of welfare across generations. (Sonja: this shouls include provision of natural services: human-friendly climate; how to know the utility function of future generations? Eg if habits are relevant, than they might be happy with less consumption). As regards renewables, it must be ensured that renewability is maintained. It is about the substitutability of natural inputs to the welfare function. \textit{Sonja: (1) How should we know how substitutable biodiversity is for humanity if it is about pleasure derived from living in a diverse world. (2) If it is about other factors of biodiversity, such as, maintaining more basal services for human live, such as safety, there might be less disagreement; more certainty on the importance also for future generations; a utility approach but based on objectively defined values: basic needs.}

\cite{Cohen2019AnnualSubstitutable} make the following distinction: \textit{within-input substitution}: =resources used in welfare production: within-input substitution allows to reduce environmental \textbf{impacts} if the same type of input can be obtained from different sources (e.g.: energy: from emission-low sources instead of emission high sources); \textit{between-input substitution}: = from energy to manufactured capital; includes more efficient energy use, recycling, reforestation.

\textit{Brundtland Comission} definition of sustainable development: \textit{development that meets the \textbf{needs} of the present without compromising the ability of future generations to meet their own needs}.
If various forms of capital are substitutable (i.e. can use manufactured capital, human capital to replace nature) then production only depends on the total capital stock. Then, economic growth is said to be \textit{weakly sustainable} when the total capital stock is non-decreasing; i..e aggregate savings rate is above depreciation rate on all forms of capital. 

\textit{Strong sustainability view}:  a minimum of natural capital must be sustained as it provides non-substitutable inputs to utility. Then, long-run growth must be able to maintain a natural capital stock.  
\ar Define $Y$ broadly to incorporate other necessary consumption goods: stable climate, breathable air, food, water; but these are partially not traded in markets, rather public goods. Then, better to model as public goods; \textbf{Or as another input in final good production}.

\begin{align*}
Y= \left[\left(\underbrace{\left(Y_c^{\frac{\varepsilon_c-1}{\varepsilon_c}}+Y_d^{\frac{\varepsilon_c-1}{\varepsilon_c}}\right)^{\frac{\varepsilon_c}{\varepsilon_c-1}}}_{\text{consumption  good}}\right)^{\frac{\varepsilon_o-1}{\varepsilon_o}}+(\underbrace{N-\bar{N}}_{\text{natural capital}})^{\frac{\varepsilon_o-1}{\varepsilon_o}}\right]^{\frac{\varepsilon_o}{\varepsilon_o-1}}
\end{align*} 
where $\varepsilon_o$  governs the substitutability of natural capital and consumption in the final good; it translates into the substitutability discussed in \cite{Cohen2019AnnualSubstitutable}. For the basic needs level ins terms of natural capital holds: $\bar{N}>0$ so that nature is a necessary good. The variable $N$ determines the consumption of natural capital, such as breathing air, stable climate.
Model technological growth on substitutability between natural capital (including nature as a waste), and other production inputs.  

 
%%%%%%%%%%%%%%%%%%%%%%%%%%%%%%%%%%%%%%%
\subsection{Greenhouse gas emissions Data}
To calibrate the relation of economic production and emissions, I use data from the EPA \url{https://www.epa.gov/newsreleases/latest-inventory-us-greenhouse-gas-emissions-and-sinks-shows-long-term-reductions-0}.
In 2019, the US greenhouse gas emissions amounted to 6,558 million metric tons of carbon dioxide equivalents; that is, 6.558Gt. 

Global greenhouse gas emissions amounted to 34.2 Gt in C02 equivalents in 2019. There was a decline in 2020 (presumably due to the pandemic, and a rebound in 2021 by 5\%)Found here: \url{https://www.iea.org/reports/greenhouse-gas-emissions-from-energy-overview/global-ghg-emissions}; the Global Energy review of the iea is to  be found here: \url{https://www.iea.org/reports/global-energy-review-2021/co2-emissions}.

Natural sinks are, for example, forests, vegetation, soils. 
The epa report (\url{https://www.epa.gov/ghgemissions/inventory-us-greenhouse-gas-emissions-and-sinks-1990-2019}) includes information on sinks. 

Net emissions after taking sinks into account are estimated by the epa to 5,769.1 in 
\paragraph{Translation metric ton to gigatonne}
1.000.000.000 metric tons are 1 gigatonne 

\paragraph{Demand and Production approach}
The OECD, \url{https://www.oecd.org/sti/ind/carbondioxideemissionsembodiedininternationaltrade.htm} differentaites between a demand and a production-side approach to determine emissions on country level! I am now using the production approach. 

For now, I assume that the global emission target is given in gros emissions. I match the US gros emission target so that contribution in 2019 equals contribution to the global gros reduction. 

\section{Model}
\subsection{Equilibrium conditions}

\begin{align*}
\text{\textbf{Household solved:}} \hspace{50mm}& \\
\text{FOCs labour supply}\hspace{4mm}&  %\log(H_t)=\frac{1}{1+\sigma}\log(1-\tau_{lt})\\
H_t=(1-\tau_{lt})^\frac{1}{1+\sigma}\\
\ \hspace{4mm} & %\log(w_{ht})=\log(w_{lt})+\log(\zeta)\\
w_{ht}=\zeta w_{lt}\\
\text{Budget}\hspace{4mm}&  %\log(c_t)= \log(\lambda_t)+ (1-\tau_{lt})\left[\frac{1}{1+\sigma}\log(1-\tau_{lt})+\log(w_{lt})\right]\\
c_t= \lambda_t (H_tw_{lt})^{(1-\tau_{lt})}\\
\text{definition}\  H_t\hspace{4mm} & %\log(H_t)=\log(h_{lt}+\zeta h_{ht})\\
H_t=\zeta h_{ht}+h_{lt}
\\
\text{\textbf{General Household Problem:}} \hspace{50mm}& \\
\text{FOC consumption}\hspace{4mm}& Mu_{ct}=p_t\mu_t\\
\text{FOC low skill}\hspace{4mm} & -Mu_{h_lt}=\mu_t \frac{\partial I_t}{\partial h_{lt}}\\
\text{FOC high skill}\hspace{4mm} & -Mu_{h_ht}=\mu_t \frac{\partial I_t}{\partial h_{ht}}\\
\text{Budget}\hspace{4mm}& c_tp_t= I_t\\
\text{definition}\  H_t\hspace{4mm} & H_t=\zeta h_{ht}+h_{lt}\\
\text{\textbf{Labour sectors:}}\hspace{50mm}&\\
\text{Production clean labour input} \hspace{4mm}& L_{ct}=l_{hct}^{\theta_c}l_{lct}^{1-\theta_c}\\ 
\text{Production dirty labour input} \hspace{4mm}& L_{dt}=l_{hdt}^{\theta_d}l_{ldt}^{1-\theta_d}\\
%
\text{Demand high skill clean sector}\hspace{4mm}&l_{hct}= \left(\frac{p_{cLt}}{w_{ht}}\right)^{\frac{1}{1-\theta_c}}\theta_c^{\frac{1}{1-\theta_c}}l_{lct}\\
%
\text{Demand low skill clean sector } \hspace{4mm}&l_{lct}= \left(\frac{p_{cLt}}{w_{lt}}\right)^{\frac{1}{\theta_c}}(1-\theta_c)^{\frac{1}{\theta_c}}l_{hct}\\
%
\text{Demand high skill dirty sector} \hspace{4mm}&l_{hdt}= \left(\frac{p_{dLt}}{w_{ht}}\right)^{\frac{1}{1-\theta_d}}\theta_d^{\frac{1}{1-\theta_d}}l_{ldt}\\
%
\text{Demand low skill dirty sector } \hspace{4mm}&l_{ldt}= \left(\frac{p_{dLt}}{w_{lt}}\right)^{\frac{1}{\theta_d}}(1-\theta_d)^{\frac{1}{\theta_d}}l_{hdt}\\
\text{\textbf{Government}}\hspace{50mm}& \nonumber\\
\text{Budget}\hspace{4mm}& G_t=H_tw_{lt}-\lambda_t(H_t w_{lt})^{(1-\tau_{lt})}
\\
\text{\textbf{Technology:}}\hspace{50mm}&\\
\text{Clean sector}\hspace{4mm}& A_{ict+1}=(1+\upsilon_{ct})A_{ict}\\
\text{Dirty sector}\hspace{4mm}& A_{idt+1}=(1+\upsilon_{dt})A_{idt}\\
%\text{Progress bound}\hspace{4mm}& \upsilon_{ct}+\upsilon_{dt}=\Upsilon\\
\text{Definition average clean technology}\hspace{4mm}& A_{ct}=\int_0^1A_{ict}di\\
\text{Definition average dirty technology}\hspace{4mm}& A_{dt}=\int_0^1A_{idt}di
\end{align*}

\begin{align}
\text{\textbf{Production:}} \hspace{4mm}
\text{\textbf{Final Good Producer}}&\\
\text{Profit maximisation}\hspace{4mm} & Y_{nt}=\left(\frac{p_{ct}}{p_{dt}}\right)^\varepsilon Y_{ct}\\
\text{Production}\hspace{4mm} & Y_t=\left[Y_{ct}^{\frac{\varepsilon-1}{\varepsilon}}+Y_{dt}^{\frac{\varepsilon-1}{\varepsilon}}\right]^{\frac{\varepsilon}{\varepsilon-1}}\\
\text{Price}\hspace{4mm}& p_t:=\left[p_{ct}^{1-\varepsilon}+p_{dt}^{1-\varepsilon}\right]^{\frac{1}{1-\varepsilon}}\\
\text{\textbf{Clean Sector}}\\
\text{Production}\hspace{4mm}& Y_{ct}=L^{1-\alpha}_{ct}\int_{0}^{1}A^{1-\alpha}_{ict}x_{ict}^{\alpha}di=  \left(\alpha\frac{p_{ct}}{\psi}\right)^{\frac{\alpha}{1-\alpha}}A_{ct} L_{ct} \label{eqbm:outputc}
\\ & =x_{ct}^{\alpha}\left(A_{ct}L_{ct}\right)^{1-\alpha} \\ 
\text{labour demand}\hspace{4mm} & p_{cLt} =
(1-\alpha)\left(\frac{\alpha}{\psi}\right)^\frac{\alpha}{1- \alpha}p_{ct}^\frac{1}{1-\alpha}A_{ct} \label{eqbm:labc} \\
\text{machine demand}\hspace{4mm} & x_{ict} = \left(\alpha\frac{ p_{ct}}{p_{ict}}\right)^\frac{1}{1-\alpha}A_{ict}L_{ct}\\
& x_{ct}:=\int_{0}^{1}x_{ict} di= \left(\alpha\frac{p_{ct}}{\psi}\right)^\frac{1}{1-\alpha}A_{ct}L_{ct}\\
%
\text{Supply machines (price)}\hspace{4mm}& p_{ict}=\psi \\
%
\text{\textbf{Dirty Sector}}\\
\text{Production}\hspace{4mm} & Y_{dt}=L^{1-\alpha}_{dt}\int_{0}^{1}A^{1-\alpha}_{idt}x_{idt}^{\alpha}di=  \left(\alpha\frac{p_{dt}}{\psi}\right)^{\frac{\alpha}{1-\alpha}}A_{dt} L_{dt}\label{eqbm:outputd}\\ & =x_{dt}^{\alpha}\left(A_{dt}L_{dt}\right)^{1-\alpha} \\ 
\text{labour demand}\hspace{4mm} & p_{dLt} =
(1-\alpha)\left(\frac{\alpha}{\psi}\right)^\frac{\alpha}{1- \alpha}p_{dt}^\frac{1}{1-\alpha}A_{dt}\label{eqbm:labd}\\
\text{machine demand}\hspace{4mm} & x_{idt} = \left(\alpha\frac{ p_{dt}}{p_{idt}}\right)^\frac{1}{1-\alpha}A_{idt}L_{dt}\\
& x_{dt}:=\int_{0}^{1}x_{idt} di= \left(\alpha\frac{p_{dt}}{\psi}\right)^\frac{1}{1-\alpha}A_{dt}L_{dt}\\
\text{Supply machines (price)}\hspace{4mm}& p_{idt}=\psi\\
\text{\textbf{Market clearing:}}\hspace{50mm}& \nonumber\\
\text{Final Good}\hspace{4mm}& Y_{t}=c_t+\psi\left(\int_{0}^1x_{idt}di+\int_{0}^1x_{ict}di\right)+G_t%\psi \left(\int_0^1\left(\alpha\frac{p_{dt}}{\psi}\right)^\frac{1}{1-\alpha}A_{idt}L_{dt}+\int_0^1\left(\alpha\frac{p_{ct}}{\psi}\right)^\frac{1}{1-\alpha}A_{ict}L_{ct}\right)
\\
%& \ (\text{Numeraire}\ \  p_t=1)\\
\text{high skill}\hspace{4mm}& l_{hct}+l_{hdt}=h_{ht}\\
\text{low skill}\hspace{4mm}&l_{lct}+l_{ldt}=h_{lt}
\end{align}
\begin{comment}
\subsection{Equilibrium conditions: Simplified model with Lc=hh, ld=hl}

\begin{align*}
\text{\textbf{Household solved:}} \hspace{50mm}& \\
\text{FOCs labour supply}\hspace{4mm}&  %\log(H_t)=\frac{1}{1+\sigma}\log(1-\tau_{lt})\\
H_t=(1-\tau_{lt})^\frac{1}{1+\sigma}\\
\ \hspace{4mm} & %\log(w_{ht})=\log(w_{lt})+\log(\zeta)\\
w_{ht}=\zeta w_{lt}\\
\text{Budget}\hspace{4mm}&  %\log(c_t)= \log(\lambda_t)+ (1-\tau_{lt})\left[\frac{1}{1+\sigma}\log(1-\tau_{lt})+\log(w_{lt})\right]\\
c_t= \lambda_t (H_tw_{lt})^{(1-\tau_{lt})}\\
\text{definition}\  H_t\hspace{4mm} & %\log(H_t)=\log(h_{lt}+\zeta h_{ht})\\
H_t=\zeta h_{ht}+h_{lt}
\\
%\text{\textbf{Labour sectors:}}\hspace{50mm}&\\
%\text{Production clean labour input} \hspace{4mm}& L_{ct}=h_{ht}\\ 
%\text{Production dirty labour input} \hspace{4mm}& L_{dt}=h_{lt}\\
%
\text{\textbf{Government}}\hspace{50mm}& \nonumber\\
\text{Budget}\hspace{4mm}& G_t=H_tw_{lt}-\lambda_t(H_t w_{lt})^{(1-\tau_{lt})}
\\
\text{\textbf{Technology:}}\hspace{50mm}&\\
\text{Clean sector}\hspace{4mm}& A_{ict+1}=(1+\upsilon_{ct})A_{ict}\\
\text{Dirty sector}\hspace{4mm}& A_{idt+1}=(1+\upsilon_{dt})A_{idt}\\
%\text{Progress bound}\hspace{4mm}& \upsilon_{ct}+\upsilon_{dt}=\Upsilon\\
\text{Definition average clean technology}\hspace{4mm}& A_{ct}=\int_0^1A_{ict}di\\
\text{Definition average dirty technology}\hspace{4mm}& A_{dt}=\int_0^1A_{idt}di
\end{align*}

\begin{align*}
\text{\textbf{Production:}} \hspace{4mm}
\text{\textbf{Final Good Producer}}&\\
\text{Profit maximisation}\hspace{4mm} & Y_{nt}=\left(\frac{p_{ct}}{p_{dt}}\right)^\varepsilon Y_{ct}\\
\text{Production}\hspace{4mm} & Y_t=\left[Y_{ct}^{\frac{\varepsilon-1}{\varepsilon}}+Y_{dt}^{\frac{\varepsilon-1}{\varepsilon}}\right]^{\frac{\varepsilon}{\varepsilon-1}}\\
\text{Price}\hspace{4mm}& p_t:=\left[p_{ct}^{1-\varepsilon}+p_{dt}^{1-\varepsilon}\right]^{\frac{1}{1-\varepsilon}}\\
\text{\textbf{Clean Sector}}\\
\text{Production}\hspace{4mm}& Y_{ct}=L^{1-\alpha}_{ct}\int_{0}^{1}A^{1-\alpha}_{ict}x_{ict}^{\alpha}di=  \left(\alpha\frac{p_{ct}}{\psi}\right)^{\frac{\alpha}{1-\alpha}}A_{ct} L_{ct}
\\ & =x_{ct}^{\alpha}\left(A_{ct}L_{ct}\right)^{1-\alpha} \\ 
\text{labour demand}\hspace{4mm} & w_{ht} =
(1-\alpha)\left(\frac{\alpha}{\psi}\right)^\frac{\alpha}{1- \alpha}p_{ct}^\frac{1}{1-\alpha}A_{ct}\\
\text{machine demand}\hspace{4mm} & x_{ict} = \left(\alpha\frac{ p_{ct}}{p_{ict}}\right)^\frac{1}{1-\alpha}A_{ict}L_{ct}\\
& x_{ct}:=\int_{0}^{1}x_{ict} di= \left(\alpha\frac{p_{ct}}{\psi}\right)^\frac{1}{1-\alpha}A_{ct}L_{ct}\\
%
\text{Supply machines (price)}\hspace{4mm}& p_{ict}=\psi \\
%
\text{\textbf{Dirty Sector}}\\
\text{Production}\hspace{4mm} & Y_{dt}=L^{1-\alpha}_{dt}\int_{0}^{1}A^{1-\alpha}_{idt}x_{idt}^{\alpha}di=  \left(\alpha\frac{p_{dt}}{\psi}\right)^{\frac{\alpha}{1-\alpha}}A_{dt} L_{dt}\\ & =x_{dt}^{\alpha}\left(A_{dt}L_{dt}\right)^{1-\alpha} \\ 
\text{labour demand}\hspace{4mm} & w_{lt} =
(1-\alpha)\left(\frac{\alpha}{\psi}\right)^\frac{\alpha}{1- \alpha}p_{dt}^\frac{1}{1-\alpha}A_{dt}\\
\text{machine demand}\hspace{4mm} & x_{idt} = \left(\alpha\frac{ p_{dt}}{p_{idt}}\right)^\frac{1}{1-\alpha}A_{idt}L_{dt}\\
& x_{dt}:=\int_{0}^{1}x_{idt} di= \left(\alpha\frac{p_{dt}}{\psi}\right)^\frac{1}{1-\alpha}A_{dt}L_{dt}\\
\text{Supply machines (price)}\hspace{4mm}& p_{idt}=\psi\\
\text{\textbf{Market clearing:}}\hspace{50mm}& \nonumber\\
\text{Final Good}\hspace{4mm}& Y_{t}=c_t+\psi\left(\int_{0}^1x_{idt}di+\int_{0}^1x_{ict}di\right)+G_t%\psi \left(\int_0^1\left(\alpha\frac{p_{dt}}{\psi}\right)^\frac{1}{1-\alpha}A_{idt}L_{dt}+\int_0^1\left(\alpha\frac{p_{ct}}{\psi}\right)^\frac{1}{1-\alpha}A_{ict}L_{ct}\right)
\\
%& \ (\text{Numeraire}\ \  p_t=1)\\
\text{high skill}\hspace{4mm}& L_{ct}=h_{ht}\\
\text{low skill}\hspace{4mm}&L_{dt}=h_{lt}
\end{align*}
\end{comment}

\section{Solution of tractable model}\label{app:solu}
Define
\begin{align*}
	\tilde{\kappa}:=\ &\frac{(1-\theta_c)(1-\theta_d)\left[\left(\frac{A_c}{A_d}\right)^{(1-\alpha)(1-\varepsilon)}\zeta^{-(\theta_c-\theta_d)(1-\alpha)(1-\varepsilon)}\tilde{\chi}+1\right]}{(1-\theta_d)+(1-\theta_c)\left[\left(\frac{A_c}{A_d}\right)^{(1-\alpha)(1-\varepsilon)}\zeta^{-(\theta_c-\theta_d)(1-\alpha)(1-\varepsilon)}\tilde{\chi}\right]}\\
	\gamma_j:=\ & \left(\frac{\theta_j}{\zeta(1-\theta_j)}\right)^{\theta_j}\\
	z_j:=\ &\theta_j^{\theta_j}(1-\theta_j)^{1-\theta_j} \\
	\chi:=\ &% \frac{(1-\theta_d)(1-\theta_c)}{\theta_c(1-\theta_d)-\theta_d(1-\theta_c)}
	\frac{(1-\theta_d)(1-\theta_c)}{\theta_c-\theta_d}\\
	\tilde{\chi}: =\ &  (\theta_c^{\theta_c}\theta_d^{-\theta_d})^{(1-\alpha) (1-\varepsilon)}(1-\theta_c)^{-\theta_c-(1-\theta_c)(\alpha+\varepsilon(1-\alpha))}(1-\theta_d)^{\theta_d+(1-\theta_d)(\alpha+\varepsilon(1-\alpha))}
\end{align*}
From profit maximisation by labour input good producers follows that the price of the labour input good relative to the skill-specific wage rate is constant. Substituting demand for low skill in the clean sector into the demand for high skill yields

\begin{align*}
	w_{h}^{\frac{1}{1-\theta_c}}w_l^{\frac{1}{\theta_c}}= p_{cL}^\frac{1}{(1-\theta_c)\theta_c}\theta_c^\frac{1}{1-\theta_c}(1-\theta_c)^\frac{1}{\theta_c}.
\end{align*}
Multiplying the left-hand side with $(w_h/w_h)^\frac{1}{\theta_c}$ and
using the FOC governing skill supply $w_h/w_l=\zeta$, it holds that

\begin{align}\label{eq:constant}
%	& \zeta^\frac{-1}{\theta_c}w_h^\frac{1}{(1-\theta_c)\theta_c}= p_{cL}^\frac{1}{(1-\theta_c)\theta_c}\theta_c^\frac{1}{1-\theta_c}(1-\theta_c)^\frac{1}{\theta_c}\nonumber\\
%	\Leftrightarrow\ 
& \frac{p_{cL}}{w_h}= \frac{\zeta^{-(1-\theta_c)}}{z_c}.
\end{align}
%\noindent \tr{Note: this result does not rely on the claim that the labour input good is constant.}

Analogously to \ref{eq:constant}, it follows that
\begin{align}
	\frac{p_{cL}}{w_l}&=\frac{\zeta^{\theta_c}}{z_c}\label{eq:pcl_wl}\\
	\frac{p_{dL}}{w_l}&=\frac{\zeta^{\theta_d}}{z_d}%\ \Leftrightarrow\ w_l= p_{dL}\zeta^{-\theta_d}\theta_d^{\theta_d}(1-\theta_d)^{1-\theta_d}
	\label{eq:pdl_wl}\\
	\frac{p_{dL}}{w_h}&=\frac{\zeta^{-(1-\theta_d)}}{z_d}.
\end{align}
Therefore, the optimal skill input ratios in the labour good production are given by
\begin{align}\label{eq:inputr}
	\frac{l_{hc}}{l_{lc}}=\frac{\theta_c}{\zeta (1-\theta_c)} \hspace{2mm} \text{and}\hspace{3mm} \frac{l_{hd}}{l_{ld}}=\frac{\theta_d}{\zeta (1-\theta_d)}.
\end{align}
This is the common result that  factor shares 
% this refers to (wh lhc)/(wl llc)
are constant over time with a Cobb-Douglas production function. 
Imposing labour market clearing for both skills and optimal skill demand yields 
\begin{align}
	&l_{ld}=\chi\left(\frac{1}{1-\theta_c}h_l-H\right)\label{eq:lld}\\ %\frac{\theta_c}{1-\theta_c}\chi h_l-\chi \zeta h_h,\\
	& l_{lc}=\chi \left(H-\frac{1}{1-\theta_d}h_l\right)\label{eq:llc} %\\
%	with \ & \chi:= \frac{(1-\theta_d)(1-\theta_c)}{\theta_c(1-\theta_d)-\theta_d(1-\theta_c)}=\frac{(1-\theta_d)(1-\theta_c)}{\theta_c-\theta_d}.
	%& l_{hc}= \frac{\theta_c}{\zeta (1-\theta_c)}l_{lc}\\
	%& l_{hd}=\frac{\theta_d}{\zeta (1-\theta_d)}l_{ld}
\end{align}
Labour good supply follows from the labour input good's production function and optimal skill inputs, equations \ref{eq:inputr}, as
\begin{align}
	L_c&=\gamma_cl_{lc}\label{eq:lab_inputc} \\
	L_d&=\gamma_dl_{ld}.\label{eq:lab_inputd}
\end{align}
%\tr{Note that now policy can affect inflation/ relative prices through changes in labour supply---NOPE: cancels}
A relation of the relative price in equilibrium results from equating demand for the labour input goods, equations \ref{eqbm:labc} and \ref{eqbm:labd}, % (which relates the price for the labour input good and the price for the sector-specific final good), 
demand for low skill input by labour producers, equations \ref{eq:pcl_wl} and \ref{eq:pdl_wl}, and free movement of skills: 
\begin{align}\label{eq:price_ratio_labourinput}
	\frac{p_c}{p_d}= \left(\frac{A_d}{A_c}\frac{z_d}{z_c}\zeta^{\theta_c-\theta_d}\right)^{1-\alpha}& \text{(optimality labour input production)}
\end{align}
%where
%\begin{align*}
%	z_j=\theta_j^{\theta_j}(1-\theta_j)^{1-\theta_j}
%\end{align*}
Together with the definition of the aggregate price level and the choice of $Y$ as numeraire, equation \ref{eq:price_ratio_labourinput} determines sector-specific prices as a function of parameters and productivity:
%\begin{align*}
%	p_c= \left(1+\left(\frac{\gamma_c}{\gamma_d}\frac{A_c}{A_d}\frac{l_{lc}}{l_{ld}}\right)^{\frac{(1-\alpha)(1-\varepsilon)}{\alpha+\varepsilon(1-\alpha)}}\right)^{-\frac{1}{1-\varepsilon}}.
%\end{align*}
%Substituting equation \ref{eq:lldllc} gives the price of the clean good in equilibrium as
\begin{align}
	p_c%& = \frac{1}{\left(1+\left(\frac{A_c}{A_d}\right)^{(1-\alpha)(1-\varepsilon)}\left(\frac{z_c}{z_d}\right)^{(1-\alpha)(1-\varepsilon)}\zeta^{-(\theta_c-\theta_d)(1-\alpha)(1-\varepsilon)}\right)^{\frac{1}{1-\varepsilon}}}\\
	&= \left(\frac{\left(A_dz_d\zeta^{\theta_c}\right)^{(1-\alpha)(1-\varepsilon)}}{\left(A_dz_d\zeta^{\theta_c}\right)^{(1-\alpha)(1-\varepsilon)}+\left(A_cz_c\zeta^{\theta_d}\right)^{(1-\alpha)(1-\varepsilon)}}\right)^{\frac{1}{1-\varepsilon}}\label{eq:eq_pc}\\
%\end{align}
%%and using equation \ref{eq:price_ratio_labourinput} yields
%\begin{align}
	p_d%& =\frac{1}{\left(\left(\frac{A_d}{A_c}\right)^{(1-\alpha)(1-\varepsilon)}\left(\frac{z_d}{z_c}\right)^{(1-\alpha)(1-\varepsilon)}\zeta^{(\theta_c-\theta_d)(1-\alpha)(1-\varepsilon)}+1\right)^{\frac{1}{1-\varepsilon}}}\\
	&= \left(\frac{\left(A_cz_c\zeta^{\theta_d}\right)^{(1-\alpha)(1-\varepsilon)}}{\left(A_dz_d\zeta^{\theta_c}\right)^{(1-\alpha)(1-\varepsilon)}+\left(A_cz_c\zeta^{\theta_d}\right)^{(1-\alpha)(1-\varepsilon)}}\right)^{\frac{1}{1-\varepsilon}} \label{eq:eq_pd}
\end{align}

\paragraph{Skill allocation}
To solve for the equilibrium ratio of skill inputs in the clean and dirty sector, I substitute labour input, equations \ref{eq:lab_inputc} and \ref{eq:lab_inputd}, in the sector-specific production functions, \ref{eqbm:outputc} and \ref{eqbm:outputd}. Exploiting demand for sector goods, $Y_d=\left(\frac{p_c}{p_d}\right)^\varepsilon Y_c$, and the price ratio in equilibrium pinned down by equation \ref{eq:price_ratio_labourinput} yields
%\begin{align}\label{eq:price_ratio_output}

%\end{align}
%Substituting equation \ref{eq:price_ratio_output} into equation \ref{eq:price_ratio_labourinput} determines the equilibrium ratio of low-skill input in the dirty to the clean sector: 
\begin{align}
	\frac{p_c}{p_d} =&\left(\frac{\gamma_d}{\gamma_c}\frac{A_d}{A_c}\frac{l_{ld}}{l_{lc}}\right)^{\frac{1-\alpha}{\alpha+\varepsilon(1-\alpha)}}\\ %& \text{(Demand for sector-specific goods)}\\
\Leftrightarrow\ 	\frac{l_{ld}}{l_{lc}}=&%\left(\frac{A_c}{A_d}\right)^{(1-\alpha)(1-\varepsilon)}\frac{\gamma_c}{\gamma_d}\left(\frac{z_d}{z_c}\right)^{\alpha+\varepsilon(1-\alpha)}\zeta^{(\theta_c-\theta_d)(\alpha+\varepsilon(1-\alpha))}\nonumber\\	=&
	\left(\frac{A_c}{A_d}\right)^{(1-\alpha)(1-\varepsilon)}\left(\zeta^{\theta_c-\theta_d}\frac{\gamma_c}{\gamma_d}\frac{z_d}{z_c}\right)^{\alpha+\varepsilon(1-\alpha)}\label{eq:lldllc}%\\
	%	\text{where}&\\
	%	\tilde{\chi}= &\  (\theta_c^{\theta_c}\theta_d^{-\theta_d})^{(1-\alpha) (1-\varepsilon)}(1-\theta_c)^{-\theta_c-(1-\theta_c)(\alpha+\varepsilon(1-\alpha))}(1-\theta_d)^{\theta_d+(1-\theta_d)(\alpha+\varepsilon(1-\alpha))}\nonumber
\end{align}

\paragraph{Skill supply}
Using equations \ref{eq:lld}, \ref{eq:llc}, and \ref{eq:lldllc} one can solve for $h_l$ as a function of total skill supply in equilibrium
\begin{align}
	h_l= \underbrace{\frac{(1-\theta_c)(1-\theta_d)\left[\left(\frac{A_c}{A_d}\right)^{(1-\alpha)(1-\varepsilon)}\zeta^{-(\theta_c-\theta_d)(1-\alpha)(1-\varepsilon)}\tilde{\chi}+1\right]}{(1-\theta_d)+(1-\theta_c)\left[\left(\frac{A_c}{A_d}\right)^{(1-\alpha)(1-\varepsilon)}\zeta^{-(\theta_c-\theta_d)(1-\alpha)(1-\varepsilon)}\tilde{\chi}\right]}}_{:=\tilde{\kappa}}H
\end{align}
Now, one can solve for labour input and sector-specific output as a function of tax progessivity  in equilibrium. 
$L_c$ and $L_d$ are
\begin{align}
	L_c&= \gamma_c \chi \left(1-\frac{\tilde{\kappa}}{1-\theta_d}\right)H\\
	L_d&= \gamma_d \chi \left(\frac{\tilde{\kappa}}{1-\theta_c}-1\right)H=\zeta^{-\theta_d}z_dp_d^{1-\varepsilon}H
\end{align}
This solves the model, since, in equilibrium,  $H$ is a function of parameters and policy variables only. 
Replacing dirty labour input and machines in dirty production leads to the expression for dirty output growth used in the text. 
\begin{comment}
\paragraph{Summary of equilibrium equations}
\begin{align*}
H=\ & (1-\tau_l)^{\frac{1}{1+\sigma}}\\
h_l=\ & \tilde{\kappa}H\\
L_c=\ & \gamma_c \chi \left(1-\frac{\tilde{\kappa}}{1-\theta_d}\right)H
\\
L_d=\ & \gamma_d \chi \left(\frac{\tilde{\kappa}}{1-\theta_c}-1\right)H%= \zeta^{-\theta_d}z_d\frac{\left(A_cz_c\zeta^{\theta_d}\right)^{(1-\alpha)(1-\varepsilon)}}{\left(A_dz_d\zeta^{\theta_c}\right)^{(1-\alpha)(1-\varepsilon)}+\left(A_cz_c\zeta^{\theta_d}\right)^{(1-\alpha)(1-\varepsilon)}}H
=\zeta^{-\theta_d}z_dp_d^{1-\varepsilon}H\\
p_d=\ &\left(\frac{\left(A_cz_c\zeta^{\theta_d}\right)^{(1-\alpha)(1-\varepsilon)}}{\left(A_dz_d\zeta^{\theta_c}\right)^{(1-\alpha)(1-\varepsilon)}+\left(A_cz_c\zeta^{\theta_d}\right)^{(1-\alpha)(1-\varepsilon)}}\right)^{\frac{1}{1-\varepsilon}} \\
p_c=\ & \left(\frac{\left(A_dz_d\zeta^{\theta_c}\right)^{(1-\alpha)(1-\varepsilon)}}{\left(A_dz_d\zeta^{\theta_c}\right)^{(1-\alpha)(1-\varepsilon)}+\left(A_cz_c\zeta^{\theta_d}\right)^{(1-\alpha)(1-\varepsilon)}}\right)^{\frac{1}{1-\varepsilon}}
\end{align*}

content...
\end{comment}
\section{Balanced Growth Path}

The model features structural transformation stemming from price effects (\cite{Ngai2007StructuralGrowth}, Baumol (1967)), since heterogeneous growth rates result in relative price changes over time. %A shown by \cite{Ngai2007StructuralGrowth}, the model features a balanced-growth path with certain parameter values: 
For certain parameter values the model exhibits a generalised balanced growth path\footnote{\ 
In contrast to a balanced growth path, which is commonly defined by constant growth in all variables, a GBGP is less strict and certain variables are allowed to grow at non-constant rates. The literature on structural transformation commonly reverts to this concept as transitions across sectors are essential to this literature.}.
\cite{Ngai2007StructuralGrowth} show that with goods being complements, employment shares shift to sectors with lower TFP growth; eventually, all labour is in the sector with the lowest TFP. In the present model, this is the clean sector. 

\section{Model Isomorphic to model with investment and rented capital}
The model is isomorphic to a model with (instantaneously productive) investment and full depreciation: 
\begin{align*}
I_t&=\psi(x_{dt}+x_{ct})\\
(LOM capital) \ K_t&=I_t= I_{ct}+I_{dt}
\end{align*}
That is, the capital good is produced by the following technology
\begin{align*}
x_{ijt}=\frac{I_{ijt}}{\psi}
\end{align*}
Machine producing firms rent the investment good, $I_{jt}$ and pay the real rate. They maximise over the choice of investment, i.e. capital, to borrow:
\begin{align*}
\underset{I_{ijt}}{\max}\hspace{2mm}p_{ijt}x_{ijt}-r_tI_{ijt}
\end{align*}
Profit maximisation of machine producing firms yields
\begin{align*}
\frac{p_{ijt}}{\psi}=r_t
\end{align*}
Free movement of capital and homogeneity of production costs imply that machine prices are equal across firms and sectors. 

Imposing market clearing for investment, $I_t=\int_{0}^{1}I_{idt}di+\int_{0}^{1}I_{ict}di$, and market clearing for machines yield a condition for the real rate in equilibrium
\begin{align*}
r_t=\alpha \psi^{-\alpha}\left(\frac{p_{dt}^{\frac{1}{1-\alpha}}A_{dt}L_{dt}+p_{ct}^{\frac{1}{1-\alpha}}A_{ct}L_{ct}}{K_t}\right)^{1-\alpha}
\end{align*}

\section{Results}
\begin{figure}[h!!]
	\centering
	\caption{Business as usual versus laissez-faire, substitutes, additional variables }\label{fig:onlyBAU_add}
	
	\begin{minipage}[]{0.32\textwidth}
		\centering{\footnotesize{(a) Clean output, $y_c$ }}
		%	\captionsetup{width=.45\linewidth}
		\includegraphics[width=1\textwidth]{../../codding_model/Own/figures/Rep_agent/staticBAU_LF_separate_yc_periods59_eppsilon4.00_zeta1.40_Ad08_Ac04_thetac0.70_thetad0.56_HetGrowth1_tauul0.181_util0_withtarget0_lgd0.png}
	\end{minipage}
	\begin{minipage}[]{0.32\textwidth}
		\centering{\footnotesize{(b) Dirty output, $y_d$}}
		%	\captionsetup{width=.45\linewidth}
		\includegraphics[width=1\textwidth]{../../codding_model/Own/figures/Rep_agent/staticBAU_LF_separate_yd_periods59_eppsilon4.00_zeta1.40_Ad08_Ac04_thetac0.70_thetad0.56_HetGrowth1_tauul0.181_util0_withtarget0_lgd0.png}
	\end{minipage}
	\begin{minipage}[]{0.32\textwidth}
		\centering{\footnotesize{(c) Labour input clean, $L_c$ }}
		%	\captionsetup{width=.45\linewidth}
		\includegraphics[width=1\textwidth]{../../codding_model/Own/figures/Rep_agent/staticBAU_LF_separate_Lc_periods59_eppsilon4.00_zeta1.40_Ad08_Ac04_thetac0.70_thetad0.56_HetGrowth1_tauul0.181_util0_withtarget0_lgd0.png}
	\end{minipage}
	\begin{minipage}[]{0.32\textwidth}
		\centering{\footnotesize{(d) Labour input dirty, $L_d$ }}
		%	\captionsetup{width=.45\linewidth}
		\includegraphics[width=1\textwidth]{../../codding_model/Own/figures/Rep_agent/staticBAU_LF_separate_Ld_periods59_eppsilon4.00_zeta1.40_Ad08_Ac04_thetac0.70_thetad0.56_HetGrowth1_tauul0.181_util0_withtarget0_lgd0.png}
	\end{minipage}
\begin{minipage}[]{0.32\textwidth}
	\centering{\footnotesize{(e) Machines clean, $x_c$}}
	%	\captionsetup{width=.45\linewidth}
	\includegraphics[width=1\textwidth]{../../codding_model/Own/figures/Rep_agent/staticBAU_LF_separate_xc_periods59_eppsilon4.00_zeta1.40_Ad08_Ac04_thetac0.70_thetad0.56_HetGrowth1_tauul0.181_util0_withtarget0_lgd0.png}
\end{minipage}
	\begin{minipage}[]{0.32\textwidth}
		\centering{\footnotesize{(f) Machines dirty, $x_d$}}
		%	\captionsetup{width=.45\linewidth}
		\includegraphics[width=1\textwidth]{../../codding_model/Own/figures/Rep_agent/staticBAU_LF_separate_xd_periods59_eppsilon4.00_zeta1.40_Ad08_Ac04_thetac0.70_thetad0.56_HetGrowth1_tauul0.181_util0_withtarget0_lgd0.png}
	\end{minipage}
\end{figure}

\begin{figure}[h!!]
	\centering
	\caption{Business as usual versus laissez-faire, complements, additional variables }\label{fig:onlyBAU_comp_add}
		\begin{minipage}[]{0.32\textwidth}
		\centering{\footnotesize{(a) Clean output }}
		%	\captionsetup{width=.45\linewidth}
		\includegraphics[width=1\textwidth]{../../codding_model/Own/figures/Rep_agent/staticBAU_LF_separate_yc_periods59_eppsilon0.40_zeta1.40_Ad08_Ac04_thetac0.70_thetad0.56_HetGrowth1_tauul0.181_util0_withtarget0_lgd0.png}
	\end{minipage}
	\begin{minipage}[]{0.32\textwidth}
		\centering{\footnotesize{(b) Dirty output }}
		%	\captionsetup{width=.45\linewidth}
		\includegraphics[width=1\textwidth]{../../codding_model/Own/figures/Rep_agent/staticBAU_LF_separate_yd_periods59_eppsilon0.40_zeta1.40_Ad08_Ac04_thetac0.70_thetad0.56_HetGrowth1_tauul0.181_util0_withtarget0_lgd0.png}
	\end{minipage}
	\begin{minipage}[]{0.32\textwidth}
		\centering{\footnotesize{(c) Labour input clean, $L_c$ }}
		%	\captionsetup{width=.45\linewidth}
		\includegraphics[width=1\textwidth]{../../codding_model/Own/figures/Rep_agent/staticBAU_LF_separate_Lc_periods59_eppsilon0.40_zeta1.40_Ad08_Ac04_thetac0.70_thetad0.56_HetGrowth1_tauul0.181_util0_withtarget0_lgd0.png}
	\end{minipage}
	\begin{minipage}[]{0.32\textwidth}
		\centering{\footnotesize{(d) Labour input dirty, $L_d$ }}
		%	\captionsetup{width=.45\linewidth}
		\includegraphics[width=1\textwidth]{../../codding_model/Own/figures/Rep_agent/staticBAU_LF_separate_Ld_periods59_eppsilon0.40_zeta1.40_Ad08_Ac04_thetac0.70_thetad0.56_HetGrowth1_tauul0.181_util0_withtarget0_lgd0.png}
	\end{minipage}
	\begin{minipage}[]{0.32\textwidth}
		\centering{\footnotesize{(e) Machines clean, $x_c$}}
		%	\captionsetup{width=.45\linewidth}
		\includegraphics[width=1\textwidth]{../../codding_model/Own/figures/Rep_agent/staticBAU_LF_separate_xc_periods59_eppsilon0.40_zeta1.40_Ad08_Ac04_thetac0.70_thetad0.56_HetGrowth1_tauul0.181_util0_withtarget0_lgd0.png}
	\end{minipage}
	\begin{minipage}[]{0.32\textwidth}
		\centering{\footnotesize{(f) Machines dirty, $x_d$}}
		%	\captionsetup{width=.45\linewidth}
		\includegraphics[width=1\textwidth]{../../codding_model/Own/figures/Rep_agent/staticBAU_LF_separate_xd_periods59_eppsilon0.40_zeta1.40_Ad08_Ac04_thetac0.70_thetad0.56_HetGrowth1_tauul0.181_util0_withtarget0_lgd0.png}
	\end{minipage}
\end{figure}

\begin{figure}[h!!]
	\centering
	\caption{Optimal allocation with emission target, complements, additional variables }\label{fig:optallo_comp_onlyR_add}
	\begin{minipage}[]{0.32\textwidth}
	\centering{\footnotesize{(a) Labour input clean, $L_c$ }}
	%	\captionsetup{width=.45\linewidth}
	\includegraphics[width=1\textwidth]{../../codding_model/Own/figures/Rep_agent/staticonlyRam_separate_Lc_periods59_eppsilon0.40_zeta1.40_Ad08_Ac04_thetac0.70_thetad0.56_HetGrowth1_tauul0.181_util0_withtarget1_lgd0.png}
\end{minipage}
	\begin{minipage}[]{0.32\textwidth}
		\centering{\footnotesize{(b) Labour input dirty, $L_d$ }}
		%	\captionsetup{width=.45\linewidth}
		\includegraphics[width=1\textwidth]{../../codding_model/Own/figures/Rep_agent/staticonlyRam_separate_Ld_periods59_eppsilon0.40_zeta1.40_Ad08_Ac04_thetac0.70_thetad0.56_HetGrowth1_tauul0.181_util0_withtarget1_lgd0.png}
	\end{minipage}
	\begin{minipage}[]{0.32\textwidth}
		\centering{\footnotesize{(c) Machines clean, $x_c$}}
		%	\captionsetup{width=.45\linewidth}
		\includegraphics[width=1\textwidth]{../../codding_model/Own/figures/Rep_agent/staticonlyRam_separate_xc_periods59_eppsilon0.40_zeta1.40_Ad08_Ac04_thetac0.70_thetad0.56_HetGrowth1_tauul0.181_util0_withtarget1_lgd0.png}
	\end{minipage}
\begin{minipage}[]{0.32\textwidth}
\centering{\footnotesize{(d) Machines dirty, $x_d$}}
%	\captionsetup{width=.45\linewidth}
\includegraphics[width=1\textwidth]{../../codding_model/Own/figures/Rep_agent/staticonlyRam_separate_xd_periods59_eppsilon0.40_zeta1.40_Ad08_Ac04_thetac0.70_thetad0.56_HetGrowth1_tauul0.181_util0_withtarget1_lgd0.png}
\end{minipage}
\begin{minipage}[]{0.32\textwidth}
	\centering{\footnotesize{(e) Price clean good, $p_c$}}
	%	\captionsetup{width=.45\linewidth}
	\includegraphics[width=1\textwidth]{../../codding_model/Own/figures/Rep_agent/staticonlyRam_separate_pc_periods59_eppsilon0.40_zeta1.40_Ad08_Ac04_thetac0.70_thetad0.56_HetGrowth1_tauul0.181_util0_withtarget1_lgd0.png}
\end{minipage}
	\begin{minipage}[]{0.32\textwidth}
		\centering{\footnotesize{(f) Price dirty good, $p_d$}}
		%	\captionsetup{width=.45\linewidth}
		\includegraphics[width=1\textwidth]{../../codding_model/Own/figures/Rep_agent/staticonlyRam_separate_pd_periods59_eppsilon0.40_zeta1.40_Ad08_Ac04_thetac0.70_thetad0.56_HetGrowth1_tauul0.181_util0_withtarget1_lgd0.png}
	\end{minipage}
	\begin{minipage}[]{0.32\textwidth}
	\centering{\footnotesize{(g) $\lambda$}}
	%	\captionsetup{width=.45\linewidth}
	\includegraphics[width=1\textwidth]{../../codding_model/Own/figures/Rep_agent/staticonlyRam_separate_lambdaa_periods59_eppsilon0.40_zeta1.40_Ad08_Ac04_thetac0.70_thetad0.56_HetGrowth1_tauul0.181_util0_withtarget1_lgd0.png}
\end{minipage}
\end{figure}



\section{Skill supply}
\paragraph{Effect of $\tau_l$ on skill investment}
From the definition of $H$ it has to hold that 
\begin{align}
&1=\frac{dh_l}{dH}+\zeta \frac{dh_h}{dH}\label{eq:ident} \\
\Leftrightarrow\ & \frac{dh_h}{dH}=\frac{1-\frac{dh_l}{dH}}{\zeta}.\label{eq:resp}
\end{align}
Using this equation, one can show that high skill supply is relatively more responsive to changes in total effective hours worked, i.e.,  $\frac{dh_h}{dH}>\frac{dh_l}{dH}$, if one excludes the case that high skill supply reduces as effective hours increase.\footnote{\ Proof: Suppose   $\frac{dh_h}{dH}>0$. Now, assume by contradiction that low skill supply is relatively more responsive. Hence, $\frac{dh_h}{dH}<\frac{dh_l}{dH}$. Using equation \ref{eq:resp}, one gets that $\frac{dh_l}{dH}>1+\zeta$. Replacing this inequality in the identity \ref{eq:ident}, it follows that $0>\zeta[1+\frac{dh_h}{dH}]$. Since $\zeta>1$ by assumption, it has to hold that $\frac{dh_h}{dH}<-1$ which contradicts the premise that $\frac{dh_h}{dH}>0$. } Thus, as the household reduces total effective hours supplied, the reduction in high skilled hours is higher. \tr{This should be due to the marginal utility from less high skill is higher than from less low skill hours.} This should show up in general equilibrium effects... but relative wages are fixed. 


\section{Calculations, partially wrong}
\noindent\rule[1ex]{\textwidth}{1pt}

\paragraph{Progressivity and emission targets}\tr{This result also rests on wrong premisses, but needs to be replicated!}
The constraint on emissions in the government's objective function implies that $Y_{dt}=\frac{\delta}{\kappa}$, thus, $(1+g_{ydt})=1$, for all time periods starting from 2050, $t\geq 30$. 

From the dirty sector's production function and equation \ref{eq:inf_d} we have that
\begin{align}
&\frac{Y_{d}'}{Y_d}=(1+\pi_d)^{\frac{\alpha}{1-\alpha}}(1+\upsilon_{d})\label{eq:gyd}\\
\Leftrightarrow\ &(1+\upsilon_{d})^{\frac{(1-\alpha)(1-\tau_l-\varepsilon)}{(1-\tau_l)-(\varepsilon(1-\alpha)+\alpha)}}=1\label{eq:def_taul}
\end{align}
The inflation rate in equation \ref{eq:gyd} captures the role of machine demand by the dirty sector. When the price at which dirty firms can sell their output is high, they demand more machines. A positive inflation, therefore, implies a rise in dirty output.

At the same time, a rise in the dirty good's price reduces demand. This counteracting mechanism is accounted for in equation \ref{eq:def_taul}. 
As will be shown below, this mechanism ensures that the government can target dirty sector production through tax progressivity. 

First, I establish an optimal policy result. Assume that the government cannot set the growth rate in the dirty sector, then equation \ref{eq:def_taul} defines $\tau_l$ on a balanced growth path.

\begin{prop}[Optimal tax progressivity]
	Assume growth of the dirty technology, $\upsilon_{d}$, is exogenously determined. 
	Then, to comply with the Paris Agreement, the government has to set the tax progressivity parameter, $\tau_l$, to $\tau^*_l=1-\varepsilon$ for $\varepsilon\neq 1$ (as otherwise the exponent in \ref{eq:def_taul} is not defined under the optimal tax rate.).
	When goods are complements, the optimal tax system is progressive. If goods are substitutes, the optimal tax system is regressive.
\end{prop}

The intuition is, that by choosing tax progressivity, the government affects price inflation in the dirty sector; compare equation \ref{eq:inf_d}. 
The result implies that inflation in the dirty sector under the optimal policy is negative when there is positive growth in dirty technology. The demand for machines has to decline by the same rate as technology growths for dirty output to be constant.  

Can this be an equilibrium as the price for dirty products declines?

How does tax progressivity affect inflation? 
First note that at a flat tax, the inflation rate is independent of the sector-specific technological growth rate. This is due to offsetting mechanisms.\tr{Continue}


\tr{Start from effect on HH:} 
(1) A rise in $\tau_l$ reduces disposable income and aggregate demand falls. This is a mechanical result from a higher tax rate, and a reduction in aggregate hours supplied.  
(1) For the dirty sector to demand labour, the costs of the labour input good has to balance its marginal product which positively depends on technological progress and the sector specific price. 

\noindent\rule[1ex]{\textwidth}{1pt} 


\noindent\rule[1ex]{\textwidth}{1pt}

\textcolor{blue}{below is wrong since the aggregate price level is not constant when G is disposed off.}
Define sector-specific inflation as: $1+\pi_{j}=\frac{p'_j}{p_j}$.
Using the definition of the aggregate price level, final good production, and optimality conditions in the clean sector, one can show that 
\begin{align}\label{eq:agg_supply}
\frac{Y'}{Y}= (1+\pi_c)^{\frac{\varepsilon(1-\alpha)+\alpha}{1-\alpha}}(1+\upsilon_{c}), \hspace{3mm} \text{(Supply side)}
\end{align}
since $\frac{L_{c}'}{L_c}=1$.
Using goods market clearance, the budget condition, and the FOC for total skill supply, it follows that 
\begin{align}\label{eq:agg_demand}
\frac{Y'}{Y}= \left(\frac{w'_h}{w_h}\right)^{1-\tau_l}. \hspace{3mm} \text{(Demand side)}\tr{\text{wrong, misses gov expenditures... or let lambdaa adjust, and machine production}}
\end{align}
Demand for the labour input good implies that 
\begin{align}\label{eq:labour income}
\frac{p'_{cL}}{p_{cL}}= (1+\pi_c)^\frac{1}{1-\alpha}(1+\upsilon_{c})
\end{align}
(independent of growth in $L_c$).

Multiplying both sides with $\left(\frac{w_h'}{w_h}\right)^{-1}$, using equation \ref{eq:agg_growth}, and that $\frac{p_{cL}}{w_h}$ is constant, it follows that 

\begin{align}
\frac{\frac{p'_{cL}}{w'_h}}{\frac{p_{cL}}{w_h}}= (1+\pi_c)^\frac{1}{1-\alpha}(1+\upsilon_{c})\left(\frac{Y'}{Y}\right)^{-\frac{1}{1-\tau_l}}=1.
\end{align}

Above equation determines inflation in the clean sector:
\begin{align}\label{eq:inf_c}
1+\pi_c=(1+\upsilon_{c})^{\frac{\tau_l(1-\alpha)}{(1-\tau_l)-\varepsilon(1-\alpha)-\alpha}}.
\end{align}
By symmetry of (i) how goods enter the production fo the final good and of (ii) sectors, it also holds that 
\begin{align}\label{eq:inf_d}
1+\pi_d=(1+\upsilon_{d})^{\frac{\tau_l(1-\alpha)}{(1-\tau_l)-\varepsilon(1-\alpha)-\alpha}}.
\end{align}


\noindent \textbf{(Aggregate output result, less relevant for main story)}

Hence, 
\begin{align}\label{eq:agg_growth}
(1+g_y)=\frac{Y'}{Y}=(1+\upsilon_{c})^\frac{(1-\tau_l)[1-(\varepsilon(1-\alpha)+\alpha)]}{(1-\tau_l)-(\varepsilon(1-\alpha)+\alpha)}.
\end{align}
Equation \ref{eq:agg_growth} implies the following proposition:
\begin{prop}[aggregate growth]
	\textit{For a proportional tax system, $\tau_l=0$, aggregate growth equals growth in the clean sector. 
		When the tax system is progressive\footnote{\ In the sense defined in \cite{Heathcote2017OptimalFramework}.}, $\tau_l>0$, then aggregate growth exceeds technology growth in the clean sector. When the tax rate is regressive, $\tau_l<0$, aggregate growth is smaller than technology growth in the clean sector. }
\end{prop}
\tr{Have to understand why.} 
With a flat tax system there is no inflation in the clean sector; compare equation \ref{eq:inf_c}. When the tax system is progressive, ...

\paragraph{Proof: labour input good constant}
%\textit{Check that the labour input good is constant:} 

First note that $\frac{l_{hc}}{l_{lc}}$ is constant over time. 
From the FOC governing high skill demand in the clean sector and equation \ref{eq:constant} we have:

\begin{align*}
\frac{l_{hc}}{l_{lc}}=\left(\frac{p_{cL}}{w_h}\theta_c\right)^{\frac{1}{1-\theta_c}}= constant.
\end{align*}

Substitution into the production function of the clean labour input good yields

\begin{align*}
\frac{L'_c}{L_c}=\frac{l_{lc}'}{l_{lc}}.
\end{align*}

\tr{To be continued.}

\textbf{\tr{To be shown next:  How $\tau_l$ affects (1) skill supply (level) and (2) externality. }}
\\

\paragraph{Overview BGP compatible growth rates}
\begin{align}
\frac{Y'}{Y}\\
\frac{w_h'}{w_h}=\frac{w_l'}{w_l}\\
tbc
\end{align}
\textbf{Conditions for BGP to exist}
Next to assuming no transition of labour input goods across sectors, a joint condition on tax progressivity and substitutability of sector goods ensures that output growth in both sectors is positive which has to be the case as otherwise production of one good tends to zero which cannot be an equilibrium when goods are no perfect substitutes.
Hence, on a BGP it has to hold that 
\begin{align*}
\frac{(1-\alpha)(1-\tau_l-\varepsilon)}{(1-\tau_l)-(\varepsilon(1-\alpha)+\alpha)}>0.
\end{align*}
There are two possible ranges of parameter values reads
\begin{align*}
\text{either}\\	&(1-\tau_l)>\varepsilon\hspace{4mm}&\text{if goods are substitutes} ;\\ \text{and}\ \ & (1-\tau_l)>\varepsilon(1-\alpha)+\alpha\hspace{4mm}&\text{if goods are complements}\\
\text{or}\\	&(1-\tau_l)<\varepsilon\hspace{4mm}&\text{if goods are complements} ;\\ \text{and}\ \ & (1-\tau_l)<\varepsilon(1-\alpha)+\alpha\hspace{4mm}&\text{if goods are substitutes}
\end{align*}
Focus on the case that goods are substitutes. Then the condition in the \textit{either}-statement prevents the government from choosing a progressive tax system, since $\tau_l<1-\varepsilon<0$.
Analogously, when goods are complements, the \textit{or}-statement excludes regressive tax systems.
I, therefore, ensure that when goods are complements, it holds that $\tau_l<(1-\alpha)(1-\varepsilon)$. When they are substitutes, it holds that $\tau_l>(1-\alpha)(1-\varepsilon)$.


\tr{Remaining problem: prices are not constant on BGP with fixed growth rates...Look at literature on positive trend inflation...}

\begin{comment}
\textbf{Below wrong Because wrong hl used}
From here,  equilibrium conditions determine prices $p_{dL}, p_{cL}$. Using \ref{eq:constant} skill wages follow. Together with the FOC on hours supply, wages determine aggregate demand. Imposing goods market clearing and using equations \ref{eq:lab_inputc} and \ref{eq:lab_inputd}, determines low skill hour demand in equilibrium.

\begin{align*}
h_l=\left( \frac{1}{\left(\frac{\alpha}{\psi}\right)^{\frac{\alpha}{1-\alpha}}\left[\left(p_c^\frac{\alpha}{1-\alpha}\chi_c A_c\right)^\frac{\varepsilon-1}{\varepsilon}+\left(p_d^\frac{\alpha}{1-\alpha}\chi_d A_d\right)^\frac{\varepsilon-1}{\varepsilon}\right]^\frac{\varepsilon}{\varepsilon-1}}\right)\ \lambda \left(H w_l\right)^{1-\tau_l}.
\end{align*}

Knowing $h_l$, the variables $L_c, \ L_d, \ h_h, \ l_{lc}, l_{ld}, l_{hc}, l_{hd}$ follow. 

Output of the clean and dirty sector read
\begin{align}
%Y_d& =  \frac{\chi_d A_d}{\left[\left(\left(\chi_d A_d\right)^\frac{\alpha}{\alpha+\varepsilon(1-\alpha)}\left(\chi_c A_c\right)^\frac{\varepsilon(1-\alpha)}{\alpha+\varepsilon(1-\alpha)}\right)^\frac{\varepsilon-1}{\varepsilon}+\left(\chi_d A_d\right)^\frac{\varepsilon-1}{\varepsilon}\right]^\frac{\varepsilon}{\varepsilon-1}} \lambda (H w_l)^{1-\tau_l}\\
&Y_d = \left(\frac{1}{\left(\frac{\chi_c A_c}{\chi_d A_d}\right)^{\frac{(\varepsilon-1)(1-\alpha)}{\alpha+\varepsilon(1-\alpha)}}+1}\right)^\frac{\varepsilon}{\varepsilon-1}\lambda (H w_l)^{1-\tau_l}\\
& Y_c= \left(\frac{1}{1+\left(\frac{\chi_d A_d}{\chi_c A_c}\right)^{\frac{(\varepsilon-1)(1-\alpha)}{\alpha+\varepsilon(1-\alpha)}}}\right)^\frac{\varepsilon}{\varepsilon-1}\lambda (H w_l)^{1-\tau_l}.
\end{align}

The government can affect dirty production by lowering aggregate demand. Note that $\chi_c,\ \chi_d$ are functions of the disutility from high skill labour supply, $\zeta$. As a result, the elasticity of diryt and clean output to tax progressivity is asymmetric.

\end{comment}