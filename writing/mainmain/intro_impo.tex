\section{Introduction}

\begin{quote}
"Mitigation pathways limiting warming to 1.5°C [...]  reduce emissions further to reach net zero $CO_2$ emissions in the 2050s."
\end{quote}

The latest reports of the International Panel on Climate Change \citep{Rogelj2018MitigationDevelopment.} highlights the importance of an absolute emission limit to comply with the Paris Agreement on limiting temperature rise to 1.5°C or well below 2°C. 
The economics literature on environmental policy has by and large allowed for a trade-off between consumption and pollution \citep{Barrage2019OptimalPolicy, Golosov2014OptimalEquilibrium} or studied relative emission targets \citep{Fried2018ClimateAnalysis}. 
The presence of an absolute emission target may change the optimal environmental policy, as it poses a limit on growth in fossil energy usage.
%Depending on the substitutability of green and fossil energy and the velocity of the green sector to grow, the absolute emission target may, first, pose a limit to consumption growth and, second, make 
In particular, untraditional policy measures in addition to corrective taxes may become optimal. %\textit{ (WHY THIS DIFFERENCE? Also with externalities in consumption pollution cannot be compensated for by consumption as the marginal utility of consumption reduces. BUT STILL MORE IS BETTER! SO IT CAN BE COMPENSATED! )} 

%For a reasonably calibrated endogenous growth model,
 I find that the optimal labour income tax is progressive when accounting for an absolute emission target. This finding highlights the importance of policy measures targeted at a \textit{reduction} of production in tandem with \textit{recomposing} policies such as carbon taxes to mitigate climate change. % Then again, I present data indicating a voluntary reduction in household consumption. Given this behavioural change, the optimal income tax progressivity could become regressive in order to boost high-skill labour supply. 

%MODEL
To investigate the effect of an exogenous emission target on the optimal policy, I study an endogenous growth model building on \cite{Fried2018ClimateAnalysis}. Allowing for endogenous growth is important to take seriously the possibility of green growth to keep consumption high while meeting emission targets. The government is characterised as a Ramsey planner who seeks to maximise Utilitarian social welfare but is constrained by an exogenous emission target. To abstract from inequality as a determinant of tax progressivity, the economy is populated by a representative family. Yet, the family supplies two types of skill. 

The model differentiates between high- and low-skilled labour to account for a skill bias found for the green sector \citep{Consoli2016DoCapital}. This asymmetry of sectors renders regressive taxes a tool to lower relative production costs in the green sector, as high-skill workers reduce their labour supply more in response to a more progressive tax valuing an additional unit of income less. %, a regressive tax functions as a green subsidy. % In fact, there is an externality arising from high-skill labour supply as it shapes the share of fossil to green energy production. 
On the other hand, labour income taxation lowers aggregate production as it renders leisure cheaper to households. 
In an endogenous growth framework, these channels affect the level and relation of growth in the economy. The government can use labour income taxes to boost growth through a market size effect. Together, the recomposing and the reductive channel shape the optimal tax progressivity from an environmental policy perspective. 


%Calibration
The calibration of the model proceeds in two steps. First, I set certain parameters to values found in the literature. Most impportantly, I use reasonable values of production and growth processes found in \cite{Fried2018ClimateAnalysis}. % who conducts a rigorous calibration exercise. 
With these parameter values at hand, I match the share of high skill in the green and non-green sectors as suggested by \cite{Consoli2016DoCapital}. The emission target is set to the values suggested by (CITE).

% Quantitative Experiment and Results
The main finding is the optimality of progressive taxes in order to reach emission targets in concert with carbon taxes. Indeed, the emission target could be satisfied by higher carbon taxes, yet, at a lower utility level.  
Meeting emission targets with a fossil tax alone, hours supplied are inefficiently high. Due to the cap on fossil energy usage and green and fossil energy being no perfect substitutes, the additional work effort is not sufficiently compensated for by rising consumption.  


\paragraph{Literature}
The paper relates to three strands of literature. First, it contributes to the literature on environmental policy by focusing on unconventional policy measures which is justified given the urgent nature of climate change mitigation.  
{Environmental Policy}
\\
{Limits to and endogenous growth}
\\{Public Policy}
Finally, the paper connects to the public policy literature which focuses on an efficiency-equity trade-off. In this project, instead, the reduction in labour effort induced by distortionary labour taxes has an advantageous effect: it reduces emissions. On the other hand, there is a recomposing effect which counteracts the reduction of emissions. This reduction-recomposition trade-off shapes the optimal tax progressivity in the present paper. 