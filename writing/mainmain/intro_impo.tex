\section{Introduction}

\begin{quote}
"Mitigation pathways limiting warming to 1.5°C [...]  reduce emissions further to reach net zero $CO_2$ emissions in the 2050s."
\end{quote}

The latest Reports of the International Panel on Climate Change highlights the importance of an absolute emission target by the 2050s to comply with the Paris Agreement on limiting temperature rise to 1.5°C. 
The economics literature on environmental policy has by and large allowed for a (limited) trade-off between consumption and pollution \citep{Barrage2019OptimalPolicy, Golosov2014OptimalEquilibrium} or studied relative emission targets \citep{Fried2018ClimateAnalysis}. 
However, the presence of an absolute emission target poses a limit to growth in fossil energy usage.
Depending on the substitutability of green and fossil energy and the velocity of the green sector to grow, the absolute emission target may, first, pose a limit to consumption growth and, second, make untraditional policy measures in addition to corrective taxes optimal.\textit{ (WHY THIS DIFFERENCE? Also with externalities in consumption pollution cannot be compensated for by consumption as the marginal utility of consumption reduces.  )} 

For a reasonably calibrated endogenous growth model, I find that the optimal labour income tax is progressive when accounting for an absolute emission target. This finding highlights the importance of policy measures targeted at a \textit{reduction} of production in tandem with policies intended at a \textit{recompostion} of the economic structure such as carbon taxes to mitigate climate change. % Then again, I present data indicating a voluntary reduction in household consumption. Given this behavioural change, the optimal income tax progressivity could become regressive in order to boost high-skill labour supply. 

%MODEL
To investigate the effect of an exogenous emission target on the optimal policy, I study an endogenous growth model building on \cite{Fried2018ClimateAnalysis}. Allowing for endogenous growth is important to take seriously the possibility of green growth to keep consumption high while meeting emission targets. The government is characterised as a Ramsey planner who seeks to maximise Utilitarian social welfare but is constrained by an exogenous emission target. To abstract from inequality as a determinant of tax progressivity, the economy is populated by a representative family. Yet, the family supplies two types of skill. 

The model differentiates between high- and low-skilled labour to account for a skill bias found for the green sector \citep{Consoli2016DoCapital}. This asymmetry of sectors renders regressive taxes a tool to lower production costs in the green sector. As high-skill workers reduce their labour supply more in response to a more progressive tax due to a relatively lower value of an additional unit of income, a regressive tax functions as a green subsidy. % In fact, there is an externality arising from high-skill labour supply as it shapes the share of fossil to green energy production. 
On the other hand, labour income taxation lowers aggregate production as it renders leisure cheaper to households. 
Together, the recomposing and the reductive channel shape the optimal tax progressivity from an environmental policy perspective. 

%Calibration
The Calibration of the model proceeds in two steps. First, I set certain parameters to values found in the literature. Most impportantly, I use reasonable value of production and growth processes as found in \cite{Fried2018ClimateAnalysis} who conducts a rigorous calibration exercise.  With these 

% Quantitative Experiment and Results

