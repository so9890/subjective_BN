\clearpage
\appendix
\section{Model}\label{app:model}

\subsection{Competitive equilibrium in simple model}

\begin{align}
\text{Utility}\hspace{5mm}& \frac{C_t^{1-\theta}-1}{1-\theta}-\chi \frac{h_t^{1+\sigma}}{1+\sigma}-\varphi(\omega F)^\eta\\
\text{Budget}\hspace{5mm}& C_t = \lambda_t(w_th_t)^{1-\tau_{\iota t}}\\
\text{optimality HH}\hspace{5mm}& h^{\sigma+\tau_{\iota t}+\theta(1-\tau_{\iota t})}=\lambda_t^{1-\theta}(1-\tau_{\iota t})w_t^{(1-\tau_{\iota t})(1-\theta)}\\
\text{Final Production}\hspace{5mm}&Y=F^{\varepsilon_y}G^{1-
	\varepsilon_y}\\ %\left[F^\frac{\varepsilon_y-1}{\varepsilon_y}+G^\frac{\varepsilon_y-1}{\varepsilon_y}\right]^\frac{\varepsilon_y}{\varepsilon_y-1}\\
%\text{price}\hspace{5mm}&1=p_y= \left(\frac{p_f}{\varepsilon_y}\right)^{\varepsilon_y}\left(\frac{p_g}{1-\varepsilon_y}\right)^{1-\varepsilon_y}\label{eq:ana_pr}\\
\text{Demand clean good}\hspace{5mm}&p_g=(1-\varepsilon)\left(\frac{F}{G}\right)^\varepsilon\label{eq:ana_dem_clean}\\
\text{Demand clean good}\hspace{5mm}&p_f=\varepsilon\left(\frac{F}{G}\right)^{\varepsilon-1}\label{eq:ana_dem_dirty}\\
\text{Production F and G}\hspace{5mm}&F=A_fL_f\label{eq:ana_prod_F}\\\
& G=A_gL_g\label{eq:ana_prod_G}\\
\text{labour demand}\hspace{5mm}& w=p_f(1-\tau_{ft})A_f\\
& w=p_gA_g\\
\text{technology}\hspace{5mm}&A_{ft+1}=(1+\nu_f)A_{ft}\\
&A_{gt+1}=(1+\nu_g)A_{gt}\\
\text{Government}\hspace{5mm}&T=\tau_{f}p_fF
\\
\text{Balanced income tax revenues}\hspace{5mm}&\lambda_t=\frac{w_t h_t}{(w_t h_t)^{1-\tau_{\iota t}}}\\
&E_{net}=\omega F-\delta
\end{align}

\subsection{Solving the model}
%Demand for intermediate goods determines the price ratio, $\frac{p_g}{p_f}$ in equilibrium, equation \ref{eq:ana_dem_fin}. 
%Intermediate good market clearing, i.e. substituting intermediate good production functions, equations \ref{eq:ana_prod_F} and \ref{eq:ana_prod_G}, in \ref{eq:ana_dem_fin} yields
%\begin{align}
%\frac{p_g}{p_f}=\frac{A_f}{Ag}\frac{s}{1-s}\frac{1-\varepsilon}{\varepsilon},
%\end{align}
%where I used that $s=\frac{L_f}{h}$ and the labor market clearing condition. Solving the price definition of the final goods price, equation \ref{eq:ana_pr} for $p_g$ as a function of $\frac{p_g}{p_f}$ and substituting the previous expression for the price ratio gives the price of the clean good in equilibrium as a function of labor shares:
%\begin{align}
%p_g=(1-\varepsilon)\left(\frac{A_f}{A_g}\right)^\varepsilon\left(\frac{s}{1-s}\right)^{1-\varepsilon}.
%\end{align}
%The equilibrium price paid for the clean good 



\subsection{Inefficiency in the wage rate with externality}

The wage rate in the competitive equilibrium is below the marginal product of hours worked. 
The reason is that the lower labor share in the dirty sector has to be sustained by a tax on dirty production. 
Otherwise, market forces would equilibrate the marginal product of labor in both sectors and $\varepsilon=s$. Yet, this disregards the negative externality of dirty production. 

In the competitive equilibrium, hence, the environmental tax serves to sustain the wedge between marginal products of labor across intermediate sectors: it is higher in the dirty and lower in the green sector.

This comes at the cost of lowering the aggregate wage rate in the economy to the marginal product of labor in the green sector. 
Nevertheless, the marginal product of labor in the fossil sector is higher. As a result, the aggregate wage rate, $w$, falls short of the aggregate marginal product of labor in the economy. This mechanism on its own renders labor supply inefficiently low in the competitive economy. However, as shown in proof \ref{prop:1}, the equilibrium hours supplied are inefficiently high in the competitve allocation. 


When the effect of a decreased wage  en gros reduces labor supply, it makes it more costly for the government to generate funds due to a smaller tax base for the income tax. This is the mechanism pointed to by the double dividend literature.
\section{Derivations and proofs}\label{app:derivations}

\subsection{Theory results \ref{sec:mod_an}}
\subsubsection{The social cost of pollution}

The social cost of pollution in my model is defined as the marginal price the representative household is willing to pay for a marginal reduction in dirty production. That is, the household maximises over dirty production for which a market exists.

The household's problem is determined as
\begin{align}
\underset{C,H,F}{\max} U(C,H,F)-\mu \left(C+\tilde{p}_fF-Y(H)\right).
\end{align}
Where $\mu$ is the Lagrange multiplier. Taking the derivative with respect to dirty production  and with respect to consumption yields
\begin{align}
U_F=\mu \tilde{p}_f,\\
U_C=\mu.
\end{align}
Substituting the Lagrange multiplier gives the negative of the equilibrium price the household is willing to pay for a reduction in dirty prodction: $\tilde{p}_f=\frac{U_F}{U_C}$. Since the environmental tax in the model is a percentage of revenues, the price producers pay per unit of dirty production is $\tau_f p_f$. Thus, the social cost of pollution to be deducted from to producers' revenues in percent is $\tau^{Pigou}=\frac{-U_F}{U_Cp_f}$.
\subsubsection{With an environmental tax, the wage rate in the competitive equilibrium is below the marginal product of labor}\label{app:wageMPL}

The aggregate marginal product of labor is defined as
\begin{align}
MPL&= \frp{Y}{H}.
\end{align}
This expression can be rewritten using relations of derivatives summarized below as follows.
\begin{align}
&= \frp{Y}{F}\frp{Y}{H}+\frp{Y}{G}\frp{G}{H}\\
&= \frp{Y}{F}\frp{F}{L_f}s+\frp{Y}{G}\frp{G}{L_g}(1-s)\\
&= \frp{Y}{G}\frp{G}{L_g}+ s\left(\frp{Y}{F}\frp{F}{L_f}-\frp{Y}{G}\frp{G}{L_g}\right).\label{eq:mpl_opt}
\end{align}
The term in brackets is positive under the optimal policy as can be seen from the first order condition with respect to $s$, equation \ref{eq:sbs}:
\begin{align}
\frp{Y}{F}\frp{F}{L_f}-\frp{Y}{G}\frp{G}{L_g}=\frac{1}{H}\left(\frp{Y}{F}\frp{F}{s}+\frp{Y}{G}\frp{G}{s}\right)=\frac{1}{H}\left(\frac{-U_F\frp{F}{s}}{U_C}\right)>0.
\end{align}
The inequality holds since the externality of polluting production is negative. %, above expression is positive.
%Therefore, the marginal product of labor in the efficient allocation equals
Now note that the first summand in equation \ref{eq:mpl_opt} is the competitive wage rate.  Hence $w<MPL$.

The gap between the wage rate and the marginal product of labor equals the gap between the marginal products of labor across sectors times the relative size of the dirty sector. 

\subsection{Sufficiency of the environmental tax when environmental tax revenues are redistributed lump sum}\label{app:incometax0}

Noticing that $\frac{\partial Y}{\partial H}= \frac{\partial Y}{\partial s}\frac{s}{H}-\frac{\partial Y}{\partial G}\frac{\partial G}{\partial s}\frac{1}{H}$ and that $\frac{\partial F}{\partial H}=\frac{\partial F}{\partial s}\frac{s}{H}$, and substituting equation \ref{eq:sbs} in equation \ref{eq:sbh} yields
\begin{align}\label{eq:pigou}
-U_C \frac{\partial Y}{\partial G}\frac{\partial G}{\partial s}\frac{1}{H}=-U_H
\end{align}
Substituting $U_H$ from household optimality, equation \ref{eq:hsup}, and the clean sectors' profit maximizing condition from equations \ref{eq:profmax} yields
\begin{align}
1=1-\tau^*_\iota.
\end{align}
Hence, $\tau^*_\iota =0$ from which follows that $\lambda =1$ so that the income tax scheme is a flat tax rate equal to zero; the labor income tax is not used in optimum.



\subsubsection{Proof proposition \ref{prop:1}: Absent lump-sum transfers, hours are inefficiently high}
\begin{proof}\textit{Absent lump-sum transfers, hours are inefficiently high when the environmental tax implements efficient share of dirty production and the wage rate is decreasing in equilibrium hours worked.}
	
	This proof proceeds by contradiction. 
	Assume by contradiction that $H^*\leq H_{FB}$. 
	It has to hold that 
	\begin{align}
	-U_H^*\leq -U_{H,FB}.
	\end{align} 
	Substituting the households' optimal labor supply and the social planner's first order condition for hours, equation \ref{eq:fbh} yields
	\begin{align}\label{eq:prH}
	U_C^*w^* \leq U_{C,FB}\frp{Y_{FB}}{G_{FB}}\frp{G_{FB}}{L_{g,FB}}
	\end{align}
	
	Rewriting equation \ref{eq:prH} above yields
	\begin{align}
	\frac{U_C^*}{U_{C,FB}}\leq \frac{\frp{Y_{FB}}{G_{FB}}\frp{G_{FB}}{L_{g,FB}}}{\frp{Y^*}{G^*}\frp{G^*}{L^*_{g}}}
	\end{align}
	where I replaced $w^*=\frp{Y^*}{G^*}\frp{G^*}{L^*_{g}}$.
	
	By assumption $s^*=s_{FB}$, $H^*\leq H_{FB}$, and the aggregate production function is increasing in its inputs. It follows that output is higher in the efficient allocation $Y_{FB}\geq Y^*$ and hence $C^*<C_{C,FB}$, since $Gov>0$ in the competitive equilibrium. By concavity of the utility function with respect to consumption (and the assumption of additive separability of hours and consumption in the utility function), we have that $\frac{U_C^*}{U_{C,FB}}>1$.
	Under the assumption that the wage rate is decreasing or constant in equilibrium hours, it holds that the right-hand side is below or equal unity since $H^*\leq H_{FB}$ by assumption. Thus,
	\begin{align}
	\frac{U_C^*}{U_{C,FB}}>1\geq \frac{\frp{Y_{FB}}{G_{FB}}\frp{G_{FB}}{L_{g,FB}}}{\frp{Y^*}{G^*}\frp{G^*}{L^*_{g}}}. 
	\end{align}
	A contradiction to the assumption that $H^*\leq H_{FB}$. Hence, it has to hold that $H^*>H_{FB}$. 
\end{proof}


\subsection{Derivation $\tau_f^*$ without lump-sum transfers, proposition \ref{prop:2} part 1}
	
Divide the Ramsey planner's first order condition with respect to $s$, equation \ref{eq:sbs}, by $U_C$ and $\frp{Y}{F}\frp{F}{s}$. Solving for $1+\frac{\frac{\partial Y}{\partial G}\frac{\partial G}{\partial s}}{\frac{\partial Y}{\partial F}\frac{\partial F}{\partial s}}$, which equals $\tau_f$, yields the desired result. 


\subsubsection{Derivation $\tau_l$ proof proposition \ref{prop:2} part 2}\label{app:subsub_nltaul}

\begin{proof}\textit{Absent lump-sum transfers, the optimal income tax scheme is progressive}
Following similar steps as in section \ref{subsec:Rams}, the optimal labor income tax progressivity parameter is given by
\begin{align}
\frac{\frac{s}{H}\frac{\partial Gov}{\partial s}- \frac{\partial Gov}{\partial H}}{\frac{\partial Y}{\partial G}\frac{\partial G}{\partial s}\frac{1}{H}}.
\end{align}

	Using the market clearing condition for final output to replace government spending and noticing the relations of derivatives with respect to aggregate labor supply and the dirty labor share, one can write above expression as
	\begin{align}
	\tau_{\iota}=1-\frac{H\frp{C}{H}-s\frp{C}{s}}{wH}.
	\end{align}
	Substituting $\frp{C}{H}=H\frp{w}{H}+w$ and $\frp{C}{s}=\frp{w}{s}$ from the household's budget constraint gives
	\begin{align}
	\tau_{\iota}=\frac{s}{w}\frp{w}{s}-\frac{H}{w}\frp{w}{H}.
	\end{align}
In a next step, I explicitly solve for $\frp{w}{s}$ and $\frp{w}{H}$, where I use that $w=\frp{Y}{G}\frp{G}{L_g}$ in equilibrium.

\begin{align}
\frp{w}{H}=\left(\frp{G}{L_g}\right)^2\frac{\partial^2Y}{\partial G^2}(1-s)+\frp{Y}{G}\frac{\partial ^2G}{\partial L_g^2}(1-s)+\frp{G}{L_g}\frac{\partial^2 Y}{\partial G \partial F}s\\
%%%%
\frp{w}{s}= \left(\frp{G}{L_g}\right)^2\frac{\partial ^2Y}{\partial G^2}(-H)+\frp{G}{L_g}\frac{\partial ^2Y}{\partial G \partial F}H+\frp{Y}{G}\frac{\partial ^2 G}{\partial L_g^2}(-H)
\end{align}
substituting derivatives and canceling terms yields:
\begin{align}
\tau_\iota= -\frac{H}{w}\left(\left(\frp{G}{L_g}\right)^2\frac{\partial ^2Y}{\partial G^2}+\frp{Y}{G}\frac{\partial ^2G}{\partial L_g ^2}\right).
\end{align}
If returns to scale of clean and aggregate production are decreasing or constant and dcereasing for at least one, than $\tau_\iota >0$ and the optimal income tax rate is progressive. 

Note that, the right-hand side of the previous expression equals the partial derivative of the wage rate with respect to the dirty labor share under the assumption that dirty production is fixed:
\begin{align}
\tau_\iota =\frac{1}{w}\frp{w}{s}_{F=\bar{F}}.
\end{align}
Since the presence of the environmental tax artificially increases labor in the green sector depressing the wage rate, the wage rate rises by a reduction of the green labor share. Thus, $\tau_\iota$ is positive. 
\tr{I guess I can show this from the first order condition of the Ramsey planner}
	\end{proof}

\subsubsection{Proof proposition \ref{prop:2} part 3: Infeasibility of efficient allocation}
\begin{proof}\textit{The efficient allocation is infeasible (under the assumption of constant or decreasing returns to scale)}
	To prove this claim, I assume that the government chooses the optimal policy; which is the highest social welfare the Ramsey planner can achieve. I show that the optimal policy does not satisfy the social planner's allocation. Since the social planner could have chosen the Ramsey planner's allocation  but did not, it follows that the social planner's allocation features a higher social welfare.
	
	For the optimal allocation to be efficient, it must be the case that $s^*=s_{FB}$, (i) $C^*=C_{FB}$, and (ii) $H^*=H_{FB}$. I show that, under the assumption that $s^*=s_{FB}$, either (i) or (ii) can hold at a time by demonstrating that assuming (i) violates (ii) and vice versa.
	
	
	
	%\begin{lemma}\textit{$\tau_f=0$ is not optimal}
	%When $\tau_f=0$ then $Gov=0$ and $\frp{Gov}{s}=0$. Furthermore, market forces then imply that the marginal products of labor are equal so that $\frp{Y}{F}\frp{F}{s}=-\frp{Y}{G}\frp{G}{s}$. Substituting this in equation \ref{eq:sbs} yields
	%\begin{align}
	%0=-U_F\frp{F}{s}>0,
	%\end{align}
	%a contradiction. 
	%\end{lemma}
	%
	%\textit{(i) Assume $C^*=C_{FB}$ and $s^*=s_{FB}$:}
	%Since $\tau_f\neq0$, it follows that $Gov>0$ and hence $Y^*=C^*+Gov>Y_{FB}$. Since the allocation of labor is the same in the efficient and the optimal allocation and output is rising in labor, it follows that $H^*>H_{FB}$. 
	%\tr{Missing: if $\tau_f<0$ then $Gov<0$ }
	
	If $s^*=s_{E>0,FB}<s_{E=0,FB}$ then it must be the case that the environmental tax is positive to sustain a gap between marginal productivities in the dirty and the clean sector: $\tau_f>0$. Then, $Gov=\tau_fp_fF>0$. 
	First assume that (i) holds true: $C^*=C_{FB}$. From the good's market clearing condition and resource constraint of the social planner's problem it follows that
	$Y^*-Gov=C^*=C_{FB}=Y_{FB}$, due to  $Gov>0$ we have that $Y^*>Y_{FB}$. Since hours are the only production input, positively affect output, and $s^*=s_{FB}$ the higher output in the optimal allocation implies that $H^*>H_{FB}$. A violation of condition (ii). 
	
	Assume now that condition (ii) holds: $H^*=H_{FB}$. Since $s^*=s_{FB}$ by assumption it holds that $Y^*=Y_{FB}$ and, by the same argument as before: $Gov>0$. Thus, by the resource and market clearing condition: $C_{FB}=Y_{FB}>Y^*-Gov=C^*$. When labor supply is efficient, then consumption is inefficiently low; condition (i) is violated. 
	
	\begin{comment} (Proof building on first order conditions)
	Assume, 
	The social planner's first order condition on labor supply can be written as
	\begin{align}
	-U_{H, FB}=U_{C, FB}\frp{Y}{G}_{FB}\frp{G}{L_g}_{FB}
	\end{align}
	and optimal labor supply is determined by
	\begin{align}
	-U^*_{H}&=U^*_C(1-\tau_\iota)w
	\end{align}
	Equalizing yields
	\begin{align}
	U_C^*(1-\tau_\iota)w=U_{C,FB}\frp{Y}{G}_{FB}\frp{G}{L_g}_{FB},
	\end{align}
	a condition for optimal labor supply to be efficient. 
	
	In the following, I demonstrate that (i) assuming $C^*=C_{FB}$ violates the condition above and $H^*\neq H_{FB}$ and that (ii) assuming $H^*=H_{FB}$ results in $C^*<C_{FB}$. 
	
	\textit{(i) Assume $C^*=C_{FB}$:}
	then
	\begin{align}
	(1-\tau_\iota)w=\frp{Y}{G}_{FB}\frp{G}{L_g}_{FB}.
	\end{align}
	Assume by contradiction that $H^*=H_{FB}$, since $s^*=s_{FB}$ by assumption, it follows that $w=\frp{Y}{G}_{FB}\frp{G}{L_g}_{FB}$. 
	Since $\tau_\iota\neq 0$ under constant or decreasing returns to scale, it holds that $H^*<H_{FB}$, a contradiction. 
	
	
	%Hence,
	%\begin{align}
	%(1-\tau_\iota)w<\frp{Y}{G}_{FB}\frp{G}{L_g}_{FB}.
	%\end{align}
	%
	%Labor supply in the competitive equilibrium is lower than in the efficient allocation when consumption is equal under the optimal policy. WHY?
	%It follows, that optimal labor supply does not equal its efficient counterpart when optimal consumption is efficient.
	
	\textit{(ii) Assume $H^*=H_{FB}$:} 
	It follows that 
	\begin{align}
	\frac{U_C^*}{U_{C,FB}}=\frac{\frp{Y}{G}_{FB}\frp{G}{L_g}_{FB}}{w}\frac{1}{1-\tau_\iota}=\frac{1}{1-\tau_\iota}>1.
	\end{align}
	From concavity of the utility function it follows that $C^*<C_{FB}$. 
	
	content...
	\end{comment}
\end{proof}

\subsection{Proof proposition \ref{prop:3}}
\begin{proof} \textit{The optimal income tax scheme is progressive}\\ % if the optimal environmental tax is positive.}\\
	Under the new policy, the household's labor supply is determined by
	\begin{align}
	-U_H=\frac{U_C (1-\tau_{\iota})(wH+\tau_f p_fF)}{H}.
	\end{align}
	Expressing the derivatives in the Ramsey planner's first order condition with respect to hours as derivatives with respect to the dirty labor share, $s$, and substituting the first order condition with respect to $s$ yields:
	\begin{align}
	U_C \frp{Y}{G}\frp{G}{L_g}=-U_H.
	\end{align}
	%This equation is equivalent to the social planner's first order condition on hours, equation \ref{eq:fbh}. The optimal policy is to choose
	%\tr{Does this give a hint to why inefficiency without redistribution? The Ramsey planner's foc and household optimality always coincide. But, when Gov does not cancel the two do not coincide! ? the two do not coincide, Bcs consumption is too low so that $U_C$ too high which increases}
	Noticing that $\frp{Y}{G}\frp{G}{L_g}=w$ and replacing household's labor supply condition gives
	\begin{align}
	& w=\frac{(1-\tau_\iota)Y}{H}\\
	\Leftrightarrow\ & \tau_\iota=1-\frac{wH}{Y}. 
	\end{align} 
	Since $Y=C=wH+\tau_fp_fF$ from the market clearing and household budget constraint, it follows that $wH<Y$ whenever $\tau^*_f>0$. Hence, $\tau_f^*>0$ implies $\tau^*_{\iota}>0$.
	%
	%Observe that $Y\geq MPL \times H$, where $MPL$ stands in for the marginal product of labor, if the aggregate production function features decreasing or constant returns to scale. Under such a production function one can rewrite the last expression as
	%\begin{align}
	%\tau_{\iota}=1-\frac{wH}{Y}\geq 1-\frac{w H}{MPL \times H}
	%\end{align}
	% Note further that the marginal product of labor exceeds the wage rate whenever the environmental tax is different from zero; compare the disucssion in subsection \ref{subsec:Rams}. It follows that the right-hand side is positive, hence
	%\begin{align}
	%\tau_{\iota}>0,
	%\end{align}
	The optimal tax scheme is progressive.
\end{proof}

\begin{proof}\textit{The optimal allocation is efficient}
	
	The idea of this proof is to show that the efficient allocation is attainable for the Ramsey planner. Since the social planner could implement any competitive allocation (which necessarily satisfies the resource constraint) and has the same objective function, the efficient allocation maximizes the Ramsey problem. 
	
	To show that the efficient allocation is feasible, I assume that $s^*=s_{FB}$. Showing that $H^*=H_{FB}$ and $C^*=C_{FB}$ are a solution to the Ramsey problem, proves that the optimal policy implements the efficient allocation for two reasons. First, by the argument in the previous paragraph any competitive allocation is a potential candidate solution to the social planner's problem and the social planner has the same objective function. Second, due to strict concavity of the utility and strict monotonicity of the production function \textit{(so that more input means more output)}, the solution is also unique.
	
	When $H^*=H_{FB}$ then $C^*=C_{FB}$ since $s^*=s_{FB}$ by assumption. It now show that under this allocation optimal labor supply, indeed, is efficient, that is:
	\begin{align}
	U_C^*\frp{Y^*}{G^*}\frp{G^*}{L^*_{g}} = U_{C,FB}\frp{Y_{FB}}{G_{FB}}\frp{G_{FB}}{L_{g,FB}}.
	\end{align}
	
	From the assumed allocation it follows that $U_C^*=U_{C,FB}$ and $\frp{Y_{FB}}{G_{FB}}\frp{G_{FB}}{L_{g,FB}}=\frp{Y^*}{G^*}\frp{G^*}{L^*_{g}}$ and above condition is satisfied. 
	
	It remains to show that under the assumed allocation $s^*=s_{FB}$ holds true. Since $Gov=0$ the Ramsey planner's first order condition with respect to $s$ equals that of the social planner. Since production and marginal utilities in the optimal allocation equal their counterparts in the efficient allocation, it has to holds that $\tau_f^*$ implements $s^*=s_{FB}$.  
	%Second, efficiency of labor supply, i.e., $H^*=H_{FB}$, as the only solution of the Ramsey planner's problem follows from demonstrating that both (i) $H^*>H_{FB}$ and (ii) $H^*<H_{FB}$ result in a contradiction under the assumption that $s^*=s_{FB}$.
	
	%Assume by contradiction that (i), $H^*>H_{FB}$. 
\end{proof}


\paragraph{Useful relations}
\begin{align}
\frac{\partial Gov}{\partial s}=\frac{\partial Y}{\partial F}\frac{\partial F}{\partial s}+\frac{\partial Y}{\partial G}\frac{\partial G}{\partial s}-\frac{\partial C}{\partial s}\\
\frac{\partial Gov}{\partial H}=\frac{\partial Y}{\partial F}\frac{\partial F}{\partial s}+\frac{\partial Y}{\partial G}\frac{\partial G}{\partial s}-\frac{\partial C}{\partial H}\\
\frac{\partial Gov}{\partial s}=\frac{\partial Y}{\partial s}-\frac{\partial C}{\partial s}\\
\frac{\partial Gov}{\partial s}=p_f F \frac{\partial \tau_f}{\partial s}+\tau_f F \frac{\partial p_f}{\partial s}+\tau_f p_f \frac{\partial F}{\partial s}\\
%\frac{\partial \tau_f}{\partial s}= -\frac{1-\varepsilon}{\varepsilon}\frac{1}{(1-s)^2}, \\
%\frac{\partial \tau_f}{\partial s}=p_f\frac{1-\varepsilon}{1-\tau_f}\frac{\partial \tau_f}{\partial s}\\
\frac{\partial F}{\partial s}=\frac{F}{s}
\\
\text{Translation derivates H and s and L}\\
\frp{Y}{H}= \frp{Y}{s}\frac{s}{H}+\frp{Y}{G}\frp{G}{Lg}
\\
\frp{G}{H}=-\frac{(1-s)}{H}\frp{G}{s}
\\\frp{G}{s}=-H\frp{G}{L_g}\\
\frp{F}{H}=\frac{s}{H}\frp{F}{s}\\
\frp{F}{s}=H\frp{F}{L_f}
\end{align}


\subsection{Numeric results in simple model}
\begin{table}[h!!]
	\caption{Linear tax scheme and lump-sum transfers}\label{tab:lin_lst}
	\begin{tabular}{lllllllll}
		Thetaa & FB hours & FB Pigou & CE hours & CE scc & Opt hours & Opt taul & Opt tauf & Opt scc \\ 
		\hline 
		<1 & 1.192 & 0.99326 & 1.192 & 0.99326 & 1.192 & -3.7748e-15 & 0.99326 & 0.99326 \\ 
		Bop & 0.13601 & 0.99959 & 0.13601 & 0.99959 & 0.13601 & -3.7748e-15 & 0.99959 & 0.99959 \\ 
		log & 0.36434 & 0.99853 & 0.36434 & 0.99853 & 0.36434 & -3.7748e-15 & 0.99853 & 0.99853 \\ 
		\hline 
	\end{tabular}
\end{table}
\begin{table}
	\caption{Linear tax scheme, env. tax revenues not transferred lump-sum}\label{tab:lin_nolst}
	\begin{tabular}{lllllllll}
		Thetaa & FB hours & FB Pigou & CE hours & CE scc & Opt hours & Opt taul & Opt tauf & Opt scc \\ 
		\hline 
		<1 & 1.192 & 0.99326 & 1.2061 & 0.97804 & 1.1706 & 0.049876 & 0.9934 & 0.94584 \\ 
		Bop & 0.13601 & 0.99959 & 0.14026 & 0.96001 & 0.13808 & 0.049876 & 0.99958 & 0.94766 \\ 
		log & 0.36434 & 0.99853 & 0.37243 & 0.97015 & 0.36435 & 0.049876 & 0.99853 & 0.94804 \\ 
		\hline 
	\end{tabular}
\end{table}
\begin{table}[h!!]
	\caption{Baseline model env. revenues transferred via income tax scheme ($\lambda$)}\label{tab:base}
	\begin{tabular}{lllllllll}
		Thetaa & FB hours & FB Pigou & CE hours & CE scc & Opt hours & Opt taul & Opt tauf & Opt scc \\ 
		\hline 
		<1 & 1.192 & 0.99326 & 1.2275 & 1.0056 & 1.192 & 0.049979 & 0.99326 & 0.99326 \\ 
		Bop & 0.13601 & 0.99959 & 0.13811 & 1.0311 & 0.13601 & 0.049979 & 0.99959 & 0.99959 \\ 
		log & 0.36434 & 0.99853 & 0.37243 & 1.0211 & 0.36434 & 0.049979 & 0.99853 & 0.99853 \\ 
		\hline 
	\end{tabular}
\end{table}



Table 1 to 3 compare the efficient allocation to an allocation resulting in the competitive equilibrium when the environmental tax is set to equal the social cost of carbon in the efficient allocation. The rationale being that without any further distortions setting environmental taxes to the social cost of carbon implements the efficient allocation. The last four columns of each table show hours worked, the optimal policy and the social cost of carbon in equilibrium resulting in the Ramsey planner allocation. 

Table \ref{tab:lin_lst} reveals that indeed, setting the corrective tax equal to the social cost of carbon under the social planner implements the first-best allocation when lump-sum transfers are available. The optimal policy chooses zero income taxes. 

The picture changes once no lump-sum transfers are available, compare table \ref{tab:lin_nolst}. In the competitive equilibrium setting the environmental tax to the social costs of carbon under the social planner results in inefficiently high hours worked for all values of $\theta$ considered; compare the columns showing the allocation resulting in the competitive equilibrium when only the efficient dirty share is implemented. 
Theoretically, the labor income tax can be used to establish the 
efficient level of hours worked given that the dirty labor share is efficient. However, since
environmental tax revenues are not redistributed lump-sum, household consumption is lower than under the social planner and the efficient level of hours worked and the efficient dirty labor share feature a lower social welfare in the competitive equilibrium. In other words, a further reduction in labor is too costly in terms of consumption and the optimal labor tax is lower than what would implement efficient hours. \textit{This might change when the household derives utility from government consumption.}

The optimal policy is to set a positive income tax rate; the optimal income tax code is progressive. When the substitution effect outweighs the income effect, i.e., $\theta<1$, then the optimal allocation results in inefficiently \textit{low} hours worked. When the income effect is at least as strong than the substitution effect, that is $\theta\geq 1$, then hours worked remain inefficiently high under the optimal policy. 

Interestingly, when the planner transfers environmental tax revenues through the income tax scheme, table \ref{tab:base}, then the efficient allocation is attainable for all values of $\theta$ considered through a progressive tax scheme. 

Only when the Ramsey planner can implement the efficient level of work, the environmental tax is set to equal the social cost of carbon.   




\textbf{In a nutshell}
\begin{itemize}
	\item hours worked without transfers are always too low even if efficient tax rate is chosen
	\item when hours are not efficient, then the environmental tax does not match the social cost of carbon
	\item when revenues are transferred through the income tax, the planner can implement the efficient allocation with the help of a progressive income tax \textit{(interesting!)}
	\item with $\theta<\frac{\varepsilon}{\varepsilon-s}$ optimal hours worked reduce, otherwise the income effect is too strong and hours worked increase! 
	Nevertheless, the allocation in LF without lump sum transfers always features too high hours worked. 
	\item why does the optimal policy with taul but no lump-sum transfers not implement the efficient level? \ar income taxes are not a measure to implement the efficient allocation; only similar when income and substitution effect cancel. Too high when income effect dominates, too low when substitution effect dominates.
	\ar general consumption tax should neither be able to implement efficient allocation! 
	%\item when there is no income tax, the optimal policy is to set the efficient dirty labour share (compare table \ref{tab:lin_nolst_notaul}). Labor supply is always too high but the optimal tax exceeds the social cost of carbon
\end{itemize}


\subsection{Derivation expression for $h^{FB}$}
Rewriting equation \ref{eq:fbh}, the efficient amount of hours worked can be indirectly expressed as:
\begin{align}
h^{FB}=\frac{1}{\chi^\frac{1}{\sigma}}\left(w_{eff}^{1-\theta}-\frac{dE}{dF}A_f s^{FB}\left(h^{FB}\right)^\theta \right)^\frac{1}{\sigma+\theta}.\label{eq:heff_1}
\end{align}

Note that an explicit expression for $h^{FB}$ follows from equation \ref{eq:fbs} when there is an externality and $\frac{dE}{dF}\neq 0$. Then 

\begin{align}
h^{FB}= \left(\frac{\varepsilon(1-s)-s(1-\varepsilon)}{s(1-s)}\frac{w_{eff}^{1-\theta}}{\frac{dE}{dF}A_f }\right)^\frac{1}{\theta}
\end{align}
and the result follows from substituting the last expression in expression  \ref{eq:heff_1}.

\subsection{Proof: Hours worked with only the efficient share of dirty labor are inefficiently high}

\begin{proof}
The proof proceeds in two steps. First, I show that the share of labor allocated to the dirty sector is smaller than its efficient level absent externality which is $s=\varepsilon$.
In the second step, I show that even if the environmental tax is set to the tax which replicates the efficient share of dirty labor, hours worked, denoted by $h_{CE, s^{eff}}$, exceed their efficient level, $h_{FB}$, when neither lump-sum transfers no labour income taxes are available. 

First note that the share of dirty labor is fully determined by the environmental tax. The environmental tax is set to implement a gap between the marginal product of labor between the clean and the dirty sector. The relation follows from labor market clearing and intermediate goods market clearing 
\begin{align}
\tau_f = \frac{\varepsilon-s}{(1-s)\varepsilon}\label{eq:tauf}
\end{align}

\textbf{Step 1:} $\frac{dE}{dF}>0$ \ar $\varepsilon>s$\\
Rewriting equation \ref{eq:fbs} yields
\begin{align}
\frac{\varepsilon(1-s)-s(1-\varepsilon)}{s(1-s)}=\frac{dE}{dF}A_fh^\theta w_{FB}^{1-\theta}.
\end{align}
When the externality is negative, i.e., $\frac{dE}{dF}>0$, then the right-hand side is positive.
Since $s\in(0,1)$ - due to both intermediate goods being necessary to produce the final good and zero consumption is not a solution - the left-hand side is positive when
\begin{align}
\varepsilon(1-s)-s(1-\varepsilon)>0,
\end{align}
which holds true if and only if $\varepsilon>s$.

\textbf{Step 2:} $\varepsilon>s$ \ar $h_{CE, s^{eff}}>h_{FB}$\\
I prove the claim by evoking a contradiction to the assumption that $h_{CE, s^{eff}}\leq h_{FB}$. Using equation \ref{eq:hopt} and \ref{eq:heff} the expression becomes

\begin{align}
&\left(\frac{w^{1-\theta}}{\chi}\right)^{\frac{1}{\sigma+\theta}}\leq \left(\frac{w_{FB}^{1-\theta}}{\chi}\frac{1-\varepsilon}{1-s}\right)^\frac{1}{\sigma+\theta}
\\
\Leftrightarrow&\left(\frac{w}{w_{FB}}\right)^{\frac{1-\theta}{\sigma+\theta}}\leq \left(\frac{1-\varepsilon}{1-s}\right)^\frac{1}{\sigma+\theta}
\end{align}

Note that the ratio of the wage in the competitive economy to the marginal product of labor is $\frac{w}{w_{FB}}=\frac{1-\varepsilon}{1-s}$, which follows from equation \ref{eq:compw} and the definition of $w_{FB}$ under the assumption that the dirty labor share is set to the first best equivalent. Substituting this in the previous equation and rearranging terms yields
\begin{align}
\left(\frac{1}{1-\varepsilon}\right)^\frac{\theta}{\sigma+\theta}\leq \left(\frac{1}{1-s}\right)^\frac{\theta}{\sigma+\theta}
\end{align}
which holds true whenever
\begin{align}
\varepsilon<s.
\end{align}
This contradicts $s<\varepsilon$ which has been shown to hold in presence of a negative externality in the dirty sector in step 1. 
\end{proof}

\textit{Intuition:} the fact that the wage rate in the competitive equilibrium understates the marginal product of labor  depresses labor supply due to a substituion effect. When the income effect is more pronounced, that is, $\theta>1$, the low wage rate increases labor supply above the efficient level. When the substitution effect is stronger when $\theta<1$, then the distortion in the wage rate decreases labor supply. 
The neglected contribution to the externality by households in the competitive equilibrium makes hours inefficiently high irrespective of parameter values. 
Hence, with $\theta<1$ the overall distortion in hours worked is mitigated has households the stronger substitution effect offsets part of the inefficient high labor supply due to the neglect of the externality. ¸

In the model, it does so by exactly the same amount as hours contribute to the externality. 
than compensated for by the neglect of the negative effect of hours on the externality due to the concave curvature of the utility function, $\theta>0$. If $\theta=0$, then the two effects would exactly offset. 



\subsection{Proof: lump-sum transfers restore the efficient allocation}
\begin{proof}\label{pr:lst_eff}
To establish that lump-sum transfers of environmental tax revenues restore the efficient allocation, I first derive the size of lump-sum transfers which implement the efficient level of hours worked given that the efficient dirty labor share is established by choice of the environmental tax. In a second step, I show that this level of transfers coincides with the revenues from the environmental tax when this is set to implement the efficient dirty labor share. 
These two steps prove that the efficient amount of hours results from lump-sum transferring environmental tax revenues. Finally, I show that the resulting level of consumption is efficient. This completes the proof.



\textbf{Step 1:} Solve for transfers which implement efficient level of hours\\
I equalize equation \ref{eq:heff} and \ref{eq:hopt} setting the income tax progressivity, $\tau_{\iota}$, to zero.
\begin{align}
\left(\frac{w^{1-\theta}\left(1+\frac{T^*}{wh}\right)^{-\theta}}{\chi}\right)^\frac{1}{\sigma+\theta}=\left(\frac{w_{FB}^{1-\theta}}{\chi}\frac{1-\varepsilon}{1-s}\right)^\frac{1}{\sigma+\theta}.
\end{align}
Using the relation of $w_{FB}$ and $w$ established in the previous proof, I can rewrite the right-hand side
\begin{align}
&\left(\frac{w^{1-\theta}\left(1+\frac{T^*}{wh}\right)^{-\theta}}{\chi}\right)^\frac{1}{\sigma+\theta}=\left(\frac{w^{1-\theta}}{\chi}\left(\frac{1-s}{1-\varepsilon}\right)^{1-\theta}\frac{1-\varepsilon}{1-s}\right)^\frac{1}{\sigma+\theta}.
\end{align}
This step is instructive in showing that transfers will correct for the two inefficiencies in the competitive economy: (i) the too low wage rate captured by the term $\left(\frac{1-s}{1-\varepsilon}\right)^{1-\theta}$, and (ii) the neglect of the effect of hours worked on the externality, captures by $\frac{1-\varepsilon}{1-s}<1$. 

Solving for transfers yields
\begin{align}
T^* = \left(\frac{\varepsilon-s}{1-\varepsilon}\right)wh_{FB},
\end{align}

\textbf{Step 2:} $T^*=\tau_f^*p_fF$\\
The environmental tax in equilibrium is determined by equation \ref{eq:tauf}: $\tau_f = \frac{\varepsilon-s}{(1-s)\varepsilon}$. Free labor movement enforcing a unique wage rate implies that $p_f=p_g\frac{A_g}{(1-\tau_f)A_f}$. The price for the clean good, $p_g$, in equilibrium, balances clean demand and wages paid: $p_g=\varepsilon^\varepsilon(1-\varepsilon)^{1-\varepsilon}\left(\frac{(1-\tau_f)A_f}{A_g}\right)^\varepsilon$. Dirty output, $F$, is given by $F=A_fsh$. 
Substituting these expressions in the expression for $T^*$ above and observing that $w=(1-\varepsilon)\left(\frac{A_f}{A_g}\frac{s}{1-s}\right)^\varepsilon A_g$ yields the result.

\textbf{Step 3: } Consumption under $T^*$ is efficient\\
Trivially, as market clearing has to hold in the competitive equilibrium, it follows that 
\begin{align}
C^*=\left(A_f s^*\right)^\varepsilon\left(A_g(1-s^*)\right)^{1-\varepsilon}h^*
\end{align} 
Since $T^*$ and $\tau_f^*$ have been set to establish the efficient dirty labor share, $s^*=s_{FB}$ and the efficient level of hours worked, $h^*=h_{FB}$, it follows that $C^*=C_{FB}$. This completes the proof.
\end{proof}

\subsection{Infeasibility of efficient allocation if environmental tax revenues are consumed by the government}

\begin{proof}
	The proof proceeds by construction. First, I assume that the efficient dirty labor share has been implemented and that consumption equals the efficient level. Solving for the competitive level of hours shows that they exceed the efficient level of hours.  Hence, the efficient allocation is not feasible when the government consumes environmental tax revenues since work effort has to be inefficiently high to sustain the first-best level of consumption. 
	
	Working hours to support the efficient level of consumption are given by the market clearing condition  (which holds true with and without income tax scheme)
	\begin{align}
		C_{FB} = \left(A_f s_{FB}\right)^\varepsilon\left(A_g(1-s_{FB})\right)^{1-\varepsilon}h-\tau_f^*p_f^*A_fs_{FB}h
	\end{align}
Since $s^*=s_{FB}$ by assumption, substituting equilibrium expressions for $\tau_f^*$ and $p_f^*$ used in proof \ref{pr:lst_eff} and solving for $h$ yields
	\begin{align}
	h=\frac{C_{FB}}{w}.
	\end{align}
	Substitution of consumption from the first best allocation, $C_{FB}=\left(A_f s_{FB}\right)^\varepsilon\left(A_g(1-s_{FB})\right)^{1-\varepsilon}h_{FB}$, gives
	\begin{align}
\frac{h}{h_{FB}}=\frac{w_{FB}}{w}.
	\end{align}
	Since $\varepsilon>s$ the right-hand side, $\frac{w_{FB}}{w}=\frac{1-s}{1-\varepsilon}$, is above unity. Hence, $h>h_{FB}$. 
\end{proof}

\subsection{Proof: redistributing environmental tax revenues through the non-linear income tax scheme restores the efficient allocation. The income tax scheme to support the efficient allocation is progressive.}

\begin{proof}
	Hours under the non-linear tax-scheme policy become
	\begin{align}
	h=\left(\frac{(1-\tau_{\iota})w^{1-\theta}(1+\tau_f p_f\frac{F}{hw})^{1-\theta}}{\chi}\right)^\frac{1}{\sigma+\theta}.
	\end{align}

	I assume that the dirty labor share is set to the efficient level, $s=s_{FB}$. This determines $\tau^*_f$ and $p^*_f$.
	It has to be shown that
	\begin{align}
	\left(\frac{(1-\tau_{\iota})w^{1-\theta}\left(1+\tau^*_f p^*_f\frac{F_{FB}}{h_{FB}w}\right)^{1-\theta}}{\chi}\right)^\frac{1}{\sigma+\theta}=\left(\frac{w^{1-\theta}}{\chi}\left(\frac{1-s}{1-\varepsilon}\right)^{1-\theta}\frac{1-\varepsilon}{1-s}\right)^\frac{1}{\sigma+\theta}
	\end{align}
	 From proof \ref{pr:lst_eff} step 2 we know that $\frac{\tau_f^*p_fF}{(wh_{FB})}=\left(\frac{\varepsilon-s}{1-\varepsilon}\right)$ and hence $\left(1+\frac{\tau_f^*p_fF}{wh_{FB}}\right)^{-\theta}=\left(\frac{1-s}{1-\varepsilon}\right)^{-\theta}$ and above condition simplifies to
	 \begin{align}
	(1-\tau_{\iota})\frac{\varepsilon-s}{1-\varepsilon}=1.
	 \end{align}
	 Rearranging terms yields
	 \begin{align}
	\tau_{\iota}= \frac{\varepsilon-s}{1-s}.
	 \end{align}
	 Since $\frac{\varepsilon-s}{1-s}\in(0,1)$ when there is a negative externality from dirty production, the optimal tax scheme, which implements the efficient allocation, exists and is progressive. \textit{Note that 1 is an upper bound on the tax scheme progressivity parameter as otherwise the marginal returns to labor would be decreasing in hours worked. It is positive since $s<\varepsilon$.}
\end{proof}

\tr{Is the tax scheme really progressive? }
\section{Results}
\begin{figure}[h!!]
	\centering
	\caption{Comparison to efficient allocation absent emission target }\label{fig:Compno_eff_BN0_notarget}
		\begin{minipage}[]{0.32\textwidth}
		\centering{\footnotesize{(a) Income tax progressivity, $\tau_{lt}$}}
		%	\captionsetup{width=.45\linewidth}
		\includegraphics[width=1\textwidth]{../../codding_model/own_basedOnFried/optimalPol_elastS_DisuSci/figures/all_1705/taul_CompEffOPT_NOT_NoTaus_spillover0_sep1_BN0_ineq0_red0_etaa0.79_lgd1.png}
	\end{minipage}
\begin{minipage}[]{0.32\textwidth}
\centering{\footnotesize{(a) Income tax progressivity, $\tau_{lt}$; No skill}}
%	\captionsetup{width=.45\linewidth}
\includegraphics[width=1\textwidth]{../../codding_model/own_basedOnFried/optimalPol_elastS_DisuSci/figures/all_1705/taul_CompEffOPT_NOT_NoTaus_spillover0_noskill1_sep1_BN0_ineq0_red0_etaa0.79_lgd1.png}
\end{minipage}
	\begin{minipage}[]{0.32\textwidth}
	\centering{\footnotesize{(b) Corrective tax, $\tau_{ft}$}}
	%	\captionsetup{width=.45\linewidth}
	\includegraphics[width=1\textwidth]{../../codding_model/own_basedOnFried/optimalPol_elastS_DisuSci/figures/all_1705/tauf_CompEffOPT_NOT_NoTaus_spillover0_sep1_BN0_ineq0_red0_etaa0.79_lgd0.png}
\end{minipage}
	\begin{minipage}[]{0.32\textwidth}
		\centering{\footnotesize{(c) High skill labour}}
		%	\captionsetup{width=.45\linewidth}
		\includegraphics[width=1\textwidth]{../../codding_model/own_basedOnFried/optimalPol_elastS_DisuSci/figures/all_1705/hh_CompEffOPT_NOT_NoTaus_spillover0_sep1_BN0_ineq0_red0_etaa0.79_lgd0.png}
	\end{minipage}
	\begin{minipage}[]{0.32\textwidth}
		\centering{\footnotesize{(d) Low skill labour}}
		%	\captionsetup{width=.45\linewidth}
		\includegraphics[width=1\textwidth]{../../codding_model/own_basedOnFried/optimalPol_elastS_DisuSci/figures/all_1705/hl_CompEffOPT_NOT_NoTaus_spillover0_sep1_BN0_ineq0_red0_etaa0.79_lgd0.png}
	\end{minipage}
	\begin{minipage}[]{0.32\textwidth}
		\centering{\footnotesize{(e) Consumption}}
		%	\captionsetup{width=.45\linewidth}
		\includegraphics[width=1\textwidth]{../../codding_model/own_basedOnFried/optimalPol_elastS_DisuSci/figures/all_1705/C_CompEffOPT_NOT_NoTaus_spillover0_sep1_BN0_ineq0_red0_etaa0.79_lgd0.png}
	\end{minipage}
%	\begin{minipage}[]{0.32\textwidth}
%		\centering{\footnotesize{(f) Technology green}}
%		%	\captionsetup{width=.45\linewidth}
%		\includegraphics[width=1\textwidth]{../../codding_model/own_basedOnFried/optimalPol_elastS_DisuSci/figures/all_1705/Ag_CompEffOPT_NOT_NoTaus_spillover0_sep1_BN0_ineq0_etaa0.79_lgd0.png}
%	\end{minipage}
%	\begin{minipage}[]{0.32\textwidth}
%		\centering{\footnotesize{(g)  Technology fossil}}
%		%	\captionsetup{width=.45\linewidth}
%		\includegraphics[width=1\textwidth]{../../codding_model/own_basedOnFried/optimalPol_elastS_DisuSci/figures/all_1705/Af_CompEffOPT_NOT_NoTaus_spillover0_sep1_BN0_ineq0_red0_etaa0.79_lgd0.png}
%	\end{minipage}
%	\begin{minipage}[]{0.32\textwidth}
%		\centering{\footnotesize{(h) Technology neutral}}
%		%	\captionsetup{width=.45\linewidth}
%		\includegraphics[width=1\textwidth]{../../codding_model/own_basedOnFried/optimalPol_elastS_DisuSci/figures/all_1705/An_CompEffOPT_NOT_NoTaus_spillover0_sep1_BN0_ineq0_red0_etaa0.79_lgd0.png}
%	\end{minipage}
	\begin{minipage}[]{0.32\textwidth}
		\centering{\footnotesize{(i) Scientists green}}
		%	\captionsetup{width=.45\linewidth}
		\includegraphics[width=1\textwidth]{../../codding_model/own_basedOnFried/optimalPol_elastS_DisuSci/figures/all_1705/sg_CompEffOPT_NOT_NoTaus_spillover0_sep1_BN0_ineq0_red0_etaa0.79_lgd0.png}
	\end{minipage}
	\begin{minipage}[]{0.32\textwidth}
		\centering{\footnotesize{(j) Scientists fossil}}
		%	\captionsetup{width=.45\linewidth}
		\includegraphics[width=1\textwidth]{../../codding_model/own_basedOnFried/optimalPol_elastS_DisuSci/figures/all_1705/sff_CompEffOPT_NOT_NoTaus_spillover0_sep1_BN0_ineq0_red0_etaa0.79_lgd0.png}
	\end{minipage}
	\begin{minipage}[]{0.32\textwidth}
		\centering{\footnotesize{(k) Scientists neutral}}
		%	\captionsetup{width=.45\linewidth}
		\includegraphics[width=1\textwidth]{../../codding_model/own_basedOnFried/optimalPol_elastS_DisuSci/figures/all_1705/sn_CompEffOPT_NOT_NoTaus_spillover0_sep1_BN0_ineq0_red0_etaa0.79_lgd0.png}
	\end{minipage}

\begin{minipage}[]{0.32\textwidth}
	\centering{\footnotesize{(i) Technology green}}
	%	\captionsetup{width=.45\linewidth}
	\includegraphics[width=1\textwidth]{../../codding_model/own_basedOnFried/optimalPol_elastS_DisuSci/figures/all_1705/Ag_CompEffOPT_NOT_NoTaus_spillover0_sep1_BN0_ineq0_red0_etaa0.79_lgd0.png}
\end{minipage}
\begin{minipage}[]{0.32\textwidth}
	\centering{\footnotesize{(j) Technology fossil}}
	%	\captionsetup{width=.45\linewidth}
	\includegraphics[width=1\textwidth]{../../codding_model/own_basedOnFried/optimalPol_elastS_DisuSci/figures/all_1705/Af_CompEffOPT_NOT_NoTaus_spillover0_sep1_BN0_ineq0_red0_etaa0.79_lgd0.png}
\end{minipage}
\begin{minipage}[]{0.32\textwidth}
	\centering{\footnotesize{(k) Technology neutral}}
	%	\captionsetup{width=.45\linewidth}
	\includegraphics[width=1\textwidth]{../../codding_model/own_basedOnFried/optimalPol_elastS_DisuSci/figures/all_1705/An_CompEffOPT_NOT_NoTaus_spillover0_sep1_BN0_ineq0_red0_etaa0.79_lgd0.png}
\end{minipage}
	\begin{minipage}[]{0.32\textwidth}
		\centering{\footnotesize{(l) labour green}}
		%	\captionsetup{width=.45\linewidth}
		\includegraphics[width=1\textwidth]{../../codding_model/own_basedOnFried/optimalPol_elastS_DisuSci/figures/all_1705/Lg_CompEffOPT_NOT_NoTaus_spillover0_sep1_BN0_ineq0_red0_etaa0.79_lgd0.png}
	\end{minipage}
	\begin{minipage}[]{0.32\textwidth}
		\centering{\footnotesize{(m) labour fossil}}
		%	\captionsetup{width=.45\linewidth}
		\includegraphics[width=1\textwidth]{../../codding_model/own_basedOnFried/optimalPol_elastS_DisuSci/figures/all_1705/Lf_CompEffOPT_NOT_NoTaus_spillover0_sep1_BN0_ineq0_red0_etaa0.79_lgd0.png}
	\end{minipage}
	\begin{minipage}[]{0.32\textwidth}
		\centering{\footnotesize{\ \\(l) labour neutral}}
		%	\captionsetup{width=.45\linewidth}
		\includegraphics[width=1\textwidth]{../../codding_model/own_basedOnFried/optimalPol_elastS_DisuSci/figures/all_1705/Ln_CompEffOPT_NOT_NoTaus_spillover0_sep1_BN0_ineq0_red0_etaa0.79_lgd0.png}
	\end{minipage}
	\begin{minipage}[]{0.32\textwidth}
		\centering{\footnotesize{(g) Final output}}
		%	\captionsetup{width=.45\linewidth}
		\includegraphics[width=1\textwidth]{../../codding_model/own_basedOnFried/optimalPol_elastS_DisuSci/figures/all_1705/Y_CompEffOPT_NOT_NoTaus_spillover0_sep1_BN0_ineq0_red0_etaa0.79_lgd0.png}
	\end{minipage}
	\begin{minipage}[]{0.32\textwidth}
		\centering{\footnotesize{(h) Energy output}}
		%	\captionsetup{width=.45\linewidth}
		\includegraphics[width=1\textwidth]{../../codding_model/own_basedOnFried/optimalPol_elastS_DisuSci/figures/all_1705/E_CompEffOPT_NOT_NoTaus_spillover0_sep1_BN0_ineq0_red0_etaa0.79_lgd0.png}
	\end{minipage}
	\begin{minipage}[]{0.32\textwidth}
		\centering{\footnotesize{(i) Neutral output}}
		%	\captionsetup{width=.45\linewidth}
		\includegraphics[width=1\textwidth]{../../codding_model/own_basedOnFried/optimalPol_elastS_DisuSci/figures/all_1705/N_CompEffOPT_NOT_NoTaus_spillover0_sep1_BN0_ineq0_red0_etaa0.79_lgd0.png}
	\end{minipage}
\end{figure}

\begin{figure}[h!!]
	\centering
	\caption{Comparison to efficient allocation absent emission target without skill }\label{fig:Compno_eff_BN0_notarget_noskill}
	\begin{minipage}[]{0.32\textwidth}
		\centering{\footnotesize{(a) Income tax progressivity, $\tau_{lt}$; No skill}}
		%	\captionsetup{width=.45\linewidth}
		\includegraphics[width=1\textwidth]{../../codding_model/own_basedOnFried/optimalPol_elastS_DisuSci/figures/all_1705/taul_CompEffOPT_NOT_NoTaus_spillover0_noskill1_sep1_BN0_ineq0_red0_etaa0.79_lgd1.png}
	\end{minipage}
	\begin{minipage}[]{0.32\textwidth}
		\centering{\footnotesize{(b) Corrective tax, $\tau_{ft}$}}
		%	\captionsetup{width=.45\linewidth}
		\includegraphics[width=1\textwidth]{../../codding_model/own_basedOnFried/optimalPol_elastS_DisuSci/figures/all_1705/tauf_CompEffOPT_NOT_NoTaus_spillover0_noskill1_sep1_BN0_ineq0_red0_etaa0.79_lgd0.png}
	\end{minipage}
	\begin{minipage}[]{0.32\textwidth}
		\centering{\footnotesize{(c) High skill labour}}
		%	\captionsetup{width=.45\linewidth}
		\includegraphics[width=1\textwidth]{../../codding_model/own_basedOnFried/optimalPol_elastS_DisuSci/figures/all_1705/hh_CompEffOPT_NOT_NoTaus_spillover0_noskill1_sep1_BN0_ineq0_red0_etaa0.79_lgd0.png}
	\end{minipage}
	\begin{minipage}[]{0.32\textwidth}
		\centering{\footnotesize{(d) Low skill labour}}
		%	\captionsetup{width=.45\linewidth}
		\includegraphics[width=1\textwidth]{../../codding_model/own_basedOnFried/optimalPol_elastS_DisuSci/figures/all_1705/hl_CompEffOPT_NOT_NoTaus_spillover0_noskill1_sep1_BN0_ineq0_red0_etaa0.79_lgd0.png}
	\end{minipage}
	\begin{minipage}[]{0.32\textwidth}
		\centering{\footnotesize{(e) Consumption}}
		%	\captionsetup{width=.45\linewidth}
		\includegraphics[width=1\textwidth]{../../codding_model/own_basedOnFried/optimalPol_elastS_DisuSci/figures/all_1705/C_CompEffOPT_NOT_NoTaus_spillover0_noskill1_sep1_BN0_ineq0_red0_etaa0.79_lgd0.png}
	\end{minipage}
	\begin{minipage}[]{0.32\textwidth}
		\centering{\footnotesize{(i) Scientists green}}
		%	\captionsetup{width=.45\linewidth}
		\includegraphics[width=1\textwidth]{../../codding_model/own_basedOnFried/optimalPol_elastS_DisuSci/figures/all_1705/sg_CompEffOPT_NOT_NoTaus_spillover0_noskill1_sep1_BN0_ineq0_red0_etaa0.79_lgd0.png}
	\end{minipage}
	\begin{minipage}[]{0.32\textwidth}
		\centering{\footnotesize{(j) Scientists fossil}}
		%	\captionsetup{width=.45\linewidth}
		\includegraphics[width=1\textwidth]{../../codding_model/own_basedOnFried/optimalPol_elastS_DisuSci/figures/all_1705/sff_CompEffOPT_NOT_NoTaus_spillover0_noskill1_sep1_BN0_ineq0_red0_etaa0.79_lgd0.png}
	\end{minipage}
	\begin{minipage}[]{0.32\textwidth}
		\centering{\footnotesize{(k) Scientists neutral}}
		%	\captionsetup{width=.45\linewidth}
		\includegraphics[width=1\textwidth]{../../codding_model/own_basedOnFried/optimalPol_elastS_DisuSci/figures/all_1705/sn_CompEffOPT_NOT_NoTaus_spillover0_noskill1_sep1_BN0_ineq0_red0_etaa0.79_lgd0.png}
	\end{minipage}
	\begin{minipage}[]{0.32\textwidth}
		\centering{\footnotesize{(i) Technology green}}
		%	\captionsetup{width=.45\linewidth}
		\includegraphics[width=1\textwidth]{../../codding_model/own_basedOnFried/optimalPol_elastS_DisuSci/figures/all_1705/Ag_CompEffOPT_NOT_NoTaus_spillover0_noskill1_sep1_BN0_ineq0_red0_etaa0.79_lgd0.png}
	\end{minipage}
	\begin{minipage}[]{0.32\textwidth}
		\centering{\footnotesize{(j) Technology fossil}}
		%	\captionsetup{width=.45\linewidth}
		\includegraphics[width=1\textwidth]{../../codding_model/own_basedOnFried/optimalPol_elastS_DisuSci/figures/all_1705/Af_CompEffOPT_NOT_NoTaus_spillover0_noskill1_sep1_BN0_ineq0_red0_etaa0.79_lgd0.png}
	\end{minipage}
	\begin{minipage}[]{0.32\textwidth}
		\centering{\footnotesize{(k) Technology neutral}}
		%	\captionsetup{width=.45\linewidth}
		\includegraphics[width=1\textwidth]{../../codding_model/own_basedOnFried/optimalPol_elastS_DisuSci/figures/all_1705/An_CompEffOPT_NOT_NoTaus_spillover0_noskill1_sep1_BN0_ineq0_red0_etaa0.79_lgd0.png}
	\end{minipage}
	\begin{minipage}[]{0.32\textwidth}
		\centering{\footnotesize{(l) labour green}}
		%	\captionsetup{width=.45\linewidth}
		\includegraphics[width=1\textwidth]{../../codding_model/own_basedOnFried/optimalPol_elastS_DisuSci/figures/all_1705/Lg_CompEffOPT_NOT_NoTaus_spillover0_noskill1_sep1_BN0_ineq0_red0_etaa0.79_lgd0.png}
	\end{minipage}
	\begin{minipage}[]{0.32\textwidth}
		\centering{\footnotesize{(m) labour fossil}}
		%	\captionsetup{width=.45\linewidth}
		\includegraphics[width=1\textwidth]{../../codding_model/own_basedOnFried/optimalPol_elastS_DisuSci/figures/all_1705/Lf_CompEffOPT_NOT_NoTaus_spillover0_noskill1_sep1_BN0_ineq0_red0_etaa0.79_lgd0.png}
	\end{minipage}
	\begin{minipage}[]{0.32\textwidth}
		\centering{\footnotesize{\ \\(l) labour neutral}}
		%	\captionsetup{width=.45\linewidth}
		\includegraphics[width=1\textwidth]{../../codding_model/own_basedOnFried/optimalPol_elastS_DisuSci/figures/all_1705/Ln_CompEffOPT_NOT_NoTaus_spillover0_noskill1_sep1_BN0_ineq0_red0_etaa0.79_lgd0.png}
	\end{minipage}
	\begin{minipage}[]{0.32\textwidth}
		\centering{\footnotesize{(g) Final output}}
		%	\captionsetup{width=.45\linewidth}
		\includegraphics[width=1\textwidth]{../../codding_model/own_basedOnFried/optimalPol_elastS_DisuSci/figures/all_1705/Y_CompEffOPT_NOT_NoTaus_spillover0_noskill1_sep1_BN0_ineq0_red0_etaa0.79_lgd0.png}
	\end{minipage}
	\begin{minipage}[]{0.32\textwidth}
		\centering{\footnotesize{(h) Energy output}}
		%	\captionsetup{width=.45\linewidth}
		\includegraphics[width=1\textwidth]{../../codding_model/own_basedOnFried/optimalPol_elastS_DisuSci/figures/all_1705/E_CompEffOPT_NOT_NoTaus_spillover0_noskill1_sep1_BN0_ineq0_red0_etaa0.79_lgd0.png}
	\end{minipage}
	\begin{minipage}[]{0.32\textwidth}
		\centering{\footnotesize{(i) Neutral output}}
		%	\captionsetup{width=.45\linewidth}
		\includegraphics[width=1\textwidth]{../../codding_model/own_basedOnFried/optimalPol_elastS_DisuSci/figures/all_1705/N_CompEffOPT_NOT_NoTaus_spillover0_sep1_BN0_ineq0_red0_etaa0.79_lgd0.png}
	\end{minipage}
\end{figure}


%---- no skill heterogeneity with target
\begin{figure}[h!!]
	\centering
	\caption{Comparison to efficient allocation without skill heterogeneity }\label{fig:Compno_eff_noskill}
		\begin{minipage}[]{0.32\textwidth}
		\centering{\footnotesize{(b) Income tax progressivity}}
		%	\captionsetup{width=.45\linewidth}
		\includegraphics[width=1\textwidth]{../../codding_model/own_basedOnFried/optimalPol_elastS_DisuSci/figures/all_1705/taul_CompEffOPT_T_NoTaus_spillover0_noskill1_sep1_BN0_ineq0_red0_etaa0.79_lgd0.png}
	\end{minipage}
	\begin{minipage}[]{0.32\textwidth}
	\centering{\footnotesize{(b) Fossil tax}}
	%	\captionsetup{width=.45\linewidth}
	\includegraphics[width=1\textwidth]{../../codding_model/own_basedOnFried/optimalPol_elastS_DisuSci/figures/all_1705/tauf_CompEffOPT_T_NoTaus_spillover0_noskill1_sep1_BN0_ineq0_red0_etaa0.79_lgd0.png}
\end{minipage}
\begin{minipage}[]{0.32\textwidth}
	\centering{\footnotesize{(b) fossil price}}
	%	\captionsetup{width=.45\linewidth}
	\includegraphics[width=1\textwidth]{../../codding_model/own_basedOnFried/optimalPol_elastS_DisuSci/figures/all_1705/pf_CompEffOPT_T_NoTaus_spillover0_noskill1_sep1_BN0_ineq0_red0_etaa0.79_lgd0.png}
\end{minipage}
	\begin{minipage}[]{0.32\textwidth}
	\centering{\footnotesize{(b) Utility labour}}
	%	\captionsetup{width=.45\linewidth}
	\includegraphics[width=1\textwidth]{../../codding_model/own_basedOnFried/optimalPol_elastS_DisuSci/figures/all_1705/Utillab_CompEffOPT_T_NoTaus_spillover0_noskill1_sep1_BN0_ineq0_red0_etaa0.79_lgd0.png}
\end{minipage}
\begin{minipage}[]{0.32\textwidth}
	\centering{\footnotesize{(b) Utility Consumption}}
	%	\captionsetup{width=.45\linewidth}
	\includegraphics[width=1\textwidth]{../../codding_model/own_basedOnFried/optimalPol_elastS_DisuSci/figures/all_1705/Utilcon_CompEffOPT_T_NoTaus_spillover0_noskill1_sep1_BN0_ineq0_red0_etaa0.79_lgd0.png}
\end{minipage}
\begin{minipage}[]{0.32\textwidth}
	\centering{\footnotesize{(b) Utility scientists}}
	%	\captionsetup{width=.45\linewidth}
	\includegraphics[width=1\textwidth]{../../codding_model/own_basedOnFried/optimalPol_elastS_DisuSci/figures/all_1705/UtilSci_CompEffOPT_T_NoTaus_spillover0_noskill1_sep1_BN0_ineq0_red0_etaa0.79_lgd0.png}
\end{minipage}
\begin{minipage}[]{0.32\textwidth}
\centering{\footnotesize{(b) green price}}
%	\captionsetup{width=.45\linewidth}
\includegraphics[width=1\textwidth]{../../codding_model/own_basedOnFried/optimalPol_elastS_DisuSci/figures/all_1705/pg_CompEffOPT_T_NoTaus_spillover0_noskill1_sep1_BN0_ineq0_red0_etaa0.79_lgd0.png}
\end{minipage}
\begin{minipage}[]{0.32\textwidth}
	\centering{\footnotesize{(b) fossil labour}}
	%	\captionsetup{width=.45\linewidth}
	\includegraphics[width=1\textwidth]{../../codding_model/own_basedOnFried/optimalPol_elastS_DisuSci/figures/all_1705/Lf_CompEffOPT_T_NoTaus_spillover0_noskill1_sep1_BN0_ineq0_red0_etaa0.79_lgd0.png}
\end{minipage}
\begin{minipage}[]{0.32\textwidth}
	\centering{\footnotesize{(b) green labour}}
	%	\captionsetup{width=.45\linewidth}
	\includegraphics[width=1\textwidth]{../../codding_model/own_basedOnFried/optimalPol_elastS_DisuSci/figures/all_1705/Lg_CompEffOPT_T_NoTaus_spillover0_noskill1_sep1_BN0_ineq0_red0_etaa0.79_lgd0.png}
\end{minipage}
\begin{minipage}[]{0.32\textwidth}
	\centering{\footnotesize{(b) wage rate}}
	%	\captionsetup{width=.45\linewidth}
	\includegraphics[width=1\textwidth]{../../codding_model/own_basedOnFried/optimalPol_elastS_DisuSci/figures/all_1705/wh_CompEffOPT_T_NoTaus_spillover0_noskill1_sep1_BN0_ineq0_red0_etaa0.79_lgd0.png}
\end{minipage}
	\begin{minipage}[]{0.32\textwidth}
		\centering{\footnotesize{(b) Hours worked}}
		%	\captionsetup{width=.45\linewidth}
		\includegraphics[width=1\textwidth]{../../codding_model/own_basedOnFried/optimalPol_elastS_DisuSci/figures/all_1705/hh_CompEffOPT_T_NoTaus_spillover0_noskill1_sep1_BN0_ineq0_red0_etaa0.79_lgd0.png}
	\end{minipage}
	\begin{minipage}[]{0.32\textwidth}
		\centering{\footnotesize{(d) Consumption}}
		%	\captionsetup{width=.45\linewidth}
		\includegraphics[width=1\textwidth]{../../codding_model/own_basedOnFried/optimalPol_elastS_DisuSci/figures/all_1705/C_CompEffOPT_T_NoTaus_spillover0_noskill1_sep1_BN0_ineq0_red0_etaa0.79_lgd0.png}
	\end{minipage}
	\begin{minipage}[]{0.32\textwidth}
		\centering{\footnotesize{\ \\(e)  Technology fossil}}
		%	\captionsetup{width=.45\linewidth}
		\includegraphics[width=1\textwidth]{../../codding_model/own_basedOnFried/optimalPol_elastS_DisuSci/figures/all_1705/Af_CompEffOPT_T_NoTaus_spillover0_noskill1_sep1_BN0_ineq0_red0_etaa0.79_lgd0.png}
	\end{minipage}
	\begin{minipage}[]{0.32\textwidth}
		\centering{\footnotesize{\ \\(f) Technology green}}
		%	\captionsetup{width=.45\linewidth}
		\includegraphics[width=1\textwidth]{../../codding_model/own_basedOnFried/optimalPol_elastS_DisuSci/figures/all_1705/Ag_CompEffOPT_T_NoTaus_spillover0_noskill1_sep1_BN0_ineq0_red0_etaa0.79_lgd0.png}
	\end{minipage}
	\begin{minipage}[]{0.32\textwidth}
		\centering{\footnotesize{\ \\(g) Technology neutral}}
		%	\captionsetup{width=.45\linewidth}
		\includegraphics[width=1\textwidth]{../../codding_model/own_basedOnFried/optimalPol_elastS_DisuSci/figures/all_1705/An_CompEffOPT_T_NoTaus_spillover0_noskill1_sep1_BN0_ineq0_red0_etaa0.79_lgd0.png}
	\end{minipage}
	\begin{minipage}[]{0.32\textwidth}
		\centering{\footnotesize{\ \\(h) Green scientists}}
		%	\captionsetup{width=.45\linewidth}
		\includegraphics[width=1\textwidth]{../../codding_model/own_basedOnFried/optimalPol_elastS_DisuSci/figures/all_1705/sg_CompEffOPT_T_NoTaus_spillover0_noskill1_sep1_BN0_ineq0_red0_etaa0.79_lgd0.png}
	\end{minipage}
	\begin{minipage}[]{0.32\textwidth}
		\centering{\footnotesize{\ \\(i) Fossil scientists}}
		%	\captionsetup{width=.45\linewidth}
		\includegraphics[width=1\textwidth]{../../codding_model/own_basedOnFried/optimalPol_elastS_DisuSci/figures/all_1705/sff_CompEffOPT_T_NoTaus_spillover0_noskill1_sep1_BN0_ineq0_red0_etaa0.79_lgd0.png}
	\end{minipage}
	%	\begin{minipage}[]{0.32\textwidth}
	%		\centering{\footnotesize{\ \\(h) Scientists fossil}}
	%		%	\captionsetup{width=.45\linewidth}
	%		\includegraphics[width=1\textwidth]{../../codding_model/own_basedOnFried/optimalPol_elastS_DisuSci/figures/all_1705/sff_CompEffOPT_T_NoTaus_spillover0_sep1_BN0_ineq0_red0_etaa0.79_lgd0.png}
	%	\end{minipage}
	%	\begin{minipage}[]{0.32\textwidth}
	%		\centering{\footnotesize{\ \\(i) Scientists neutral}}
	%		%	\captionsetup{width=.45\linewidth}
	%		\includegraphics[width=1\textwidth]{../../codding_model/own_basedOnFried/optimalPol_elastS_DisuSci/figures/all_1705/sn_CompEffOPT_T_NoTaus_spillover0_sep1_BN0_ineq0_red0_etaa0.79_lgd0.png}
	%	\end{minipage}
	\begin{minipage}[]{0.32\textwidth}
		\centering{\footnotesize{\ \\(j) labour green}}
		%	\captionsetup{width=.45\linewidth}
		\includegraphics[width=1\textwidth]{../../codding_model/own_basedOnFried/optimalPol_elastS_DisuSci/figures/all_1705/Lg_CompEffOPT_T_NoTaus_spillover0_noskill1_sep1_BN0_ineq0_red0_etaa0.79_lgd0.png}
	\end{minipage}
	\begin{minipage}[]{0.32\textwidth}
		\centering{\footnotesize{\ \\(k) labour fossil}}
		%	\captionsetup{width=.45\linewidth}
		\includegraphics[width=1\textwidth]{../../codding_model/own_basedOnFried/optimalPol_elastS_DisuSci/figures/all_1705/Lf_CompEffOPT_T_NoTaus_spillover0_noskill1_sep1_BN0_ineq0_red0_etaa0.79_lgd0.png}
	\end{minipage}
	\begin{minipage}[]{0.32\textwidth}
		\centering{\footnotesize{\ \\(l) labour neutral}}
		%	\captionsetup{width=.45\linewidth}
		\includegraphics[width=1\textwidth]{../../codding_model/own_basedOnFried/optimalPol_elastS_DisuSci/figures/all_1705/Ln_CompEffOPT_T_NoTaus_spillover0_noskill1_sep1_BN0_ineq0_red0_etaa0.79_lgd0.png}
	\end{minipage}
\end{figure}



\begin{figure}[h!!]
	\centering
	\caption{Comparison to efficient allocation without target but externality }\label{fig:Compno_eff_extern}
	\begin{minipage}[]{0.32\textwidth}
		\centering{\footnotesize{(b) Income tax progressivity}}
		%	\captionsetup{width=.45\linewidth}
		\includegraphics[width=1\textwidth]{../../codding_model/own_basedOnFried/optimalPol_elastS_DisuSci/figures/all_1705/Extern_CompEff_taul_spillover0_noskill0_sep1_BN0_ineq0_red0_etaa0.79_lgd1.png}
	\end{minipage}
	\begin{minipage}[]{0.32\textwidth}
		\centering{\footnotesize{(b) Fossil tax}}
		%	\captionsetup{width=.45\linewidth}
		\includegraphics[width=1\textwidth]{../../codding_model/own_basedOnFried/optimalPol_elastS_DisuSci/figures/all_1705/Extern_CompEff_tauf_spillover0_noskill0_sep1_BN0_ineq0_red0_etaa0.79_lgd0.png}
	\end{minipage}
	\begin{minipage}[]{0.32\textwidth}
		\centering{\footnotesize{(b) fossil price}}
		%	\captionsetup{width=.45\linewidth}
		\includegraphics[width=1\textwidth]{../../codding_model/own_basedOnFried/optimalPol_elastS_DisuSci/figures/all_1705/Extern_CompEff_pf_spillover0_noskill0_sep1_BN0_ineq0_red0_etaa0.79_lgd0.png}
	\end{minipage}
	\begin{minipage}[]{0.32\textwidth}
		\centering{\footnotesize{(b) Utility labour}}
		%	\captionsetup{width=.45\linewidth}
		\includegraphics[width=1\textwidth]{../../codding_model/own_basedOnFried/optimalPol_elastS_DisuSci/figures/all_1705/Extern_CompEff_Utillab_spillover0_noskill0_sep1_BN0_ineq0_red0_etaa0.79_lgd0.png}
	\end{minipage}
	\begin{minipage}[]{0.32\textwidth}
		\centering{\footnotesize{(b) Utility Consumption}}
		%	\captionsetup{width=.45\linewidth}
		\includegraphics[width=1\textwidth]{../../codding_model/own_basedOnFried/optimalPol_elastS_DisuSci/figures/all_1705/Extern_CompEff_UtilCon_spillover0_noskill0_sep1_BN0_ineq0_red0_etaa0.79_lgd0.png}
	\end{minipage}
\end{figure}
\begin{figure}[h!!]
	\centering
	\caption{Comparison to efficient allocation without target but externality }\label{fig:Compno_eff_extern22}
	\begin{minipage}[]{0.32\textwidth}
		\centering{\footnotesize{(b) Utility scientists}}
		%	\captionsetup{width=.45\linewidth}
		\includegraphics[width=1\textwidth]{../../codding_model/own_basedOnFried/optimalPol_elastS_DisuSci/figures/all_1705/Extern_CompEff_UtilSci_spillover0_noskill0_sep1_BN0_ineq0_red0_etaa0.79_lgd0.png}
	\end{minipage}
	\begin{minipage}[]{0.32\textwidth}
		\centering{\footnotesize{(b) green price}}
		%	\captionsetup{width=.45\linewidth}
		\includegraphics[width=1\textwidth]{../../codding_model/own_basedOnFried/optimalPol_elastS_DisuSci/figures/all_1705/Extern_CompEff_pg_spillover0_noskill0_sep1_BN0_ineq0_red0_etaa0.79_lgd0.png}
	\end{minipage}
	\begin{minipage}[]{0.32\textwidth}
		\centering{\footnotesize{(b) fossil labour}}
		%	\captionsetup{width=.45\linewidth}
		\includegraphics[width=1\textwidth]{../../codding_model/own_basedOnFried/optimalPol_elastS_DisuSci/figures/all_1705/Extern_CompEff_Lf_spillover0_noskill0_sep1_BN0_ineq0_red0_etaa0.79_lgd0.png}
	\end{minipage}
	\begin{minipage}[]{0.32\textwidth}
		\centering{\footnotesize{(b) green labour}}
		%	\captionsetup{width=.45\linewidth}
		\includegraphics[width=1\textwidth]{../../codding_model/own_basedOnFried/optimalPol_elastS_DisuSci/figures/all_1705/Extern_CompEff_Lg_spillover0_noskill0_sep1_BN0_ineq0_red0_etaa0.79_lgd0.png}
	\end{minipage}
\begin{minipage}[]{0.32\textwidth}
\centering{\footnotesize{\ \\(l) labour neutral}}
%	\captionsetup{width=.45\linewidth}
\includegraphics[width=1\textwidth]{../../codding_model/own_basedOnFried/optimalPol_elastS_DisuSci/figures/all_1705/Extern_CompEff_Ln_spillover0_noskill0_sep1_BN0_ineq0_red0_etaa0.79_lgd0.png}
\end{minipage}
	\begin{minipage}[]{0.32\textwidth}
		\centering{\footnotesize{(b) wage rate high skill}}
		%	\captionsetup{width=.45\linewidth}
		\includegraphics[width=1\textwidth]{../../codding_model/own_basedOnFried/optimalPol_elastS_DisuSci/figures/all_1705/Extern_CompEff_wh_spillover0_noskill0_sep1_BN0_ineq0_red0_etaa0.79_lgd0.png}
	\end{minipage}
\begin{minipage}[]{0.32\textwidth}
	\centering{\footnotesize{(b) wage rate low skill}}
	%	\captionsetup{width=.45\linewidth}
	\includegraphics[width=1\textwidth]{../../codding_model/own_basedOnFried/optimalPol_elastS_DisuSci/figures/all_1705/Extern_CompEff_wl_spillover0_noskill0_sep1_BN0_ineq0_red0_etaa0.79_lgd0.png}
\end{minipage}
\begin{minipage}[]{0.32\textwidth}
	\centering{\footnotesize{(b) Hours high skill}}
	%	\captionsetup{width=.45\linewidth}
	\includegraphics[width=1\textwidth]{../../codding_model/own_basedOnFried/optimalPol_elastS_DisuSci/figures/all_1705/Extern_CompEff_hh_spillover0_noskill0_sep1_BN0_ineq0_red0_etaa0.79_lgd0.png}
\end{minipage}
	\begin{minipage}[]{0.32\textwidth}
		\centering{\footnotesize{(b) Hours low skill}}
		%	\captionsetup{width=.45\linewidth}
		\includegraphics[width=1\textwidth]{../../codding_model/own_basedOnFried/optimalPol_elastS_DisuSci/figures/all_1705/Extern_CompEff_hl_spillover0_noskill0_sep1_BN0_ineq0_red0_etaa0.79_lgd0.png}
	\end{minipage}
	\begin{minipage}[]{0.32\textwidth}
		\centering{\footnotesize{(d) Consumption}}
		%	\captionsetup{width=.45\linewidth}
		\includegraphics[width=1\textwidth]{../../codding_model/own_basedOnFried/optimalPol_elastS_DisuSci/figures/all_1705/Extern_CompEff_C_spillover0_noskill0_sep1_BN0_ineq0_red0_etaa0.79_lgd0.png}
	\end{minipage}
\end{figure}

\begin{figure}[h!!]
	\centering
	\caption{Comparison to efficient allocation without target but externality 2}\label{fig:Compno_eff_extern2}
	\begin{minipage}[]{0.32\textwidth}
	\centering{\footnotesize{\ \\(e)  Technology fossil}}
	%	\captionsetup{width=.45\linewidth}
	\includegraphics[width=1\textwidth]{../../codding_model/own_basedOnFried/optimalPol_elastS_DisuSci/figures/all_1705/Extern_CompEff_Af_spillover0_noskill0_sep1_BN0_ineq0_red0_etaa0.79_lgd0.png}
\end{minipage}
\begin{minipage}[]{0.32\textwidth}
	\centering{\footnotesize{\ \\(f) Technology green}}
	%	\captionsetup{width=.45\linewidth}
	\includegraphics[width=1\textwidth]{../../codding_model/own_basedOnFried/optimalPol_elastS_DisuSci/figures/all_1705/Extern_CompEff_Ag_spillover0_noskill0_sep1_BN0_ineq0_red0_etaa0.79_lgd0.png}
\end{minipage}
\begin{minipage}[]{0.32\textwidth}
	\centering{\footnotesize{\ \\(g) Technology neutral}}
	%	\captionsetup{width=.45\linewidth}
	\includegraphics[width=1\textwidth]{../../codding_model/own_basedOnFried/optimalPol_elastS_DisuSci/figures/all_1705/Extern_CompEff_An_spillover0_noskill0_sep1_BN0_ineq0_red0_etaa0.79_lgd0.png}
\end{minipage}
\begin{minipage}[]{0.32\textwidth}
	\centering{\footnotesize{\ \\(h) Green scientists}}
	%	\captionsetup{width=.45\linewidth}
	\includegraphics[width=1\textwidth]{../../codding_model/own_basedOnFried/optimalPol_elastS_DisuSci/figures/all_1705/Extern_CompEff_sg_spillover0_noskill0_sep1_BN0_ineq0_red0_etaa0.79_lgd0.png}
\end{minipage}
\begin{minipage}[]{0.32\textwidth}
	\centering{\footnotesize{\ \\(i) Fossil scientists}}
	%	\captionsetup{width=.45\linewidth}
	\includegraphics[width=1\textwidth]{../../codding_model/own_basedOnFried/optimalPol_elastS_DisuSci/figures/all_1705/Extern_CompEff_sff_spillover0_noskill0_sep1_BN0_ineq0_red0_etaa0.79_lgd0.png}
\end{minipage}

\end{figure}

\begin{figure}[h!!]
	\centering
	\caption{Comparison to efficient allocation without target but externality 2}\label{fig:Compno_eff_Growth}
	\begin{minipage}[]{0.32\textwidth}
		\centering{\footnotesize{\ \\(e)  Technology fossil}}
		%	\captionsetup{width=.45\linewidth}
		\includegraphics[width=1\textwidth]{../../codding_model/own_basedOnFried/optimalPol_elastS_DisuSci/figures/all_1705/Extern_CompEff_Af_spillover0_noskill0_sep1_BN0_ineq0_red0_etaa0.79_lgd0.png}
	\end{minipage}
	\begin{minipage}[]{0.32\textwidth}
		\centering{\footnotesize{\ \\(f) Technology green}}
		%	\captionsetup{width=.45\linewidth}
		\includegraphics[width=1\textwidth]{../../codding_model/own_basedOnFried/optimalPol_elastS_DisuSci/figures/all_1705/Extern_CompEff_Ag_spillover0_noskill0_sep1_BN0_ineq0_red0_etaa0.79_lgd0.png}
	\end{minipage}
	\begin{minipage}[]{0.32\textwidth}
		\centering{\footnotesize{\ \\(g) Technology neutral}}
		%	\captionsetup{width=.45\linewidth}
		\includegraphics[width=1\textwidth]{../../codding_model/own_basedOnFried/optimalPol_elastS_DisuSci/figures/all_1705/Extern_CompEff_An_spillover0_noskill0_sep1_BN0_ineq0_red0_etaa0.79_lgd0.png}
	\end{minipage}
	\begin{minipage}[]{0.32\textwidth}
		\centering{\footnotesize{\ \\(h) Green scientists}}
		%	\captionsetup{width=.45\linewidth}
		\includegraphics[width=1\textwidth]{../../codding_model/own_basedOnFried/optimalPol_elastS_DisuSci/figures/all_1705/Extern_CompEff_sg_spillover0_noskill0_sep1_BN0_ineq0_red0_etaa0.79_lgd0.png}
	\end{minipage}
	\begin{minipage}[]{0.32\textwidth}
		\centering{\footnotesize{\ \\(i) Fossil scientists}}
		%	\captionsetup{width=.45\linewidth}
		\includegraphics[width=1\textwidth]{../../codding_model/own_basedOnFried/optimalPol_elastS_DisuSci/figures/all_1705/Extern_CompEff_sff_spillover0_noskill0_sep1_BN0_ineq0_red0_etaa0.79_lgd0.png}
	\end{minipage}
	
\end{figure}