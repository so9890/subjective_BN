\section{Quantitative results}\label{sec:res}

\tr{integrated policy has an advantage even absent externality when there is endogenous growth!}
In this section, I present and discuss the quantitative results.
Subsection \ref{subsec:mr} depicts the optimal policy under the baseline policy regime: environmental tax revenues are consumed by the government and an income tax is available. Subsection \ref{subsec:dis} discusses the results in comparison to the efficient allocation focusing on the role of labor income taxes.
%I focus on analyzing the mechanisms and welfare benefits from integrating the income tax scheme into the environmental policy. I also discuss the costs of not using lump-sum transfers.


\subsection{Results}\label{subsec:mr}
\tr{To be adjusted to new numbers}
%This section depicts results on the optimal policy followed by the implied allocation in the benchmark model where environmental tax revenues are redistributed via the income tax scheme. 

\begin{figure}[h!!]
	\centering
	\caption{Optimal Policy }\label{fig:optPol}
	\begin{minipage}[]{0.4\textwidth}
		\centering{\footnotesize{(a) Income tax progressivity, $\tau_{\iota t}$}}
		%	\captionsetup{width=.45\linewidth}
		\includegraphics[width=1\textwidth]{../../codding_model/own_basedOnFried/optimalPol_190722_tidiedUp/figures/all_10Aout22/Single_OPT_T_NoTaus_taul_regime3_spillover0_noskill0_sep1_xgrowth0_extern0_etaa0.79.png}
	\end{minipage}
	\begin{minipage}[]{0.1\textwidth}
		\
	\end{minipage}
	\begin{minipage}[]{0.4\textwidth}
		\centering{\footnotesize{(b) Environmental tax, $\tau_{ft}$ }}
		%	\captionsetup{width=.45\linewidth}
		\includegraphics[width=1\textwidth]{../../codding_model/own_basedOnFried/optimalPol_190722_tidiedUp/figures/all_10Aout22/Single_OPT_T_NoTaus_tauf_regime3_spillover0_noskill0_sep1_xgrowth0_extern0_etaa0.79.png}
	\end{minipage}
\end{figure} 

\subsubsection{Optimal policy}
To meet the emission limits suggested by the IPCC, the optimal income tax is progressive for all periods; see panel (a) in figure \ref{fig:optPol}.  The x-axis indicates the first year of the 5 year period to which the variable value corresponds. 
% optimal taul over time:   
%  0.1042    0.0977    0.0926    0.0875    0.0838    0.0804    0.0708    0.0683    0.0661    0.0641    0.0622    0.0665

The optimal income tax progressivity is decreasing over time. Starting from a value of 0.104 in 2020 it steadily decreases until the net-zero emission limit is implimented in 2050. Then, the progressivity parameter displays some discontinuity dropping from 0.08 to 0.071 from which it continues to decline to 0.067 in 2070; yet at a lower rate than before 2050. 
Overall, the optimal tax progressivity is approximately  around half the size found for the US in \cite{Heathcote2017OptimalFramework}: $\tau_{l}=0.181$.

%   optimal tauf
%   0.8598    0.8662    0.8727    0.8907    0.8967    0.9023    0.9481    0.9500    0.9519    0.9537    0.9555    0.9570
Consider panel (b). The optimal fossil tax is increasing over the period considered and jumps to higher levels when emission limits become stricter, i.e. in 2035 and in 2050.
In 2020, the environmental tax equals 86\% of the fossil sector's revenues, from where it rises steadily to 87\% in 2030.  In 2035, the environmental tax jumps to around 89\% as the emission target is to reduce emissions by 50\% relative to 2019. Over the years from 2035 to 2045 the planner gradually increases the fossil tax to 90\%. As the emission limit declines to net-zero in 2050, the tax rapidly surges to 95\% and gradually increases afterwards reaching 96\% in 2070. 

\subsubsection{Allocation}
Figure \ref{fig:optAll} depicts the optimal allocation. Limiting emissions in line with the Paris Agreement is concomitant with both a reduction and recomposition of consumption and production over time. 
Panel (a) shows consumption which reduces significantly when new emission limits become active, in 2030 and in 2050, but starting from the new low levels it growths modestly. Labor effort of both skill types reduces as stricter emission targets are enforced with the exception of high skill which slightly increases as the net-zero limit becomes binding; panel (b). The rise in high-skilled hours can be explained by the drop in income tax progressivity. Hours of low-skill workers appear constant over time after the initial reduction in 2030.  In comparison to hours supplied by low-skilled workers, high-skilled workers reduce hours more as the first emission limit gets introduced in 2030; compare panel (c) which shows the ratio of hours worked by high to low skill workers. Yet, the small rise in high-skill hours in 2050 causes an increase of the high-to-low skill ratio. The increase is decaying in the subsequent years.

The rise in consumption after each reduction is driven by technological progress in all sectors; compare panel (d) which shows growth rates by sector and as aggregate in per cent. 
The green sector sees a rise in technological progress, the dashed black line, while growth in the fossil and the non-energy sector is positive, yet diminishing over time. Overall, aggregate growth is positive but decreasing; compare the grey dashed graph. 
Summing up the last two paragraphs, the emission target is best achieved with more leisure at higher technology levels in all sectors. 

The optimal allocation of scientists over time nicely captures the combination of reductive and recomposing policies; see panel (e). There is a recomposition towards the green sector: while research in the non-energy and the fossil sector decrease over time, green research effort rises. Yet, overall, the amount of scientists reduces; compare the gray graph which depicts the sum of researchers across sectors.  
Finally, labor input goods are redirected towards the green sector; see panel (f). 

\begin{figure}[h!!]
	\centering
	\caption{Optimal Allocation }\label{fig:optAll}
	
	
	\begin{minipage}[]{0.32\textwidth}
		\centering{\footnotesize{(a) Consumption}}
		%	\captionsetup{width=.45\linewidth}
		\includegraphics[width=1\textwidth]{../../codding_model/own_basedOnFried/optimalPol_190722_tidiedUp/figures/all_10Aout22/Single_OPT_T_NoTaus_C_regime3_spillover0_noskill0_sep1_xgrowth0_extern0_etaa0.79.png}
	\end{minipage}
	\begin{minipage}[]{0.32\textwidth}
		\centering{\footnotesize{(b) Hours worked }}
		%	\captionsetup{width=.45\linewidth}
		\includegraphics[width=1\textwidth]{../../codding_model/own_basedOnFried/optimalPol_190722_tidiedUp/figures/all_10Aout22/SingleJointTOT_regime3_OPT_T_NoTaus_Labour_spillover0_noskill0_sep1_xgrowth0_extern0_PV1_etaa0.79_lgd1.png}
	\end{minipage}
	\begin{minipage}[]{0.32\textwidth}
		\centering{\footnotesize{(c) High-to-low-skill ratio hours}}
		%	\captionsetup{width=.45\linewidth}
		\includegraphics[width=1\textwidth]{../../codding_model/own_basedOnFried/optimalPol_190722_tidiedUp/figures/all_10Aout22/Single_OPT_T_NoTaus_hhhl_regime3_spillover0_noskill0_sep1_xgrowth0_extern0_etaa0.79.png}
	\end{minipage}
	\begin{minipage}[]{0.32\textwidth}
		\centering{\footnotesize{\ \\ (d) Technology growth}}
		%	\captionsetup{width=.45\linewidth}
		\includegraphics[width=1\textwidth]{../../codding_model/own_basedOnFried/optimalPol_190722_tidiedUp/figures/all_10Aout22/SingleJointTOT_regime3_OPT_T_NoTaus_Growth_spillover0_noskill0_sep1_xgrowth0_extern0_PV1_etaa0.79_lgd1.png}
	\end{minipage}
%\begin{minipage}[]{0.32\textwidth}
%	\centering{\footnotesize{\ \\ (d) Technology growth}}
%	%	\captionsetup{width=.45\linewidth}
%	\includegraphics[width=1\textwidth]{../../codding_model/own_basedOnFried/optimalPol_190722_tidiedUp/figures/all_July22/SingleJointTOT_regime0_OPT_T_NoTaus_Growth_spillover0_noskill0_sep1_xgrowth0_extern0_etaa0.79_lgd1.png}
%\end{minipage}
	\begin{minipage}[]{0.32\textwidth}
		\centering{\footnotesize{\ \\(e) Scientists }}
		%	\captionsetup{width=.45\linewidth}
		\includegraphics[width=1\textwidth]{../../codding_model/own_basedOnFried/optimalPol_190722_tidiedUp/figures/all_10Aout22/SingleJointTOT_regime3_OPT_T_NoTaus_Science_spillover0_noskill0_sep1_xgrowth0_extern0_PV1_etaa0.79_lgd1.png}
	\end{minipage}
	\begin{minipage}[]{0.32\textwidth}
		\centering{\footnotesize{\ \\(f) Labor input}}
		%	\captionsetup{width=.45\linewidth}
		\includegraphics[width=1\textwidth]{../../codding_model/own_basedOnFried/optimalPol_190722_tidiedUp/figures/all_10Aout22/SingleJointTOT_regime3_OPT_NOT_NoTaus_LabourInp_spillover0_noskill0_sep1_xgrowth0_extern0_PV1_etaa0.79_lgd1.png}
	\end{minipage}
\end{figure} 



\subsubsection{Consumption equivalence}
\tr{How important is the income tax}

%\begin{figure}[h!!]
%	\centering
%	\caption{Change relative to first period }\label{fig:optAll_percEffOpt}
%	
%	
%	\begin{minipage}[]{0.32\textwidth}
%		\centering{\footnotesize{(a) Consumption}}
%		%	\captionsetup{width=.45\linewidth}
%		\includegraphics[width=1\textwidth]{../../codding_model/own_basedOnFried/optimalPol_190722_tidiedUp/figures/all_10Aout22/C_PercentageEffOptFirstPeriod_Target_regime3_spillover0_noskill0_sep1_xgrowth0_etaa0.79_lgd1.png}
%	\end{minipage}
%	\begin{minipage}[]{0.32\textwidth}
%		\centering{\footnotesize{(b) High-skilled hours worked }}
%		%	\captionsetup{width=.45\linewidth}
%		\includegraphics[width=1\textwidth]{../../codding_model/own_basedOnFried/optimalPol_190722_tidiedUp/figures/all_10Aout22/hh_PercentageEffOptFirstPeriod_Target_regime3_spillover0_noskill0_sep1_xgrowth0_etaa0.79_lgd0.png}
%	\end{minipage}
%	\begin{minipage}[]{0.32\textwidth}
%		\centering{\footnotesize{(c) Low-skilled hours worked}}
%		%	\captionsetup{width=.45\linewidth}
%		\includegraphics[width=1\textwidth]{../../codding_model/own_basedOnFried/optimalPol_190722_tidiedUp/figures/all_10Aout22/hl_PercentageEffOptFirstPeriod_Target_regime3_spillover0_noskill0_sep1_xgrowth0_etaa0.79_lgd0.png}
%	\end{minipage}
%	\begin{minipage}[]{0.32\textwidth}
%		\centering{\footnotesize{\ \\ (d) Green-to-fossil technology ratio }}
%		%	\captionsetup{width=.45\linewidth}
%		\includegraphics[width=1\textwidth]{../../codding_model/own_basedOnFried/optimalPol_190722_tidiedUp/figures/all_10Aout22/AgAf_PercentageEffOptFirstPeriod_Target_regime3_spillover0_noskill0_sep1_xgrowth0_etaa0.79_lgd1.png}
%	\end{minipage}
%	%\begin{minipage}[]{0.32\textwidth}
%	%	\centering{\footnotesize{\ \\ (d) Technology growth}}
%	%	%	\captionsetup{width=.45\linewidth}
%	%	\includegraphics[width=1\textwidth]{../../codding_model/own_basedOnFried/optimalPol_190722_tidiedUp/figures/all_July22/SingleJointTOT_regime0_OPT_T_NoTaus_Growth_spillover0_noskill0_sep1_xgrowth0_extern0_etaa0.79_lgd1.png}
%	%\end{minipage}
%	\begin{minipage}[]{0.32\textwidth}
%		\centering{\footnotesize{\ \\(e) Green-to-fossil scientists ratio }}
%		%	\captionsetup{width=.45\linewidth}
%		\includegraphics[width=1\textwidth]{../../codding_model/own_basedOnFried/optimalPol_190722_tidiedUp/figures/all_10Aout22/sgsff_PercentageEffOptFirstPeriod_Target_regime3_spillover0_noskill0_sep1_xgrowth0_etaa0.79_lgd1.png}
%	\end{minipage}
%	\begin{minipage}[]{0.32\textwidth}
%		\centering{\footnotesize{\ \\(f) Green-to-fossil labor input}}
%		%	\captionsetup{width=.45\linewidth}
%		\includegraphics[width=1\textwidth]{../../codding_model/own_basedOnFried/optimalPol_190722_tidiedUp/figures/all_10Aout22/LgLf_PercentageEffOptFirstPeriod_Target_regime3_spillover0_noskill0_sep1_xgrowth0_etaa0.79_lgd1.png}
%	\end{minipage}
%\end{figure} 
%



%%%%%%%%%%%%%%%%%%%%%%%%%%%%%%%%%%%%%%%%%%%%%%%%%%%%%%%%%%%%%%%%%%%%%%%%%%%%%%%%%%%%
%% DISCUSSION 
%%%%%%%%%%%%%%%%%%%%%%%%%%%%%%%%%%%%%%%%%%%%%%%%%%%%%%%%%%%%%%%%%%%%%%%%%%%%%%%%%%%%

\subsection{Discussion}\label{subsec:dis}
%\tr{Questions}
%\begin{itemize}
%	\item why progressive tax? and why the drop in progressivity in 2050? (a means to boost high-skill supply and keeping low skill stable)
%	\item what are the costs of the progressive tax
%\end{itemize}

 What explains the optimal policy?
 In section \ref{subsec:notaul}, I contrast the optimal allocation under the benchmark policy regime to a scenario where no income tax is available.  This comparison is informative on the benefits of an income tax. The role of endogenous growth and skill heterogeneity is analyzed in section \ref{subsec:xgrnsk}.
Finally, in section \ref{subsec:comp_lumpsum}, I turn to analyze the optimal allocation under the alternative policy regimes: redistribution of environmental tax revenues via (1) lump-sum transfers and (2) the income tax scheme. 
%In subsection \ref{subsec:simpler}, I discuss the results when the benchmark model is simplified: that is, assuming exogenous growth and/or skill homogeneity.

%\begin{enumerate}
%	\item What is the goal of policy intervention? \ar social planner allocation
%	\item (Benefits) What is different when no integrated policy is run and instead revs consumed by government \ar Benefits of an integrated policy
%	\item double dividend literature: use of labor income tax when all env tax revenues are consumed by the government.
%	\item (Costs) What cannot be reached by integrated policy as compared to lump-sum transfers: is taul used for different purpose? without endogenous growth should be zero; eg. can use taul to boost growth as lump-sum transfers take care of labor supply 
%	\item What could be reached if there was no trade-off with heterogenous skills or growth? no heterogeneous skills, no endogenous growth \ar how does the optimal policy differ?
%\end{enumerate}

%\subsubsection{Comparison to other policy regimes}
%\tr{To be rewritten}
%How does the optimal allocation and especially its relation to the efficient allocation change under alternative policy scenarios?
%In this section, I discuss two policy alternations which have already been discussed in the analytical section. First, a version where environmental tax revenues are consumed by the government and no labor income tax scheme is available, henceforth referred to as \textit{separate policy}. The comparison of this scenario serves to assess the benefits of an integrated environmental-fiscal policy when no lump-sum transfers are available. 
%Second, I look at the optimal allocation 

\subsubsection{The role of income taxes}\label{subsec:notaul}



In figure \ref{fig:optAll_percLf_dyn}, I contrast the optimal allocation under the benchmark regime with income tax scheme, black solid graph, with the optimal allocation without labor income tax, the blue dashed graph. As a benchmark to the optimal policy, the figure depicts the social planner's allocation by the orange dotted graph. The efficient allocation can be perceived as the allocation the Ramsey planner seeks to implement. However, she may not be able to achieve the efficient allocation due to the reliance on tax instruments.
All graphs depict percentage changes relative to the laissez-faire allocation of the same period. Except for panel (f) which compares the environmental tax in the model with and without income tax. 

\begin{figure}[h!!]
	\centering
	\caption{Costs and benefits of progressive income taxes }\label{fig:optAll_percLf_dyn}
	\begin{minipage}[]{0.32\textwidth}
		\centering{\footnotesize{(a) Consumption}}
		%	\captionsetup{width=.45\linewidth}
		\includegraphics[width=1\textwidth]{../../codding_model/own_basedOnFried/optimalPol_190722_tidiedUp/figures/all_10Aout22/C_PercentageLFDynNT_Target_regime3_spillover0_noskill0_sep1_xgrowth0_etaa0.79_lgd1.png}
	\end{minipage}
	\begin{minipage}[]{0.32\textwidth}
		\centering{\footnotesize{(b) High-skilled hours worked }}
		%	\captionsetup{width=.45\linewidth}
		\includegraphics[width=1\textwidth]{../../codding_model/own_basedOnFried/optimalPol_190722_tidiedUp/figures/all_10Aout22/hh_PercentageLfDynNT_Target_regime3_spillover0_noskill0_sep1_xgrowth0_etaa0.79_lgd0.png}
	\end{minipage}
	\begin{minipage}[]{0.32\textwidth}
		\centering{\footnotesize{(c) Low-skilled hours worked}}
		%	\captionsetup{width=.45\linewidth}
		\includegraphics[width=1\textwidth]{../../codding_model/own_basedOnFried/optimalPol_190722_tidiedUp/figures/all_10Aout22/hl_PercentageLfDynNT_Target_regime3_spillover0_noskill0_sep1_xgrowth0_etaa0.79_lgd0.png}
	\end{minipage}
	\begin{minipage}[]{0.32\textwidth}
	\centering{\footnotesize{\ \\(d) Aggregate growth\\ \tr{check time axis}\ }}
	%	\captionsetup{width=.45\linewidth}
	\includegraphics[width=1\textwidth]{../../codding_model/own_basedOnFried/optimalPol_190722_tidiedUp/figures/all_10Aout22/gAagg_PercentageLfDynNT_noeff_Target_regime3_spillover0_noskill0_sep1_xgrowth0_etaa0.79_lgd0.png}
\end{minipage}
\begin{minipage}[]{0.32\textwidth}
	\centering{\footnotesize{\ \\(e) Environmental tax, $\tau_{ft}$}}
	%	\captionsetup{width=.45\linewidth}
	\includegraphics[width=1\textwidth]{../../codding_model/own_basedOnFried/optimalPol_190722_tidiedUp/figures/all_10Aout22/comp_benchregime3_notaul2_OPT_T_NoTaus_tauf_spillover0_noskill0_sep1_xgrowth0_PV1_etaa0.79_lgd0.png}
\end{minipage}
\begin{minipage}[]{0.32\textwidth}
	\centering{\footnotesize{\ \\(f) Aggregate research }}
	%	\captionsetup{width=.45\linewidth}
	\includegraphics[width=1\textwidth]{../../codding_model/own_basedOnFried/optimalPol_190722_tidiedUp/figures/all_10Aout22/S_PercentageLfDynNT_noeff_Target_regime3_spillover0_noskill0_sep1_xgrowth0_etaa0.79_lgd0.png}
\end{minipage}
%\begin{minipage}[]{0.32\textwidth}
%	\centering{\footnotesize{\ \\(d) Non-energy research }}
%	%	\captionsetup{width=.45\linewidth}
%	\includegraphics[width=1\textwidth]{../../codding_model/own_basedOnFried/optimalPol_190722_tidiedUp/figures/all_10Aout22/sn_PercentageLfDynNT_Target_regime3_spillover0_noskill0_sep1_xgrowth0_etaa0.79_lgd0.png}
%\end{minipage}
	%	\begin{minipage}[]{0.32\textwidth}
	%		\centering{\footnotesize{\ \\ (d) Green-to-fossil technology ratio }}
	%		%	\captionsetup{width=.45\linewidth}
	%		\includegraphics[width=1\textwidth]{../../codding_model/own_basedOnFried/optimalPol_190722_tidiedUp/figures/all_10Aout22/AgAf_PercentageLfDynNT_Target_regime3_spillover0_noskill0_sep1_xgrowth0_etaa0.79_lgd0.png}
	%	\end{minipage}
	%\begin{minipage}[]{0.32\textwidth}
	%	\centering{\footnotesize{\ \\ (d) Technology growth}}
	%	%	\captionsetup{width=.45\linewidth}
	%	\includegraphics[width=1\textwidth]{../../codding_model/own_basedOnFried/optimalPol_190722_tidiedUp/figures/all_July22/SingleJointTOT_regime0_OPT_T_NoTaus_Growth_spillover0_noskill0_sep1_xgrowth0_extern0_etaa0.79_lgd1.png}
	%\end{minipage}	
	\begin{minipage}[]{0.32\textwidth}
		\centering{\footnotesize{\ \\(g) Green-to-fossil energy }}
		%	\captionsetup{width=.45\linewidth}
		\includegraphics[width=1\textwidth]{../../codding_model/own_basedOnFried/optimalPol_190722_tidiedUp/figures/all_10Aout22/GFF_PercentageLfDynNT_Target_regime3_spillover0_noskill0_sep1_xgrowth0_etaa0.79_lgd0.png}
	\end{minipage}
	\begin{minipage}[]{0.32\textwidth}
		\centering{\footnotesize{\ \\(h) Green-to-fossil scientists ratio\\ \ }}
		%	\captionsetup{width=.45\linewidth}
		\includegraphics[width=1\textwidth]{../../codding_model/own_basedOnFried/optimalPol_190722_tidiedUp/figures/all_10Aout22/sgsff_PercentageLfDynNT_Target_regime3_spillover0_noskill0_sep1_xgrowth0_etaa0.79_lgd0.png}
	\end{minipage}
	\begin{minipage}[]{0.32\textwidth}
		\centering{\footnotesize{\ \\(i) Green-to-fossil labor input}}
		%	\captionsetup{width=.45\linewidth}
		\includegraphics[width=1\textwidth]{../../codding_model/own_basedOnFried/optimalPol_190722_tidiedUp/figures/all_10Aout22/LgLf_PercentageLfDynNT_Target_regime3_spillover0_noskill0_sep1_xgrowth0_etaa0.79_lgd0.png}
	\end{minipage}
\end{figure} 
%
% Labor supply
In comparison to a policy scenario without income tax, the availability of an income tax allows to more closely resemble the efficient levels of labor, panels (b) and (c). 
The social planner reduces hours worked for both the high- and the low-skill type by between 3 to 4 percent relative to the laissez-faire allocation. The reduction in  hours worked by the low type remains close to zero when no labor income tax is available, the dashed graph. Under the same policy regime, hours of high skill workers even increases slightly. When the government has income taxes available, it reduces hours worked of both types closer to the efficient allocation. While the efficient allocation sees a similar decline in working time of both types, the optimal policy allocation features a stronger decrease in working time by high-skill labor and too low a reduction of the low-skill ones. 

%- Labor income taxes have advantage in terms of growth
A second benefit in the quantitative model stems from endogenous growth. 
In the periods before the net-zero emission limit (2020 to 2050), the optimal policy with progressive income tax achieves higher growths rate; consider panel (d) which shows aggregate growth in the Ramsey allocations. In these periods, the Ramsey planner is able to achieve a higher growth rate by partly substituting fossil taxation with labor income taxation: in presence of a progressive income tax a lower fossil tax is feasible; see panel (g). The mechanism is as follows. 
As the fossil sector is the technologically most advanced,  cross-sector spillovers make environmental taxes especially costly in terms of growth \citep[this mechanism has been discussed in][]{Fried2018ClimateAnalysis}. Switching from fossil to income taxes allows the government to reap the benefits from technology spillovers while meeting emission limits and reducing research, panel (f). Although the increase in growth rates is small - not a percentage point difference in growth rate reduction per period - the total effect on the future is substantial as highlighted by the consumption equivalence. 
 This mechanism underlines an advantage of reductive environmental policies as opposed to recomposing strategies.  While income taxes and environmental taxes have shown to be complements in the analytical model, they act as substitutes through this endogenous growth channel.\footnote{\ \tr{An alternative explanation for the advantage of income taxes above environmental taxes under the presented policy regime could be that labor taxes are redistributed to households, so there is no reduction in consumption via government consumption. To test this alternative explanation, I run a model version where both income and fossil taxes are consumed by the government.} Then again, the use of labor income taxes reduces consumption, so if there was a channel through government consumption it does not dominate.}

Once the net-zero emission limit becomes binding in 2050, however, the gap between environmental taxes reduces and aggregate growth in the model without income taxes is minimally higher.   This suggests that the net-zero emission limit prevents to substitute fossil with labor income taxes. 

% costs
The benefits of the progressive income tax, more leisure and technology spillovers, come at the cost of less consumption, panel (a),  and a lower green-to-fossil energy mix, panel (g). The social planner implements continues consumption growth and only reduces consumption below laissez-faire levels during the first periods. In contrast, the optimal allocation reduces consumption relative to laissez-faire in all periods. The reduction is increasing over time.\footnote{\ Figure \ref{fig:LF} in section \ref{app:quant_res} shows the laissez-faire allocation. } 

The higher green-to-fossil energy ratio in the efficient allocation is driven by a reallocation of input factors towards the green sector under the social planner; see panel (i). In contrast, the ratio of scientists remains unchanged (panel (h)). This observation suggests that the planner recomposes the economy by inputs and through an adjustment in technologies in order to further profit from technology spillovers. In fact, these spillovers in favor of the less advanced sector enable the social planner to implement a higher green to fossil technology ratio than in the laissez-faire economy (not shown). 

In the optimal allocation, irrespective of the policy regime, the increase in the green-to-fossil energy mix is inefficiently low. The progressive income tax minimizes the increase even further. The reduction is explained by both less green-to-fossil research and labor input. This recomposing effect of the labor income tax arises from the higher responsiveness of high skill labor to the tax progressivity. As the typical green input good is in smaller supply, production shifts to the fossil sector. This again transmits to research efforts, as machine producers' profits from research in the fossil sector are higher amplifying the recomposing effect of income taxes. 

\begin{comment}
COMMENT ON WEAK DD

These results speak to the weak double-dividend literature. %When the government consumes environmental tax revenues, hours worked are inefficiently high. 
The weak double-dividend result posits that when environmental tax revenues suffice to cover all government funding requirements, it would be optimal to lower distortionary income taxes. The results presented herein, however, show that there is a lower bound. Lowering distortionary income taxes too much results in inefficiently high hours worked. Hence, even though there is no motive to fund government expenses  labor income taxation is not zero due to the environmental externality.
%Indeed, this reduces consumption further away from the efficient level, but, hours worked are aligned closer to the efficient level, panels (b) and (c). Next to consumption, the planner also forfeits an advantageous green-to-fossil energy ratio, panel (e). 

%The use of a progressive labor income tax contributes minimally to meeting the emission limit as can be seen by scrutinizing the optimal environmental tax, panel (b) in figure \ref{fig:comp_nored_pol}: when income taxes can be used, the environmental tax is lower. Still, the difference is minimal, supporting the thesis of complementarity of income and environmental taxes. 
%Environmental tax revenues are lower as the tax rate reduces, and income taxes reduce labor supply and hence the tax base of the environmental tax. 
%Even though labor income taxes have the advantage of being redistributed to households and lowering the externality, they are not used to substitute environmental tax revenues.\footnote{\ This might be a motive to prefer labor income taxes as an instrument to reduce emissions since labor income tax revenues are redistributed back to the household while environmental tax revenues are not in this setting. Nevertheless, the observation that the environmental tax only adjusts slightly once an income tax tool is available points to the advantage of environmental taxes in handling too high emissions.
%}
\end{comment}

\subsubsection{Endogenous growth and skill heterogeneity}\label{subsec:xgrnsk}
\tr{How do the two factors affect the optimal policy and welfare?}

\begin{figure}[h!!]
	\centering
	\caption{Optimal policy with and without income tax \tr{Plot interaction and single effects: decomposition}}\label{fig:comp_mod}
	
	\begin{minipage}[]{0.4\textwidth}
		\centering{\footnotesize{(a) Income tax progressivity, $\tau_{\iota t}$}}
		%	\captionsetup{width=.45\linewidth}
		\includegraphics[width=1\textwidth]{../../codding_model/own_basedOnFried/optimalPol_190722_tidiedUp/figures/all_10Aout22/CompMod1_OPT_T_NoTaus_taul_regime3_spillover0_noskill0_sep1_xgrowth0_extern0_etaa0.79_lgd1.png}
	\end{minipage}
\begin{minipage}[]{0.1\textwidth}
	\
\end{minipage}
\begin{minipage}[]{0.4\textwidth}
	\centering{\footnotesize{(b) Environmental tax, $\tau_{ft}$}}
	%	\captionsetup{width=.45\linewidth}
	\includegraphics[width=1\textwidth]{../../codding_model/own_basedOnFried/optimalPol_190722_tidiedUp/figures/all_10Aout22/CompMod1_OPT_T_NoTaus_tauf_regime3_spillover0_noskill0_sep1_xgrowth0_extern0_etaa0.79_lgd0.png}
\end{minipage}
\end{figure}

 
\begin{comment}
\paragraph{Comparison integrated policy to separate policy}

Consider figure \ref{fig:bench_nored_notaul}. The figure presents the optimal allocation in the integrated policy scenario,  the orange-dashed graph, the optimal allocation under the separate policy, the blue-dotted graph, and the efficient allocation, the black-solid graph.

In comparison to a policy scenario where environmental tax revenues are not redistributed, the integrated policy closer resembles the efficient allocation in terms of consumption, panel (a) and of labor, panels (b) and (c). %In total, the utility level of the representative household is at least as close to the efficient level for all time periods considered. 
The benefits of an integrated-policy regime come at the cost of a lower green-to-fossil energy mix, panel (e), and a reduction in growth, panel (d). Nevertheless, if a planner could choose between the two regimes, it would select the integrated-policy regime. The gains from the integrated regime amount to xxx. \tr{Do CEV}

Interestingly, the optimal environmental tax is only negligibly smaller in the integrated-policy regime. This suggests, that environmental taxes and labor income taxes are complements in the optimal environmental policy to lower inefficiently high hours worked. Only in the period from 2030 to 2050 the environmental tax necessary to meet emission limits is slightly smaller which can be rationalized by a lower level of production.\footnote{\ Absent an emission limit before 2030, the optimal environmental tax is slightly negative to subsidize fossil research which again spills over to research in the other sectors. }  

\begin{figure}[h!!]
	\centering
	\caption{Comparison to separate policy scenario; \tr{drop efficient from tauf graph }}\label{fig:bench_nored_notaul}
	
	\begin{minipage}[]{0.32\textwidth}
		\centering{\footnotesize{(a) Consumption}}
		%	\captionsetup{width=.45\linewidth}
		\includegraphics[width=1\textwidth]{../../codding_model/own_basedOnFried/optimalPol_190722_tidiedUp/figures/all_July22/C_CompEffOPT_T_NoTaus_pol2_spillover0_noskill0_sep1_xgrowth0_etaa0.79_lgd1_lff0.png}
	\end{minipage}
	\begin{minipage}[]{0.32\textwidth}
		\centering{\footnotesize{(b) High skill hours worked}}
		%	\captionsetup{width=.45\linewidth}
		\includegraphics[width=1\textwidth]{../../codding_model/own_basedOnFried/optimalPol_190722_tidiedUp/figures/all_July22/hh_CompEffOPT_T_NoTaus_pol2_spillover0_noskill0_sep1_xgrowth0_etaa0.79_lgd0_lff0.png}
	\end{minipage}
	\begin{minipage}[]{0.32\textwidth}
		\centering{\footnotesize{(c) Low skill hours worked}}
		%	\captionsetup{width=.45\linewidth}
		\includegraphics[width=1\textwidth]{../../codding_model/own_basedOnFried/optimalPol_190722_tidiedUp/figures/all_July22/hl_CompEffOPT_T_NoTaus_pol2_spillover0_noskill0_sep1_xgrowth0_etaa0.79_lgd0_lff0.png}
	\end{minipage}
	\begin{minipage}[]{0.32\textwidth}
		\centering{\footnotesize{(d) Aggregate growth}}
		%	\captionsetup{width=.45\linewidth}
		\includegraphics[width=1\textwidth]{../../codding_model/own_basedOnFried/optimalPol_190722_tidiedUp/figures/all_July22/gAagg_CompEffOPT_T_NoTaus_pol2_spillover0_noskill0_sep1_xgrowth0_etaa0.79_lgd0_lff0.png}
	\end{minipage}
	\begin{minipage}[]{0.32\textwidth}
		\centering{\footnotesize{(e) Energy mix, $\frac{G}{F}$}}
		%	\captionsetup{width=.45\linewidth}
		\includegraphics[width=1\textwidth]{../../codding_model/own_basedOnFried/optimalPol_190722_tidiedUp/figures/all_July22/GFF_CompEffOPT_T_NoTaus_pol2_spillover0_noskill0_sep1_xgrowth0_etaa0.79_lgd0_lff0.png}
	\end{minipage}
	%	\begin{minipage}[]{0.32\textwidth}
	%	\centering{\footnotesize{(f) Utility}}
	%	%	\captionsetup{width=.45\linewidth}
	%	\includegraphics[width=1\textwidth]{../../codding_model/own_basedOnFried/optimalPol_190722_tidiedUp/figures/all_July22/SWF_CompEffOPT_T_NoTaus_pol2_spillover0_noskill0_sep1_xgrowth0_etaa0.79_lgd0_lff0.png}
	%\end{minipage}
	\begin{minipage}[]{0.32\textwidth}
		\centering{\footnotesize{(f) Environmental tax, $\tau_{ft}$}}
		%	\captionsetup{width=.45\linewidth}
		\includegraphics[width=1\textwidth]{../../codding_model/own_basedOnFried/optimalPol_190722_tidiedUp/figures/all_July22/tauf_CompEffOPT_T_NoTaus_pol2_spillover0_noskill0_sep1_xgrowth0_etaa0.79_lgd0_lff0.png}
	\end{minipage}
\end{figure}


	content...
\end{comment}


\subsubsection{Optimal policy and allocation with lump-sum transfers}\label{subsec:comp_lumpsum}
\tr{Or better: other policy regimes}

How do lump-sum transfers change the role of labor income taxes?\footnote{\ In appendix section \tr{To be added}, I present the optimal allocation under a policy regime where environmental tax revenues are redistributed through the income tax scheme. This scenario is relevant when the government wants to redistribute environmental tax revenues but lump-sum transfers are not feasible.}
 According to the theory in section \ref{sec:mod_an}, the use of lump-sum taxes should (i)  allow to attain an allocation closer to the efficient one and (ii) deprive the income tax scheme of its use as reductive environmental policy tool. Indeed, the optimal allocation under lump-sum transfers is much closer to the efficient one. 
 Yet, the emission limit still shapes the optimal income tax due to endogenous growth. 
  %This is so despite the advantageous recomposing effect of regressive income taxes through skill supply.
  
% YES, one can speak of a separation of environmental and fiscal policies as the goal of income taxes is to boost or lower growth in the first place. We also dont speak of the environmental tax being targeted at 

\begin{figure}[h!!]
	\centering
	\caption{Comparison integrated regime and regime lump-sum transfers}\label{fig:bench_lumpsum}
	
	\begin{minipage}[]{0.32\textwidth}
		\centering{\footnotesize{(a) Consumption}}
		%	\captionsetup{width=.45\linewidth}
		\includegraphics[width=1\textwidth]{../../codding_model/own_basedOnFried/optimalPol_190722_tidiedUp/figures/all_July22/C_CompEffOPT_T_NoTaus_bb3_pol4_spillover0_noskill0_sep1_xgrowth0_etaa0.79_lgd1_lff0.png}
	\end{minipage}
	\begin{minipage}[]{0.32\textwidth}
		\centering{\footnotesize{(b) High skill hours worked}}
		%	\captionsetup{width=.45\linewidth}
		\includegraphics[width=1\textwidth]{../../codding_model/own_basedOnFried/optimalPol_190722_tidiedUp/figures/all_July22/hh_CompEffOPT_T_NoTaus_bb3_pol4_spillover0_noskill0_sep1_xgrowth0_etaa0.79_lgd0_lff0.png}
	\end{minipage}
	\begin{minipage}[]{0.32\textwidth}
		\centering{\footnotesize{(c) Low skill hours worked}}
		%	\captionsetup{width=.45\linewidth}
		\includegraphics[width=1\textwidth]{../../codding_model/own_basedOnFried/optimalPol_190722_tidiedUp/figures/all_July22/hl_CompEffOPT_T_NoTaus_bb3_pol4_spillover0_noskill0_sep1_xgrowth0_etaa0.79_lgd0_lff0.png}
	\end{minipage}
	\begin{minipage}[]{0.32\textwidth}
		\centering{\footnotesize{(d) Aggregate growth}}
		%	\captionsetup{width=.45\linewidth}
		\includegraphics[width=1\textwidth]{../../codding_model/own_basedOnFried/optimalPol_190722_tidiedUp/figures/all_July22/gAagg_CompEffOPT_T_NoTaus_bb3_pol4_spillover0_noskill0_sep1_xgrowth0_etaa0.79_lgd0_lff0.png}
	\end{minipage}
	\begin{minipage}[]{0.32\textwidth}
		\centering{\footnotesize{(e) Energy mix, $\frac{G}{F}$}}
		%	\captionsetup{width=.45\linewidth}
		\includegraphics[width=1\textwidth]{../../codding_model/own_basedOnFried/optimalPol_190722_tidiedUp/figures/all_July22/GFF_CompEffOPT_T_NoTaus_bb3_pol4_spillover0_noskill0_sep1_xgrowth0_etaa0.79_lgd0_lff0.png}
	\end{minipage}
	\begin{minipage}[]{0.32\textwidth}
		\centering{\footnotesize{(f) Utility}}
		%	\captionsetup{width=.45\linewidth}
		\includegraphics[width=1\textwidth]{../../codding_model/own_basedOnFried/optimalPol_190722_tidiedUp/figures/all_July22/SWF_CompEffOPT_T_NoTaus_bb3_pol4_spillover0_noskill0_sep1_xgrowth0_etaa0.79_lgd0_lff0.png}
	\end{minipage}
\end{figure}

 Figure \ref{fig:bench_lumpsum} contrasts the efficient allocation, the black solid graphs, the allocation under the benchmark policy, the orange-dashed graphs, and the optimal allocation when lump-sum transfers are available, the blue-dotted graphs. 
When lump-sum transfers of environmental tax revenues are in the policy set, the Ramsey planner can implement hours worked closer to the efficient allocation, panels (b) and (c). Consumption under the regime with lump-sum transfers, as well, mirrors the efficient level more closely see panel (a). 
Growth in the scenario with lump-sum transfers is at least as high as under the benchmark policy due to the use of regressive income taxes to accelerate growth. % I DONT KNOW WHY: However, the green-to-fossil energy ratio is slightly higher under the benchmark policy under the net-zero emission limit. The recomposing mechanism of the income tax via a relatively higher supply of high-skilled labor contributes to this finding. 
Utility gains from the availability of lump-sum transfers overall seem sizable especially under the net-zero emission limit; compare panel (f).

\begin{figure}[h!!]
	\centering
	\caption{Optimal policy in integrated regime and with lump-sum transfers}\label{fig:bench_lumpsum_pol}
	
	\begin{minipage}[]{0.32\textwidth}
		\centering{\footnotesize{(a) Income tax progressivity, $\tau_{\iota t}$}}
		%	\captionsetup{width=.45\linewidth}
		\includegraphics[width=1\textwidth]{../../codding_model/own_basedOnFried/optimalPol_190722_tidiedUp/figures/all_July22/comp_bb3_notaul4_OPT_T_NoTaus_taul_spillover0_noskill0_sep1_xgrowth0_etaa0.79_lgd1.png}
	\end{minipage}
	\begin{minipage}[]{0.32\textwidth}
		\centering{\footnotesize{(b) Environmental tax, $\tau_{ft}$}}
		%	\captionsetup{width=.45\linewidth}
		\includegraphics[width=1\textwidth]{../../codding_model/own_basedOnFried/optimalPol_190722_tidiedUp/figures/all_July22/comp_bb3_notaul4_OPT_T_NoTaus_tauf_spillover0_noskill0_sep1_xgrowth0_etaa0.79_lgd0.png}
	\end{minipage}
	\begin{minipage}[]{0.32\textwidth}
		\centering{\footnotesize{(c) Lump-sum transfers}}
		%	\captionsetup{width=.45\linewidth}
		\includegraphics[width=1\textwidth]{../../codding_model/own_basedOnFried/optimalPol_190722_tidiedUp/figures/all_July22/comp_bb3_notaul4_OPT_T_NoTaus_Tls_spillover0_noskill0_sep1_xgrowth0_etaa0.79_lgd0.png}
	\end{minipage}
\end{figure}


% finding 1) income tax to boost growth, 2) no use of recomposing effect of income tax
Figure \ref{fig:bench_lumpsum_pol} shows the optimal policy when lump-sum transfers are available. 
Now, the optimal income tax scheme is regressive. Since lump-sum transfers ensure a reduction in labor supply, the inefficiency in labor supply as a motive for progressive income taxes vanishes.
Instead, the motive to boost growth and consumption dominates: absent endogenous growth, the income tax remains untouched.\footnote{\ Compare results in the model with exogenous growth depicted in figure \ref{fig:lumpsum_xgr_vglNotaul} in appendix section \ref{app:lumps}.}  In fact, the allocation attained in the model without endogenous growth but lump-sum transfers is similar to the efficient one at a zero income tax progressivity; compare figure \ref{fig:lumpsum_xgr_vglNotaul}. Hence, the environmental policy does not use income tax regressivity to recompose production towards green energy by subsidizing high-skill supply.
 
 \tr{Has to be rewritten: it is rather, that the costs of more research seem to be too high in terms of disutility.}
Nevertheless, note, though, that regressivity of the optimal income tax reduces over time; compare the orange-dashed graph in panel (a) in figure \ref{fig:bench_lumpsum_pol}. This points to the income tax, again, being shaped by the environmental externality.
This is so even though consumption is inefficiently low and higher growth rates would be efficiency improving. I conclude from this observation that it is the environmental externality which makes it optimal to forfeit growth. 
The conflict between growth and emissions, however, does not seem to arise from growth itself, since the social planner satisfies the emission limit at higher growth rates. Instead, the issue stems from labor supply as a means to boost growth in the competitive economy. 
Then, the decreasing pattern of income tax regressivity can be rationalized as follows: as growth increases, more labor means more production and hence emissions. Therefore, boost growth becomes more costly in terms of emissions. Only growth which is concomitant with lower production is feasible in the market economy.\footnote{\ If the Ramsey planner had tools at hand to boost growth without necessarily boosting production, more growth would be optimal. \tr{\textit{Does a research subsidy imply more production?}}}

%It suggests itself that the conflict with growth does not stem from growth itself but rather that it is fostered in the competitive economy through a market size effect, that is, more production, since the social planner chooses higher technology levels. This would speak against progressive income taxes as growth accelerates. Thus, the environmental indeed prevents usage of the income tax to boost demand, but I conclude that it is not used with the goal to reduce emissions through the endogenous growth channel. Since growth as such does not pose the conflict to the emission limit. 
I conclude from this discussion that it is solely the environmental tax and not the income tax which addresses the environmental externality when lump-sum transfers are available.

\subsection{Sensitivity}
I will now briefly discuss sensitivity analyses to the quantitative exercise. 
\subsubsection{Wage elasticity of labor}

Recent papers have examined the wage elasticity of labor. \cite{Boppart2019LaborPerspectiveb} present evidence that hours worked per worker have been falling steadily over time 

\subsubsection{Research subsidy}
but finding should be similar to version without endogenous growth
\subsubsection{Changing emission limits calculation}
The \textit{equal-per-capita} approach is favorable for population high countries like the US. Therefore, in the sensitivity analysis, I rerun the model where US emission limits follow from \textit{Equal cumulative per capita} approach, where countries with historically  high emissions per capita mitigate more.
% \textit{constant-emission-ratio} approach which is achieved by all countries reducing emissions by 50\% in the 2030s relative to 2019. This principle limits US emissions to 2.309Gt in the 2030s. THIS APPROACH IS EVEN MORE FAVOURABLE  TO THE US BECAUSE CURRENTLY THEY EMIT A HIGHER SHARE PER CAPITA THAN THE REST OF THE WORLD. 

\subsubsection{Technology gap}