\section{Introduction}
%\tr{I show that carbon taxes are only efficient if lump-sum transfers are available.}

\begin{comment}
\tr{Think about:
	1) when labor income taxes are not used, then need to have  a higher environmental tax to meet emission limits? \ar Yes, because of advantageous level effect which outweighs recomposing effect of income tax.
	2) When staying at level optimal under the assumption of lump-sum redistribution, but then not redistributing, than absent labor income tax emissions are too high; by how much? Counterfactual}
	
	content...
	\end{comment}

The latest assessment report of the Intergovernmental Panel on Climate Change (IPCC) \citep{IPCC2022} highlights the urgency to reduce greenhouse-gas emissions.%relative to the previous report from 2018 \citep{Rogelj2018MitigationDevelopment.}.
\footnote{ \  The report stresses the decreasing likelihood of meeting the Paris Agreement and limiting climate warming to 1.5°. The Paris Agreement of 2015 formulates clear political goals to mitigate climate change. Under this treaty, states have agreed on a legally binding maximum increase in temperature to well below 2°C, preferably 1.5° over pre-industrial levels, and the global community seeks to be climate-neutral in 2050  (compare: \url{https://unfccc.int/process-and-meetings/the-paris-agreement/the-paris-agreement}). 
}
On the other hand, scholars have pointed to reductive policy measures to handle environmental limits \citep{Arrow2004AreMuch, Schor2005SustainableReduction, Dasgupta2021}. A reduction in work effort and consumption mitigates pollution by diminishing economic activity. Such a reduction could be achieved by using distortionary fiscal policy tools.
However, the economic literature on environmental policy has focused on the recomposing aspect of environmental policies: environmental taxes. %\citep{Fried2018ClimateAnalysis}. 
Given the exigency to act, this paper addresses the question whether fiscal policies can help meet climate targets.

I show analytically that, indeed, once 
labor supply is elastic, reductive policy measures optimally complement the environmental tax. 
It is established in the literature that absent any other distortion, an environmental tax equal to the social cost of the externality implements the efficient allocation. 
%Environmental taxes are perceived as a cost-effective way to reduce emissions. 
I argue that this result crucially depends on the use of lump-sum transfers to redistribute environmental tax revenues; transfers reduce labor supply through an income effect. Thus, indeed there is a role for reductive policy measures. 
%\textcolor{blue}{This is interesting independent of whether they are feasible or not. Could relate to the fact that there is a discussion how to use revenues. Yet, one might argue that we are always in a setting with distortionary labor income taxes; so that recycling lump-sum is never needed; numbers on size of expected revenues and government spending}
 As a consequence,  when lump-sum transfers are not in the policy set, environmental taxes are optimally combined with progressive labor income taxes. The use of income taxes as a reductive policy measure is not directly targeted at the externality: the motive for labor taxation follows indirectly from a distortion in labor supply due to lower income. Hence, (i) the two tax instruments are complements, and % to lower inefficiently high hours worked. 
% I show that redistributing environmental tax revenues through an income tax scheme allows to implement the efficient allocation. The optimal income tax scheme is progressive.
(ii) the optimal environmental policy equalizes the distribution of income as  a side effect. The theoretic analysis forms the first part of the paper.

In the second part, I study a quantitative model where environmental tax revenues are consumed by the government. Current debates on how to optimally use environmental tax revenues in politics \citep{Baker2017TheDividends} and academia \citep[e.g.][]{Fried2018TheGenerations, Carattini2018} motivate to focus on this policy regime. In this setting, I scrutinize whether progressive income taxes remain optimal in a more realistic model with endogenous growth and heterogeneous skills. This is unclear a-priori since a progressive tax scheme reduces incentives to innovate through a market size effect. Furthermore, a skill bias documented for the green sector \citep{Consoli2016DoCapital} in combination with a relatively more elastic high-skilled labor supply causes a higher tax progressivity to recompose the economic structure towards dirty production. The model suggests that despite these adverse mechanisms the optimal income tax scheme is progressive. To lower hours worked the government forfeits growth and accepts a less green production ratio.  
%\textit{I quantify the welfare gains of setting progressive income taxes to equal yyy in consumption equivalent measure. TO BE DONE  }

% relation to literature
The paper's results are relevant for the political and academic debate on how best to use environmental tax revenues. The paper points to the importance of lump-sum transfers as a reductive policy tool in the optimal environmental policy; an aspect which appears overlooked in today's discussion.%\footnote{\ POLICY debate; \cite{Fried2018TheGenerations}}
When thinking about how to recycle environmental tax revenues other than by lump-sum transfers, then, one should think about alternative reductive tools such as progressive labor income taxes. 
If the reductive part of the environmental policy is neglected, environmental taxes have to be higher to meet emission limits, as I demonstrate in the quantitative exercise.

The results address the academic debate on the so-called \textit{weak double-dividend} \citep[for example:][]{LansBovenberg1994EnvironmentalTaxation, LansBovenberg1996OptimalAnalyses}. The hypothesis posits that recycling environmental tax revenues to reduce pre-existing tax distortions is advantageous to recycling  revenues as lump-sum transfers. The rationale is that transfers decrease labor supply thereby diminishing the tax base of the income tax. A conflict between generating government funds and environmental protection arises. The findings in the present paper suggest a lower bound on the reduction in distortionary income taxes: when environmental tax revenues are not redistributed lump-sum, some reduction in labor supply via distortionary income taxes is in fact efficient from an environmental policy perspective. In other words, even if environmental tax revenues suffice to satisfy a government revenue requirement, there is a motive for progressive income taxation. 

The finding is especially interesting as the provision of the environmental public good and equity have been perceived as competing targets in the literature. First, when the poor consume more of the polluting good, a corrective tax is regressive \citep{Sager2019IncomeCurves, Fried2018ClimateAnalysis} \textit{Metcalf 2007, Hassett 2009 as  in Fried 2018}. Second and more indirectly, a fossil tax exerts efficiency costs by lowering labor efforts\footnote{\ The reduction in hours worked is per se not inefficient. The reduction in dirty production reduces the marginal product of labor, so that the disutility of labor is not compensated enough. However, when the government seeks to tax labor income using distortionary policy tools the reduced labor supply diminishes the tax base of the labor tax making it more costly to redistribute.} which again makes it more costly for the government to redistribute \citep{Dobkowitz2022}. 
In contrast to this literature, the present paper provides an argument for progressive income taxes even under perfect income risk sharing. This suggests a double dividend of redistribution: equity on the one hand and efficiency gains from less labor as part of the environmental policy.


\begin{comment}
When labor supply is fixed, environmental taxes alone can establish the efficient allocation in a representative agent economy absent fiscal distortions. Then, such a tax instrument is optimally set to the social cost of an externality, and originators internalize these social costs: the Pigou principle.
However, not redistributing environmental tax revenues reduces consumption below the efficient level and, as I demonstrate, the optimal environmental tax does not follow the Pigou principle.  If, on top, the  labor supply decision is endogenous, the environmental tax alone features too high labor supply. \tr{This results in too high environmental externality. \textbf{To be shown!}}

Lump-sum transfers of environmental tax revenues restore the efficient allocation: as households become richer, labor supply reduces. When lump-sum transfers are not available, the government can establish the efficient allocation by redistributing environmental tax revenues through an income tax scheme which I demonstrate to be progressive.

content...
\end{comment}

%%%%%%%%%%%%%%%%%%%%%%%%%%%%%%%%%%%%%%%
%\paragraph{Answer WHAT I DO }
%%%%%%%%%%%%%%%%%%%%%%%%%%%%%%%%%%%%%%%
%\subsubsection*{First part: Analytic stuff}
%\paragraph{simple model}
I propose a simple and comparatively general model to derive the main theoretical results. There are two intermediate sectors of production, one of which exerts a negative environmental externality. The environmental externality is the only distortion motivating government action. In the model, a Ramsey planner seeks to maximize welfare of a representative agent having  an environmental and a labor income tax at its disposal. The income tax scheme is generally non-linear and a common specification in the public finance literature \citep[e.g.][]{Benabou2002TaxEfficiency, Heathcote2017OptimalFramework}.
The model abstracts from  endogenous growth, inequality, and an exogenous government funding constraint. The last two are important abstractions since they traditionally motivate income taxation.

%\paragraph{Analytic findings 1}
%\textbf{MAIN finding: Progressive income tax} 
I show that the optimal labor income tax is progressive when no lump-sum transfers are available. 
The optimality of progressive income taxes results from inefficiently high labor supply. The mechanism is simple: when labor income taxes are not redistributed lump sum a wedge between households' shadow value of income and the social one emerges.  As a result, labor supply is too high. 
This is a novel motive for income taxation so far overlooked in the literature.
Following this intuition, environmental and labor income taxes complement each other in the optimal environmental policy absent lump-sum transfers. The higher environmental tax revenues, the higher the distortion in households' income and output. 
% I argue that environmental and income taxes are complements. The first is targeted at the environmental externality while the second serves to mitigate distortions in the labor market resulting from environmental taxation.
  
%\paragraph{Analytic findings 2}
%\textbf{Pigou principle violated absent lump-sum transfers and efficient allocation not feasible}
I consider two cases how environmental tax revenues are recycled. First, revenues are consumed by the government, and, second, the government redistributes revenues via the income tax scheme. In the first scenario, I find that the optimal environmental tax does not satisfy the Pigou principle; it does not necessarily equal the social cost of the externality. %As demonstrated by \textit{Pigou xxx}, setting the environmental tax equal to the marginal costs arising from the polluting activity that are not private (the so-called social cost) is optimal. The idea is that polluters behave as if internalizing the social cost of their action when confronted with the environmental tax. I show that when environmental tax revenues are consumed by the government, the Pigou principle is violated. 
In this case, the motive to increase private consumption by lowering environmental tax revenues makes a deviation of the environmental tax from the social cost of the externality optimal.
Furthermore, when labor supply is elastic non-redistribution of environmental tax revenues results in inefficiently high hours worked. Then, the optimal labor tax is progressive to diminish work effort closer to the efficient level.
Nevertheless, the efficient allocation is not feasible under this policy regime, since either consumption is inefficiently low or work effort is too high. 

%
%\paragraph{Analytic findings 3}
%\textbf{Efficient allocation can be implemented through redistribution via income tax scheme}
%The inefficiency of the optimal allocation when environmental tax revenues are consumed by the government motivates to point to a 
In the second scenario, the Ramsey planner redistributes environmental tax revenues through the income tax scheme, now the efficient allocation is attainable. The optimal income tax scheme is again progressive. By redistributing environmental tax revenues, consumption can be chosen efficiently high while the labor income tax scheme handles too high labor supply. In fact, redistribution via the income tax scheme incentivizes more labor supply. A progressive income tax scheme counteracts this tendency restoring the efficient level of hours worked. 
In this setting, the Pigou principle holds, since there is no trade-off between lowering the externality and the allocation of consumption.
%More precisely, I study the optimal policy mix of environmental and labor income taxes to meet emission limits. 
%I find that progressive labor income taxes are used in concert with fossil taxes to optimally reduce emissions. 
% First, I propose a tractable model to provide intuition for this result: Non-lump-sum redistribution of environmental tax revenues increase the gains from labor. Leisure becomes more expensive and households do not reduce their labor supply efficiently in response to the fossil tax. To reduce hours to the efficient level, income taxes complement the fossil tax to lower hours to the efficient level.  
% 
% 
% 
%Second, I assess the importance of this novel role for income taxes in a quantitative endogenous growth model:
%Even though a more progressive tax reduces research effort  and recomposes production towards the fossil sector, the optimal tax is progressive. 


%\subsubsection*{Quantitative exercise}
In this and the subsequent three paragraphs, I motivate the quantitative exercise and expound the model.
Even though I have shown theoretically that progressive income taxes form one pillar of the optimal environmental policy when environmental tax revenues are not redistributed lump sum, it is not clear whether progressivity remains optimal in a more realistic, quantitative model. 
Two countering mechanisms of tax progressivity shall be considered here.
First, endogenous growth could make a more regressive income tax optimal to incentivize innovation. The size of the market for successful innovation positively affects the profitability of research investment. Subsidizing labor supply through regressive taxes then boosts technological growth. Second, a skill bias in the green sector renders regressive income taxes advantageous by enhancing supply of the green sector-specific input good. In the model, high-skill workers supply more labor due to a wage premium. Therefore, they are more responsive to the substitution effect as income tax progressivity makes leisure cheaper. Then, the economy transitions to a higher dirty production share. Directed technical change amplifies this recomposing effect of the income tax. 

% relation to core model and to fried;
In light of these two mechanisms calling for income tax regressivity, I extend the core model studied in the analytical section by adding endogenous growth and skill heterogeneity. The resulting model builds on \cite{Fried2018ClimateAnalysis}. It enhances aforementioned work by an optimal policy analysis, the availability of income taxes, and skill heterogeneity. 
% no income inequality: trade off as in classical 
I maintain the assumption of a representative family which consists of two skill types. Within the family, all workers consume the same. Thus, workers remain perfectly insured against income differentials so that equity concerns do not drive the optimal policy. The trade-off shaping the optimal policy stays within the boundaries of traditional environmental policy considerations: growth versus externality mitigation \citep{Stokey1998AreGrowth, Jones2016LifeGrowth, Acemoglu2012TheChange}.  

%\paragraph{How the externality is modeled}
In the quantitative model, the government cares about the externality due to an exogenous constraint on emissions and not via household utility. This approach, known as a cost-effectiveness approach in the environmental literature, has the advantage of reducing modeling and parametric uncertainty when specifying how greenhouse-gas emissions affect the climate and the damages it produces in terms of well-being and production. Instead, the cost-effectiveness approach uses estimated emission targets stemming from meta studies of more complex integrated assessment models. Additionally, the emission limits I use to calibrate the model are designed to meet climate targets constituting a politically relevant environment. 
% think about the relation of an absolute emission target and not reducing hours

%\paragraph{Endogenous growth}
Allowing for endogenous growth is important to take seriously the possibility of green growth to keep consumption high while meeting emission targets. Endogenous growth introduces dynamics into the model since today's level of technology positively depends on yesterday's technology: \tr{ a \textit{standing on the shoulder of giants} mechanism. LOOK THIS UP \cite{Acemoglu2012TheChange}} 


%The model differentiates between high- and low-skilled labor to account for a skill bias found for the green sector \citep{Consoli2016DoCapital}. This asymmetry of sectors renders regressive taxes a tool to lower relative production costs in the green sector: high-skill workers reduce their labor supply more in response to a more progressive tax as leisure is more valuable to them. This recomposing mechanism counteracts intentions to lower emissions. %, a regressive tax functions as a green subsidy. % In fact, there is an externality arising from high-skill labor supply as it shapes the share of fossil to green energy production. 
%On the other hand, progressive income taxes lower aggregate production by diminishing the price of leisure. 
%Endogenous growth amplifies the repercussions of progressive income taxes:
%First, a lower labor supply reduces the profitability of research in general. By reducing hours worked the planner sacrifices technological progress. Secondly, the recomposing effect is aggravated as research is directed towards the sector with the increased labor share, i.e. the fossil sector.

%\paragraph{Calibration}
The model is calibrated to the US in the baseline period from 2015 to 2019. To do so, I proceed in two steps. First, I set certain parameters to values found in the literature. Most importantly, I use reasonable values of production and growth processes found in \cite{Fried2018ClimateAnalysis}. % who conducts a rigorous calibration exercise. 
With these parameter values at hand, I match the share of high skill in the green and non-green sectors building on \cite{Consoli2016DoCapital}. The emission target is set to the values suggested in the latest IPCC draft on mitigation pathways \citep{IPCC2022}: A 50\% reduction by 2030 relative to 2019 levels and  net-zero emissions from 2050 onward.

%\paragraph{Quantitative exercise}
I perform the following quantitative exercise. 
 The planner sets the environmental tax and the progressivity parameter of the non-linear income tax scheme to maximize welfare while satisfying the emission limit. Environmental tax revenues are consumed by the government. As discussed in the analytical section, under this policy regime the efficient allocation is not feasible, but, it is  more policy relevant. 

I solve the Ramsey model explicitly for 60 years starting from 2020.\footnote{\ The focus on utility of households living during the transition is legitimate given findings which suggest that this generation is the most affected. Furthermore, from a political economy perspective, environmental policies need to be supported by the current living population. \tr{Read \cite{Fried2018TheGenerations} and \cite{Kotlikoff2021MakingWin} on this point! }}
Under the usual assumption that taxes remain fixed afterwards in order to solve for the balanced growth path \textit{look at manuelli rossi; also look at \cite{Conesa2009TaxingAll} who make some argument about bgp preferences: to allow for a broader set of parameters, they abstract from growth}, the emission limit is violated due to market forces.
Furthermore, as the emission limit is absolute, an ever increasing technology level (which is the case under decreasing returns to research), the use of fossil labor and machines has to decline towards zero.

 Instead, I constrain the planner to focus on the 60 year period in which the economy needs to transition to the zero-emission target. To make the planner care about future periods, I adopt a sustainability-type of condition: the potential to grow in terms of technology has to be at least as high as today.\footnote{\ Absent this condition, technological progress today is less valuable. The value of the technology level for future periods - due to the dynamic structure of innovation - is not taken into account. } 
 What is left to future generations in terms of production has some lower bound. 
 The productive asset inherited from predecessors is the technology level in the present model. 
 
 Defining the balanced growth path as in \cite{Fried2018ClimateAnalysis} is not possible, as she requires output ratios to stay constant. This assumption would force the economy not to grow any further once fossil output has reached its limit. 

When no BGP is feasible, then alternative, eg some sustainability condition. But, whenever there is growth in just one period, the next generation will have potential to grow as much. 
Successors will always have as much productive resources as predecessors left.  

 The planner is constrained by a sustainability motive to leave at least as much resources to future periods than for the explicitly modeled periods.\footnote{\ \tr{To be done}}
This approach has the advantage of not having to assume the existence of a balanced growth path. Given that one factor of production, fossil energy, is bounded by an exogenous limit, I seek to stay agnostic on this aspect.
\tr{\textit{Alternative: assume the economy reaches a BGP in some future period; all variables grow at constant rates. Possible?}}

% main finding; then understand why; what are the drivers: how does the result change as things are added or taken from the model
%\paragraph{Findings}
%\textbf{Progressive income tax}
I find that the planner chooses a progressive income tax despite its repercussions on growth and emissions. Indeed, this policy increases the share of fossil to green energy and reduces research efforts and consumption. %As the emission limit becomes tighter over time, the Ramsey planner augments both the environmental tax and income tax progressivity. 
%In general, income tax progressivity and the environmental tax appear to be complements: as the emission limit becomes tighter, the two increase. Yet, with introduction of the net-zero emission limit, income tax progressivity decreases on impact. 
%The positive correlation between income tax progressivity and the environmental tax is in line with the analytical finding. 
 

% comparison to a model without income tax
To investigate the importance and the welfare gains of income taxes, I rerun the main experiment but let the government only choose the environmental tax; labor is not taxed. The welfare gains from the use of the environmental tax amount to xxx. \textit{\tr{do CEV}}

% complements
The results suggest a clear cut between responsibilities of policy instruments: The environmental tax targets the externality, while the income tax handles the inefficiency arising from environmental tax revenues. When depriving the government from an income tax scheme, the environmental tax remains largely unchanged compared to the benchmark setting. 

% importance of endogenous growth
To shed light on the role of endogenous growth and skill heterogeneity, I rerun the analysis in modified models simplifying on either dimension. 
Endogenous growth accounts for the gradual increase in labor tax progressivity. 
The reason is that more research effort loses in value as consumption increases. As dynamic spill overs of the research process advance, an additional unit of research effort is not compensated by enough from more consumption. \textit{\tr{This might be due to the missing continuation value! }}

%To understand this finding, it is informative to look at a setting with lump-sum transfers. Then, the motive to use lump-sum taxes to reduce labor supply vanishes. The labor tax can, however, be used to boost growth through a market-size mechanism by enhancing labor supply. Indeed, in such a setting the optimal labor income tax is progressive. Yet, regressivity reduces similar to the result without lump-sum transfers. The reason for this gradual increase is that in  the emission limit prevents the usage of labor income taxes to boost growth.
%More labor supply causes more emissions especially the more progressed the technology. With only a labor income tax as a tool to raise growth, accelerating technology growth issions. Indeed, this policy increases the share of fossil to green energy and reduces research efforts and consumption. %As the emission limit becomes tighter over time, the Ramsey planner augments both the environmental tax and income tax progressivity. 
%In general, income tax progressivity ands not feasible as it is concomitant with more production and emissions. 
%\tr{Should find no gradual decrease in labor income tax absent the emission limit then!}
%If I find it, though, then it is that the mmore in research is not 
%% Lump-sum taxes

\tr{Continue writing up results with lump-sum transfers and skill heterogeneity}

% sensitivity
\paragraph{Sensitivity}
\begin{itemize}
	\item utility specification (Building on Bick can think of European version when substitution effect is stronger)
	\item spillovers across scientists: with positive spillovers potentially no growth
	\item counterfactual technology gap
\end{itemize}

Since the target of the labor tax in the environmental setting presented here is to align hours worked with their efficient level, results are sensitive to the elasticity of labor with respect to after-tax wages. 
Quantitative finding to be shaped by income and substitution effect!
Literature on how households react to changes in income \cite{Bick2018HowImplications} and \cite{Boppart2019LaborPerspectiveb}


%\paragraph{Literature}
%\paragraph{Literature}

\paragraph{Optimal environmental policy}
\textbf{Main claim: focus on environmental taxes and recomposition}
\begin{itemize}
	\item with exogenous growth
	\item with endogenous growth
\end{itemize}

In general, papers on optimal environmental policy focus on the optimal environmental tax and analyze settings with inelastic labor supply \citep{Golosov2014OptimalEquilibrium, Acemoglu2012TheChang, Fried2018ClimateAnalysis}. Therefore, the main finding of the present paper, the necessity of reductive policy measures to implement the efficient allocation, complement this literature. Furthermore, I argue that the Pigou principle does generally not apply when no lump-sum transfers are available.  


\paragraph{Recycling of environmental tax revenues}
\textbf{Main claim: they overlook that when lump-sum transfers are not available, then labor supply is inefficiently high, and that env. tax exceeds (?) scc in optimal policy}
\begin{itemize}
	\item in general: \cite{Fried2018TheGenerations}
	\item double dividend literature
\end{itemize}
These findings have important consequences for the literature on the so-called double dividend of environmental policy and the question how to recycle environmental tax revenues. While this literature argues for the recycling of environmental tax revenues to lower pre-existing tax distortions, my paper constitutes an argument for a lower bound on distortionary income taxes: some reduction of labor supply is in fact efficient. 
Furthermore, when revenues are not redistributed lump-sum, the Pigouvian tax does not implement the efficient allocation. 

\paragraph{Environmental protection and inequality}
\ar 1) Inequality and environment as competing goods.

In the literature discussing environmental policies in an unequal framework, a competition between equity and environmental good provision have been discussed. 
This trade-off can be separated into (i) the competition for public funds \citep{LansBovenberg1996OptimalAnalyses, Jacobs2019RedistributionCurves} and (ii) effects of either environmental policies on equity or equalizing policies on environmental quality \citep{Jacobs2019RedistributionCurves, Sager2019IncomeCurves, Dobkowitz2022}. 

Since hours are inefficiently high, equalizing policies become part of the optimal environmental policy. As a byproduct, the distribution of income becomes more equal.


\ar 2) Inequality to shape effects of environmental policies and effect of fiscal policy on environment due to heterogeneity

 Furthermore, the differentiation of skills and the skill-bias of the green sector in the paper give rise to a new channel through which labor taxation affects environmental protection. The literature has primarily focused on a demand channel arising from non-linear Engel curves through which inequality and redistribution shape the degree of dirty production in the economy.   

%\textbf{Non-linear Engel Curves: redistribution}
\cite{Jacobs2019RedistributionCurves} the motive to redistribute and to provide an environmental good compete for government resources due to a negative effect of environmental taxes on the wage rate. Even with lump-sum transfers, the optimal environmental tax does not follow the Pigou principle when the government seeks to enhance equity.

\cite{Sager2019IncomeCurves} argues empirically, that redistribution to poorer households may result in a higher demand for polluting goods. 
\paragraph{Pubic Finance literature}
\textbf{I add: a new perspective on labor income taxes as a tool to lower inefficiently high hours worked. }

 \cite{Heathcote2017OptimalFramework}, \cite{Loebbing2019NationalChange}

\paragraph{Reductive policies in the literature}
The finding relates to the literature discussing rationales for the usage of reductive policy measures. These arise from o
Negative externalities of consumption and hours worked such as
 envy \cite{Alvarez-Cuadrado2007EnvyHours}, habits \cite{Ravn2006DeepHabits} \tr{Check if this is on inefficiency} or a positive externality of leisure \cite{Alesina2005WorkDifferent}. The present paper relates to this literature by identifying an externality of work which emerges from the existence of an environmental externality of production. In addition, once an environmental tax recomposes production towards a cleaner alternative, the wage rate understates the marginal product of labor. \tr{This needs to be made clearer.}
\begin{itemize}
	\item literature which \cite{Alvarez-Cuadrado2007EnvyHours}
\end{itemize}: includes deeped understanding of Bov De Mooji
\paragraph{Literature}

The paper relates to several strands of literature. 
First, to the literature on environmental policies. Second, it connects to the literature on how to recycle environmental tax revenues. Third, as the paper combines environmental and fiscal policies it naturally connects to the public finance literature. Finally, the results speak to the literature discussing overconsumption  which may be social preferences or environmental constraints. 

 
\begin{itemize}
	\item How to use environmental tax revenues \citep{Fried2018TheGenerations}
	\item Optimal environmental policy \ar focuses on environmental taxes
	\item weak do
\end{itemize}


In general, macro papers on (optimal) environmental policy focus on environmental taxation and analyze settings with inelastic labor supply \citep{Golosov2014OptimalEquilibrium, Acemoglu2012TheChang, Fried2018ClimateAnalysis, Acemoglu2016TransitionTechnology}. The mentioned papers assume lump-sum transfers of environmental tax revenues, hence endogenizing labor supply would not affect the optimal policy; yet, the role of lump-sum transfers changes to a reductive policy measure.
% 
% Acemoglu 2016 have lump-sum transfers and taxes
% Acemoglu Aghion 2012: lump-sum transfers, no optimal policy
%Golosov: hightlight the need of lump-sum transfers! but exogenous labor supply
% Therefore, the main finding of the present paper, the necessity of reductive policy measures to implement the efficient allocation, complements this literature. 
%Especially, when environmental tax revenues are not redistributed lump-sum in these papers, a variable labor supply would give an argument for labor income taxation. 

%
%Furthermore, I argue that the Pigou principle does generally not apply when no lump-sum transfers are available.  

Another focus of the literature on environmental policy is the redistribution of environmental tax revenues. In contrast to the previous strand of literature, this one generally assumes labor supply to be elastic. 
The dominant focus of this literature is the weak double dividend of environmental taxes \citep{LansBovenberg1994EnvironmentalTaxation, LansBovenberg1996OptimalAnalyses, Bovenberg2002EnvironmentalRegulation,  Barrage2019OptimalPolicy}: given an exogenous government funding constraint it is optimal to recycle environmental tax revenues to lower distortionary labor income taxes as opposed to higher lump-sum transfers. The latter decreases labor supply through an income effect thereby decreasing the tax base of the labor income tax. Consequently, it becomes more expensive for the government to generate revenues.
The weak double-dividend literature rests on the assumption that no lump-sum transfers and taxes are available. Yet, when this is the case, this paper shows that absent a government funding constraint, distortionary taxes should be set to reduce labor supply: some reduction in hours is in fact efficient. Hence, this paper provides an upper bound on the reduction of distortionary income taxes. This becomes clear when environmental tax revenues suffice to meet the government's funding constraint, then labor supply would be inefficiently high when the labor income tax is unused. 
The analytical literature on optimal environmental policy treats the labor income tax as pre-existing \citep{LansBovenberg1994EnvironmentalTaxation, LansBovenberg1996OptimalAnalyses} in the settings when no lump-sum transfers are possible.
 

Building on the weak double-dividend literature, \cite{Fried2018TheGenerations} compare distinct scenarios of how to recycle environmental tax revenues and investigate the impact on inter- and intragenerational inequality in an overlapping generations model. Lump-sum transfers are preferred by the  living generation. 
In this literature, the advantage of different recycling means is often evaluated with respect to equity or political feasibility \cite{Carattini2018, VANDERPLOEG2022103966}. The present paper employs an efficient allocation as a benchmark to assess lump-sum transfers, government consumption, and redistribution via the income tax scheme. 

\cite{VANDERPLOEG2022103966} do comment on efficiency costs. 
What they do: 
structural estimate effects of carbon tax and lump-sum redistribution on households across income distribution; "better off" measured by utility: to capture both: consumption and labor supply!
The term efficiency is synonym to labor supply redcutions. Abstracting from the possibility that labor supply reductions may be advantageous;\tr{ do they assume an endogenous funding condition?} The politically most feasible recycling is through lower income tax rates, which boosts labor supply and economic activity, but hurts the poor. If equity is relevant, the government should do split env tax revenues: some recycling as carbon dividend, the other to lower income taxes.

\tr{1) What if there is a pre-existing income tax but not gov funding constraint? Do BLÖDsinnn; maybe not since could still find an increase in labor income tax if optimal level is above initial level;\\
 2) look at a policy where income taxes are used to fund government spending \ar then this would generate more gov. revenues }

Why do bov and de Mooji not find that the reduction in labor tax revenues depends on the level of the income tax? \ar because its a non-formal argument

equity: horowitz 2017: equity gains from lump-sum redistribution

Metcalf 2007/ 2008
Kotlikoff making carbon pricing a generational win win

\cite{LansBovenberg1996OptimalAnalyses}: they focus on how the optimal environmental policy deviates from the Pigou principle due to pre-existing distortionary taxes; 

\textbf{Goulder1995} defines the weak double dividend as: recycling revenues to lower pre existing tax distortions: one achieves \textbf{cost savings} relative to the case where the tax revenues are returned to taxpayers in lump-sum fashion; 
\textbf{It seems like the weak double dividend is defined in terms of costs}; costs (i.e. non-environmental costs) are the reduction in price times quantity relative to the equilibrium without tax; p.18: \textbf{the double dividend literature focuses on non-environmental costs of the environmental policy. And earlier works focus on consumption and output as only measure of gains/ missing gains from leisure. } 
Linking paper to double dividend literature means to compare efficiency gains to costs. Leisure versus consumption \ar but I show analytically that consumption costs are lower. 

\textcolor{blue}{so fare havent seen a paper which analytically shows the weak double dividend holds \ar could be that they indeed miss the lower bound, size dependency of advantage to reduce labor taxes}

\textit{To be fair, the double dividend literature focuses on cost advantages by using environmental tax revenues to substitute for distortionary labor income taxes. However, it remains unmentioned that under the assumption of elastic labor supply, which the literature necessarily assumes, the environmental tax alone is not efficient. }

\textbf{Bovenber 1998}: "\textit{environmental taxes are  generally  an  efficient  instrument  for  protecting  the  environment}"
\tr{This statement is wrong! they are only efficient in combination with reductive measures \ar there is no double-dividend as - to efficiently reduce emissions - environmental tax revenues have to be redistributed lump-sum}
\textcolor{blue}{the question arises what is better from an equity perspective: lump-sum transfers or additional progressive taxes?} Evaluate by looking at high and low skill wages. 
\\
Environmental tax revenues are not a free lunch. By not redistributing them lump-sum,there are efficiency costs because work effort would be too high.




\textit{Citation in Fried: Pigou 1920, Dales 1968, Montgomery 1972, Baumol and Oates 1988 }: "\textit{Establishin a price on carbon [...] is well understood to be the most efficient approach for reducing greenhouse gas emissions.}"

\paragraph{Reduction policies}

\paragraph{Outline}
The remainder of the paper is structured as follows. The next section \ref{sec:mod_an} presents the core model which is used to derive the analytical results in section \ref{sec:theory}. In section \ref{sec:model}, I extend the model to a quantitative framework and calibrate it. I present and dicuss the quantitative results in section \ref{sec:res}. Section \ref{sec:con} concludes.