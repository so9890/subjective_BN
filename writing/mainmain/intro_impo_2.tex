\section{Introduction}

%\paragraph{Recomposition versus reduction discussion plus ever stricter emission limits}
The latest assessment report of the Intergovernmental Panel on Climate Change (IPCC) \citep{IPCC2022} highlights the decreasing likelihood of meeting the Paris Agreement and limiting climate warming to 1.5°.%relative to the previous report from 2018 \citep{Rogelj2018MitigationDevelopment.}.
\footnote{ \ The Paris Agreement of 2015 formulates clear political goals to mitigate climate change. Under this treaty, states have agreed on a legally binding maximum increase in temperature to well below 2°C, preferably 1.5° over pre-industrial levels, and the global community seeks to be climate-neutral in 2050  (compare:\\ \url{https://unfccc.int/process-and-meetings/the-paris-agreement/the-paris-agreement}). 
}
On the other hand, scholars have pointed to reductive policy measures to handle environmental limits \citep{Arrow2004AreMuch, Schor2005SustainableReduction, Dasgupta2021}. A reduction in work effort and consumption would mitigate pollution by diminishing economic activity. Such a reduction in production could be achieved by using distortionary policy tools.
However, the economics literature on environmental policy has by and large focused on carbon taxes \citep{Golosov2014OptimalEquilibrium, Barrage2019OptimalPolicy}. %\citep{Fried2018ClimateAnalysis}. 
Given the urgency to act, this paper addresses the question whether fiscal policies can help meet climate targets.

In theory, corrective, environmental taxes can establish the efficient allocation in a representative agent economy. Absent inequality, such a tax instrument is optimally set to the social cost of an externality. Originators then internalise these costs in addition to their private ones. However, a role for income taxes arises when government revenues cannot be redistributed lump sum.

%\paragraph{Answer WHAT}
More precisely, I study the optimal policy mix of fossil and labour income taxes to meet emission limits. 
I find that progressive labour income taxes are used in concert with fossil taxes to optimally reduce emissions. 
 First, I propose a tractable model to provide intuition for this result: Non-lump-sum redistribution of environmental tax revenues increase the gains from labour. Leisure becomes more expensive and households do not reduce their labour supply efficiently in response to the fossil tax. To reduce hours to the efficient level, income taxes complement the fossil tax to lower hours to the efficient level.  
Second, I assess the importance of this novel role for income taxes in a quantitative endogenous growth model:
Even though a more progressive tax reduces research effort  and recomposes production towards the fossil sector, the optimal tax is progressive. 

The finding is especially interesting as the provision of the environmental public good and equity have been perceived as competing targets in the literature. First, when the poor consume more of the polluting good, a corrective tax is regressive. Second and more indirectly, a fossil tax exerts efficiency costs by lowering labour efforts\footnote{\ The reduction in hours worked is per se not inefficient. A social planner reduces hours worked when confronted with an environmental externality. The reduction in dirty production reduces the marginal product of labour, so that the disutility of labour is not compensated enough. However, when the government seeks to tax labour income using distortionary policy tools the reduced labour supply diminishes the tax base of the labour tax making it more costly to redistribute.} which again makes it more costly for the government to redistribute. 
In contrast to this literature, the present paper provides an argument for progressive income taxes even under perfect income risk sharing. This suggests a double dividend of redistribution: utility gains from more leisure.

\paragraph{Tractable model}
To highlight the novel mechanism interacting 

Even though progressive income taxes implement efficiency gains, it is not clear whether the optimal labour income tax is indeed progressive in a more realistic quantitative model as I will expound when describing the quantitative model in the following paragraphs. 

\paragraph{Quantitative model in more details}
I embed a non-linear income tax scheme commonly studied in the public finance literature \citep[e.g.][]{Heathcote2017OptimalFramework} into an endogenous growth model designed to investigate environmental policies building on \cite{Fried2018ClimateAnalysis}.  
Allowing for endogenous growth is important to take seriously the possibility of green growth to keep consumption high while meeting emission targets.
The government is characterised as a Ramsey planner who seeks to maximise Utilitarian social welfare but is constrained by an exogenous emission target. To abstract from inequality as a determinant of tax progressivity, the economy is populated by a representative family. Yet, the family supplies two types of skill. 

The model differentiates between high- and low-skilled labour to account for a skill bias found for the green sector \citep{Consoli2016DoCapital}. This asymmetry of sectors renders regressive taxes a tool to lower relative production costs in the green sector: high-skill workers reduce their labour supply more in response to a more progressive tax as leisure is more valuable to them. This recomposing mechanism counteracts intentions to lower emissions. %, a regressive tax functions as a green subsidy. % In fact, there is an externality arising from high-skill labour supply as it shapes the share of fossil to green energy production. 
%On the other hand, progressive income taxes lower aggregate production by diminishing the price of leisure. 
Endogenous growth amplifies the repercussions of progressive income taxes:
First, a lower labor supply reduces the profitability of research in general. By reducing hours worked the planner sacrifices technological progress. Secondly, the recomposing effect is aggravated as research is directed towards the sector with the increased labour share, i.e. the fossil sector.

\paragraph{Calibration}
The calibration of the model proceeds in two steps. First, I set certain parameters to values found in the literature. Most impportantly, I use reasonable values of production and growth processes found in \cite{Fried2018ClimateAnalysis}. % who conducts a rigorous calibration exercise. 
With these parameter values at hand, I match the share of high skill in the green and non-green sectors building on \cite{Consoli2016DoCapital}. The emission target is set to the values suggested in the latest IPCC draft on mitigation pathways: A 50\% reduction by 2030 relative to 2019 levels and  net-zero emissions from 2050 onwards \citep{IPCC2022}.

\paragraph{Main findings quantitative analysis}
\textbf{qualitative i}
Despite the repercussions of income taxes on growth and emissions, the planer chooses a progressive income tax. 
Indeed, this policy increases the share of fossil to green energy and reduces research efforts and consumption. Taking these indirect effects of labor supply into account, the optimal tax is progressive. 

\textbf{qualitative i}
The results suggest a clear cut between responsibilities of policy instruments: The environmental tax targets the externality, while the income tax handles the inefficiency arising from environmental tax revenues.
\tr{To do: investigate if even with lump sum redistribution there is a role for income tax in full model.}
\textbf{quantitative:} the availability of an income tax increases social welfare by 0.11\% over the period from 2020 to 2080 which corresponds to a consumption equivalent of xxx. 

\paragraph{Quantitative finding to be shaped by income and substitution effect!}
Literature on how households react to changes in income \cite{Bick2018HowImplications} and \cite{Boppart2019LaborPerspectiveb}


\paragraph{Literature}
\paragraph{Lit: environmental policy}
%Inequality-environment nexus: normally motivated by a demand-side perspective; in this project I focus on a supply side explanation
The literature studying the interaction of environmental taxes and income taxes generally focuses on the gap between the social cost of carbon and the optimal tax \citep{Bovenberg1997EnvironmentalGrowth,  Kaplow2012OPTIMALTAXATION, Jacobs2019RedistributionCurves, Barrage2019OptimalPolicy}. Labour income taxes form a passive component of these analyses. 
In contrast, I focus the paper on the role of income taxes in the optimal environmental policy. The inefficiency of environmental taxes arises absent pre-existing income tax distortions or the motive to redistribute.
In difference to this literature, where the presence of a distortionary income tax shapes the optimal level of the environmental tax, the existence of the environmental tax rationalises a progressive income tax in the present paper.

\paragraph{Environment and endogenous growth}

\paragraph{Public finance}
An equity-efficiency trade-off is central to the discussion of optimal labour income taxation and tax progressivity in the public finance literature.  The benefits of labour taxes and progressivity arise, inter alia, from redistribution. %and from generating government revenues. 
With concave utility specifications full redistribution is efficient. However, the optimal tax system does not feature full redistribution when labour supply is endogenous. Instead, redistribution is traded off against aggregate output as individuals reduce their labour supply and skill investment in response to labour income taxation. 

One closely related work is \cite{Loebbing2019NationalChange} who studies optimal income taxation in a model of directed technical change. The redistributive effect of tax progressivity is amplified through a compression of the wage rate distribution xxx