\section{Introduction}
\tr{I show that carbon taxes are only efficient if lump-sum transfers are available.}


%\paragraph{Recomposition versus reduction discussion plus ever stricter emission limits}
The latest assessment report of the Intergovernmental Panel on Climate Change (IPCC) \citep{IPCC2022} highlights the urgency to reduce greenhouse-gas emissions.%relative to the previous report from 2018 \citep{Rogelj2018MitigationDevelopment.}.
\footnote{ \  The report stresses the decreasing likelihood of meeting the Paris Agreement and limiting climate warming to 1.5°. The Paris Agreement of 2015 formulates clear political goals to mitigate climate change. Under this treaty, states have agreed on a legally binding maximum increase in temperature to well below 2°C, preferably 1.5° over pre-industrial levels, and the global community seeks to be climate-neutral in 2050  (compare:\\ \url{https://unfccc.int/process-and-meetings/the-paris-agreement/the-paris-agreement}). 
}
On the other hand, scholars have pointed to reductive policy measures to handle environmental limits \citep{Arrow2004AreMuch, Schor2005SustainableReduction, Dasgupta2021}. A reduction in work effort and consumption mitigates pollution by diminishing economic activity. Such a reduction could be achieved by using distortionary fiscal policy tools.
However, the economic literature on environmental policy has focused on the recomposing aspect of optimal environmental policy: environmental taxes. %\citep{Fried2018ClimateAnalysis}. 
Given the urgency to act, this paper addresses the question whether fiscal policies can help meet climate targets.

I show in this paper that  once 
labor supply is elastic, reductive policy measures become a necessary complement to an environmental tax to implement the efficient allocation. 
When labor supply is fixed, environmental taxes alone can establish the efficient allocation in a representative agent economy absent fiscal distortions. Then, such a tax instrument is optimally set to the social cost of an externality, and originators internalize these social costs: the Pigou principle.
However, not redistributing environmental tax revenues reduces consumption below the efficient level and, as I demonstrate, the optimal environmental tax does not follow the Pigou principle.  If, on top, the  labor supply decision is endogenous, the environmental tax alone features too high labor supply. \tr{This results in too high environmental externality. \textbf{To be shown!}}


Lump-sum transfers of environmental tax revenues restore the efficient allocation: as households become richer, labor supply reduces. When lump-sum transfers are not available, the government can establish the efficient allocation by redistributing environmental tax revenues through an income tax scheme which I demonstrate to be progressive.
The optimal environmental policy, therefore, has an equalizing effect on the income distribution.

These findings have important consequences for the literature on the so-called double dividend of environmental policy tools and the question how to recycle environmental tax revenues. While this literature argues for the recycling of environmental tax revenues to lower pre-existing tax distortions, my paper constitutes an argument for a lower bound on distortionary income taxes: some reduction of labor supply is in fact efficient. 
Furthermore, when revenues are not redistributed lump-sum, the Pigouvian tax does not implement the efficient allocation. 

%%%%%%%%%%%%%%%%%%%%%%%%%%%%%%%%%%%%%%%
%\paragraph{Answer WHAT}
%%%%%%%%%%%%%%%%%%%%%%%%%%%%%%%%%%%%%%%
More precisely, I study the optimal policy mix of fossil and labor income taxes to meet emission limits. 
I find that progressive labor income taxes are used in concert with fossil taxes to optimally reduce emissions. 
 First, I propose a tractable model to provide intuition for this result: Non-lump-sum redistribution of environmental tax revenues increase the gains from labor. Leisure becomes more expensive and households do not reduce their labor supply efficiently in response to the fossil tax. To reduce hours to the efficient level, income taxes complement the fossil tax to lower hours to the efficient level.  
Second, I assess the importance of this novel role for income taxes in a quantitative endogenous growth model:
Even though a more progressive tax reduces research effort  and recomposes production towards the fossil sector, the optimal tax is progressive. 

The finding is especially interesting as the provision of the environmental public good and equity have been perceived as competing targets in the literature. First, when the poor consume more of the polluting good, a corrective tax is regressive. Second and more indirectly, a fossil tax exerts efficiency costs by lowering labor efforts\footnote{\ The reduction in hours worked is per se not inefficient. A social planner reduces hours worked when confronted with an environmental externality. The reduction in dirty production reduces the marginal product of labor, so that the disutility of labor is not compensated enough. However, when the government seeks to tax labor income using distortionary policy tools the reduced labor supply diminishes the tax base of the labor tax making it more costly to redistribute.} which again makes it more costly for the government to redistribute. 
In contrast to this literature, the present paper provides an argument for progressive income taxes even under perfect income risk sharing. This suggests a double dividend of redistribution: utility gains from more leisure.

\paragraph{Tractable model}
To highlight the novel mechanism interacting 

Even though progressive income taxes implement efficiency gains, it is not clear whether the optimal labor income tax is indeed progressive in a more realistic quantitative model as I will expound when describing the quantitative model in the following paragraphs. 

\paragraph{Quantitative model in more details}
I embed a non-linear income tax scheme commonly studied in the public finance literature \citep[e.g.][]{Heathcote2017OptimalFramework} into an endogenous growth model designed to investigate environmental policies building on \cite{Fried2018ClimateAnalysis}.  
Allowing for endogenous growth is important to take seriously the possibility of green growth to keep consumption high while meeting emission targets.
The government is characterised as a Ramsey planner who seeks to maximise Utilitarian social welfare but is constrained by an exogenous emission target. To abstract from inequality as a determinant of tax progressivity, the economy is populated by a representative family. Yet, the family supplies two types of skill. 

The model differentiates between high- and low-skilled labor to account for a skill bias found for the green sector \citep{Consoli2016DoCapital}. This asymmetry of sectors renders regressive taxes a tool to lower relative production costs in the green sector: high-skill workers reduce their labor supply more in response to a more progressive tax as leisure is more valuable to them. This recomposing mechanism counteracts intentions to lower emissions. %, a regressive tax functions as a green subsidy. % In fact, there is an externality arising from high-skill labor supply as it shapes the share of fossil to green energy production. 
%On the other hand, progressive income taxes lower aggregate production by diminishing the price of leisure. 
Endogenous growth amplifies the repercussions of progressive income taxes:
First, a lower labor supply reduces the profitability of research in general. By reducing hours worked the planner sacrifices technological progress. Secondly, the recomposing effect is aggravated as research is directed towards the sector with the increased labor share, i.e. the fossil sector.

\paragraph{Calibration}
The calibration of the model proceeds in two steps. First, I set certain parameters to values found in the literature. Most impportantly, I use reasonable values of production and growth processes found in \cite{Fried2018ClimateAnalysis}. % who conducts a rigorous calibration exercise. 
With these parameter values at hand, I match the share of high skill in the green and non-green sectors building on \cite{Consoli2016DoCapital}. The emission target is set to the values suggested in the latest IPCC draft on mitigation pathways: A 50\% reduction by 2030 relative to 2019 levels and  net-zero emissions from 2050 onwards \citep{IPCC2022}.

\paragraph{Main findings quantitative analysis}
\textbf{qualitative i}
Despite the repercussions of income taxes on growth and emissions, the planer chooses a progressive income tax. 
Indeed, this policy increases the share of fossil to green energy and reduces research efforts and consumption. Taking these indirect effects of labor supply into account, the optimal tax is progressive. 

\textbf{qualitative i}
The results suggest a clear cut between responsibilities of policy instruments: The environmental tax targets the externality, while the income tax handles the inefficiency arising from environmental tax revenues.
\tr{To do: investigate if even with lump sum redistribution there is a role for income tax in full model.}
\textbf{quantitative:} the availability of an income tax increases social welfare by 0.11\% over the period from 2020 to 2080 which corresponds to a consumption equivalent of xxx. 

\paragraph{Quantitative finding to be shaped by income and substitution effect!}
Literature on how households react to changes in income \cite{Bick2018HowImplications} and \cite{Boppart2019LaborPerspectiveb}


\paragraph{Literature}
\tr{This observation relates to the literature in several ways: first, the literature which discusses the optimal recycling of carbon tax revenues. Because when revenues are not recycled as lump-sum transfers, then labor supply is inefficiently high and additional policy measures are necessary to implement the efficient allocation today. In general, the literature argues that environmental tax revenues should be used to reduce pre-existing tax distortions. This tax distortions arise under the assumption that no lump-sum transfers are available. Yet, when this is the case, distortionary taxes should be set in a way that reduces production. Hence, I provide an argument for the optimal size of distortionary labor income taxes. In other words: because lump-sum transfers are not available, the literature argues, the government should use corrective tax revenues to lower pre-existing tax distortions. But by how much? I argue, that there is an optimal size of positive tax distortions when lump-sum transfers are not available. Hence, under the premise of non-lump sum transfers, distortionary labor income taxes arise as an optimal policy tool even absent an exogenous financing condition or inequality. }

 The paper relates broadly to the literature discussing optimal environmental policies. I separate the two into two strandy: one with inelastic and one with elastic labor supply. 
  \paragraph{Optimal environmental policy: exogenous labor supply}
The main finding of the present paper is generally overlooked in this literature because labor supply is inelastic \citep{Golosov2014OptimalEquilibrium, Acemoglu2012TheChang, Fried2018ClimateAnalysis}. 
 
\paragraph{Lit: environmental policy and distortionary fiscal setting}
\tr{}
\begin{itemize}
	\item Williams 2013 Double dividend 
	\item talk to Mireille Chiroleu Assouline: paper on double dividend
	\item Mireille with Aubert or Fodha (PSE)
\end{itemize}
%Inequality-environment nexus: normally motivated by a demand-side perspective; in this project I focus on a supply side explanation
labor supply becomes elastically in the literature studying the interaction of environmental taxes and distortionary taxes.  This strand of the literature generally focuses on the gap between the social cost of carbon and the optimal environmental tax arising from pre-existing distortionary labor income taxes or an exogenous requirement on government funds \citep{Bovenberg1997EnvironmentalGrowth,  Kaplow2012OPTIMALTAXATION, Jacobs2019RedistributionCurves, Barrage2019OptimalPolicy}. labor income taxes form a passive component of these analyses. 
The general findings of this literature is that the optimal environmental tax falls below the social cost of carbon to mitigate efficiency costs and enable the government to raise revenues. 
Furthermore, the literature argues for a recycling of environmental tax revenues to be used to lower income taxes. A recycling through transfers would intensify reductions in labor supply. These arguments rely on the premise that no lump-sum transfers are available. I add to this literature the perspective that a reduction in labor supply is part of the efficient policy. If lump-sum transfers are not available - as is to be assumed in this literature to motivate the existence of distortionary taxes - then labor income taxes should be positive to cope with distortions in the labor supply. Hence, there is a lower bound up to which environmental tax revenues are optimally used to lower distortionary taxes. This is not recognised by the literature. \tr{How do \cite{LansBovenberg1994EnvironmentalTaxation} argue for the use of env. tax revenues to lower distortionary taxes? Verbally or analytically?}

In contrast, I focus the paper on the role of income taxes in the optimal environmental policy and abstract from an exogenous financing condition on the government. Still, the model rationalizes a positive or progressive income tax.
 The inefficiency of environmental taxes arises absent pre-existing income tax distortions or the motive to redistribute.
In difference to this literature, where the presence of a distortionary income tax shapes the optimal level of the environmental tax, the existence of the environmental tax rationalises a progressive income tax in the present paper.
Importantly, the equity and the environmental targets of government intervention are perceived as competing goals as both tax instruments exert efficiency costs through a reduction in labor supply. 
I argue in this paper that what has commonly been perceived as an efficiency cost -  the reduction in labor supply in response to environmental and income taxation - is part of the optimal environmental policy. Hence, income taxation has a double dividend: an environmental and an equity one.  

In this literature, there are either no transfers to households at all \citep{Bovenberg2002EnvironmentalRegulation, LansBovenberg1994EnvironmentalTaxation} or an exogenously given requirement for transfers \citep{Barrage2019OptimalPolicy}. Hence, there is no lump-sum transfer instrument.

\citep{Fullerton1997EnvironmentalComment} writes in its introduction 
\begin{quote}
	With no revenue requirement, or where government can use lump-sum taxes, Arthur C. Pigou (1947) shows that the first-best tax on pollution is equal to the marginal environmental damage.
\end{quote}
\ar What is the optimal environmental tax when labor supply is elastic and there are no lump-sum funds?
\paragraph{Environment and elastic labor supply}
\cite{Oueslati2002EnvironmentalSupply} studies the optimal environmental policy with elastic labor supply. Yet, he allows for lump-sum transfers of environmental revenues. \textit{He should find something on reduction of hours}: No: capital is the only polluting factor, and labor is the clean factor of production.
\paragraph{Recycling of environmental tax revenues}
\paragraph{Environment and endogenous growth}

\paragraph{Public finance}
An equity-efficiency trade-off is central to the discussion of optimal labor income taxes in the public finance literature.  The benefits of labor taxes and progressivity arise, inter alia, from redistribution. %and from generating government revenues. 
With concave utility specifications full redistribution is efficient. However, the optimal tax system does not feature full redistribution when labor supply is endogenous. Instead, redistribution is traded off against aggregate output as individuals reduce their labor supply and skill investment in response to labor income taxation \citep{Heathcote2017OptimalFramework, Conesa2009TaxingAll, Domeij2004OnTaxes}.

To this literature I add another motive for the use of distortionary fiscal policies; namely to reduce inefficiently high labor supply. Furthermore, by abstracting from income inequality or income risk heterogeneity - the present framework
One closely related work is \cite{Loebbing2019NationalChange} who studies optimal income taxation in a model of directed technical change. The redistributive effect of tax progressivity is amplified through a compression of the wage rate distribution xxx