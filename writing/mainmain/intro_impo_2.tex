\section{Introduction}
%\tr{I show that carbon taxes are only efficient if lump-sum transfers are available.}

\begin{comment}
\tr{Think about:
	1) when labor income taxes are not used, then need to have  a higher environmental tax to meet emission limits? \ar Yes, because of advantageous level effect which outweighs recomposing effect of income tax.
	2) When staying at level optimal under the assumption of lump-sum redistribution, but then not redistributing, than absent labor income tax emissions are too high; by how much? Counterfactual}
	
	content...
	\end{comment}
%\paragraph{Recomposition versus reduction discussion plus ever stricter emission limits}
The latest assessment report of the Intergovernmental Panel on Climate Change (IPCC) \citep{IPCC2022} highlights the urgency to reduce greenhouse-gas emissions.%relative to the previous report from 2018 \citep{Rogelj2018MitigationDevelopment.}.
\footnote{ \  The report stresses the decreasing likelihood of meeting the Paris Agreement and limiting climate warming to 1.5°. The Paris Agreement of 2015 formulates clear political goals to mitigate climate change. Under this treaty, states have agreed on a legally binding maximum increase in temperature to well below 2°C, preferably 1.5° over pre-industrial levels, and the global community seeks to be climate-neutral in 2050  (compare: \url{https://unfccc.int/process-and-meetings/the-paris-agreement/the-paris-agreement}). 
}
On the other hand, scholars have pointed to reductive policy measures to handle environmental limits \citep{Arrow2004AreMuch, Schor2005SustainableReduction, Dasgupta2021}. A reduction in work effort and consumption mitigates pollution by diminishing economic activity. Such a reduction could be achieved by using distortionary fiscal policy tools.
However, the economic literature on environmental policy has focused on the recomposing aspect of environmental policies: environmental taxes. %\citep{Fried2018ClimateAnalysis}. 
Given the exigency to act, this paper addresses the question whether fiscal policies can help meet climate targets.

I show that, indeed, once 
labor supply is elastic, reductive policy measures optimally complement the environmental tax. 
It is established in the literature that absent any other distortion, an environmental tax equal to the social cost of the externality implements the efficient allocation. I argue, first, that this result crucially depends on the use of lump-sum transfers to redistribute environmental tax revenues; otherwise work effort is inefficiently high.
%\textcolor{blue}{This is interesting independent of whether they are feasible or not. Could relate to the fact that there is a discussion how to use revenues. Yet, one might argue that we are always in a setting with distortionary labor income taxes; so that recycling lump-sum is never needed; numbers on size of expected revenues and government spending}
 When, second, environmental tax revenues are not redistributed lump-sum, then  environmental taxes are optimally combined with progressive labor income taxes. However, the use of income taxes is not directly targeted at the externality: the motive for labor taxation follows rather indirectly from a distortion in labor markets induced by the environmental policy. Hence, (i) the two tax instruments are complements, and % to lower inefficiently high hours worked. 
% I show that redistributing environmental tax revenues through an income tax scheme allows to implement the efficient allocation. The optimal income tax scheme is progressive.
(ii) the optimal environmental policy equalizes the distribution of income as  a side effect. The theoretic analysis forms the first part of the paper.

In the second part, I scrutinize whether progressive income taxes remain optimal in a more realistic quantitative model with endogenous growth and heterogeneous skills \textbf{when environmental tax revenues are not redistributed. 
}This is unclear a-priori since a progressive tax scheme reduces incentives to innovate through a market size effect. Furthermore, a skill bias documented for the green sector \citep{Consoli2016DoCapital} in combination with a relatively more elastic high-skilled labor causes a higher tax progressivity to recompose economic structure towards dirty production. The model suggests that despite these countering mechanisms the optimal income tax scheme is progressive. To lower hours worked the government forfeits growth and accepts a less green production ratio.  I quantify the welfare gains of setting progressive income taxes to equal yyy in consumption equivalent measure.  
% take model with no redistribution as benchmark

The paper's results are relevant for the political and academic debate on how best to use environmental tax revenues. The paper points to the importance of lump-sum transfers as a reductive policy tool in the optimal environmental policy; an aspect which appears overlooked in today's discussion.\footnote{\ POLICY debate; \cite{Fried2018TheGenerations}}
When thinking about how to recycle environmental tax revenues other than as lump-sum transfer, then, one should also think about alternative reductive tools such as progressive labor income taxes. 
If the reductive part of the environmental policy is neglected, environmental taxes have to be higher to meet emission limits, as I demonstrate in the quantitative exercise.

The results address the academic debate on the so-called \textit{weak double-dividend} \citep[for example:][]{LansBovenberg1994EnvironmentalTaxation, LansBovenberg1996OptimalAnalyses}. The hypothesis posits that recycling environmental tax revenues to reduce pre-existing tax distortions is advantageous to recycling  revenues as lump-sum transfers. The rationale is that transfers decrease labor supply thereby diminishing the tax base of the income tax. A conflict between generating government funds and environmental protection arises. The findings in the present paper suggest a lower bound on the reduction in distortionary income taxes: when environmental tax revenues are not redistributed lump-sum, some reduction in labor supply via distortionary income taxes is in fact efficient from an environmental policy perspective. In other words, even if environmental tax revenues suffice to satisfy an government revenue requirement, the optimal labor income tax is progressive.  %\footnote{\ The set-up in this paper can be integrated in the  of as a situation How the trade-off characterizing the optimal level of work effort plays out when the government seeks to generate funds and to mitigate an externality while environmental tax revenues suffice to satisfy the revenue requirement  is left for further research. Clearly: Then the optimal income tax is progressive. }  %\tr{Think about that labor income tax revenues are redistributed back to households but they would not under the weak double dividend hypossis} 

\begin{comment}
When labor supply is fixed, environmental taxes alone can establish the efficient allocation in a representative agent economy absent fiscal distortions. Then, such a tax instrument is optimally set to the social cost of an externality, and originators internalize these social costs: the Pigou principle.
However, not redistributing environmental tax revenues reduces consumption below the efficient level and, as I demonstrate, the optimal environmental tax does not follow the Pigou principle.  If, on top, the  labor supply decision is endogenous, the environmental tax alone features too high labor supply. \tr{This results in too high environmental externality. \textbf{To be shown!}}

Lump-sum transfers of environmental tax revenues restore the efficient allocation: as households become richer, labor supply reduces. When lump-sum transfers are not available, the government can establish the efficient allocation by redistributing environmental tax revenues through an income tax scheme which I demonstrate to be progressive.

content...
\end{comment}

%%%%%%%%%%%%%%%%%%%%%%%%%%%%%%%%%%%%%%%
%\paragraph{Answer WHAT I DO }
%%%%%%%%%%%%%%%%%%%%%%%%%%%%%%%%%%%%%%%
\subsubsection*{First part: Analytic stuff}
\paragraph{simple model}
I propose a simple and comparatively general model to derive the main theoretical results. There are two intermediate sectors of production, one of which exerts a negative environmental externality. The environmental externality is the only distortion motivating government action. In the model, a Ramsey planner seeks to maximize welfare of a representative agent having  an environmental and a labor income tax at its disposal. The income tax scheme is generally non-linear and a common specification in the public finance literature \citep[e.g.][]{Benabou2002TaxEfficiency, Heathcote2017OptimalFramework}.
The model abstracts from  endogenous growth, inequality, and an exogenous government funding constraint. The last two are important abstractions since they traditionally motivate income taxation.

\paragraph{Analytic findings 1}
\textbf{MAIN finding: Progressive income tax} I show that, under mild assumptions, the optimal labor income tax is progressive when no lump-sum transfers are available. 
The optimality of progressive income taxes results from inefficiently high labor supply. The mechanism runs as follows: the use of environmental taxes induces an additional distortion on the labor market by driving a wedge between households' shadow value of income and the social one. The environmental tax reduces the returns to labor below its marginal product. When environmental tax revenues are not redistributed lump-sum, labor supply is too high. 
This is a novel motive for income taxation so far overlooked in the literature.
Following this intuition, environmental and labor income taxes complement each other in the optimal environmental policy.
% I argue that environmental and income taxes are complements. The first is targeted at the environmental externality while the second serves to mitigate distortions in the labor market resulting from environmental taxation.
  
\paragraph{Analytic findings 2}
\textbf{Pigou principle violated absent lump-sum transfers and efficient allocation not feasible}
I consider two cases how environmental tax revenues are recycled. First, revenues are consumed by the government, and, second, the government redistributes revenues via the income tax scheme. In the first scenario, I find that the optimal environmental tax does not satisfy the Pigou principle, i.e., it does not equal the social cost of the externality. As demonstrated by \textit{Pigou xxx}, setting the environmental tax equal to the marginal costs arising from the polluting activity that are not private (the so-called social cost) is optimal. The idea is that polluters behave as if internalizing the social cost of their action when confronted with the environmental tax. I show that when environmental tax revenues are consumed by the government, the Pigou principle is violated. In this case, the motive to increase private consumption by lowering environmental tax revenues makes a deviation of the environmental tax from the social cost of the externality optimal.
Furthermore, when labor supply is elastic non-redistribution of environmental tax revenues results in inefficiently high hours worked. Then, the optimal labor tax is progressive to diminish work effort closer to the efficient level.
Nevertheless, the efficient allocation is not feasible in this policy framework since either consumption is inefficiently low or work effort is too high. 


\paragraph{Analytic findings 3}
\textbf{Efficient allocation can be implemented through redistribution via income tax scheme}
In the second scenario, the Ramsey planner redistributes environmental tax revenues through the income tax scheme, now the efficient allocation is attainable. Thus, even absent lump-sum transfers, there exists a possibility to implement the efficient allocation. By redistributing environmental tax revenues, consumption can be chosen efficiently high while the labor income tax scheme handles too high labor supply. In fact, redistribution via the income tax scheme incentivizes more labor supply. A progressive income tax scheme counteracts this tendency restoring the efficient level of hours worked. 
In this setting, the Pigou principle holds since there is no trade-off between lowering the externality and the allocation of consumption.
%More precisely, I study the optimal policy mix of environmental and labor income taxes to meet emission limits. 
%I find that progressive labor income taxes are used in concert with fossil taxes to optimally reduce emissions. 
% First, I propose a tractable model to provide intuition for this result: Non-lump-sum redistribution of environmental tax revenues increase the gains from labor. Leisure becomes more expensive and households do not reduce their labor supply efficiently in response to the fossil tax. To reduce hours to the efficient level, income taxes complement the fossil tax to lower hours to the efficient level.  
% 
% 
% 
%Second, I assess the importance of this novel role for income taxes in a quantitative endogenous growth model:
%Even though a more progressive tax reduces research effort  and recomposes production towards the fossil sector, the optimal tax is progressive. 


\subsubsection*{Quantitative exercise}
In this and the subsequent three paragraphs, I motivate the quantitative exercise and expound the model.
Even though I have shown theoretically that progressive income taxes form one pillar of the optimal environmental policy when no lump-sum transfers are available, it is not clear whether progressivity remains optimal in a more realistic, quantitative model. 
Two countering mechanisms of tax progressivity shall be considered here.
First, endogenous growth could make a more regressive income tax optimal to incentivize innovation. The size of the market for successful innovation positively affects the profitability of research investment. Subsidizing labor supply through regressive taxes then boosts technological growth. Second, a skill bias in the green sector also renders regressive income taxes advantageous by enhancing supply of the green sector-specific input good. In the model, high-skill workers supply more labor due to a wage premium. They are more responsive to the substitution effect as income tax progressivity makes leisure cheaper. Then, the economy transitions to a higher dirty production share. Directed technical change amplifies this recomposing effect of the income tax. 

% relation to core model and to fried;
In light of these two mechanisms calling for income tax regressivity, I extend the core model studied in the analytical section with endogenous growth and skill heterogeneity. The resulting model builds on \cite{Fried2018ClimateAnalysis}. It extends aforementioned work by an optimal policy analysis, the availability of income taxes, and skill heterogeneity. 
% no income inequality: trade off as in classical 
I maintain the assumption of a representative family which consists of two skill types. Within the family, all workers consume the same. Thus, workers remain perfectly insured against income differentials so that equity concerns do not drive the optimal policy. The trade-off shaping the optimal policy stays within the boundaries of traditional environmental policy considerations: growth versus externality mitigation \citep{Stokey1998AreGrowth, Jones2016LifeGrowth, Acemoglu2012TheChange}.  

%\paragraph{How the externality is modeled}
In the quantitative model, the government cares about the externality due to an exogenous constraint on emissions and not via household utility. This approach, known as a cost-effectiveness approach in the environmental literature, has the advantage of reducing modeling and parametric uncertainty when specifying how greenhouse-gas emissions affect the climate and the damages it produces in terms of well-being and production. Instead, the cost-effectiveness approach uses estimated emission targets stemming from meta studies of more complex integrated assessment models. Additionally, the emission limits I use to calibrate the model are designed to meet climate targets constituting a politically relevant environment. 
% think about the relation of an absolute emission target and not reducing hours

%\paragraph{Endogenous growth}
Allowing for endogenous growth is important to take seriously the possibility of green growth to keep consumption high while meeting emission targets. Endogenous growth introduces dynamics into the model since today's level of technology positively depends on yesterday's technology: \tr{ a \textit{standing on the shoulder of giants} mechanism. LOOK THIS UP \cite{Acemoglu2012TheChange}} 


%The model differentiates between high- and low-skilled labor to account for a skill bias found for the green sector \citep{Consoli2016DoCapital}. This asymmetry of sectors renders regressive taxes a tool to lower relative production costs in the green sector: high-skill workers reduce their labor supply more in response to a more progressive tax as leisure is more valuable to them. This recomposing mechanism counteracts intentions to lower emissions. %, a regressive tax functions as a green subsidy. % In fact, there is an externality arising from high-skill labor supply as it shapes the share of fossil to green energy production. 
%On the other hand, progressive income taxes lower aggregate production by diminishing the price of leisure. 
%Endogenous growth amplifies the repercussions of progressive income taxes:
%First, a lower labor supply reduces the profitability of research in general. By reducing hours worked the planner sacrifices technological progress. Secondly, the recomposing effect is aggravated as research is directed towards the sector with the increased labor share, i.e. the fossil sector.

%\paragraph{Calibration}
The model is calibrated to the US in the baseline period from 2015 to 2019. To do so, I proceed in two steps. First, I set certain parameters to values found in the literature. Most importantly, I use reasonable values of production and growth processes found in \cite{Fried2018ClimateAnalysis}. % who conducts a rigorous calibration exercise. 
With these parameter values at hand, I match the share of high skill in the green and non-green sectors building on \cite{Consoli2016DoCapital}. The emission target is set to the values suggested in the latest IPCC draft on mitigation pathways \citep{IPCC2022}: A 50\% reduction by 2030 relative to 2019 levels and  net-zero emissions from 2050 onward.

%\paragraph{Quantitative exercise}
I perform the following quantitative exercise. 
The Ramsey planner maximizes social welfare as in the core model. However, the externality enters as an exogenous emission limit which constraints government action. The planner sets the environmental tax and the progressivity parameter of the non-linear income tax scheme. \textbf{Environmental tax revenues are consumed by the government.} As discussed in the analytical section, under this policy regime the efficient allocation is not feasible; but, it is  more policy relevant. 

I solve the Ramsey model explicitly for 60 years starting from 2020. From 2080 onward taxes are fixed. The planner is constrained by a sustainability motive to leave at least as much resources to future periods than for the explicitly modeled periods.
This approach has the advantage of not having to assume the existence of a balanced growth path. Given that one factor of production, namely fossil energy, is bounded by an exogenous limit, I seek to stay agnostic on this aspect.
\tr{\textit{Alternative: assume the economy reaches a BGP in some future period; all variables grow at constant rates. Possible?}}

% main finding; then understand why; what are the drivers: how does the result change as things are added or taken from the model
\paragraph{Findings}
I find that the planner chooses a progressive income tax despite its repercussions on growth and emissions. Indeed, this policy increases the share of fossil to green energy and reduces research efforts and consumption. As the emission limit becomes tighter over time, the Ramsey planner augments both the environmental tax and income tax progressivity. The positive correlation between income tax progressivity and the environmental tax is in line with the analytical finding. 
 

% comparison to a model without income tax
To investigate the importance and the welfare gains of an integrated environmental policy, I rerun the main experiment but let the government only choose the environmental tax. Environmental tax revenues are transferred back to households through the income tax scheme.\footnote{\ \textit{This setting does neither correspond to any analytical model version- Could also think of modeling the alternative as gov consumes environmental tax revenues  and no income tax scheme.\textcolor{blue}{Compare: benchmark model where fiscal policy and environmental policy are integrated versus a version where there are neither an income tax scheme nor lump-sum transfers. }} 
}
Comparing the resulting allocation to the one under the benchmark model is informative on the mechanisms and importance of an integrated environmental and fiscal policy when no lump-sum transfers are available. 

\textit{ This means two aspects: (i) redistribution via the income tax scheme and (ii) the use of progressive taxes. Named: Integrated Scenario }

% results
The results suggest a clear cut between responsibilities of policy instruments: The environmental tax targets the externality, while the income tax handles the inefficiency arising from environmental tax revenues. When depriving the government from an income tax scheme, the environmental tax remains largely unchanged compared to the benchmark setting. 
\tr{To do: investigate if even with lump-sum redistribution there is a role for income tax in full model.}
The availability of an income tax increases social welfare by 0.11\% over the period from 2020 to 2080 which corresponds to a consumption equivalent of xxx. 

% optimal income tax with exogenous growth; optimal income tax without skill heterogeneity 

% sensitivity
\tr{In progress}
\begin{itemize}
	\item utility specification (Building on Bick can think of European version when substitution effect is stronger)
	\item no skill bias in the green sector
	\item no endogenous growth
	\item spillovers across scientists: with positive spillovers potentially no growth
	\item counterfactual technology gap
\end{itemize}
Since the target of the labor tax in the environmental setting presented here is to align hours worked with their efficient level, results are sensitive to the elasticity of labor with respect to after-tax wages. 
\paragraph{Quantitative finding to be shaped by income and substitution effect!}
Literature on how households react to changes in income \cite{Bick2018HowImplications} and \cite{Boppart2019LaborPerspectiveb}


%\paragraph{Literature}
\paragraph{Literature}

\paragraph{Optimal environmental policy}
\textbf{Main claim: focus on environmental taxes and recomposition}
\begin{itemize}
	\item with exogenous growth
	\item with endogenous growth
\end{itemize}

In general, papers on optimal environmental policy focus on the optimal environmental tax and analyze settings with inelastic labor supply \citep{Golosov2014OptimalEquilibrium, Acemoglu2012TheChang, Fried2018ClimateAnalysis}. Therefore, the main finding of the present paper, the necessity of reductive policy measures to implement the efficient allocation, complement this literature. Furthermore, I argue that the Pigou principle does generally not apply when no lump-sum transfers are available.  


\paragraph{Recycling of environmental tax revenues}
\textbf{Main claim: they overlook that when lump-sum transfers are not available, then labor supply is inefficiently high, and that env. tax exceeds (?) scc in optimal policy}
\begin{itemize}
	\item in general: \cite{Fried2018TheGenerations}
	\item double dividend literature
\end{itemize}
These findings have important consequences for the literature on the so-called double dividend of environmental policy and the question how to recycle environmental tax revenues. While this literature argues for the recycling of environmental tax revenues to lower pre-existing tax distortions, my paper constitutes an argument for a lower bound on distortionary income taxes: some reduction of labor supply is in fact efficient. 
Furthermore, when revenues are not redistributed lump-sum, the Pigouvian tax does not implement the efficient allocation. 

\paragraph{Environmental protection and inequality}
\ar 1) Inequality and environment as competing goods.

In the literature discussing environmental policies in an unequal framework, a competition between equity and environmental good provision have been discussed. 
This trade-off can be separated into (i) the competition for public funds \citep{LansBovenberg1996OptimalAnalyses, Jacobs2019RedistributionCurves} and (ii) effects of either environmental policies on equity or equalizing policies on environmental quality \citep{Jacobs2019RedistributionCurves, Sager2019IncomeCurves, Dobkowitz2022}. 

Since hours are inefficiently high, equalizing policies become part of the optimal environmental policy. As a byproduct, the distribution of income becomes more equal.


\ar 2) Inequality to shape effects of environmental policies and effect of fiscal policy on environment due to heterogeneity

 Furthermore, the differentiation of skills and the skill-bias of the green sector in the paper give rise to a new channel through which labor taxation affects environmental protection. The literature has primarily focused on a demand channel arising from non-linear Engel curves through which inequality and redistribution shape the degree of dirty production in the economy.   

%\textbf{Non-linear Engel Curves: redistribution}
\cite{Jacobs2019RedistributionCurves} the motive to redistribute and to provide an environmental good compete for government resources due to a negative effect of environmental taxes on the wage rate. Even with lump-sum transfers, the optimal environmental tax does not follow the Pigou principle when the government seeks to enhance equity.

\cite{Sager2019IncomeCurves} argues empirically, that redistribution to poorer households may result in a higher demand for polluting goods. 
\paragraph{Pubic Finance literature}
\textbf{I add: a new perspective on labor income taxes as a tool to lower inefficiently high hours worked. }

 \cite{Heathcote2017OptimalFramework}, \cite{Loebbing2019NationalChange}

\paragraph{Reductive policies in the literature}
The finding relates to the literature discussing rationales for the usage of reductive policy measures. These arise from o
Negative externalities of consumption and hours worked such as
 envy \cite{Alvarez-Cuadrado2007EnvyHours}, habits \cite{Ravn2006DeepHabits} \tr{Check if this is on inefficiency} or a positive externality of leisure \cite{Alesina2005WorkDifferent}. The present paper relates to this literature by identifying an externality of work which emerges from the existence of an environmental externality of production. In addition, once an environmental tax recomposes production towards a cleaner alternative, the wage rate understates the marginal product of labor. \tr{This needs to be made clearer.}
\begin{itemize}
	\item literature which \cite{Alvarez-Cuadrado2007EnvyHours}
\end{itemize}
\tr{This observation relates to the literature in several ways: first, the literature which discusses the optimal recycling of carbon tax revenues. Because when revenues are not recycled as lump-sum transfers, then labor supply is inefficiently high and additional policy measures are necessary to implement the efficient allocation today. 
	 In other words: because lump-sum transfers are not available, the literature argues, the government should use corrective tax revenues to lower pre-existing tax distortions. But by how much? I argue, that there is an optimal size of positive tax distortions when lump-sum transfers are not available. Hence, under the premise of non-lump sum transfers, distortionary labor income taxes arise as an optimal policy tool even absent an exogenous financing condition or inequality. }

 The paper relates broadly to the literature discussing optimal environmental policies. I separate them into two strands: one with inelastic and one with elastic labor supply. 
  \paragraph{Optimal environmental policy: exogenous labor supply}
 
\paragraph{Lit: environmental policy and distortionary fiscal setting}

\begin{itemize}
	\item Williams 2013 Double dividend 
	\item talk to Mireille Chiroleu Assouline: paper on double dividend
	\item Mireille with Aubert or Fodha (PSE)
\end{itemize}
%Inequality-environment nexus: normally motivated by a demand-side perspective; in this project I focus on a supply side explanation
labor supply becomes elastically in the literature studying the interaction of environmental taxes and distortionary taxes.  This strand of the literature generally focuses on the gap between the social cost of carbon and the optimal environmental tax arising from pre-existing distortionary labor income taxes or an exogenous requirement on government funds \citep{Bovenberg1997EnvironmentalGrowth,  Kaplow2012OPTIMALTAXATION, Jacobs2019RedistributionCurves, Barrage2019OptimalPolicy}. labor income taxes form a passive component of these analyses. 
The general findings of this literature is that the optimal environmental tax falls below the social cost of carbon to mitigate efficiency costs and enable the government to raise revenues. 
Furthermore, the literature argues for a recycling of environmental tax revenues to be used to lower income taxes. A recycling through transfers would intensify reductions in labor supply. These arguments rely on the premise that no lump-sum transfers are available. I add to this literature the perspective that a reduction in labor supply is part of the efficient policy. If lump-sum transfers are not available - as is to be assumed in this literature to motivate the existence of distortionary taxes - then labor income taxes should be positive to cope with distortions in the labor supply. Hence, there is a lower bound up to which environmental tax revenues are optimally used to lower distortionary taxes. This is not recognised by the literature. \tr{How do \cite{LansBovenberg1994EnvironmentalTaxation} argue for the use of env. tax revenues to lower distortionary taxes? Verbally or analytically?}

In contrast, I focus the paper on the role of income taxes in the optimal environmental policy and abstract from an exogenous financing condition on the government. Still, the model rationalizes a positive or progressive income tax.
 The inefficiency of environmental taxes arises absent pre-existing income tax distortions or the motive to redistribute.
In difference to this literature, where the presence of a distortionary income tax shapes the optimal level of the environmental tax, the existence of the environmental tax rationalises a progressive income tax in the present paper.
Importantly, the equity and the environmental targets of government intervention are perceived as competing goals as both tax instruments exert efficiency costs through a reduction in labor supply. 
I argue in this paper that what has commonly been perceived as an efficiency cost -  the reduction in labor supply in response to environmental and income taxation - is part of the optimal environmental policy. Hence, income taxation has a double dividend: an environmental and an equity one.  

In this literature, there are either no transfers to households at all \citep{Bovenberg2002EnvironmentalRegulation, LansBovenberg1994EnvironmentalTaxation} or an exogenously given requirement for transfers \citep{Barrage2019OptimalPolicy}. Hence, there is no lump-sum transfer instrument.

\citep{Fullerton1997EnvironmentalComment} writes in its introduction 
\begin{quote}
	With no revenue requirement, or where government can use lump-sum taxes, Arthur C. Pigou (1947) shows that the first-best tax on pollution is equal to the marginal environmental damage.
\end{quote}
\ar What is the optimal environmental tax when labor supply is elastic and there are no lump-sum funds?
\paragraph{Environment and elastic labor supply}
\cite{Oueslati2002EnvironmentalSupply} studies the optimal environmental policy with elastic labor supply. Yet, he allows for lump-sum transfers of environmental revenues. \textit{He should find something on reduction of hours}: No: capital is the only polluting factor, and labor is the clean factor of production.
\paragraph{Recycling of environmental tax revenues}
\cite{Fried2018TheGenerations}
\paragraph{Environment and (endogenous) growth}
\begin{itemize}
	\item limits to growth
	\item general literature on end growth and the environment
\end{itemize}
\paragraph{Public finance}
An equity-efficiency trade-off is central to the discussion of optimal labor income taxes in the public finance literature.  The benefits of labor taxes and progressivity arise, inter alia, from redistribution. %and from generating government revenues. 
With concave utility specifications full redistribution is efficient. However, the optimal tax system does not feature full redistribution when labor supply is endogenous. Instead, redistribution is traded off against aggregate output as individuals reduce their labor supply and skill investment in response to labor income taxation \citep{Heathcote2017OptimalFramework, Conesa2009TaxingAll, Domeij2004OnTaxes}.

To this literature I add another motive for the use of distortionary fiscal policies; namely to reduce inefficiently high labor supply. Furthermore, by abstracting from income inequality or income risk heterogeneity - the present framework
One closely related work is \cite{Loebbing2019NationalChange} who studies optimal income taxation in a model of directed technical change. The redistributive effect of tax progressivity is amplified through a compression of the wage rate distribution xxx