\section{Quantitative results}\label{sec:res}

In this section, I present and discuss the quantitative results.
Subsection \ref{subsec:mr} depicts the optimal policy given the emission target. Subsection \ref{subsec:dis} discusses the results in comparison to the efficient allocation focusing on the role of labor income taxes. Finally, I discuss results when environmental tax revenues are redistributed lump-sum. 
%I focus on analyzing the mechanisms and welfare benefits from integrating the income tax scheme into the environmental policy. I also discuss the costs of not using lump-sum transfers.


\subsection{Results}\label{subsec:mr}
%This section depicts results on the optimal policy followed by the implied allocation in the benchmark model where environmental tax revenues are redistributed via the income tax scheme. 

\begin{figure}[h!!]
	\centering
	\caption{Optimal Policy }\label{fig:optPol}
	\begin{minipage}[]{0.4\textwidth}
		\centering{\footnotesize{(a) Income tax progressivity, $\tau_{lt}$}}
		%	\captionsetup{width=.45\linewidth}
		\includegraphics[width=1\textwidth]{../../codding_model/own_basedOnFried/optimalPol_190722_tidiedUp/figures/all_July22/taul_SingleAltPolOPT_T_NoTaus_regime3_spillover0_noskill0_sep1_xgrowth0_etaa0.79.png}
	\end{minipage}
	\begin{minipage}[]{0.1\textwidth}
		\
	\end{minipage}
	\begin{minipage}[]{0.4\textwidth}
		\centering{\footnotesize{(b) Environmental tax, $\tau_{ft}$ }}
		%	\captionsetup{width=.45\linewidth}
		\includegraphics[width=1\textwidth]{../../codding_model/own_basedOnFried/optimalPol_190722_tidiedUp/figures/all_July22/tauf_SingleAltPolOPT_T_NoTaus_regime3_spillover0_noskill0_sep1_xgrowth0_etaa0.79.png}
	\end{minipage}
\end{figure} 

To meet the IPCCs suggested emission target, the optimal income tax is progressive for all periods between 2030 and 2080; see panel (a) in figure \ref{fig:optPol}. As the emission target is less strict, between 2030 to 2045, optimal income tax progressivity is around $\tau_{lt}=0.04$. As the emission target jumps to net-zero emissions in 2050, optimal tax progressivity accelerates to above 0.08 and gradually increases in the subsequent years to around 0.09. This is approximately half the size found for the US in \cite{Heathcote2017OptimalFramework}: $\tau_{l}=0.181$. 
In the period without emission target from 2020 to 2030, the optimal income tax is slightly regressive.

Consider panel (b). The optimal fossil tax displays a similar step pattern as the income tax progressivity. From 2020 to the beginning of 2030, it is negative. It jumps to around 70\% as the emission target is to reduce emissions by 50\% relative to 2019 emissions. As the emission target rises  to net-zero emissions in 2050, the optimal tax on fossil sales is close to 90\%. 

Figure \ref{fig:optAll} depicts the optimal allocation under the constraint on emissions. Limiting emissions in line with the Paris Agreement is concomitant with both a reduction and recomposition of consumption and production over time. 

Panel (a) shows consumption which reduces significantly when new emission limits become active, in 2030 and in 2050, but starting from the new low levels continues to grow. labour effort of both skill types also reduces visibly as stricter emission targets are enforced; panel (b). In contrast to consumption, hours worked for both types of labour decrease over time. In comparison to hours supplied by low-skilled workers, high-skilled workers reduce hours more; compare panel (c) which shows the ratio of hours worked by high to low skill workers. 

The rise in consumption after each reduction is driven by technological progress in all sectors; compare panel (d) which shows growth rates by sector and as aggregate in per cent. 
The green sector sees a rise in technological progress, the dashed black line, while growth in the fossil and the non-energy sector is positive, yet diminishing over time. Overall, aggregate growth is positive but decreasing; compare the grey dashed graph. 
Summing up the last two paragraphs, the emission target is best achieved with more leisure at higher technology levels in all sectors. 

A regards scientists, panel (e), there is a recomposition           towards the green sector: while research in the non-energy and the fossil sector decrease over time, green research effort rises. Yet, overall, the amount of scientists reduces; compare the grey graph which depicts the sum of researchers across sectors. 
Finally, labour input goods are also redirected towards the green sector; see panel (f). 

\begin{figure}[h!!]
	\centering
	\caption{Optimal Allocation }\label{fig:optAll}
	
	
	\begin{minipage}[]{0.32\textwidth}
		\centering{\footnotesize{(a) Consumption}}
		%	\captionsetup{width=.45\linewidth}
		\includegraphics[width=1\textwidth]{../../codding_model/own_basedOnFried/optimalPol_elastS_DisuSci/figures/all_1705/Single_OPT_T_NoTaus_C_spillover0_sep1_BN0_ineq0_red0_etaa0.79.png}
	\end{minipage}
	\begin{minipage}[]{0.32\textwidth}
		\centering{\footnotesize{(b) Hours worked }}
		%	\captionsetup{width=.45\linewidth}
		\includegraphics[width=1\textwidth]{../../codding_model/own_basedOnFried/optimalPol_elastS_DisuSci/figures/all_1705/SingleJointTOT_OPT_T_NoTaus_labour_spillover0_sep1_BN0_ineq0_red0_etaa0.79_lgd1.png}
	\end{minipage}
	\begin{minipage}[]{0.32\textwidth}
		\centering{\footnotesize{(c) High-to-low-skill ratio hours}}
		%	\captionsetup{width=.45\linewidth}
		\includegraphics[width=1\textwidth]{../../codding_model/own_basedOnFried/optimalPol_elastS_DisuSci/figures/all_1705/Single_OPT_T_NoTaus_hhhl_spillover0_sep1_BN0_ineq0_red0_etaa0.79.png}
	\end{minipage}
	\begin{minipage}[]{0.32\textwidth}
		\centering{\footnotesize{\ \\ (d) Technology growth}}
		%	\captionsetup{width=.45\linewidth}
		\includegraphics[width=1\textwidth]{../../codding_model/own_basedOnFried/optimalPol_elastS_DisuSci/figures/all_1705/SingleJointTOT_OPT_T_NoTaus_Growth_spillover0_sep1_BN0_ineq0_red0_etaa0.79_lgd1.png}
	\end{minipage}
	\begin{minipage}[]{0.32\textwidth}
		\centering{\footnotesize{\ \\(e) Scientists }}
		%	\captionsetup{width=.45\linewidth}
		\includegraphics[width=1\textwidth]{../../codding_model/own_basedOnFried/optimalPol_elastS_DisuSci/figures/all_1705/SingleJointTOT_OPT_T_NoTaus_Science_spillover0_sep1_BN0_ineq0_red0_etaa0.79_lgd1.png}
	\end{minipage}
	\begin{minipage}[]{0.32\textwidth}
		\centering{\footnotesize{\ \\(f) Labor input}}
		%	\captionsetup{width=.45\linewidth}
		\includegraphics[width=1\textwidth]{../../codding_model/own_basedOnFried/optimalPol_elastS_DisuSci/figures/all_1705/SingleJointTOT_OPT_T_NoTaus_labourInp_spillover0_sep1_BN0_ineq0_red0_etaa0.79_lgd1.png}
	\end{minipage}
\end{figure} 



\subsection{Discussion}\label{subsec:dis}
The discussion of the optimal policy centers on the questions of what drives the optimal policy. In the first paragraph, I compare the optimal to the efficient allocation. Thereafter I contrast the optimal allocation under the benchmark regime, referred to as \textit{integrated} regime, to a regime where environmental and fiscal policy are \textit{separate}: environmental tax revenues are consumed by the government and no income taxes are available. This comparison helps to understand the benefits of an integrated-policy regime when no lump-sum transfers are available. Finally, I turn to the optimal allocation when lump-sum transfers are feasible: the optimal environmental and fiscal policy are separate, and income taxes serve to boost growth as long as this does not conflict with meeting emission limits.
In subsection \ref{subsec:simpler}, I discuss the results when the benchmark model is simplified: that is, assuming exogenous growth and/or skill homogeneity.

%\begin{enumerate}
%	\item What is the goal of policy intervention? \ar social planner allocation
%	\item (Benefits) What is different when no integrated policy is run and instead revs consumed by government \ar Benefits of an integrated policy
%	\item double dividend literature: use of labor income tax when all env tax revenues are consumed by the government.
%	\item (Costs) What cannot be reached by integrated policy as compared to lump-sum transfers: is taul used for different purpose? without endogenous growth should be zero; eg. can use taul to boost growth as lump-sum transfers take care of labor supply 
%	\item What could be reached if there was no trade-off with heterogenous skills or growth? no heterogeneous skills, no endogenous growth \ar how does the optimal policy differ?
%\end{enumerate}

\subsubsection{Social planner allocation}
As a benchmark to the Ramsey planner allocation, I present the social planner's allocation. The efficient allocation can be perceived as the intended allocation which the Ramsey planner seeks to implement. However, it may not be able to achieve the efficient allocation due to the reliance on tax instruments. Figure \ref{fig:fb_opt} depicts the efficient and the optimal allocation by the black-solid and the orange-dashed graphs. 
\begin{figure}[h!!]
	\centering
	\caption{Comparison to efficient allocation }\label{fig:fb_opt}
	
	\begin{minipage}[]{0.32\textwidth}
		\centering{\footnotesize{(a) Consumption}}
		%	\captionsetup{width=.45\linewidth}
		\includegraphics[width=1\textwidth]{../../codding_model/own_basedOnFried/optimalPol_190722_tidiedUp/figures/all_July22/C_CompEffOPT_T_NoTaus_opteff_spillover0_noskill0_sep1_xgrowth0_countec0_etaa0.79_lgd1_lff0.png}
	\end{minipage}
	\begin{minipage}[]{0.32\textwidth}
	\centering{\footnotesize{(b) High skill hours worked}}
	%	\captionsetup{width=.45\linewidth}
	\includegraphics[width=1\textwidth]{../../codding_model/own_basedOnFried/optimalPol_190722_tidiedUp/figures/all_July22/hh_CompEffOPT_T_NoTaus_opteff_spillover0_noskill0_sep1_xgrowth0_countec0_etaa0.79_lgd0_lff0.png}
\end{minipage}
	\begin{minipage}[]{0.32\textwidth}
	\centering{\footnotesize{(c) Low skill hours worked}}
	%	\captionsetup{width=.45\linewidth}
	\includegraphics[width=1\textwidth]{../../codding_model/own_basedOnFried/optimalPol_190722_tidiedUp/figures/all_July22/hl_CompEffOPT_T_NoTaus_opteff_spillover0_noskill0_sep1_xgrowth0_countec0_etaa0.79_lgd0_lff0.png}
\end{minipage}

	\begin{minipage}[]{0.32\textwidth}
	\centering{\footnotesize{(d) Aggregate growth}}
	%	\captionsetup{width=.45\linewidth}
	\includegraphics[width=1\textwidth]{../../codding_model/own_basedOnFried/optimalPol_190722_tidiedUp/figures/all_July22/gAagg_CompEffOPT_T_NoTaus_opteff_spillover0_noskill0_sep1_xgrowth0_countec0_etaa0.79_lgd0_lff0.png}
\end{minipage}
\begin{minipage}[]{0.32\textwidth}
	\centering{\footnotesize{(e) Energy mix, $\frac{G}{F}$}}
	%	\captionsetup{width=.45\linewidth}
	\includegraphics[width=1\textwidth]{../../codding_model/own_basedOnFried/optimalPol_190722_tidiedUp/figures/all_July22/GFF_CompEffOPT_T_NoTaus_opteff_spillover0_noskill0_sep1_xgrowth0_countec0_etaa0.79_lgd0_lff0.png}
\end{minipage}
\begin{minipage}[]{0.32\textwidth}
	\centering{\footnotesize{(f) Utility}}
	%	\captionsetup{width=.45\linewidth}
	\includegraphics[width=1\textwidth]{../../codding_model/own_basedOnFried/optimalPol_190722_tidiedUp/figures/all_July22/SWF_CompEffOPT_T_NoTaus_opteff_spillover0_noskill0_sep1_xgrowth0_countec0_etaa0.79_lgd0_lff0.png}
\end{minipage}
\end{figure}

The social planner allocation, too, is a combination of recomposing and reductive measures. 
The social planner reduces consumption less than in the Ramsey planner allocation in order to reach emission limits; compare panel (a). There is a reduction in hours worked for high- and low-skilled labor over time as emission limits become stricter. Relative to a scenario without emission limit the efficient level of hours also reduces; see panels (b) and (c).\footnote{\ In appendix section \ref{app:eff_notarg}, I show how the social planner allocation with emission limit compares to the efficient allocation without limit in figure \ref{fig:eff_with_notarget}. For the given calibration, the planner reduces hours worked once there is an emission limit. Since with log-utility the income and substitution effect cancel, the social planner does not increase hours worked to compensate for lower output. Compare the discussion of the efficient allocation in section \ref{sec:theory}.} 
The optimal allocation mimics the reduction in hours worked; yet, for the high-skilled the reduction is too strong starting from 2050m and it is too small for low-skilled labor. 

The higher consumption levels at lower hours worked in the efficient allocation are driven by higher growth rates; see panel (d) showing aggregate growth. The ratio of green to fossil energy, depicted in panel (e), moves similarly in the efficient and the optimal allocation. This illustrates the recomposing quality of the efficient allocation to cope with stricter emission limits. However, the Ramsey planner chooses a lower green-to-fossil energy mix starting from 2050. Utility of the representative household in the efficient and the optimal allocation is decreasing overall. The reduction happens in steps when a tighter emission limit becomes active. Utility is rising when the emission limit is stable and more so under the social planner. Recall that utility is net of environmental concerns solely determined by consumption and leisure in the present framework. 

\subsubsection{Comparison to other policy regimes}
How does the optimal allocation and especially its relation to the efficient allocation change under alternative policy scenarios?
In this section, I discuss two policy alternations which have already been discussed in the analytical section. First, a version where environmental tax revenues are consumed by the government and no labor income tax scheme is available, henceforth referred to as \textit{separate policy}. The comparison of this scenario serves to assess the benefits of an integrated environmental-fiscal policy when no lump-sum transfers are available. 
Second, I look at the optimal allocation when lump-sum transfers are available. According to the theory in section \ref{sec:mod_an}, the availability of lump-sum taxes should (i) deprive the income tax scheme of its use for the environmental policy. I show that labor income taxes are now used to boost growth and the environmental tax and fiscal policy are separate in optimum. 

\paragraph{Welfare comparison of policy regimes}
Table \ref{tab:swf} contrasts social welfare levels across policy regimes with and without income tax. The highest level of social welfare is attained in the regime with lump-sum transfers of environmental tax revenues and with income tax. When no lump-sum redistribution of environmental tax revenues is feasible, an integrated policy regime, that is, environmental tax revenues are redistributed via the income tax scheme, is preferable: social welfare increases by almost 7.45\%.  The social welfare level attained under the integrated policy comes close to the regime with lump-sum transfers. 
\begin{table}\caption{Social welfare across policy regimes}\label{tab:swf}
\begin{tabular}{l|ll}
	Policy Regime & With income tax & No income tax\\ 
	\hline 
	Integrated&-7.1575 & -7.1656\\
	Separate & -7.7334 & -7.7486 \\
	Lump-sum transfers & -7.1559 & -7.1561\\
%	BAU & -6.6269 & 0 & 0 & 0 & 0 & 0 \\ 
%	LF & -6.5382 & 0 & 0 & 0 & 0 & 0 \\ 
	%OPT NOT & -6.5378 & -6.5381 & -6.5382 & -6.5378 & -6.5378 & -6.5382 \\ 
%	&  & & \\ 
	%SP NOT & -6.3283 & 0 & 0 & 0 & 0 & 0 \\ 
%	SP T & -6.9511 & 0 & 0 & 0 & 0 & 0 \\ 
	\hline 
\end{tabular}
\end{table}
\paragraph{Comparison integrated policy to separate policy}

Consider figure \ref{fig:bench_nored_notaul}. The figure presents the optimal allocation in the integrated policy scenario,  the orange-dashed graph, the optimal allocation under the separate policy, the blue-dotted graph, and the efficient allocation, the black-solid graph.

In comparison to a policy scenario where environmental tax revenues are not redistributed, the integrated policy closer resembles the efficient allocation in terms of consumption, panel (a) and of labor, panels (b) and (c). %In total, the utility level of the representative household is at least as close to the efficient level for all time periods considered. 
The benefits of an integrated-policy regime come at the cost of a lower green-to-fossil energy mix, panel (e), and a reduction in growth, panel (d). Nevertheless, if a planner could choose between the two regimes, it would select the integrated-policy regime. The gains from the integrated regime amount to xxx. \tr{Do CEV}

Interestingly, the optimal environmental tax is only negligibly smaller in the integrated-policy regime. This suggests, that environmental taxes and labor income taxes are complements in the optimal environmental policy to lower inefficiently high hours worked. Only in the period from 2030 to 2050 the environmental tax necessary to meet emission limits is slightly smaller which can be rationalized by a lower level of production.\footnote{\ Absent an emission limit before 2030, the optimal environmental tax is slightly negative to subsidize fossil research which again spills over to research in the other sectors. }  

\begin{figure}[h!!]
	\centering
	\caption{Comparison to separate policy scenario; \tr{drop efficient from tauf graph }}\label{fig:bench_nored_notaul}
	
	\begin{minipage}[]{0.32\textwidth}
		\centering{\footnotesize{(a) Consumption}}
		%	\captionsetup{width=.45\linewidth}
		\includegraphics[width=1\textwidth]{../../codding_model/own_basedOnFried/optimalPol_190722_tidiedUp/figures/all_July22/C_CompEffOPT_T_NoTaus_pol2_spillover0_noskill0_sep1_xgrowth0_etaa0.79_lgd1_lff0.png}
	\end{minipage}
	\begin{minipage}[]{0.32\textwidth}
		\centering{\footnotesize{(b) High skill hours worked}}
		%	\captionsetup{width=.45\linewidth}
		\includegraphics[width=1\textwidth]{../../codding_model/own_basedOnFried/optimalPol_190722_tidiedUp/figures/all_July22/hh_CompEffOPT_T_NoTaus_pol2_spillover0_noskill0_sep1_xgrowth0_etaa0.79_lgd0_lff0.png}
	\end{minipage}
	\begin{minipage}[]{0.32\textwidth}
		\centering{\footnotesize{(c) Low skill hours worked}}
		%	\captionsetup{width=.45\linewidth}
		\includegraphics[width=1\textwidth]{../../codding_model/own_basedOnFried/optimalPol_190722_tidiedUp/figures/all_July22/hl_CompEffOPT_T_NoTaus_pol2_spillover0_noskill0_sep1_xgrowth0_etaa0.79_lgd0_lff0.png}
	\end{minipage}
	\begin{minipage}[]{0.32\textwidth}
		\centering{\footnotesize{(d) Aggregate growth}}
		%	\captionsetup{width=.45\linewidth}
		\includegraphics[width=1\textwidth]{../../codding_model/own_basedOnFried/optimalPol_190722_tidiedUp/figures/all_July22/gAagg_CompEffOPT_T_NoTaus_pol2_spillover0_noskill0_sep1_xgrowth0_etaa0.79_lgd0_lff0.png}
	\end{minipage}
	\begin{minipage}[]{0.32\textwidth}
		\centering{\footnotesize{(e) Energy mix, $\frac{G}{F}$}}
		%	\captionsetup{width=.45\linewidth}
		\includegraphics[width=1\textwidth]{../../codding_model/own_basedOnFried/optimalPol_190722_tidiedUp/figures/all_July22/GFF_CompEffOPT_T_NoTaus_pol2_spillover0_noskill0_sep1_xgrowth0_etaa0.79_lgd0_lff0.png}
	\end{minipage}
	%	\begin{minipage}[]{0.32\textwidth}
	%	\centering{\footnotesize{(f) Utility}}
	%	%	\captionsetup{width=.45\linewidth}
	%	\includegraphics[width=1\textwidth]{../../codding_model/own_basedOnFried/optimalPol_190722_tidiedUp/figures/all_July22/SWF_CompEffOPT_T_NoTaus_pol2_spillover0_noskill0_sep1_xgrowth0_etaa0.79_lgd0_lff0.png}
	%\end{minipage}
	\begin{minipage}[]{0.32\textwidth}
		\centering{\footnotesize{(f) Environmental tax, $\tau_{ft}$}}
		%	\captionsetup{width=.45\linewidth}
		\includegraphics[width=1\textwidth]{../../codding_model/own_basedOnFried/optimalPol_190722_tidiedUp/figures/all_July22/tauf_CompEffOPT_T_NoTaus_pol2_spillover0_noskill0_sep1_xgrowth0_etaa0.79_lgd0_lff0.png}
	\end{minipage}
\end{figure}



\paragraph{Comparison to regime with lump-sum transfers}\label{par:comp_lumpsum}


\begin{figure}[h!!]
	\centering
	\caption{Comparison integrated regime and regime lump-sum transfers}\label{fig:bench_lumpsum}
	
	\begin{minipage}[]{0.32\textwidth}
		\centering{\footnotesize{(a) Consumption}}
		%	\captionsetup{width=.45\linewidth}
		\includegraphics[width=1\textwidth]{../../codding_model/own_basedOnFried/optimalPol_190722_tidiedUp/figures/all_July22/C_CompEffOPT_T_NoTaus_pol4_spillover0_noskill0_sep1_xgrowth0_etaa0.79_lgd1_lff0.png}
	\end{minipage}
	\begin{minipage}[]{0.32\textwidth}
		\centering{\footnotesize{(b) High skill hours worked}}
		%	\captionsetup{width=.45\linewidth}
		\includegraphics[width=1\textwidth]{../../codding_model/own_basedOnFried/optimalPol_190722_tidiedUp/figures/all_July22/hh_CompEffOPT_T_NoTaus_pol4_spillover0_noskill0_sep1_xgrowth0_etaa0.79_lgd0_lff0.png}
	\end{minipage}
	\begin{minipage}[]{0.32\textwidth}
		\centering{\footnotesize{(c) Low skill hours worked}}
		%	\captionsetup{width=.45\linewidth}
		\includegraphics[width=1\textwidth]{../../codding_model/own_basedOnFried/optimalPol_190722_tidiedUp/figures/all_July22/hl_CompEffOPT_T_NoTaus_pol4_spillover0_noskill0_sep1_xgrowth0_etaa0.79_lgd0_lff0.png}
	\end{minipage}
	\begin{minipage}[]{0.32\textwidth}
		\centering{\footnotesize{(d) Aggregate growth}}
		%	\captionsetup{width=.45\linewidth}
		\includegraphics[width=1\textwidth]{../../codding_model/own_basedOnFried/optimalPol_190722_tidiedUp/figures/all_July22/gAagg_CompEffOPT_T_NoTaus_pol4_spillover0_noskill0_sep1_xgrowth0_etaa0.79_lgd0_lff0.png}
	\end{minipage}
	\begin{minipage}[]{0.32\textwidth}
		\centering{\footnotesize{(e) Energy mix, $\frac{G}{F}$}}
		%	\captionsetup{width=.45\linewidth}
		\includegraphics[width=1\textwidth]{../../codding_model/own_basedOnFried/optimalPol_190722_tidiedUp/figures/all_July22/GFF_CompEffOPT_T_NoTaus_pol4_spillover0_noskill0_sep1_xgrowth0_etaa0.79_lgd0_lff0.png}
	\end{minipage}
	\begin{minipage}[]{0.32\textwidth}
		\centering{\footnotesize{(f) Utility}}
		%	\captionsetup{width=.45\linewidth}
		\includegraphics[width=1\textwidth]{../../codding_model/own_basedOnFried/optimalPol_190722_tidiedUp/figures/all_July22/SWF_CompEffOPT_T_NoTaus_pol4_spillover0_noskill0_sep1_xgrowth0_etaa0.79_lgd0_lff0.png}
	\end{minipage}
\end{figure}

When lump-sum transfers of environmental tax revenues are in the policy set, the Ramsey planner can implement hours worked closer to the efficient allocation: figure \ref{fig:bench_lumpsum} contrasts the efficient allocation in black, the allocation under the benchmark policy, the blue-dotted graph, and the optimal allocation when lump-sum transfers are available in orange-dashed graph. 

While hours more closely mirror the efficient level, panels (b) and (c), consumption under the regime with lump-sum transfers is even lower than in the integrated policy regime, see panel (a). Utility gains from the availability of lump-sum transfers are minor but positive especially under the net-zero emission limit; compare panel (f).


\begin{figure}[h!!]
	\centering
	\caption{Optimal policy in integrated regime and with lump-sum transfers}\label{fig:bench_lumpsum_pol}
	
	\begin{minipage}[]{0.32\textwidth}
		\centering{\footnotesize{(a) Income tax progressivity, $\tau_{\iota t}$}}
		%	\captionsetup{width=.45\linewidth}
		\includegraphics[width=1\textwidth]{../../codding_model/own_basedOnFried/optimalPol_190722_tidiedUp/figures/all_July22/comp_notaul4_OPT_T_NoTaus_taul_spillover0_noskill0_sep1_xgrowth0_etaa0.79_lgd1.png}
	\end{minipage}
	\begin{minipage}[]{0.32\textwidth}
		\centering{\footnotesize{(b) Environmental tax, $\tau_{ft}$}}
		%	\captionsetup{width=.45\linewidth}
		\includegraphics[width=1\textwidth]{../../codding_model/own_basedOnFried/optimalPol_190722_tidiedUp/figures/all_July22/comp_notaul4_OPT_T_NoTaus_tauf_spillover0_noskill0_sep1_xgrowth0_etaa0.79_lgd0.png}
	\end{minipage}
	\begin{minipage}[]{0.32\textwidth}
		\centering{\footnotesize{(c) Lump-sum transfers}}
		%	\captionsetup{width=.45\linewidth}
		\includegraphics[width=1\textwidth]{../../codding_model/own_basedOnFried/optimalPol_190722_tidiedUp/figures/all_July22/comp_notaul4_OPT_T_NoTaus_Tls_spillover0_noskill0_sep1_xgrowth0_etaa0.79_lgd0.png}
	\end{minipage}
\end{figure}


Figure \ref{fig:bench_lumpsum_pol} shows the optimal policy the planner chooses when lump-sum transfers are available. 
Now, the optimal income tax scheme is regressive. Since lump-sum transfers ensure a reduction in labor supply the motive for progressive income taxes vanished.
The income tax in the model with lump-sum transfers of environmental tax revenues is used to boost growth and consumption. This becomes obvious when looking at a version of the model with exogenous growth, see figure \ref{fig:lumpsum_xgr_vglNotaul} in appendix section \ref{app:lumps}. Then, the income tax is not used. In fact, the allocation attained in the model without endogenous growth but lump-sum transfers suggests to be efficient despite skill heterogeneity. Hence, the environmental policy does not rely on the recomposing effect of regressive income taxes towards green production. It is solely the environmental tax and not the income tax which addresses the environmental externality. 

The use of regressive income taxes to boost growth comes at the cost of less leisure, see panel (b) and (c) in figure \ref{fig:bench_lumpsum_vglNotaul}, and a slightly higher environmental tax to meet the emission target due to higher production, see panel (i) in the same figure. Note that regressivity of the optimal income tax is declining as the emission limit becomes stricter. This is so even though consumption is lower. I conclude from this that due to the severity of the environmental externality, the planner forfeits enhancing growth through more production. The costs a higher environmental tax would incure to meet the emission limit at higher labor effort exceed the benefits of more growth. 

\subsubsection{The role of skill heterogeneity and endogenous growth}\label{subsec:simpler}
%\ar Does integrated policy come close to efficient one or the one with lump-sum transfers (should be the case according to theory where I have shown that the integrated policy replicates the efficient one without endogenous growth and skill heterogeneity)

In this section, I briefly discuss the role of skill heterogeneity and of endogenous growth as drivers of the quantitative results.
To do so, I simplify the model on the different dimensions: the next section discusses results without skill heterogeneity, the subsequent one results with exogenous growth. The last section verifies that the theoretic results hold true in the model with neither heterogeneous skills nor endogenous growth, i.e. a quantitative version of the model used to derive the theoretic results.\footnote{\ The models still differ by the assumption of a non-energy sector in the quantitative model. Nevertheless, the results hold true that the efficient allocation is attainable under both an integrated-policy regime and a regime with lump-sum transfers.} 

%\paragraph{No skill heterogeneity}
With homogeneous skills, the optimal allocation with lump-sum transfers and the integrated policy is exactly the same as shown by figure \ref{fig:bench_lumpsum_noskill}. Due to endogenous growth, however, they do not attain the efficient allocation: growth is too low. 
The environmental tax equals the wedge a social planner chooses for the marginal product of labor across sectors; compare panel (h). 

%\paragraph{Exogenous growth}
Figure \ref{fig:bench_lumpsum_xgr} shows the optimal allocation without endogenous growth. When lump-sum transfers are available, the Ramsey planner attains the efficient allocation. In line with the literature, it is endogenous growth which makes environmental policy costly \citep{Fried2018ClimateAnalysis, Acemoglu2012TheChange}. 
With lump-sum transfers, the income tax scheme is not used as discussed above. When lump-sum transfers are not available, the planner reverts to income taxes to lower hours worked; since growth is exogenous the level of consumption in the integrated-policy regime exceeds the efficient level.  



%\paragraph{Homogeneous skills and exogenous growth}
Absent both endogenous growth and skill heterogeneity the Ramsey allocation mirrors the efficient allocation.  The progressive income tax scheme does not face a trade-off between reducing high-skill supply too much and low-skill supply too little. The integrated-policy regime, therefore, is capable of implementing the efficient allocation. In contrast, lump-sum transfers also implement the efficient allocation with skill heterogeneity.  
