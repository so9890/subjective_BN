\section{Quantitative results}\label{sec:res}

In this section, I present and discuss the quantitative results.
Subsection \ref{subsec:mr} depicts the optimal policy given the emission target. Subsection \ref{subsec:dis} discusses the results. In particular, I focus on analyzing the mechanisms and welfare benefits from integrating the income tax scheme into the environmental policy. 

\subsection{Results}\label{subsec:mr}
%This section depicts results on the optimal policy followed by the implied allocation in the benchmark model where environmental tax revenues are redistributed via the income tax scheme. 

\begin{figure}[h!!]
	\centering
	\caption{Optimal Policy }\label{fig:optPol}
	\begin{minipage}[]{0.4\textwidth}
		\centering{\footnotesize{(a) Income tax progressivity, $\tau_{lt}$}}
		%	\captionsetup{width=.45\linewidth}
		\includegraphics[width=1\textwidth]{../../codding_model/own_basedOnFried/optimalPol_elastS_DisuSci/figures/all_1705/Single_OPT_T_NoTaus_taul_spillover0_sep1_BN0_ineq0_red0_etaa0.79.png}
	\end{minipage}
	\begin{minipage}[]{0.1\textwidth}
		\
	\end{minipage}
	\begin{minipage}[]{0.4\textwidth}
		\centering{\footnotesize{(b) Environmental tax, $\tau_{ft}$ }}
		%	\captionsetup{width=.45\linewidth}
		\includegraphics[width=1\textwidth]{../../codding_model/own_basedOnFried/optimalPol_elastS_DisuSci/figures/all_1705/Single_OPT_T_NoTaus_tauf_spillover0_sep1_BN0_ineq0_red0_etaa0.79.png}
	\end{minipage}
\end{figure} 

To meet the IPCCs suggested emission target, the optimal income tax is progressive for all periods between 2030 and 2080; see panel (a) in figure \ref{fig:optPol}. As the emission target is less strict, between 2030 to 2045, optimal income tax progressivity is around $\tau_{lt}=0.04$. As the emission target jumps to net-zero emissions in 2050, optimal tax progressivity accelerates to above 0.08 and gradually increases in the subsequent years to around 0.09. This is approximately half the size found for the US in \cite{Heathcote2017OptimalFramework}: $\tau_{l}=0.181$. 
In the period without emission target from 2020 to 2030, the optimal income tax is slightly regressive.

Consider panel (b). The optimal fossil tax displays a similar step pattern as the income tax progressivity. From 2020 to the beginning of 2030, it is negative. It jumps to around 70\% as the emission target is to reduce emissions by 50\% relative to 2019 emissions. As the emission target rises  to net-zero emissions in 2050, the optimal tax on fossil sales is close to 90\%. 

Figure \ref{fig:optAll} depicts the optimal allocation while meeting emission targets. Limiting emissions in line with the Paris Agreement is concomitant with both a reduction and recomposition of consumption and production over time. 

Panel (a) shows consumption which reduces significantly when new emission limits become active, in 2030 and in 2050, but starting from the new low levels continues to grow. labour effort of both skill types also reduces visibly as stricter emission targets are enforced; panel (b). In contrast to consumption, hours worked for both types of labour decrease over time. In comparison to hours supplied by low-skilled workers, high-skilled workers reduce hours more; compare panel (c) which shows the ratio of hours worked by high to low skill workers. 

The rise in consumption after each reduction is driven by technological progress in all sectors; compare panel (d) which shows growth rates by sector and as aggregate in per cent. 
The green sector sees a rise in technological progress, the dashed black line, while growth in the fossil and the non-energy sector is positive, yet diminishing over time. Overall, aggregate growth is positive but decreasing; compare the grey dashed graph. 
Summing up the last two paragraphs, the emission target is best achieved with more leisure at higher technology levels in all sectors. 

Nevertheless, there would be potential for more growth which is forfeited to meet emission targets. This becomes apparent when looking at the allocation of scientists in panel (e). Again, there is a recomposition towards the green sector: while research in the non-energy and the fossil sector decrease over time, green research effort rises. Yet, overall, the amount of scientists reduces; compare the grey graph which depicts the sum of researchers across sectors. 
Finally, labour input goods are also redirected towards the green sector; see panel (f). 

\begin{figure}[h!!]
	\centering
	\caption{Optimal Allocation }\label{fig:optAll}
	
	
	\begin{minipage}[]{0.32\textwidth}
		\centering{\footnotesize{(a) Consumption}}
		%	\captionsetup{width=.45\linewidth}
		\includegraphics[width=1\textwidth]{../../codding_model/own_basedOnFried/optimalPol_elastS_DisuSci/figures/all_1705/Single_OPT_T_NoTaus_C_spillover0_sep1_BN0_ineq0_red0_etaa0.79.png}
	\end{minipage}
	\begin{minipage}[]{0.32\textwidth}
		\centering{\footnotesize{(b) Hours worked }}
		%	\captionsetup{width=.45\linewidth}
		\includegraphics[width=1\textwidth]{../../codding_model/own_basedOnFried/optimalPol_elastS_DisuSci/figures/all_1705/SingleJointTOT_OPT_T_NoTaus_labour_spillover0_sep1_BN0_ineq0_red0_etaa0.79_lgd1.png}
	\end{minipage}
	\begin{minipage}[]{0.32\textwidth}
		\centering{\footnotesize{(c) High-to-low-skill ratio hours}}
		%	\captionsetup{width=.45\linewidth}
		\includegraphics[width=1\textwidth]{../../codding_model/own_basedOnFried/optimalPol_elastS_DisuSci/figures/all_1705/Single_OPT_T_NoTaus_hhhl_spillover0_sep1_BN0_ineq0_red0_etaa0.79.png}
	\end{minipage}
	\begin{minipage}[]{0.32\textwidth}
		\centering{\footnotesize{\ \\ (d) Technology growth}}
		%	\captionsetup{width=.45\linewidth}
		\includegraphics[width=1\textwidth]{../../codding_model/own_basedOnFried/optimalPol_elastS_DisuSci/figures/all_1705/SingleJointTOT_OPT_T_NoTaus_Growth_spillover0_sep1_BN0_ineq0_red0_etaa0.79_lgd1.png}
	\end{minipage}
	\begin{minipage}[]{0.32\textwidth}
		\centering{\footnotesize{\ \\(e) Scientists }}
		%	\captionsetup{width=.45\linewidth}
		\includegraphics[width=1\textwidth]{../../codding_model/own_basedOnFried/optimalPol_elastS_DisuSci/figures/all_1705/SingleJointTOT_OPT_T_NoTaus_Science_spillover0_sep1_BN0_ineq0_red0_etaa0.79_lgd1.png}
	\end{minipage}
	\begin{minipage}[]{0.32\textwidth}
		\centering{\footnotesize{\ \\(f) Labor input}}
		%	\captionsetup{width=.45\linewidth}
		\includegraphics[width=1\textwidth]{../../codding_model/own_basedOnFried/optimalPol_elastS_DisuSci/figures/all_1705/SingleJointTOT_OPT_T_NoTaus_labourInp_spillover0_sep1_BN0_ineq0_red0_etaa0.79_lgd1.png}
	\end{minipage}
\end{figure} 



\subsection{Discussion}\label{subsec:dis}
The discussion of the optimal policy centers on the questions (i) what drives the optimal policy and (ii) how to assess the optimal allocation: what are the benefits of labor income taxes, finally, (iii) what effects do skill heterogeneity and endogeneous growth have on the optimal policy (effect on tauf, taul and on gap to efficient allocation)
\begin{enumerate}
	\item What is the goal of policy intervention? \ar social planner allocation
	\item (Benefits) What is different when no integrated policy is run and instead revs consumed by government \ar Benefits of an integrated policy
	\item double dividend literature: use of labor income tax when all env tax revenues are consumed by the government.
	\item (Costs) What cannot be reached by integrated policy as compared to lump-sum transfers: is taul used for different purpose? without endogenous growth should be zero; eg. can use taul to boost growth as lump-sum transfers take care of labor supply 
	\item What could be reached if there was no trade-off with heterogenous skills or growth? no heterogeneous skills, no endogenous growth \ar how does the optimal policy differ?
\end{enumerate}

\subsubsection{Social planner allocation}
As a benchmark to the Ramsey planner allocation, I present the social planner's allocation. The efficient allocation can be perceived as the intended allocation which the Ramsey planner may not be able to implement due to the reliance on tax instruments. Figure \ref{fig:fb_opt} depicts the efficient and the optimal allocation by the black-solid and the orange-dashed graphs. 
\begin{figure}[h!!]
	\centering
	\caption{Comparison to efficient allocation }\label{fig:fb_opt}
	
	\begin{minipage}[]{0.32\textwidth}
		\centering{\footnotesize{(a) Consumption}}
		%	\captionsetup{width=.45\linewidth}
		\includegraphics[width=1\textwidth]{../../codding_model/own_basedOnFried/optimalPol_190722_tidiedUp/figures/all_July22/C_CompEffOPT_T_NoTaus_opteff_spillover0_noskill0_sep1_xgrowth0_countec0_etaa0.79_lgd1_lff0.png}
	\end{minipage}
	\begin{minipage}[]{0.32\textwidth}
	\centering{\footnotesize{(b) High skill hours worked}}
	%	\captionsetup{width=.45\linewidth}
	\includegraphics[width=1\textwidth]{../../codding_model/own_basedOnFried/optimalPol_190722_tidiedUp/figures/all_July22/hh_CompEffOPT_T_NoTaus_opteff_spillover0_noskill0_sep1_xgrowth0_countec0_etaa0.79_lgd0_lff0.png}
\end{minipage}
	\begin{minipage}[]{0.32\textwidth}
	\centering{\footnotesize{(c) Low skill hours worked}}
	%	\captionsetup{width=.45\linewidth}
	\includegraphics[width=1\textwidth]{../../codding_model/own_basedOnFried/optimalPol_190722_tidiedUp/figures/all_July22/hl_CompEffOPT_T_NoTaus_opteff_spillover0_noskill0_sep1_xgrowth0_countec0_etaa0.79_lgd0_lff0.png}
\end{minipage}

	\begin{minipage}[]{0.32\textwidth}
	\centering{\footnotesize{(d) Aggregate growth}}
	%	\captionsetup{width=.45\linewidth}
	\includegraphics[width=1\textwidth]{../../codding_model/own_basedOnFried/optimalPol_190722_tidiedUp/figures/all_July22/gAagg_CompEffOPT_T_NoTaus_opteff_spillover0_noskill0_sep1_xgrowth0_countec0_etaa0.79_lgd0_lff0.png}
\end{minipage}
\begin{minipage}[]{0.32\textwidth}
	\centering{\footnotesize{(e) Energy mix, $\frac{G}{F}$}}
	%	\captionsetup{width=.45\linewidth}
	\includegraphics[width=1\textwidth]{../../codding_model/own_basedOnFried/optimalPol_190722_tidiedUp/figures/all_July22/GFF_CompEffOPT_T_NoTaus_opteff_spillover0_noskill0_sep1_xgrowth0_countec0_etaa0.79_lgd0_lff0.png}
\end{minipage}
\begin{minipage}[]{0.32\textwidth}
	\centering{\footnotesize{(f) Utility}}
	%	\captionsetup{width=.45\linewidth}
	\includegraphics[width=1\textwidth]{../../codding_model/own_basedOnFried/optimalPol_190722_tidiedUp/figures/all_July22/SWF_CompEffOPT_T_NoTaus_opteff_spillover0_noskill0_sep1_xgrowth0_countec0_etaa0.79_lgd0_lff0.png}
\end{minipage}
\end{figure}

The social planner reduces consumption less than in the Ramsey planner allocation in order to reach emission limits; compare panel (a). There is a reduction in hours worked for high- and low-skilled labor over time as emission limits become stricter and also relative to a scenario without emission limit; see panels (b) and (c).\footnote{\ In appendix section \ref{app:eff_notarg}, I show how the social planner allocation with emission limit compares to the efficient allocation without limit. For the given calibration, the planner reduces hours worked once there is an emission limit.} 
The optimal allocation mimics the reduction in hours worked; yet, for the high-skilled the reduction is too strong starting from 2050. It is too small for low-skilled labor. 

The higher consumption levels at lower hours worked in the efficient allocation are driven by higher growth rates; see panel (d) which shows aggregate growth. The ratio of green to fossil energy, depicted in panel (e), moves similarly in the efficient and the optimal allocation. However, the Ramsey planner chooses a lower green-to-fossil energy mix starting from 2050. Utility of the representative household in the efficient and the optimal allocation is decreasing overall. The reduction happens in steps when a tighter emission limit becomes active. Utility is rising when the emission limit is stable and more so under the social planner. 

\subsubsection{Comparison other policy scenarios}
How does the optimal allocation and especially its relation to the efficient allocation change under alternative policy scenarios?
In this section, I discuss three policy alternations which have already been discussed in the analytical section. First, a version where environmental tax revenues are consumed by the government and no labor income tax scheme is available, henceforth referred to as \textit{separate policy}. The comparison of this scenario serves to assess the benefits of an integrated environmental-fiscal policy when no lump-sum transfers are available. 


Second, I show the optimal allocation in the scenario without redistribution but the government can use labor income taxes in comparison to the separate regime. Contrasting these two regimes highlights the benefits of labor income taxes  when environmental tax revenues are not redistributed. The results speak to the double-dividend literature arguing that there is a lower bound on the reduction in distortionary labor income taxes when environmental tax revenues are not redistributed lump-sum. \tr{Think of how to deal with channel that labor income taxes have the advantage to be redistributed....JUST DISCUSS IT AND THAT IT IS UNLIKELY;  }

Finally, I look at the optimal allocation when lump-sum transfers are available. According to the theory in section \ref{sec:mod_an}, the availability of lump-sum taxes should (i) deprive the income tax scheme of its use for the environmental policy, and (ii) result in an allocation closer to the efficient allocation than the benchmark scenario, i.e. the integrated policy. 


\paragraph{Comparison integrated policy to separate policy without lump-sum transfers}

Consider figure \ref{fig:bench_nored_notaul}. The figure presents the optimal allocation in the integrated policy scenario, the benchmark policy,  the orange-dashed graph, the optimal allocation under the separate policy, the blue-dotted graph, and the efficient allocation, the black-solid graph.

In comparison to a policy scenario where environmental tax revenues are not redistributed, the integrated policy closer resembles the efficient allocation in terms of consumption, panel (a) and of labor, panels (b) and (c). %In total, the utility level of the representative household is at least as close to the efficient level for all time periods considered. 
The benefits of an integrated policy regime come at the cost of a lower green-to-fossil energy mix. 

Interestingly, the optimal environmental tax is only negligibly smaller in the integrated policy regime. This suggests, that environmental taxes and labor income taxes are complements in the optimal environmental policy to lower inefficiently high hours worked. 

\begin{figure}[h!!]
	\centering
	\caption{Comparison to separate policy scenario; \tr{drop efficient from tauf graph }}\label{fig:bench_nored_notaul}
	
	\begin{minipage}[]{0.32\textwidth}
		\centering{\footnotesize{(a) Consumption}}
		%	\captionsetup{width=.45\linewidth}
		\includegraphics[width=1\textwidth]{../../codding_model/own_basedOnFried/optimalPol_190722_tidiedUp/figures/all_July22/C_CompEffOPT_T_NoTaus_pol2_spillover0_noskill0_sep1_xgrowth0_etaa0.79_lgd1_lff0.png}
	\end{minipage}
	\begin{minipage}[]{0.32\textwidth}
		\centering{\footnotesize{(b) High skill hours worked}}
		%	\captionsetup{width=.45\linewidth}
		\includegraphics[width=1\textwidth]{../../codding_model/own_basedOnFried/optimalPol_190722_tidiedUp/figures/all_July22/hh_CompEffOPT_T_NoTaus_pol2_spillover0_noskill0_sep1_xgrowth0_etaa0.79_lgd0_lff0.png}
	\end{minipage}
	\begin{minipage}[]{0.32\textwidth}
		\centering{\footnotesize{(c) Low skill hours worked}}
		%	\captionsetup{width=.45\linewidth}
		\includegraphics[width=1\textwidth]{../../codding_model/own_basedOnFried/optimalPol_190722_tidiedUp/figures/all_July22/hl_CompEffOPT_T_NoTaus_pol2_spillover0_noskill0_sep1_xgrowth0_etaa0.79_lgd0_lff0.png}
	\end{minipage}
	\begin{minipage}[]{0.32\textwidth}
		\centering{\footnotesize{(d) Aggregate growth}}
		%	\captionsetup{width=.45\linewidth}
		\includegraphics[width=1\textwidth]{../../codding_model/own_basedOnFried/optimalPol_190722_tidiedUp/figures/all_July22/gAagg_CompEffOPT_T_NoTaus_pol2_spillover0_noskill0_sep1_xgrowth0_etaa0.79_lgd0_lff0.png}
	\end{minipage}
	\begin{minipage}[]{0.32\textwidth}
		\centering{\footnotesize{(e) Energy mix, $\frac{G}{F}$}}
		%	\captionsetup{width=.45\linewidth}
		\includegraphics[width=1\textwidth]{../../codding_model/own_basedOnFried/optimalPol_190722_tidiedUp/figures/all_July22/GFF_CompEffOPT_T_NoTaus_pol2_spillover0_noskill0_sep1_xgrowth0_etaa0.79_lgd0_lff0.png}
	\end{minipage}
	%	\begin{minipage}[]{0.32\textwidth}
	%	\centering{\footnotesize{(f) Utility}}
	%	%	\captionsetup{width=.45\linewidth}
	%	\includegraphics[width=1\textwidth]{../../codding_model/own_basedOnFried/optimalPol_190722_tidiedUp/figures/all_July22/SWF_CompEffOPT_T_NoTaus_pol2_spillover0_noskill0_sep1_xgrowth0_etaa0.79_lgd0_lff0.png}
	%\end{minipage}
	\begin{minipage}[]{0.32\textwidth}
		\centering{\footnotesize{(f) Environmental tax, $\tau_{ft}$}}
		%	\captionsetup{width=.45\linewidth}
		\includegraphics[width=1\textwidth]{../../codding_model/own_basedOnFried/optimalPol_190722_tidiedUp/figures/all_July22/tauf_CompEffOPT_T_NoTaus_pol2_spillover0_noskill0_sep1_xgrowth0_etaa0.79_lgd0_lff0.png}
	\end{minipage}
\end{figure}


\paragraph{Comparison separate polcy with and without labor income tax}


In figure \ref{fig:comp_nored}, I contrast the optimal allocation in (i) the separate policy regime, orange-dotted graph, with (ii) a regime in which the government consumes environmental tax revenues but has the option to use a labor income tax. The solid black graph depicts the efficient allocation. 

These results speak to the double-dividend literature. When the government consumes environmental tax revenues completely - think of it as a scenario where environmental tax revenues are sufficient to meet the exogenous revenue constraint - then hours worked are inefficiently high; compare the orange-dashed to  the solid graph. The Ramsey planner can converge to the efficient allocation by employing a progressive labor income tax schedule. Indeed, this reduces consumption further away from the efficient level, but, hours worked are aligned closer to the efficient level, panels (b) and (c). Next to consumption, the planner also forfeits an advantageous green-to-fossil energy ratio, panel (e). 

The use of a progressive labor income tax contributes minimally to meeting the emission limit as can be seen by scrutinizing the optimal environmental tax, panel (b) in figure \ref{fig:comp_nored_pol}: when income taxes can be used, the environmental tax is slightly lower. Still, the difference is minimal, supporting the thesis of complementarity of income and environmental taxes. 
Environmental tax revenues are lower as the tax rate reduces, and income taxes reduce labor supply and hence the tax base of the environmental tax. 

This might be a motive to prefer labor income taxes as an instrument to reduce emissions since labor income tax revenues are redistributed back to the household while environmental tax revenues are not in this setting. Nevertheless, the observation that the environmental tax only adjusts slightly once an income tax tool is available points to the advantage of environmental taxes in handling too high emissions. 


%\tr{Could find that the labor income tax is used to reduce emissions because it does not reduce consumption! \ar maybe better to compare to a scenario where government revenues have to be equal to the env. tax revenues in the separate scenario, and then check if labor income taxes are still used? But this would reduce government revenues from the exogenous target! What would be a good comparison?}

\begin{figure}[h!!]
	\centering
	\caption{Comparison to separate policy scenario}\label{fig:comp_nored}
	
	\begin{minipage}[]{0.32\textwidth}
		\centering{\footnotesize{(a) Consumption}}
		%	\captionsetup{width=.45\linewidth}
		\includegraphics[width=1\textwidth]{../../codding_model/own_basedOnFried/optimalPol_190722_tidiedUp/figures/all_July22/C_DDCompEffOPT_T_NoTaus_pol3_spillover0_noskill0_sep1_xgrowth0_etaa0.79_lgd1_lff0.png}
	\end{minipage}
	\begin{minipage}[]{0.32\textwidth}
		\centering{\footnotesize{(b) High skill hours worked}}
		%	\captionsetup{width=.45\linewidth}
		\includegraphics[width=1\textwidth]{../../codding_model/own_basedOnFried/optimalPol_190722_tidiedUp/figures/all_July22/hh_DDCompEffOPT_T_NoTaus_pol3_spillover0_noskill0_sep1_xgrowth0_etaa0.79_lgd0_lff0.png}
	\end{minipage}
	\begin{minipage}[]{0.32\textwidth}
		\centering{\footnotesize{(c) Low skill hours worked}}
		%	\captionsetup{width=.45\linewidth}
		\includegraphics[width=1\textwidth]{../../codding_model/own_basedOnFried/optimalPol_190722_tidiedUp/figures/all_July22/hl_DDCompEffOPT_T_NoTaus_pol3_spillover0_noskill0_sep1_xgrowth0_etaa0.79_lgd0_lff0.png}
	\end{minipage}
	\begin{minipage}[]{0.32\textwidth}
		\centering{\footnotesize{(d) Aggregate growth}}
		%	\captionsetup{width=.45\linewidth}
		\includegraphics[width=1\textwidth]{../../codding_model/own_basedOnFried/optimalPol_190722_tidiedUp/figures/all_July22/gAagg_DDCompEffOPT_T_NoTaus_pol3_spillover0_noskill0_sep1_xgrowth0_etaa0.79_lgd0_lff0.png}
	\end{minipage}
	\begin{minipage}[]{0.32\textwidth}
		\centering{\footnotesize{(e) Energy mix, $\frac{G}{F}$}}
		%	\captionsetup{width=.45\linewidth}
		\includegraphics[width=1\textwidth]{../../codding_model/own_basedOnFried/optimalPol_190722_tidiedUp/figures/all_July22/GFF_DDCompEffOPT_T_NoTaus_pol3_spillover0_noskill0_sep1_xgrowth0_etaa0.79_lgd0_lff0.png}
	\end{minipage}
	\begin{minipage}[]{0.32\textwidth}
		\centering{\footnotesize{(f)Utility}}
		%	\captionsetup{width=.45\linewidth}
		\includegraphics[width=1\textwidth]{../../codding_model/own_basedOnFried/optimalPol_190722_tidiedUp/figures/all_July22/SWF_DDCompEffOPT_T_NoTaus_pol3_spillover0_noskill0_sep1_xgrowth0_etaa0.79_lgd0_lff0.png}
	\end{minipage}
\end{figure}

\begin{figure}[h!!]
	\centering
	\caption{Separate regime with and without income tax}\label{fig:comp_nored_pol}

\begin{minipage}[]{0.32\textwidth}
	\centering{\footnotesize{(a) Income tax progressivity, $\tau_{\iota t}$ }}
	%	\captionsetup{width=.45\linewidth}
	\includegraphics[width=1\textwidth]{../../codding_model/own_basedOnFried/optimalPol_190722_tidiedUp/figures/all_July22/taul_DDCompEffOPT_T_NoTaus_pol3_spillover0_noskill0_sep1_xgrowth0_etaa0.79_lgd1_lff0.png}
\end{minipage}
\begin{minipage}[]{0.32\textwidth}
	\centering{\footnotesize{(b) Environmental tax, $\tau_{ft}$}}
	%	\captionsetup{width=.45\linewidth}
	\includegraphics[width=1\textwidth]{../../codding_model/own_basedOnFried/optimalPol_190722_tidiedUp/figures/all_July22/tauf_DDCompEffOPT_T_NoTaus_pol3_spillover0_noskill0_sep1_xgrowth0_etaa0.79_lgd0_lff0.png}
\end{minipage}
\begin{minipage}[]{0.32\textwidth}
	\centering{\footnotesize{(c) Government consumption }}
	%	\captionsetup{width=.45\linewidth}
	\includegraphics[width=1\textwidth]{../../codding_model/own_basedOnFried/optimalPol_190722_tidiedUp/figures/all_July22/GovCon_DDCompEffOPT_T_NoTaus_pol3_spillover0_noskill0_sep1_xgrowth0_etaa0.79_lgd0_lff0.png}
\end{minipage}
\end{figure}

\paragraph{Comparison to model with lump-sum transfers}

When lump-sum transfers of environmental tax revenues are in the policy set, the Ramsey planner can implement an allocation closer to the efficient allocation but only minimally though. Figure \ref{fig:bench_lumpsum} contrasts the efficient allocation in black, the allocation under the benchmark policy, the blue-dotted graph, and the optimal allocation when lump-sum transfers are available in orange-dashed graph. 

Figure \ref{fig:bench_lumpsum_pol} shows the optimal policy the planner chooses when lump-sum transfers are available. 
Now, the optimal income tax scheme is regressive. Since lump-sum transfers ensure a reduction in labor supply the motive for progressive income taxes vanished. Instead, labor income taxes are used to 
\tr{Compare to version with lump sum transfers but without income taxes \ar what are the gains from labor income tax regressivity?} 



\begin{figure}[h!!]
	\centering
	\caption{Comparison integrates policy and lump-sum transferss}\label{fig:bench_lumpsum}
	
	\begin{minipage}[]{0.32\textwidth}
		\centering{\footnotesize{(a) Consumption}}
		%	\captionsetup{width=.45\linewidth}
		\includegraphics[width=1\textwidth]{../../codding_model/own_basedOnFried/optimalPol_190722_tidiedUp/figures/all_July22/C_CompEffOPT_T_NoTaus_pol4_spillover0_noskill0_sep1_xgrowth0_etaa0.79_lgd1_lff0.png}
	\end{minipage}
	\begin{minipage}[]{0.32\textwidth}
		\centering{\footnotesize{(b) High skill hours worked}}
		%	\captionsetup{width=.45\linewidth}
		\includegraphics[width=1\textwidth]{../../codding_model/own_basedOnFried/optimalPol_190722_tidiedUp/figures/all_July22/hh_CompEffOPT_T_NoTaus_pol4_spillover0_noskill0_sep1_xgrowth0_etaa0.79_lgd0_lff0.png}
	\end{minipage}
	\begin{minipage}[]{0.32\textwidth}
		\centering{\footnotesize{(c) Low skill hours worked}}
		%	\captionsetup{width=.45\linewidth}
		\includegraphics[width=1\textwidth]{../../codding_model/own_basedOnFried/optimalPol_190722_tidiedUp/figures/all_July22/hl_CompEffOPT_T_NoTaus_pol4_spillover0_noskill0_sep1_xgrowth0_etaa0.79_lgd0_lff0.png}
	\end{minipage}
	\begin{minipage}[]{0.32\textwidth}
		\centering{\footnotesize{(d) Aggregate growth}}
		%	\captionsetup{width=.45\linewidth}
		\includegraphics[width=1\textwidth]{../../codding_model/own_basedOnFried/optimalPol_190722_tidiedUp/figures/all_July22/gAagg_CompEffOPT_T_NoTaus_pol4_spillover0_noskill0_sep1_xgrowth0_etaa0.79_lgd0_lff0.png}
	\end{minipage}
	\begin{minipage}[]{0.32\textwidth}
		\centering{\footnotesize{(e) Energy mix, $\frac{G}{F}$}}
		%	\captionsetup{width=.45\linewidth}
		\includegraphics[width=1\textwidth]{../../codding_model/own_basedOnFried/optimalPol_190722_tidiedUp/figures/all_July22/GFF_CompEffOPT_T_NoTaus_pol4_spillover0_noskill0_sep1_xgrowth0_etaa0.79_lgd0_lff0.png}
	\end{minipage}
	\begin{minipage}[]{0.32\textwidth}
		\centering{\footnotesize{(f) Utility}}
		%	\captionsetup{width=.45\linewidth}
		\includegraphics[width=1\textwidth]{../../codding_model/own_basedOnFried/optimalPol_190722_tidiedUp/figures/all_July22/SWF_CompEffOPT_T_NoTaus_pol4_spillover0_noskill0_sep1_xgrowth0_etaa0.79_lgd0_lff0.png}
	\end{minipage}
\end{figure}

\begin{figure}[h!!]
	\centering
	\caption{Optimal policy with and without lump-sum transfers}\label{fig:bench_lumpsum_pol}
	
	\begin{minipage}[]{0.32\textwidth}
		\centering{\footnotesize{(a) Income tax progressivity, $\tau_{\iota t}$}}
		%	\captionsetup{width=.45\linewidth}
		\includegraphics[width=1\textwidth]{../../codding_model/own_basedOnFried/optimalPol_190722_tidiedUp/figures/all_July22/comp_notaul4_OPT_T_NoTaus_taul_spillover0_noskill0_sep1_xgrowth0_etaa0.79_lgd1.png}
	\end{minipage}
	\begin{minipage}[]{0.32\textwidth}
		\centering{\footnotesize{(b) Environmental tax, $\tau_{ft}$}}
		%	\captionsetup{width=.45\linewidth}
		\includegraphics[width=1\textwidth]{../../codding_model/own_basedOnFried/optimalPol_190722_tidiedUp/figures/all_July22/comp_notaul4_OPT_T_NoTaus_tauf_spillover0_noskill0_sep1_xgrowth0_etaa0.79_lgd0.png}
	\end{minipage}
	\begin{minipage}[]{0.32\textwidth}
		\centering{\footnotesize{(c) Lump-sum transfers}}
		%	\captionsetup{width=.45\linewidth}
		\includegraphics[width=1\textwidth]{../../codding_model/own_basedOnFried/optimalPol_190722_tidiedUp/figures/all_July22/comp_notaul4_OPT_T_NoTaus_Tls_spillover0_noskill0_sep1_xgrowth0_etaa0.79_lgd0.png}
	\end{minipage}
\end{figure}


\subsection{No heterogeneous skills, no endogenous growth}
\ar Does integrated policy come close to efficient one or the one with lump-sum transfers (should be the case according to theory)
