\paragraph{Literature}

The paper relates to several strands of literature. 
First, to the literature on environmental policies. Second, it connects to the literature on how to recycle environmental tax revenues. Third, as the paper combines environmental and fiscal policies it naturally connects to the public finance literature. Finally, the results speak to the literature discussing overconsumption  which may be social preferences or environmental constraints. 

 
\begin{itemize}
	\item How to use environmental tax revenues \citep{Fried2018TheGenerations}
	\item Optimal environmental policy \ar focuses on environmental taxes
	\item weak do
\end{itemize}


In general, macro papers on (optimal) environmental policy focus on environmental taxation and analyze settings with inelastic labor supply \citep{Golosov2014OptimalEquilibrium, Acemoglu2012TheChang, Fried2018ClimateAnalysis, Acemoglu2016TransitionTechnology}. The mentioned papers assume lump-sum transfers of environmental tax revenues, hence endogenizing labor supply would not affect the optimal policy; yet, the role of lump-sum transfers changes to a reductive policy measure.
% 
% Acemoglu 2016 have lump-sum transfers and taxes
% Acemoglu Aghion 2012: lump-sum transfers, no optimal policy
%Golosov: hightlight the need of lump-sum transfers! but exogenous labor supply
% Therefore, the main finding of the present paper, the necessity of reductive policy measures to implement the efficient allocation, complements this literature. 
%Especially, when environmental tax revenues are not redistributed lump-sum in these papers, a variable labor supply would give an argument for labor income taxation. 

%
%Furthermore, I argue that the Pigou principle does generally not apply when no lump-sum transfers are available.  

Another focus of the literature on environmental policy is the redistribution of environmental tax revenues. In contrast to the previous strand of literature, this one generally assumes labor supply to be elastic. 
The dominant focus of this literature is the weak double dividend of environmental taxes \citep{LansBovenberg1994EnvironmentalTaxation, LansBovenberg1996OptimalAnalyses, Bovenberg2002EnvironmentalRegulation,  Barrage2019OptimalPolicy}: given an exogenous government funding constraint it is optimal to recycle environmental tax revenues to lower distortionary labor income taxes as opposed to higher lump-sum transfers. The latter decreases labor supply through an income effect thereby decreasing the tax base of the labor income tax. Consequently, it becomes more expensive for the government to generate revenues.
The weak double-dividend literature rests on the assumption that no lump-sum transfers and taxes are available. Yet, when this is the case, this paper shows that absent a government funding constraint, distortionary taxes should be set to reduce labor supply: some reduction in hours is in fact efficient. Hence, this paper provides an upper bound on the reduction of distortionary income taxes. This becomes clear when environmental tax revenues suffice to meet the government's funding constraint, then labor supply would be inefficiently high when the labor income tax is unused. 
The analytical literature on optimal environmental policy treats the labor income tax as pre-existing \citep{LansBovenberg1994EnvironmentalTaxation, LansBovenberg1996OptimalAnalyses} in the settings when no lump-sum transfers are possible.
 

Building on the weak double-dividend literature, \cite{Fried2018TheGenerations} compare distinct scenarios of how to recycle environmental tax revenues and investigate the impact on inter- and intragenerational inequality in an overlapping generations model. Lump-sum transfers are preferred by the  living generation. 
In this literature, the advantage of different recycling means is often evaluated with respect to equity or political feasibility \cite{Carattini2018, VANDERPLOEG2022103966}. The present paper employs an efficient allocation as a benchmark to assess lump-sum transfers, government consumption, and redistribution via the income tax scheme. 

\cite{VANDERPLOEG2022103966} do comment on efficiency costs. 
What they do: 
structural estimate effects of carbon tax and lump-sum redistribution on households across income distribution; "better off" measured by utility: to capture both: consumption and labor supply!
The term efficiency is synonym to labor supply redcutions. Abstracting from the possibility that labor supply reductions may be advantageous;\tr{ do they assume an endogenous funding condition?} The politically most feasible recycling is through lower income tax rates, which boosts labor supply and economic activity, but hurts the poor. If equity is relevant, the government should do split env tax revenues: some recycling as carbon dividend, the other to lower income taxes.

\tr{1) What if there is a pre-existing income tax but not gov funding constraint? Do BLÖDsinnn; maybe not since could still find an increase in labor income tax if optimal level is above initial level;\\
 2) look at a policy where income taxes are used to fund government spending \ar then this would generate more gov. revenues }

Why do bov and de Mooji not find that the reduction in labor tax revenues depends on the level of the income tax? \ar because its a non-formal argument

equity: horowitz 2017: equity gains from lump-sum redistribution

Metcalf 2007/ 2008
Kotlikoff making carbon pricing a generational win win

\cite{LansBovenberg1996OptimalAnalyses}: they focus on how the optimal environmental policy deviates from the Pigou principle due to pre-existing distortionary taxes; 

\textbf{Goulder1995} defines the weak double dividend as: recycling revenues to lower pre existing tax distortions: one achieves \textbf{cost savings} relative to the case where the tax revenues are returned to taxpayers in lump-sum fashion; 
\textbf{It seems like the weak double dividend is defined in terms of costs}; costs (i.e. non-environmental costs) are the reduction in price times quantity relative to the equilibrium without tax; p.18: \textbf{the double dividend literature focuses on non-environmental costs of the environmental policy. And earlier works focus on consumption and output as only measure of gains/ missing gains from leisure. } 
Linking paper to double dividend literature means to compare efficiency gains to costs. Leisure versus consumption \ar but I show analytically that consumption costs are lower. 

\textcolor{blue}{so fare havent seen a paper which analytically shows the weak double dividend holds \ar could be that they indeed miss the lower bound, size dependency of advantage to reduce labor taxes}

\textit{To be fair, the double dividend literature focuses on cost advantages by using environmental tax revenues to substitute for distortionary labor income taxes. However, it remains unmentioned that under the assumption of elastic labor supply, which the literature necessarily assumes, the environmental tax alone is not efficient. }

\textbf{Bovenber 1998}: "\textit{environmental taxes are  generally  an  efficient  instrument  for  protecting  the  environment}"
\tr{This statement is wrong! they are only efficient in combination with reductive measures \ar there is no double-dividend as - to efficiently reduce emissions - environmental tax revenues have to be redistributed lump-sum}
\textcolor{blue}{the question arises what is better from an equity perspective: lump-sum transfers or additional progressive taxes?} Evaluate by looking at high and low skill wages. 
\\
Environmental tax revenues are not a free lunch. By not redistributing them lump-sum,there are efficiency costs because work effort would be too high.




\textit{Citation in Fried: Pigou 1920, Dales 1968, Montgomery 1972, Baumol and Oates 1988 }: "\textit{Establishin a price on carbon [...] is well understood to be the most efficient approach for reducing greenhouse gas emissions.}"

\paragraph{Reduction policies}