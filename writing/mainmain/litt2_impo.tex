\paragraph{Literature}
\tr{Divide into the quantitative part speaks to.... the analytical section contributes to..... }
The paper relates to several strands of literature. 
First, to the literature on environmental policies. Second, it connects to the literature connecting environmental and fiscal policies and how to recycle environmental tax revenues. Third, as the paper combines environmental and fiscal policies it naturally connects to the public finance literature. Finally, the results speak to the literature discussing inefficiently high production.
 
%\begin{itemize}
%	\item How to use environmental tax revenues \citep{Fried2018TheGenerations}
%	\item Optimal environmental policy \ar focuses on environmental taxes
%	\item weak double dividend
%	\item to be incorporated: \tr{\cite{Metcalf2003EnvironmentalPollution} why does he find that the optimal pigou tax equals first best when gov spending is satisfied with tax revenues? }
%	\\
%	Williams III: Welfare improvement with xxx \citep{Parry1999WhenMarkets} \tr{is this weak or strong dd?}
%\end{itemize}

%---------------------------------------
%.. optimal environmental policy
%---------------------------------------
In general, macro papers on (optimal) environmental policy focus on environmental taxation and analyze settings with inelastic labor supply \citep{Golosov2014OptimalEquilibrium, Acemoglu2012TheChang, Fried2018ClimateAnalysis, Acemoglu2016TransitionTechnology}. The mentioned papers assume lump-sum transfers of environmental tax revenues, hence endogenizing labor supply would not affect the optimal policy; yet, the role of lump-sum transfers changes to a reductive policy measure. The reduction in labor supply induced by the environmental policy might, nevertheless, affect the results. \textit{HOW?}
The quantitative analysis sheds light on the effect of reductive policies on a green transition and growth. \textit{sss}
% OVERVIEW LITERATURE
% Acemoglu 2016 have lump-sum transfers and taxes
% Acemoglu Aghion 2012: lump-sum transfers, no optimal policy
%Golosov: hightlight the need of lump-sum transfers! but exogenous labor supply
% Therefore, the main finding of the present paper, the necessity of reductive policy measures to implement the efficient allocation, complements this literature. 
%Especially, when environmental tax revenues are not redistributed lump-sum in these papers, a variable labor supply would give an argument for labor income taxation. 


\paragraph{Endogenous growth, elastic labor supply and optimal environmental policy}
\cite{Hemous2021DirectedEconomics}
\cite{Oueslati2002EnvironmentalSupply} studies the optimal environmental policy with elastic labor supply and endogenous growth. Yet, he allows for lump-sum transfers of environmental revenues. \textit{He should find something on reduction of hours}: No: capital is the only polluting factor, and labor is the clean factor of production.


%%%--------------------------------------------------------------------
% How to recycle environmental tax revenues: weak double dividend
%%%-------------------------------------------------------------------- 
\paragraph{Recycling environmental tax revenues}
\tr{Read:\cite{Freire-Gonzalez2018EnvironmentalReview} yet on strong dd, I assume}
Another focal point of the discussion on environmental policy is the redistribution of environmental tax revenues. In contrast to the previous strand of literature, this one generally assumes labor supply to be elastic. 
The dominant focus of this literature is the weak double dividend of environmental taxes \citep{Goulder1995EnvironmentalGuide, Bovenberg2002EnvironmentalRegulation}: given an exogenous government funding constraint it is cost saving to recycle environmental tax revenues to lower distortionary labor income taxes as opposed to higher lump-sum transfers. The latter decreases labor supply through an income effect thereby decreasing the tax base of the labor income tax. Consequently, it becomes more expensive for the government to generate revenues.
Therefore, this literature advocates recycling environmental tax revenues through a reduction in other distortionary fiscal taxes as opposed to lump-sum rebates.

This literature is close to the present paper as it combines fiscal and environmental policies.  The primary distinction is that there the use of income taxes by an exogenous funding constraint or is fixed. In contrast, I focus the paper on the role of income taxes in the optimal environmental policy and abstract from an exogenous financing condition on the government. Still, the model rationalizes a progressive income tax.
\cite{Barrage2019OptimalPolicy} also optimizes jointly over fiscal and environmental policy, but her focus rests on the deviation of the optimal environmental tax from the social costs of carbon.

The weak double-dividend literature rests on the assumption that no lump-sum transfers and taxes are available. Yet, when this is the case, this paper shows that absent a government funding constraint, distortionary taxes should be set to reduce labor supply: some reduction in hours is in fact efficient from an environmental perspective. Hence, this paper provides an upper bound on the reduction of distortionary income taxes. To the best of my knowledge, the papers theoretically discussing the weak double-dividend \citep{LansBovenberg1996OptimalAnalyses, Goulder1995EnvironmentalGuide} do not formally derive the result; a possible explanation for why the lower bound on distortionary tax reduction remained unnoticed. 

 %This becomes clear when environmental tax revenues suffice to meet the government's funding constraint, then labor supply would be inefficiently high when the labor income tax is unused. 
In contrast to the present paper, the double-dividend literature focuses on non-environmental cost advantages of environmental taxation either via interactions with other taxes and their bases or via their revenues. However, it remains unmentioned that under the assumption of elastic labor supply, which the literature necessarily assumes, the environmental tax alone is not efficient.\footnote{\ \cite{LansBovenberg1999GreenGuide} writes "\textit{Environmental taxes are  generally  an  efficient  instrument  for  protecting  the  environment.}" thereby neglecting the role environmental tax revenue redistribution. Or "\textit{Establishin a price on carbon [...] is well understood to be the most efficient approach for reducing greenhouse gas emissions.}" \citep{Fried2018TheGenerations} and } When revenues are used other than as lump-sum rebates, then a motive for other reductive means becomes desirable from an environmental perspective. 

Importantly, the equity and the environmental targets of government intervention are perceived as competing goals as both tax instruments exert efficiency costs through a reduction in labor supply.
I argue in this paper that what is perceived as an efficiency cost -  the reduction in labor supply - is part of the optimal environmental policy. Hence, income taxation has a double dividend: an environmental and an equity one.  

%----------------------------------------
%---- optimal revenue recycling 
%---- empirical and quantitative-------
%----------------------------------------
The advantage of different recycling means is often evaluated with respect to cost savings, equity, or political feasibility \citep{Carattini2018, Goulder2019IncomeGroups, VANDERPLOEG2022103966, Kotlikoff2019NBERWin, Carbone2013DeficitImpacts}. Building on the weak double-dividend literature, \cite{Fried2018TheGenerations} compare distinct recycling scenarios investigating the impact on inter- and intragenerational inequality in an overlapping generations model. Lump-sum transfers are preferred by the  living generation. 
 \cite{VANDERPLOEG2022103966} find an equity advantage of lump-sum transfers using German data. The authors suggest the government to split environmental tax revenues to both lower pre-existing tax distortions and as lump-sum transfers. The present paper employs the first-best allocation as a benchmark to assess distinct recycling methods.
In general, I draw into question that environmental tax revenues constitute a free lunch as their redistribution constitutes an essential part of the optimal environmental policy next to taxes.

%%%------------------------------------------------
% Deviation from the pigou principle: The effect of fiscal distortions on the optimal environmental policy
%%%--------------------------------------------------------------------

Another realm the literature which combines fiscal and environmental policy considers is the deviation of the environmental tax from the Pigou principle. In the classic setup, the government faces an exogenous funding condition. This motivates using labor income taxes to generate funds. 
The environmental tax reduces the wage rate - an efficient decline from an environmental perspective - depressing the tax base of the income tax if the uncompensated wage elasticity of labor is positive.  This is why the two motives of government intervention, the environmental externality and generating revenues, compete.\footnote{\ Importantly, using environmental tax revenues to fund the government is more costly as opposed to labor income taxes due to an additional distortion generated from envrionmental taxes: environmental taxes distort commodity in addition to reducing labor supply. This argument rejects the strong double dividend hypothesis and was brought forward by \cite{LansBovenberg1994EnvironmentalTaxation}. } 
In order to satisfy the funding constraint, the optimal environmental tax falls short of the social costs of the externality; the Pigou principle is violates \citep{LansBovenberg1996OptimalAnalyses}. \cite{Barrage2019OptimalPolicy} studies the role of fiscal distortions for the optimal environmental policy in a dynamic setting with climate cycle. 
The results presented in this paper connect to the considerations on whether and why the environmental tax deviates from the social cost of the externality. While it is the second target, i.e., to generate funds, which rationalizes a lower environmental tax, the reason of the deviation in my paper emerges from the non-redistribution of environmental tax revenues. This reduces consumption and thereby utility so that the government seeks to reduce environmental tax revenues. 

\paragraph{endogenous growth and distortionary taxes}
Fullerton and Kim 2008
\cite{Bovenberg1997EnvironmentalGrowth}
Hettich 1998
Lighart and van der Ploeg 1994

\paragraph{Public finance}
An equity-efficiency trade-off is central to the discussion of optimal labor income taxes in the public finance literature.  The benefits of labor taxes and progressivity arise, inter alia, from redistribution. %and from generating government revenues. 
With concave utility specifications full redistribution is efficient. However, the optimal tax system does not feature full redistribution when labor supply is endogenous. Instead, redistribution is traded off against aggregate output as individuals reduce their labor supply and skill investment in response to labor income taxation \citep{Heathcote2017OptimalFramework, Conesa2009TaxingAll, Domeij2004OnTaxes}.

To this literature I add another motive for the use of distortionary fiscal policies: to reduce inefficiently high labor supply induced by environmental policy. 
One closely related work is \cite{Loebbing2019NationalChange} who studies optimal income taxation in a model of directed technical change. The redistributive effect of tax progressivity is amplified through a compression of the wage rate distribution \textit{to be continued}

\paragraph{reduction policies and their optimality}
\begin{itemize}
	\item due to social preferences (envy, keeping up with the Joneses, habits) \ar read Layard 2006 on Happiness
	\item due to environmental limits
	\item do not include LIMITS to GROWTH literature (this seems to be a different question )
\end{itemize}
Finally, the paper relates from its motivation and finding to the discussion on whether production levels are too high. 
The finding relates to the literature discussing rationales for the usage of reductive policy measures. These arise from o
Negative externalities of consumption and hours worked such as
envy \cite{Alvarez-Cuadrado2007EnvyHours}, or a positive externality of leisure \cite{Alesina2005WorkDifferent}. The present paper relates to this literature by identifying an externality resulting from mitigating an environmental externality when tax revenues are not redistributed.
