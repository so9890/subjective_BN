\section{Results}

In this section, I, present and discuss the quantitative results.
Subsection \ref{subsec:mr} presents the optimal policy given the emission target. Subsection \ref{subsec:dis} discusses the results. In particular, I focus on understanding the role of income tax progressivity. 

\subsection{Main results}\label{subsec:mr}
\begin{figure}[h!!]
	\centering
	\caption{Optimal Policy }\label{fig:optPol}
	\begin{minipage}[]{0.4\textwidth}
		\centering{\footnotesize{(a) Income tax progressivity, $\tau_{lt}$}}
		%	\captionsetup{width=.45\linewidth}
		\includegraphics[width=1\textwidth]{../../codding_model/own_basedOnFried/optimalPol_elastS_DisuSci/figures/all_1705/Single_OPT_T_NoTaus_taul_spillover0_sep1_BN0_ineq0_etaa0.79.png}
	\end{minipage}
\begin{minipage}[]{0.1\textwidth}
\
\end{minipage}
	\begin{minipage}[]{0.4\textwidth}
		\centering{\footnotesize{(b) Fossil tax, $\tau_{ft}$ }}
		%	\captionsetup{width=.45\linewidth}
		\includegraphics[width=1\textwidth]{../../codding_model/own_basedOnFried/optimalPol_elastS_DisuSci/figures/all_1705/Single_OPT_T_NoTaus_tauf_spillover0_sep1_BN0_ineq0_etaa0.79.png}
	\end{minipage}
\end{figure} 
%\begin{figure}[h!!]
%	\centering
%	\caption{Optimal Policy }\label{fig:optPol}
%	\begin{minipage}[]{0.4\textwidth}
%		\centering{\footnotesize{(a) Income tax progressivity, $\tau_{lt}$}}
%		%	\captionsetup{width=.45\linewidth}
%		\includegraphics[width=1\textwidth]{../../codding_model/own_basedOnFried/optimalPol_elastS_DisuSci/figures/all_1705/Single_OPT_T_NoTaus_taul_spillover0_sep1_BN1_ineq0_etaa0.79.png}
%	\end{minipage}
%	\begin{minipage}[]{0.1\textwidth}
%		\
%	\end{minipage}
%	\begin{minipage}[]{0.4\textwidth}
%		\centering{\footnotesize{(b) Fossil tax, $\tau_{ft}$ }}
%		%	\captionsetup{width=.45\linewidth}
%		\includegraphics[width=1\textwidth]{../../codding_model/own_basedOnFried/optimalPol_elastS_DisuSci/figures/all_1705/Single_OPT_T_NoTaus_tauf_spillover0_sep1_BN1_ineq0_etaa0.79.png}
%	\end{minipage}
%\end{figure}

To meet the IPCCs suggested emission target, the optimal income tax is progressive for all periods between 2030 and 2080; see panel (a) in figure \ref{fig:optPol}. As the emission target is less strict, between 2030 to 2045, optimal income tax progressivity is around $\tau_{lt}=0.02$. As the emission target jumps to net-zero emissions in 2050, optimal tax progressivity accelerates to above 0.08 and gradually increases in the subsequent years to around 0.09. This is approximately half the size found for the US in \cite{Heathcote2017OptimalFramework}: $\tau_{l}=0.181$. 
In the period without emission target from 2020 to 2030, the optimal income tax is slightly regressive.

Consider panel (b). The optimal fossil tax displays a similar step pattern as the income tax progressivity. From 2020 to the beginning of 2030, it is negative. It jumps to around 50\% as the emission target is to reduce emissions by 50\% relative to 2019 emissions. As the emission target rises  to net-zero emissions in 2050, the optimal tax on fossil sales is close to 90\%. 

Figure \ref{fig:optAll} depicts the optimal allocation while meeting emission targets.
Meeting emission targets is concomitant with both a reduction and recomposition of consumption. Panel (a) shows consumption which reduces significantly as emission targets become stricter, in 2030 and 2050, but starting from the new low levels is allowed to grow. Labour effort also reduces visibly as stricter emission targets become active; panels (b) and (c). In contrast to consumption, hours worked for both types of labour decrease over time. Overall, social welfare reduces but grows after new emission targets get implemented. The rise in consumption is driven by technological progress in all sectors; compare panels (d) to (e).INCLUDE GRAPHS ON TECHNOLOGY RATIOS!

The recomposing aspect of the optimal policy is best underlined by looking at labour inputs. Panels (g) to (i) show the labour composite used in the distinct sectors. While labour input in the fossil sector reduces, it increases in the green sector. The economy recomposes its energy consumption towards green energy. The reductive aspect of the optimal policy, is highlighted by the reduction of non-energy labour input; panel (i). 
%\begin{comment}
\begin{figure}[h!!]
	\centering
	\caption{Optimal Allocation }\label{fig:optAll}
	
	
	\begin{minipage}[]{0.32\textwidth}
		\centering{\footnotesize{(d) Consumption}}
		%	\captionsetup{width=.45\linewidth}
		\includegraphics[width=1\textwidth]{../../codding_model/own_basedOnFried/optimalPol_elastS_DisuSci/figures/all_1705/Single_OPT_T_NoTaus_C_spillover0_sep1_BN0_ineq0_etaa0.79.png}
	\end{minipage}
	\begin{minipage}[]{0.32\textwidth}
		\centering{\footnotesize{(e) Low-skilled labour }}
		%	\captionsetup{width=.45\linewidth}
		\includegraphics[width=1\textwidth]{../../codding_model/own_basedOnFried/optimalPol_elastS_DisuSci/figures/all_1705/Single_OPT_T_NoTaus_hl_spillover0_sep1_BN0_ineq0_etaa0.79.png}
	\end{minipage}
	\begin{minipage}[]{0.32\textwidth}
		\centering{\footnotesize{(f) High-skilled labour}}
		%	\captionsetup{width=.45\linewidth}
		\includegraphics[width=1\textwidth]{../../codding_model/own_basedOnFried/optimalPol_elastS_DisuSci/figures/all_1705/Single_OPT_T_NoTaus_hh_spillover0_sep1_BN0_ineq0_etaa0.79.png}
	\end{minipage}
	\begin{minipage}[]{0.32\textwidth}
		\centering{\footnotesize{(a) Growth fossil sector}}
		%	\captionsetup{width=.45\linewidth}
		\includegraphics[width=1\textwidth]{../../codding_model/own_basedOnFried/optimalPol_elastS_DisuSci/figures/all_1705/Single_OPT_T_NoTaus_Af_spillover0_sep1_BN0_ineq0_etaa0.79.png}
	\end{minipage}
	\begin{minipage}[]{0.32\textwidth}
		\centering{\footnotesize{(b) Growth green sector }}
		%	\captionsetup{width=.45\linewidth}
		\includegraphics[width=1\textwidth]{../../codding_model/own_basedOnFried/optimalPol_elastS_DisuSci/figures/all_1705/Single_OPT_T_NoTaus_Ag_spillover0_sep1_BN0_ineq0_etaa0.79.png}
	\end{minipage}
\begin{minipage}[]{0.32\textwidth}
	\centering{\footnotesize{(c) Growth neutral sector}}
	%	\captionsetup{width=.45\linewidth}
	\includegraphics[width=1\textwidth]{../../codding_model/own_basedOnFried/optimalPol_elastS_DisuSci/figures/all_1705/Single_OPT_T_NoTaus_An_spillover0_sep1_BN0_ineq0_etaa0.79.png}
\end{minipage}
	\begin{minipage}[]{0.32\textwidth}
	\centering{\footnotesize{(d) Labour fossil sector}}
	%	\captionsetup{width=.45\linewidth}
	\includegraphics[width=1\textwidth]{../../codding_model/own_basedOnFried/optimalPol_elastS_DisuSci/figures/all_1705/Single_OPT_T_NoTaus_Lf_spillover0_sep1_BN0_ineq0_etaa0.79.png}
\end{minipage}
\begin{minipage}[]{0.32\textwidth}
	\centering{\footnotesize{(e) Labour green}}
	%	\captionsetup{width=.45\linewidth}
	\includegraphics[width=1\textwidth]{../../codding_model/own_basedOnFried/optimalPol_elastS_DisuSci/figures/all_1705/Single_OPT_T_NoTaus_Lg_spillover0_sep1_BN0_ineq0_etaa0.79.png}
\end{minipage}
\begin{minipage}[]{0.32\textwidth}
	\centering{\footnotesize{(f) Labour neutral}}
	%	\captionsetup{width=.45\linewidth}
	\includegraphics[width=1\textwidth]{../../codding_model/own_basedOnFried/optimalPol_elastS_DisuSci/figures/all_1705/Single_OPT_T_NoTaus_Ln_spillover0_sep1_BN0_ineq0_etaa0.79.png}
\end{minipage}
\end{figure} 

%\end{comment}


\subsection{Discussion}\label{subsec:dis}

\begin{figure}[h!!]
	\centering
	\caption{Comparison to regime without income tax }\label{fig:Compno_taul_BN0}
	\begin{minipage}[]{0.32\textwidth}
		\centering{\footnotesize{(a) Emissions}}
		%	\captionsetup{width=.45\linewidth}
		\includegraphics[width=1\textwidth]{../../codding_model/own_basedOnFried/optimalPol_elastS_DisuSci/figures/all_1705/comp_notaul_OPT_T_NoTaus_Emnet_spillover0_sep1_BN0_ineq0_etaa0.79_lgd1.png}
	\end{minipage}
	\begin{minipage}[]{0.32\textwidth}
		\centering{\footnotesize{(b) Fossil tax}}
		%	\captionsetup{width=.45\linewidth}
		\includegraphics[width=1\textwidth]{../../codding_model/own_basedOnFried/optimalPol_elastS_DisuSci/figures/all_1705/comp_notaul_OPT_T_NoTaus_tauf_spillover0_sep1_BN0_ineq0_etaa0.79.png}
	\end{minipage}
	\begin{minipage}[]{0.32\textwidth}
		\centering{\footnotesize{(c) Consumption}}
		%	\captionsetup{width=.45\linewidth}
		\includegraphics[width=1\textwidth]{../../codding_model/own_basedOnFried/optimalPol_elastS_DisuSci/figures/all_1705/comp_notaul_OPT_T_NoTaus_C_spillover0_sep1_BN0_ineq0_etaa0.79.png}
	\end{minipage}
	\begin{minipage}[]{0.32\textwidth}
		\centering{\footnotesize{\ \\(d) High skill hours worked }}
		%	\captionsetup{width=.45\linewidth}
		\includegraphics[width=1\textwidth]{../../codding_model/own_basedOnFried/optimalPol_elastS_DisuSci/figures/all_1705/comp_notaul_OPT_T_NoTaus_hh_spillover0_sep1_BN0_ineq0_etaa0.79.png}
	\end{minipage}
	\begin{minipage}[]{0.32\textwidth}
		\centering{\footnotesize{\ \\(e) Low skill hours worked}}
		%	\captionsetup{width=.45\linewidth}
		\includegraphics[width=1\textwidth]{../../codding_model/own_basedOnFried/optimalPol_elastS_DisuSci/figures/all_1705/comp_notaul_OPT_T_NoTaus_hl_spillover0_sep1_BN0_ineq0_etaa0.79.png}
	\end{minipage}
	\begin{minipage}[]{0.32\textwidth}
		\centering{\footnotesize{\ \\(f) Social welfare}}
		%	\captionsetup{width=.45\linewidth}
		\includegraphics[width=1\textwidth]{../../codding_model/own_basedOnFried/optimalPol_elastS_DisuSci/figures/all_1705/comp_notaul_OPT_T_NoTaus_SWF_spillover0_sep1_BN0_ineq0_etaa0.79.png}
	\end{minipage}
	\begin{minipage}[]{0.32\textwidth}
		\centering{\footnotesize{\ \\(g) Labour fossil}}
		%	\captionsetup{width=.45\linewidth}
		\includegraphics[width=1\textwidth]{../../codding_model/own_basedOnFried/optimalPol_elastS_DisuSci/figures/all_1705/comp_notaul_OPT_T_NoTaus_Lf_spillover0_sep1_BN0_ineq0_etaa0.79.png}
	\end{minipage}
	\begin{minipage}[]{0.32\textwidth}
		\centering{\footnotesize{\ \\(h) Labour green}}
		%	\captionsetup{width=.45\linewidth}
		\includegraphics[width=1\textwidth]{../../codding_model/own_basedOnFried/optimalPol_elastS_DisuSci/figures/all_1705/comp_notaul_OPT_T_NoTaus_Lg_spillover0_sep1_BN0_ineq0_etaa0.79.png}
	\end{minipage}
	\begin{minipage}[]{0.32\textwidth}
		\centering{\footnotesize{\ \\(i) Labour non-energy}}
		%	\captionsetup{width=.45\linewidth}
		\includegraphics[width=1\textwidth]{../../codding_model/own_basedOnFried/optimalPol_elastS_DisuSci/figures/all_1705/comp_notaul_OPT_T_NoTaus_Ln_spillover0_sep1_BN0_ineq0_etaa0.79.png}
	\end{minipage}
	\begin{minipage}[]{0.32\textwidth}
		\centering{\footnotesize{\ \\(j) Research fossil}}
		%	\captionsetup{width=.45\linewidth}
		\includegraphics[width=1\textwidth]{../../codding_model/own_basedOnFried/optimalPol_elastS_DisuSci/figures/all_1705/comp_notaul_OPT_T_NoTaus_sff_spillover0_sep1_BN0_ineq0_etaa0.79.png}
	\end{minipage}
	\begin{minipage}[]{0.32\textwidth}
		\centering{\footnotesize{\ \\(k) Green research}}
		%	\captionsetup{width=.45\linewidth}
		\includegraphics[width=1\textwidth]{../../codding_model/own_basedOnFried/optimalPol_elastS_DisuSci/figures/all_1705/comp_notaul_OPT_T_NoTaus_sg_spillover0_sep1_BN0_ineq0_etaa0.79.png}
	\end{minipage}
	\begin{minipage}[]{0.32\textwidth}
		\centering{\footnotesize{\ \\(l) Non-energy research }}
		%	\captionsetup{width=.45\linewidth}
		\includegraphics[width=1\textwidth]{../../codding_model/own_basedOnFried/optimalPol_elastS_DisuSci/figures/all_1705/comp_notaul_OPT_T_NoTaus_sn_spillover0_sep1_BN0_ineq0_etaa0.79.png}
	\end{minipage}
	
%	\begin{minipage}[]{0.32\textwidth}
%		\centering{\footnotesize{\ \\(m) Fossil output}}
%		%	\captionsetup{width=.45\linewidth}
%		\includegraphics[width=1\textwidth]{../../codding_model/own_basedOnFried/optimalPol_elastS_DisuSci/figures/all_1705/comp_notaul_OPT_T_NoTaus_F_spillover0_sep1_BN0_ineq0_etaa0.79.png}
%	\end{minipage}
%	\begin{minipage}[]{0.32\textwidth}
%		\centering{\footnotesize{\ \\(n) Green output}}
%		%	\captionsetup{width=.45\linewidth}
%		\includegraphics[width=1\textwidth]{../../codding_model/own_basedOnFried/optimalPol_elastS_DisuSci/figures/all_1705/comp_notaul_OPT_T_NoTaus_G_spillover0_sep1_BN0_ineq0_etaa0.79.png}
%	\end{minipage}
%	\begin{minipage}[]{0.32\textwidth}
%		\centering{\footnotesize{\ \\(o) Non-energy output}}
%		%	\captionsetup{width=.45\linewidth}
%		\includegraphics[width=1\textwidth]{../../codding_model/own_basedOnFried/optimalPol_elastS_DisuSci/figures/all_1705/comp_notaul_OPT_T_NoTaus_N_spillover0_sep1_BN0_ineq0_etaa0.79.png}
%	\end{minipage}
\end{figure} 

To study the role of income tax progressivity, I compare the optimal policy and allocation in the full model to a  model where no labour income tax is available. Second, I compare these figures to the allocation a social planner would choose, which I label \textit{first best}.

Figure  \ref{fig:Compno_taul_BN0} compares the allocation under the policy regime with income tax, the black solid lines, to the counterfactual regime without income tax, the orange dashed graphs. 
The first thing to note is that a fossil tax suffices to meet the emission target: As shown by panel (a), net emissions are similar under both policy regimes. To achieve this level of emissions, the optimal fossil tax is slightly higher when no income tax is available from 2030 onwards; compare panel (b).

% how labour income tax contributes to meeting the target
The advantage from relying on labour income taxes to meet the emission target stems from a higher utility from leisure. Less time spent working, especially for high-skilled workers,  outweighs lower consumption levels; compare panels (c) to (e). In fact, the rise in social welfare arises from the periods with net-zero emission target as shown by panel (f) which compares social welfare levels across policy regimes. 

Hence, households work inefficiently high hours absent an income tax. % The higher hours worked do not translate into a sufficient rise in consumption which would outweigh the disutility from labour since fossil output is constrained. 
As green energy is not a perfect substitute for fossil energy and energy and non-energy goods are complements, the cap on fossil output prevents a rise in final production which would compensate the additional hours worked. 
Panels (g) to (l) compare labour and research effort for the three different sectors. Higher hours worked result in too high labour effort and research in the green and non-energy sector while the allocation in the fossil sector, panels (g) and (j), remains unchanged to meet the emission target. In other words, the planner optimally forfeits productivity growth for a higher utility from leisure.

%Although consumption rises due to the higher work effort when there is no income tax, the gains from labour effort are diminished due to the cap on fossil energy. Since green and fossil energy are no perfect substitutes, the economy cannot profit as much from the rise in green energy. HYPOTHESIS: WITH ENERGY SOURCES BEING BETTER COMPLEMENTS, WORK EFFORTS WOULD BE MORE FRUITEFUL. The muted effect of green energy on total energy output is intensified when considering total output where input goods are complements. 
%\begin{figure}
%
%\begin{minipage}[]{0.32\textwidth}
%	\centering{\footnotesize{\ \\(p) Energy output}}
%	%	\captionsetup{width=.45\linewidth}
%	\includegraphics[width=1\textwidth]{../../codding_model/own_basedOnFried/optimalPol_elastS_DisuSci/figures/all_1705/comp_notaul_OPT_T_NoTaus_E_spillover0_sep1_BN0_ineq0_etaa0.79.png}
%\end{minipage}
%\begin{minipage}[]{0.32\textwidth}
%	\centering{\footnotesize{\ \\(q) Final output}}
%	%	\captionsetup{width=.45\linewidth}
%	\includegraphics[width=1\textwidth]{../../codding_model/own_basedOnFried/optimalPol_elastS_DisuSci/figures/all_1705/comp_notaul_OPT_T_NoTaus_Y_spillover0_sep1_BN0_ineq0_etaa0.79.png}
%\end{minipage}
%\end{figure}
 
\begin{figure}[h!!]
	\centering
	\caption{Comparison to efficient allocation }\label{fig:Compno_eff_BN0}
	\begin{minipage}[]{0.32\textwidth}
		\centering{\footnotesize{(a) High skill hours worked}}
		%	\captionsetup{width=.45\linewidth}
		\includegraphics[width=1\textwidth]{../../codding_model/own_basedOnFried/optimalPol_elastS_DisuSci/figures/all_1705/hh_CompEffOPT_T_NoTaus_spillover0_sep1_BN0_ineq0_etaa0.79_lgd1.png}
	\end{minipage}
	\begin{minipage}[]{0.32\textwidth}
		\centering{\footnotesize{(b) Low skill hours worked}}
		%	\captionsetup{width=.45\linewidth}
		\includegraphics[width=1\textwidth]{../../codding_model/own_basedOnFried/optimalPol_elastS_DisuSci/figures/all_1705/hl_CompEffOPT_T_NoTaus_spillover0_sep1_BN0_ineq0_etaa0.79_lgd0.png}
	\end{minipage}
	\begin{minipage}[]{0.32\textwidth}
		\centering{\footnotesize{(c) Consumption}}
		%	\captionsetup{width=.45\linewidth}
		\includegraphics[width=1\textwidth]{../../codding_model/own_basedOnFried/optimalPol_elastS_DisuSci/figures/all_1705/C_CompEffOPT_T_NoTaus_spillover0_sep1_BN0_ineq0_etaa0.79_lgd0.png}
	\end{minipage}
	\begin{minipage}[]{0.32\textwidth}
		\centering{\footnotesize{\ \\(d) Technology green}}
		%	\captionsetup{width=.45\linewidth}
		\includegraphics[width=1\textwidth]{../../codding_model/own_basedOnFried/optimalPol_elastS_DisuSci/figures/all_1705/Ag_CompEffOPT_T_NoTaus_spillover0_sep1_BN0_ineq0_etaa0.79_lgd0.png}
	\end{minipage}
	\begin{minipage}[]{0.32\textwidth}
		\centering{\footnotesize{\ \\(e)  Technology fossil}}
		%	\captionsetup{width=.45\linewidth}
		\includegraphics[width=1\textwidth]{../../codding_model/own_basedOnFried/optimalPol_elastS_DisuSci/figures/all_1705/Af_CompEffOPT_T_NoTaus_spillover0_sep1_BN0_ineq0_etaa0.79_lgd0.png}
	\end{minipage}
	\begin{minipage}[]{0.32\textwidth}
		\centering{\footnotesize{\ \\(f) Technology neutral}}
		%	\captionsetup{width=.45\linewidth}
		\includegraphics[width=1\textwidth]{../../codding_model/own_basedOnFried/optimalPol_elastS_DisuSci/figures/all_1705/An_CompEffOPT_T_NoTaus_spillover0_sep1_BN0_ineq0_etaa0.79_lgd0.png}
	\end{minipage}
	\begin{minipage}[]{0.32\textwidth}
		\centering{\footnotesize{\ \\(g) Scientists green}}
		%	\captionsetup{width=.45\linewidth}
		\includegraphics[width=1\textwidth]{../../codding_model/own_basedOnFried/optimalPol_elastS_DisuSci/figures/all_1705/sg_CompEffOPT_T_NoTaus_spillover0_sep1_BN0_ineq0_etaa0.79_lgd0.png}
	\end{minipage}
	\begin{minipage}[]{0.32\textwidth}
		\centering{\footnotesize{\ \\(h) Scientists fossil}}
		%	\captionsetup{width=.45\linewidth}
		\includegraphics[width=1\textwidth]{../../codding_model/own_basedOnFried/optimalPol_elastS_DisuSci/figures/all_1705/sff_CompEffOPT_T_NoTaus_spillover0_sep1_BN0_ineq0_etaa0.79_lgd0.png}
	\end{minipage}
	\begin{minipage}[]{0.32\textwidth}
		\centering{\footnotesize{\ \\(i) Scientists neutral}}
		%	\captionsetup{width=.45\linewidth}
		\includegraphics[width=1\textwidth]{../../codding_model/own_basedOnFried/optimalPol_elastS_DisuSci/figures/all_1705/sn_CompEffOPT_T_NoTaus_spillover0_sep1_BN0_ineq0_etaa0.79_lgd0.png}
	\end{minipage}
	\begin{minipage}[]{0.32\textwidth}
		\centering{\footnotesize{\ \\(g) Labour green}}
		%	\captionsetup{width=.45\linewidth}
		\includegraphics[width=1\textwidth]{../../codding_model/own_basedOnFried/optimalPol_elastS_DisuSci/figures/all_1705/Lg_CompEffOPT_T_NoTaus_spillover0_sep1_BN0_ineq0_etaa0.79_lgd0.png}
	\end{minipage}
	\begin{minipage}[]{0.32\textwidth}
		\centering{\footnotesize{\ \\(h) Labour fossil}}
		%	\captionsetup{width=.45\linewidth}
		\includegraphics[width=1\textwidth]{../../codding_model/own_basedOnFried/optimalPol_elastS_DisuSci/figures/all_1705/Lf_CompEffOPT_T_NoTaus_spillover0_sep1_BN0_ineq0_etaa0.79_lgd0.png}
	\end{minipage}
	\begin{minipage}[]{0.32\textwidth}
		\centering{\footnotesize{\ \\(i) Labour neutral}}
		%	\captionsetup{width=.45\linewidth}
		\includegraphics[width=1\textwidth]{../../codding_model/own_basedOnFried/optimalPol_elastS_DisuSci/figures/all_1705/Ln_CompEffOPT_T_NoTaus_spillover0_sep1_BN0_ineq0_etaa0.79_lgd0.png}
	\end{minipage}
\end{figure}

%\paragraph{Comparison to social planner allocation}
In figure \ref{fig:Compno_eff_BN0}, I contrast the optimal allocation under the regime with income tax, the blue dashed lines, and without income tax, the orange dotted graph, to the efficient allocation  a social planner would choose, the black solid graph. 
Without income tax, the Ramsey planner matches the low and high skill levels a social planner would choose absent an emission target in the period 2020-2030; compare panels (a) and (b). However, both skills are supplied in too high amounts in periods with emission target. The gap widens the stricter the target. 

Interestingly, with income tax, the supply of both skills is inefficiently high when there is no emission target although levels closer to the efficient allocation would be attainable at zero tax progressivity. This suggests that the dynamic nature of the economy makes it beneficial to increase hours worked today to make up for the inefficiently low supply of high-skilled labour under the net-zero emission starting from 2050. 

When net emissions have to be zero, the optimal supply of high-skilled labour falls below the efficient level while the supply of low-skill labour is inefficiently high. The reason is that the response of high-skilled labour to the income tax is more pronounced than that of low-skilled labour due to the skill premium. 

Despite the lower work effort until 2050, the efficient allocation implies higher consumption levels for all years considered, panel (c). More consumption is caused by a higher research effort and technology levels under the social planner; see panels (d) to (i) in all sectors.\footnote{\ The abrupt reduction in skill effort under the social planner could be driven by the missing continuation value in the social and Ramsey planner problem. }

In sum, the social planner meets the emission target at higher technology levels and generally lower hours worked. This allocation is not attainable for the Ramsey planner for the externality of work effort on technological growth: The lower work effort affects the wage of scientists negatively, and technological growth decreases. However, higher work effort at higher productivity would violate the emission target. 
(WITH ONE SKILL THE RAMSEY PLANNER SHOULD BE ABLE TO MEET SAME WORK ALLOCATION... OR IS THE TRADE OFF (A) HOUSEHOLDS WORK LESS \ar (B) LOWER RESEARCH INPUT \ar Compare to version without endogenous growth (say gov can choose technology level within a range)!
)

% IDEA: study how the Ramsey planner achieves emission target when having to accept efficient growth levels!
 
 %IDEQ2: Interpretation: we cannot grow more because households would work too much. WRONG as in this specification work effort is independent of technology growth. But instead: More work effort would be needed to foster more research but this would violate the emission target!
 
%\subsubsection{Role of skill heterogeneity}

\subsection{Sensitivity}
In this subsection, I discuss results under counterfactual parameter values to elicit the robustness of the main result: the preference of progressive labour taxation above higher fossil taxes. 
First, the productivity gap between sectors might be driving the results. Second, I study how results change as returns to research are increasing within sector. 