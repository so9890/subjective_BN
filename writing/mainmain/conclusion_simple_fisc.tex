\section{Conclusion}\label{sec:con}

\begin{comment}
	content...
\tr{In contrast to \cite{Acemoglu2012TheChange}, perceiving dirty output as the source of the externality is misleading in my model: As productivity in this sector rises, so does the externality. Instead, one could think of one input factor to dirty production (i.e. energy) which causes pollution, then a more productive technology can generate more output at a lower externality. \\
Substitutability as the focus of technological advancement: (a) the economy grows as more clean goods can be used to replace dirty goods; this corresponds to less emissions in the final good production. (b) or progress on dirtiness of energy use...CONSULT \cite{Acemoglu2012TheChange} who discuss distinct forms of progress. }

\end{comment}
Another central aspect of the paper is the importance of inequality for the optimal environmental policy. How does household heterogeneity in labour supply shape the optimal environmental policy? First, I hypothesise that the skill bias of the green sector makes a less progressive income tax optimal. 


I am planning to extend the project in the following directions.
First, I want to introduce dynamics and heterogeneity in the acquisition of skills to capture mechanisms through which fiscal policy affects skill accumulation and hence the output ratio in the long run. Second, as argued by \cite{Boppart2019LaborPerspectiveb}, the intensive margin of hours worked have been falling steadily over the last 130 years. They argue for the consistency of preferences which feature a slightly higher income effect than substitution effect. In the current model with log-utility of consumption, income and substitution effects offset each other. With the preferences suggested in \cite{Boppart2019LaborPerspectiveb}, high-income, high-skilled households might increase their labour supply. 

The literature advocating a reduction in consumption levels proposes a restriction of hours worked as policy instrument to lower the consumption of resources. Therefore, I am planning to look at a reduction in the maximum number of hours worked on the externality. 

Finally, endogenising growth constitutes another interesting trade-off when the impact of fiscal policy is skill specific. 
As regards growth, it seems reasonable to consider growth as a change in the substitutability of dirty and clean goods in the final consumption good. As it stands now, growth in the dirty sector results in emission growth, ceteris paribus. Growth might instead be associated with a more efficient use of dirty energy sources, so that more output can be generated at lower emissions.

Think about effects of government using revenues for other consumption. Then reducing demand will diminish demand for the final good. 
Broadly speaking, there are two channels through which distortionary labour taxation affects emissions. First, by affecting households' labour supply decision (efficiency channel) and second in a mechanical way by changing households disposable income. The latter effect cancels out when tax revenues are used by the government to consume the final output good. Allowing the government to recycle revenues in a different way than for final good consumption uncloses another instrument to reduce emissions. 

Further ideas for extensions: include behavioural aspects: a voluntary reduction in demand, and a lower disutility from working in the green sector.
\begin{comment}
\paragraph{Ways forward}
How to introduce compositional effects:
\begin{enumerate}
	\item 	Utility function: With substitution and income effect not canceling (u(c)=$\frac{c^{1-\gamma}}{1-\gamma},\ \gamma\neq 1$), the wage rate might play a role, depends on GE effects.
	\item endogenising skill supply (rep agent chooses how much skill to supply, but this he already does... / might need to introduce structure as in HSV)
	\item government revenues are not used for final good consumption. Instead,  disposed of/ used for sth useful (this could be an extension and contribute to benefits of progressivity) THINK THIS ONLY CHANGES THE LEVEL TOO!
\end{enumerate}
\paragraph{Point 1 above}
change the utility function in the code to see what happens, if $\frac{Y_d}{Y_c}$ is constant in particular 
\paragraph{Point 3 above}
\textcolor{blue}{2) Government consumption wasted}
Letting the government not consume the final output good may alter the result. 
Now, the aggregate price level is determined endogenously as the goods market does not clear by Walras' law. 

In the equilibrium equations, I drop $p_t=1$ and use goods market clearing instead\\ $Y=c+\psi (x_c+x_d)$.

Blödsinn, only changes level

content...
\end{comment}