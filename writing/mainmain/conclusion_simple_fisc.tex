\section{Conclusion}\label{sec:con}

\begin{comment}
	content...
\tr{In contrast to \cite{Acemoglu2012TheChange}, perceiving dirty output as the source of the externality is misleading in my model: As productivity in this sector rises, so does the externality. Instead, one could think of one input factor to dirty production (i.e. energy) which causes pollution, then a more productive technology can generate more output at a lower externality. \\
Substitutability as the focus of technological advancement: (a) the economy grows as more clean goods can be used to replace dirty goods; this corresponds to less emissions in the final good production. (b) or progress on dirtiness of energy use...CONSULT \cite{Acemoglu2012TheChange} who discuss distinct forms of progress. }

\end{comment}

%What I do and what I find
The IPCC defines absolute emission targets to meet climate goals, with net-zero emissions starting from 2050. This paper is the first to introduce these exogenously given emission targets into an optimal fiscal policy framework. I find that even absent inequality, the constraint to meet emission targets makes a progressive income tax an optimal complement to corrective taxes on fossil energy. 
The importance of income taxes comes from utility advantages through a reduction in hours worked. Absent income taxes, hours worked are inefficiently high since the cap on fossil energy prevents big enough a rise in consumption to compensate for the disutility from labour. 
This finding obtains even though work effort exerts a positive externality on growth through a market size effect. 

\paragraph{Extension: What if the low skilled get a higher share \ar they reduce even less \ar more fossil input supply}

% I plan to discuss results under counterfactual parameter values to elicit the robustness of the main result: the preference of progressive labour taxation above higher fossil taxes. 
%First, the productivity gap between sectors might be driving the results. Second, I will abstract from endogenous growth to learn about the labour-supply-innovation channel as a driver of the optimal policy. Finally, I plan to study how results change as returns to research are increasing within sector. 
%Due to the endogeneity of technological growth in the model, the reduction in work effort fosters less research especially in the non-energy sector.  %However, more hours worked in the Ramsey model fostering research would violate the emission target. As a result, growth in technology and in consumption is inefficiently low in order to meet the emission target. 

\begin{comment}
To shed more light on the main findings, I plan conduct several additional quantitative experiments. First, I want to reduce the size of the emission target, second, I allow for a longer time frame until net-zero emissions have to be reached. The IPCC report states that for a temperature target of 2°C net-zero emissions have to be reached by 2070 only. How does this laxer target affect the importance of labour income taxes. Given the wider time frame, the green sector might be able to catch up and growth could continue. Finally, how does a change in spillovers shape the result? % \textit{(Question: I guess that substitutability is key here! Growth in green implies growths in fossil when goods are no perfect substitutes! )}
content...

%Another central aspect of the paper is the importance of inequality for the optimal environmental policy. How does household heterogeneity in labour supply shape the optimal environmental policy? First, I hypothesise that the skill bias of the green sector makes a less progressive income tax optimal. 
One main result of the paper is reduction of consumption and work effort as an optimal policy. So far, I have assumed that households are passive and preferences are fixed; there is no trade-off between environmental quality and consumption from a household perspective.
In an extension to the baseline model, I plan to depart from the representative agent assumption and explicitly model household heterogeneity. This setting allows to capture a change in household behaviour: A share of households is willing to voluntarily reduce consumption. I provide evidence for such behaviour using a representative Dutch dataset. More than 50\% of households are willing to reduce consumption in order to help the economy. Importantly, these households have a higher likelihood to work in the green sector. How does such a change in behaviour affect the optimal policy? Given the additional reduction in green-specific labour supply, the planner might find it optimal to set a more regressive tax to booster green production and research.    

\end{comment}

%However, data suggests, that households do care, and they express a willingness to reduce consumption.\footnote{\ The data I have studied comes from the Liss Panel, a representative sample of Dutch households, more than 50\% of participants indicate a readiness to change their behaviour to help the environment.} I want to study the effect of such behavioural  change on the optimal policy. Interestingly, households in high-skilled jobs are more likely to declare their willingness to reduce. This linkage may intensify the trade-off between reduction and green labour supply. 


%1) BN and inequality
%2) preferences for labour
\begin{comment}
Preferences and the trade-off between leisure and consumption determining household behaviour seem to be key to the results. As argued by \cite{Boppart2019labourPerspectiveb}, the intensive margin of hours worked have been falling steadily over the last 130 years. They argue for the consistency of preferences which feature a slightly higher income effect than substitution effect. In the current model with log-utility and representative family framework,  the substitution effects offset each other. With the preferences suggested in \cite{Boppart2019labourPerspectiveb}, growth would affect hours worked, assumably changing the optimal policy. It could, for instance, be the case, that growth has to be slowed down even more, to prevent too high work efforts and consumption levels. % high-income, high-skilled households might increase their labour supply with growth. 

content...
\end{comment}


In an extension, I am planning to give the Ramsey planner the opportunity to reduce hours worked through a cap on the maximum hours which can be supplied. The literature advocating a reduction in consumption levels \cite{Schor2005SustainableReduction} proposes a restriction of hours worked as policy instrument to lower the consumption of resources.

%Finally, endogenising growth constitutes another interesting trade-off when the impact of fiscal policy is skill specific. 
%As regards growth, it seems reasonable to consider growth as a change in the substitutability of dirty and clean goods in the final consumption good. As it stands now, growth in the dirty sector results in emission growth, ceteris paribus. Growth might instead be associated with a more efficient use of dirty energy sources, so that more output can be generated at lower emissions.
%
%Think about effects of government using revenues for other consumption. Then reducing demand will diminish demand for the final good. 
%Broadly speaking, there are two channels through which distortionary labour taxation affects emissions. First, by affecting households' labour supply decision (efficiency channel) and second in a mechanical way by changing households disposable income. The latter effect cancels out when tax revenues are used by the government to consume the final output good. Allowing the government to recycle revenues in a different way than for final good consumption uncloses another instrument to reduce emissions. 

%Further ideas for extensions: include behavioural aspects: a voluntary reduction in demand, and a lower disutility from working in the green sector.
\begin{comment}
\paragraph{Ways forward}
How to introduce compositional effects:
\begin{enumerate}
	\item 	Utility function: With substitution and income effect not canceling (u(c)=$\frac{c^{1-\gamma}}{1-\gamma},\ \gamma\neq 1$), the wage rate might play a role, depends on GE effects.
	\item endogenising skill supply (rep agent chooses how much skill to supply, but this he already does... / might need to introduce structure as in HSV)
	\item government revenues are not used for final good consumption. Instead,  disposed of/ used for sth useful (this could be an extension and contribute to benefits of progressivity) THINK THIS ONLY CHANGES THE LEVEL TOO!
\end{enumerate}
\paragraph{Point 1 above}
change the utility function in the code to see what happens, if $\frac{Y_d}{Y_c}$ is constant in particular 
\paragraph{Point 3 above}
\textcolor{blue}{2) Government consumption wasted}
Letting the government not consume the final output good may alter the result. 
Now, the aggregate price level is determined endogenously as the goods market does not clear by Walras' law. 

In the equilibrium equations, I drop $p_t=1$ and use goods market clearing instead\\ $Y=c+\psi (x_c+x_d)$.

Blödsinn, only changes level

content...
\end{comment}