\section{Theoretic Results}
\subsection{An emission target calls for a reduction policy under likely parameter values}
\subsection{Tax progressivity affects the composition of total output}
In the model, tax progressivity affects the innovation decision due to heterogeneous effects on skill supply. 
The optimal ratio of skills supplied by the household is
\begin{align}
\frac{h_{ht}}{h_{lt}}=\left(\frac{w_{ht}}{w_{lt}}\right)^\frac{1-\tau_{lt}}{\tau_{lt}+\sigma}.
\end{align}
The semi-elasticity of the ratio of aggregate skill supply, defined as $\frac{H_h}{H_l}:=\frac{z_hh_h}{z_lh_l}$, in response to a change in tax progressivity is then given by
\begin{align}
\frac{d\log\left(\frac{H_h}{H_l}\right)}{d\tau_l}=-\frac{1+\sigma}{(\tau_l+\sigma)^2}\log\left(\frac{w_{h}}{w_l}\right). \end{align}
The direct effect, with fixed prices is negative. 
The term negative given a positive wage premium for high skill labour. Hence, a higher tax progressivity implies a decline in the relative supply of high skill labour. 

\paragraph{Effect on the externality }

\subsection{Growth in the dirty sector has to stop}
otherwise, price increases to infinity and labour input falls.