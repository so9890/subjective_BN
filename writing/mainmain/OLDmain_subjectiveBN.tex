\documentclass[12pt]{article}
\usepackage[utf8]{inputenc}
\usepackage{xcolor}
\usepackage{graphicx}
\usepackage{epstopdf}
\usepackage{pdflscape} % landsacpe package
% set font to times
%\usepackage{mathptmx} % times!!! 
%\usepackage[T1]{fontenc}
\usepackage{amsmath}
\usepackage{soul}
\usepackage[left=2.5cm, right=2.5cm, top=2.5cm, bottom =2.5cm]{geometry}
\usepackage{natbib}
\bibliographystyle{apa}
%\usepackage{natbib}
%\renewcommand{\footnotesize}{\fontsize{10pt}{11pt}\selectfont}
\usepackage[onehalfspacing]{setspace}
\usepackage{listings}
\renewcommand{\figurename}{\textbf{Figure}}
\renewcommand{\hat}{\widehat}
\usepackage[bf]{caption}
\usepackage{tikz}
\usepackage[headsepline,footsepline]{scrlayer-scrpage} % has to come before package!!! otherwise option clash
\usepackage{scrlayer-scrpage}
\pagestyle{scrheadings} % kopfzeile/ fußzeile
\clearpairofpagestyles
\ohead{April 2021\\ Sonja Dobkowitz}
\ihead{ Subjective Basic Needs, Sustainability and Growth }
\cfoot{\thepage}
%\pagestyle{plain}
\usepackage{comment}
 \usepackage{siunitx}
  \usepackage{textcomp}
\definecolor{sonja}{cmyk}{0.9,0,0.3,0}
%\definecolor{purple}{model}{color-spec}
\usepackage{amssymb}
\newcommand{\ar}{$\Rightarrow$ \ }
\newcommand{\frp}[2]{\frac{\partial{#1}}{\partial{#2}}}
\newcommand{\tr}[1]{\textcolor{red}{#1}}
\newcommand{\vlt}[1]{\textcolor{violet}{#1}}

\newcommand{\sn}[1]{\textcolor{sonja}{#1}}
%%% TIKZS
\usepackage{tikz}
\usetikzlibrary{tikzmark}
\usetikzlibrary{decorations.markings}
\usepackage{tikz-cd}
\usetikzlibrary{arrows,calc,fit}
\tikzset{mainbox/.style={draw=sonja, text=black, fill=white, ellipse, rounded corners, thick, node distance=5em, text width=8em, text centered, minimum height=3.5em}}
\tikzset{mainboxbig/.style={draw=sonja, text=black, fill=white, ellipse, rounded corners, thick, node distance=5em, text width=13em, text centered, minimum height=3.5em}}
\tikzset{dummybox/.style={draw=none, text=black , rectangle, rounded corners, thick, node distance=4em, text width=20em, text centered, minimum height=3.5em}}
\tikzset{box/.style={draw , rectangle, rounded corners, thick, node distance=7em, text width=8em, text centered, minimum height=3.5em}}
\tikzset{container/.style={draw, rectangle, dashed, inner sep=2em}}
\tikzset{line/.style={draw, very thick, -latex'}}
\tikzset{    pil/.style={
		->,
		thick,
		shorten <=2pt,
		shorten >=2pt,}}
	
% other stuff
\newcommand{\innermid}{\nonscript\;\delimsize\vert\nonscript\;}
\newcommand{\activatebar}{%
	\begingroup\lccode`\~=`\|
	\lowercase{\endgroup\let~}\innermid 
	\mathcode`|=\string"8000
}
%\usepackage{biblatex}
%\addbibresource{bib_mt.bib}
\usepackage{ulem}
\title{Proposal \\ Subjective Basic Needs, Sustainability and Growth}
\date{}
\usepackage{graphicx,caption}
\usepackage{hyperref}
\usetikzlibrary{shapes.geometric}
\begin{document}
	\maketitle
	\tableofcontents
\section{potentially helpful}
Survey on values and SDG \url{https://www.globalsurvey-sdgs.com/#social-media}
\section{Motivation}

\paragraph{Climate change and a reduction of consumption}
Climate change is one of, if not the central problem threatening humanity today. 
What is needed is a reduction in resource usage. Two competing and potentially complementing views about how to achieve such a decoupling exist. 
Proponents of a \textit{recomposition} perspective argue that a shift towards green production alone is sufficient to fight climate change. 
The recomposition approach has been dominant in (macro)economics research. However, there is uncertainty about whether this approach alone is sufficient to fight climate change.
OTHERS argue  that a \textit{reduction} in consumption levels is inevitable \citep[compare][]{Gough2015ClimateNeeds}. 
There is evidence that a full transition to green production is not possible given today's high levels of consumption. 
For example, the less extensive use of land in organic agriculture does not allow to meet today's consumption levels with organic produce alone. 
Therefore,  this paper focuses on a reduction in consumption as a measure to fight climate change and its interactions with a recomposition approach.\footnote{We find it necessary to study both pillars of climate measures jointly. Abstracting from recomposition would well overestimate the required reduction in demand THIS IS NOT OUR FOCUS PROVIDE BETTER ARGUMENT }.

\paragraph{Introduction Basic needs}
 More precisely, the central object of this study are subjective basic needs. The term refers to what an individual subjectively believes she needs as a minimum consumption level. This measure comprises objective basic needs which a human being needs to live a humane live. 
 For example, objective basic needs are such that biological needs are satisfied, that shelter is secured, and that the individual can participate in society. 
Subjective basic needs in today's consumption societies most likely reach beyond that level. They are formed and shaped by the individual's susceptibility to advertisement, to common consumption levels observed in society, and habits, to name a few. PROVIDE REFERENCES. 

\paragraph{Consequences of subjective basic needs for reduction and recomposition}
The most direct impact of subjective basic needs on resource usage follows from high aggregate consumption levels causing high levels of resource usage absent a big enough recomposition of production. \footnote{Assuming a decoupling of resource usage and consumption is impossible}. (INTERACTION WITH RECOMPOSITION VIEW) Less obvious, a high level of subjective basic needs can impede individuals to switch to green consumption and hence a recomposition towards green production. The mechanism is the following. Studies have provided evidence for the presence of individual social responsibility, i.e. the willingness to pay a price premium to avoid negative externalities, as a determinant of demand \citep[compare][]{Bartling2015DoResponsibility}. Yet, the willingness to pay for green goods most likely is negatively related to subjective basic needs: when income is not high enough to allow for green goods to cover subjective basic needs, the individual might prefer to keep consumption of non-green alternatives high to satisfy what is subjectively perceived a need. 
  
  
In light of climate change it is crucial to learn about potential policy measures to lower subjective basic needs (BEHAVIOURAL/EXPERIMENTAL), the effectiveness of such measures on an aggregate level(MODEL), and political considerations that could prevent governmental action in this direction (POLITICAL ECONOMY: TRADE OFF LABOUR VS ENVIRONMENT).  

Therefore, the paper is separated into three parts. The first one presents a field study to analyse measures to lower subjective basic needs. The findings are then modelled into a \textbf{disequilibrium} structural model to learn about their aggregate effectiveness and the interaction of recomposition and reduction of production. Particular emphasis is put on the effect on labour demand.
The final part studies the political economy of subjective basic needs. Policies which lower consumption potentially\footnote{ Could be that the recomposition which is possible under lower consumption, in fact, keeps labour demand high} have  negative effects on labour demand and output. Both labour and GDP are commonly used to measure policymakers' success. Hence, even if a reduction of subjective basic needs is optimal from a Ramsey planner's point of view there is good reason to question the political attractiveness of such policies.


\section{Basic needs}
\paragraph{Households} A generic household maximises lifetime utility according to:
\begin{align}
\underset{c_{s},c_{n}, l}{\max} \ \hspace{2mm} U(c_s, c_n, l; h_n)= \underset{c_{s},c_{n}, l}{\max} \ \hspace{2mm} \log(c(c_s,c_n))-\chi\frac{l^{1+\frac{1}{\theta}}}{1+\frac{1}{\theta}}  -penalty(c_s, c_n),
\\
s.t. \  p_{s}c_{s}+c_{n}\leq w(1-\tau_l)zl+T,\\
l\leq L \\
c =
\left(\omega_s^{\frac{1}{\sigma}}c_{s}^{\frac{\sigma-1}{\sigma}}+(1-\omega_s)^{\frac{1}{\sigma}}c_{n}^{\frac{\sigma-1}{\sigma}}\right)^{\frac{\sigma}{\sigma-1}}& \hspace{2mm} \text{with} \hspace*{2mm} \sigma \neq 1
\end{align}
Where the penalty function is given, for example, by
\begin{align*}
penalty(\hat{c_t}-\bar{c})=\frac{1}{\phi}\exp(-\phi(c_{nt}+c_{st}-\bar{c}))\\
\end{align*}
\textbf{Notes}: \textit{subjective basic needs} refer to a lower bound on what an indivdiual feels he needs. 
On the other hand, one could think of \textit{satiation points} as an upper level of consumption. 
Even with subjective basic needs being reduced, we would observe a rise in demand as prices fall. Instead, consumption levels have an upper bound with satiation points. 
\\
Even with satiation points: could still have that the reduction in satiation points allows the quality of a good to be more important for demand, that is: social responsibility plays a more important role. 

\section{Behavioural Economics}
This section is dedicated to sketch opportunities to alter subjective basic needs or consumption levels in general: 

\begin{itemize}
	\item goals on resource usage \ar information on resource usage when consuming a given product
	\item behavioural bias would play a role \ar discrepency between one consumption and environmental consequences
	\item optimal targeting could also play a role here: depending on the observed level of consumption 
\end{itemize}

Schedule of project
\begin{itemize}
	\item \textbf{Survey}:
	\\ Part I on consumption behaviour
	\begin{itemize}
	\item  Why do you consume as much as you do?
	\\ Potential answers I could imagine
	\begin{itemize}
		\item to catch up with others
		\item bcs I need to \ar subjective bn
		\item bounded by income (for low levels)
	\end{itemize}
	\item How satisfied are you with your consumption level?
	\item How would you like to change it, if possible?
\end{itemize}
Part II on economic theory and mechanisms
\begin{itemize}
	\item How would your judge the impact of consumption on climate change in general?
	\item How do you assess the impact of your consumption on climate change/ resource usage?
\end{itemize}
Part III on political interventions
\begin{itemize}
	\item how would you assess a political intervention to lower demand: information on resource usage on consumer packaged goods
\end{itemize}
\item \textbf{Experiment}
\begin{itemize}
	\item elicit information on beliefs on how consumption and climate relate
	\item provide information on economic mechanisms
	\item info on resource usage 
\end{itemize}
\end{itemize}

Key words in behavioural
\begin{itemize}
	\item conservations
	\item rebound effect
	\item goals
\end{itemize}

\clearpage
\bibliography{../../bib_2_0}
\end{document} 