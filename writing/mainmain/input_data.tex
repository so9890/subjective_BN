\section{Empirical evidence/ work}
\tr{@Pavel: This part can be skipped, model part more important for now. For clarification on this part: The first two subsections are the ones I want to focus on in the data part. The third one seems to have been studied already. I have done a quick search for possible data sources... need to write it up and make up my mind what data is best}


\subsection{Consumption cross-section }
Measuring resource/emission consumption over the income distribution.\\
MY CONTRIBUTION: allow for quality of good to matter for emission (I think this is missing in the literature...the data there only depends on product types)

\subsection{Consumption time series}
Is there evidence for households voluntarily reducing their consumption? If yes, what determines such a reduction?\\
 In a psychological experiment, \cite{Brown2005AreLifestyle} find that environmentally-friendly behaviour and well-being are compatible (THEY DO NOT STUDY CONSUMPTION BEHAVIOUR OVER TIME). 
(\ar for more literature consult \cite{Heikkinen2015DegrowthConsumers} (Macro model with upper bound, refers to empirical literature on that)
)



\textbf{Together with the previous subsection: Is the reduction in resource consumption, if there is any, driven by the amounts consumed or by a recomposition of consumption?
}
\subsection{Labour market: skills and green sector}
TO READ: \cite{Bowen2018CharacterisingComposition} jobs that would benefit from a green transition; \\
\cite{Consoli2016DoCapital} do green and non-green jobs differ wrt skills? (US): ``\textit{green jobs use more intensively high-level cognitive and interpersonal skills compared to non-green jobs. Green occupations also exhibit higher levels of standard dimensions of human capital such as formal eductaion, work experience and on-the-job training.}''

\subsection{Possible EXTENSION I: Durables versus non-Durable}
\textbf{Which one accounts for more resource usage?\ar if the answer is durables then would make more sense to look at these}
\\
Durables: early replacement as a cause for overconsumption due to social drivers \citep{Hou2020FeelingsIntentions}
\\ 
Including durables in model to look at policies which target usage of durables.
\subsection{Possible EXTENSION II: Green investment decision}
What mixture of green to non-green bonds do private investors hold? Include decision in which sector to invest into the model.