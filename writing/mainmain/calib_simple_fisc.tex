\subsection{Calibration}\label{subsec:calib}
I calibrate the model to the US in the baseline period from 2015 to 2019 procreeding in three steps. First, I set certain parameters to values used in the literature. Second, I calibrate the remaining variables requiring that equilibrium conditions hold. Third, I choose parameters relating production, emissions and the emission limit. Tables \ref{tab:calib} and \ref{tab:calib2} summarise the calibrated parameter values.

In the first step, I mainly rely on \cite{Fried2018ClimateAnalysis} to calibrate the parameters governing research processes, $\eta, \rho_f,\rho_n, \rho_g, \phi $, and production, $\varepsilon_e, \varepsilon_y, \alpha_f, \alpha_g, \alpha_n$.  Important for the present project: the labour share in the green sector is remarkably low with $\alpha_g=0.91$. This diminishes the significance of skill supply for green innovation and production. Furthermore, fossil and green energy are no close substitutes with $\varepsilon_e=1.5$ so that the cap on fossil energy cannot be fully substituted for by green energy. The utility parameters, $\beta, \sigma$ are set to $0.984^5$ and $0.75$ following \cite{Barrage2019OptimalPolicy} and \cite{Chetty2011AreMargins}, respectively. The business as usual (BAU) policy is set to $\tau_l=0.181, \tau_f=0$, where I borrow the tax progressivity parameter from \cite{Heathcote2017OptimalFramework}. 

In the second step, I calibrate the weight on energy in final good production by matching the average expenditure share on energy relative to GDP over the period from 2015 to 2019 taken from the U.S. Energy Information Administration (EIA). The expenditure share equals 6\%.\footnote{\ The data on energy was retrieved from \url{https://www.eia.gov/totalenergy/data/monthly/}
 on April 06, 2022.} The resulting weight on energy is $\delta_y=0.45$\footnote{\ Note that in difference to \cite{Fried2018ClimateAnalysis} I raise the weight on intermediate inputs in final production to the power $\frac{1}{\varepsilon_y}$, so that in the limit the function approaches the Leontief specification as $\varepsilon_y\rightarrow 0$ \citep{Herrendorf2014GrowthTransformation}.}. The data to match the high-skill share in labour production are taken from table 3 in \cite{Consoli2016DoCapital}. In particular, I derive the share of high-skill labour in the green sector and for the non-green sector. I also assume that within the non-green sector, i.e. sectors N and F in the model, high-skilled labour shares are the same, $\theta_f=\theta_n$.  These three conditions determine $\theta_g=0.57$ and $\theta_f=\theta_n=0.42$. The share of high-skilled workers, $z_h$, is chosen to match a skill premium for the period 2005-2016 of $\frac{w_h}{w_l}=1.9$ following \cite{Slavik2020WagePremium}. The disutility of labour $\chi$ is set to match equilibrium average hours worked to average hours over the period from 2015-2019, 0.34, drawing from OECD data on annual average hours per worker worked. I normalise total economic time endowment per day, which I set to 14.5 as found in \cite{Jones1993OptimalGrowth}, to 1. The resulting disutility of labour in the model equals $\chi=10.02$. 

 To find the remaining variables which determine research, these are  research productivity, $\gamma$, and the disutility from research, $\chi_s$, I match an annual growth rate in the non-energy sector of 2\% following \cite{Fried2018ClimateAnalysis} and set total hours supplied by scientists to 0.34 similar to workers. As a result,  $\chi_s=0.032$ and $\gamma=0.042$. Initial productivity levels follow from normalising output in the base period to $Y=1$ and matching the ratio of fossil to green energy utilisation over the years 2014-2019 which equals 7.33 according to the EIA. I find that $A_{n0}=4.3$, $A_{f0}=3350$, and $A_{g0}=95.4$ which refer to the period 2010-2014. The enormous difference in energy productivity levels follows from the low labour share in the green sector, $\alpha_g=0.09$ compared to $\alpha_f= 0.28$. 
 \thispagestyle{empty}
 \begin{table}[h!]
 	\begin{center}
 		\captionsetup{width=0.9\textwidth}
 		\caption{ Calibration baseline model: Households, Research and Production}
 		\label{tab:calib}
 		\begin{tabular}{c|lll}
 			%			\hline \hline
 			%			\multicolumn{7}{c}{Calibration based on basic needs}\\
 			\hline \hline
 			Parameter& Target/Source& \makecell[l]{Calibration}& \makecell[l]{Meaning}\\ 
 			\hline
 			\hline
 			Household&\multicolumn{3}{c}{}\\
 			\hline 
 			
 			\hline
 			$\sigma$ &  \makecell[l]{\cite{Chetty2011AreMargins}}& $4/3$ & inverse Frisch elasticity  \\
 			\hline
 			$z_h$& \makecell[l]{skill premium 2005-2016:\\ $w_h/w_l=1.9$\\ \citep{Slavik2020WagePremium}}&0.2121&\makecell[l]{share of\\ high-skilled workers} \\	
 			\hline			
 			$\chi$ &  \makecell[l]{average hours worked per\\ economic time endowment\\ by worker: 0.34 \cite{OECDHoursworked}}& 10.021 & inverse Frisch elasticity  \\
 			\hline
 			$\beta$ &  \makecell[l]{\cite{Barrage2019OptimalPolicy}}& 0.9272 & 5 year discount factor  \\
 			\hline
 			$\bar{H}$& \makecell[l]{14.5 hours per day\\ \cite{Jones1993OptimalGrowth}}&5&\makecell[l]{economic time endowment \\per 5 years, normalised} \\
 			\hline
 			\hline
 			Research&\multicolumn{3}{c}{}\\
 			\hline
 			
 			\hline
 			$\sigma_s$ &  \makecell[l]{\cite{Chetty2011AreMargins}}& $4/3$ & inverse Frisch elasticity  \\
 			\hline
 			$\chi_s$ &\makecell[l]{average hours worked per \\ economic time endowment\\ by worker: 0.34 \cite{OECDHoursworked}} & 0.032 & disutility from science\\
 			\hline
 			$\eta$ &\makecell[l]{\cite{Fried2018ClimateAnalysis}} & 0.79 & spillover research\\
 			\hline			
 			$\rho_f$ &\makecell[l]{\cite{Fried2018ClimateAnalysis}} & 0.01 &\makecell[l]{research tasks in\\ fossil sector}\\
 			\hline			
 			$\rho_g$ &\makecell[l]{\cite{Fried2018ClimateAnalysis}} & 0.01 &\makecell[l]{research tasks in\\ green sector}\\
 			\hline			
 			$\rho_n$ &\makecell[l]{\cite{Fried2018ClimateAnalysis}} & 1 &\makecell[l]{research tasks in \\non-energy sector}\\
 			\hline			
 			$\phi$ &\makecell[l]{\cite{Fried2018ClimateAnalysis}} & 0.5 &across-sector research spillovers\\
 			\hline
 			$\gamma$ &\makecell[l]{growth in non-energy sector:\\2\% per annum \cite{Fried2018ClimateAnalysis}} & 0.0042 & productivity of research\\
 			\hline
 			\hline
 			Production&\multicolumn{3}{c}{}\\
 			\hline
 			
 			\hline
 			$\varepsilon_y$&\cite{Fried2018ClimateAnalysis}&0.05& \makecell[l]{substitutability \\ energy and non-energy}\\			
 			\hline
 			$\delta_y$&\makecell[l]{expenditure share \\ on energy IEA}&0.4496& \makecell[l]{weight on energy in\\ final good production}\\	
 			\hline
 			$\varepsilon_e$&\cite{Fried2018ClimateAnalysis}&1.5& \makecell[l]{substitutability \\ green and fossil energy}\\	
 			\hline
 			$\alpha_f$&\cite{Fried2018ClimateAnalysis} &0.72& capital share fossil  \\
 			\hline
 			$\alpha_g$&\cite{Fried2018ClimateAnalysis} &0.91& capital share green \\
 			\hline
 			$\alpha_n$&\cite{Fried2018ClimateAnalysis} &0.36& capital share non-energy  \\
 			%\hline
 			%$\beta$&\makecell{ annual nominal rate 3\%\\ and annual inflation rate of 2\%}& 0.9903& discount factor\\ 
 			\hline
 			\hline
 			Initial Productivity&\multicolumn{3}{c}{}\\
 			\hline
 			
 			\hline
 			$A_{f0}$&- &3350.5& \makecell[l]{initial productivity \\ fossil sector, 2014-2019}  \\
 			\hline
 			$A_{g0}$&- &95.4& \makecell[l]{initial productivity \\ green sector, 2014-2019}  \\
 			\hline
 			$A_{n0}$&- &4.3& \makecell[l]{initial productivity \\ non-energy sector, 2014-2019}  \\
 			\hline \hline
 		\end{tabular}
 	\end{center}
 \end{table}
 \clearpage
 
 %According to the IEA, global greenhouse gas emissions from fuel combustion amounted to 34.2 Gt in CO2 equivalents in 2019.\footnote{\ Retrieved from \url{https://www.iea.org/reports/global-energy-review-2021/co2-emissions} on February 2, 2022.} I use the share the US contributed to global emissions in 2019, 19.18\%, to proxy the share in reductions I require the US to contribute to total reductions from 2019 to 2030. 
 The most recent IPCC report \citep{IPCC2022} formulates a reduction of emissions in the 2030s by 50\% relative to 2019\footnote{\ p. 5 Chapter 3: "\textit{Mitigation pathways limiting warming to 1.5°C [...] reach 50\% reductions of CO2 in the 2030s, relative to 2019, then reduce emissions further to reach net zero CO2 emissions in the 2050s [...] (\textnormal{medium confidence}).}"}
  as essential to meeting 1.5°C climate targets.  In 2019, US greenhouse-gas emissions amounted to 6,558 million metric tons (6.558Gt) of carbon dioxide equivalents and to net-emissions of 5.7691Gt. Information on emissions comes from the United States Environmental Protection Agency (EPA).\footnote{\ Retrieved on February 2, 2022 from  \url{https://www.epa.gov/newsreleases/latest-inventory-us-greenhouse-gas-emissions-and-sinks-shows-long-term-reductions-0}. }
   Expressed in model periods, which correspond to five years, the net emission limit in the 2030s amounts to: 14.4228Gt.  % This decline in emissions amounts to a gros emission target for the US of  $3.2792$Gt.
 I assume here that each country contributes to the global reduction by the same percentage of 50\% of its own emissions.\footnote{\ Alternatively, one could assume that the global reduction is allocated in the same share as countries contributed to global emissions in 2019. This would result in an even stricter target for the US which contributed almost 20\% to global greenhouse-gas emissions in 2019 (based on own calculations where total emissions come from the EIA global greenhouse-gas information, to be found here \url{https://www.iea.org/reports/global-energy-review-2021/co2-emissions}).}
 Starting from 2050, the net-emission target is zero. 
 
 I calibrate the sink capacity to match the average difference between gross and net emissions over the baseline period from 2015 to 2019. The resulting sink capacity per model period is  $\delta=3.9633$Gt, which I assume to be constant, as discussed above. 
 In summary, I calibrate the net-emission target vector for the period from 2030 to 2080 as
 $\Omega_{2030-2050}$= 2.4899Gt and $\Omega_{2050-2080}$= 0Gt. The parameter relating emissions and the use of fossil energy in the base period equals $\omega=126.0602$.\footnote{\ Note that I perceive fossil energy as the source of all greenhouse-gas emissions.}  
 \clearpage

 \begin{table}[hh!!!!!]
 	\begin{center}
 		\captionsetup{width=0.9\textwidth}
 		\caption{ Calibration baseline model: labour, Government, and Emissions}
 		\label{tab:calib2}
 		\begin{tabular}{c|lll}
 			%			\hline \hline
 			%			\multicolumn{7}{c}{Calibration based on basic needs}\\
 			\hline \hline
 			Parameter& Target/Source& \makecell[l]{Calibration}& \makecell[l]{Meaning}\\ 
 			\hline
 			\hline
 			labour Production&\multicolumn{3}{c}{}\\
 			\hline 
 			
 			\hline
 			$\theta_f$&\makecell[l]{share of high skill\\ non-green occupations: \\27.55\% }&0.4194&\makecell[l]{income share high \\ skill fossil sector}\\
 			\hline
 			$\theta_g$&\makecell[l]{share of high skill\\ green occupations: \\40.71\% }&0.5661&\makecell[l]{indome share high \\skill green sector}\\
 			\hline
 			$\theta_n$&\makecell[l]{share of high skill\\ non-green occupations: \\27.55\% }&0.4194&\makecell[l]{income share high \\ skillnon-energy sector}\\
 			\hline
 			\hline
 			Government&\multicolumn{3}{c}{}\\
 			\hline
 			
 			\hline
 			$\tau_f$&- &0& sales tax on fossil energy\\
 			\hline
 			$\tau_l$&\cite{Heathcote2017OptimalFramework} &0.181& income tax progressivity\\
 			\hline	
 			\hline
 			Emissions&\multicolumn{3}{c}{}\\
 			\hline
 			
 			\hline
 			$\delta$& \makecell[l]{EPA}&0.7893&carbon sinks \\
 			\hline
 			$\omega$& EPA&45.5634& \makecell[l]{ gross emissions as a\\ fraction of fossil output}\\
 			\hline
 			$\Omega$& \cite{IPCC2022}&\makecell[l]{from 2030-2050: 2.4899Gt\\2050-2080: 0Gt}& \makecell[l]{vector of \\ net emission limits}\\
 			\hline \hline
 		\end{tabular}
 	\end{center}
 \end{table}
%\thispagestyle{plain}
% \clearpage
%
%\paragraph{Sources data}
%%\url{https://www.eia.gov/totalenergy/data/monthly/#prices}
%
%Total energy data: 
%For data on skill and premium see references in 
%paper saved in data \citep{Slavik2020WagePremium}
%
%The model is calibrated to parameter values common in the literature. I bestow more care on  calibrating the emission target. 
%I match emissions in the model to emission targets suggested in the IPCC report \citep{Rogelj2018MitigationDevelopment.}. 
%%How to determine the economy in 2050? Should the economy have reached a steady state? or should it be in a transitional path? Maybe no need to specify this...it will be a outcome. All I have to use is that for all years after 2050 net-emissions have to be zero. Whether the economy is on the transitional path or in a steady state is an outcome. 
%The IPCC prescribes net-zero emissions starting from 2050. In 2030 emissions should be between 25 and 30 GtCO2e per year.
%
\thispagestyle{empty}
	\begin{table}[h!]
		\begin{center}
			\captionsetup{width=0.9\textwidth}
			\caption{ Calibration baseline model: Households, Research and Production}
			\label{tab:calib}
			\begin{tabular}{c|lll}
				%			\hline \hline
				%			\multicolumn{7}{c}{Calibration based on basic needs}\\
				\hline \hline
				Parameter& Target/Source& \makecell[l]{Calibration}& \makecell[l]{Meaning}\\ 
				\hline
				\hline
				Household&\multicolumn{3}{c}{}\\
				\hline 
				
				\hline
				$\sigma$ &  \makecell[l]{\cite{Chetty2011AreMargins}}& $4/3$ & inverse Frisch elasticity  \\
				\hline
				$z_h$& \makecell[l]{skill premium 2005-2016:\\ $w_h/w_l=1.9$\\ \citep{Slavik2020WagePremium}}&0.2121&\makecell[l]{share of\\ high-skilled workers} \\	
				\hline			
				$\chi$ &  \makecell[l]{average hours worked per\\ economic time endowment\\ by worker: 0.34 \cite{OECDHoursworked}}& 10.021 & inverse Frisch elasticity  \\
				\hline
				$\beta$ &  \makecell[l]{\cite{Barrage2019OptimalPolicy}}& 0.9272 & 5 year discount factor  \\
				\hline
				$\bar{H}$& \makecell[l]{14.5 hours per day\\ \cite{Jones1993OptimalGrowth}}&5&\makecell[l]{economic time endowment \\per 5 years, normalised} \\
				\hline
				\hline
				Research&\multicolumn{3}{c}{}\\
				\hline
				
				\hline
				$\sigma_s$ &  \makecell[l]{\cite{Chetty2011AreMargins}}& $4/3$ & inverse Frisch elasticity  \\
				\hline
				$\chi_s$ &\makecell[l]{average hours worked per \\ economic time endowment\\ by worker: 0.34 \cite{OECDHoursworked}} & 0.032 & disutility from science\\
				\hline
				$\eta$ &\makecell[l]{\cite{Fried2018ClimateAnalysis}} & 0.79 & spillover research\\
				\hline			
				$\rho_f$ &\makecell[l]{\cite{Fried2018ClimateAnalysis}} & 0.01 &\makecell[l]{research tasks in\\ fossil sector}\\
				\hline			
				$\rho_g$ &\makecell[l]{\cite{Fried2018ClimateAnalysis}} & 0.01 &\makecell[l]{research tasks in\\ green sector}\\
				\hline			
				$\rho_n$ &\makecell[l]{\cite{Fried2018ClimateAnalysis}} & 1 &\makecell[l]{research tasks in \\non-energy sector}\\
				\hline			
				$\phi$ &\makecell[l]{\cite{Fried2018ClimateAnalysis}} & 0.5 &across-sector research spillovers\\
				\hline
					$\gamma$ &\makecell[l]{growth in non-energy sector:\\2\% per annum \cite{Fried2018ClimateAnalysis}} & 0.0042 & productivity of research\\
				\hline
				\hline
				Production&\multicolumn{3}{c}{}\\
				\hline
				
				\hline
				$\varepsilon_y$&\cite{Fried2018ClimateAnalysis}&0.05& \makecell[l]{substitutability \\ energy and non-energy}\\			
				\hline
				$\delta_y$&\makecell[l]{expenditure share \\ on energy IEA}&0.4496& \makecell[l]{weight on energy in\\ final good production}\\	
				\hline
				$\varepsilon_e$&\cite{Fried2018ClimateAnalysis}&1.5& \makecell[l]{substitutability \\ green and fossil energy}\\	
				\hline
				$\alpha_f$&\cite{Fried2018ClimateAnalysis} &0.72& capital share fossil  \\
				\hline
				$\alpha_g$&\cite{Fried2018ClimateAnalysis} &0.91& capital share green \\
				\hline
				$\alpha_n$&\cite{Fried2018ClimateAnalysis} &0.36& capital share non-energy  \\
				%\hline
				%$\beta$&\makecell{ annual nominal rate 3\%\\ and annual inflation rate of 2\%}& 0.9903& discount factor\\ 
				\hline
				\hline
				Initial Productivity&\multicolumn{3}{c}{}\\
				\hline
				
				\hline
				$A_{f0}$&- &3350.5& \makecell[l]{initial productivity \\ fossil sector, 2014-2019}  \\
				\hline
				$A_{g0}$&- &95.4& \makecell[l]{initial productivity \\ green sector, 2014-2019}  \\
				\hline
				$A_{n0}$&- &4.3& \makecell[l]{initial productivity \\ non-energy sector, 2014-2019}  \\
				\hline \hline
			\end{tabular}
		\end{center}
	\end{table}
\begin{table}[hh!!!!!]
	\begin{center}
		\captionsetup{width=0.9\textwidth}
		\caption{ Calibration baseline model: Labour, Government, and Emissions}
		\label{tab:calib2}
		\begin{tabular}{c|lll}
			%			\hline \hline
			%			\multicolumn{7}{c}{Calibration based on basic needs}\\
			\hline \hline
			Parameter& Target/Source& \makecell[l]{Calibration}& \makecell[l]{Meaning}\\ 
			\hline
			\hline
			Labour Production&\multicolumn{3}{c}{}\\
			\hline 
			
			\hline
			$\theta_f$&\makecell[l]{share of high skill\\ non-green occupations: \\27.55\% }&0.4194&\makecell[l]{income share high \\ skill fossil sector}\\
			\hline
			$\theta_g$&\makecell[l]{share of high skill\\ green occupations: \\40.71\% }&0.5661&\makecell[l]{indome share high \\skill green sector}\\
			\hline
			$\theta_n$&\makecell[l]{share of high skill\\ non-green occupations: \\27.55\% }&0.4194&\makecell[l]{income share high \\ skillnon-energy sector}\\
			\hline
			\hline
			Government&\multicolumn{3}{c}{}\\
			\hline
			
			\hline
			$\tau_f$&- &0& sales tax on fossil energy\\
			\hline
			$\tau_l$&\cite{Heathcote2017OptimalFramework} &0.181& income tax progressivity\\
			\hline	
			\hline
			Emissions&\multicolumn{3}{c}{}\\
			\hline
			
			\hline
			$\delta$& \makecell[l]{EPA}&0.7893&carbon sinks \\
			\hline
			$\omega$& EPA&45.5634& \makecell[l]{ gross emissions as a\\ fraction of fossil output}\\
				$\Omega$& IPCC report April 2022&\makecell[l]{from 2030-2050: 4.0684Gt\\2050-2080: 0Gt}& \makecell[l]{net emission target}\\
			\hline \hline
		\end{tabular}
	\end{center}
\end{table}

%
\thispagestyle{empty}
	\begin{table}[h!]
		\begin{center}
			\captionsetup{width=0.9\textwidth}
			\caption{ Calibration baseline model: Households, Research and Production}
			\label{tab:calib}
			\begin{tabular}{c|lll}
				%			\hline \hline
				%			\multicolumn{7}{c}{Calibration based on basic needs}\\
				\hline \hline
				Parameter& Target/Source& \makecell[l]{Calibration}& \makecell[l]{Meaning}\\ 
				\hline
				\hline
				Household&\multicolumn{3}{c}{}\\
				\hline 
				
				\hline
				$\sigma$ &  \makecell[l]{\cite{Chetty2011AreMargins}}& $4/3$ & inverse Frisch elasticity  \\
				\hline
				$z_h$& \makecell[l]{skill premium 2005-2016:\\ $w_h/w_l=1.9$\\ \citep{Slavik2020WagePremium}}&0.2121&\makecell[l]{share of\\ high-skilled workers} \\	
				\hline			
				$\chi$ &  \makecell[l]{average hours worked per\\ economic time endowment\\ by worker: 0.34 \cite{OECDHoursworked}}& 10.021 & inverse Frisch elasticity  \\
				\hline
				$\beta$ &  \makecell[l]{\cite{Barrage2019OptimalPolicy}}& 0.9272 & 5 year discount factor  \\
				\hline
				$\bar{H}$& \makecell[l]{14.5 hours per day\\ \cite{Jones1993OptimalGrowth}}&5&\makecell[l]{economic time endowment \\per 5 years, normalised} \\
				\hline
				\hline
				Research&\multicolumn{3}{c}{}\\
				\hline
				
				\hline
				$\sigma_s$ &  \makecell[l]{\cite{Chetty2011AreMargins}}& $4/3$ & inverse Frisch elasticity  \\
				\hline
				$\chi_s$ &\makecell[l]{average hours worked per \\ economic time endowment\\ by worker: 0.34 \cite{OECDHoursworked}} & 0.032 & disutility from science\\
				\hline
				$\eta$ &\makecell[l]{\cite{Fried2018ClimateAnalysis}} & 0.79 & spillover research\\
				\hline			
				$\rho_f$ &\makecell[l]{\cite{Fried2018ClimateAnalysis}} & 0.01 &\makecell[l]{research tasks in\\ fossil sector}\\
				\hline			
				$\rho_g$ &\makecell[l]{\cite{Fried2018ClimateAnalysis}} & 0.01 &\makecell[l]{research tasks in\\ green sector}\\
				\hline			
				$\rho_n$ &\makecell[l]{\cite{Fried2018ClimateAnalysis}} & 1 &\makecell[l]{research tasks in \\non-energy sector}\\
				\hline			
				$\phi$ &\makecell[l]{\cite{Fried2018ClimateAnalysis}} & 0.5 &across-sector research spillovers\\
				\hline
					$\gamma$ &\makecell[l]{growth in non-energy sector:\\2\% per annum \cite{Fried2018ClimateAnalysis}} & 0.0042 & productivity of research\\
				\hline
				\hline
				Production&\multicolumn{3}{c}{}\\
				\hline
				
				\hline
				$\varepsilon_y$&\cite{Fried2018ClimateAnalysis}&0.05& \makecell[l]{substitutability \\ energy and non-energy}\\			
				\hline
				$\delta_y$&\makecell[l]{expenditure share \\ on energy IEA}&0.4496& \makecell[l]{weight on energy in\\ final good production}\\	
				\hline
				$\varepsilon_e$&\cite{Fried2018ClimateAnalysis}&1.5& \makecell[l]{substitutability \\ green and fossil energy}\\	
				\hline
				$\alpha_f$&\cite{Fried2018ClimateAnalysis} &0.72& capital share fossil  \\
				\hline
				$\alpha_g$&\cite{Fried2018ClimateAnalysis} &0.91& capital share green \\
				\hline
				$\alpha_n$&\cite{Fried2018ClimateAnalysis} &0.36& capital share non-energy  \\
				%\hline
				%$\beta$&\makecell{ annual nominal rate 3\%\\ and annual inflation rate of 2\%}& 0.9903& discount factor\\ 
				\hline
				\hline
				Initial Productivity&\multicolumn{3}{c}{}\\
				\hline
				
				\hline
				$A_{f0}$&- &3350.5& \makecell[l]{initial productivity \\ fossil sector, 2014-2019}  \\
				\hline
				$A_{g0}$&- &95.4& \makecell[l]{initial productivity \\ green sector, 2014-2019}  \\
				\hline
				$A_{n0}$&- &4.3& \makecell[l]{initial productivity \\ non-energy sector, 2014-2019}  \\
				\hline \hline
			\end{tabular}
		\end{center}
	\end{table}
\begin{table}[hh!!!!!]
	\begin{center}
		\captionsetup{width=0.9\textwidth}
		\caption{ Calibration baseline model: Labour, Government, and Emissions}
		\label{tab:calib2}
		\begin{tabular}{c|lll}
			%			\hline \hline
			%			\multicolumn{7}{c}{Calibration based on basic needs}\\
			\hline \hline
			Parameter& Target/Source& \makecell[l]{Calibration}& \makecell[l]{Meaning}\\ 
			\hline
			\hline
			Labour Production&\multicolumn{3}{c}{}\\
			\hline 
			
			\hline
			$\theta_f$&\makecell[l]{share of high skill\\ non-green occupations: \\27.55\% }&0.4194&\makecell[l]{income share high \\ skill fossil sector}\\
			\hline
			$\theta_g$&\makecell[l]{share of high skill\\ green occupations: \\40.71\% }&0.5661&\makecell[l]{indome share high \\skill green sector}\\
			\hline
			$\theta_n$&\makecell[l]{share of high skill\\ non-green occupations: \\27.55\% }&0.4194&\makecell[l]{income share high \\ skillnon-energy sector}\\
			\hline
			\hline
			Government&\multicolumn{3}{c}{}\\
			\hline
			
			\hline
			$\tau_f$&- &0& sales tax on fossil energy\\
			\hline
			$\tau_l$&\cite{Heathcote2017OptimalFramework} &0.181& income tax progressivity\\
			\hline	
			\hline
			Emissions&\multicolumn{3}{c}{}\\
			\hline
			
			\hline
			$\delta$& \makecell[l]{EPA}&0.7893&carbon sinks \\
			\hline
			$\omega$& EPA&45.5634& \makecell[l]{ gross emissions as a\\ fraction of fossil output}\\
				$\Omega$& IPCC report April 2022&\makecell[l]{from 2030-2050: 4.0684Gt\\2050-2080: 0Gt}& \makecell[l]{net emission target}\\
			\hline \hline
		\end{tabular}
	\end{center}
\end{table}

%\clearpage