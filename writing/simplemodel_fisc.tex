\section{Simple Model and Hypothesis}
Simple model with representative agent who provides two skills: high and low. I look at steady states: laissez faire versus the steady state under the optimal policy. 
The focus rests on matching emissions in the model to emission targets suggested by the IPCC report \citep{Rogelj2018MitigationDevelopment.}. 

How to determine the economy in 2050? Should the economy have reached a steady state? or should it be in a transitional path? Maybe no need to specify this...it will be a outcome. All I have to use is that for all years after 2050 net-emissions have to be zero. Whether the economy is on the transitional path or in a steady state is an outcome. 
\subsection{Model and Hypothesis}

\paragraph{Households}
% the rep agent
The economy behaves as if there was a representative household. The household chooses between high and low-skilled labour. The household experiences utility costs from skill accumulation. In equilibrium, the high-skilled labour receives a higher wage rate. 

\begin{align}
U=\underset{\{c_{t}\}_{t=0}^{\infty}, \{h_{lt}\}_{t=0}^{\infty}, \{h_{ht}\}_{t=0}^{\infty}}{max}&
\sum_{t=0}^{\infty}\beta^t u(c_{t}, h_{lt}, h_{ht}, h_{ht+1})\\
%U_{s}=\underset{\{c_{st}\}_{t=0}^{\infty}, \{h_{st}\}_{t=0}^{\infty}}{max}&\sum_{t=0}^{\infty}\beta^t u_s(c_{st}, h_{st}; S_t)\\
s.t.& \ \ c_{t}p_{t}=% (1-\tau_{lt})(h_{ht}w_{ht}+h_{lt}w_{lt})+T_t\\ 
\lambda \left(h_{ht}w_{ht}+h_{lt}w_{lt}\right)^{1-\tau}\\
\ & h_{ht}+h_{lt}\leq \bar{H}_t\\
\ & h_{st}\geq 0 \ \forall s\in \{l,h\}
\end{align}
The period utility function includes costs for on-the-job training which is required for high-skilled labour in the next period. 
\begin{align}
	u(c_{t}, h_{lt}, h_{ht}, h_{ht+1})&= %\frac{c_t^{1-\gamma}}{1-\gamma}
	\log(c_t)-\frac{(h_{lt}+\zeta h_{ht})^{1+\sigma}}{1+\sigma}%-v(h_{ht+1})%,\\
	%\text{where}\  & v(h_{ht+1})=\left\{\begin{array}{lll}\zeta& \hspace{2mm} \text{if} \hspace*{2mm}  h_{ht+1}> 0, &\\
%0  &\hspace{2mm}\text{if}\hspace{2mm}  h_{ht+1}= 0.
%	\end{array}
%	\right. 		
\end{align}
The positive parameter $\zeta>1$ implies a higher marginal disutility for high-skilled labour than unskilled labour. As a result, in equilibirum,  
high-skilled labour earns a premium to compensate the representative household for the higher disutility. When labour income gets taxed, the returns to learning reduce and skilled labour becomes scarcer on impact. 

The log-utility from consumption ensures balanced-growth-path compatibility of hours worked. However, this makes the reduction in hours supplied independent of the wage rate. Still, the extensive margin through learning should remain.

I define the variable $H_t:=\zeta h_{ht}+h_{lt}$ which facilitates the derivation of results. 

\paragraph{Production}
There are two production sectors: a clean and a dirty one, indexed by $c$ and $d$. Sector specific goods are imperfect substitutes in final consumption good production.\footnote{\ This ensures that the dirty good is always produced and overall, production is never perfectly clean.}  
The final good producing sector is perfectly competitive:
$Y_t=\left(Y_{ct}^{\frac{\varepsilon-1}{\varepsilon}}+Y_{dt}^{\frac{\varepsilon-1}{\varepsilon}}\right)^\frac{\varepsilon}{\varepsilon-1}$. 
I take the composite good as the numeraire so that $\left[p_{dt}^{1-\varepsilon}+p_{ct}^{1-\varepsilon}\right]^{\frac{1}{1-\varepsilon}}=1$.

In both sectors, a unit mass of competitive firms $i$ produces an individual consumption good. All firms use machines, $x_{jit}$ and an intermediate labour good, $L_{jt}$ as input.\footnote{\ For now I abstract from a natural resource.} 
\begin{align*}
&Y_{dt}= L_{dt}^{1-\alpha}\int_{0}^{1}A_{dit}^{1-\alpha}x_{dit}^{\alpha} di,\ \hspace{2mm} Y_{ct}= L_{ct}^{1-\alpha}\int_{0}^{1}A_{cit}^{1-\alpha}x_{cit}^{\alpha} di.
\end{align*}

The labour input good is produced by a perfectly competitive and sector-specific labour industry according to: 
\begin{align}
L_{jt}=l_{jht}^{\theta_j}l_{jlt}^{1-\theta_j}, \ for \ j \in\{c,d\},
\end{align}
where $\theta_c>\theta_d$ so that the clean sectors labour input has a higher share of high-skilled labour. 


\paragraph{Machine producing firms}
A perfectly competitive sector produces machines, $x_{ijt}$, and sells them to final good firms in the respective sector at price $p_{ijt}$. It is assumed that the costs to produce one machine, $\psi$, are homogeneous across firms. It follows that $p_{ijt}=\psi$.


\paragraph{Technological progress}
Technological progress is exogenous:
\begin{align}
A_{ijt+1}=(1+\upsilon_{jt}) A_{ijt}\ for \ j \in\{c,d\}. 
\end{align}

\paragraph{Impossibility of reaching target in laissez-faire with exogenous growth}
\tr{Note that this is wrong! There is an option for the gov to affect inflation which then redirects demand.}
Note that with exogenous growth in each sector there is no possibility for the government to stop emissions from growing, since production of the dirty good is essential for the consumption good (no perfect substitution: $\varepsilon<\infty$). To meet the emission target, the government either needs to affect the growth rate in the economy; i.e., $\upsilon_j$ is a choice variable, or work and consumption need to be set to zero; or the emission target has to be defined in relative terms. The latter possibility contradicts the Paris Agreement which is concerned with absolute emissions.  
I therefore assume, that the government can change the growth rate.

The government chooses the growth rate in each sector, taking into account that research is constrained by an exogenous  amount of scientists
\begin{align}
\upsilon_{ct}+\upsilon_{dt}\leq\Upsilon
\end{align}
 
  
\paragraph{Government}

The government maximises social welfare but is constrained by meeting emission targets in line with the Paris Agreement. Furthermore, the government does not have corrective taxes at its disposal. Instead, only already established tax instruments: distortionary labour taxes (consumption taxes) are available. 

\begin{align*}
\underset{\{\tau_{lt}, \upsilon_{ct}, \upsilon_{dt}\}_{t=0}^{\infty}}{max}&\sum_{t=0}^{\infty}\beta^t u(c_{t}, h_{ht}, h_{lt})\\
s.t.\ & (1)\  \tau_{lt}(h_{ht}w_{ht}+h_{lt}w_{lt})=T_t\  \forall \ t\geq 0\\
& (2)\ \underbrace{\kappa Y_{nt}}_{\text{emissions in t}} -\delta \leq E_t \  \forall \ t\geq 0\\
& (3)\ \upsilon_{ct}+\upsilon_{dt}\leq\Upsilon\  \forall \ t\geq 0\\
& (4)\ \text{behaviour of firms and households}
\end{align*}

$E_t$ are flow emissions per year. The IPCC prescribes net-zero emissions starting from 2050 and in 2030 to $E_t= 25-30GtCO2e\ yr^{-1}$. The parameter $\delta$ captures the capacity of the environment to reduce emitted $CO2$ through sinks, such as forests and moors. Hence, in the net-zero steady state it has to hold that $Y_{nt}=\frac{\delta}{\kappa}\ \forall t\geq 30$ assuming that the analysis starts in 2020. 

\subsection{Hypothesised outcome}
How do I expect the optimal steady state to differ from the laissez-faire one? 
In the representative agent model, the government faces a trade-off  between efficiency and the externality. 
On the one hand, the distortionary labour tax reduces output and thereby the externality of production. On the other hand, it reduces utility from consumption.

Allowing for two skill types and a skill bias of the cleaner sector adds an additional layer to the effect of labour taxes on the environment. Instead of merely reducing output there is also a recomposition effect. 
In response to the labour tax, the household reduces its labour supply. Since the high-skilled labour earns a higher wage rate, unskilled labour becomes more attractive to the representative agent. The lower supply of skilled labour increases production costs of the cleaner sector. The price of the cleaner good increases. Hence, the share of clean to dirty output falls. This indirect recomposition effect counteracts the direct reduction of the externality. 

I hypothesise that under this assumption growth in the clean sector, too, will have to stop. Why? Consider that only the clean sector growths, then the price for clean goods has to fall so that the final good sector  continues demand the supply of the clean good. The price will be driven towards zero. Which cannot be an equilibrium solution since the clean sector would stop producing as costs exceed revenues (only if marginal production costs tend to zero but they don't as labour exerts disutility).

How can then be there a role for distortionary labour taxes if the government can choose growth rates? Maybe not. But once growth is endogenous? Maybe during the transition? 
Maybe because reducing labour supply is better in terms of utility? 

Some rationale for setting hours restriction? 
\subsection{Equilibrium conditions}

\begin{align*}
\text{\textbf{Household}} & \\
\text{FOC consumption}\hspace{4mm}&  \log(c_t)= \log(\lambda_t)+ (1-\tau_{lt})\left[\frac{1}{1+\sigma}\log(1-\tau_{lt})+\log(w_{lt})\right]\\
\text{FOCs labour supply}\hspace{4mm}&  \log(H_t)=\frac{1}{1+\sigma}\log(1-\tau_{lt})\\
\ \hspace{4mm} & \log(w_{ht})=\log(w_{lt})+\log(\zeta)\\
\text{definition}\  H_t\hspace{4mm} & \log(H_t)=\log(h_{lt}+\zeta h_{ht})\\
\ 
\\
\text{\textbf{Production}}& \\
\text{\textbf{Final Good Producer}}\\
\text{Profit maximisation}\hspace{4mm} & Y_{nt}=\left(\frac{p_{ct}}{p_{dt}}\right)^\varepsilon Y_{ct}\\
\text{Production}\hspace{4mm} & Y_t=\left[Y_{ct}^{\frac{\varepsilon-1}{\varepsilon}}+Y_{dt}^{\frac{\varepsilon-1}{\varepsilon}}\right]^{\frac{\varepsilon}{\varepsilon-1}}\\
\text{Price}\hspace{4mm}& p_t:=\left[p_{ct}^{1-\varepsilon}+p_{dt}^{1-\varepsilon}\right]^{\frac{1}{1-\varepsilon}}\\
\text{\textbf{Clean Sector}}\\
\text{Production}\hspace{4mm}& Y_{ct}=L^{1-\alpha}_{ct}\int_{0}^{1}A^{1-\alpha}_{ict}x_{ict}^{\alpha}di=  \left(\alpha\frac{p_{ct}}{\psi}\right)^{\frac{\alpha}{1-\alpha}}A_{ct} L_{ct} \\ 
\text{labour demand}\hspace{4mm} & p_{cLt} =
(1-\alpha)\left(\frac{\alpha}{\psi}\right)^\frac{\alpha}{1- \alpha}p_{ct}^\frac{1}{1-\alpha}A_{ct}\\
\text{machine demand}\hspace{4mm} & x_{ict} = \left(\alpha\frac{ p_{ct}}{p_{ict}}\right)^\frac{1}{1-\alpha}A_{ict}L_{ct}\\
%
\text{Supply machines (price)}\hspace{4mm}& p_{ict}=\psi \\
\text{\textbf{Dirty Sector}}\\
\text{Production}\hspace{4mm} & Y_{dt}=L^{1-\alpha}_{dt}\int_{0}^{1}A^{1-\alpha}_{idt}x_{idt}^{\alpha}di=  \left(\alpha\frac{p_{dt}}{\psi}\right)^{\frac{\alpha}{1-\alpha}}A_{dt} L_{dt} \\ 
\text{labour demand}\hspace{4mm} & p_{dLt} =
(1-\alpha)\left(\frac{\alpha}{\psi}\right)^\frac{\alpha}{1- \alpha}p_{dt}^\frac{1}{1-\alpha}A_{dt}\\
\text{machine demand}\hspace{4mm} & x_{idt} = \left(\alpha\frac{ p_{dt}}{p_{idt}}\right)^\frac{1}{1-\alpha}A_{idt}L_{dt}\\
\text{Supply machines (price)}\hspace{4mm}& p_{idt}=\psi
\end{align*}

\begin{align*}
\text{\textbf{Labour sectors}}\\
\text{Production clean labour input} \hspace{4mm}& L_{ct}=l_{hct}^{\theta_c}l_{lct}^{1-\theta_c}\\ 
\text{Production dirty labour input} \hspace{4mm}& L_{dt}=l_{hdt}^{\theta_d}l_{ldt}^{1-\theta_d}\\
%
\text{Demand high skill clean sector}\hspace{4mm}&l_{hct}= \left(\frac{p_{cLt}}{w_{ht}}\right)^{\frac{1}{1-\theta_c}}\theta_c^{\frac{1}{1-\theta_c}}l_{lct}\\
%
\text{Demand low skill clean sector } \hspace{4mm}&l_{lct}= \left(\frac{p_{cLt}}{w_{lt}}\right)^{\frac{1}{\theta_c}}(1-\theta_c)^{\frac{1}{\theta_c}}l_{hct}\\
%
\text{Demand high skill dirty sector} \hspace{4mm}&l_{hdt}= \left(\frac{p_{dLt}}{w_{ht}}\right)^{\frac{1}{1-\theta_d}}\theta_d^{\frac{1}{1-\theta_d}}l_{ldt}\\
%
\text{Demand low skill dirty sector } \hspace{4mm}&l_{ldt}= \left(\frac{p_{dLt}}{w_{lt}}\right)^{\frac{1}{\theta_d}}(1-\theta_d)^{\frac{1}{\theta_d}}l_{hdt}\\
\text{\textbf{Market clearing}}& \nonumber\\
\text{Consumption Good}\hspace{4mm}& Y_{t}=c_t; \ 
(\text{Numeraire}\ \  p_t=1)\\
\text{high skill}\hspace{4mm}& l_{hct}+l_{hdt}=h_{ht}\\
\text{low skill}\hspace{4mm}&l_{lct}+l_{ldt}=h_{lt}\\
\text{\textbf{Technology}}\hspace{4mm}\\
\text{Clean sector}\hspace{4mm}& A_{ict+1}=(1+\upsilon_{ct})A_{ict}\\
\text{Dirty sector}\hspace{4mm}& A_{idt+1}=(1+\upsilon_{dt})A_{idt}\\
\text{Progress bound}\hspace{4mm}& \upsilon_{ct}+\upsilon_{dt}=\Upsilon\\
\text{Definition average clean technology}\hspace{4mm}& A_{ct}=\int_0^1A_{ict}di\\
\text{Definition average dirty technology}\hspace{4mm}& A_{dt}=\int_0^1A_{idt}di
\end{align*}
\subsection{Balanced Growth path}
A balanced growth path  is defined as follows. The economy is in equilibrium as specified above. Furthermore, I impose that all variables grow at constant rate. A dash $'$ indicates next period variables. 

It follows that the labour input good does not grow, since hours worked are constant and transitional dynamics are ruled out by definition.\footnote{\ In a subsection below, I prove this claim.}

I write the evolution of the model as a function of growth rates and initial conditions $A_{c0}, A_{d0}$. I also impose that policy variables, $\upsilon_{c}, \upsilon_{d}, \tau_l, \lambda$, are constant on the balanced growth path. 

From the FOCs on skill demand follows that the price of the labour input good relative to the skill-specific wage rate is constant. Substituting demand for low skill in the clean sector into the demand for high skill yields

\begin{align*}
 w_{h}^{\frac{1}{1-\theta_c}}w_l^{\frac{1}{\theta_c}}= p_{cL}^\frac{1}{(1-\theta_c)\theta_c}\theta_c^\frac{1}{1-\theta_c}(1-\theta_c)^\frac{1}{\theta_c}.
\end{align*}
Multiplying the left-hand side with $(w_h/w_h)^\frac{1}{\theta_c}$ and
using the FOC governing skill supply $w_h/w_l=\zeta$, it holds that

\begin{align}\label{eq:constant}
& \zeta^\frac{-1}{\theta_c}w_h^\frac{1}{(1-\theta_c)\theta_c}= p_{cL}^\frac{1}{(1-\theta_c)\theta_c}\theta_c^\frac{1}{1-\theta_c}(1-\theta_c)^\frac{1}{\theta_c}\nonumber\\
	\Leftrightarrow\ & \frac{p_{cL}}{w_h}= \zeta^{-(1-\theta_c)}\theta_c^{-\theta_c}(1-\theta_c)^{-(1-\theta_c)}.
\end{align}
\noindent \tr{Note: this result does not rely on the claim that the labour input good is constant.}

Define sector-specific inflation as: $1+\pi_{j}=\frac{p'_j}{p_j}$.
Using the definition of the aggregate price level, final good production, and optimality conditions in the clean sector, one can show that 
\begin{align}\label{eq:agg_supply}
\frac{Y'}{Y}= (1+\pi_c)^{\frac{\varepsilon(1-\alpha)+\alpha}{1-\alpha}}(1+\upsilon_{c}), \hspace{3mm} \text{(Supply side)}
\end{align}
since $\frac{L_{c}'}{L_c}=1$.
Using goods market clearance, the budget condition, and the FOC for total skill supply, it follows that 
\begin{align}\label{eq:agg_demand}
\frac{Y'}{Y}= \left(\frac{w'_h}{w_h}\right)^{1-\tau_l}. \hspace{3mm} \text{(Demand side)}
\end{align}
Demand for the labour input good implies that 
\begin{align}\label{eq:labour income}
\frac{p'_{cL}}{p_{cL}}= (1+\pi_c)^\frac{1}{1-\alpha}(1+\upsilon_{c})
\end{align}
(independent of growth in $L_c$).

Multiplying both sides with $\left(\frac{w_h'}{w_h}\right)^{-1}$, using equation \ref{eq:agg_growth}, and that $\frac{p_{cL}}{w_h}$ is constant, it follows that 

\begin{align}
		\frac{\frac{p'_{cL}}{w'_h}}{\frac{p_{cL}}{w_h}}= (1+\pi_c)^\frac{1}{1-\alpha}(1+\upsilon_{c})\left(\frac{Y'}{Y}\right)^{-\frac{1}{1-\tau_l}}=1.
\end{align}

Above equation determines inflation in the clean sector:
\begin{align}\label{eq:inf_c}
1+\pi_c=(1+\upsilon_{c})^{\frac{\tau_l(1-\alpha)}{(1-\tau_l)-\varepsilon(1-\alpha)-\alpha}}.
\end{align}
By symmetry of (i) how goods enter the production fo the final good and of (ii) sectors, it also holds that 
\begin{align}\label{eq:inf_d}
1+\pi_d=(1+\upsilon_{d})^{\frac{\tau_l(1-\alpha)}{(1-\tau_l)-\varepsilon(1-\alpha)-\alpha}}.
\end{align}


\noindent\rule[1ex]{\textwidth}{1pt}

\noindent \textbf{(Aggregate output result, less relevant for main story)}

Hence, 
\begin{align}\label{eq:agg_growth}
	(1+g_y)=\frac{Y'}{Y}=(1+\upsilon_{c})^\frac{(1-\tau_l)[1-(\varepsilon(1-\alpha)+\alpha)]}{(1-\tau_l)-(\varepsilon(1-\alpha)+\alpha)}.
\end{align}
Equation \ref{eq:agg_growth} implies the following proposition:
\begin{prop}[aggregate growth]
\textit{For a proportional tax system, $\tau_l=0$, aggregate growth equals growth in the clean sector. 
When the tax system is progressive\footnote{\ In the sense defined in \cite{Heathcote2017OptimalFramework}.}, $\tau_l>0$, then aggregate growth exceeds technology growth in the clean sector. When the tax rate is regressive, $\tau_l<0$, aggregate growth is smaller than technology growth in the clean sector. }
\end{prop}
\tr{Have to understand why.} 
With a flat tax system there is no inflation in the clean sector; compare equation \ref{eq:inf_c}. When the tax system is progressive, ...

\paragraph{Proof: labour input good constant}
%\textit{Check that the labour input good is constant:} 

First note that $\frac{l_{hc}}{l_{lc}}$ is constant over time. 
From the FOC governing high skill demand in the clean sector and equation \ref{eq:constant} we have:

\begin{align*}
	\frac{l_{hc}}{l_{lc}}=\left(\frac{p_{cL}}{w_h}\theta_c\right)^{\frac{1}{1-\theta_c}}= constant.
\end{align*}

Substitution into the production function of the clean labour input good yields

\begin{align*}
\frac{L'_c}{L_c}=\frac{l_{lc}'}{l_{lc}}.
\end{align*}

\tr{To be continued.}

\textbf{\tr{To be shown next:  How $\tau_l$ affects (1) skill supply (level) and (2) externality. }}
\\

\noindent\rule[1ex]{\textwidth}{1pt}


\paragraph{Progressivity and emission targets}
The constraint on emissions in the government's objective function implies that $Y_{dt}=\frac{\delta}{\kappa}$, thus, $(1+g_{ydt})=1$, for all time periods starting from 2050, $t\geq 30$. 

From the dirty sector's production function and equation \ref{eq:inf_d} we have that
\begin{align}
&\frac{Y_{d}'}{Y_d}=(1+\pi_d)^{\frac{\alpha}{1-\alpha}}(1+\upsilon_{d})\label{eq:gyd}\\
\Leftrightarrow\ &(1+\upsilon_{d})^{\frac{(1-\alpha)(1-\tau_l-\varepsilon)}{(1-\tau_l)-(\varepsilon(1-\alpha)+\alpha)}}=1\label{eq:def_taul}
\end{align}
The inflation rate in equation \ref{eq:gyd} captures the role of machine demand by the dirty sector. When the price at which dirty firms can sell their output is high, they demand more machines. A positive inflation, therefore, implies a rise in dirty output.

At the same time, a rise in the dirty good's price reduces demand. This counteracting mechanism is accounted for in equation \ref{eq:def_taul}. 
 As will be shown below, this mechanism ensures that the government can target dirty sector production through tax progressivity. 

First, I establish an optimal policy result. Assume that the government cannot set the growth rate in the dirty sector, then equation \ref{eq:def_taul} defines $\tau_l$ on a balanced growth path.

\begin{prop}[Optimal tax progressivity]
Assume growth of the dirty technology, $\upsilon_{d}$, is exogenously determined. 
Then, to comply with the Paris Agreement, the government has to set the tax progressivity parameter, $\tau_l$, to $\tau^*_l=1-\varepsilon$ for $\varepsilon\neq 1$ (as otherwise the exponent in \ref{eq:def_taul} is not defined under the optimal tax rate.).
When goods are complements, the optimal tax system is progressive. If goods are substitutes, the optimal tax system is regressive.
\end{prop}

The intuition is, that by choosing tax progressivity, the government affects price inflation in the dirty sector; compare equation \ref{eq:inf_d}. 
The result implies that inflation in the dirty sector under the optimal policy is negative when there is positive growth in dirty technology. The demand for machines has to decline by the same rate as technology growths for dirty output to be constant.  

Can this be an equilibrium as the price for dirty products declines?

How does tax progressivity affect inflation? 
First note that at a flat tax, the inflation rate is independent of the sector-specific technological growth rate. This is due to offsetting mechanisms.\tr{Continue}

 
\tr{Start from effect on HH:} 
(1) A rise in $\tau_l$ reduces disposable income and aggregate demand falls. This is a mechanical result from a higher tax rate, and a reduction in aggregate hours supplied.  
(1) For the dirty sector to demand labour, the costs of the labour input good has to balance its marginal product which positively depends on technological progress and the sector specific price. 

\paragraph{Effect of $\tau_l$ on skill investment}
From the definition of $H$ it has to hold that 
\begin{align}
&1=\frac{dh_l}{dH}+\zeta \frac{dh_h}{dH}\label{eq:ident} \\
\Leftrightarrow\ & \frac{dh_h}{dH}=\frac{1-\frac{dh_l}{dH}}{\zeta}.\label{eq:resp}
\end{align}
Using this equation, one can show that high skill supply is relatively more responsive to changes in total effective hours worked, i.e.,  $\frac{dh_h}{dH}>\frac{dh_l}{dH}$, if one excludes the case that high skill supply reduces as effective hours increase.\footnote{\ Proof: Suppose   $\frac{dh_h}{dH}>0$. Now, assume by contradiction that low skill supply is relatively more responsive. Hence, $\frac{dh_h}{dH}<\frac{dh_l}{dH}$. Using equation \ref{eq:resp}, one gets that $\frac{dh_l}{dH}>1+\zeta$. Replacing this inequality in the identity \ref{eq:ident}, it follows that $0>\zeta[1+\frac{dh_h}{dH}]$. Since $\zeta>1$ by assumption, it has to hold that $\frac{dh_h}{dH}<-1$ which contradicts the premise that $\frac{dh_h}{dH}>0$. } Thus, as the household reduces total effective hours supplied, the reduction in high skilled hours is higher. \tr{This should be due to the marginal utility from less high skill is higher than from less low skill hours.} This should show up in general equilibrium effects... but relative wages are fixed. 


\textbf{Supply side}
Analogously to \ref{eq:constant}, one can show that
\begin{align*}
\frac{p_{cL}}{w_l}&=\zeta^{\theta_c}\theta_c^{-\theta_c}(1-\theta_c)^{-(1-\theta_c)}\\
\frac{p_{dL}}{w_l}&=\zeta^{\theta_d}\theta_d^{-\theta_d}(1-\theta_d)^{-(1-\theta_d)}\ \Leftrightarrow\ w_l= p_{dL}\zeta^{-\theta_d}\theta_d^{\theta_d}(1-\theta_d)^{1-\theta_d}\\
\frac{p_{dL}}{w_h}&=\zeta^{-(1-\theta_d)}\theta_d^{-\theta_d}(1-\theta_d)^{-(1-\theta_d)}.
\end{align*}
\paragraph{Which variable ratios are constant on a BGP?}

\subsection{Solution in levels - for coding}
Using the values for $p_{jL}/w_s$ it follows that the optimal skill input ratios in the labour good production are
\begin{align*}
\frac{l_{hc}}{l_{lc}}=\frac{\theta_c}{\zeta (1-\theta_c)}; \hspace{2mm} \frac{l_{hd}}{l_{ld}}=\frac{\theta_d}{\zeta (1-\theta_d)}.
\end{align*}
Imposing labour market clearing for both skills and optimal skill demand  yields 
\begin{align*}
&l_{ld}=\frac{\theta_c}{1-\theta_c}\chi h_l-\chi \zeta h_h,\\
& l_{lc}=\chi \zeta h_h-\frac{\theta_d}{1-\theta_d}\chi h_l\\
with \ & \chi:= \frac{(1-\theta_d)(1-\theta_c)}{\theta_c(1-\theta_d)-\theta_d(1-\theta_c)}.
%& l_{hc}= \frac{\theta_c}{\zeta (1-\theta_c)}l_{lc}\\
%& l_{hd}=\frac{\theta_d}{\zeta (1-\theta_d)}l_{ld}
\end{align*}
Labour good supply is then given by (where I substituted $h_h$)
\begin{align}
L_c&=\underbrace{\left(\frac{\theta_c}{\zeta(1-\theta_c)}\right)^{\theta_c}}_{=:\gamma_c}\chi\left(H +\frac{\theta_d}{1-\theta_d} h_l\right)\label{eq:lab_inputc} \\
L_d&= \underbrace{\left(\frac{\theta_d}{\zeta (1-\theta_d)}\right)^{\theta_d}}_{=:\gamma_d}\chi\left(\frac{1}{1-\theta_c} h_l-H\right).\label{eq:lab_inputd}
\end{align}
Substituting labour input in the sector-specific production functions, and using demand for sector goods, $Y_d=\left(\frac{p_c}{p_d}\right)^\varepsilon Y_c$, yields
\begin{align*}
p_c =p_d\left(\frac{\gamma_d}{\gamma_c}\frac{A_d}{A_c}\left(\frac{\frac{1}{1-\theta_c}h_l-H}{H+\frac{\theta_d}{1-\theta_d}h_l}\right)\right)^{\frac{1-\alpha}{\alpha+\varepsilon(1-\alpha)}}.
\end{align*}
\tr{Note that now policy can affect inflation/ relative prices through changes in labour supply---}
Together with the definition of the aggregate price level and the choice of $Y$ as numeraire determines sector-specific prices as a function of labour supply:
\begin{align*}
p_d= \left(1+\left(\frac{\gamma_d}{\gamma_c}\frac{A_d}{A_c}\left(\frac{\frac{1}{1-\theta_c}h_l-H}{H+\frac{\theta_d}{1-\theta_d}h_l}\right)\right)^{\frac{(1-\alpha)(1-\varepsilon)}{\alpha+\varepsilon(1-\alpha)}}\right)^{-\frac{1}{1-\varepsilon}}.
\end{align*}

\textbf{Below wrong}
From here,  equilibrium conditions determine prices $p_{dL}, p_{cL}$. Using \ref{eq:constant} skill wages follow. Together with the FOC on hours supply, wages determine aggregate demand. Imposing goods market clearing and using equations \ref{eq:lab_inputc} and \ref{eq:lab_inputd}, determines low skill hour demand in equilibrium.

\begin{align*}
h_l=\left( \frac{1}{\left(\frac{\alpha}{\psi}\right)^{\frac{\alpha}{1-\alpha}}\left[\left(p_c^\frac{\alpha}{1-\alpha}\chi_c A_c\right)^\frac{\varepsilon-1}{\varepsilon}+\left(p_d^\frac{\alpha}{1-\alpha}\chi_d A_d\right)^\frac{\varepsilon-1}{\varepsilon}\right]^\frac{\varepsilon}{\varepsilon-1}}\right)\ \lambda \left(H w_l\right)^{1-\tau_l}.
\end{align*}

Knowing $h_l$, the variables $L_c, \ L_d, \ h_h, \ l_{lc}, l_{ld}, l_{hc}, l_{hd}$ follow. 

Output of the clean and dirty sector read
\begin{align}
%Y_d& =  \frac{\chi_d A_d}{\left[\left(\left(\chi_d A_d\right)^\frac{\alpha}{\alpha+\varepsilon(1-\alpha)}\left(\chi_c A_c\right)^\frac{\varepsilon(1-\alpha)}{\alpha+\varepsilon(1-\alpha)}\right)^\frac{\varepsilon-1}{\varepsilon}+\left(\chi_d A_d\right)^\frac{\varepsilon-1}{\varepsilon}\right]^\frac{\varepsilon}{\varepsilon-1}} \lambda (H w_l)^{1-\tau_l}\\
&Y_d = \left(\frac{1}{\left(\frac{\chi_c A_c}{\chi_d A_d}\right)^{\frac{(\varepsilon-1)(1-\alpha)}{\alpha+\varepsilon(1-\alpha)}}+1}\right)^\frac{\varepsilon}{\varepsilon-1}\lambda (H w_l)^{1-\tau_l}\\
& Y_c= \left(\frac{1}{1+\left(\frac{\chi_d A_d}{\chi_c A_c}\right)^{\frac{(\varepsilon-1)(1-\alpha)}{\alpha+\varepsilon(1-\alpha)}}}\right)^\frac{\varepsilon}{\varepsilon-1}\lambda (H w_l)^{1-\tau_l}.
\end{align}

The government can affect dirty production by lowering aggregate demand. Note that $\chi_c,\ \chi_d$ are functions of the disutility from high skill labour supply, $\zeta$. As a result, the elasticity of diryt and clean output to tax progressivity is asymmetric.