\documentclass[12pt]{article}
\usepackage[utf8]{inputenc}
\usepackage{xcolor}
\usepackage{graphicx}
\usepackage{listings}
\usepackage{epstopdf}
\usepackage{etoc}
\usepackage{pdfpages}
\usepackage[capposition=top]{floatrow}
\usepackage{pdflscape} % landsacpe package
% set font to times
%\usepackage{mathptmx} % times!!! 
%\usepackage[T1]{fontenc}
\usepackage{amsmath}
\usepackage{soul}
\usepackage[left=2.5cm, right=2.5cm, top=2.5cm, bottom =2.5cm]{geometry}
\usepackage{natbib}
%\usepackage[natbibapa]{apacite}
%\usepackage{apacite}
%\bibliographystyle{apacite}
\bibliographystyle{apa}
%\renewcommand{\footnotesize}{\fontsize{10pt}{11pt}\selectfont}
\usepackage[onehalfspacing]{setspace}
\usepackage{listings}
\renewcommand{\figurename}{\textbf{Figure}}
\renewcommand{\hat}{\widehat}
\usepackage[bf]{caption}
\usepackage{tikz}
%\begin{comment}
%\usepackage[headsepline,footsepline]{scrlayer-scrpage} % has to come before package!!! otherwise option clash
%\usepackage{scrlayer-scrpage}
%\pagestyle{scrheadings} % kopfzeile/ fußzeile
%\clearpairofpagestyles
%\ohead{}
%\ihead{\textit{Redistribution, Demand and  Sustainable Production}}
%\cfoot{\thepage}
%\pagestyle{plain} % comment this one to have header
%\end{comment}
\usepackage{comment}
 \usepackage{siunitx}
  \usepackage{textcomp}
\definecolor{sonja}{cmyk}{0.9,0,0.3,0}
%\definecolor{purple}{model}{color-spec}
\usepackage{amssymb}
\newcommand{\ar}{$\Rightarrow$ \ }
\newcommand{\frp}[2]{\frac{\partial{#1}}{\partial{#2}}}
\newcommand{\tr}[1]{\textcolor{red}{#1}}
\newcommand{\vlt}[1]{\textcolor{violet}{#1}}
\newcommand{\bl}[1]{\textcolor{blue}{#1}}
\newcommand{\sn}[1]{\textcolor{sonja}{#1}}
%%% TIKZS
\usepackage{tikz}
\usetikzlibrary{tikzmark}
\usetikzlibrary{decorations.markings}
\usepackage{tikz-cd}
\usetikzlibrary{arrows,calc,fit}
\tikzset{mainbox/.style={draw=sonja, text=black, fill=white, ellipse, rounded corners, thick, node distance=5em, text width=8em, text centered, minimum height=3.5em}}
\tikzset{mainboxbig/.style={draw=sonja, text=black, fill=white, ellipse, rounded corners, thick, node distance=5em, text width=13em, text centered, minimum height=3.5em}}
\tikzset{dummybox/.style={draw=none, text=black , rectangle, rounded corners, thick, node distance=4em, text width=20em, text centered, minimum height=3.5em}}
\tikzset{box/.style={draw , rectangle, rounded corners, thick, node distance=7em, text width=8em, text centered, minimum height=3.5em}}
\tikzset{container/.style={draw, rectangle, dashed, inner sep=2em}}
\tikzset{line/.style={draw, very thick, -latex'}}
\tikzset{    pil/.style={
		->,
		thick,
		shorten <=2pt,
		shorten >=2pt,}}
	
% other stuff
\newcommand{\innermid}{\nonscript\;\delimsize\vert\nonscript\;}
\newcommand{\activatebar}{%
	\begingroup\lccode`\~=`\|
	\lowercase{\endgroup\let~}\innermid 
	\mathcode`|=\string"8000
}
%\usepackage{biblatex}
%\addbibresource{bib_mt.bib}
\usepackage{ulem}
\title{Inequality, Nature, and Limits to Sustainable Production}
\date{Sonja Dobkowitz\\ Bonn Graduate School of Economics\\ %University of Bonn\\
\vspace{1mm}
%Preliminary and incomplete\\
First version: November 27, 2021\\
This version: \today }
\usepackage{graphicx,caption}
\usepackage{hyperref}
\usepackage{minitoc}
\setcounter{secttocdepth}{5}
\usetikzlibrary{shapes.geometric}

% for tabular

%\usepackage{array}
\usepackage{makecell}
\usepackage{multirow}
\usepackage{bigdelim}

\renewenvironment{abstract}
{\small
	\list{}{
		\setlength{\leftmargin}{0.025\textwidth}%
		\setlength{\rightmargin}{\leftmargin}%
	}%
	\item\relax}
{\endlist}
\begin{document}
%	\includepdf[pages=-]{../titlepage.pdf}
	\maketitle
	
\section{Motivation}

Due to climate change and vital threats to biodiversity we need to reduce the consumption of resources. % why this?/ Resource consumption= to emissions??? 
Macroeconomic research has largely focused on a green \textit{recomposition} of consumption. However, doubt has been raised if a recomposition of production alone suffices to fight climate change. 
Arguments which question the pure recomposition approach are, for instance, the use of nature as a sink \citep{Dasgupta2021}, failure of a decoupling of production and resource usage despite a  shift to sustainable technologies in recent years (\textit{degrowth book}), rebound effects of demand, or the existence of limiting factors  and the regenerative capacity of the planet. 
Given the uncertainty about the possibilities of sustainable technologies to decouple resources and production, this paper draws attention to a world, in which a recomposition of production is not enough to comply with climate targets and planetary boundaries. 
The novelty is the introduction of an upper bound on the reduction of resource usage in the sustainable sector (\textit{hence there is some resource needed in the sustainable sector!}) in a macroeconomic analysis of environmental policies. (\textit{Parameter values such that with the upper bound on technology today's consumption levels (of the rich) cannot be met.}) 

HAVE TO READ AA PAPER AGAIN, HOW DO THEY MODEL THIS?

As a result, consumption growth will have to cease or become negative in order to respect planetary boundaries. (\textit{postgrowth! not degrowth with this motivation }) How much time remains for consumption growth to stop? One could argue not much due to uncertainty, irreversibility. What is the optimal policy in such a scenario, that is, a scenario where sustainable technology also exerts some externality, and the risk of a disaster/ irreversibility (tipping point) is already very high? 

What is the optimal policy? What is the socially acceptable policy? How do the two matter?
More specifically, I am interested in two aspects. First, the paper focuses on the political conflicts which may arise in such a scenario where a reduction of economic output is inevitable (\textit{Might already be present in other papers but just not discussed}). 
For example, what are the effects on distinct sources of income: the wage rate paid in green and non-green sectors, firm profits and dividends?  Due to a heterogeneous distribution of skills across sectors \cite{Bowen2018CharacterisingComposition, Consoli2016DoCapital}, low and high income households might be affected differently by the reduction in output. 
Second, the analysis draws attention to general equilibrium effects which might matter for the total effect an initial reduction in output has on the externality. 
A reduction in demand or labour, for instance, could well affect the incentives for R\&D. To capture the effect I integrate  endogenous, directed technolgical innovation following  \cite{Acemoglu2012TheChange} into the model. Furthermore, in an extension, I allow for income-dependent marginal propensities to consume so that aggregate demand for green products varies with the distribution of income. When income of the poor falls in reaction to a drop in consumption by the rich, for instance, they might revert to consume a higher share of an unsustainable yet cheaper good. This again mitigates the reduction in resource consumption. 


\clearpage
\bibliography{../../bib_2_0}
\addcontentsline{toc}{section}{References}
\end{document}