\documentclass[12pt]{article}
\usepackage[utf8]{inputenc}
\usepackage{xcolor}
\usepackage{graphicx}
\usepackage{listings}
\usepackage{epstopdf}
\usepackage{etoc}
\usepackage{pdfpages}
\usepackage[capposition=top]{floatrow}
\usepackage{pdflscape} % landsacpe package
% set font to times
%\usepackage{mathptmx} % times!!! 
%\usepackage[T1]{fontenc}
\usepackage{amsmath}
\usepackage{soul}
\usepackage[left=2.5cm, right=2.5cm, top=2.5cm, bottom =2.5cm]{geometry}
\usepackage{natbib}
%\usepackage[natbibapa]{apacite}
%\usepackage{apacite}
%\bibliographystyle{apacite}
\bibliographystyle{apa}
%\renewcommand{\footnotesize}{\fontsize{10pt}{11pt}\selectfont}
\usepackage[onehalfspacing]{setspace}
\usepackage{listings}
\renewcommand{\figurename}{\textbf{Figure}}
\renewcommand{\hat}{\widehat}
\usepackage[bf]{caption}
\usepackage{tikz}
%\begin{comment}
%\usepackage[headsepline,footsepline]{scrlayer-scrpage} % has to come before package!!! otherwise option clash
%\usepackage{scrlayer-scrpage}
%\pagestyle{scrheadings} % kopfzeile/ fußzeile
%\clearpairofpagestyles
%\ohead{}
%\ihead{\textit{Redistribution, Demand and  Sustainable Production}}
%\cfoot{\thepage}
%\pagestyle{plain} % comment this one to have header
%\end{comment}
\usepackage{comment}
 \usepackage{siunitx}
  \usepackage{textcomp}
\definecolor{sonja}{cmyk}{0.9,0,0.3,0}
%\definecolor{purple}{model}{color-spec}
\usepackage{amssymb}
\newcommand{\ar}{$\Rightarrow$ \ }
\newcommand{\frp}[2]{\frac{\partial{#1}}{\partial{#2}}}
\newcommand{\tr}[1]{\textcolor{red}{#1}}
\newcommand{\vlt}[1]{\textcolor{violet}{#1}}
\newcommand{\bl}[1]{\textcolor{blue}{#1}}
\newcommand{\sn}[1]{\textcolor{sonja}{#1}}
%%% TIKZS
\usepackage{tikz}
\usetikzlibrary{tikzmark}
\usetikzlibrary{decorations.markings}
\usepackage{tikz-cd}
\usetikzlibrary{arrows,calc,fit}
\tikzset{mainbox/.style={draw=sonja, text=black, fill=white, ellipse, rounded corners, thick, node distance=5em, text width=8em, text centered, minimum height=3.5em}}
\tikzset{mainboxbig/.style={draw=sonja, text=black, fill=white, ellipse, rounded corners, thick, node distance=5em, text width=13em, text centered, minimum height=3.5em}}
\tikzset{dummybox/.style={draw=none, text=black , rectangle, rounded corners, thick, node distance=4em, text width=20em, text centered, minimum height=3.5em}}
\tikzset{box/.style={draw , rectangle, rounded corners, thick, node distance=7em, text width=8em, text centered, minimum height=3.5em}}
\tikzset{container/.style={draw, rectangle, dashed, inner sep=2em}}
\tikzset{line/.style={draw, very thick, -latex'}}
\tikzset{    pil/.style={
		->,
		thick,
		shorten <=2pt,
		shorten >=2pt,}}
	
% other stuff
\newcommand{\innermid}{\nonscript\;\delimsize\vert\nonscript\;}
\newcommand{\activatebar}{%
	\begingroup\lccode`\~=`\|
	\lowercase{\endgroup\let~}\innermid 
	\mathcode`|=\string"8000
}
%\usepackage{biblatex}
%\addbibresource{bib_mt.bib}
\usepackage{ulem}
\title{Growth, the Environment, and Inequality\\ \small{the political economy of environmental policies}}
\date{Sonja Dobkowitz\\ Bonn Graduate School of Economics\\ %University of Bonn\\
\vspace{1mm}
%Preliminary and incomplete\\
First version: November 27, 2021\\
This version: \today }
\usepackage{graphicx,caption}
\usepackage{hyperref}
\usepackage{minitoc}
\setcounter{secttocdepth}{5}
\usetikzlibrary{shapes.geometric}

% for tabular

%\usepackage{array}
\usepackage{makecell}
\usepackage{multirow}
\usepackage{bigdelim}

\renewenvironment{abstract}
{\small
	\list{}{
		\setlength{\leftmargin}{0.025\textwidth}%
		\setlength{\rightmargin}{\leftmargin}%
	}%
	\item\relax}
{\endlist}
\begin{document}
%	\includepdf[pages=-]{../titlepage.pdf}
	\maketitle
	
\section{Motivation}

Due to climate change and vital threats to biodiversity we need to reduce the consumption of resources. % why this?/ Resource consumption= to emissions??? 
Macroeconomic research has largely focused on a green \textit{recomposition} of consumption. However, doubt has been raised if a recomposition of production alone suffices to fight climate change. 
Arguments which question the pure recomposition approach are, for instance, the use of nature as a sink \citep{Dasgupta2021}, failure of a decoupling of production and resource usage despite a  shift to sustainable technologies in recent years (\textit{degrowth book}), rebound effects of demand, or the existence of limiting factors  and the regenerative capacity of the planet. 
Given the uncertainty about the possibilities of sustainable technologies to decouple resources and production, this paper draws attention to a world, in which a recomposition of production is not enough to comply with climate targets and planetary boundaries. 
The novelty is the introduction of an upper bound on the reduction of resource usage in the sustainable sector (\textit{hence there is some resource needed in the sustainable sector!}) in a macroeconomic analysis of environmental policies. (\textit{Parameter values such that with the upper bound on technology today's consumption levels (of the rich) cannot be met.}) 

HAVE TO READ AA PAPER AGAIN, HOW DO THEY MODEL THIS?: Disaster and risk of non-existence \ar There is a disaster in their paper, and a point of no return. However, pollution does not harm productivity, 

As a result, consumption growth will have to cease or become negative in order to respect planetary boundaries. (\textit{postgrowth! not degrowth with this motivation }) How much time remains for consumption growth to stop? One could argue not much due to uncertainty, irreversibility. What is the optimal policy in such a scenario, that is, a scenario where sustainable technology also exerts some externality, and the risk of a disaster/ irreversibility (tipping point) is already very high (assumption of where we are relative to tipping point)? 

\textbf{What is the optimal policy? (With a focus on avoiding environmental disaster, when is a reduction in consumption optimal? BCS counteracting effects on the externality! On the one hand, bigger market size means more innovation and especially if the rich, high-skilled provide more labour! On the other hand, a direct effect on resource consumption.  if a reduction in habits implies a relative rise in sustainable consumption, then there could be  an acceleration of sustainable innovation. Second step, what are the effects on inequality (profits, wages)? Would such an environmentally optimal policy be socially acceptable? Is it implementable? What would change given inequality in the objective function?)
}

Evaluating the results, I give special attention on the effect on consumption and labour growth. Is it the case that as advocated by proponents of a reduction approach that consumption falls under the optimal environmental policy? And if so, who bears the costs?

What is the socially acceptable policy? How do the two matter?
More specifically, I am interested in two aspects. First, the paper focuses on the political conflicts which may arise in such a scenario where a reduction of economic output is inevitable (\textit{Might already be present in other papers but just not discussed}). 
For example, what are the effects on distinct sources of income: the wage rate paid in green and non-green sectors, firm profits and dividends?  Due to a heterogeneous distribution of skills across sectors \cite{Bowen2018CharacterisingComposition, Consoli2016DoCapital}, low and high income households might be affected differently by the reduction in output. 
Second, the analysis draws attention to general equilibrium effects which might matter for the total effect an initial reduction in output has on the externality. 
A reduction in demand or labour, for instance, could well affect the incentives for R\&D. To capture the effect I integrate  endogenous, directed technolgical innovation following  \cite{Acemoglu2012TheChange} into the model. Furthermore, in an extension, I allow for income-dependent marginal propensities to consume so that aggregate demand for green products varies with the distribution of income. When income of the poor falls in reaction to a drop in consumption by the rich, for instance, they might revert to consume a higher share of an unsustainable yet cheaper good. This again mitigates the reduction in resource consumption. 

\section{Plan for the paper: The quantitative experiment}
\begin{enumerate}
\item Calculate the optimal policy in a representative agent model (should be the same as for some politician?)\ar gives the optimal environmental policy absent inequality\\
\ar trade-off: reduction might have a negative effect on recomposition


\item A reduction in consumption growth/ employment is politically delicate \ar look at political economy of such a policy. \\
1) What effects does the policy resulting in bullet point above have on inequality?
2) What would a political government choose given inequality? 
\ar study the effects of a reduction on inequality

\item Plug in optimal environmental policy in a model with inequality (two households) 
\item Here analyse: how does inequality change the effect of the optimal environmental policy on the externality?; What are the consequences for different income groups?
\item In full model: Ramsey planner and Politicians (democratic politician bases policy on opinion) 
\end{enumerate}

\section{Model}
\subsection{Preferences}
In the baseline version of the model, there are standard preferences. Households care about the absolute amount of consumption and utility is monotonically increasing in consumption. 
In an extension: 
High consumption levels feature prominently in the literature suggesting a reduction of consumption for environmental reasons. In the economics literature, overconsumption has been shown to result from habits or social preferences. This paper places the environmental policy analysis in a model with habits/ social preferences as these seem to be the reason for high consumption levels, so that they should be taken into account in an analysis of potential reductions in consumption. 

\subsection{Inequality}

\noindent Income heterogeneity is introduced by
\begin{itemize}
\item skill heterogeneity
\item firm ownership
\item investment into research
\end{itemize}

\subsection{Pollution}
Build on \cite{Acemoglu2012TheChange} but make the sustainable sector also exert pollution but less (motivate by production of electric vehicles, solar panels, infrastructure for railways) or leading to waste.  
\subsection{Government}
\subsubsection{Objective function}
Political considerations and societal acceptance of environmental policies is a crucial aspect for environmental protection to succeed. 
In the second part of the paper, I, therefore, consider a government which searches to increases its chances on reelection. 
Three alternative ways to set up the government's objective function
\begin{enumerate}
	\item aggregation of utilities according to median voter?
	\item assuming the gov does not know utilities, it could search to maximise value measures such as employment rates, environment with weights that follow polls, opinions in society (use WVS data)\\ \ar How important is a change in electorate values towards environmental priority to prevent environmental disaster?\\
	Endogenise values relevant for votes \ar economic downturn \ar surge of prioritising economic growth, for instance
\end{enumerate}

\subsubsection{What policies can the government choose from?}
How is working time reduction in \cite{Schor2005SustainableReduction} implemented?
What structural biases does she see on the firm level? i.e., why do firms have an advantage to keep working hours per worker high?
\textit{Extension:} Model jobs posted as having a fixed number of hours.

Tax policies which might imply a reduction in consumption over time.

\subsection{The environment and planetary boundaries}
\clearpage
\bibliography{../../bib_2_0}
\addcontentsline{toc}{section}{References}
\end{document}