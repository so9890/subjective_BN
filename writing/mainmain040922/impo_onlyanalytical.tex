\documentclass[12pt]{article}
\usepackage[utf8]{inputenc}
\usepackage{xcolor}
\usepackage{graphicx}
\usepackage{listings}
\usepackage{epstopdf}
\usepackage{etoc}
\usepackage{pdfpages}
\usepackage[capposition=top]{floatrow}
\usepackage{pdflscape} % landsacpe package
% set font to times
%\usepackage{mathptmx} % times!!! 
%\usepackage[T1]{fontenc}
\usepackage{amsmath}
\usepackage{amsthm}
\usepackage{soul}
\usepackage[left=2.5cm, right=2.5cm, top=2.5cm, bottom =2.5cm]{geometry}
\usepackage{natbib}
%\usepackage[natbibapa]{apacite}
%\usepackage{apacite}
%\bibliographystyle{apacite}
\bibliographystyle{apa}
%\renewcommand{\footnotesize}{\fontsize{10pt}{11pt}\selectfont}
\usepackage[onehalfspacing]{setspace}
\usepackage{listings}
\renewcommand{\figurename}{\textbf{Figure}}
\renewcommand{\hat}{\widehat}
\usepackage[bf]{caption}
\usepackage{tikz}
%\begin{comment}
%\usepackage[headsepline,footsepline]{scrlayer-scrpage} % has to come before package!!! otherwise option clash
%\usepackage{scrlayer-scrpage}
%\pagestyle{scrheadings} % kopfzeile/ fußzeile
%\clearpairofpagestyles
%\ohead{}
%\ihead{\textit{Redistribution, Demand and  Sustainable Production}}
%\cfoot{\thepage}
%\pagestyle{plain} % comment this one to have header
%\end{comment}
\allowdisplaybreaks
\usepackage{comment}
\usepackage{siunitx}
\usepackage{textcomp}
\definecolor{sonja}{cmyk}{0.9,0,0.3,0}
%\definecolor{purple}{model}{color-spec}
\usepackage{amssymb}
\newcommand{\ar}{$\Rightarrow$ \ }
\newcommand{\frp}[2]{\frac{\partial{#1}}{\partial{#2}}}
\newcommand{\tr}[1]{\textcolor{red}{#1}}
\newcommand{\vlt}[1]{\textcolor{violet}{#1}}
\newcommand{\bl}[1]{\textcolor{blue}{#1}}
\newcommand{\sn}[1]{\textcolor{sonja}{#1}}
%%% TIKZS
\usepackage{tikz}
\usetikzlibrary{mindmap,trees}
\usetikzlibrary{backgrounds}
\tikzstyle{every edge}=  [fill=orange]  
\usetikzlibrary{tikzmark}
\usetikzlibrary{decorations.markings}
\usepackage{tikz-cd}
\usetikzlibrary{arrows,calc,fit}
\tikzset{mainbox/.style={draw=sonja, text=black, fill=white, ellipse, rounded corners, thick, node distance=5em, text width=8em, text centered, minimum height=3.5em}}
\tikzset{mainboxbig/.style={draw=sonja, text=black, fill=white, ellipse, rounded corners, thick, node distance=5em, text width=13em, text centered, minimum height=3.5em}}
\tikzset{dummybox/.style={draw=none, text=black , rectangle, rounded corners, thick, node distance=4em, text width=20em, text centered, minimum height=3.5em}}
\tikzset{box/.style={draw , rectangle, rounded corners, thick, node distance=7em, text width=8em, text centered, minimum height=3.5em}}
\tikzset{container/.style={draw, rectangle, dashed, inner sep=2em}}
\tikzset{line/.style={draw, very thick, -latex'}}
\tikzset{    pil/.style={
		->,
		thick,
		shorten <=2pt,
		shorten >=2pt,}}

% other stuff
\newcommand{\innermid}{\nonscript\;\delimsize\vert\nonscript\;}
\newcommand{\activatebar}{%
	\begingroup\lccode`\~=`\|
	\lowercase{\endgroup\let~}\innermid 
	\mathcode`|=\string"8000
}
%\usepackage{biblatex}
%\addbibresource{bib_mt.bib}
\usepackage{ulem}
\title{On the Role of Fiscal Policies in the Optimal Environmental Policy}
%\title{The Environment, Inequality, and Growth\\ \small{ optimal fiscal policy in an endogenous growth model with inequality and emission targets}}
\date{Sonja Dobkowitz\\ Bonn Graduate School of Economics\\ %University of Bonn\\
	\vspace{1mm}
	%Preliminary and incomplete\\
	%First version: January 9, 2022\\
	This version: \today }
\usepackage{graphicx,caption}
%\usepackage{hyperref}
\usepackage[colorlinks,linkcolor=aaltoblue,citecolor=aaltoblue,urlcolor=aaltoblue,unicode=true]{hyperref} %can create hyperlinks. ALWAYS LOAD LAST
\definecolor{aaltoblue}{RGB}{0,94,184}
\usepackage{minitoc}
\setcounter{secttocdepth}{5}
\usetikzlibrary{shapes.geometric}

% for tabular

%\usepackage{array}
\usepackage{makecell}
\usepackage{multirow}
\usepackage{bigdelim}

%propositions etc
\newtheorem{prop}{Proposition}
\newtheorem{corollary}{Corollary}[prop]
\newtheorem{lemma}[prop]{Lemma}

\renewenvironment{abstract}
{\small
	\list{}{
		\setlength{\leftmargin}{0.025\textwidth}%
		\setlength{\rightmargin}{\leftmargin}%
	}%
	\item\relax}
{\endlist}
\begin{document}
	%	\includepdf[pages=-]{../titlepage.pdf}
	\maketitle
	\begin{abstract}
		\begin{singlespacing}
			\textbf{Abstract \ }
			Some scholars call for reductive policies to handle tightening environmental limits. I show that, indeed, when labor supply is elastic, 
			the optimal policy consists of both a recomposing and a reductive element: an environmental tax does not suffice to implement the efficient allocation. Lump-sum transfers complement environmental taxes by reducing labor supply through an income effect.
			When environmental tax revenues are not redistributed lump sum, the planner uses progressive income taxes to decrease work effort.
			Therefore, the optimal environmental policy - as a byproduct - increases equity either through lump-sum transfers or progressive income taxation.
			I quantify the optimal income tax in an endogenous growth model with skill heterogeneity. The optimal income tax is progressive even though this reduces growth and increases the share of fossil to green energy usage due to a skill bias in the green sector.
			%	The model suggests that the use of income taxes when environmental tax revenues are not redistributed raises social welfare by 0.1\% over the 60 years from 2020 to 2080.
			%The model suggests that integration of environmental and fiscal policy when no lump-sum transfers are available raises social welfare by 0.1\% over the 60 years from 2020 to 2080.
			%			To reach climate targets, the International Panel on Climate Change has identified net-zero emissions by 2050 an essential element. I show that progressive income taxes are optimally used in concert with corrective taxes to satisfy an absolute emission limit.
			%			On the one hand, progressive income taxation reduces labour efforts as leisure becomes relatively cheaper. The overall reduction in production lowers emissions. On the other hand, a progressive income tax (i) reduces general research efforts through a market size effect  and (ii) recomposes the structure of the economy away from green energy through a skill-supply channel. Both effects call for a more regressive tax to foster a green transition. For a reasonable calibration, I find that the reduction effect dominates and the optimal income tax schedule is progressive. The welfare advantage of income tax progressivity arises from a reduction of inefficiently high hours worked. 
			%The model suggests that including progressive income taxes as a tool to lower emissions accounts for a rise in social welfare by 0.1\% over the 60 years from 2020 to 2080.  % as relative supply of the through a skill-supply channel. As richer, high-skilled workers reduce their labour supply more in response to a more progressive tax, low-skill labour becomes more abundant. 
			
			%			Natural scientists have identified the reduction of demand %for energy and land %thus, a change in lifestyle 
			%			as an important contributor to meeting global climate targets. However, a general equilibrium analysis of reduction policies is missing.
			%		%	I study the general equilibrium effects of reduction policies: such as income taxes, a restriction of hours worked, or consumption taxes. 
			%		A higher labour income tax progressivity can achieve such a reduction as it lowers labour supply. 
			%		Then again, tax progressivity alters the relative skill supply. As the green sector is relatively more skill-biased, the economy recomposes production towards the dirty sector. What is the optimal policy when the government has to meet an exogenous emission and demand target?
			%This additional benefit of income taxes changes the  equity-efficiency trade-off which classically determines optimal fiscal policies. 
			%To answer the question, I build a model of directed technical change and skill heterogeneity.
			
			
			%In a set up with representative agent, the necessity to meet emission targets makes the optimal income tax highly progressive. The more goods are substitutes the higher the optimal tax progressivity. When goods are complements, the more slowly growing clean sector dampens production in the dirty sector making a lower progressivity sufficient. 
		%	\noindent \textit{JEL classification}: E71, H21, H23,  O11, O13, Q58
			
		\end{singlespacing}
		
	\end{abstract}
	%\tableofcontents
	%\section{Introduction}

\begin{quote}
"Mitigation pathways limiting warming to 1.5°C [...]  reduce emissions further to reach net zero $CO_2$ emissions in the 2050s."
\end{quote}

The latest Reports of the International Panel on Climate Change highlights the importance of an absolute emission target by the 2050s to comply with the Paris Agreement on limiting temperature rise to 1.5°C. 
The economics literature on environmental policy has by and large allowed for a (limited) trade-off between consumption and pollution \citep{Barrage2019OptimalPolicy, Golosov2014OptimalEquilibrium} or studied relative emission targets \citep{Fried2018ClimateAnalysis}. 
However, the presence of an absolute emission target poses a limit to growth in fossil energy usage.
Depending on the substitutability of green and fossil energy and the velocity of the green sector to grow, the absolute emission target may, first, pose a limit to consumption growth and, second, make untraditional policy measures in addition to corrective taxes optimal.\textit{ (WHY THIS DIFFERENCE? Also with externalities in consumption pollution cannot be compensated for by consumption as the marginal utility of consumption reduces.  )} 

For a reasonably calibrated endogenous growth model, I find that the optimal labour income tax is progressive when accounting for an absolute emission target. This finding highlights the importance of policy measures targeted at a \textit{reduction} of production in tandem with policies intended at a \textit{recompostion} of the economic structure such as carbon taxes to mitigate climate change. % Then again, I present data indicating a voluntary reduction in household consumption. Given this behavioural change, the optimal income tax progressivity could become regressive in order to boost high-skill labour supply. 

%MODEL
To investigate the effect of an exogenous emission target on the optimal policy, I study an endogenous growth model building on \cite{Fried2018ClimateAnalysis}. Allowing for endogenous growth is important to take seriously the possibility of green growth to keep consumption high while meeting emission targets. The government is characterised as a Ramsey planner who seeks to maximise Utilitarian social welfare but is constrained by an exogenous emission target. To abstract from inequality as a determinant of tax progressivity, the economy is populated by a representative family. Yet, the family supplies two types of skill. 

The model differentiates between high- and low-skilled labour to account for a skill bias found for the green sector \citep{Consoli2016DoCapital}. This asymmetry of sectors renders regressive taxes a tool to lower production costs in the green sector. As high-skill workers reduce their labour supply more in response to a more progressive tax due to a relatively lower value of an additional unit of income, a regressive tax functions as a green subsidy. % In fact, there is an externality arising from high-skill labour supply as it shapes the share of fossil to green energy production. 
On the other hand, labour income taxation lowers aggregate production as it renders leisure cheaper to households. 
Together, the recomposing and the reductive channel shape the optimal tax progressivity from an environmental policy perspective. 

In an extension to the baseline model I depart from the representative agent assumption and explicitly model household heterogeneity. This setting allows to capture a change in household behaviour: A share of households is willing to voluntarily reduce consumption. I provide evidence for such behaviour using a representative Dutch dataset. More than 50\% of households are willing to reduce consumption in order to help the economy. Importantly, these households have a higher likelihood to work in the green sector. How does such a change in behaviour affect the optimal policy? Given the additional reduction in green-specific labour supply, the planner might find it optimal to set a more regressive tax.    

%Calibration
The Calibration of the model proceeds in two steps. First, I set certain parameters to values found in the literature. Most impportantly, I use reasonable value of production and growth processes as found in \cite{Fried2018ClimateAnalysis} who conducts a rigorous calibration exercise.  With these 

% Quantitative Experiment and Results
The main finding is the optimality of progressive taxes in order to reach emission targets. 


To scrutinise the reasons for the main finding. I conduct several additional quantitative experiments. First, I reduce the size of the emission target, second, I allow for a longer time frame until net-zero emissions have to be reached. The IPCC report states that for a temperature target of 2°C net-zero emissions have to be reached by 2070 only. How does this laxer target affect the importance of labour income taxes. Given the wider time frame, the green sector might be able to catch up and growth could continue. \textit{(Question: I guess that substitutability is key here! Growth in green implies growths in fossil when goods are no perfect substitutes! )}
	%\section{Introduction}
%\tr{I show that carbon taxes are only efficient if lump-sum transfers are available.}

\begin{comment}
\tr{Think about:
	1) when labor income taxes are not used, then need to have  a higher environmental tax to meet emission limits? \ar Yes, because of advantageous level effect which outweighs recomposing effect of income tax.
	2) When staying at level optimal under the assumption of lump-sum redistribution, but then not redistributing, than absent labor income tax emissions are too high; by how much? Counterfactual}
	
	content...
	\end{comment}
%\paragraph{Recomposition versus reduction discussion plus ever stricter emission limits}
The latest assessment report of the Intergovernmental Panel on Climate Change (IPCC) \citep{IPCC2022} highlights the urgency to reduce greenhouse-gas emissions.%relative to the previous report from 2018 \citep{Rogelj2018MitigationDevelopment.}.
\footnote{ \  The report stresses the decreasing likelihood of meeting the Paris Agreement and limiting climate warming to 1.5°. The Paris Agreement of 2015 formulates clear political goals to mitigate climate change. Under this treaty, states have agreed on a legally binding maximum increase in temperature to well below 2°C, preferably 1.5° over pre-industrial levels, and the global community seeks to be climate-neutral in 2050  (compare: \url{https://unfccc.int/process-and-meetings/the-paris-agreement/the-paris-agreement}). 
}
On the other hand, scholars have pointed to reductive policy measures to handle environmental limits \citep{Arrow2004AreMuch, Schor2005SustainableReduction, Dasgupta2021}. A reduction in work effort and consumption mitigates pollution by diminishing economic activity. Such a reduction could be achieved by using distortionary fiscal policy tools.
However, the economic literature on environmental policy has focused on the recomposing aspect of environmental policies: environmental taxes. %\citep{Fried2018ClimateAnalysis}. 
Given the exigency to act, this paper addresses the question whether fiscal policies can help meet climate targets.

I show that, indeed, once 
labor supply is elastic, reductive policy measures optimally complement the environmental tax. 
It is established in the literature that absent any other distortion, an environmental tax equal to the social cost of the externality implements the efficient allocation. I argue, first, that this result crucially depends on the use of lump-sum transfers to redistribute environmental tax revenues; otherwise work effort is inefficiently high.
%\textcolor{blue}{This is interesting independent of whether they are feasible or not. Could relate to the fact that there is a discussion how to use revenues. Yet, one might argue that we are always in a setting with distortionary labor income taxes; so that recycling lump-sum is never needed; numbers on size of expected revenues and government spending}
 When, second, environmental tax revenues are not redistributed lump-sum, then  environmental taxes are optimally combined with progressive labor income taxes. However, the use of income taxes is not directly targeted at the externality: the motive for labor taxation follows rather indirectly from a distortion in labor markets induced by the environmental policy. Hence, (i) the two tax instruments are complements, and % to lower inefficiently high hours worked. 
% I show that redistributing environmental tax revenues through an income tax scheme allows to implement the efficient allocation. The optimal income tax scheme is progressive.
(ii) the optimal environmental policy equalizes the distribution of income as  a side effect. The theoretic analysis forms the first part of the paper.

In the second part, I scrutinize whether progressive income taxes remain optimal in a more realistic quantitative model with endogenous growth and heterogeneous skills \textbf{when environmental tax revenues are not redistributed. 
}This is unclear a-priori since a progressive tax scheme reduces incentives to innovate through a market size effect. Furthermore, a skill bias documented for the green sector \citep{Consoli2016DoCapital} in combination with a relatively more elastic high-skilled labor causes a higher tax progressivity to recompose economic structure towards dirty production. The model suggests that despite these countering mechanisms the optimal income tax scheme is progressive. To lower hours worked the government forfeits growth and accepts a less green production ratio.  I quantify the welfare gains of setting progressive income taxes to equal yyy in consumption equivalent measure.  
% take model with no redistribution as benchmark

The paper's results are relevant for the political and academic debate on how best to use environmental tax revenues. The paper points to the importance of lump-sum transfers as a reductive policy tool in the optimal environmental policy; an aspect which appears overlooked in today's discussion.\footnote{\ POLICY debate; \cite{Fried2018TheGenerations}}
When thinking about how to recycle environmental tax revenues other than as lump-sum transfer, then, one should also think about alternative reductive tools such as progressive labor income taxes. 
If the reductive part of the environmental policy is neglected, environmental taxes have to be higher to meet emission limits, as I demonstrate in the quantitative exercise.

The results address the academic debate on the so-called \textit{weak double-dividend} \citep[for example:][]{LansBovenberg1994EnvironmentalTaxation, LansBovenberg1996OptimalAnalyses}. The hypothesis posits that recycling environmental tax revenues to reduce pre-existing tax distortions is advantageous to recycling  revenues as lump-sum transfers. The rationale is that transfers decrease labor supply thereby diminishing the tax base of the income tax. A conflict between generating government funds and environmental protection arises. The findings in the present paper suggest a lower bound on the reduction in distortionary income taxes: when environmental tax revenues are not redistributed lump-sum, some reduction in labor supply via distortionary income taxes is in fact efficient from an environmental policy perspective. In other words, even if environmental tax revenues suffice to satisfy an government revenue requirement, the optimal labor income tax is progressive.  %\footnote{\ The set-up in this paper can be integrated in the  of as a situation How the trade-off characterizing the optimal level of work effort plays out when the government seeks to generate funds and to mitigate an externality while environmental tax revenues suffice to satisfy the revenue requirement  is left for further research. Clearly: Then the optimal income tax is progressive. }  %\tr{Think about that labor income tax revenues are redistributed back to households but they would not under the weak double dividend hypossis} 

\begin{comment}
When labor supply is fixed, environmental taxes alone can establish the efficient allocation in a representative agent economy absent fiscal distortions. Then, such a tax instrument is optimally set to the social cost of an externality, and originators internalize these social costs: the Pigou principle.
However, not redistributing environmental tax revenues reduces consumption below the efficient level and, as I demonstrate, the optimal environmental tax does not follow the Pigou principle.  If, on top, the  labor supply decision is endogenous, the environmental tax alone features too high labor supply. \tr{This results in too high environmental externality. \textbf{To be shown!}}

Lump-sum transfers of environmental tax revenues restore the efficient allocation: as households become richer, labor supply reduces. When lump-sum transfers are not available, the government can establish the efficient allocation by redistributing environmental tax revenues through an income tax scheme which I demonstrate to be progressive.

content...
\end{comment}

%%%%%%%%%%%%%%%%%%%%%%%%%%%%%%%%%%%%%%%
%\paragraph{Answer WHAT I DO }
%%%%%%%%%%%%%%%%%%%%%%%%%%%%%%%%%%%%%%%
\subsubsection*{First part: Analytic stuff}
\paragraph{simple model}
I propose a simple and comparatively general model to derive the main theoretical results. There are two intermediate sectors of production, one of which exerts a negative environmental externality. The environmental externality is the only distortion motivating government action. In the model, a Ramsey planner seeks to maximize welfare of a representative agent having  an environmental and a labor income tax at its disposal. The income tax scheme is generally non-linear and a common specification in the public finance literature \citep[e.g.][]{Benabou2002TaxEfficiency, Heathcote2017OptimalFramework}.
The model abstracts from  endogenous growth, inequality, and an exogenous government funding constraint. The last two are important abstractions since they traditionally motivate income taxation.

\paragraph{Analytic findings 1}
\textbf{MAIN finding: Progressive income tax} I show that, under mild assumptions, the optimal labor income tax is progressive when no lump-sum transfers are available. 
The optimality of progressive income taxes results from inefficiently high labor supply. The mechanism runs as follows: the use of environmental taxes induces an additional distortion on the labor market by driving a wedge between households' shadow value of income and the social one. The environmental tax reduces the returns to labor below its marginal product. When environmental tax revenues are not redistributed lump-sum, labor supply is too high. 
This is a novel motive for income taxation so far overlooked in the literature.
Following this intuition, environmental and labor income taxes complement each other in the optimal environmental policy.
% I argue that environmental and income taxes are complements. The first is targeted at the environmental externality while the second serves to mitigate distortions in the labor market resulting from environmental taxation.
  
\paragraph{Analytic findings 2}
\textbf{Pigou principle violated absent lump-sum transfers and efficient allocation not feasible}
I consider two cases how environmental tax revenues are recycled. First, revenues are consumed by the government, and, second, the government redistributes revenues via the income tax scheme. In the first scenario, I find that the optimal environmental tax does not satisfy the Pigou principle, i.e., it does not equal the social cost of the externality. As demonstrated by \textit{Pigou xxx}, setting the environmental tax equal to the marginal costs arising from the polluting activity that are not private (the so-called social cost) is optimal. The idea is that polluters behave as if internalizing the social cost of their action when confronted with the environmental tax. I show that when environmental tax revenues are consumed by the government, the Pigou principle is violated. In this case, the motive to increase private consumption by lowering environmental tax revenues makes a deviation of the environmental tax from the social cost of the externality optimal.
Furthermore, when labor supply is elastic non-redistribution of environmental tax revenues results in inefficiently high hours worked. Then, the optimal labor tax is progressive to diminish work effort closer to the efficient level.
Nevertheless, the efficient allocation is not feasible in this policy framework since either consumption is inefficiently low or work effort is too high. 


\paragraph{Analytic findings 3}
\textbf{Efficient allocation can be implemented through redistribution via income tax scheme}
In the second scenario, the Ramsey planner redistributes environmental tax revenues through the income tax scheme, now the efficient allocation is attainable. Thus, even absent lump-sum transfers, there exists a possibility to implement the efficient allocation. By redistributing environmental tax revenues, consumption can be chosen efficiently high while the labor income tax scheme handles too high labor supply. In fact, redistribution via the income tax scheme incentivizes more labor supply. A progressive income tax scheme counteracts this tendency restoring the efficient level of hours worked. 
In this setting, the Pigou principle holds since there is no trade-off between lowering the externality and the allocation of consumption.
%More precisely, I study the optimal policy mix of environmental and labor income taxes to meet emission limits. 
%I find that progressive labor income taxes are used in concert with fossil taxes to optimally reduce emissions. 
% First, I propose a tractable model to provide intuition for this result: Non-lump-sum redistribution of environmental tax revenues increase the gains from labor. Leisure becomes more expensive and households do not reduce their labor supply efficiently in response to the fossil tax. To reduce hours to the efficient level, income taxes complement the fossil tax to lower hours to the efficient level.  
% 
% 
% 
%Second, I assess the importance of this novel role for income taxes in a quantitative endogenous growth model:
%Even though a more progressive tax reduces research effort  and recomposes production towards the fossil sector, the optimal tax is progressive. 


\subsubsection*{Quantitative exercise}
In this and the subsequent three paragraphs, I motivate the quantitative exercise and expound the model.
Even though I have shown theoretically that progressive income taxes form one pillar of the optimal environmental policy when no lump-sum transfers are available, it is not clear whether progressivity remains optimal in a more realistic, quantitative model. 
Two countering mechanisms of tax progressivity shall be considered here.
First, endogenous growth could make a more regressive income tax optimal to incentivize innovation. The size of the market for successful innovation positively affects the profitability of research investment. Subsidizing labor supply through regressive taxes then boosts technological growth. Second, a skill bias in the green sector also renders regressive income taxes advantageous by enhancing supply of the green sector-specific input good. In the model, high-skill workers supply more labor due to a wage premium. They are more responsive to the substitution effect as income tax progressivity makes leisure cheaper. Then, the economy transitions to a higher dirty production share. Directed technical change amplifies this recomposing effect of the income tax. 

% relation to core model and to fried;
In light of these two mechanisms calling for income tax regressivity, I extend the core model studied in the analytical section with endogenous growth and skill heterogeneity. The resulting model builds on \cite{Fried2018ClimateAnalysis}. It extends aforementioned work by an optimal policy analysis, the availability of income taxes, and skill heterogeneity. 
% no income inequality: trade off as in classical 
I maintain the assumption of a representative family which consists of two skill types. Within the family, all workers consume the same. Thus, workers remain perfectly insured against income differentials so that equity concerns do not drive the optimal policy. The trade-off shaping the optimal policy stays within the boundaries of traditional environmental policy considerations: growth versus externality mitigation \citep{Stokey1998AreGrowth, Jones2016LifeGrowth, Acemoglu2012TheChange}.  

%\paragraph{How the externality is modeled}
In the quantitative model, the government cares about the externality due to an exogenous constraint on emissions and not via household utility. This approach, known as a cost-effectiveness approach in the environmental literature, has the advantage of reducing modeling and parametric uncertainty when specifying how greenhouse-gas emissions affect the climate and the damages it produces in terms of well-being and production. Instead, the cost-effectiveness approach uses estimated emission targets stemming from meta studies of more complex integrated assessment models. Additionally, the emission limits I use to calibrate the model are designed to meet climate targets constituting a politically relevant environment. 
% think about the relation of an absolute emission target and not reducing hours

%\paragraph{Endogenous growth}
Allowing for endogenous growth is important to take seriously the possibility of green growth to keep consumption high while meeting emission targets. Endogenous growth introduces dynamics into the model since today's level of technology positively depends on yesterday's technology: \tr{ a \textit{standing on the shoulder of giants} mechanism. LOOK THIS UP \cite{Acemoglu2012TheChange}} 


%The model differentiates between high- and low-skilled labor to account for a skill bias found for the green sector \citep{Consoli2016DoCapital}. This asymmetry of sectors renders regressive taxes a tool to lower relative production costs in the green sector: high-skill workers reduce their labor supply more in response to a more progressive tax as leisure is more valuable to them. This recomposing mechanism counteracts intentions to lower emissions. %, a regressive tax functions as a green subsidy. % In fact, there is an externality arising from high-skill labor supply as it shapes the share of fossil to green energy production. 
%On the other hand, progressive income taxes lower aggregate production by diminishing the price of leisure. 
%Endogenous growth amplifies the repercussions of progressive income taxes:
%First, a lower labor supply reduces the profitability of research in general. By reducing hours worked the planner sacrifices technological progress. Secondly, the recomposing effect is aggravated as research is directed towards the sector with the increased labor share, i.e. the fossil sector.

%\paragraph{Calibration}
The model is calibrated to the US in the baseline period from 2015 to 2019. To do so, I proceed in two steps. First, I set certain parameters to values found in the literature. Most importantly, I use reasonable values of production and growth processes found in \cite{Fried2018ClimateAnalysis}. % who conducts a rigorous calibration exercise. 
With these parameter values at hand, I match the share of high skill in the green and non-green sectors building on \cite{Consoli2016DoCapital}. The emission target is set to the values suggested in the latest IPCC draft on mitigation pathways \citep{IPCC2022}: A 50\% reduction by 2030 relative to 2019 levels and  net-zero emissions from 2050 onward.

%\paragraph{Quantitative exercise}
I perform the following quantitative exercise. 
The Ramsey planner maximizes social welfare as in the core model. However, the externality enters as an exogenous emission limit which constraints government action. The planner sets the environmental tax and the progressivity parameter of the non-linear income tax scheme. \textbf{Environmental tax revenues are consumed by the government.} As discussed in the analytical section, under this policy regime the efficient allocation is not feasible; but, it is  more policy relevant. 

I solve the Ramsey model explicitly for 60 years starting from 2020. From 2080 onward taxes are fixed. The planner is constrained by a sustainability motive to leave at least as much resources to future periods than for the explicitly modeled periods.
This approach has the advantage of not having to assume the existence of a balanced growth path. Given that one factor of production, namely fossil energy, is bounded by an exogenous limit, I seek to stay agnostic on this aspect.
\tr{\textit{Alternative: assume the economy reaches a BGP in some future period; all variables grow at constant rates. Possible?}}

% main finding; then understand why; what are the drivers: how does the result change as things are added or taken from the model
\paragraph{Findings}
I find that the planner chooses a progressive income tax despite its repercussions on growth and emissions. Indeed, this policy increases the share of fossil to green energy and reduces research efforts and consumption. As the emission limit becomes tighter over time, the Ramsey planner augments both the environmental tax and income tax progressivity. The positive correlation between income tax progressivity and the environmental tax is in line with the analytical finding. 
 

% comparison to a model without income tax
To investigate the importance and the welfare gains of an integrated environmental policy, I rerun the main experiment but let the government only choose the environmental tax. Environmental tax revenues are transferred back to households through the income tax scheme.\footnote{\ \textit{This setting does neither correspond to any analytical model version- Could also think of modeling the alternative as gov consumes environmental tax revenues  and no income tax scheme.\textcolor{blue}{Compare: benchmark model where fiscal policy and environmental policy are integrated versus a version where there are neither an income tax scheme nor lump-sum transfers. }} 
}
Comparing the resulting allocation to the one under the benchmark model is informative on the mechanisms and importance of an integrated environmental and fiscal policy when no lump-sum transfers are available. 

\textit{ This means two aspects: (i) redistribution via the income tax scheme and (ii) the use of progressive taxes. Named: Integrated Scenario }

% results
The results suggest a clear cut between responsibilities of policy instruments: The environmental tax targets the externality, while the income tax handles the inefficiency arising from environmental tax revenues. When depriving the government from an income tax scheme, the environmental tax remains largely unchanged compared to the benchmark setting. 
\tr{To do: investigate if even with lump-sum redistribution there is a role for income tax in full model.}
The availability of an income tax increases social welfare by 0.11\% over the period from 2020 to 2080 which corresponds to a consumption equivalent of xxx. 

% optimal income tax with exogenous growth; optimal income tax without skill heterogeneity 

% sensitivity
\tr{In progress}
\begin{itemize}
	\item utility specification (Building on Bick can think of European version when substitution effect is stronger)
	\item no skill bias in the green sector
	\item no endogenous growth
	\item spillovers across scientists: with positive spillovers potentially no growth
	\item counterfactual technology gap
\end{itemize}
Since the target of the labor tax in the environmental setting presented here is to align hours worked with their efficient level, results are sensitive to the elasticity of labor with respect to after-tax wages. 
\paragraph{Quantitative finding to be shaped by income and substitution effect!}
Literature on how households react to changes in income \cite{Bick2018HowImplications} and \cite{Boppart2019LaborPerspectiveb}


%\paragraph{Literature}
\paragraph{Literature}

\paragraph{Optimal environmental policy}
\textbf{Main claim: focus on environmental taxes and recomposition}
\begin{itemize}
	\item with exogenous growth
	\item with endogenous growth
\end{itemize}

In general, papers on optimal environmental policy focus on the optimal environmental tax and analyze settings with inelastic labor supply \citep{Golosov2014OptimalEquilibrium, Acemoglu2012TheChang, Fried2018ClimateAnalysis}. Therefore, the main finding of the present paper, the necessity of reductive policy measures to implement the efficient allocation, complement this literature. Furthermore, I argue that the Pigou principle does generally not apply when no lump-sum transfers are available.  


\paragraph{Recycling of environmental tax revenues}
\textbf{Main claim: they overlook that when lump-sum transfers are not available, then labor supply is inefficiently high, and that env. tax exceeds (?) scc in optimal policy}
\begin{itemize}
	\item in general: \cite{Fried2018TheGenerations}
	\item double dividend literature
\end{itemize}
These findings have important consequences for the literature on the so-called double dividend of environmental policy and the question how to recycle environmental tax revenues. While this literature argues for the recycling of environmental tax revenues to lower pre-existing tax distortions, my paper constitutes an argument for a lower bound on distortionary income taxes: some reduction of labor supply is in fact efficient. 
Furthermore, when revenues are not redistributed lump-sum, the Pigouvian tax does not implement the efficient allocation. 

\paragraph{Environmental protection and inequality}
\ar 1) Inequality and environment as competing goods.

In the literature discussing environmental policies in an unequal framework, a competition between equity and environmental good provision have been discussed. 
This trade-off can be separated into (i) the competition for public funds \citep{LansBovenberg1996OptimalAnalyses, Jacobs2019RedistributionCurves} and (ii) effects of either environmental policies on equity or equalizing policies on environmental quality \citep{Jacobs2019RedistributionCurves, Sager2019IncomeCurves, Dobkowitz2022}. 

Since hours are inefficiently high, equalizing policies become part of the optimal environmental policy. As a byproduct, the distribution of income becomes more equal.


\ar 2) Inequality to shape effects of environmental policies and effect of fiscal policy on environment due to heterogeneity

 Furthermore, the differentiation of skills and the skill-bias of the green sector in the paper give rise to a new channel through which labor taxation affects environmental protection. The literature has primarily focused on a demand channel arising from non-linear Engel curves through which inequality and redistribution shape the degree of dirty production in the economy.   

%\textbf{Non-linear Engel Curves: redistribution}
\cite{Jacobs2019RedistributionCurves} the motive to redistribute and to provide an environmental good compete for government resources due to a negative effect of environmental taxes on the wage rate. Even with lump-sum transfers, the optimal environmental tax does not follow the Pigou principle when the government seeks to enhance equity.

\cite{Sager2019IncomeCurves} argues empirically, that redistribution to poorer households may result in a higher demand for polluting goods. 
\paragraph{Pubic Finance literature}
\textbf{I add: a new perspective on labor income taxes as a tool to lower inefficiently high hours worked. }

 \cite{Heathcote2017OptimalFramework}, \cite{Loebbing2019NationalChange}

\paragraph{Reductive policies in the literature}
The finding relates to the literature discussing rationales for the usage of reductive policy measures. These arise from o
Negative externalities of consumption and hours worked such as
 envy \cite{Alvarez-Cuadrado2007EnvyHours}, habits \cite{Ravn2006DeepHabits} \tr{Check if this is on inefficiency} or a positive externality of leisure \cite{Alesina2005WorkDifferent}. The present paper relates to this literature by identifying an externality of work which emerges from the existence of an environmental externality of production. In addition, once an environmental tax recomposes production towards a cleaner alternative, the wage rate understates the marginal product of labor. \tr{This needs to be made clearer.}
\begin{itemize}
	\item literature which \cite{Alvarez-Cuadrado2007EnvyHours}
\end{itemize}
\tr{This observation relates to the literature in several ways: first, the literature which discusses the optimal recycling of carbon tax revenues. Because when revenues are not recycled as lump-sum transfers, then labor supply is inefficiently high and additional policy measures are necessary to implement the efficient allocation today. 
	 In other words: because lump-sum transfers are not available, the literature argues, the government should use corrective tax revenues to lower pre-existing tax distortions. But by how much? I argue, that there is an optimal size of positive tax distortions when lump-sum transfers are not available. Hence, under the premise of non-lump sum transfers, distortionary labor income taxes arise as an optimal policy tool even absent an exogenous financing condition or inequality. }

 The paper relates broadly to the literature discussing optimal environmental policies. I separate them into two strands: one with inelastic and one with elastic labor supply. 
  \paragraph{Optimal environmental policy: exogenous labor supply}
 
\paragraph{Lit: environmental policy and distortionary fiscal setting}

\begin{itemize}
	\item Williams 2013 Double dividend 
	\item talk to Mireille Chiroleu Assouline: paper on double dividend
	\item Mireille with Aubert or Fodha (PSE)
\end{itemize}
%Inequality-environment nexus: normally motivated by a demand-side perspective; in this project I focus on a supply side explanation
labor supply becomes elastically in the literature studying the interaction of environmental taxes and distortionary taxes.  This strand of the literature generally focuses on the gap between the social cost of carbon and the optimal environmental tax arising from pre-existing distortionary labor income taxes or an exogenous requirement on government funds \citep{Bovenberg1997EnvironmentalGrowth,  Kaplow2012OPTIMALTAXATION, Jacobs2019RedistributionCurves, Barrage2019OptimalPolicy}. labor income taxes form a passive component of these analyses. 
The general findings of this literature is that the optimal environmental tax falls below the social cost of carbon to mitigate efficiency costs and enable the government to raise revenues. 
Furthermore, the literature argues for a recycling of environmental tax revenues to be used to lower income taxes. A recycling through transfers would intensify reductions in labor supply. These arguments rely on the premise that no lump-sum transfers are available. I add to this literature the perspective that a reduction in labor supply is part of the efficient policy. If lump-sum transfers are not available - as is to be assumed in this literature to motivate the existence of distortionary taxes - then labor income taxes should be positive to cope with distortions in the labor supply. Hence, there is a lower bound up to which environmental tax revenues are optimally used to lower distortionary taxes. This is not recognised by the literature. \tr{How do \cite{LansBovenberg1994EnvironmentalTaxation} argue for the use of env. tax revenues to lower distortionary taxes? Verbally or analytically?}

In contrast, I focus the paper on the role of income taxes in the optimal environmental policy and abstract from an exogenous financing condition on the government. Still, the model rationalizes a positive or progressive income tax.
 The inefficiency of environmental taxes arises absent pre-existing income tax distortions or the motive to redistribute.
In difference to this literature, where the presence of a distortionary income tax shapes the optimal level of the environmental tax, the existence of the environmental tax rationalises a progressive income tax in the present paper.
Importantly, the equity and the environmental targets of government intervention are perceived as competing goals as both tax instruments exert efficiency costs through a reduction in labor supply. 
I argue in this paper that what has commonly been perceived as an efficiency cost -  the reduction in labor supply in response to environmental and income taxation - is part of the optimal environmental policy. Hence, income taxation has a double dividend: an environmental and an equity one.  

In this literature, there are either no transfers to households at all \citep{Bovenberg2002EnvironmentalRegulation, LansBovenberg1994EnvironmentalTaxation} or an exogenously given requirement for transfers \citep{Barrage2019OptimalPolicy}. Hence, there is no lump-sum transfer instrument.

\citep{Fullerton1997EnvironmentalComment} writes in its introduction 
\begin{quote}
	With no revenue requirement, or where government can use lump-sum taxes, Arthur C. Pigou (1947) shows that the first-best tax on pollution is equal to the marginal environmental damage.
\end{quote}
\ar What is the optimal environmental tax when labor supply is elastic and there are no lump-sum funds?
\paragraph{Environment and elastic labor supply}
\cite{Oueslati2002EnvironmentalSupply} studies the optimal environmental policy with elastic labor supply. Yet, he allows for lump-sum transfers of environmental revenues. \textit{He should find something on reduction of hours}: No: capital is the only polluting factor, and labor is the clean factor of production.
\paragraph{Recycling of environmental tax revenues}
\cite{Fried2018TheGenerations}
\paragraph{Environment and (endogenous) growth}
\begin{itemize}
	\item limits to growth
	\item general literature on end growth and the environment
\end{itemize}
\paragraph{Public finance}
An equity-efficiency trade-off is central to the discussion of optimal labor income taxes in the public finance literature.  The benefits of labor taxes and progressivity arise, inter alia, from redistribution. %and from generating government revenues. 
With concave utility specifications full redistribution is efficient. However, the optimal tax system does not feature full redistribution when labor supply is endogenous. Instead, redistribution is traded off against aggregate output as individuals reduce their labor supply and skill investment in response to labor income taxation \citep{Heathcote2017OptimalFramework, Conesa2009TaxingAll, Domeij2004OnTaxes}.

To this literature I add another motive for the use of distortionary fiscal policies; namely to reduce inefficiently high labor supply. Furthermore, by abstracting from income inequality or income risk heterogeneity - the present framework
One closely related work is \cite{Loebbing2019NationalChange} who studies optimal income taxation in a model of directed technical change. The redistributive effect of tax progressivity is amplified through a compression of the wage rate distribution xxx
	\section{Introduction}
	The latest assessment report of the Intergovernmental Panel on Climate Change (IPCC) \citep{IPCC2022} highlights the urgency to reduce greenhouse-gas emissions.%relative to the previous report from 2018 \citep{Rogelj2018MitigationDevelopment.}.
\footnote{ \  The report stresses the decreasing likelihood of meeting the Paris Agreement and limiting climate warming to 1.5°. The Paris Agreement of 2015 formulates clear political goals to mitigate climate change. Under this treaty, states have agreed on a legally binding maximum increase in temperature to well below 2°C, preferably 1.5° over pre-industrial levels, and the global community seeks to be climate-neutral in 2050  (compare: \url{https://unfccc.int/process-and-meetings/the-paris-agreement/the-paris-agreement}). 
}
On the other hand, scholars have pointed to reductive policy measures to handle environmental limits \citep{Arrow2004AreMuch, Schor2005SustainableReduction, Dasgupta2021}. A reduction in work effort and consumption mitigates pollution by diminishing economic activity. Such a reduction could be achieved by using distortionary fiscal policy tools.
However, the economic literature on environmental policy has focused on the recomposing aspect of environmental policies: environmental taxes. %\citep{Fried2018ClimateAnalysis}. 
Given the exigency to act, this paper addresses the question whether fiscal policies can help meet climate targets.

I show analytically that, indeed, once 
labor supply is elastic, reductive policy measures optimally complement the environmental tax. 
It is established in the literature that absent any other distortion, an environmental tax equal to the social cost of the externality implements the efficient allocation. 
%Environmental taxes are perceived as a cost-effective way to reduce emissions. 
I argue that this result crucially depends on the use of lump-sum transfers to redistribute environmental tax revenues; transfers reduce labor supply through an income effect. Thus, indeed there is a role for reductive policy measures. 
%\textcolor{blue}{This is interesting independent of whether they are feasible or not. Could relate to the fact that there is a discussion how to use revenues. Yet, one might argue that we are always in a setting with distortionary labor income taxes; so that recycling lump-sum is never needed; numbers on size of expected revenues and government spending}
 As a consequence,  when lump-sum transfers are not in the policy set, environmental taxes are optimally combined with progressive labor income taxes. The use of income taxes as a reductive policy measure is not directly targeted at the externality: the motive for labor taxation follows indirectly from a distortion in labor supply due to lower income. Hence, (i) the two tax instruments are complements, and % to lower inefficiently high hours worked. 
% I show that redistributing environmental tax revenues through an income tax scheme allows to implement the efficient allocation. The optimal income tax scheme is progressive.
(ii) the optimal environmental policy equalizes the distribution of income as  a side effect. The theoretic analysis forms the first part of the paper.

In the second part, I study a quantitative model where environmental tax revenues are consumed by the government. Current debates on how to optimally use environmental tax revenues in politics \citep{Baker2017TheDividends} and academia \citep[e.g.][]{Fried2018TheGenerations, Carattini2018} motivate to focus on this policy regime. In this setting, I scrutinize whether progressive income taxes remain optimal in a more realistic model with endogenous growth and heterogeneous skills. This is unclear a-priori since a progressive tax scheme reduces incentives to innovate through a market size effect. Furthermore, a skill bias documented for the green sector \citep{Consoli2016DoCapital} in combination with a relatively more elastic high-skilled labor supply causes a higher tax progressivity to recompose the economic structure towards dirty production. The model suggests that despite these adverse mechanisms the optimal income tax scheme is progressive. To lower hours worked the government forfeits growth and accepts a less green production ratio.  
%\textit{I quantify the welfare gains of setting progressive income taxes to equal yyy in consumption equivalent measure. TO BE DONE  }

% relation to literature
The paper's results are relevant for the political and academic debate on how best to use environmental tax revenues. The paper points to the importance of lump-sum transfers as a reductive policy tool in the optimal environmental policy; an aspect which appears overlooked in today's discussion.%\footnote{\ POLICY debate; \cite{Fried2018TheGenerations}}
When thinking about how to recycle environmental tax revenues other than by lump-sum transfers, then, one should think about alternative reductive tools such as progressive labor income taxes. 
If the reductive part of the environmental policy is neglected, environmental taxes have to be higher to meet emission limits, as I demonstrate in the quantitative exercise.

The results address the academic debate on the so-called \textit{weak double-dividend} \citep[for example:][]{LansBovenberg1994EnvironmentalTaxation, LansBovenberg1996OptimalAnalyses}. The hypothesis posits that recycling environmental tax revenues to reduce pre-existing tax distortions is advantageous to recycling  revenues as lump-sum transfers. The rationale is that transfers decrease labor supply thereby diminishing the tax base of the income tax. A conflict between generating government funds and environmental protection arises. The findings in the present paper suggest a lower bound on the reduction in distortionary income taxes: when environmental tax revenues are not redistributed lump-sum, some reduction in labor supply via distortionary income taxes is in fact efficient from an environmental policy perspective. In other words, even if environmental tax revenues suffice to satisfy a government revenue requirement, there is a motive for progressive income taxation. 

The finding is especially interesting as the provision of the environmental public good and equity have been perceived as competing targets in the literature. First, when the poor consume more of the polluting good, a corrective tax is regressive \citep{Sager2019IncomeCurves, Fried2018ClimateAnalysis} \textit{Metcalf 2007, Hassett 2009 as  in Fried 2018}. Second and more indirectly, a fossil tax exerts efficiency costs by lowering labor efforts\footnote{\ The reduction in hours worked is per se not inefficient. The reduction in dirty production reduces the marginal product of labor, so that the disutility of labor is not compensated enough. However, when the government seeks to tax labor income using distortionary policy tools the reduced labor supply diminishes the tax base of the labor tax making it more costly to redistribute.} which again makes it more costly for the government to redistribute \citep{Dobkowitz2022}. 
In contrast to this literature, the present paper provides an argument for progressive income taxes even under perfect income risk sharing. This suggests a double dividend of redistribution: equity on the one hand and efficiency gains from less labor as part of the environmental policy.

	\paragraph{Outline} Section \ref{sec:mod_an} presents the model. I derive and discuss the results in section \ref{sec:theory}. Section \ref{sec:con} concludes.
	\section{Core model and theoretic results}\label{sec:mod_an}

This section develops a tractable model  to derive the theoretic results. I show that scaling the level of production is part of the efficient environmental policy. Yet, absent endogenous growth, this is fully implemented by the use of an environmental tax and lump-sum transfers. There is no role for labor income taxes.

\subsection{Model}
The representative household faces a consumption and labor supply decision. The final consumption good is a composite of a fossil and a green good. Labor is the only input to production. The fossil sector causes an environmental externality.\footnote{ For simplicity, the green sector does not induce any externality; yet, whenever intermediate goods are no perfect substitutes, final good production is never perfectly green.} There is no growth, and the model is static.

\paragraph{Representative household}
Throughout the paper, the household's decision is static. Each period, the household maximizes its period utility
\begin{align*}
U(C,H; F).
\end{align*} 

The household derives utility from consumption, $C$, but experiences disutility from hours worked, $H$. An externality from fossil production, $F$, decreases household utility. The level of fossil production is taken as given by the household.
I assume additive separability of consumption, hours, and the externality. Utility of consumption is increasing and strictly concave. Utility is decreasing and strictly convex in hours worked and fossil production.
Utility maximization is subject to a period budget constraint:
\begin{align}
	 C= \lambda(wH)^{1-\tau_{\iota}}+T_{ls}. \label{eq:hhbudget}
\end{align}
The variable $w$ indicates the wage rate.  Lump-sum transfers from the government are denoted by $T_{ls}$.
The government levies income taxes on labor income using a non-linear tax scheme common in the public finance literature \citep{Heathcote2017OptimalFramework, Benabou2002TaxEfficiency}. The tax scheme is
characterized by (i) a scaling factor, $\lambda$, which determines the level of average tax revenues in the economy, and (ii) a measure of tax progressivity denoted by $\tau_{\iota}$. 
\cite{Heathcote2017OptimalFramework} show that whenever $\tau_{\iota}>0$, the tax scheme is progressive since the marginal tax rate exceeds the average tax rate irrespective of  pre-tax labor income. Hence, average tax payments increase with labor income.\footnote{ An alternative intuition is that when $\tau_{\iota}>0$, the elasticity of post- to pre-tax  income is smaller unity for all levels of pre-tax income.  } %\footnote{ I show that the result is equivalent with a linear tax rate in the appendix.} 
With a representative household, $\tau_{\iota}$ can be understood as an instrument to regulate labor supply and, thus, the overall level of production. When $\tau_{\iota}<0$, the government subsidizes labor, with $\tau_{\iota}>0$, it discourages labor. 

\paragraph{Production}
All sectors of production are perfectly competitive, and production functions have decreasing returns to scale. %\footnote{ \textit{With increasing returns to scale the assumption of perfect competition would be violated. With constant returns to scale, the solution is not unique.}}. The final consumption good, $Y$, is a composite of the fossil, $F$, and the green intermediate good, $G$. 
Intermediate goods, indicated by $J\in \{F,G\}$ for fossil and green, are produced from the labor input good, $L_J$, using technology, $A_J$. The variable $Y$ stands in for final output and is the numeraire. Production is given by:
\begin{align}
Y=Y(F, G), \hspace{5mm} F=F(A_F, L_F),\hspace{5mm} G=G(A_G, L_G). \label{eq:prod}
\end{align}

\paragraph{Government}
The government raises income taxes from households and levies an environmental tax, $\tau_F$, per unit of fossil energy bought by final good producers. The environmental tax, thus, is modeled in parallel to a carbon tax which poses a price on emissions. Revenues from the income tax and the environmental tax are treated separately by the government. Income tax revenues are fully redistributed through the income tax schedule. Environmental tax revenues are rebated lump sum to households:
\begin{align}
\tau_{F}F=T_{ls}, \hspace{7mm}
0={w H}-\lambda(w H)^{1-\tau_{\iota}}. \label{eq:gov_but}
\end{align}
The scaling parameter $\lambda$ adjusts to balance the income tax scheme. 
%Environmental tax revenues are either transferred lump-sum, fully consumed by the government, or transferred through the income tax schedule.

\paragraph{Markets}
Markets for labor and the final good both clear: 
\begin{align}
H=L_F+L_G,\ \hspace{5mm} Y=C. \label{eq:market_clear}
\end{align}
%I summarize the eq.s determining the competitive equilibrium in appendix Section \ref{app:model}.
\paragraph{Competitive equilibrium}
In a competitive equilibrium, household behavior is determined by the budget constraint, eq. \eqref{eq:hhbudget}, and labor supply which follows from the household's first order conditions and substitution of $\lambda$ from the government's budget on the income tax:
\begin{align}
-U_H=U_C(1-\tau_{\iota})w. \label{eq:hsup}
\end{align}
Firms choose the quantity of input goods to maximize their profits taking prices as given. The following equations describe this behavior in equilibrium:
\begin{align}
p_G=\frac{\partial Y}{\partial G}, \hspace{5mm}
p_F +\tau_{F} = \frac{\partial Y}{\partial F}, \hspace{5mm}
w= p_F\frac{\partial F}{\partial L_F}=p_G\frac{\partial G}{\partial L_G}.\label{eq:profmax}
\end{align}

The competitive equilibrium is defined as prices and allocations so that households and firms behave optimally; i.e. eqs. \eqref{eq:hhbudget}, \eqref{eq:hsup}, and \eqref{eq:profmax} hold. Production happens according to eqs. \eqref{eq:prod}.  Equilibrium prices and the wage rate adjust to clear markets, eqs. \eqref{eq:market_clear}. Finally, the government's budgets are satisfied eqs. \eqref{eq:gov_but}. Policy variables $\tau_F$ and $\tau_\iota$ are taken as given. 

\subsection{Theoretic results}\label{sec:theory}
Section \ref{subsec:sp2} defines and discusses the efficient allocation. It constitutes a benchmark for the optimal allocation discussed in Section \ref{subsec:decen_ec}. 

\subsubsection{Social planner}\label{subsec:sp2}
Let the share of fossil to total labor be denoted by $s=\frac{L_F}{H}$. The social planner's problem reads
\begin{align*}
\underset{s, H}{\max}\ & U(C,H; F)\\ s.t.\ \ & C=Y.
\end{align*}
The first order conditions are given by
\begin{align}
wrt. s:\hspace{4mm} & U_C \cdot \left(\frp{Y}{F}\frp{F}{s}+\frp{Y}{G}\frp{G}{s}\right)=-U_F\frp{F}{s}, \label{eq:fbs2}\\
wrt. H:\hspace{4mm}& U_C\frp{Y}{H}+U_F\frp{F}{H}=-U_H. \label{eq:fbh}
\end{align}
Where $U_X$ denotes the partial derivative of utility with respect to the variable $X$.
These equations determine the efficient or first-best allocation. 
Absent an externality, $U_F=0$, the efficient distribution of labor equalizes the marginal product of labor across sectors; compare eq. \eqref{eq:fbs2}. Efficient hours balance the marginal utility gain from consumption and the marginal disutility from working, as formalized by eq. \eqref{eq:fbh}. 

When there is an externality, the social planner adjusts the allocation in two ways: (i) a compositional adjustment, that targets the share of fossil production, and (ii) a scaling adjustment amending the level of production. 
The compositional adjustment is determined by eq. \eqref{eq:fbs2}.
The negative externality of fossil production makes it efficient to alter the share of fossil labor so that  a marginal reallocation of labor to the fossil sector would raise output.\footnote{ Note that $U_F<0$ by assumption so that the right-hand side is positive and that $\frac{dG}{ds}<0$. Hence,  in the efficient allocation, the marginal product of labor in the fossil sector is higher than in the green sector.} %Hence,$\frp{Y}{F}\frp{F}{s}>-\frp{Y}{G}\frp{G}{s}$ is efficient. }
I show in Appendix \ref{app:redeffs} that the social planner reduces the fossil labor share when the aggregate production function features decreasing returns to scale in its labor inputs, $L_G$ and $L_F$.


The scaling effect is summarized by eq. \eqref{eq:fbh}.
First note that eq. \eqref{eq:fbh} can be rewritten by substituting eq. \eqref{eq:fbs2} and noticing the relation of derivatives with respect to $H$ and $s$.\footnote{ This is done in more detail for the optimal allocation in Appendix \ref{app:incometax0}. Relations of derivatives are summarized in Appendix \ref{app:dervs_use}.}  
The second first order condition becomes:
\begin{align}\label{eq:fbh_simp}
-U_H=U_C\frac{\partial Y}{\partial G}\frp{G}{L_G}.
\end{align}
Hence, the efficient level of the externality lowers hours as if the marginal product of labor was equal to the marginal product of labor in the clean sector.

The recomposition of labor towards the  green sector reduces the average marginal product of labor in the economy. An additional unit of labor results in a smaller increase in consumption.  This effect has two opposing impacts on the efficient level of labor. On the one hand, there is a substitution effect: as leisure becomes less costly, the efficient amount of hours reduces (note that the right-hand side of eq. \eqref{eq:fbh} is increasing in $H$). On the other hand, the economy becomes poorer in terms of consumption, and more work effort might be efficient. This is captured by the term $U_C$ and equivalent to an income effect. 
%In total, which effect dominates depends on the curvature of the utility from consumption, $\theta$. With $\theta>1$ the  lower marginal product of labor decreases the efficient amount of hours worked. 
%Second, the social planner reduces hours worked due to their negative exeternality through fossil production. This effect is introduced by the term $U_F\frac{dF}{dH}<0$. 
Proposition \ref{prop:0} summarizes this discussion.
\begin{prop}\label{prop:0}
	Efficient externality mitigation consists of a compositional and a scaling adjustment. 
\end{prop}


Depending on the importance of the income effect, efficient hours worked may be higher or lower than  absent an externality. I will show in the following, however, that there is no role for labor income taxation in implementing the efficient allocation. In fact, under the optimal policy, the wage rate is set so that households internalize the effect of work effort on emissions. %\footnote{ \ I discuss in the appendix conditions on parameter values when assuming functional forms of the model.}
%I will show in the following, that irrespective of whether the social planner de- or increases hours, the decentralized economy always features higher hours when environmental tax revenues are not redistributed lump-sum. 


\subsubsection{Decentralized economy}\label{subsec:decen_ec}

Governments use tax and transfer instruments to correct for distortions, such as an environmental externality. The question arises if the efficient allocation can be decentralized by the use of taxes and transfers in a competitive economy.  %For now, I assume that the income tax is not available and $\tau_{\iota}=0$, $\lambda=0$.

%I show in this section that lump-sum redistribution of environmental tax revenues is essential to implement the first-best allocation in the competitive equilibrium. Only in combination with lump-sum transfers of  environmental tax revenues does an environmental tax suffice to implement the efficient allocation. %Then the environmental tax equals the social cost of the externality as shown by \textit{PIGOU}. 
%When environmental tax revenues are not redistributed lump-sum, hours worked exceed their efficient level, and a role for income taxes to lower hours worked arises. I consider two cases.

%\begin{enumerate}
%\item lump-sum transfers important for Pigou tax to implement efficient allocation: Proposition \ref{prop:1}
%\item when transfers are not redistributed: infeasibility of efficient allocation,  role for labor tax, and violation of Pigou principle \ref{prop:2}.
%\item redistribution through income tax scheme with progressive income tax restores efficient allocation \ref{prop:3}
%\end{enumerate}

%\subsubsection{Government problem}\label{subsec:Rams}
The government is characterized by a Ramsey planner who maximizes utility of the representative household by use of tax and transfer instruments. The behavior of firms and households constrain the government's optimization problem. 
The Ramsey problem is defined as
\begin{align*}
\underset{s, H}{\max}\ & U(C,H; F)\\ s.t.\ \ &  C=Y,
\end{align*}
subject to the behavior of households and firms.
The first order conditions are equivalent to the social planner ones:
\begin{align}
wrt.\ s:\hspace{4mm} & U_C\cdot\left(\frac{\partial Y}{\partial F}\frac{\partial F}{\partial s}+\frac{\partial Y}{\partial G}\frac{\partial G}{\partial s}\right)=-U_F\frac{\partial F}{\partial s}, \label{eq:sbs}
\\
wrt.\ H:\hspace{4mm} & U_C\frac{\partial Y}{\partial H}+U_F\frac{\partial F}{\partial H}=-U_H\label{eq:sbh}. 
\end{align}
%-- paragraph to show that with Gov=0 and lump-sum transfers, the efficient allocation is implemented
When environmental tax revenues are fully redistributed lump sum, an environmental tax equal to the marginal social cost of fossil production implements the efficient allocation.\footnote{ I define and derive the social cost of fossil production in Appendix \ref{app:scp}.} This observation is known as the \textit{Pigou principle} in the literature. 
To see this, note that eq. \eqref{eq:sbs} ensures that the social planner's first order condition, eq. \eqref{eq:fbs2}, is satisfied. 
Rewriting eq. \eqref{eq:fbs2} reveals that the Pigou principle holds: %\footnote{ I derive the social cost of pollution as the price the representative household is willing to pay for a marginal reduction in fossil production. The derivation is exponded in appendix Section \ref{sec:mod_an}. 
%	To be precise, social cost of pollution refers to the marginal cost evaluated at the resulting equilibrium allocation.}: The Pigou principle. 
\begin{align*}
\underbrace{\frac{-U_F}{U_C}}_{\text{marginal social cost of fossil production}}=\left(1+\frac{\frac{\partial Y}{\partial G}\frac{\partial G}{\partial s}}{\frac{\partial Y}{\partial F}\frac{\partial F}{\partial s}}\right)\frac{\partial Y}{\partial F}=\tau^*_F.
\end{align*}
Where the second equality follows from substituting firms' profit maximization conditions from eqs. \eqref{eq:profmax}.

Absent an externality of production, it is efficient to balance marginal products of labor across sectors.
When there is an externality, the social planner lowers the share of labor in the fossil sector. As a result, the marginal product of labor in this sector increases. It falls in the green sector. To sustain this gap between marginal products in the competitive equilibrium, the government has to introduce a corrective tax. Otherwise, market forces would direct labor towards the sector with the higher marginal product. Consequently, the equilibrium wage rate is below the marginal product of labor.\footnote{ I formally discuss this statement in Appendix \ref{app:wageMPL}.} 

Setting the environmental tax equal to the social cost of fossil production implies that the second first order condition of the Ramsey planner, eq. \eqref{eq:sbh}, is satisfied without use of the income tax instrument: $\tau_{\iota}^*=0$. 
The reason is, that in this case, the wage rate reflects the marginal social costs of hours through raising emissions. I show in Appendix \ref{app:incometax0} that the wage rate can be written as:
\begin{align*}
w = \frp{Y}{H}+\frac{U_F}{U_C}\frp{F}{H}.
\end{align*}
Since $U_F<0$, the second summand reduces the wage rate beyond the marginal product of labor in the economy.
Therefore, households internalize the marginal social costs of the externality of hours worked in their labor supply decision. Relative to no policy intervention, labor supply declines. Proposition \ref{prop:1} condenses this result.

\begin{prop}\label{prop:1}
	The efficient allocation is implemented by an environmental tax and lump-sum transfers.  When the environmental tax implements the efficient share of dirty labor, the wage rate fully captures the marginal effect of hours worked on the externality. There is no role for distortive labor income taxation, $\tau_{\iota}^*=0$.
\end{prop}
Proof: Appendix \ref{app:incometax0}. 

%As discussed previously,  However,

%Due to this effect of the environmental tax on the wage rate, lump-sum transfers and environmental taxes alone suffice to implement the efficient level of hours worked. 


%	\input{impo_model_fin}
%	\section{Calibration}\label{sec:calib}
%	\section{Quantitative results}\label{sec:res}

In this section, I present and discuss the quantitative results.
Subsection \ref{subsec:mr} depicts the optimal policy given the emission target. Subsection \ref{subsec:dis} discusses the results. In particular, I focus on analyzing the mechanisms and welfare benefits from integrating the income tax scheme into the environmental policy. 

\subsection{Results}\label{subsec:mr}
%This section depicts results on the optimal policy followed by the implied allocation in the benchmark model where environmental tax revenues are redistributed via the income tax scheme. 

\begin{figure}[h!!]
	\centering
	\caption{Optimal Policy }\label{fig:optPol}
	\begin{minipage}[]{0.4\textwidth}
		\centering{\footnotesize{(a) Income tax progressivity, $\tau_{lt}$}}
		%	\captionsetup{width=.45\linewidth}
		\includegraphics[width=1\textwidth]{../../codding_model/own_basedOnFried/optimalPol_elastS_DisuSci/figures/all_1705/Single_OPT_T_NoTaus_taul_spillover0_sep1_BN0_ineq0_red0_etaa0.79.png}
	\end{minipage}
	\begin{minipage}[]{0.1\textwidth}
		\
	\end{minipage}
	\begin{minipage}[]{0.4\textwidth}
		\centering{\footnotesize{(b) Environmental tax, $\tau_{ft}$ }}
		%	\captionsetup{width=.45\linewidth}
		\includegraphics[width=1\textwidth]{../../codding_model/own_basedOnFried/optimalPol_elastS_DisuSci/figures/all_1705/Single_OPT_T_NoTaus_tauf_spillover0_sep1_BN0_ineq0_red0_etaa0.79.png}
	\end{minipage}
\end{figure} 

To meet the IPCCs suggested emission target, the optimal income tax is progressive for all periods between 2030 and 2080; see panel (a) in figure \ref{fig:optPol}. As the emission target is less strict, between 2030 to 2045, optimal income tax progressivity is around $\tau_{lt}=0.04$. As the emission target jumps to net-zero emissions in 2050, optimal tax progressivity accelerates to above 0.08 and gradually increases in the subsequent years to around 0.09. This is approximately half the size found for the US in \cite{Heathcote2017OptimalFramework}: $\tau_{l}=0.181$. 
In the period without emission target from 2020 to 2030, the optimal income tax is slightly regressive.

Consider panel (b). The optimal fossil tax displays a similar step pattern as the income tax progressivity. From 2020 to the beginning of 2030, it is negative. It jumps to around 70\% as the emission target is to reduce emissions by 50\% relative to 2019 emissions. As the emission target rises  to net-zero emissions in 2050, the optimal tax on fossil sales is close to 90\%. 

Figure \ref{fig:optAll} depicts the optimal allocation while meeting emission targets. Limiting emissions in line with the Paris Agreement is concomitant with both a reduction and recomposition of consumption and production over time. 

Panel (a) shows consumption which reduces significantly when new emission limits become active, in 2030 and in 2050, but starting from the new low levels continues to grow. labour effort of both skill types also reduces visibly as stricter emission targets are enforced; panel (b). In contrast to consumption, hours worked for both types of labour decrease over time. In comparison to hours supplied by low-skilled workers, high-skilled workers reduce hours more; compare panel (c) which shows the ratio of hours worked by high to low skill workers. 

The rise in consumption after each reduction is driven by technological progress in all sectors; compare panel (d) which shows growth rates by sector and as aggregate in per cent. 
The green sector sees a rise in technological progress, the dashed black line, while growth in the fossil and the non-energy sector is positive, yet diminishing over time. Overall, aggregate growth is positive but decreasing; compare the grey dashed graph. 
Summing up the last two paragraphs, the emission target is best achieved with more leisure at higher technology levels in all sectors. 

Nevertheless, there would be potential for more growth which is forfeited to meet emission targets. This becomes apparent when looking at the allocation of scientists in panel (e). Again, there is a recomposition towards the green sector: while research in the non-energy and the fossil sector decrease over time, green research effort rises. Yet, overall, the amount of scientists reduces; compare the grey graph which depicts the sum of researchers across sectors. 
Finally, labour input goods are also redirected towards the green sector; see panel (f). 

\begin{figure}[h!!]
	\centering
	\caption{Optimal Allocation }\label{fig:optAll}
	
	
	\begin{minipage}[]{0.32\textwidth}
		\centering{\footnotesize{(a) Consumption}}
		%	\captionsetup{width=.45\linewidth}
		\includegraphics[width=1\textwidth]{../../codding_model/own_basedOnFried/optimalPol_elastS_DisuSci/figures/all_1705/Single_OPT_T_NoTaus_C_spillover0_sep1_BN0_ineq0_red0_etaa0.79.png}
	\end{minipage}
	\begin{minipage}[]{0.32\textwidth}
		\centering{\footnotesize{(b) Hours worked }}
		%	\captionsetup{width=.45\linewidth}
		\includegraphics[width=1\textwidth]{../../codding_model/own_basedOnFried/optimalPol_elastS_DisuSci/figures/all_1705/SingleJointTOT_OPT_T_NoTaus_labour_spillover0_sep1_BN0_ineq0_red0_etaa0.79_lgd1.png}
	\end{minipage}
	\begin{minipage}[]{0.32\textwidth}
		\centering{\footnotesize{(c) High-to-low-skill ratio hours}}
		%	\captionsetup{width=.45\linewidth}
		\includegraphics[width=1\textwidth]{../../codding_model/own_basedOnFried/optimalPol_elastS_DisuSci/figures/all_1705/Single_OPT_T_NoTaus_hhhl_spillover0_sep1_BN0_ineq0_red0_etaa0.79.png}
	\end{minipage}
	\begin{minipage}[]{0.32\textwidth}
		\centering{\footnotesize{\ \\ (d) Technology growth}}
		%	\captionsetup{width=.45\linewidth}
		\includegraphics[width=1\textwidth]{../../codding_model/own_basedOnFried/optimalPol_elastS_DisuSci/figures/all_1705/SingleJointTOT_OPT_T_NoTaus_Growth_spillover0_sep1_BN0_ineq0_red0_etaa0.79_lgd1.png}
	\end{minipage}
	\begin{minipage}[]{0.32\textwidth}
		\centering{\footnotesize{\ \\(e) Scientists }}
		%	\captionsetup{width=.45\linewidth}
		\includegraphics[width=1\textwidth]{../../codding_model/own_basedOnFried/optimalPol_elastS_DisuSci/figures/all_1705/SingleJointTOT_OPT_T_NoTaus_Science_spillover0_sep1_BN0_ineq0_red0_etaa0.79_lgd1.png}
	\end{minipage}
	\begin{minipage}[]{0.32\textwidth}
		\centering{\footnotesize{\ \\(f) Labor input}}
		%	\captionsetup{width=.45\linewidth}
		\includegraphics[width=1\textwidth]{../../codding_model/own_basedOnFried/optimalPol_elastS_DisuSci/figures/all_1705/SingleJointTOT_OPT_T_NoTaus_labourInp_spillover0_sep1_BN0_ineq0_red0_etaa0.79_lgd1.png}
	\end{minipage}
\end{figure} 



\subsection{Discussion}\label{subsec:dis}
The discussion of the optimal policy centers on the questions (i) what drives the optimal policy and (ii) how to assess the optimal allocation: what are the benefits of labor income taxes, finally, (iii) what effects do skill heterogeneity and endogeneous growth have on the optimal policy (effect on tauf, taul and on gap to efficient allocation)
\begin{enumerate}
	\item What is the goal of policy intervention? \ar social planner allocation
	\item (Benefits) What is different when no integrated policy is run and instead revs consumed by government \ar Benefits of an integrated policy
	\item double dividend literature: use of labor income tax when all env tax revenues are consumed by the government.
	\item (Costs) What cannot be reached by integrated policy as compared to lump-sum transfers: is taul used for different purpose? without endogenous growth should be zero; eg. can use taul to boost growth as lump-sum transfers take care of labor supply 
	\item What could be reached if there was no trade-off with heterogenous skills or growth? no heterogeneous skills, no endogenous growth \ar how does the optimal policy differ?
\end{enumerate}

\subsubsection{Social planner allocation}
As a benchmark to the Ramsey planner allocation, I present the social planner's allocation. The efficient allocation can be perceived as the intended allocation which the Ramsey planner may not be able to implement due to the reliance on tax instruments. Figure \ref{fig:fb_opt} depicts the efficient and the optimal allocation by the black-solid and the orange-dashed graphs. 
\begin{figure}[h!!]
	\centering
	\caption{Comparison to efficient allocation }\label{fig:fb_opt}
	
	\begin{minipage}[]{0.32\textwidth}
		\centering{\footnotesize{(a) Consumption}}
		%	\captionsetup{width=.45\linewidth}
		\includegraphics[width=1\textwidth]{../../codding_model/own_basedOnFried/optimalPol_190722_tidiedUp/figures/all_July22/C_CompEffOPT_T_NoTaus_opteff_spillover0_noskill0_sep1_xgrowth0_countec0_etaa0.79_lgd1_lff0.png}
	\end{minipage}
	\begin{minipage}[]{0.32\textwidth}
	\centering{\footnotesize{(b) High skill hours worked}}
	%	\captionsetup{width=.45\linewidth}
	\includegraphics[width=1\textwidth]{../../codding_model/own_basedOnFried/optimalPol_190722_tidiedUp/figures/all_July22/hh_CompEffOPT_T_NoTaus_opteff_spillover0_noskill0_sep1_xgrowth0_countec0_etaa0.79_lgd0_lff0.png}
\end{minipage}
	\begin{minipage}[]{0.32\textwidth}
	\centering{\footnotesize{(c) Low skill hours worked}}
	%	\captionsetup{width=.45\linewidth}
	\includegraphics[width=1\textwidth]{../../codding_model/own_basedOnFried/optimalPol_190722_tidiedUp/figures/all_July22/hl_CompEffOPT_T_NoTaus_opteff_spillover0_noskill0_sep1_xgrowth0_countec0_etaa0.79_lgd0_lff0.png}
\end{minipage}

	\begin{minipage}[]{0.32\textwidth}
	\centering{\footnotesize{(d) Aggregate growth}}
	%	\captionsetup{width=.45\linewidth}
	\includegraphics[width=1\textwidth]{../../codding_model/own_basedOnFried/optimalPol_190722_tidiedUp/figures/all_July22/gAagg_CompEffOPT_T_NoTaus_opteff_spillover0_noskill0_sep1_xgrowth0_countec0_etaa0.79_lgd0_lff0.png}
\end{minipage}
\begin{minipage}[]{0.32\textwidth}
	\centering{\footnotesize{(e) Energy mix, $\frac{G}{F}$}}
	%	\captionsetup{width=.45\linewidth}
	\includegraphics[width=1\textwidth]{../../codding_model/own_basedOnFried/optimalPol_190722_tidiedUp/figures/all_July22/GFF_CompEffOPT_T_NoTaus_opteff_spillover0_noskill0_sep1_xgrowth0_countec0_etaa0.79_lgd0_lff0.png}
\end{minipage}
\begin{minipage}[]{0.32\textwidth}
	\centering{\footnotesize{(f) Utility}}
	%	\captionsetup{width=.45\linewidth}
	\includegraphics[width=1\textwidth]{../../codding_model/own_basedOnFried/optimalPol_190722_tidiedUp/figures/all_July22/SWF_CompEffOPT_T_NoTaus_opteff_spillover0_noskill0_sep1_xgrowth0_countec0_etaa0.79_lgd0_lff0.png}
\end{minipage}
\end{figure}

The social planner reduces consumption less than in the Ramsey planner allocation in order to reach emission limits; compare panel (a). There is a reduction in hours worked for high- and low-skilled labor over time as emission limits become stricter and also relative to a scenario without emission limit; see panels (b) and (c).\footnote{\ In appendix section \ref{app:eff_notarg}, I show how the social planner allocation with emission limit compares to the efficient allocation without limit. For the given calibration, the planner reduces hours worked once there is an emission limit.} 
The optimal allocation mimics the reduction in hours worked; yet, for the high-skilled the reduction is too strong starting from 2050. It is too small for low-skilled labor. 

The higher consumption levels at lower hours worked in the efficient allocation are driven by higher growth rates; see panel (d) which shows aggregate growth. The ratio of green to fossil energy, depicted in panel (e), moves similarly in the efficient and the optimal allocation. However, the Ramsey planner chooses a lower green-to-fossil energy mix starting from 2050. Utility of the representative household in the efficient and the optimal allocation is decreasing overall. The reduction happens in steps when a tighter emission limit becomes active. Utility is rising when the emission limit is stable and more so under the social planner. 

\subsubsection{Comparison other policy scenarios}
How does the optimal allocation and especially its relation to the efficient allocation change under alternative policy scenarios?
In this section, I discuss three policy alternations which have already been discussed in the analytical section. First, a version where environmental tax revenues are consumed by the government and no labor income tax scheme is available, henceforth referred to as \textit{separate policy}. The comparison of this scenario serves to assess the benefits of an integrated environmental-fiscal policy when no lump-sum transfers are available. 


Second, I show the optimal allocation in the scenario without redistribution but the government can use labor income taxes in comparison to the separate regime. Contrasting these two regimes highlights the benefits of labor income taxes  when environmental tax revenues are not redistributed. The results speak to the double-dividend literature arguing that there is a lower bound on the reduction in distortionary labor income taxes when environmental tax revenues are not redistributed lump-sum. \tr{Think of how to deal with channel that labor income taxes have the advantage to be redistributed....JUST DISCUSS IT AND THAT IT IS UNLIKELY;  }

Finally, I look at the optimal allocation when lump-sum transfers are available. According to the theory in section \ref{sec:mod_an}, the availability of lump-sum taxes should (i) deprive the income tax scheme of its use for the environmental policy, and (ii) result in an allocation closer to the efficient allocation than the benchmark scenario, i.e. the integrated policy. 


\paragraph{Comparison integrated policy to separate policy without lump-sum transfers}

Consider figure \ref{fig:bench_nored_notaul}. The figure presents the optimal allocation in the integrated policy scenario, the benchmark policy,  the orange-dashed graph, the optimal allocation under the separate policy, the blue-dotted graph, and the efficient allocation, the black-solid graph.

In comparison to a policy scenario where environmental tax revenues are not redistributed, the integrated policy closer resembles the efficient allocation in terms of consumption, panel (a) and of labor, panels (b) and (c). %In total, the utility level of the representative household is at least as close to the efficient level for all time periods considered. 
The benefits of an integrated policy regime come at the cost of a lower green-to-fossil energy mix. 

Interestingly, the optimal environmental tax is only negligibly smaller in the integrated policy regime. This suggests, that environmental taxes and labor income taxes are complements in the optimal environmental policy to lower inefficiently high hours worked. 

\begin{figure}[h!!]
	\centering
	\caption{Comparison to separate policy scenario; \tr{drop efficient from tauf graph }}\label{fig:bench_nored_notaul}
	
	\begin{minipage}[]{0.32\textwidth}
		\centering{\footnotesize{(a) Consumption}}
		%	\captionsetup{width=.45\linewidth}
		\includegraphics[width=1\textwidth]{../../codding_model/own_basedOnFried/optimalPol_190722_tidiedUp/figures/all_July22/C_CompEffOPT_T_NoTaus_pol2_spillover0_noskill0_sep1_xgrowth0_etaa0.79_lgd1_lff0.png}
	\end{minipage}
	\begin{minipage}[]{0.32\textwidth}
		\centering{\footnotesize{(b) High skill hours worked}}
		%	\captionsetup{width=.45\linewidth}
		\includegraphics[width=1\textwidth]{../../codding_model/own_basedOnFried/optimalPol_190722_tidiedUp/figures/all_July22/hh_CompEffOPT_T_NoTaus_pol2_spillover0_noskill0_sep1_xgrowth0_etaa0.79_lgd0_lff0.png}
	\end{minipage}
	\begin{minipage}[]{0.32\textwidth}
		\centering{\footnotesize{(c) Low skill hours worked}}
		%	\captionsetup{width=.45\linewidth}
		\includegraphics[width=1\textwidth]{../../codding_model/own_basedOnFried/optimalPol_190722_tidiedUp/figures/all_July22/hl_CompEffOPT_T_NoTaus_pol2_spillover0_noskill0_sep1_xgrowth0_etaa0.79_lgd0_lff0.png}
	\end{minipage}
	\begin{minipage}[]{0.32\textwidth}
		\centering{\footnotesize{(d) Aggregate growth}}
		%	\captionsetup{width=.45\linewidth}
		\includegraphics[width=1\textwidth]{../../codding_model/own_basedOnFried/optimalPol_190722_tidiedUp/figures/all_July22/gAagg_CompEffOPT_T_NoTaus_pol2_spillover0_noskill0_sep1_xgrowth0_etaa0.79_lgd0_lff0.png}
	\end{minipage}
	\begin{minipage}[]{0.32\textwidth}
		\centering{\footnotesize{(e) Energy mix, $\frac{G}{F}$}}
		%	\captionsetup{width=.45\linewidth}
		\includegraphics[width=1\textwidth]{../../codding_model/own_basedOnFried/optimalPol_190722_tidiedUp/figures/all_July22/GFF_CompEffOPT_T_NoTaus_pol2_spillover0_noskill0_sep1_xgrowth0_etaa0.79_lgd0_lff0.png}
	\end{minipage}
	%	\begin{minipage}[]{0.32\textwidth}
	%	\centering{\footnotesize{(f) Utility}}
	%	%	\captionsetup{width=.45\linewidth}
	%	\includegraphics[width=1\textwidth]{../../codding_model/own_basedOnFried/optimalPol_190722_tidiedUp/figures/all_July22/SWF_CompEffOPT_T_NoTaus_pol2_spillover0_noskill0_sep1_xgrowth0_etaa0.79_lgd0_lff0.png}
	%\end{minipage}
	\begin{minipage}[]{0.32\textwidth}
		\centering{\footnotesize{(f) Environmental tax, $\tau_{ft}$}}
		%	\captionsetup{width=.45\linewidth}
		\includegraphics[width=1\textwidth]{../../codding_model/own_basedOnFried/optimalPol_190722_tidiedUp/figures/all_July22/tauf_CompEffOPT_T_NoTaus_pol2_spillover0_noskill0_sep1_xgrowth0_etaa0.79_lgd0_lff0.png}
	\end{minipage}
\end{figure}


\paragraph{Comparison separate polcy with and without labor income tax}


In figure \ref{fig:comp_nored}, I contrast the optimal allocation in (i) the separate policy regime, orange-dotted graph, with (ii) a regime in which the government consumes environmental tax revenues but has the option to use a labor income tax. The solid black graph depicts the efficient allocation. 

These results speak to the double-dividend literature. When the government consumes environmental tax revenues completely - think of it as a scenario where environmental tax revenues are sufficient to meet the exogenous revenue constraint - then hours worked are inefficiently high; compare the orange-dashed to  the solid graph. The Ramsey planner can converge to the efficient allocation by employing a progressive labor income tax schedule. Indeed, this reduces consumption further away from the efficient level, but, hours worked are aligned closer to the efficient level, panels (b) and (c). Next to consumption, the planner also forfeits an advantageous green-to-fossil energy ratio, panel (e). 

The use of a progressive labor income tax contributes minimally to meeting the emission limit as can be seen by scrutinizing the optimal environmental tax, panel (b) in figure \ref{fig:comp_nored_pol}: when income taxes can be used, the environmental tax is slightly lower. Still, the difference is minimal, supporting the thesis of complementarity of income and environmental taxes. 
Environmental tax revenues are lower as the tax rate reduces, and income taxes reduce labor supply and hence the tax base of the environmental tax. 

This might be a motive to prefer labor income taxes as an instrument to reduce emissions since labor income tax revenues are redistributed back to the household while environmental tax revenues are not in this setting. Nevertheless, the observation that the environmental tax only adjusts slightly once an income tax tool is available points to the advantage of environmental taxes in handling too high emissions. 


%\tr{Could find that the labor income tax is used to reduce emissions because it does not reduce consumption! \ar maybe better to compare to a scenario where government revenues have to be equal to the env. tax revenues in the separate scenario, and then check if labor income taxes are still used? But this would reduce government revenues from the exogenous target! What would be a good comparison?}

\begin{figure}[h!!]
	\centering
	\caption{Comparison to separate policy scenario}\label{fig:comp_nored}
	
	\begin{minipage}[]{0.32\textwidth}
		\centering{\footnotesize{(a) Consumption}}
		%	\captionsetup{width=.45\linewidth}
		\includegraphics[width=1\textwidth]{../../codding_model/own_basedOnFried/optimalPol_190722_tidiedUp/figures/all_July22/C_DDCompEffOPT_T_NoTaus_pol3_spillover0_noskill0_sep1_xgrowth0_etaa0.79_lgd1_lff0.png}
	\end{minipage}
	\begin{minipage}[]{0.32\textwidth}
		\centering{\footnotesize{(b) High skill hours worked}}
		%	\captionsetup{width=.45\linewidth}
		\includegraphics[width=1\textwidth]{../../codding_model/own_basedOnFried/optimalPol_190722_tidiedUp/figures/all_July22/hh_DDCompEffOPT_T_NoTaus_pol3_spillover0_noskill0_sep1_xgrowth0_etaa0.79_lgd0_lff0.png}
	\end{minipage}
	\begin{minipage}[]{0.32\textwidth}
		\centering{\footnotesize{(c) Low skill hours worked}}
		%	\captionsetup{width=.45\linewidth}
		\includegraphics[width=1\textwidth]{../../codding_model/own_basedOnFried/optimalPol_190722_tidiedUp/figures/all_July22/hl_DDCompEffOPT_T_NoTaus_pol3_spillover0_noskill0_sep1_xgrowth0_etaa0.79_lgd0_lff0.png}
	\end{minipage}
	\begin{minipage}[]{0.32\textwidth}
		\centering{\footnotesize{(d) Aggregate growth}}
		%	\captionsetup{width=.45\linewidth}
		\includegraphics[width=1\textwidth]{../../codding_model/own_basedOnFried/optimalPol_190722_tidiedUp/figures/all_July22/gAagg_DDCompEffOPT_T_NoTaus_pol3_spillover0_noskill0_sep1_xgrowth0_etaa0.79_lgd0_lff0.png}
	\end{minipage}
	\begin{minipage}[]{0.32\textwidth}
		\centering{\footnotesize{(e) Energy mix, $\frac{G}{F}$}}
		%	\captionsetup{width=.45\linewidth}
		\includegraphics[width=1\textwidth]{../../codding_model/own_basedOnFried/optimalPol_190722_tidiedUp/figures/all_July22/GFF_DDCompEffOPT_T_NoTaus_pol3_spillover0_noskill0_sep1_xgrowth0_etaa0.79_lgd0_lff0.png}
	\end{minipage}
	\begin{minipage}[]{0.32\textwidth}
		\centering{\footnotesize{(f)Utility}}
		%	\captionsetup{width=.45\linewidth}
		\includegraphics[width=1\textwidth]{../../codding_model/own_basedOnFried/optimalPol_190722_tidiedUp/figures/all_July22/SWF_DDCompEffOPT_T_NoTaus_pol3_spillover0_noskill0_sep1_xgrowth0_etaa0.79_lgd0_lff0.png}
	\end{minipage}
\end{figure}

\begin{figure}[h!!]
	\centering
	\caption{Separate regime with and without income tax}\label{fig:comp_nored_pol}

\begin{minipage}[]{0.32\textwidth}
	\centering{\footnotesize{(a) Income tax progressivity, $\tau_{\iota t}$ }}
	%	\captionsetup{width=.45\linewidth}
	\includegraphics[width=1\textwidth]{../../codding_model/own_basedOnFried/optimalPol_190722_tidiedUp/figures/all_July22/taul_DDCompEffOPT_T_NoTaus_pol3_spillover0_noskill0_sep1_xgrowth0_etaa0.79_lgd1_lff0.png}
\end{minipage}
\begin{minipage}[]{0.32\textwidth}
	\centering{\footnotesize{(b) Environmental tax, $\tau_{ft}$}}
	%	\captionsetup{width=.45\linewidth}
	\includegraphics[width=1\textwidth]{../../codding_model/own_basedOnFried/optimalPol_190722_tidiedUp/figures/all_July22/tauf_DDCompEffOPT_T_NoTaus_pol3_spillover0_noskill0_sep1_xgrowth0_etaa0.79_lgd0_lff0.png}
\end{minipage}
\begin{minipage}[]{0.32\textwidth}
	\centering{\footnotesize{(c) Government consumption }}
	%	\captionsetup{width=.45\linewidth}
	\includegraphics[width=1\textwidth]{../../codding_model/own_basedOnFried/optimalPol_190722_tidiedUp/figures/all_July22/GovCon_DDCompEffOPT_T_NoTaus_pol3_spillover0_noskill0_sep1_xgrowth0_etaa0.79_lgd0_lff0.png}
\end{minipage}
\end{figure}

\paragraph{Comparison to model with lump-sum transfers}

When lump-sum transfers of environmental tax revenues are in the policy set, the Ramsey planner can implement an allocation closer to the efficient allocation but only minimally though. Figure \ref{fig:bench_lumpsum} contrasts the efficient allocation in black, the allocation under the benchmark policy, the blue-dotted graph, and the optimal allocation when lump-sum transfers are available in orange-dashed graph. 

Figure \ref{fig:bench_lumpsum_pol} shows the optimal policy the planner chooses when lump-sum transfers are available. 
Now, the optimal income tax scheme is regressive. Since lump-sum transfers ensure a reduction in labor supply the motive for progressive income taxes vanished. Instead, labor income taxes are used to 
\tr{Compare to version with lump sum transfers but without income taxes \ar what are the gains from labor income tax regressivity?} 



\begin{figure}[h!!]
	\centering
	\caption{Comparison integrates policy and lump-sum transferss}\label{fig:bench_lumpsum}
	
	\begin{minipage}[]{0.32\textwidth}
		\centering{\footnotesize{(a) Consumption}}
		%	\captionsetup{width=.45\linewidth}
		\includegraphics[width=1\textwidth]{../../codding_model/own_basedOnFried/optimalPol_190722_tidiedUp/figures/all_July22/C_CompEffOPT_T_NoTaus_pol4_spillover0_noskill0_sep1_xgrowth0_etaa0.79_lgd1_lff0.png}
	\end{minipage}
	\begin{minipage}[]{0.32\textwidth}
		\centering{\footnotesize{(b) High skill hours worked}}
		%	\captionsetup{width=.45\linewidth}
		\includegraphics[width=1\textwidth]{../../codding_model/own_basedOnFried/optimalPol_190722_tidiedUp/figures/all_July22/hh_CompEffOPT_T_NoTaus_pol4_spillover0_noskill0_sep1_xgrowth0_etaa0.79_lgd0_lff0.png}
	\end{minipage}
	\begin{minipage}[]{0.32\textwidth}
		\centering{\footnotesize{(c) Low skill hours worked}}
		%	\captionsetup{width=.45\linewidth}
		\includegraphics[width=1\textwidth]{../../codding_model/own_basedOnFried/optimalPol_190722_tidiedUp/figures/all_July22/hl_CompEffOPT_T_NoTaus_pol4_spillover0_noskill0_sep1_xgrowth0_etaa0.79_lgd0_lff0.png}
	\end{minipage}
	\begin{minipage}[]{0.32\textwidth}
		\centering{\footnotesize{(d) Aggregate growth}}
		%	\captionsetup{width=.45\linewidth}
		\includegraphics[width=1\textwidth]{../../codding_model/own_basedOnFried/optimalPol_190722_tidiedUp/figures/all_July22/gAagg_CompEffOPT_T_NoTaus_pol4_spillover0_noskill0_sep1_xgrowth0_etaa0.79_lgd0_lff0.png}
	\end{minipage}
	\begin{minipage}[]{0.32\textwidth}
		\centering{\footnotesize{(e) Energy mix, $\frac{G}{F}$}}
		%	\captionsetup{width=.45\linewidth}
		\includegraphics[width=1\textwidth]{../../codding_model/own_basedOnFried/optimalPol_190722_tidiedUp/figures/all_July22/GFF_CompEffOPT_T_NoTaus_pol4_spillover0_noskill0_sep1_xgrowth0_etaa0.79_lgd0_lff0.png}
	\end{minipage}
	\begin{minipage}[]{0.32\textwidth}
		\centering{\footnotesize{(f) Utility}}
		%	\captionsetup{width=.45\linewidth}
		\includegraphics[width=1\textwidth]{../../codding_model/own_basedOnFried/optimalPol_190722_tidiedUp/figures/all_July22/SWF_CompEffOPT_T_NoTaus_pol4_spillover0_noskill0_sep1_xgrowth0_etaa0.79_lgd0_lff0.png}
	\end{minipage}
\end{figure}

\begin{figure}[h!!]
	\centering
	\caption{Optimal policy with and without lump-sum transfers}\label{fig:bench_lumpsum_pol}
	
	\begin{minipage}[]{0.32\textwidth}
		\centering{\footnotesize{(a) Income tax progressivity, $\tau_{\iota t}$}}
		%	\captionsetup{width=.45\linewidth}
		\includegraphics[width=1\textwidth]{../../codding_model/own_basedOnFried/optimalPol_190722_tidiedUp/figures/all_July22/comp_notaul4_OPT_T_NoTaus_taul_spillover0_noskill0_sep1_xgrowth0_etaa0.79_lgd1.png}
	\end{minipage}
	\begin{minipage}[]{0.32\textwidth}
		\centering{\footnotesize{(b) Environmental tax, $\tau_{ft}$}}
		%	\captionsetup{width=.45\linewidth}
		\includegraphics[width=1\textwidth]{../../codding_model/own_basedOnFried/optimalPol_190722_tidiedUp/figures/all_July22/comp_notaul4_OPT_T_NoTaus_tauf_spillover0_noskill0_sep1_xgrowth0_etaa0.79_lgd0.png}
	\end{minipage}
	\begin{minipage}[]{0.32\textwidth}
		\centering{\footnotesize{(c) Lump-sum transfers}}
		%	\captionsetup{width=.45\linewidth}
		\includegraphics[width=1\textwidth]{../../codding_model/own_basedOnFried/optimalPol_190722_tidiedUp/figures/all_July22/comp_notaul4_OPT_T_NoTaus_Tls_spillover0_noskill0_sep1_xgrowth0_etaa0.79_lgd0.png}
	\end{minipage}
\end{figure}


\subsection{No heterogeneous skills, no endogenous growth}
\ar Does integrated policy come close to efficient one or the one with lump-sum transfers (should be the case according to theory)

	\section{Conclusion}\label{sec:con}
Some scholars argue that  reductive policies are necessary to handle environmental limits \citep{Schor2005SustainableReductionb, VanVuuren2018AlternativeTechnologies, Bertram2018TargetedScenarios}, and the question has been raised whether consumption is too high \citep{Arrow2004AreMuch}. On the other hand, the focus of environmental policy discussions in economics rests on corrective environmental taxation. In the light of tightening environmental limits \citep{Rockstrom2009AHumanity, IPCC2022}, I study whether labor income taxes - as a reductive policy tool - can help mitigate externalities. 

In the analytical part of the paper, I show in a simple model that labor income taxes are  progressive as part of the optimal environmental policy. %The model does not feature inequality.
% Quantitative results
% baseline model
When environmental tax revenues are not redistributed lump sum, labor supply is inefficiently high. Then, income taxes serve to diminish hours worked closer to the efficient level. The result prevails absent income inequality.


% quantitative
In the second part of the paper, I analyze in a quantitative model with skill heterogeneity and endogenous growth whether the optimal labor income tax remains progressive. Again, there are no equity concerns, but workers are perfectly ensured against income differences. 
The optimal income tax is progressive to reduce inefficiently high hours worked. The quantitative model reveals that income taxes also serve as a substitute for corrective taxes. Knowledge spillovers from the non-energy sector render environmental taxes especially costly. 
Fossil taxes make energy relatively more expensive which directs research from non-energy to energy sectors. As the non-energy sector features the most research processes it is especially important for aggregate technology and knowledge spillovers. Using income taxes instead of fossil taxes to lower emissions allows to direct more research to the non-energy sector and to profit from knowledge spillovers.
In sum, however, the reduction in labor supply outweighs the positive effect on growth and consumption decreases compared to a scenario where no income tax is used. 

In the quantitative setting, the income tax affects the economic structure through two channels. First, because the fossil sector is comparably labor intense, a reduction in labor supply favors the green sector. This mechanism makes a higher tax progressivity optimal. However, the effect vanishes in equilibrium due to endogenous growth.
Second, a skill-recomposition channel makes green energy production more costly compared to fossil production. This effect arises from a skill bias in the green sector and high-skill labor being more responsive to income taxation. 
The second channel dominates the recomposing effect of  income tax progressivity in equilibrium. A market size effect amplifies the skill-recomposition channel directing research to the fossil sector. 

%Initially, the intention not to harm growth too much makes a lower progressivity optimal. As growth in the fossil sector accelerates due to the dynamic structure of endogenous growth too low progressive income taxes conflict with meeting the emission limit. As a result, optimal progressivity increases over time.
%The optimal path of income tax progressivity is decreasing, a feature mainly driven by endogenous growth. As a result, the optimal income tax progressivity and the optimal fossil tax seem to behave like substitutes in the quantitative model. 

%Skill heterogeneity depresses optimal tax progressivity due to the adverse recomposing effect of a lower high-to-low skill labor supply on the green-to-fossil energy ratio. A higher corrective tax is required to meet emission limits when there is only one skill type: with only one skill the supply of fossil-specific inputs increases thereby violating the emission limit.

%% lump-sum transfers
%When environmental tax revenues are redistributed lump-sum, the motive to use labor income taxes to deal with inefficiently high labor supply vanishes. Instead, income taxes serve to boost growth as long as this does not conflict with meeting emission limits. Therefore, they are regressive. 
%\tr{not true! it is rather that the more in research is not worth it given the dynamics! and decreasing utility gains}
%However, the regressivity decreases since more labor supply causes more emissions especially the more progressed the technology. With only a labor income tax as a tool to raise growth, accelerating technology growth is not feasible as it is concomitant with more production and emissions. 

% extensions
In an extension, I am planning to give the Ramsey planner the opportunity to limit working hours directly. The literature advocating a reduction in consumption levels \citep[e.g.,][]{Schor2005SustainableReductionb} proposes a restriction of hours worked as policy instrument to lower the consumption of resources.
Even though advocated in the literature, there is evidence for political difficulties in reducing working hours. In 2020, the French Citizens' Convention on Climate voted against reducing working hours as a measure to handle climate change. Potentially, ignorance about economic consequences is an explanation. The extension would serve to better understand economic consequences. 



\begin{comment}
\paragraph{Extension: What if the low skilled get a higher share \ar they reduce even less \ar more fossil input supply}

Redistribution to households with a higher marginal propensity to consume emissions counteracts the externality. This effect is amplified by a market size effect  of dirty goods. 

content...
\end{comment}

% I plan to discuss results under counterfactual parameter values to elicit the robustness of the main result: the preference of progressive labour taxation above higher fossil taxes. 
%First, the productivity gap between sectors might be driving the results. Second, I will abstract from endogenous growth to learn about the labour-supply-innovation channel as a driver of the optimal policy. Finally, I plan to study how results change as returns to research are increasing within sector. 
%Due to the endogeneity of technological growth in the model, the reduction in work effort fosters less research especially in the non-energy sector.  %However, more hours worked in the Ramsey model fostering research would violate the emission target. As a result, growth in technology and in consumption is inefficiently low in order to meet the emission target. 

\begin{comment}
To shed more light on the main findings, I plan conduct several additional quantitative experiments. First, I want to reduce the size of the emission target, second, I allow for a longer time frame until net-zero emissions have to be reached. The IPCC report states that for a temperature target of 2°C net-zero emissions have to be reached by 2070 only. How does this laxer target affect the importance of labour income taxes. Given the wider time frame, the green sector might be able to catch up and growth could continue. Finally, how does a change in spillovers shape the result? % \textit{(Question: I guess that substitutability is key here! Growth in green implies growths in fossil when goods are no perfect substitutes! )}
content...

%Another central aspect of the paper is the importance of inequality for the optimal environmental policy. How does household heterogeneity in labour supply shape the optimal environmental policy? First, I hypothesise that the skill bias of the green sector makes a less progressive income tax optimal. 
One main result of the paper is reduction of consumption and work effort as an optimal policy. So far, I have assumed that households are passive and preferences are fixed; there is no trade-off between environmental quality and consumption from a household perspective.
In an extension to the baseline model, I plan to depart from the representative agent assumption and explicitly model household heterogeneity. This setting allows to capture a change in household behaviour: A share of households is willing to voluntarily reduce consumption. I provide evidence for such behaviour using a representative Dutch dataset. More than 50\% of households are willing to reduce consumption in order to help the economy. Importantly, these households have a higher likelihood to work in the green sector. How does such a change in behaviour affect the optimal policy? Given the additional reduction in green-specific labour supply, the planner might find it optimal to set a more regressive tax to booster green production and research.    

\end{comment}

%However, data suggests, that households do care, and they express a willingness to reduce consumption.\footnote{\ The data I have studied comes from the Liss Panel, a representative sample of Dutch households, more than 50\% of participants indicate a readiness to change their behaviour to help the environment.} I want to study the effect of such behavioural  change on the optimal policy. Interestingly, households in high-skill jobs are more likely to declare their willingness to reduce. This linkage may intensify the trade-off between reduction and green labour supply. 


%1) BN and inequality
%2) preferences for labour
\begin{comment}
Preferences and the trade-off between leisure and consumption determining household behaviour seem to be key to the results. As argued by \cite{Boppart2019labourPerspectiveb}, the intensive margin of hours worked have been falling steadily over the last 130 years. They argue for the consistency of preferences which feature a slightly higher income effect than substitution effect. In the current model with log-utility and representative family framework,  the substitution effects offset each other. With the preferences suggested in \cite{Boppart2019labourPerspectiveb}, growth would affect hours worked, assumably changing the optimal policy. It could, for instance, be the case, that growth has to be slowed down even more, to prevent too high work efforts and consumption levels. % high-income, high-skill households might increase their labour supply with growth. 

content...
\end{comment}



%Finally, endogenising growth constitutes another interesting trade-off when the impact of fiscal policy is skill specific. 
%As regards growth, it seems reasonable to consider growth as a change in the substitutability of dirty and clean goods in the final consumption good. As it stands now, growth in the dirty sector results in emission growth, ceteris paribus. Growth might instead be associated with a more efficient use of dirty energy sources, so that more output can be generated at lower emissions.
%
%Think about effects of government using revenues for other consumption. Then reducing demand will diminish demand for the final good. 
%Broadly speaking, there are two channels through which distortionary labour taxation affects emissions. First, by affecting households' labour supply decision (efficiency channel) and second in a mechanical way by changing households disposable income. The latter effect cancels out when tax revenues are used by the government to consume the final output good. Allowing the government to recycle revenues in a different way than for final good consumption uncloses another instrument to reduce emissions. 

%Further ideas for extensions: include behavioural aspects: a voluntary reduction in demand, and a lower disutility from working in the green sector.
\begin{comment}
\paragraph{Ways forward}
How to introduce compositional effects:
\begin{enumerate}
	\item 	Utility function: With substitution and income effect not canceling (u(c)=$\frac{c^{1-\gamma}}{1-\gamma},\ \gamma\neq 1$), the wage rate might play a role, depends on GE effects.
	\item endogenising skill supply (rep agent chooses how much skill to supply, but this he already does... / might need to introduce structure as in HSV)
	\item government revenues are not used for final good consumption. Instead,  disposed of/ used for sth useful (this could be an extension and contribute to benefits of progressivity) THINK THIS ONLY CHANGES THE LEVEL TOO!
\end{enumerate}
\paragraph{Point 1 above}
change the utility function in the code to see what happens, if $\frac{Y_d}{Y_c}$ is constant in particular 
\paragraph{Point 3 above}
\textcolor{blue}{2) Government consumption wasted}
Letting the government not consume the final output good may alter the result. 
Now, the aggregate price level is determined endogenously as the goods market does not clear by Walras' law. 

In the equilibrium equations, I drop $p_t=1$ and use goods market clearing instead\\ $Y=c+\psi (x_c+x_d)$.

Blödsinn, only changes level

content...
\end{comment}
	\clearpage
\appendix
\section{Derivations and proofs}\label{app:derivations}

\subsection{Theory results \ref{sec:mod_an}}
\subsubsection{Useful relations in the simple model}\label{app:dervs_use}
\begin{align*}
\frac{\partial Gov}{\partial s}=\frac{\partial Y}{\partial F}\frac{\partial F}{\partial s}+\frac{\partial Y}{\partial G}\frac{\partial G}{\partial s}-\frac{\partial C}{\partial s}\\
\frac{\partial Gov}{\partial H}=\frac{\partial Y}{\partial F}\frac{\partial F}{\partial s}+\frac{\partial Y}{\partial G}\frac{\partial G}{\partial s}-\frac{\partial C}{\partial H}\\
\frac{\partial Gov}{\partial s}=\frac{\partial Y}{\partial s}-\frac{\partial C}{\partial s}\\
\frac{\partial Gov}{\partial s}=p_f F \frac{\partial \tau_F}{\partial s}+\tau_F F \frac{\partial p_f}{\partial s}+\tau_F p_f \frac{\partial F}{\partial s}\\
%\frac{\partial \tau_F}{\partial s}= -\frac{1-\varepsilon}{\varepsilon}\frac{1}{(1-s)^2}, \\
%\frac{\partial \tau_F}{\partial s}=p_f\frac{1-\varepsilon}{1-\tau_F}\frac{\partial \tau_F}{\partial s}\\
\frac{\partial F}{\partial s}=\frac{F}{s}
\\
\frp{Y}{H}= \frp{Y}{s}\frac{s}{H}+\frp{Y}{G}\frp{G}{Lg}
\\
\frp{G}{H}=-\frac{(1-s)}{H}\frp{G}{s}
\\\frp{G}{s}=-H\frp{G}{L_G}\\
\frp{F}{H}=\frac{s}{H}\frp{F}{s}\\
\frp{F}{s}=H\frp{F}{L_F}
\end{align*}

\subsubsection{Reduction in dirty labor share is efficient}
\begin{proof}
	With a negative externality of dirty production it has to hold that 
	\begin{align}
	\frp{Y}{F}\frp{F}{s}>-\frp{Y}{G}\frp{G}{s},
	\end{align}
	which can be rewritten to 
	\begin{align}\label{eq:mpl_eff}
	\frp{Y}{L_F}>\frp{Y}{L_G}. 
	\end{align}
	In the efficient allocation absent externality, marginal products of dirty and green labor are equalized. 
	Under decreasing returns to scale it holds that the left-hand side is decreasing in $L_F$ and the right-hand side of equation \ref{eq:mpl_eff} is decreasing in $L_G$. Hence, the adjustment to satisfy equation \ref{eq:mpl_eff} relative to the efficient allocation without externality requires a decrease in $L_F$ and/or a rise in $L_G$  .
	This reallocation is achieved by reducing $s$, since $L_F=sH$ and $L_G=(1-s)H$.	
\end{proof}


\begin{comment}
content...
\paragraph{If a reduction in dirty labor share is efficient, then the aggregate production function features decreasing returns to scale in labor}
\begin{proof}
	\textit{The proof rest on the assumption that returns to scale are symmetric across dirty and clean production; either both decreasing or both are non-decreasing.}
It holds by assumption that $s_{FB,E>0}<s_{FB,E=0}$, where $E>0$ indicates that the externality is active. 
Assume by contradiction that the aggregate production function features non-decreasing returns to scale. This implies that:
\begin{align}
\left. \frp{Y}{L_F} \right|_{s_{FB,E>0}}\leq \left. \frp{Y}{L_F} \right|_{s_{FB,E=0}},\\
\left. \frp{Y}{L_G} \right|_{s_{FB,E>0}}\geq \left. \frp{Y}{L_G} \right|_{s_{FB,E=0}}.
\end{align}
When there is no externality, the efficient allocation is characterized by
\begin{align}
\left. \frp{Y}{L_F} \right|_{s_{FB,E=0}}= \left. \frp{Y}{L_G} \right|_{s_{FB,E=0}}.
\end{align}
Using the inequalities above yields
\begin{align}
\left. \frp{Y}{L_F} \right|_{s_{FB,E>0}}\leq \left. \frp{Y}{L_G} \right|_{s_{FB,E>0}}.
\end{align}
This contradicts the optimality condition which requires 
\begin{align}
\left. \frp{Y}{L_F} \right|_{s_{FB,E>0}}> \left. \frp{Y}{L_G} \right|_{s_{FB,E>0}}.
\end{align}
Hence, when a reduction in the dirty labor share is efficient, then the aggregate production function features decreasing returns to scale in both labor input goods. 
\end{proof}
\end{comment}

\subsubsection{The social cost of pollution and the Pigouvian tax rate}\label{app:scp}

The social cost of pollution in my model is defined as the marginal price the representative household is willing to pay for a marginal reduction in dirty production. That is, the household maximises over dirty production for which a market exists.

The household's problem is determined as
\begin{align}
\underset{C,H,F}{\max} U(C,H,F)-\mu \left(C+\tilde{p}_FF-Y(H)\right).
\end{align}
Where $\mu$ is the Lagrange multiplier. Taking the derivative with respect to dirty production  and with respect to consumption yields
\begin{align}
U_F=\mu \tilde{p}_F,\\
U_C=\mu.
\end{align}
Substituting the Lagrange multiplier gives the negative of the equilibrium price the household is willing to pay for a reduction in dirty prodction: $\tilde{p}_F=\frac{U_F}{U_C}$. Since the environmental tax in the model is a percentage of revenues, the price producers pay per unit of dirty production is $\tau_F p_F$. Thus, the social cost of pollution to be deducted from to producers' revenues in percent is $\tau^{Pigou}=\frac{-U_F}{U_Cp_F}$.


\subsubsection{With a positive environmental tax, the wage rate in the competitive equilibrium is below the marginal product of labor}\label{app:wageMPL}

The aggregate marginal product of labor is defined as
\begin{align}
MPL&= \frp{Y}{H}.
\end{align}
This expression can be rewritten using relations of derivatives summarized in \ref{app:dervs_use} as follows.
\begin{align}
&= \frp{Y}{F}\frp{Y}{H}+\frp{Y}{G}\frp{G}{H}\\
&= \frp{Y}{F}\frp{F}{L_F}s+\frp{Y}{G}\frp{G}{L_G}(1-s)\\
&= \frp{Y}{G}\frp{G}{L_G}+ s\left(\frp{Y}{F}\frp{F}{L_F}-\frp{Y}{G}\frp{G}{L_G}\right).\label{eq:mpl_opt}
\end{align}
The term in brackets is positive under the optimal policy as can be seen from the first order condition with respect to $s$, equation \ref{eq:sbs}:
\begin{align}
\frp{Y}{F}\frp{F}{L_F}-\frp{Y}{G}\frp{G}{L_G}=\frac{1}{H}\left(\frp{Y}{F}\frp{F}{s}+\frp{Y}{G}\frp{G}{s}\right)=\frac{1}{H}\left(\frac{-U_F\frp{F}{s}}{U_C}\right)>0.
\end{align}
The inequality holds since the externality of polluting production is negative. %, above expression is positive.
%Therefore, the marginal product of labor in the efficient allocation equals
Now note that the first summand in equation \ref{eq:mpl_opt} is the competitive wage rate.  Hence $w<MPL$.

The gap between the wage rate and the marginal product of labor equals the gap between the marginal products of labor across sectors times the relative size of the dirty sector. 

\subsubsection{Sufficiency of the environmental tax when environmental tax revenues are redistributed lump sum}\label{app:incometax0}

Noticing that $\frac{\partial Y}{\partial H}= \frac{\partial Y}{\partial s}\frac{s}{H}-\frac{\partial Y}{\partial G}\frac{\partial G}{\partial s}\frac{1}{H}$ and that $\frac{\partial F}{\partial H}=\frac{\partial F}{\partial s}\frac{s}{H}$, and substituting equation \ref{eq:sbs} in equation \ref{eq:sbh} yields
\begin{align}\label{eq:pigou}
-U_C \frac{\partial Y}{\partial G}\frp{G}{L_G}=-U_H.
\end{align}
Hence, if the environmental tax is set to guarantee that condition \ref{eq:sbh} holds, then optimal hours worked only trade-off the disutility from labor and the utility from more consumption when environmental tax revenues are redistributed lump-sum.

Equation \ref{eq:pigou} also holds for the social planner allocation simplifying the second first order condition, equation \ref{eq:fbh}.


Substituting $U_H$ from household optimality, equation \ref{eq:hsup}, and the clean sectors' profit maximizing condition from equations \ref{eq:profmax} yields
\begin{align}
1=1-\tau^*_\iota.
\end{align}
Hence, $\tau^*_\iota =0$ from which follows that $\lambda =1$ so that the income tax scheme is a flat tax rate equal to zero; the labor income tax is not used in optimum.

%\subsubsection{Simplifying social planner's first order conditions}
%
%The social planner's first order condition on labor can be rewritten as in the previous section to
%\begin{align}
%-U_H=U_C\frac{\partial Y}{\partial G}\frp{G}{L_G}
%\end{align}
\subsubsection{Proof proposition \ref{prop:1}: Absent lump-sum transfers, hours are inefficiently high under decreasing returns to scale}\label{app:nolumpsum_hourshigh}
\begin{proof}\textit{Absent lump-sum transfers, hours are inefficiently high when the environmental tax implements efficient share of dirty production and the aggregate production function features decreasing returns to scale in labor inputs.}
	
	This proof proceeds by contradiction. 
	Assume by contradiction that $H^*\leq H_{FB}$. 
	It has to hold that 
	\begin{align}
	-U_H^*\leq -U_{H,FB}.
	\end{align} 
	
	Substituting the households' optimal labor supply and the social planner's first order condition for hours, equation \ref{eq:fbh_simp} yields
	\begin{align}\label{eq:prH}
	U_C^*w^* \leq U_{C,FB}\frp{Y_{FB}}{G_{FB}}\frp{G_{FB}}{L_{G,FB}}.
	\end{align}
	
	Rewriting equation \ref{eq:prH} above yields
	\begin{align}
	\frac{U_C^*}{U_{C,FB}}\leq \frac{\frp{Y_{FB}}{G_{FB}}\frp{G_{FB}}{L_{G,FB}}}{\frp{Y^*}{G^*}\frp{G^*}{L^*_{G}}},
	\end{align}
	where I replaced $w^*=\frp{Y^*}{G^*}\frp{G^*}{L^*_{G}}$.
	
	By assumption $s^*=s_{FB}$, $H^*\leq H_{FB}$, and the aggregate production function is increasing in its inputs. It follows that output is higher in the efficient allocation $Y_{FB}\geq Y^*$ and hence $C^*<C_{FB}$, since $Gov>0$ in the competitive equilibrium. By additive separability of the utility function and strict concavity with respect to consumption, we have that $\frac{U_C^*}{U_{C,FB}}>1$.
	
	Now note that $H^*\leq H_{FB}$ implies  $L_G^*\leq L_{G,FB}$, since the dirty labor share is equal. Under decreasing returns to scale of aggregate production to clean labor, it holds that the right-hand side is below or equal unity.Thus,
	\begin{align}
	\frac{U_C^*}{U_{C,FB}}>1\geq \frac{\frp{Y_{FB}}{G_{FB}}\frp{G_{FB}}{L_{G,FB}}}{\frp{Y^*}{G^*}\frp{G^*}{L^*_{G}}}. 
	\end{align}
	A contradiction to the assumption that $H^*\leq H_{FB}$. Hence, it has to hold that $H^*>H_{FB}$. 
\end{proof}


\subsubsection{Derivation $\tau_F^*$ without lump-sum transfers}\label{app:reiv_tauf}
	
Divide the Ramsey planner's first order condition with respect to $s$, equation \ref{eq:sbs}, by $U_C$ and $\frp{Y}{F}\frp{F}{s}$. Solving for $1+\frac{\frac{\partial Y}{\partial G}\frac{\partial G}{\partial s}}{\frac{\partial Y}{\partial F}\frac{\partial F}{\partial s}}$, which equals $\tau_F$, yields the desired result:

\begin{align}
\tau_{F}=SCC + \frp{Gov}{s}.
\end{align}

\begin{comment}
The latter summand can be rewritten to 
\begin{align}
\frp{Gov}{s}= \frp{Y}{s}+H^2 \frp{\left(\frp{Y}{L_G}\right)}{L_G}.
\end{align}
Where under decreasing returns to scale the second summand is negative and the first is positive. \textit{To be continued.} 

content...
\end{comment}
\subsubsection{Derivation $\tau_l$ without lump-sum transfers }\label{app:subsub_nltaul}

\begin{proof}\textit{Absent lump-sum transfers, the optimal income tax scheme is progressive}
Following similar steps as in section \ref{app:incometax0}, the optimal labor income tax progressivity parameter is given by
\begin{align}
\tau_{\iota}^*=\frac{\frac{s}{H}\frac{\partial Gov}{\partial s}- \frac{\partial Gov}{\partial H}}{\frac{\partial Y}{\partial G}\frac{\partial G}{\partial s}\frac{1}{H}}.
\end{align}

	Using the market clearing condition for final output to replace government spending and noticing the relations of derivatives with respect to aggregate labor supply and the dirty labor share, one can write above expression as
	\begin{align}
	\tau_{\iota}=1-\frac{H\frp{C}{H}-s\frp{C}{s}}{wH}.
	\end{align}
	Substituting $\frp{C}{H}=H\frp{w}{H}+w$ and $\frp{C}{s}=H\frp{w}{s}$ from the household's budget constraint gives
	\begin{align}
	\tau_{\iota}=\frac{s}{w}\frp{w}{s}-\frac{H}{w}\frp{w}{H}.
	\end{align}
In a next step, I explicitly solve for $\frp{w}{s}$ and $\frp{w}{H}$, where I use that $w=\frp{Y}{G}\frp{G}{L_G}$ in equilibrium.

\begin{align}
\frp{w}{H}=\left(\frp{G}{L_G}\right)^2\frac{\partial^2Y}{\partial G^2}(1-s)+\frp{Y}{G}\frac{\partial ^2G}{\partial L_G^2}(1-s)+\frp{G}{L_G}\frac{\partial^2 Y}{\partial G \partial F}s\\
%%%%
\frp{w}{s}= \left(\frp{G}{L_G}\right)^2\frac{\partial ^2Y}{\partial G^2}(-H)+\frp{G}{L_G}\frac{\partial ^2Y}{\partial G \partial F}H+\frp{Y}{G}\frac{\partial ^2 G}{\partial L_G^2}(-H)
\end{align}
substituting derivatives and canceling terms yields:
\begin{align}
\tau_\iota= -\frac{H}{w}\frp{\left(\frp{Y}{L_G}\right)}{L_G}.=-\frac{H}{w}\left(\left(\frp{G}{L_G}\right)^2\frac{\partial ^2Y}{\partial G^2}+\frp{Y}{G}\frac{\partial ^2G}{\partial L_G ^2}\right).
\end{align}
Under the assumption of decreasing returns to scale of aggregate production with respect to green labor the term in brackets is negative, and it holds that $\tau_\iota >0$ and the optimal income tax rate is progressive. 

For intuition, note that the right-hand side of the previous expression equals the partial derivative of the wage rate with respect to the dirty labor share under the assumption that dirty production is fixed divided by the wage rate:
\begin{align}
\tau^*_\iota =\left. \frac{1}{w}\frp{w}{s} \right|_{F=\bar{F}}.
\end{align}
%Since the presence of the environmental tax artificially increases labor in the green sector depressing the wage rate (under the assumption of decreasing returns to scale), the wage rate rises by a reduction of the green labor share. 

The equation makes clear that environmental taxation and the labor income tax are complements. When the environmental tax rises, thereby increasing the share of labor allocated to the green sector, the marginal product of green labor decreases further. A marginal reduction in the green labor share would increase the wage rate more the higher the green labor share, hence, the optimal labor tax progressivity increases with the environmental tax. 
Secondly, the wage rate decreases with $\tau_F$ which as well inflates the optimal labor tax progressivity. 
	\end{proof}

\subsubsection{Proof proposition: Infeasibility of efficient allocation}\label{app:ineff}
\begin{proof}\textit{The efficient allocation is infeasible (under the assumption of constant or decreasing returns to scale)}
	To prove this claim, I assume that the government chooses the optimal policy; which is the highest social welfare the Ramsey planner can achieve. I show that the optimal policy does not satisfy the social planner's allocation. Since the social planner could have chosen the Ramsey planner's allocation  but did not, it follows that the social planner's allocation features a higher social welfare.
	
	For the optimal allocation to be efficient, it must be the case that $s^*=s_{FB}$, (i) $C^*=C_{FB}$, and (ii) $H^*=H_{FB}$. I show that, under the assumption that $s^*=s_{FB}$, either (i) or (ii) can hold at a time by demonstrating that assuming (i) violates (ii) and vice versa.
	
	
	
	%\begin{lemma}\textit{$\tau_F=0$ is not optimal}
	%When $\tau_F=0$ then $Gov=0$ and $\frp{Gov}{s}=0$. Furthermore, market forces then imply that the marginal products of labor are equal so that $\frp{Y}{F}\frp{F}{s}=-\frp{Y}{G}\frp{G}{s}$. Substituting this in equation \ref{eq:sbs} yields
	%\begin{align}
	%0=-U_F\frp{F}{s}>0,
	%\end{align}
	%a contradiction. 
	%\end{lemma}
	%
	%\textit{(i) Assume $C^*=C_{FB}$ and $s^*=s_{FB}$:}
	%Since $\tau_F\neq0$, it follows that $Gov>0$ and hence $Y^*=C^*+Gov>Y_{FB}$. Since the allocation of labor is the same in the efficient and the optimal allocation and output is rising in labor, it follows that $H^*>H_{FB}$. 
	%\tr{Missing: if $\tau_F<0$ then $Gov<0$ }
	
	If $s^*=s_{E>0,FB}<s_{E=0,FB}$ then it must be the case that the environmental tax is positive to sustain a gap between marginal productivities in the dirty and the clean sector: $\tau_F>0$. Then, $Gov=\tau_Fp_fF>0$. 
	First assume that (i) holds true: $C^*=C_{FB}$. From the good's market clearing condition and resource constraint of the social planner's problem it follows that
	$Y^*-Gov=C^*=C_{FB}=Y_{FB}$, due to  $Gov>0$ we have that $Y^*>Y_{FB}$. Since hours are the only production input, positively affect output, and $s^*=s_{FB}$ the higher output in the optimal allocation implies that $H^*>H_{FB}$. A violation of condition (ii). 
	
	Assume now that condition (ii) holds: $H^*=H_{FB}$. Since $s^*=s_{FB}$ by assumption it holds that $Y^*=Y_{FB}$ and, by the same argument as before: $Gov>0$. Thus, by the resource and market clearing condition: $C_{FB}=Y_{FB}>Y^*-Gov=C^*$. When labor supply is efficient, then consumption is inefficiently low; condition (i) is violated. 
	
	\begin{comment} (Proof building on first order conditions)
	Assume, 
	The social planner's first order condition on labor supply can be written as
	\begin{align}
	-U_{H, FB}=U_{C, FB}\frp{Y}{G}_{FB}\frp{G}{L_G}_{FB}
	\end{align}
	and optimal labor supply is determined by
	\begin{align}
	-U^*_{H}&=U^*_C(1-\tau_\iota)w
	\end{align}
	Equalizing yields
	\begin{align}
	U_C^*(1-\tau_\iota)w=U_{C,FB}\frp{Y}{G}_{FB}\frp{G}{L_G}_{FB},
	\end{align}
	a condition for optimal labor supply to be efficient. 
	
	In the following, I demonstrate that (i) assuming $C^*=C_{FB}$ violates the condition above and $H^*\neq H_{FB}$ and that (ii) assuming $H^*=H_{FB}$ results in $C^*<C_{FB}$. 
	
	\textit{(i) Assume $C^*=C_{FB}$:}
	then
	\begin{align}
	(1-\tau_\iota)w=\frp{Y}{G}_{FB}\frp{G}{L_G}_{FB}.
	\end{align}
	Assume by contradiction that $H^*=H_{FB}$, since $s^*=s_{FB}$ by assumption, it follows that $w=\frp{Y}{G}_{FB}\frp{G}{L_G}_{FB}$. 
	Since $\tau_\iota\neq 0$ under constant or decreasing returns to scale, it holds that $H^*<H_{FB}$, a contradiction. 
	
	
	%Hence,
	%\begin{align}
	%(1-\tau_\iota)w<\frp{Y}{G}_{FB}\frp{G}{L_G}_{FB}.
	%\end{align}
	%
	%Labor supply in the competitive equilibrium is lower than in the efficient allocation when consumption is equal under the optimal policy. WHY?
	%It follows, that optimal labor supply does not equal its efficient counterpart when optimal consumption is efficient.
	
	\textit{(ii) Assume $H^*=H_{FB}$:} 
	It follows that 
	\begin{align}
	\frac{U_C^*}{U_{C,FB}}=\frac{\frp{Y}{G}_{FB}\frp{G}{L_G}_{FB}}{w}\frac{1}{1-\tau_\iota}=\frac{1}{1-\tau_\iota}>1.
	\end{align}
	From concavity of the utility function it follows that $C^*<C_{FB}$. 
	
	content...
	\end{comment}
\end{proof}

\subsubsection{Proofs proposition \ref{prop:3}}\label{app:proofintegrated}
\begin{proof} \textit{The optimal income tax scheme is progressive}\\ % if the optimal environmental tax is positive.}\\
	Under the new policy, the household's labor supply is determined by
	\begin{align}
	-U_H=\frac{U_C (1-\tau_{\iota})(wH+\tau_F p_fF)}{H}.
	\end{align}
	Expressing the derivatives in the Ramsey planner's first order condition with respect to hours as derivatives with respect to the dirty labor share, $s$, and substituting the first order condition with respect to $s$ yields:
	\begin{align}
	U_C \frp{Y}{G}\frp{G}{L_G}=-U_H.
	\end{align}
	%This equation is equivalent to the social planner's first order condition on hours, equation \ref{eq:fbh}. The optimal policy is to choose
	%\tr{Does this give a hint to why inefficiency without redistribution? The Ramsey planner's foc and household optimality always coincide. But, when Gov does not cancel the two do not coincide! ? the two do not coincide, Bcs consumption is too low so that $U_C$ too high which increases}
	Noticing that $\frp{Y}{G}\frp{G}{L_G}=w$ and replacing household's labor supply condition gives
	\begin{align}
	& w=\frac{(1-\tau_\iota)Y}{H}\\
	\Leftrightarrow\ & \tau_\iota=1-\frac{wH}{Y}. 
	\end{align} 
	Since $Y=C=wH+\tau_Fp_fF$ from the market clearing and household budget constraint, it follows that $wH<Y$ whenever $\tau^*_F>0$. Hence, $\tau_F^*>0$ implies $\tau^*_{\iota}>0$.
	%
	%Observe that $Y\geq MPL \times H$, where $MPL$ stands in for the marginal product of labor, if the aggregate production function features decreasing or constant returns to scale. Under such a production function one can rewrite the last expression as
	%\begin{align}
	%\tau_{\iota}=1-\frac{wH}{Y}\geq 1-\frac{w H}{MPL \times H}
	%\end{align}
	% Note further that the marginal product of labor exceeds the wage rate whenever the environmental tax is different from zero; compare the disucssion in subsection \ref{subsec:Rams}. It follows that the right-hand side is positive, hence
	%\begin{align}
	%\tau_{\iota}>0,
	%\end{align}
	The optimal tax scheme is progressive.
\end{proof}

\begin{proof}\textit{The optimal allocation is efficient}
	
	The idea of this proof is to show that the efficient allocation is attainable for the Ramsey planner. Since the social planner could implement any competitive allocation (which necessarily satisfies the resource constraint) and has the same objective function, the efficient allocation maximizes the Ramsey problem. 
	
	To show that the efficient allocation is feasible, I assume that $s^*=s_{FB}$. Showing that $H^*=H_{FB}$ and $C^*=C_{FB}$ are a solution to the Ramsey problem, proves that the optimal policy implements the efficient allocation for two reasons. First, by the argument in the previous paragraph any competitive allocation is a potential candidate solution to the social planner's problem and the social planner has the same objective function. Second, due to strict concavity of the utility and strict monotonicity of the production function \textit{(so that more input means more output)}, the solution is also unique.
	
	When $H^*=H_{FB}$ then $C^*=C_{FB}$ since $s^*=s_{FB}$ by assumption. It now show that under this allocation optimal labor supply, indeed, is efficient, that is:
	\begin{align}
	U_C^*\frp{Y^*}{G^*}\frp{G^*}{L^*_{G}} = U_{C,FB}\frp{Y_{FB}}{G_{FB}}\frp{G_{FB}}{L_{G,FB}}.
	\end{align}
	
	From the assumed allocation it follows that $U_C^*=U_{C,FB}$ and $\frp{Y_{FB}}{G_{FB}}\frp{G_{FB}}{L_{G,FB}}=\frp{Y^*}{G^*}\frp{G^*}{L^*_{G}}$ and above condition is satisfied. 
	
	It remains to show that under the assumed allocation, $s^*=s_{FB}$ holds true. Since $Gov=0$ the Ramsey planner's first order condition with respect to $s$ equals that of the social planner. Since production and marginal utilities in the optimal allocation equal their counterparts in the efficient allocation, it has to holds that $\tau_F^*$ implements $s^*=s_{FB}$.  
	%Second, efficiency of labor supply, i.e., $H^*=H_{FB}$, as the only solution of the Ramsey planner's problem follows from demonstrating that both (i) $H^*>H_{FB}$ and (ii) $H^*<H_{FB}$ result in a contradiction under the assumption that $s^*=s_{FB}$.
	
	%Assume by contradiction that (i), $H^*>H_{FB}$. 
\end{proof}


	%-------------------------------------
	\clearpage
	\bibliography{../../../bib_2_0}
	\addcontentsline{toc}{section}{References}
\end{document}