\section{Quantitative results}\label{sec:res}

In this section, I present and discuss the quantitative results. Under the benchmark policy regime, the government runs a consolidated budget: environmental tax revenues are redistributed via the income tax scheme. 
First, in subsection \ref{subsec:exp}, I use the model to learn how a constant carbon tax affects the economy, how it interacts with a progressive income tax, and what carbon tax would be required to meet the emission limit.
Second, I show how the government can optimally satisfy the emission limit in subsection \ref{subsec:mr} by jointly choosing the progressivity of the income tax scheme and the carbon tax. Section \ref{subsec:dis} discusses the results of the optimality analysis. 

%I focus on analyzing the mechanisms and welfare benefits from integrating the income tax scheme into the environmental policy. I also discuss the costs of not using lump-sum transfers.

\subsection{Results }
This section depicts the environmental tax which is required to meet the emission limits and the evolution of key variables under distinct policy regimes. The income tax scheme is kept fixed at its calibrated levels: $\tau_{\iota t}=0.181$, $\lambda=0.43$.
I consider the following regimes: first, environmental tax revenues are redistributed lump sum to households. Second, the government runs a consolidated budget and environmental tax revenues are redistributed via the income tax scheme; that is, $\lambda$ adjusts. Third, environmental tax revenues are recycled as subsidies to the green sector. Fourth, the government consumes environmental tax revenues. 



\subsection{Optimal Policy}\label{subsec:mr}
This section seeks to answer the question how a benevolent planner optimally attains the emission limit. After showing the results, I discuss the intention behind the optimal policy with a special attention given to the role of income taxes. 
%This section depicts results on the optimal policy followed by the implied allocation in the benchmark model where environmental tax revenues are redistributed via the income tax scheme. 

\begin{figure}[h!!]
	\centering
	\caption{Optimal Policy }\label{fig:optPol}
	\begin{minipage}[]{0.32\textwidth}
		\centering{\footnotesize{(a) Income tax progressivity, $\tau_{\iota t}$}}
		%	\captionsetup{width=.45\linewidth}
		\includegraphics[width=1\textwidth]{../../codding_model/own_basedOnFried/optimalPol_010922_revision/figures/all_13Sept22_Tplus30/Single_OPT_T_NoTaus_taul_regime0_spillover0_knspil0_noskill0_sep0_xgrowth0_extern0_PV1_sizeequ0_GOV0_etaa0.79.png}
	\end{minipage}
	\begin{minipage}[]{0.32\textwidth}
		\centering{\footnotesize{(b) Environmental tax, $\tau_{Ft}$ }}
		%	\captionsetup{width=.45\linewidth}
		\includegraphics[width=1\textwidth]{../../codding_model/own_basedOnFried/optimalPol_010922_revision/figures/all_13Sept22_Tplus30/Single_OPT_T_NoTaus_Tauf_regime0_spillover0_knspil0_noskill0_sep0_xgrowth0_extern0_PV1_sizeequ0_GOV0_etaa0.79.png}
	\end{minipage}
\begin{minipage}[]{0.32\textwidth}
	\centering{\footnotesize{(c) Net emissions }}
	%	\captionsetup{width=.45\linewidth}
	\includegraphics[width=1\textwidth]{../../codding_model/own_basedOnFried/optimalPol_010922_revision/figures/all_13Sept22_Tplus30/Single_OPT_T_NoTaus_Emnet_regime0_spillover0_knspil0_noskill0_sep0_xgrowth0_extern0_PV1_sizeequ0_GOV0_etaa0.79.png}
\end{minipage}
\end{figure} 


%\paragraph{Optimal policy}
To meet the emission limits suggested by the IPCC, the optimal income tax is progressive for all periods; see panel (a) in figure \ref{fig:optPol}.  The x-axis indicates the first year of the 5 year period to which the variable value corresponds. 
% optimal taul over time:   
%   0.0824    0.0830    0.0835    0.0839    0.0842    0.0843    0.0927    0.0919    0.0913    0.0907    0.0902    0.0898

The optimal income tax scheme is progressive. The progressivity  increases during the 2020 to 2045 period and decreases afterwards during the net-zero periods. Starting from a value of 0.082 in 2020. When the net-zero emission limit is implemented in 2050, the progressivity parameter displays some discontinuity rising from 0.084 to 0.093 from where it  declines to 0.090 in 2070.
Overall, the optimal tax progressivity is approximately  around half the size found for the US in \cite{Heathcote2017OptimalFramework}: $\tau_{l}=0.181$.

%   optimal tauf
%        1.8967    2.0254    2.1536    2.2826    2.4135    2.5461    5.4378    5.6006    5.7675    5.9382    6.1120    6.2887


Consider panel (b). The optimal fossil tax is increasing over the period considered and jumps to higher levels when the zero-net emission limit is introduced in 2050.
In 2020, the environmental tax equals 86\% of the fossil sector's revenues, from where it rises steadily to 90\% in 2030.  As the emission limit declines to net-zero in 2050, the tax rapidly surges to 95\% and gradually increases afterwards reaching 97\% in 2070. 

%\paragraph{Allocation}
Figure \ref{fig:optAll} depicts the optimal allocation. Limiting emissions in line with the Paris Agreement is concomitant with a reduction of research over time, panel (c), and a shift to more green research; see the black dashed graph in panel (c).  When the net-zero limit becomes binding, there is a recomposition towards the green research. 
 
Panel (a) shows consumption which increases over time; yet, its level reduces when the net-emission limit become active in  2050. From the lower level it continues to grow comparibly fast as before. Labor effort of both skill types decreases over time.  In comparison to hours supplied by low-skill workers, high-skill workers lower hours more as the tax progressivity jumps in 2050; compare panel (c). 
The rise in consumption after each reduction is driven by technological progress.
%; compare panel (d) which shows 5-year growth rates by sector and as aggregate in per cent. 
%The green sector sees a rise in technological progress, the dashed black line, while growth in the fossil and the non-energy sector is positive, yet diminishing over time. Overall, aggregate growth is positive and increasing; compare the gray dashed graph. 

%Research efforts, shown in panel (e) decrease over time; compare the gray graph which depicts the sum of researchers across sectors. When the net-zero limit is binding, there is a recomposition towards the green sector: while research in the non-energy and the fossil sector decrease over time, green research effort rises. 
%Finally, the fossil sector has a higher labor input than the green sector. 

\begin{figure}[h!!]
	\centering
	\caption{Optimal Allocation }\label{fig:optAll}
	
	
	\begin{minipage}[]{0.32\textwidth}
		\centering{\footnotesize{(a) Consumption}}
		%	\captionsetup{width=.45\linewidth}
		\includegraphics[width=1\textwidth]{../../codding_model/own_basedOnFried/optimalPol_010922_revision/figures/all_13Sept22_Tplus30/Single_OPT_T_NoTaus_C_regime0_spillover0_knspil0_noskill0_sep0_xgrowth0_extern0_PV1_sizeequ0_GOV0_etaa0.79.png}
	\end{minipage}
	\begin{minipage}[]{0.32\textwidth}
		\centering{\footnotesize{(b) Hours worked }}
		%	\captionsetup{width=.45\linewidth}
		\includegraphics[width=1\textwidth]{../../codding_model/own_basedOnFried/optimalPol_010922_revision/figures/all_13Sept22_Tplus30/SingleJointTOT_regime0_OPT_T_NoTaus_Labour_spillover0_knspil0_noskill0_sep0_xgrowth0_extern0_PV1_etaa0.79_lgd1.png}
	\end{minipage}
%	\begin{minipage}[]{0.32\textwidth}
%		\centering{\footnotesize{(c) High-to-low-skill ratio}}
%		%	\captionsetup{width=.45\linewidth}
%		\includegraphics[width=1\textwidth]{../../codding_model/own_basedOnFried/optimalPol_010922_revision/figures/all_13Sept22_Tplus30/Single_OPT_T_NoTaus_hhhl_regime0_spillover0_knspil0_noskill0_sep0_xgrowth0_extern0_PV1_sizeequ0_GOV0_etaa0.79.png}
%	\end{minipage}
%	\begin{minipage}[]{0.32\textwidth}
%		\centering{\footnotesize{\ \\ (d) Technology growth}}
%		%	\captionsetup{width=.45\linewidth}
%		\includegraphics[width=1\textwidth]{../../codding_model/own_basedOnFried/optimalPol_010922_revision/figures/all_13Sept22_Tplus30/SingleJointTOT_regime0_OPT_T_NoTaus_Growth_spillover0_knspil0_noskill0_sep0_xgrowth0_extern0_PV1_etaa0.79_lgd1.png}
%	\end{minipage}
	\begin{minipage}[]{0.32\textwidth}
		\centering{\footnotesize{\ \\(c) Scientists }}
		%	\captionsetup{width=.45\linewidth}
		\includegraphics[width=1\textwidth]{../../codding_model/own_basedOnFried/optimalPol_010922_revision/figures/all_13Sept22_Tplus30/SingleJointTOT_regime0_OPT_T_NoTaus_Science_spillover0_knspil0_noskill0_sep0_xgrowth0_extern0_PV1_etaa0.79_lgd1.png}
	\end{minipage}
%	\begin{minipage}[]{0.32\textwidth}
%		\centering{\footnotesize{\ \\(f) Labor input}}
%		%	\captionsetup{width=.45\linewidth}
%		\includegraphics[width=1\textwidth]{../../codding_model/own_basedOnFried/optimalPol_010922_revision/figures/all_13Sept22_Tplus30/SingleJointTOT_regime0_OPT_NOT_NoTaus_LabourInp_spillover0_knspil0_noskill0_sep0_xgrowth0_extern0_PV1_etaa0.79_lgd1.png}
%	\end{minipage}
\end{figure} 



%\subsubsection{Consumption equivalence}
%
%The importance of the income tax schedule amounts to 9.28\% of per period consumption. The bulk of this utility gain is driven by future periods. When limiting the measure of the consumption equivalence to the 55 years considered in the explicit optimization, the CEV reduces to 0.18\%.  




%%%%%%%%%%%%%%%%%%%%%%%%%%%%%%%%%%%%%%%%%%%%%%%%%%%%%%%%%%%%%%%%%%%%%%%%%%%%%%%%%%%%
%% DISCUSSION 
%%%%%%%%%%%%%%%%%%%%%%%%%%%%%%%%%%%%%%%%%%%%%%%%%%%%%%%%%%%%%%%%%%%%%%%%%%%%%%%%%%%%

\subsection{Discussion}\label{subsec:dis}
%\tr{Questions}
%\begin{itemize}
%	\item why progressive tax? and why the drop in progressivity in 2050? (a means to boost high-skill supply and keeping low skill stable)
%	\item what are the costs of the progressive tax
%\end{itemize}

 What explains the optimal policy?
 In section \ref{subsec:notaul}, I contrast the optimal allocation under the benchmark policy regime to a scenario where no income tax is available. The impact of endogenous growth and skill heterogeneity on the optimal policy is analyzed in section \ref{subsec:xgrnsk}.
%Finally, in section \ref{subsec:comp_lumpsum}, I turn to analyze the optimal allocation under the alternative policy regimes: redistribution of environmental tax revenues via (1) lump-sum transfers and (2) the income tax scheme. 

%\begin{enumerate}
%	\item What is the goal of policy intervention? \ar social planner allocation
%	\item (Benefits) What is different when no integrated policy is run and instead revs consumed by government \ar Benefits of an integrated policy
%	\item double dividend literature: use of labor income tax when all env tax revenues are consumed by the government.
%	\item (Costs) What cannot be reached by integrated policy as compared to lump-sum transfers: is taul used for different purpose? without endogenous growth should be zero; eg. can use taul to boost growth as lump-sum transfers take care of labor supply 
%	\item What could be reached if there was no trade-off with heterogenous skills or growth? no heterogeneous skills, no endogenous growth \ar how does the optimal policy differ?
%\end{enumerate}

%\subsubsection{Comparison to other policy regimes}
%\tr{To be rewritten}
%How does the optimal allocation and especially its relation to the efficient allocation change under alternative policy scenarios?
%In this section, I discuss two policy alternations which have already been discussed in the analytical section. First, a version where environmental tax revenues are consumed by the government and no labor income tax scheme is available, henceforth referred to as \textit{separate policy}. The comparison of this scenario serves to assess the benefits of an integrated environmental-fiscal policy when no lump-sum transfers are available. 
%Second, I look at the optimal allocation 

\subsubsection{The role of income taxes}\label{subsec:notaul}



In figure \ref{fig:optAll_percLf_dyn}, I contrast the optimal allocation under the benchmark regime with income tax scheme, black solid graph, with the optimal allocation without labor income tax, the blue dashed graph. As a benchmark to the optimal policy, the figure depicts the social planner's allocation by the orange dotted graph.\footnote{\ I formulate the social planner's problem in appendix section \ref{app:sp_prob}.} 
The efficient allocation can be perceived as the allocation the Ramsey planner seeks to implement. However, she may not be able to achieve the efficient allocation due to the reliance on tax instruments.
All graphs depict percentage changes relative to the laissez-faire allocation of the same period.\footnote{\ Figure \ref{fig:LF} in appendix section \ref{app:quant_res} shows the laissez-faire allocation. } Except for panel (f) which compares the environmental tax in the model with and without income tax. 

\begin{figure}[h!!!]
	\centering
	\caption{Costs and benefits of progressive income taxes \tr{plot: consumption, hours worked, growth rates 2 points: a) more leisure, 2) more growth} }\label{fig:optAll_percLf_dyn}
	\begin{minipage}[]{0.32\textwidth}
		\centering{\footnotesize{(a) Consumption\\ \ }}
		%	\captionsetup{width=.45\linewidth}
		\includegraphics[width=1\textwidth]{../../codding_model/own_basedOnFried/optimalPol_010922_revision/figures/all_13Sept22_Tplus30/C_PercentageLFDynNT_Target_regime0_spillover0_noskill0_sep0_xgrowth0_PV1_etaa0.79_lgd1.png}
	\end{minipage}
	\begin{minipage}[]{0.32\textwidth}
		\centering{\footnotesize{(b) High-skill hours worked\\ \  }}
		%	\captionsetup{width=.45\linewidth}
		\includegraphics[width=1\textwidth]{../../codding_model/own_basedOnFried/optimalPol_010922_revision/figures/all_13Sept22_Tplus30/hh_PercentageLFDynNT_Target_regime0_spillover0_noskill0_sep0_xgrowth0_PV1_etaa0.79_lgd0.png}
	\end{minipage}
	\begin{minipage}[]{0.32\textwidth}
		\centering{\footnotesize{(c) Low-skill hours worked\\ \ }}
		%	\captionsetup{width=.45\linewidth}
		\includegraphics[width=1\textwidth]{../../codding_model/own_basedOnFried/optimalPol_010922_revision/figures/all_13Sept22_Tplus30/hl_PercentageLFDynNT_Target_regime0_spillover0_noskill0_sep0_xgrowth0_PV1_etaa0.79_lgd0.png}
	\end{minipage}
	\begin{minipage}[]{0.32\textwidth}
		\centering{\footnotesize{\ \\(d) Fossil growth\\ \ }}
		%	\captionsetup{width=.45\linewidth}
		\includegraphics[width=1\textwidth]{../../codding_model/own_basedOnFried/optimalPol_010922_revision/figures/all_13Sept22_Tplus30/gAf_PercentageLFDynNT_noeff_Target_regime0_spillover0_knspil0_noskill0_sep0_xgrowth0_PV1_etaa0.79_lgd0.png}
	\end{minipage}
\begin{minipage}[]{0.32\textwidth}
\centering{\footnotesize{\ \\(d) Non-energy growth\\ \ }}
%	\captionsetup{width=.45\linewidth}
\includegraphics[width=1\textwidth]{../../codding_model/own_basedOnFried/optimalPol_010922_revision/figures/all_13Sept22_Tplus30/gAn_PercentageLFDynNT_noeff_Target_regime0_spillover0_knspil0_noskill0_sep0_xgrowth0_PV1_etaa0.79_lgd0.png}
\end{minipage}
	\begin{minipage}[]{0.32\textwidth}
		\centering{\footnotesize{\ \\(e) Environmental tax, $\tau_{Ft}$\\ \ }}
		%	\captionsetup{width=.45\linewidth}
		\includegraphics[width=1\textwidth]{../../codding_model/own_basedOnFried/optimalPol_010922_revision/figures/all_13Sept22_Tplus30/tauf_OPT_COMPtaul_regime0_spillover0_knspil0_noskill0_sep0_xgrowth0_PV1_etaa0.79_lgd0.png}
	\end{minipage}
	\floatfoot{Notes: \footnotesize{ The figure shows the percentage deviation of the allocation resulting under the benchmark policy, that is, with income tax (the black solid graph), the allocation under the benchmark policy but the income tax is not available (the blue dashed graph), and the efficient allocation (the orange dotted graph), in relation to the laissez-faire allocation. 
			Panel (d) shows aggregate growth where the variable value in $t$ refers to the growth rate from  period $t$ to period $t+1$. Hence, from 2045 to 2050 growth reduces significantly, since in 2050 the net-emission limit has to be satisfied. Panel (f) shows the level of the environmental tax under the two policy regimes compared.
}}
\end{figure} 
%
%\clearpage
%%
% Labor supply
In comparison to a policy scenario without income tax, the availability of an income tax allows to more closely resemble the efficient levels of labor, panels (b) and (c). 
The social planner reduces hours worked for both the high- and the low-skill type by between 3 to 4 percent relative to the laissez-faire allocation.\footnote{\ Over time, the reduction in hours declines; the social planner chooses a higher work effort especially for the high-skill type. The reason is that more consumption becomes more valuable as the emission limit becomes stricter. This is the income effect of  externality mitigation on hours discussed in the analytical section. For the impact of the emission limit on the efficient allocation see figure \ref{fig:eff_with_notarget} in appendix section \ref{app:quant_res}.}


When no labor income tax is available, the reduction in  hours worked by the low type remains close to zero, the dashed graph. Under the same policy regime, hours of high-skill workers even increase slightly above laissez-faire level due to the strengthened importance of high-skill-intense green energy. When the government has income taxes available, it reduces hours worked of both types closer to the efficient allocation. While the efficient allocation sees a similar decline in working time of both types, the optimal policy allocation features a stronger decrease in working time by high-skill labor and too low a reduction of the low-skill ones. This result emerges from the higher wage elasticity of substitution for the high-skill type which works more hours in the laissez-faire allocation so that leisure is more valuable. The income channel of the wage rate is similar to both worker types because of perfect income insurance.

%- Labor income taxes have advantage in terms of growth
A second benefit of progressive income taxation in the quantitative model stems from endogenous growth and knowledge spillovers. 
In the periods before the net-zero emission limit (2020 to 2050), the optimal policy with progressive income tax achieves a higher aggregate growth rate; consider panel (d). The additional gains in terms of growth arise from substituting fossil taxes with income taxes; see the stronger reduction in fossil taxes in this period in panel (g).  The intuition goes as follows: by partly substituting fossil taxes with labor income taxation the economy can profit more from knowledge spillovers from the biggest research sector: the non-energy sector.\footnote{\ The higher growth rate, indeed, emerges from spillover effects and not a higher research effort; in fact, the amount of scientists is reduced more when an income tax is available, panel (f). However, the share of non-energy research increases (compare panel (c) in figure \ref{fig:optAll_percLf_dyn_app}). }
As the corrective tax reduces, the price for energy diminishes less, and energy becomes relatively less expensive. A price effect directs research to the more expensive non-energy sector. % the fossil tax is especially costly because it redirects research away from the non-energy sector which becomes relatively cheaper.

%Although the increase in growth rates is small - not a percentage point difference in growth rate reduction per period - the total effect on the future is substantial as highlighted by the consumption equivalence. 

 This mechanism underlines an advantage of reductive environmental policies as opposed to recomposing strategies. Here, the income tax and the fossil tax act as substitutes.  %\footnote{\ An alternative explanation for the advantage of income taxes above environmental taxes under the presented policy regime could be that labor taxes are redistributed to households, so there is no reduction in consumption via government consumption. To test this alternative explanation, I run a model version where both income and fossil taxes are redistributed through the income tax. And the results persist.}
Once the net-zero emission limit becomes binding in 2050, however, the gap between environmental taxes reduces and aggregate growth in the model without income taxes is minimally higher.   This suggests that the net-zero emission limit prevents to substitute fossil with labor income taxes to reap the gains from knowledge spillovers.\footnote{\ \cite{Acemoglu2012TheChange} demonstrate that consumption growth is hampered by a transition from dirty to green production when the dirty sector is more productive. However, their channel persists absent knowledge spillovers. Rather, final good production is slowed down more when dirty and fossil goods are easily substitutabel: then, the good with the higher technology is favored in production and technology improvements in the backward sector do not contribute to overall output growth.} 

% costs
The benefits of the progressive income tax, more leisure and knowledge spillovers, come at the cost of less consumption, panel (a),  and a lower green-to-fossil energy mix, panel (g), and a higher energy share, panel (j). The social planner implements continues consumption growth and only reduces consumption below laissez-faire levels during the first periods. In contrast, the optimal allocation reduces consumption relative to the laissez-faire world in all periods. The reduction is increasing over time,\footnote{\ This observation speaks to the literature investigating limits to growth. When the Ramsey planner can only use fossil and income taxes, the emission limit is best satisfied by a continues reduction in growth relative to the laissez-faire allocation. Yet, the government cannot use research subsidies in the present setting. However, the efficient allocation with emission limit is characterized by a reduction in consumption and in consumption growth; consider figure \ref{fig:eff_with_notarget}.} and it is stronger in the model with income tax. As growth rates are higher, or only minimally lower in the model with income tax, the additional reduction in consumption is explained by lower work effort. The additional decrease in consumption as the net-zero limit is established, occurs despite more work effort (note that hours worked in the laissez-faire equilibrium are constant over time). The reduction in growth and a lower marginal product of labor as the fossil tax increases explain this result. 


\begin{comment}
The higher green-to-fossil energy ratio (panel (g)) in the efficient allocation is driven by a reallocation of input factors towards the green sector; see panel (i). In contrast, the ratio of scientists remains largely unchanged (panel (h)). This observation suggests that the social planner recomposes the economy by inputs and not through an adjustment in technology growth in order to further profit from knowledge spillovers. In fact, these spillovers in favor of the less advanced sector enable the social planner to implement a higher green to fossil technology ratio than in the laissez-faire economy (panel (l)). 

In the optimal allocation, irrespective of the policy regime, the increase in the green-to-fossil energy mix is inefficiently low. When an income tax is available, the optimal policy mix mutes the rise even further. The reduction is explained by both less green-to-fossil research (panel (h)) and labor input (panel(i)). 
The adverse recomposition, however, does not, as hypothesized,  arise from the higher income tax progressivity. Rather, the lower fossil tax explains this result. 
%This recomposing effect of the labor income tax arises from the higher responsiveness of high skill labor to the tax progressivity.
To back this claim, I only feed the optimal income tax into the model and compare the resulting allocation to the laissez-faire economy. Figure \ref{fig:LF_vs_onlytaul} in the appendix shows the results. The green-to-fossil energy mix only reduces slightly in response to the progressive income tax. The following three paragraphs serve to understand this result.

There are two mechanisms shaping the recomposing effect of the labor income tax on the economic structure. Abstract for now from endogenous growth.
The first asymmetry results from a higher labor share in the fossil sector compared to the green one; I will refer to this channel as \textit{labor-share} channel. Therefore, an overall reduction in labor supply has a stronger effect on the fossil sector. Via this mechanism, income tax progressivity boosts green production. Figure \ref{fig:LF_vs_onlytaul_xgrnsk} shows the effect of income tax progressivity in the model with neither endogenous growth nor skill heterogeneity. Therefore, the income tax affects the economic structure solely through the level of labor supply. The reduction in labor supply makes the fossil good relatively cheaper, demand for green energy increases, and the labor share employed in the green sector grows.  

However, the labor-share channel is muted by endogenous growth. Adding endogenous growth to the model with exogenous growth and one skill type (see figure \ref{fig:LF_vs_onlytaul_nsk}), the green-to-fossil energy ratio remains unchanged relative to the laissez-faire allocation. 
Therefore, in the benchmark model with endogenous growth and skill heterogeneity the \textit{skill-recomposition} channel dominates and the effect of the progressive income tax on the green-to-fossil energy share is negative. Nevertheless, it is relatively small as price adjustments absorb the effect of a relatively higher low-skill supply on the direction of innovation; consider again figure \ref{fig:LF_vs_onlytaul}.\footnote{\ A similar analysis to the one below applies to the share of energy and non-energy goods in final consumption. The non-energy good is more labor intense than the energy good and features a lower high-skill share. Again, a reduction in the energy-share dominates in the benchmark model. Research effort, again, is invariant to the changes in labor supply. }

Given the small recomposing effect of the labor income tax absent the fossil tax leads to the conclusion that it is the reduction of the fossil tax which drives the adverse effect on the energy mix once the government can use an income tax scheme. 
Because of the reductive effect of the progressive income tax, a higher fossil share does not conflict with meeting the emission limit. 

content...
\end{comment}

\begin{comment}
  This again transmits to research efforts, as machine producers' profits from research in the fossil sector are higher thereby amplifying the recomposing effect of income taxes. This finding is in line with the theoretic considerations in the literature: first,  since green and fossil energy are substitutes, a market size effect may dominate the price effect attracting research efforts in the sector with higher input supply. Second, complementarity of the non-energy and energy goods combined with a reliance of energy on the scarcer sector implies that research is directed towards the sector where input goods are scarcer; i.e., energy (compare figure \ref{fig:optAll_percLf_dyn_app} in the appendix). 

content...
\end{comment}
\begin{comment}
COMMENT ON WEAK DD

These results speak to the weak double-dividend literature. %When the government consumes environmental tax revenues, hours worked are inefficiently high. 
The weak double-dividend result posits that when environmental tax revenues suffice to cover all government funding requirements, it would be optimal to lower distortionary income taxes. The results presented herein, however, show that there is a lower bound. Lowering distortionary income taxes too much results in inefficiently high hours worked. Hence, even though there is no motive to fund government expenses  labor income taxation is not zero due to the environmental externality.
%Indeed, this reduces consumption further away from the efficient level, but, hours worked are aligned closer to the efficient level, panels (b) and (c). Next to consumption, the planner also forfeits an advantageous green-to-fossil energy ratio, panel (e). 

%The use of a progressive labor income tax contributes minimally to meeting the emission limit as can be seen by scrutinizing the optimal environmental tax, panel (b) in figure \ref{fig:comp_nored_pol}: when income taxes can be used, the environmental tax is lower. Still, the difference is minimal, supporting the thesis of complementarity of income and environmental taxes. 
%Environmental tax revenues are lower as the tax rate reduces, and income taxes reduce labor supply and hence the tax base of the environmental tax. 
%Even though labor income taxes have the advantage of being redistributed to households and lowering the externality, they are not used to substitute environmental tax revenues.\footnote{\ This might be a motive to prefer labor income taxes as an instrument to reduce emissions since labor income tax revenues are redistributed back to the household while environmental tax revenues are not in this setting. Nevertheless, the observation that the environmental tax only adjusts slightly once an income tax tool is available points to the advantage of environmental taxes in handling too high emissions.
%}
\end{comment}
\subsubsection{Optimal policy with lump-sum transfers}
To challenge the claim that income taxes serve as a substitute to carbon taxes, I now turn to discuss the optimal policy tuples in a policy regime with lump-sum transfers. According to the theory established in section \ref{sec:theory}, lump-sum transferring environmental tax revenues implements the efficient level of labor. 
Hence, observing a progressive income tax in this policy regime points to another advantage of taxing labor. 

The optimal income tax scheme is mildly progressive in the first periods and becomes regressive afterwards. The gain of this policy is more fossil and non-energy growth during initial periods until 2035 and higher growth rates in the green sector during the net-zero emission periods. \textit{is the higher green growth due to knowledge spillovers or due to the regressive labor tax?}

The driving model feature of this result are knowledge spillovers. Absent knowledge spillovers, the optimal income tax scheme is regressive throughout which boosts labor supply in general and raises the high-to-low skill ratio supplied.  As a result of both the higher carbon tax and the regressive income tax, green growth is higher, and fossil and non-energy growth reduces.\footnote{\ Figure \ref{fig:opt_TLs_noknow} in appendix section \ref{app:TLS } replicates figure \ref{fig:opt_TLs} above in a model without knowledge spillovers. 
	 When skill are homogeneous so that there is no compositional effect of income tax progressivity, the optimal income tax is even more regressive and the carbon tax is raised more above the counterpart in a model without income tax. This policy achieves a stronger composition towards green production and growth through the higher carbon tax, while the regressive income tax mitigates the reductive effect of the higher carbon tax on labor supply. Figure \ref{fig:opt_TLs_noknow_homoskill} in appendix section \ref{app:TLS }shows the results. }
With knowledge spillovers, the non-energy and green sector profit from technology levels in the fossil sector.


The utility gains of this policy arise during the initial 10 years from 2020 to 2030, panel (f). 
 

%- carbon tax
Thanks to the reductive effect of the progressive income tax, the carbon tax can be set lower during the initial periods considered. Once the income tax scheme becomes regressive the optimal carbon tax exceeds the counterpart in the model absent income tax. 
\begin{figure}[h!!!]
	\centering
	\caption{Optimal policy with lump-sum transfers}\label{fig:opt_TLs}
	\begin{minipage}[]{0.32\textwidth}
	\centering{\footnotesize{Income tax progressivity, $\tau_{\iota t}$ }}
	%	\captionsetup{width=.45\linewidth}
	\includegraphics[width=1\textwidth]{../../codding_model/own_basedOnFried/optimalPol_010922_revision/figures/all_13Sept22_Tplus30/taul_OPT_COMPtaul_regime4_spillover0_knspil0_noskill0_sep0_xgrowth0_PV1_etaa0.79_lgd0.png}
\end{minipage}
\begin{minipage}[]{0.32\textwidth}
\centering{\footnotesize{(b) Carbon tax, $\tau_{Ft}$}}
%	\captionsetup{width=.45\linewidth}
\includegraphics[width=1\textwidth]{../../codding_model/own_basedOnFried/optimalPol_010922_revision/figures/all_13Sept22_Tplus30/tauf_OPT_COMPtaulPer_regime4_spillover0_knspil0_noskill0_sep0_xgrowth0_PV1_etaa0.79.png}
\end{minipage}
	\begin{minipage}[]{0.32\textwidth}
	\centering{\footnotesize{(c) Fossil growth }}
	%	\captionsetup{width=.45\linewidth}
	\includegraphics[width=1\textwidth]{../../codding_model/own_basedOnFried/optimalPol_010922_revision/figures/all_13Sept22_Tplus30/gAf_OPT_COMPtaulPer_regime4_spillover0_knspil0_noskill0_sep0_xgrowth0_PV1_etaa0.79.png}
\end{minipage}
	\begin{minipage}[]{0.32\textwidth}
		\centering{\footnotesize{(d) Green growth }}
		%	\captionsetup{width=.45\linewidth}
		\includegraphics[width=1\textwidth]{../../codding_model/own_basedOnFried/optimalPol_010922_revision/figures/all_13Sept22_Tplus30/gAg_OPT_COMPtaulPer_regime4_spillover0_knspil0_noskill0_sep0_xgrowth0_PV1_etaa0.79.png}
	\end{minipage}
	\begin{minipage}[]{0.32\textwidth}
		\centering{\footnotesize{(e) Non-energy growth }}
		%	\captionsetup{width=.45\linewidth}
		\includegraphics[width=1\textwidth]{../../codding_model/own_basedOnFried/optimalPol_010922_revision/figures/all_13Sept22_Tplus30/gAn_OPT_COMPtaulPer_regime4_spillover0_knspil0_noskill0_sep0_xgrowth0_PV1_etaa0.79.png}
	\end{minipage}
\begin{minipage}[]{0.32\textwidth}
\centering{\footnotesize{(f) Period utility }}
%	\captionsetup{width=.45\linewidth}
\includegraphics[width=1\textwidth]{../../codding_model/own_basedOnFried/optimalPol_010922_revision/figures/all_13Sept22_Tplus30/SWF_OPT_COMPtaulPer_regime4_spillover0_knspil0_noskill0_sep0_xgrowth0_PV1_etaa0.79.png}
\end{minipage}
	\floatfoot{Notes: \footnotesize{ 	}}
\end{figure} 


