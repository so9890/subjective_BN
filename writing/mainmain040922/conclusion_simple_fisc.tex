\section{Conclusion}\label{sec:con}
Some scholars argue that  reductive policies are necessary to handle environmental limits \citep{Schor2005SustainableReductionb, VanVuuren2018AlternativeTechnologies, Bertram2018TargetedScenarios}, and the question has been raised whether consumption is too high \citep{Arrow2004AreMuch}. On the other hand, the focus of environmental policy discussions in economics rests on corrective environmental taxation. In the light of tightening environmental limits \citep{Rockstrom2009AHumanity, IPCC2022}, I study whether labor income taxes - as a reductive policy tool - can help mitigate externalities. 

In the analytical part of the paper, I show in a simple model that labor income taxes are  progressive as part of the optimal environmental policy. %The model does not feature inequality.
% Quantitative results
% baseline model
When environmental tax revenues are not redistributed lump sum, labor supply is inefficiently high. Then, income taxes serve to diminish hours worked closer to the efficient level. The result prevails absent income inequality.


% quantitative
In the second part of the paper, I analyze in a quantitative model with skill heterogeneity and endogenous growth whether the optimal labor income tax remains progressive. Again, there are no equity concerns, but workers are perfectly ensured against income differences. 
The optimal income tax is progressive mainly to reduce inefficiently high hours worked. The quantitative model reveals, secondly, that income taxes also serve as a substitute for corrective taxes. Knowledge spillovers from the fossil sector render environmental taxes especially costly. 
During the periods before the net-zero emission, therefore, the optimal policy consists in setting carbon taxes lower to prevent too strong a disruption of fossil growth and to ensure satisfaction of the emission limit through a progressive income tax. Albeit interesting, the use of this channel is limited in size. Also, when emissions have to be reduced to net zero, the economy cannot profit from this channel. 

% extensions
In an extension, I am planning to give the Ramsey planner the opportunity to limit working hours directly. The literature advocating a reduction in consumption levels \citep[e.g.,][]{Schor2005SustainableReductionb} proposes a restriction of hours worked as policy instrument to lower the consumption of resources.
Even though advocated in the literature, there is evidence for political difficulties in reducing working hours. In 2020, the French Citizens' Convention on Climate voted against reducing working hours as a measure to handle climate change. Potentially, ignorance about economic consequences is an explanation. The extension would serve to better understand economic consequences. 

Secondly, the model features log-utility of consumption in order to keep labor supply unaffected from the level of xxx. 
%Secondly, the model abstracts from income inequality and government funding constraints which constitute traditional motives for income taxation.  Integrating these aspects into the model 
