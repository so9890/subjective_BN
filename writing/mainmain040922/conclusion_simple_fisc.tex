\section{Conclusion}\label{sec:con}
Some scholars argue that  reductive policies are necessary to handle environmental limits \citep{Schor2005SustainableReductionb, VanVuuren2018AlternativeTechnologies, Bertram2018TargetedScenarios}, and the question has been raised whether consumption is too high \citep{Arrow2004AreMuch}. On the other hand, the focus of environmental policy discussions in economics rests on corrective environmental taxation. In the light of tightening environmental limits \citep{Rockstrom2009AHumanity, IPCC2022}, I study whether labor income taxes - as a reductive policy tool - can help mitigate externalities. 

In the analytical part of the paper, I show in a simple model that the optimal environmental policy features a reductive policy element: lump-sum transfers of environmental tax revenues. When lump-sum transfers are not available, progressive labor income taxes act as a complement to the environmental tax. The intention is to mitigate distortions on the labor market arising from the lack of lump-sum rebates. %The model does not feature inequality.
% Quantitative results
% baseline model
%When environmental tax revenues are not redistributed lump sum, labor supply is inefficiently high. Then, income taxes serve to diminish hours worked closer to the efficient level. The result prevails absent income inequality.


% quantitative
In the second part of the paper, I analyze in a quantitative model with skill heterogeneity and endogenous growth whether the optimal labor income tax remains progressive.
This is unclear given that a progressive income tax (i) may lower research efforts in general, and (ii) recomposes the energy ratio towards more fossil energy. The reason for the latter is that the green sector is skill biased plus high-skill workers being more responsive to a more progressive tax scheme as they work more hours. 
A progressive income tax makes high-skill less abundant thereby raising the production costs of the green sector. 
% Again, there are no equity concerns, but workers are perfectly ensured against income differences. 
Nevertheless, the optimal income tax is progressive. The primary objective is to reduce inefficiently high hours worked. 

Secondly, income taxes also serve as a substitute for carbon taxes.
By combining carbon taxes with progressive income taxes, a lower carbon tax suffices to satisfy emission limits. A lower carbon tax boosts fossil and non-energy research. Knowledge spillovers mitigate the drop in green research so that meeting the net-zero emission limit in the 2050s and thereafter does not become too expensive. Absent knowledge spillovers such a combined policy becomes too costly in the future. Yet, the use of progressive income taxes with the intention to boost growth is limited in time: Once the net-zero emission limit is binding there is no room to substitute carbon taxes with a progressive income tax. 

% extensions
In an extension, I am planning to give the Ramsey planner the opportunity to limit working hours directly. The literature advocating a reduction in consumption levels \citep[e.g.,][]{Schor2005SustainableReductionb} proposes a restriction of hours worked as policy instrument to lower the consumption of resources.
Even though advocated in the literature, there is evidence for political difficulties in reducing working hours. In 2020, the French Citizens' Convention on Climate voted against reducing working hours as a measure to handle climate change. Potentially, ignorance about economic consequences is an explanation. The extension would serve to better understand economic consequences. 

Secondly, the model features log-utility of consumption in order to keep labor supply unaffected from the scale of the income tax scheme. A substitution and income effect offset each other as the wage rate rises.  However, recent evidence exists that  shows that working hours reduce over time with productivity \citep{Boppart2019LaborPerspectiveb}. An income effect dominates the substitution effect. A lack of lump-sum transfers may exacerbate the need for progressive income taxes. 

Thirdly, it would be interesting to investigate whether there is a role for progressive income taxes in the optimal environmental policy when green subsidies are available. \cite{Acemoglu2012TheChange} point to the importance of green subsidies within the optimal environmental policy. The use of progressive taxes to boost growth might be obsolete. Then again, subsidies may boost labor demand aggravating the inefficiency in working hours. 
%Secondly, the model abstracts from income inequality and government funding constraints which constitute traditional motives for income taxation.  Integrating these aspects into the model 
