\documentclass[12pt]{article}
\usepackage[utf8]{inputenc}
\usepackage{xcolor}
\usepackage{graphicx}
\usepackage{listings}
\usepackage{epstopdf}
\usepackage{etoc}
\usepackage{pdfpages}
\usepackage[capposition=top]{floatrow}
\usepackage{pdflscape} % landsacpe package
% set font to times
%\usepackage{mathptmx} % times!!! 
%\usepackage[T1]{fontenc}
\usepackage{amsmath}
\usepackage{amsthm}
\usepackage{soul}
\usepackage[left=2.5cm, right=2.5cm, top=2.5cm, bottom =2.5cm]{geometry}
\usepackage{natbib}
%\usepackage[natbibapa]{apacite}
%\usepackage{apacite}
%\bibliographystyle{apacite}
\bibliographystyle{apa}
%\renewcommand{\footnotesize}{\fontsize{10pt}{11pt}\selectfont}
\usepackage[onehalfspacing]{setspace}
\usepackage{listings}
\renewcommand{\figurename}{\textbf{Figure}}
\renewcommand{\hat}{\widehat}
\usepackage[bf]{caption}
\usepackage{tikz}
%\begin{comment}
%\usepackage[headsepline,footsepline]{scrlayer-scrpage} % has to come before package!!! otherwise option clash
%\usepackage{scrlayer-scrpage}
%\pagestyle{scrheadings} % kopfzeile/ fußzeile
%\clearpairofpagestyles
%\ohead{}
%\ihead{\textit{Redistribution, Demand and  Sustainable Production}}
%\cfoot{\thepage}
%\pagestyle{plain} % comment this one to have header
%\end{comment}
\allowdisplaybreaks
\usepackage{comment}
 \usepackage{siunitx}
  \usepackage{textcomp}
\definecolor{sonja}{cmyk}{0.9,0,0.3,0}
%\definecolor{purple}{model}{color-spec}
\usepackage{amssymb}
\newcommand{\ar}{$\Rightarrow$ \ }
\newcommand{\frp}[2]{\frac{\partial{#1}}{\partial{#2}}}
\newcommand{\tr}[1]{\textcolor{red}{#1}}
\newcommand{\vlt}[1]{\textcolor{violet}{#1}}
\newcommand{\bl}[1]{\textcolor{blue}{#1}}
\newcommand{\sn}[1]{\textcolor{sonja}{#1}}
%%% TIKZS
\usepackage{tikz}
\usetikzlibrary{mindmap,trees}
\usetikzlibrary{backgrounds}
\tikzstyle{every edge}=  [fill=orange]  
\usetikzlibrary{tikzmark}
\usetikzlibrary{decorations.markings}
\usepackage{tikz-cd}
\usetikzlibrary{arrows,calc,fit}
\tikzset{mainbox/.style={draw=sonja, text=black, fill=white, ellipse, rounded corners, thick, node distance=5em, text width=8em, text centered, minimum height=3.5em}}
\tikzset{mainboxbig/.style={draw=sonja, text=black, fill=white, ellipse, rounded corners, thick, node distance=5em, text width=13em, text centered, minimum height=3.5em}}
\tikzset{dummybox/.style={draw=none, text=black , rectangle, rounded corners, thick, node distance=4em, text width=20em, text centered, minimum height=3.5em}}
\tikzset{box/.style={draw , rectangle, rounded corners, thick, node distance=7em, text width=8em, text centered, minimum height=3.5em}}
\tikzset{container/.style={draw, rectangle, dashed, inner sep=2em}}
\tikzset{line/.style={draw, very thick, -latex'}}
\tikzset{    pil/.style={
		->,
		thick,
		shorten <=2pt,
		shorten >=2pt,}}
	
% other stuff
\newcommand{\innermid}{\nonscript\;\delimsize\vert\nonscript\;}
\newcommand{\activatebar}{%
	\begingroup\lccode`\~=`\|
	\lowercase{\endgroup\let~}\innermid 
	\mathcode`|=\string"8000
}
%\usepackage{biblatex}
%\addbibresource{bib_mt.bib}
\usepackage{ulem}
\title{The Importance of Fiscal Policies to the Optimal Environmental Policy}
%\title{The Environment, Inequality, and Growth\\ \small{ optimal fiscal policy in an endogenous growth model with inequality and emission targets}}
\date{Sonja Dobkowitz\\ Bonn Graduate School of Economics\\ %University of Bonn\\
\vspace{1mm}
%Preliminary and incomplete\\
%First version: January 9, 2022\\
This version: \today }
\usepackage{graphicx,caption}
%\usepackage{hyperref}
\usepackage[colorlinks,linkcolor=aaltoblue,citecolor=aaltoblue,urlcolor=aaltoblue,unicode=true]{hyperref} %can create hyperlinks. ALWAYS LOAD LAST
\definecolor{aaltoblue}{RGB}{0,94,184}
\usepackage{minitoc}
\setcounter{secttocdepth}{5}
\usetikzlibrary{shapes.geometric}

% for tabular

%\usepackage{array}
\usepackage{makecell}
\usepackage{multirow}
\usepackage{bigdelim}

%propositions etc
\newtheorem{prop}{Proposition}
\newtheorem{corollary}{Corollary}[prop]
\newtheorem{lemma}[prop]{Lemma}

\renewenvironment{abstract}
{\small
	\list{}{
		\setlength{\leftmargin}{0.025\textwidth}%
		\setlength{\rightmargin}{\leftmargin}%
	}%
	\item\relax}
{\endlist}
\begin{document}
%	\includepdf[pages=-]{../titlepage.pdf}
	\maketitle
	\begin{abstract}
		\begin{singlespacing}
			\textbf{Abstract \ }
			Some scholars argue for limiting consumption to handle tightening environmental constraints \citep{Schor2005SustainableReductionb, VanVuuren2018AlternativeTechnologies}. Can reductive measures help mitigate environmental externalities?
			 I show that an environmental tax does not suffice to implement the efficient allocation. Instead, the optimal policy also contains a reductive element: lump-sum transfers complement environmental taxes by reducing labor supply through an income effect.
			 When environmental tax revenues are not redistributed lump sum, the planner uses progressive labor income taxes to decrease work effort. %Therefore, the optimal environmental policy - as a byproduct - increases equity either through lump-sum transfers or progressive income taxation.
			I quantify the optimal income tax in an endogenous growth model with skill heterogeneity, which is progressive. Next to the benefits from more leisure, the income tax is used to substitute for environmental taxes so that the economy can profit from knowledge spillovers from the  non-energy sector. The reason is that a higher price of energy goods induced by the environmental tax directs research away from the non-energy sector. As the non-energy sector is the biggest in terms of research processes, knowledge spillovers from this sector are especially important.%
		%	Two compositional and opposed effects of the income tax arise through (i) a higher labor share in the fossil sector and (ii) a skill bias in the green sector. Overall, the compositional effect is small.	
%			 On the one hand, a smaller labor supply discourages fossil production which is more labor intense. On the other hand, a skill bias in the green sector implies an adverse effect of income tax progressivity on the green-to-fossil energy mix. Overall, the latter mechanism dominates and is amplified through a market size effect, but the compositional effect remains small overall.

		%	The model suggests that the use of income taxes when environmental tax revenues are not redistributed raises social welfare by 0.1\% over the 60 years from 2020 to 2080.
			%The model suggests that integration of environmental and fiscal policy when no lump-sum transfers are available raises social welfare by 0.1\% over the 60 years from 2020 to 2080.
%			To reach climate targets, the International Panel on Climate Change has identified net-zero emissions by 2050 an essential element. I show that progressive income taxes are optimally used in concert with corrective taxes to satisfy an absolute emission limit.
%			On the one hand, progressive income taxation reduces labour efforts as leisure becomes relatively cheaper. The overall reduction in production lowers emissions. On the other hand, a progressive income tax (i) reduces general research efforts through a market size effect  and (ii) recomposes the structure of the economy away from green energy through a skill-supply channel. Both effects call for a more regressive tax to foster a green transition. For a reasonable calibration, I find that the reduction effect dominates and the optimal income tax schedule is progressive. The welfare advantage of income tax progressivity arises from a reduction of inefficiently high hours worked. 
%The model suggests that including progressive income taxes as a tool to lower emissions accounts for a rise in social welfare by 0.1\% over the 60 years from 2020 to 2080.  % as relative supply of the through a skill-supply channel. As richer, high-skill workers reduce their labour supply more in response to a more progressive tax, low-skill labour becomes more abundant. 
			
%			Natural scientists have identified the reduction of demand %for energy and land %thus, a change in lifestyle 
%			as an important contributor to meeting global climate targets. However, a general equilibrium analysis of reduction policies is missing.
%		%	I study the general equilibrium effects of reduction policies: such as income taxes, a restriction of hours worked, or consumption taxes. 
%		A higher labour income tax progressivity can achieve such a reduction as it lowers labour supply. 
%		Then again, tax progressivity alters the relative skill supply. As the green sector is relatively more skill-biased, the economy recomposes production towards the dirty sector. What is the optimal policy when the government has to meet an exogenous emission and demand target?
		%This additional benefit of income taxes changes the  equity-efficiency trade-off which classically determines optimal fiscal policies. 
		%To answer the question, I build a model of directed technical change and skill heterogeneity.
		
		
%In a set up with representative agent, the necessity to meet emission targets makes the optimal income tax highly progressive. The more goods are substitutes the higher the optimal tax progressivity. When goods are complements, the more slowly growing clean sector dampens production in the dirty sector making a lower progressivity sufficient. 
%	\noindent \textit{JEL classification}: E71, H21, H23,  O11, O13, Q58
			
		\end{singlespacing}
		
		\end{abstract}
%\tableofcontents
%\section{Introduction}

\begin{quote}
"Mitigation pathways limiting warming to 1.5°C [...]  reduce emissions further to reach net zero $CO_2$ emissions in the 2050s."
\end{quote}

The latest Reports of the International Panel on Climate Change highlights the importance of an absolute emission target by the 2050s to comply with the Paris Agreement on limiting temperature rise to 1.5°C. 
The economics literature on environmental policy has by and large allowed for a (limited) trade-off between consumption and pollution \citep{Barrage2019OptimalPolicy, Golosov2014OptimalEquilibrium} or studied relative emission targets \citep{Fried2018ClimateAnalysis}. 
However, the presence of an absolute emission target poses a limit to growth in fossil energy usage.
Depending on the substitutability of green and fossil energy and the velocity of the green sector to grow, the absolute emission target may, first, pose a limit to consumption growth and, second, make untraditional policy measures in addition to corrective taxes optimal.\textit{ (WHY THIS DIFFERENCE? Also with externalities in consumption pollution cannot be compensated for by consumption as the marginal utility of consumption reduces.  )} 

For a reasonably calibrated endogenous growth model, I find that the optimal labour income tax is progressive when accounting for an absolute emission target. This finding highlights the importance of policy measures targeted at a \textit{reduction} of production in tandem with policies intended at a \textit{recompostion} of the economic structure such as carbon taxes to mitigate climate change. % Then again, I present data indicating a voluntary reduction in household consumption. Given this behavioural change, the optimal income tax progressivity could become regressive in order to boost high-skill labour supply. 

%MODEL
To investigate the effect of an exogenous emission target on the optimal policy, I study an endogenous growth model building on \cite{Fried2018ClimateAnalysis}. Allowing for endogenous growth is important to take seriously the possibility of green growth to keep consumption high while meeting emission targets. The government is characterised as a Ramsey planner who seeks to maximise Utilitarian social welfare but is constrained by an exogenous emission target. To abstract from inequality as a determinant of tax progressivity, the economy is populated by a representative family. Yet, the family supplies two types of skill. 

The model differentiates between high- and low-skilled labour to account for a skill bias found for the green sector \citep{Consoli2016DoCapital}. This asymmetry of sectors renders regressive taxes a tool to lower production costs in the green sector. As high-skill workers reduce their labour supply more in response to a more progressive tax due to a relatively lower value of an additional unit of income, a regressive tax functions as a green subsidy. % In fact, there is an externality arising from high-skill labour supply as it shapes the share of fossil to green energy production. 
On the other hand, labour income taxation lowers aggregate production as it renders leisure cheaper to households. 
Together, the recomposing and the reductive channel shape the optimal tax progressivity from an environmental policy perspective. 

In an extension to the baseline model I depart from the representative agent assumption and explicitly model household heterogeneity. This setting allows to capture a change in household behaviour: A share of households is willing to voluntarily reduce consumption. I provide evidence for such behaviour using a representative Dutch dataset. More than 50\% of households are willing to reduce consumption in order to help the economy. Importantly, these households have a higher likelihood to work in the green sector. How does such a change in behaviour affect the optimal policy? Given the additional reduction in green-specific labour supply, the planner might find it optimal to set a more regressive tax.    

%Calibration
The Calibration of the model proceeds in two steps. First, I set certain parameters to values found in the literature. Most impportantly, I use reasonable value of production and growth processes as found in \cite{Fried2018ClimateAnalysis} who conducts a rigorous calibration exercise.  With these 

% Quantitative Experiment and Results
The main finding is the optimality of progressive taxes in order to reach emission targets. 


To scrutinise the reasons for the main finding. I conduct several additional quantitative experiments. First, I reduce the size of the emission target, second, I allow for a longer time frame until net-zero emissions have to be reached. The IPCC report states that for a temperature target of 2°C net-zero emissions have to be reached by 2070 only. How does this laxer target affect the importance of labour income taxes. Given the wider time frame, the green sector might be able to catch up and growth could continue. \textit{(Question: I guess that substitutability is key here! Growth in green implies growths in fossil when goods are no perfect substitutes! )}
\section{Introduction}
%\tr{I show that carbon taxes are only efficient if lump-sum transfers are available.}

\begin{comment}
\tr{Think about:
	1) when labor income taxes are not used, then need to have  a higher environmental tax to meet emission limits? \ar Yes, because of advantageous level effect which outweighs recomposing effect of income tax.
	2) When staying at level optimal under the assumption of lump-sum redistribution, but then not redistributing, than absent labor income tax emissions are too high; by how much? Counterfactual}
	
	content...
	\end{comment}
%\paragraph{Recomposition versus reduction discussion plus ever stricter emission limits}
The latest assessment report of the Intergovernmental Panel on Climate Change (IPCC) \citep{IPCC2022} highlights the urgency to reduce greenhouse-gas emissions.%relative to the previous report from 2018 \citep{Rogelj2018MitigationDevelopment.}.
\footnote{ \  The report stresses the decreasing likelihood of meeting the Paris Agreement and limiting climate warming to 1.5°. The Paris Agreement of 2015 formulates clear political goals to mitigate climate change. Under this treaty, states have agreed on a legally binding maximum increase in temperature to well below 2°C, preferably 1.5° over pre-industrial levels, and the global community seeks to be climate-neutral in 2050  (compare: \url{https://unfccc.int/process-and-meetings/the-paris-agreement/the-paris-agreement}). 
}
On the other hand, scholars have pointed to reductive policy measures to handle environmental limits \citep{Arrow2004AreMuch, Schor2005SustainableReduction, Dasgupta2021}. A reduction in work effort and consumption mitigates pollution by diminishing economic activity. Such a reduction could be achieved by using distortionary fiscal policy tools.
However, the economic literature on environmental policy has focused on the recomposing aspect of environmental policies: environmental taxes. %\citep{Fried2018ClimateAnalysis}. 
Given the exigency to act, this paper addresses the question whether fiscal policies can help meet climate targets.

I show that, indeed, once 
labor supply is elastic, reductive policy measures optimally complement the environmental tax. 
It is established in the literature that absent any other distortion, an environmental tax equal to the social cost of the externality implements the efficient allocation. I argue, first, that this result crucially depends on the use of lump-sum transfers to redistribute environmental tax revenues; otherwise work effort is inefficiently high.
%\textcolor{blue}{This is interesting independent of whether they are feasible or not. Could relate to the fact that there is a discussion how to use revenues. Yet, one might argue that we are always in a setting with distortionary labor income taxes; so that recycling lump-sum is never needed; numbers on size of expected revenues and government spending}
 When, second, environmental tax revenues are not redistributed lump-sum, then  environmental taxes are optimally combined with progressive labor income taxes. However, the use of income taxes is not directly targeted at the externality: the motive for labor taxation follows rather indirectly from a distortion in labor markets induced by the environmental policy. Hence, (i) the two tax instruments are complements, and % to lower inefficiently high hours worked. 
% I show that redistributing environmental tax revenues through an income tax scheme allows to implement the efficient allocation. The optimal income tax scheme is progressive.
(ii) the optimal environmental policy equalizes the distribution of income as  a side effect. The theoretic analysis forms the first part of the paper.

In the second part, I scrutinize whether progressive income taxes remain optimal in a more realistic quantitative model with endogenous growth and heterogeneous skills \textbf{when environmental tax revenues are not redistributed. 
}This is unclear a-priori since a progressive tax scheme reduces incentives to innovate through a market size effect. Furthermore, a skill bias documented for the green sector \citep{Consoli2016DoCapital} in combination with a relatively more elastic high-skilled labor causes a higher tax progressivity to recompose economic structure towards dirty production. The model suggests that despite these countering mechanisms the optimal income tax scheme is progressive. To lower hours worked the government forfeits growth and accepts a less green production ratio.  I quantify the welfare gains of setting progressive income taxes to equal yyy in consumption equivalent measure.  
% take model with no redistribution as benchmark

The paper's results are relevant for the political and academic debate on how best to use environmental tax revenues. The paper points to the importance of lump-sum transfers as a reductive policy tool in the optimal environmental policy; an aspect which appears overlooked in today's discussion.\footnote{\ POLICY debate; \cite{Fried2018TheGenerations}}
When thinking about how to recycle environmental tax revenues other than as lump-sum transfer, then, one should also think about alternative reductive tools such as progressive labor income taxes. 
If the reductive part of the environmental policy is neglected, environmental taxes have to be higher to meet emission limits, as I demonstrate in the quantitative exercise.

The results address the academic debate on the so-called \textit{weak double-dividend} \citep[for example:][]{LansBovenberg1994EnvironmentalTaxation, LansBovenberg1996OptimalAnalyses}. The hypothesis posits that recycling environmental tax revenues to reduce pre-existing tax distortions is advantageous to recycling  revenues as lump-sum transfers. The rationale is that transfers decrease labor supply thereby diminishing the tax base of the income tax. A conflict between generating government funds and environmental protection arises. The findings in the present paper suggest a lower bound on the reduction in distortionary income taxes: when environmental tax revenues are not redistributed lump-sum, some reduction in labor supply via distortionary income taxes is in fact efficient from an environmental policy perspective. In other words, even if environmental tax revenues suffice to satisfy an government revenue requirement, the optimal labor income tax is progressive.  %\footnote{\ The set-up in this paper can be integrated in the  of as a situation How the trade-off characterizing the optimal level of work effort plays out when the government seeks to generate funds and to mitigate an externality while environmental tax revenues suffice to satisfy the revenue requirement  is left for further research. Clearly: Then the optimal income tax is progressive. }  %\tr{Think about that labor income tax revenues are redistributed back to households but they would not under the weak double dividend hypossis} 

\begin{comment}
When labor supply is fixed, environmental taxes alone can establish the efficient allocation in a representative agent economy absent fiscal distortions. Then, such a tax instrument is optimally set to the social cost of an externality, and originators internalize these social costs: the Pigou principle.
However, not redistributing environmental tax revenues reduces consumption below the efficient level and, as I demonstrate, the optimal environmental tax does not follow the Pigou principle.  If, on top, the  labor supply decision is endogenous, the environmental tax alone features too high labor supply. \tr{This results in too high environmental externality. \textbf{To be shown!}}

Lump-sum transfers of environmental tax revenues restore the efficient allocation: as households become richer, labor supply reduces. When lump-sum transfers are not available, the government can establish the efficient allocation by redistributing environmental tax revenues through an income tax scheme which I demonstrate to be progressive.

content...
\end{comment}

%%%%%%%%%%%%%%%%%%%%%%%%%%%%%%%%%%%%%%%
%\paragraph{Answer WHAT I DO }
%%%%%%%%%%%%%%%%%%%%%%%%%%%%%%%%%%%%%%%
\subsubsection*{First part: Analytic stuff}
\paragraph{simple model}
I propose a simple and comparatively general model to derive the main theoretical results. There are two intermediate sectors of production, one of which exerts a negative environmental externality. The environmental externality is the only distortion motivating government action. In the model, a Ramsey planner seeks to maximize welfare of a representative agent having  an environmental and a labor income tax at its disposal. The income tax scheme is generally non-linear and a common specification in the public finance literature \citep[e.g.][]{Benabou2002TaxEfficiency, Heathcote2017OptimalFramework}.
The model abstracts from  endogenous growth, inequality, and an exogenous government funding constraint. The last two are important abstractions since they traditionally motivate income taxation.

\paragraph{Analytic findings 1}
\textbf{MAIN finding: Progressive income tax} I show that, under mild assumptions, the optimal labor income tax is progressive when no lump-sum transfers are available. 
The optimality of progressive income taxes results from inefficiently high labor supply. The mechanism runs as follows: the use of environmental taxes induces an additional distortion on the labor market by driving a wedge between households' shadow value of income and the social one. The environmental tax reduces the returns to labor below its marginal product. When environmental tax revenues are not redistributed lump-sum, labor supply is too high. 
This is a novel motive for income taxation so far overlooked in the literature.
Following this intuition, environmental and labor income taxes complement each other in the optimal environmental policy.
% I argue that environmental and income taxes are complements. The first is targeted at the environmental externality while the second serves to mitigate distortions in the labor market resulting from environmental taxation.
  
\paragraph{Analytic findings 2}
\textbf{Pigou principle violated absent lump-sum transfers and efficient allocation not feasible}
I consider two cases how environmental tax revenues are recycled. First, revenues are consumed by the government, and, second, the government redistributes revenues via the income tax scheme. In the first scenario, I find that the optimal environmental tax does not satisfy the Pigou principle, i.e., it does not equal the social cost of the externality. As demonstrated by \textit{Pigou xxx}, setting the environmental tax equal to the marginal costs arising from the polluting activity that are not private (the so-called social cost) is optimal. The idea is that polluters behave as if internalizing the social cost of their action when confronted with the environmental tax. I show that when environmental tax revenues are consumed by the government, the Pigou principle is violated. In this case, the motive to increase private consumption by lowering environmental tax revenues makes a deviation of the environmental tax from the social cost of the externality optimal.
Furthermore, when labor supply is elastic non-redistribution of environmental tax revenues results in inefficiently high hours worked. Then, the optimal labor tax is progressive to diminish work effort closer to the efficient level.
Nevertheless, the efficient allocation is not feasible in this policy framework since either consumption is inefficiently low or work effort is too high. 


\paragraph{Analytic findings 3}
\textbf{Efficient allocation can be implemented through redistribution via income tax scheme}
In the second scenario, the Ramsey planner redistributes environmental tax revenues through the income tax scheme, now the efficient allocation is attainable. Thus, even absent lump-sum transfers, there exists a possibility to implement the efficient allocation. By redistributing environmental tax revenues, consumption can be chosen efficiently high while the labor income tax scheme handles too high labor supply. In fact, redistribution via the income tax scheme incentivizes more labor supply. A progressive income tax scheme counteracts this tendency restoring the efficient level of hours worked. 
In this setting, the Pigou principle holds since there is no trade-off between lowering the externality and the allocation of consumption.
%More precisely, I study the optimal policy mix of environmental and labor income taxes to meet emission limits. 
%I find that progressive labor income taxes are used in concert with fossil taxes to optimally reduce emissions. 
% First, I propose a tractable model to provide intuition for this result: Non-lump-sum redistribution of environmental tax revenues increase the gains from labor. Leisure becomes more expensive and households do not reduce their labor supply efficiently in response to the fossil tax. To reduce hours to the efficient level, income taxes complement the fossil tax to lower hours to the efficient level.  
% 
% 
% 
%Second, I assess the importance of this novel role for income taxes in a quantitative endogenous growth model:
%Even though a more progressive tax reduces research effort  and recomposes production towards the fossil sector, the optimal tax is progressive. 


\subsubsection*{Quantitative exercise}
In this and the subsequent three paragraphs, I motivate the quantitative exercise and expound the model.
Even though I have shown theoretically that progressive income taxes form one pillar of the optimal environmental policy when no lump-sum transfers are available, it is not clear whether progressivity remains optimal in a more realistic, quantitative model. 
Two countering mechanisms of tax progressivity shall be considered here.
First, endogenous growth could make a more regressive income tax optimal to incentivize innovation. The size of the market for successful innovation positively affects the profitability of research investment. Subsidizing labor supply through regressive taxes then boosts technological growth. Second, a skill bias in the green sector also renders regressive income taxes advantageous by enhancing supply of the green sector-specific input good. In the model, high-skill workers supply more labor due to a wage premium. They are more responsive to the substitution effect as income tax progressivity makes leisure cheaper. Then, the economy transitions to a higher dirty production share. Directed technical change amplifies this recomposing effect of the income tax. 

% relation to core model and to fried;
In light of these two mechanisms calling for income tax regressivity, I extend the core model studied in the analytical section with endogenous growth and skill heterogeneity. The resulting model builds on \cite{Fried2018ClimateAnalysis}. It extends aforementioned work by an optimal policy analysis, the availability of income taxes, and skill heterogeneity. 
% no income inequality: trade off as in classical 
I maintain the assumption of a representative family which consists of two skill types. Within the family, all workers consume the same. Thus, workers remain perfectly insured against income differentials so that equity concerns do not drive the optimal policy. The trade-off shaping the optimal policy stays within the boundaries of traditional environmental policy considerations: growth versus externality mitigation \citep{Stokey1998AreGrowth, Jones2016LifeGrowth, Acemoglu2012TheChange}.  

%\paragraph{How the externality is modeled}
In the quantitative model, the government cares about the externality due to an exogenous constraint on emissions and not via household utility. This approach, known as a cost-effectiveness approach in the environmental literature, has the advantage of reducing modeling and parametric uncertainty when specifying how greenhouse-gas emissions affect the climate and the damages it produces in terms of well-being and production. Instead, the cost-effectiveness approach uses estimated emission targets stemming from meta studies of more complex integrated assessment models. Additionally, the emission limits I use to calibrate the model are designed to meet climate targets constituting a politically relevant environment. 
% think about the relation of an absolute emission target and not reducing hours

%\paragraph{Endogenous growth}
Allowing for endogenous growth is important to take seriously the possibility of green growth to keep consumption high while meeting emission targets. Endogenous growth introduces dynamics into the model since today's level of technology positively depends on yesterday's technology: \tr{ a \textit{standing on the shoulder of giants} mechanism. LOOK THIS UP \cite{Acemoglu2012TheChange}} 


%The model differentiates between high- and low-skilled labor to account for a skill bias found for the green sector \citep{Consoli2016DoCapital}. This asymmetry of sectors renders regressive taxes a tool to lower relative production costs in the green sector: high-skill workers reduce their labor supply more in response to a more progressive tax as leisure is more valuable to them. This recomposing mechanism counteracts intentions to lower emissions. %, a regressive tax functions as a green subsidy. % In fact, there is an externality arising from high-skill labor supply as it shapes the share of fossil to green energy production. 
%On the other hand, progressive income taxes lower aggregate production by diminishing the price of leisure. 
%Endogenous growth amplifies the repercussions of progressive income taxes:
%First, a lower labor supply reduces the profitability of research in general. By reducing hours worked the planner sacrifices technological progress. Secondly, the recomposing effect is aggravated as research is directed towards the sector with the increased labor share, i.e. the fossil sector.

%\paragraph{Calibration}
The model is calibrated to the US in the baseline period from 2015 to 2019. To do so, I proceed in two steps. First, I set certain parameters to values found in the literature. Most importantly, I use reasonable values of production and growth processes found in \cite{Fried2018ClimateAnalysis}. % who conducts a rigorous calibration exercise. 
With these parameter values at hand, I match the share of high skill in the green and non-green sectors building on \cite{Consoli2016DoCapital}. The emission target is set to the values suggested in the latest IPCC draft on mitigation pathways \citep{IPCC2022}: A 50\% reduction by 2030 relative to 2019 levels and  net-zero emissions from 2050 onward.

%\paragraph{Quantitative exercise}
I perform the following quantitative exercise. 
The Ramsey planner maximizes social welfare as in the core model. However, the externality enters as an exogenous emission limit which constraints government action. The planner sets the environmental tax and the progressivity parameter of the non-linear income tax scheme. \textbf{Environmental tax revenues are consumed by the government.} As discussed in the analytical section, under this policy regime the efficient allocation is not feasible; but, it is  more policy relevant. 

I solve the Ramsey model explicitly for 60 years starting from 2020. From 2080 onward taxes are fixed. The planner is constrained by a sustainability motive to leave at least as much resources to future periods than for the explicitly modeled periods.
This approach has the advantage of not having to assume the existence of a balanced growth path. Given that one factor of production, namely fossil energy, is bounded by an exogenous limit, I seek to stay agnostic on this aspect.
\tr{\textit{Alternative: assume the economy reaches a BGP in some future period; all variables grow at constant rates. Possible?}}

% main finding; then understand why; what are the drivers: how does the result change as things are added or taken from the model
\paragraph{Findings}
I find that the planner chooses a progressive income tax despite its repercussions on growth and emissions. Indeed, this policy increases the share of fossil to green energy and reduces research efforts and consumption. As the emission limit becomes tighter over time, the Ramsey planner augments both the environmental tax and income tax progressivity. The positive correlation between income tax progressivity and the environmental tax is in line with the analytical finding. 
 

% comparison to a model without income tax
To investigate the importance and the welfare gains of an integrated environmental policy, I rerun the main experiment but let the government only choose the environmental tax. Environmental tax revenues are transferred back to households through the income tax scheme.\footnote{\ \textit{This setting does neither correspond to any analytical model version- Could also think of modeling the alternative as gov consumes environmental tax revenues  and no income tax scheme.\textcolor{blue}{Compare: benchmark model where fiscal policy and environmental policy are integrated versus a version where there are neither an income tax scheme nor lump-sum transfers. }} 
}
Comparing the resulting allocation to the one under the benchmark model is informative on the mechanisms and importance of an integrated environmental and fiscal policy when no lump-sum transfers are available. 

\textit{ This means two aspects: (i) redistribution via the income tax scheme and (ii) the use of progressive taxes. Named: Integrated Scenario }

% results
The results suggest a clear cut between responsibilities of policy instruments: The environmental tax targets the externality, while the income tax handles the inefficiency arising from environmental tax revenues. When depriving the government from an income tax scheme, the environmental tax remains largely unchanged compared to the benchmark setting. 
\tr{To do: investigate if even with lump-sum redistribution there is a role for income tax in full model.}
The availability of an income tax increases social welfare by 0.11\% over the period from 2020 to 2080 which corresponds to a consumption equivalent of xxx. 

% optimal income tax with exogenous growth; optimal income tax without skill heterogeneity 

% sensitivity
\tr{In progress}
\begin{itemize}
	\item utility specification (Building on Bick can think of European version when substitution effect is stronger)
	\item no skill bias in the green sector
	\item no endogenous growth
	\item spillovers across scientists: with positive spillovers potentially no growth
	\item counterfactual technology gap
\end{itemize}
Since the target of the labor tax in the environmental setting presented here is to align hours worked with their efficient level, results are sensitive to the elasticity of labor with respect to after-tax wages. 
\paragraph{Quantitative finding to be shaped by income and substitution effect!}
Literature on how households react to changes in income \cite{Bick2018HowImplications} and \cite{Boppart2019LaborPerspectiveb}


%\paragraph{Literature}
\paragraph{Literature}

\paragraph{Optimal environmental policy}
\textbf{Main claim: focus on environmental taxes and recomposition}
\begin{itemize}
	\item with exogenous growth
	\item with endogenous growth
\end{itemize}

In general, papers on optimal environmental policy focus on the optimal environmental tax and analyze settings with inelastic labor supply \citep{Golosov2014OptimalEquilibrium, Acemoglu2012TheChang, Fried2018ClimateAnalysis}. Therefore, the main finding of the present paper, the necessity of reductive policy measures to implement the efficient allocation, complement this literature. Furthermore, I argue that the Pigou principle does generally not apply when no lump-sum transfers are available.  


\paragraph{Recycling of environmental tax revenues}
\textbf{Main claim: they overlook that when lump-sum transfers are not available, then labor supply is inefficiently high, and that env. tax exceeds (?) scc in optimal policy}
\begin{itemize}
	\item in general: \cite{Fried2018TheGenerations}
	\item double dividend literature
\end{itemize}
These findings have important consequences for the literature on the so-called double dividend of environmental policy and the question how to recycle environmental tax revenues. While this literature argues for the recycling of environmental tax revenues to lower pre-existing tax distortions, my paper constitutes an argument for a lower bound on distortionary income taxes: some reduction of labor supply is in fact efficient. 
Furthermore, when revenues are not redistributed lump-sum, the Pigouvian tax does not implement the efficient allocation. 

\paragraph{Environmental protection and inequality}
\ar 1) Inequality and environment as competing goods.

In the literature discussing environmental policies in an unequal framework, a competition between equity and environmental good provision have been discussed. 
This trade-off can be separated into (i) the competition for public funds \citep{LansBovenberg1996OptimalAnalyses, Jacobs2019RedistributionCurves} and (ii) effects of either environmental policies on equity or equalizing policies on environmental quality \citep{Jacobs2019RedistributionCurves, Sager2019IncomeCurves, Dobkowitz2022}. 

Since hours are inefficiently high, equalizing policies become part of the optimal environmental policy. As a byproduct, the distribution of income becomes more equal.


\ar 2) Inequality to shape effects of environmental policies and effect of fiscal policy on environment due to heterogeneity

 Furthermore, the differentiation of skills and the skill-bias of the green sector in the paper give rise to a new channel through which labor taxation affects environmental protection. The literature has primarily focused on a demand channel arising from non-linear Engel curves through which inequality and redistribution shape the degree of dirty production in the economy.   

%\textbf{Non-linear Engel Curves: redistribution}
\cite{Jacobs2019RedistributionCurves} the motive to redistribute and to provide an environmental good compete for government resources due to a negative effect of environmental taxes on the wage rate. Even with lump-sum transfers, the optimal environmental tax does not follow the Pigou principle when the government seeks to enhance equity.

\cite{Sager2019IncomeCurves} argues empirically, that redistribution to poorer households may result in a higher demand for polluting goods. 
\paragraph{Pubic Finance literature}
\textbf{I add: a new perspective on labor income taxes as a tool to lower inefficiently high hours worked. }

 \cite{Heathcote2017OptimalFramework}, \cite{Loebbing2019NationalChange}

\paragraph{Reductive policies in the literature}
The finding relates to the literature discussing rationales for the usage of reductive policy measures. These arise from o
Negative externalities of consumption and hours worked such as
 envy \cite{Alvarez-Cuadrado2007EnvyHours}, habits \cite{Ravn2006DeepHabits} \tr{Check if this is on inefficiency} or a positive externality of leisure \cite{Alesina2005WorkDifferent}. The present paper relates to this literature by identifying an externality of work which emerges from the existence of an environmental externality of production. In addition, once an environmental tax recomposes production towards a cleaner alternative, the wage rate understates the marginal product of labor. \tr{This needs to be made clearer.}
\begin{itemize}
	\item literature which \cite{Alvarez-Cuadrado2007EnvyHours}
\end{itemize}
\tr{This observation relates to the literature in several ways: first, the literature which discusses the optimal recycling of carbon tax revenues. Because when revenues are not recycled as lump-sum transfers, then labor supply is inefficiently high and additional policy measures are necessary to implement the efficient allocation today. 
	 In other words: because lump-sum transfers are not available, the literature argues, the government should use corrective tax revenues to lower pre-existing tax distortions. But by how much? I argue, that there is an optimal size of positive tax distortions when lump-sum transfers are not available. Hence, under the premise of non-lump sum transfers, distortionary labor income taxes arise as an optimal policy tool even absent an exogenous financing condition or inequality. }

 The paper relates broadly to the literature discussing optimal environmental policies. I separate them into two strands: one with inelastic and one with elastic labor supply. 
  \paragraph{Optimal environmental policy: exogenous labor supply}
 
\paragraph{Lit: environmental policy and distortionary fiscal setting}

\begin{itemize}
	\item Williams 2013 Double dividend 
	\item talk to Mireille Chiroleu Assouline: paper on double dividend
	\item Mireille with Aubert or Fodha (PSE)
\end{itemize}
%Inequality-environment nexus: normally motivated by a demand-side perspective; in this project I focus on a supply side explanation
labor supply becomes elastically in the literature studying the interaction of environmental taxes and distortionary taxes.  This strand of the literature generally focuses on the gap between the social cost of carbon and the optimal environmental tax arising from pre-existing distortionary labor income taxes or an exogenous requirement on government funds \citep{Bovenberg1997EnvironmentalGrowth,  Kaplow2012OPTIMALTAXATION, Jacobs2019RedistributionCurves, Barrage2019OptimalPolicy}. labor income taxes form a passive component of these analyses. 
The general findings of this literature is that the optimal environmental tax falls below the social cost of carbon to mitigate efficiency costs and enable the government to raise revenues. 
Furthermore, the literature argues for a recycling of environmental tax revenues to be used to lower income taxes. A recycling through transfers would intensify reductions in labor supply. These arguments rely on the premise that no lump-sum transfers are available. I add to this literature the perspective that a reduction in labor supply is part of the efficient policy. If lump-sum transfers are not available - as is to be assumed in this literature to motivate the existence of distortionary taxes - then labor income taxes should be positive to cope with distortions in the labor supply. Hence, there is a lower bound up to which environmental tax revenues are optimally used to lower distortionary taxes. This is not recognised by the literature. \tr{How do \cite{LansBovenberg1994EnvironmentalTaxation} argue for the use of env. tax revenues to lower distortionary taxes? Verbally or analytically?}

In contrast, I focus the paper on the role of income taxes in the optimal environmental policy and abstract from an exogenous financing condition on the government. Still, the model rationalizes a positive or progressive income tax.
 The inefficiency of environmental taxes arises absent pre-existing income tax distortions or the motive to redistribute.
In difference to this literature, where the presence of a distortionary income tax shapes the optimal level of the environmental tax, the existence of the environmental tax rationalises a progressive income tax in the present paper.
Importantly, the equity and the environmental targets of government intervention are perceived as competing goals as both tax instruments exert efficiency costs through a reduction in labor supply. 
I argue in this paper that what has commonly been perceived as an efficiency cost -  the reduction in labor supply in response to environmental and income taxation - is part of the optimal environmental policy. Hence, income taxation has a double dividend: an environmental and an equity one.  

In this literature, there are either no transfers to households at all \citep{Bovenberg2002EnvironmentalRegulation, LansBovenberg1994EnvironmentalTaxation} or an exogenously given requirement for transfers \citep{Barrage2019OptimalPolicy}. Hence, there is no lump-sum transfer instrument.

\citep{Fullerton1997EnvironmentalComment} writes in its introduction 
\begin{quote}
	With no revenue requirement, or where government can use lump-sum taxes, Arthur C. Pigou (1947) shows that the first-best tax on pollution is equal to the marginal environmental damage.
\end{quote}
\ar What is the optimal environmental tax when labor supply is elastic and there are no lump-sum funds?
\paragraph{Environment and elastic labor supply}
\cite{Oueslati2002EnvironmentalSupply} studies the optimal environmental policy with elastic labor supply. Yet, he allows for lump-sum transfers of environmental revenues. \textit{He should find something on reduction of hours}: No: capital is the only polluting factor, and labor is the clean factor of production.
\paragraph{Recycling of environmental tax revenues}
\cite{Fried2018TheGenerations}
\paragraph{Environment and (endogenous) growth}
\begin{itemize}
	\item limits to growth
	\item general literature on end growth and the environment
\end{itemize}
\paragraph{Public finance}
An equity-efficiency trade-off is central to the discussion of optimal labor income taxes in the public finance literature.  The benefits of labor taxes and progressivity arise, inter alia, from redistribution. %and from generating government revenues. 
With concave utility specifications full redistribution is efficient. However, the optimal tax system does not feature full redistribution when labor supply is endogenous. Instead, redistribution is traded off against aggregate output as individuals reduce their labor supply and skill investment in response to labor income taxation \citep{Heathcote2017OptimalFramework, Conesa2009TaxingAll, Domeij2004OnTaxes}.

To this literature I add another motive for the use of distortionary fiscal policies; namely to reduce inefficiently high labor supply. Furthermore, by abstracting from income inequality or income risk heterogeneity - the present framework
One closely related work is \cite{Loebbing2019NationalChange} who studies optimal income taxation in a model of directed technical change. The redistributive effect of tax progressivity is amplified through a compression of the wage rate distribution xxx
%



\paragraph{Quantitative model}
Recent work has shown, that  higher tax progressivity is amplified in lowering inequality through a compression of the wage rate. (there is a second effect...). On the other hand, high skill labour is used in a higher share in green sectors \cite{Consoli2016DoCapital}. Therefore, progressivity implies a shift to dirty innovations and a higher externality. These channels constitute a new trade-off between inequality and climate change mitigation. 

% motivation from consumption reduction proponents
Furthermore, a higher tax progressivity reduces consumption at higher income levels (only if not smoothed by savings), production and externalities. Then again, high skill labour may be missing for green production. The reduction and recomposition mechanism counteract each other once accounting for skill heterogeneity. 

%\begin{itemize}
%	\item What is the optimal policy to achieve emission targets by 2050?
%	\item role for fiscal policy due to time frame and skill supply?
%	\item inefficient low supply of high skill labour \ar regressive tax optimal?
%	\item demand side: to counteract the negative effect of redistribution through lower skill supply? \ar demand side effect
%	\item voluntary reduction in consumption \ar even lower skill supply? \ar who are these households? (rich/ high low skill?)
%	\item non-monetary motive for scientists? 
%\end{itemize}

\subsection*{Comment: 31/03/22}
It seems difficult to solve the problem when no balanced growth path exist since fossil output has to be constant under the optimal policy. (But that is a constant growth rate, just that output ratios are not constant) 
Therefore,
\begin{enumerate}
	\item solve under assumption of constant growth rates \ar that would be the limit and determines the continuation value of the economy
	\item assume a planner who only cares about the transition to the net-zero emission economy \ar most important to satisfy voters today
	\begin{itemize}
		\item the objective function is the sum of transition periods (2020 to 2080), with the length of a period =5 years \ar 30 periods.
		\item using numeric method as in Barrage should be solvable; no continuation value; (later adding continuation value not a big deal)
		\item no need to make it stationary
		\item how does the economy evolve afterwards?
	\end{itemize}

\end{enumerate}
\subsection*{Comment: 25/03/2022}
In the models on endogenous growth, emissions positively depend on innovations in the dirty sector. Technology in this sector has to be perceived as more goods being produced each of which exerts the same level of the externality. This seems sensible, when these goods need the same input of emissions generating factors but can be produced with the same number of machines. Also sensible when thinking about waste which is on product level. But then again could think of progress as needing less input goods to achieve the same number of outputs, then should measure emissions by input factors and not output. This is captured by growth in the green sector.

Now, assuming emissions are proportional to output, and introducing the exogenous limit on emissions s.t. net-emissions have to equal zero from mid-century onward, then the assumption that on the BGP all technology ratios are constant implies that all growth has to stop. (also assuming here that carbon capture cannot grow without end). This seems very restrictive. Could divert from assumption of constant technology ratios on BGP: instead assuming a generalised BGP which allows for transitions across sectors.  On this GBGP fossil output has to remain at the same level (as an upper bound assume the one prescribed by the IPCC). Then growth in the fossil sector is zero (compatible with BGP, but technology ratios are not constant.) In fact, a BGP. 

Due to spill overs there might be room for ever growing corrective taxes to counter market forces. 
As the green sector growth, green energy becomes relatively cheaper and cheaper compared to fossil energy. This market effect could redirect production and innovation to green energy. 
The labour income tax could support by changing relative skill supply. 

\subsection*{Comment: 19/03/2022}
\begin{itemize}

	\item \textbf{Question 1 a)}: \textbf{how does the presence of a progressive tax (as calibrated) (or a demand target) change the optimal environmental policy?}\\ \ \\
%	\\ Steps
%	\begin{enumerate}
%		\item max swf st emission constraints; save optimal policy
%		\item max swf st emission constraint and demand constraint
%		\item compare optimal policies
%	\end{enumerate}
\textbf{Motivation:} How tax progressivity affects the externality: lowering demand on the one hand reduces emissions as output decreases; on the other hand, labour supply incentives change, potentially more so for high skilled than for low skilled workers.\\
In the literature on optimal environmental policies, positive labour income taxes lower the optimal environmental tax below the Pigouvian rate, due to efficiency costs. In this setting, however, the optimal environmental tax might be higher to counteract the positive effect of income tax progressivity on the externality. \\
Starting from \cite{Fried2018ClimateAnalysis}; set up:  model with rep agent, max social welfare function plus target;  the planner can choose corrective taxes the income tax is given exogenously
\begin{enumerate}
	\item amend model to incorporate income taxes and skill heterogeneity
\item set up her model to find the optimal environmental tax as done in \cite{Barrage2019OptimalPolicy}
\item 
\end{enumerate}
\item[\ar] check model implications in the data: is there heterogeneity in the effect of tax progressvity on the externality across countries which differ in their wage-hour elasticity. 

	\item Question 1 b): (Optimal policy) Is there the potential for the income tax code to be targeted at the externality even if a corrective tax is present and there is no demand target?
Why? Rather, a further reduction of tax progressivity might be optimal to increase high-skill labour supply. 
\ar \textbf{another trade-off between inequality and the externality}. Not only via the efficiency channel but due to redistribution, innovations will be directed towards the polluting sector. \\
Add exogenous demand target as suggested by natural scientists. What is the optimal policy in this case? Could also find that the environemntal tax is used to lower aggregate output when goods are complements! 

\item \textbf{Adding inequality}
On the other hand, the environmental tax could be lower than absent inequality as it increases wage dispersion. 

	\item Question 2 (This refers to \cite{Loebbing2019NationalChange}): reducing consumption bears the potential of social unrest. \ar add inequality. How would a social planner choose to meet the targets if he searches to minimise social tension/ impact on the poor/ utilitarian swf? 
	\\
	set up: two household types, max swf and targets; additional motive to avoid inequality
	\\
	Steps
	\begin{enumerate}
	\item what are the distributional effects of the policy found for question 1?
	\item How would the optimal planner set the optimal policy now? 
	\end{enumerate}
\item why not look at good specific taxes? \ar because (1) hit the poor (regressivity), (2) corresponds to corrective tax! \ar but then no need for income tax
\end{itemize}

New motivation: \\
Natural scientists have identified a reduction in demand for energy and land-intense products as key to meeting climate targets and sustainability goals jointly, or to diminish the reliance on risky carbon dioxide reduction technologies. More broadly, \cite{Arrow2004AreMuch} argue for the efficiency gains of lowering aggregate consumption through the use of public policy instruments. 
However, a dynamic general equilibrium analysis is missing. 

\paragraph{How to get to this?}
\ar Amend \cite{Fried2018ClimateAnalysis}. 
\begin{enumerate}
\item add income tax and elastic labour supply
\end{enumerate}
\subsection*{Comment: 16/03/2022}
\begin{itemize}
	\item this version: the planner maximises social welfare but is constrained by an emission target; the planner only has labour income taxes at its disposal; the economy can be represented by a representative agent
	\item way forward (quantitatively):
	\begin{enumerate}
		\item model rep agent as only supplying one skill \ar $\zeta=1$
			\item depart from log utility of consumption; use preferences with a slightly higher income effect than substitution effect as suggested by \cite{Boppart2019labourPerspectiveb}.
		\item  add a target on demand to the Ramsey problem\ar the planner not only has to meet emission targets but also a target on demand which is motivated by the natural sciences debate on climate change: 
		\begin{quote}Reduction policies alleviate the pressure to meet other sustainability goals \citep{Bertram2018TargetedScenarios}, they reduce the necessity to rely on \textit{carbon dioxide removal} technologies which are not without risk as they rely on  underground CO2 storage and compete with land needed for food production and biodiversity protection \citep{VanVuuren2018AlternativeTechnologies}.
		\end{quote}
	then, the planner might choose income taxes to meet the emission target even if corrective taxes are available.

		\item  introduce heterogeneity  (skills) 
		\item  introduce directed technical change 
		\item what if households want to lower consumption absent any policy intervention? How does this change the analysis?
	\end{enumerate}
	
\end{itemize}

\section{Introduction}
% this intro refers to the following setup:
% The government can only use labour taxes and corrective taxes are not available
% e.g what can the government achieve with common 

% Structure Intro
% 1. Motivation: (2) setting real world, (3) Why is the question interesting? (Tradeoff)

% 2. What I do: Contribution and main finding

% 3. Model (several layers)

% 4. Calibration

% 5. Main quantitative experiment and results
% 
% To do: 
%\tr{ (i) Connect paragraphs,
% (ii) guide reader, 
% (iii) make smooth }

\begin{comment}
\textcolor{violet}{Still to do:
\begin{itemize}
	\item possibilities to model technical change: substitutability of goods, growth in sector, innovation on substitutability versus consumption growth
\end{itemize}
}

content...
\end{comment}

%\paragraph{Classical use of fiscal instruments}
An equity-efficiency trade-off is central to the discussion of optimal labour income taxation and tax progressivity in the public finance literature.  The benefits of labour taxes and progressivity arise, inter alia, from redistribution. %and from generating government revenues. 
With concave utility specifications full redistribution is efficient. However, the optimal tax system does not feature full redistribution when labour supply is endogenous. Instead, redistribution is traded off against aggregate output as individuals reduce their labour supply and skill investment in response to labour income taxation. 

%\paragraph{Environmental Externality}
Adding environmental externalities to the classical public finance framework changes the perception of efficiency costs. Instead of merely reducing welfare, direct benefits through a reduction of the externality arise by lowering output. 
In theory, corrective, environmental taxes can establish the efficient allocation in a representative agent economy. Absent inequality, such a tax instrument is optimally set to the social cost of an externality. Originators then internalise these costs in addition to their private ones. However, governments face political difficulties in implementing such policy instruments.\footnote{\ Compare, for instance, the Yellow Vest movement in France in 2018.} On the other hand, scientific research has emphasised the urgency to act and highlighted the advantages of lowering demand for land and energy.
For example, reduction policies alleviate the pressure to meet other sustainability goals \citep{Bertram2018TargetedScenarios}, they reduce the necessity to rely on \textit{carbon dioxide removal} technologies which are not without risk as they rely on  underground CO2 storage and compete with land needed for food production and biodiversity protection \citep{VanVuuren2018AlternativeTechnologies}.
 Therefore, this paper shifts the focus of optimal environmental policies  to fiscal tax instruments as tools to lower demand and meet emission targets. What can be achieved in terms of climate targets and what are the costs?

%\paragraph{Trade-off/ Mechanisms}
 Consumption reduction in affluent countries has been promoted as an environmental policy \citep{Schor2005SustainableReduction, Pullinger2014WorkingDesign, Arrow2004AreMuch}. But, the general equilibrium effects are less well understood.
While having an advantageous direct effect on the externality, counteracting indirect effects may exist in a general equilibrium framework. Proponents of a reduction policy especially focus on consumption by the rich which consume a higher amount of natural resources.\footnote{\ There is a bunch of research on the consumption of resources by income groupd; see for instance \cite{Sager2019IncomeCurves}.} %\footnote{\ Note Sonja: Abstracting from inequality, would it still be best to reduce consumption by the rich when the poor have a higher marginal propensity to consume dirty? \textit{Could be an important aspect in the model}. Not in the baseline, look at it in an extension...}
This concern could add to the benefits of tax progressivity.
In contrast, targeting rich households in particular for environmental reasons will lower the supply of high skilled labour.\footnote{\ The relation of labour income tax progressivity and skill investment has been studied by \cite{Heathcote2017OptimalFramework}.} Yet, these skills are essentially important in greener sectors of the economy \citep{Consoli2016DoCapital}. As a result, dirty production becomes relatively cheaper and the dirty share of production rises. 
% I want to add endogenous innovation later

%\paragraph{Model}
% this version: With Rep agent
I build a tractable model which incorporates the key aspects sketched above. There are two sectors one of which emits pollutants: the dirty sector. Both clean and dirty goods are necessary inputs to the final consumption good. Sectors produce with a sector-specific labour input good. The labour input good in the clean sector contains a higher share of high-skilled labour. 
The economy behaves as if there was a representative household which provides high and low-skilled labour. The former exerts a higher utility cost for the household generating a wage premium for high-skilled labour. 

The government maximises social welfare from a Ramsey planner's perspective. However, it is constrained by an exogenous limit on emissions. The advantage of this approach is that it suffers less from  model misspesifications due to  uncertainties about how emissions affect the environment. Furthermore, it is closely related to the current political debate.\footnote{\ Compare appendix section \ref{app:emission_climate_targets} for a more in depth discussion of this aspect. } 

%\paragraph{Calibration}
I inform the exogenous emission limit by the  targets proposed in the 2018 report of the Intergovernmental Panel on Climate Change (IPCC)\footnote{\ A body of the United Nations established to assess the science related to climate change.},  \cite{Rogelj2018MitigationDevelopment.}. These targets are designed for states to comply with the Paris Agreement: global net greenhouse-gas emissions in 2030 shall equal 25-30 GtCO2e per year and zero in 2050 (p.95 in \cite{Rogelj2018MitigationDevelopment.}).%Indeed,  the agreement  foresees a tight time frame for emission reductions: climate neutrality should be achieved by mid-century.
\footnote{\ Under this treaty, states have agreed on limiting temperature rise to well below 2°C, preferably to 1.5°C, and to achieve climate neutrality by mid-century \url{https://unfccc.int/process-and-meetings/the-paris-agreement/the-paris-agreement}. }
%Compared to integrated climate assessment models, (CHECK DEFINITION) this approach requires less assumptions concerning the relation of emissions and the climate. What

Another important calibration choice is the substitutability of clean and dirty production in the final consumption good. I make the cautious assumption that goods are no perfect substitutes. In other words, there is always at least a small amount of dirty production necessary to produce the final consumption good. \tr{\cite{Cohen2019AnnualSubstitutable} discuss and estimate the substitutability of natural capital in production with a focus on energy. }

%\paragraph{Quantitative Exercise and Results}
The paper is divided into two parts: an analytical part where I derive propositions concerning the role of fiscal policy and a quantitative part which discusses the optimal policy and transitions. 

The main theoretical result is that in the laissez-faire economy, emissions grow without bound. Irrespective of whether the clean and dirty good are substitutes or compliments.

In the quantitative exercise I let a planner choose the optimal policy by maximizing a utilitarian social welfare function but it faces an constraint on emissions. I solve explicitly for the optimal policy in each period making the optimal tax progressivity time dependent. 


\paragraph{Literature}

The paper is related to 3 strands of literature. 

First, to the public finance literature.  \cite{Heathcote2017OptimalFramework} study optimal labour tax progressivity on 



Second, to the literature on optimal environmental policy. 

Third, to the literature on directed technical change. 
\textbf{HEMOUS and Olsen} discuss an endogenous growth model with heterogeneous labour input:
\begin{itemize}
	\item the wage premium is not constant on a BGP which they specify as stable if innovation occurs in both sectors
	\item hence: a non stable BGP is one where innovation does not occur in both sectors at some point
	\item need to allow for this option< when solving the model
	\item quality ladder model: each scientist after having chosen a sector of production, there is no congestion (each scientist works on one machine) legitimate due to within-sector spillovers
	\item on a BGP with equal growth the wage premium may grow (Result in \cite{Acemoglu2002DirectedChange}) 
	\item in \cite{Acemoglu2012TheChange} 
	\begin{itemize}
		\item as emissions are proportional to dirty output implicit assumption of a Leontief production function if there was energy (and also this as only source of emissions); 
		 \item endogenous labour (no sector-specific labour supply) \ar the more productive sector attracts more labour (the MPL is higher at an equal ratio so that more labour ends up in the more productive sector to have equal wages)
		 \item for substitutes innovation might be stuck in the more advanced market as the price effect (which directs innovation to the less productive market) is muted
		 \item[\ar] in \cite{Fried2018ClimateAnalysis} fossil and green energy are substitutes \ar stuck in fossil innovation; but non-energy goods and energy are complements \ar price effect strong; equalising effect
		 \item in LF stuck with dirty innovation, government can redirect innovation towards the clean sector until it catches up \ar policy intervention needer for a " sufficient amount of time" \ar might be missing in today's world
		 \item with DTC need postponing intervention problematic!
		 \item subsidies and corrective taxes needed to implement first best 
	\end{itemize}
\item \cite{Acemoglu2016TransitionTechnology} incremental (sector-specific) and radical (building on the leading technology irrespective of sector) innovation \ar cross-sectoral spillovers which were absent in \cite{Acemoglu2012TheChange}
\end{itemize}

%Finally, the paper is meant to add to the discussion on reduction versus recomposition policies as tools to reduce human impact on the environment. 

\paragraph{Outline}
The paper is structured as follows. In the next section, I define a simple model. % to analyse the role of fiscal taxes on the environment. 
Section \ref{sec:theory} discusses theoretical optimal policy results. Section \ref{sec:calib} argues for the plausibility of chosen parameter values. In section \ref{sec:simul}, I show dynamics of the economy under the laissez-faire and the optimal policy regimes. 


%\section{Literature}


\subsection{Models}
\begin{itemize}
\item \cite{Bilbiie2012EndogenousCycles}
\begin{itemize}
\item a model with rep agent
\item investment in the form of stock 
\item innovation as a form of new products
\item one final good sector
\item monopolistic competition
\item homothetic preferences
\end{itemize}
\item \cite{Ravn2006DeepHabits}
\begin{itemize}
 \item habits over average previous consumption of specific good! not over total consumption
 \item rep agent 
 \item habits: marginal utility rises as habits rise \ar could look at what happens as habits are reduced! \ar marginal utility at given consumption level reduces!
 \item more is always better! Plus increases habits \ar I want: that more might not be better after some point
\end{itemize}
\item \cite{McKay2021LumpyPolicy}
\begin{itemize}
\item New Keynesian model with durable and non-durable consumption 
\end{itemize}
\item \cite{Acemoglu2012TheChange}
\begin{itemize}
\item endogenous growth
\item rep agent
\item single labour market
\item no resource use in clean sector! ; abstracts from waste
\item disaster risk!: There is a lower bound on the quality of the environment 
\item environmental externality only affects Utility! So no chance for \textbf{environmental quality} to drive production to zero!
BUT there is a natural resource which is used in production; \textit{How do the two relate?} \ar when environmental quality affects regeneration of exhaustible resource, then there would be some connection, but there is no regeneration of the resource, I think
\item there is degradation of the environment through unsustainable production (only!) and 
\end{itemize}
Functional forms
\begin{align*}
S\in[0,\bar{S}],\ & \text{where}\ \bar{S}\ \text{is the quality of the environment without pollution;}\\
S_v=0 \Rightarrow S_t=0 \forall t\geq v,\ &  0 \ \text{is the point of no return.}\\
\underset{S\rightarrow0}{lim} U(C,S)=-\infty\ & \text{S=0 is a disaster!}\\
\underset{S\rightarrow0}{lim}\frac{\partial U(C,S)}{\partial S}=\infty\ &\\
S_{t+1}= -\xi Y_{dt}+(1+\delta)S_t& \\ 
\text{evolution of environmental quality:} & \text{ falls in dirty production; regeneration rate }\\
 \text{both are exponential relationships}\Rightarrow&\text{ smaller env. quality slower regeneration}\\ 
 &\text{ higher pollution, stronger degradation}
\end{align*}
The dirty sector uses an exploitable resource in the production process
\begin{align*}
Y_{dt}= R_t^{\alpha_2}L_{dt}^{1-\alpha}\int_{0}^{1}A_{dit}^{1-\alpha_1}x_{dit}^{\alpha_1}di
\end{align*}
$R_t$ is the exhaustible resource
\begin{align*}
Q_{t+1}=Q_t-R_t
\end{align*}
they look at a version where the resource is common property (water, air) or owned (Hotelling rule)
\item \cite{Heikkinen2015DegrowthConsumers}: macro model with voluntary reduction in consumption
\item \cite{Borissov2019CarbonDevelopment}: model labour sector in more detail: skill, sectors, and transition
\item \cite{Michaillat2015AggregateUnemployment, Auerbach2021InequalityEconomy} examples of models with economic slack. But both do not feature a satiation point of consumption. 
\end{itemize}

\subsection{Motivation}
\begin{itemize}
\item \cite{Schor2005SustainableReduction}
\begin{itemize}
	\item arguments against unlimited growth
	\begin{itemize}
\item hhh
	\end{itemize}
\end{itemize}
\item \cite{Dasgupta2021}
\begin{itemize}
\item emphasises the use of nature as a sink (stock) and as an input to production \ar can the two be combined?
\end{itemize}
\end{itemize}
 %<- contains summaries of potentially relevant papers
\section{Core model and theoretic results}\label{sec:mod_an}

This section develops a tractable model  to derive the theoretic results. I show that scaling the level of production is part of the efficient environmental policy. Yet, absent endogenous growth, this is fully implemented by the use of an environmental tax and lump-sum transfers. There is no role for labor income taxes.

\subsection{Model}
The representative household faces a consumption and labor supply decision. The final consumption good is a composite of a fossil and a green good. Labor is the only input to production. The fossil sector causes an environmental externality.\footnote{ For simplicity, the green sector does not induce any externality; yet, whenever intermediate goods are no perfect substitutes, final good production is never perfectly green.} There is no growth, and the model is static.

\paragraph{Representative household}
Throughout the paper, the household's decision is static. Each period, the household maximizes its period utility
\begin{align*}
U(C,H; F).
\end{align*} 

The household derives utility from consumption, $C$, but experiences disutility from hours worked, $H$. An externality from fossil production, $F$, decreases household utility. The level of fossil production is taken as given by the household.
I assume additive separability of consumption, hours, and the externality. Utility of consumption is increasing and strictly concave. Utility is decreasing and strictly convex in hours worked and fossil production.
Utility maximization is subject to a period budget constraint:
\begin{align}
	 C= \lambda(wH)^{1-\tau_{\iota}}+T_{ls}. \label{eq:hhbudget}
\end{align}
The variable $w$ indicates the wage rate.  Lump-sum transfers from the government are denoted by $T_{ls}$.
The government levies income taxes on labor income using a non-linear tax scheme common in the public finance literature \citep{Heathcote2017OptimalFramework, Benabou2002TaxEfficiency}. The tax scheme is
characterized by (i) a scaling factor, $\lambda$, which determines the level of average tax revenues in the economy, and (ii) a measure of tax progressivity denoted by $\tau_{\iota}$. 
\cite{Heathcote2017OptimalFramework} show that whenever $\tau_{\iota}>0$, the tax scheme is progressive since the marginal tax rate exceeds the average tax rate irrespective of  pre-tax labor income. Hence, average tax payments increase with labor income.\footnote{ An alternative intuition is that when $\tau_{\iota}>0$, the elasticity of post- to pre-tax  income is smaller unity for all levels of pre-tax income.  } %\footnote{ I show that the result is equivalent with a linear tax rate in the appendix.} 
With a representative household, $\tau_{\iota}$ can be understood as an instrument to regulate labor supply and, thus, the overall level of production. When $\tau_{\iota}<0$, the government subsidizes labor, with $\tau_{\iota}>0$, it discourages labor. 

\paragraph{Production}
All sectors of production are perfectly competitive, and production functions have decreasing returns to scale. %\footnote{ \textit{With increasing returns to scale the assumption of perfect competition would be violated. With constant returns to scale, the solution is not unique.}}. The final consumption good, $Y$, is a composite of the fossil, $F$, and the green intermediate good, $G$. 
Intermediate goods, indicated by $J\in \{F,G\}$ for fossil and green, are produced from the labor input good, $L_J$, using technology, $A_J$. The variable $Y$ stands in for final output and is the numeraire. Production is given by:
\begin{align}
Y=Y(F, G), \hspace{5mm} F=F(A_F, L_F),\hspace{5mm} G=G(A_G, L_G). \label{eq:prod}
\end{align}

\paragraph{Government}
The government raises income taxes from households and levies an environmental tax, $\tau_F$, per unit of fossil energy bought by final good producers. The environmental tax, thus, is modeled in parallel to a carbon tax which poses a price on emissions. Revenues from the income tax and the environmental tax are treated separately by the government. Income tax revenues are fully redistributed through the income tax schedule. Environmental tax revenues are rebated lump sum to households:
\begin{align}
\tau_{F}F=T_{ls}, \hspace{7mm}
0={w H}-\lambda(w H)^{1-\tau_{\iota}}. \label{eq:gov_but}
\end{align}
The scaling parameter $\lambda$ adjusts to balance the income tax scheme. 
%Environmental tax revenues are either transferred lump-sum, fully consumed by the government, or transferred through the income tax schedule.

\paragraph{Markets}
Markets for labor and the final good both clear: 
\begin{align}
H=L_F+L_G,\ \hspace{5mm} Y=C. \label{eq:market_clear}
\end{align}
%I summarize the eq.s determining the competitive equilibrium in appendix Section \ref{app:model}.
\paragraph{Competitive equilibrium}
In a competitive equilibrium, household behavior is determined by the budget constraint, eq. \eqref{eq:hhbudget}, and labor supply which follows from the household's first order conditions and substitution of $\lambda$ from the government's budget on the income tax:
\begin{align}
-U_H=U_C(1-\tau_{\iota})w. \label{eq:hsup}
\end{align}
Firms choose the quantity of input goods to maximize their profits taking prices as given. The following equations describe this behavior in equilibrium:
\begin{align}
p_G=\frac{\partial Y}{\partial G}, \hspace{5mm}
p_F +\tau_{F} = \frac{\partial Y}{\partial F}, \hspace{5mm}
w= p_F\frac{\partial F}{\partial L_F}=p_G\frac{\partial G}{\partial L_G}.\label{eq:profmax}
\end{align}

The competitive equilibrium is defined as prices and allocations so that households and firms behave optimally; i.e. eqs. \eqref{eq:hhbudget}, \eqref{eq:hsup}, and \eqref{eq:profmax} hold. Production happens according to eqs. \eqref{eq:prod}.  Equilibrium prices and the wage rate adjust to clear markets, eqs. \eqref{eq:market_clear}. Finally, the government's budgets are satisfied eqs. \eqref{eq:gov_but}. Policy variables $\tau_F$ and $\tau_\iota$ are taken as given. 

\subsection{Theoretic results}\label{sec:theory}
Section \ref{subsec:sp2} defines and discusses the efficient allocation. It constitutes a benchmark for the optimal allocation discussed in Section \ref{subsec:decen_ec}. 

\subsubsection{Social planner}\label{subsec:sp2}
Let the share of fossil to total labor be denoted by $s=\frac{L_F}{H}$. The social planner's problem reads
\begin{align*}
\underset{s, H}{\max}\ & U(C,H; F)\\ s.t.\ \ & C=Y.
\end{align*}
The first order conditions are given by
\begin{align}
wrt. s:\hspace{4mm} & U_C \cdot \left(\frp{Y}{F}\frp{F}{s}+\frp{Y}{G}\frp{G}{s}\right)=-U_F\frp{F}{s}, \label{eq:fbs2}\\
wrt. H:\hspace{4mm}& U_C\frp{Y}{H}+U_F\frp{F}{H}=-U_H. \label{eq:fbh}
\end{align}
Where $U_X$ denotes the partial derivative of utility with respect to the variable $X$.
These equations determine the efficient or first-best allocation. 
Absent an externality, $U_F=0$, the efficient distribution of labor equalizes the marginal product of labor across sectors; compare eq. \eqref{eq:fbs2}. Efficient hours balance the marginal utility gain from consumption and the marginal disutility from working, as formalized by eq. \eqref{eq:fbh}. 

When there is an externality, the social planner adjusts the allocation in two ways: (i) a compositional adjustment, that targets the share of fossil production, and (ii) a scaling adjustment amending the level of production. 
The compositional adjustment is determined by eq. \eqref{eq:fbs2}.
The negative externality of fossil production makes it efficient to alter the share of fossil labor so that  a marginal reallocation of labor to the fossil sector would raise output.\footnote{ Note that $U_F<0$ by assumption so that the right-hand side is positive and that $\frac{dG}{ds}<0$. Hence,  in the efficient allocation, the marginal product of labor in the fossil sector is higher than in the green sector.} %Hence,$\frp{Y}{F}\frp{F}{s}>-\frp{Y}{G}\frp{G}{s}$ is efficient. }
I show in Appendix \ref{app:redeffs} that the social planner reduces the fossil labor share when the aggregate production function features decreasing returns to scale in its labor inputs, $L_G$ and $L_F$.


The scaling effect is summarized by eq. \eqref{eq:fbh}.
First note that eq. \eqref{eq:fbh} can be rewritten by substituting eq. \eqref{eq:fbs2} and noticing the relation of derivatives with respect to $H$ and $s$.\footnote{ This is done in more detail for the optimal allocation in Appendix \ref{app:incometax0}. Relations of derivatives are summarized in Appendix \ref{app:dervs_use}.}  
The second first order condition becomes:
\begin{align}\label{eq:fbh_simp}
-U_H=U_C\frac{\partial Y}{\partial G}\frp{G}{L_G}.
\end{align}
Hence, the efficient level of the externality lowers hours as if the marginal product of labor was equal to the marginal product of labor in the clean sector.

The recomposition of labor towards the  green sector reduces the average marginal product of labor in the economy. An additional unit of labor results in a smaller increase in consumption.  This effect has two opposing impacts on the efficient level of labor. On the one hand, there is a substitution effect: as leisure becomes less costly, the efficient amount of hours reduces (note that the right-hand side of eq. \eqref{eq:fbh} is increasing in $H$). On the other hand, the economy becomes poorer in terms of consumption, and more work effort might be efficient. This is captured by the term $U_C$ and equivalent to an income effect. 
%In total, which effect dominates depends on the curvature of the utility from consumption, $\theta$. With $\theta>1$ the  lower marginal product of labor decreases the efficient amount of hours worked. 
%Second, the social planner reduces hours worked due to their negative exeternality through fossil production. This effect is introduced by the term $U_F\frac{dF}{dH}<0$. 
Proposition \ref{prop:0} summarizes this discussion.
\begin{prop}\label{prop:0}
	Efficient externality mitigation consists of a compositional and a scaling adjustment. 
\end{prop}


Depending on the importance of the income effect, efficient hours worked may be higher or lower than  absent an externality. I will show in the following, however, that there is no role for labor income taxation in implementing the efficient allocation. In fact, under the optimal policy, the wage rate is set so that households internalize the effect of work effort on emissions. %\footnote{ \ I discuss in the appendix conditions on parameter values when assuming functional forms of the model.}
%I will show in the following, that irrespective of whether the social planner de- or increases hours, the decentralized economy always features higher hours when environmental tax revenues are not redistributed lump-sum. 


\subsubsection{Decentralized economy}\label{subsec:decen_ec}

Governments use tax and transfer instruments to correct for distortions, such as an environmental externality. The question arises if the efficient allocation can be decentralized by the use of taxes and transfers in a competitive economy.  %For now, I assume that the income tax is not available and $\tau_{\iota}=0$, $\lambda=0$.

%I show in this section that lump-sum redistribution of environmental tax revenues is essential to implement the first-best allocation in the competitive equilibrium. Only in combination with lump-sum transfers of  environmental tax revenues does an environmental tax suffice to implement the efficient allocation. %Then the environmental tax equals the social cost of the externality as shown by \textit{PIGOU}. 
%When environmental tax revenues are not redistributed lump-sum, hours worked exceed their efficient level, and a role for income taxes to lower hours worked arises. I consider two cases.

%\begin{enumerate}
%\item lump-sum transfers important for Pigou tax to implement efficient allocation: Proposition \ref{prop:1}
%\item when transfers are not redistributed: infeasibility of efficient allocation,  role for labor tax, and violation of Pigou principle \ref{prop:2}.
%\item redistribution through income tax scheme with progressive income tax restores efficient allocation \ref{prop:3}
%\end{enumerate}

%\subsubsection{Government problem}\label{subsec:Rams}
The government is characterized by a Ramsey planner who maximizes utility of the representative household by use of tax and transfer instruments. The behavior of firms and households constrain the government's optimization problem. 
The Ramsey problem is defined as
\begin{align*}
\underset{s, H}{\max}\ & U(C,H; F)\\ s.t.\ \ &  C=Y,
\end{align*}
subject to the behavior of households and firms.
The first order conditions are equivalent to the social planner ones:
\begin{align}
wrt.\ s:\hspace{4mm} & U_C\cdot\left(\frac{\partial Y}{\partial F}\frac{\partial F}{\partial s}+\frac{\partial Y}{\partial G}\frac{\partial G}{\partial s}\right)=-U_F\frac{\partial F}{\partial s}, \label{eq:sbs}
\\
wrt.\ H:\hspace{4mm} & U_C\frac{\partial Y}{\partial H}+U_F\frac{\partial F}{\partial H}=-U_H\label{eq:sbh}. 
\end{align}
%-- paragraph to show that with Gov=0 and lump-sum transfers, the efficient allocation is implemented
When environmental tax revenues are fully redistributed lump sum, an environmental tax equal to the marginal social cost of fossil production implements the efficient allocation.\footnote{ I define and derive the social cost of fossil production in Appendix \ref{app:scp}.} This observation is known as the \textit{Pigou principle} in the literature. 
To see this, note that eq. \eqref{eq:sbs} ensures that the social planner's first order condition, eq. \eqref{eq:fbs2}, is satisfied. 
Rewriting eq. \eqref{eq:fbs2} reveals that the Pigou principle holds: %\footnote{ I derive the social cost of pollution as the price the representative household is willing to pay for a marginal reduction in fossil production. The derivation is exponded in appendix Section \ref{sec:mod_an}. 
%	To be precise, social cost of pollution refers to the marginal cost evaluated at the resulting equilibrium allocation.}: The Pigou principle. 
\begin{align*}
\underbrace{\frac{-U_F}{U_C}}_{\text{marginal social cost of fossil production}}=\left(1+\frac{\frac{\partial Y}{\partial G}\frac{\partial G}{\partial s}}{\frac{\partial Y}{\partial F}\frac{\partial F}{\partial s}}\right)\frac{\partial Y}{\partial F}=\tau^*_F.
\end{align*}
Where the second equality follows from substituting firms' profit maximization conditions from eqs. \eqref{eq:profmax}.

Absent an externality of production, it is efficient to balance marginal products of labor across sectors.
When there is an externality, the social planner lowers the share of labor in the fossil sector. As a result, the marginal product of labor in this sector increases. It falls in the green sector. To sustain this gap between marginal products in the competitive equilibrium, the government has to introduce a corrective tax. Otherwise, market forces would direct labor towards the sector with the higher marginal product. Consequently, the equilibrium wage rate is below the marginal product of labor.\footnote{ I formally discuss this statement in Appendix \ref{app:wageMPL}.} 

Setting the environmental tax equal to the social cost of fossil production implies that the second first order condition of the Ramsey planner, eq. \eqref{eq:sbh}, is satisfied without use of the income tax instrument: $\tau_{\iota}^*=0$. 
The reason is, that in this case, the wage rate reflects the marginal social costs of hours through raising emissions. I show in Appendix \ref{app:incometax0} that the wage rate can be written as:
\begin{align*}
w = \frp{Y}{H}+\frac{U_F}{U_C}\frp{F}{H}.
\end{align*}
Since $U_F<0$, the second summand reduces the wage rate beyond the marginal product of labor in the economy.
Therefore, households internalize the marginal social costs of the externality of hours worked in their labor supply decision. Relative to no policy intervention, labor supply declines. Proposition \ref{prop:1} condenses this result.

\begin{prop}\label{prop:1}
	The efficient allocation is implemented by an environmental tax and lump-sum transfers.  When the environmental tax implements the efficient share of dirty labor, the wage rate fully captures the marginal effect of hours worked on the externality. There is no role for distortive labor income taxation, $\tau_{\iota}^*=0$.
\end{prop}
Proof: Appendix \ref{app:incometax0}. 

%As discussed previously,  However,

%Due to this effect of the environmental tax on the wage rate, lump-sum transfers and environmental taxes alone suffice to implement the efficient level of hours worked. 


%\section{Analytic results}
\textbf{Points to be made}
\begin{enumerate}
\item the efficient allocation consists of both a recomposing and a reductive element \ar discuss social planner allocation \checkmark
\item lump-sum transfers implement the efficient reduction in hours worked \checkmark
\item absent lump-sum transfers, households work too much \checkmark
\item the income tax which implements the efficient allocation is positive/ progressive \checkmark
\end{enumerate}

This section develops a tractable model to investigate the inefficiency arising in hours worked when an environmental externality has to be taken care of. 

\subsection{Core model}
The representative household faces a consumption and labor supply decision. The final consumption good is a composite of a dirty and a clean good. labor is the only input to production. For simplicity the clean sector does not induce any externality. The model also abstracts from endogenous growth and becomes static. The planner levies income taxes on labor income using a tax scheme common in the public finance literature \citep{Heathcote2017OptimalFramework, Benabou2002TaxEfficiency}.\footnote{\ I show that the result is equivalent with a linear tax rate in the appendix.} 

The representative household maximises its period utility
\begin{align}
\frac{C^{1-\theta}-1}{1-\theta}-\chi \frac{h^{1+\sigma}}{1+\sigma}-E(F)
\end{align}
subject to a budget constraint
\begin{align}
	 C= \lambda(wh)^{1-\tau_{\iota}}+T.
\end{align}
The household derives utility from consumption but experiences disutility from working. An externality from dirty (or fossil) production, $F$, decreases household utility and is taken as given by the household.

All sectors of production are perfectly competitive. The final consumption goodm $Y$, is a Cobb-Douglas composite of the dirty, $F$, and the clean good, $G$. Intermediate good $j\in \{f,g\}$ is produced from labor, $L_j$m and total factor technology, $A_j$. 
\begin{align}
Y=F^{\varepsilon}G^{1-
	\varepsilon}, \hspace{4mm}
F=A_fL_f, \hspace{4mm}
G=A_gL_g.
\end{align}
The government raises income taxes and levies sales taxes, $\tau_f$, on dirty production revenues $p_fF$. Revenues from the income tax and the environmental tax are treated separately by the government. Environmental tax revenues are transferred lump-sum, while income tax revenues are fully redistributed through the income tax schedule:
\begin{align}
T=\tau_{f}p_fF, \hspace{4mm}
0={w h}-\lambda(w h)^{1-\tau_{\iota}}.
\end{align}
The market for labor clears, $h=L_f+L_g$, and the final consumption good is the numéraire.
I summarize the equations determining the competitive equilibrium in appendix section \ref{app:model}.

\subsection{Social planner}
Let the share of dirty to total labor be denoted by $s=L_f/h$. The social planner's problem reads
\begin{align}
\underset{s, h}{\max}\ & \frac{C^{1-\theta}-1}{1-\theta}-\chi \frac{h^{1+\sigma}}{1+\sigma}-E(F)\\
s.t\ \ & C=\left(A_fs\right)^{\varepsilon}\left(A_g(1-s)\right)^{1-\varepsilon}h
\end{align}
The first order conditions are given by
\begin{align}
wrt.\ h:\ & C^{-\theta}\underbrace{(A_fs)^{\varepsilon}(A_g(1-s))^{1-\varepsilon}}_{MPL}=\chi h^\sigma+\frac{dE}{dF}\frac{dF}{dh}\label{eq:fbh},\\
wrt.\ s:\ & C^{-\theta}\underbrace{(A_fs)^{\varepsilon}(A_g(1-s))^{1-\varepsilon}}_{MPL}h\underbrace{\left(\frac{\varepsilon(1-s)-s(1-\varepsilon)}{s(1-s)}\right)}_{\text{how s changes MPL}}=\frac{dE}{dF}\frac{dF}{ds}. \label{eq:fbs}
\end{align}
These equations determine the efficient or first-best allocation. 
Absent an externality, $\frac{dE}{dF}=0$, the efficient share of labor in the dirty sector is to maximise the marginal product of each hour worked and $s^{FB,E=0}=\varepsilon$, compare equation \ref{eq:fbs}. Efficient hours equalize the marginal utility gain and the disutility from working. 

When there is an externality, the social planner adjusts the allocation by two modulations: a (i) recomposing and a (ii) scaling one. 
The recomposition is determined by equation \ref{eq:fbs}. The negative externality of dirty production lowers the efficient share of labor allocated to  the dirty sector and $s^{FB,E>0}<s^{FB,E=0}$. 

The scaling effect is summarized by equation \ref{eq:fbh}. There are two reasons for why the efficient amount of hours worked changes. 

\begin{align}
h_{FB}= \left(\frac{w_{FB}^{1-\theta}}{\chi}\frac{1-\varepsilon}{1-s}\right)^\frac{1}{\sigma+\theta}\label{eq:heff}
\end{align}
where $w_{FB}=(A_f s)^{\varepsilon}(A_g(1-s))^{1-\varepsilon}$.\footnote{\ For the derivation see appendix section \ref{app:derivations}.}


First, the recomposition of labor input towards the  clean sector reduces the marginal product of labor, and the marginal increase in consumption for an additional hour worked declines.  This effect is captured by $w_{FB}$. On the one hand, there is a substitution effect and leisure becomes less costly and the efficient amount of hours reduces. On the other hand, the economy becomes poorer and more work effort might therefore be efficient; an income effect. In total, which effect dominates depends on the curvature of the utility from consumption, $\theta$. With $\theta>1$ the  lower marginal product of labor decreases the efficient amount of hours worked. 
Second, the social planner reduces hours worked due to their negative exeternality through production. This effect is captured by the fraction $\frac{1-\varepsilon}{1-s}$; a measure of the distance of $s$ to $\varepsilon$ and hence the severity of the externality. This factor is shown to be below unity in step 1 in the appendix section \ref{app:derivations}.

One can show that the total effect of a drop in the dirty labour share on hours worked is positive, i.e. $\frac{dh_{FB}}{ds}>0$, if $\theta<\frac{\varepsilon}{\varepsilon-s}$. If the income effect dominates, the social planner increases hours worked as the economy becomes less productive. 
Under the value for $\theta$ suggested by \cite{Boppart2019LaborPerspectiveb}, the efficient scale effect is to increase hours worked. When, however, the substitution effect outweighs or dominates the income effect - as commonly assumed in the public finance literature \citep{Heathcote2017OptimalFramework, LansBovenberg1994EnvironmentalTaxation, LansBovenberg1996OptimalAnalyses} \tr{CHECK this}- 
When the efficient level of hours increases, though, the dirty labor share reduces even more to outweigh the increase in the externality.

\subsection{Decentralized economy}

How can the efficient allocation be decentralized by the use of taxes in a competitive economy? For now, I assume that the income tax is not available and $\tau_{\iota}=0$, $\lambda=0$.
I will show that lump-sum redistribution of environmental tax revenues are essential to implement the first-best allocation in the competitive equilibrium. This is the first main result summarized by proposition \ref{prop:1}. 

\begin{prop}\label{prop:1}
Even if the Ramsey planner sets the environmental tax to the social cost of carbon, %to implement the first-best share of clean labor, $s$,
the optimal allocation is inefficient absent additional measures to adjust hours worked. Lump-sum transferring environmental tax revenues are one means to change hours worked to the efficient level. 
\end{prop}

%\textbf{Show: derive optimal amount of lump-sum transfers.}

To proof this claim, I first show that hours worked in an economy with neither transfers nor labor income taxes exceed the level in the efficient allocation. The proof is depicted in appendix section \ref{app:derivations}. 
show that lump-sum transfers which implement the efficient amount of hours worked are positive when the corrective tax is set to the social cost of carbon. Through an income effect the additional income reduces work effort to the efficient level. Transfers equal exactly the amount raised by the environmental tax. 

Note that in the competitive economy, the environmental tax determines the share of dirty labor, $s$. 

The competitive level of hours worked as a function of lump-sum transfers and the income tax is given by

\begin{align}
h = \left[\frac{w^{1-\theta}\left(1+\frac{T}{wh}\right)^{-\theta}(1-\tau_{\iota})}{\chi}\right]^{\frac{1}{\sigma+\theta}}.\label{eq:hopt}
\end{align}
When neither transfers nor income taxation is available the expression simplifies to $h=\left(\frac{w^{1-\theta}}{\chi}\right)^\frac{1}{\sigma +\theta}$.


In the competitive equilibrium, the wage rate is
\begin{align}
w= (1-\varepsilon)(A_fs)^\varepsilon (A_g)^{1-\varepsilon}(1-s)^{-\varepsilon}. \label{eq:compw}
\end{align}


\begin{prop}
	If lump-sum transfers are not available, income taxes allow to replicate the efficient amount of hours worked.
\end{prop}

I will first proof that lump-sum transfers are positive. Then I will derive the income tax progressivity which replicates the efficient allocation. 
The Ramsey planner's problem is given by
\begin{align}
\underset{s, h}{\max}\ & \frac{C^{1-\theta}}{1-\theta}-\chi \frac{h^{1+\sigma}}{1+\sigma}-E(F)\\
s.t\ \ & C=\left(A_fs\right)^{\varepsilon}\left(A_g(1-s)\right)^{1-\varepsilon}h
\end{align}

\subsection{Source of inefficiency}
With a linear tax schedule and lump-sum transfers, labor supply reacts sufficiently to the reduction in income and the efficient allocation can be implemented solely by the use of the corrective tax. Under the more flexible benchmark tax schedule, transfers, $\lambda$, are multiplicative which raises the returns to labor intensifying the substitution (?) effect of the wage rate. As the wage rate falls, households work more. To counteract this tendency, the government has to impose a progressive tax scheme, given that 


\subsection{Numeric results in simple model}
\begin{table}[h!!]
	\caption{Linear tax scheme and lump-sum transfers}\label{tab:lin_lst}
	\begin{tabular}{lllllllll}
		Thetaa & FB hours & FB Pigou & CE hours & CE scc & Opt hours & Opt taul & Opt tauf & Opt scc \\ 
		\hline 
		<1 & 1.192 & 0.99326 & 1.192 & 0.99326 & 1.192 & -3.7748e-15 & 0.99326 & 0.99326 \\ 
		Bop & 0.13601 & 0.99959 & 0.13601 & 0.99959 & 0.13601 & -3.7748e-15 & 0.99959 & 0.99959 \\ 
		log & 0.36434 & 0.99853 & 0.36434 & 0.99853 & 0.36434 & -3.7748e-15 & 0.99853 & 0.99853 \\ 
		\hline 
	\end{tabular}
\end{table}
\begin{table}
	\caption{Linear tax scheme, env. tax revenues not transferred lump-sum}\label{tab:lin_nolst}
	\begin{tabular}{lllllllll}
		Thetaa & FB hours & FB Pigou & CE hours & CE scc & Opt hours & Opt taul & Opt tauf & Opt scc \\ 
		\hline 
		<1 & 1.192 & 0.99326 & 1.2061 & 0.97804 & 1.1706 & 0.049876 & 0.9934 & 0.94584 \\ 
		Bop & 0.13601 & 0.99959 & 0.14026 & 0.96001 & 0.13808 & 0.049876 & 0.99958 & 0.94766 \\ 
		log & 0.36434 & 0.99853 & 0.37243 & 0.97015 & 0.36435 & 0.049876 & 0.99853 & 0.94804 \\ 
		\hline 
	\end{tabular}
\end{table}
\begin{table}[h!!]
	\caption{Baseline model env. revenues transferred via income tax scheme ($\lambda$)}\label{tab:base}
	\begin{tabular}{lllllllll}
		Thetaa & FB hours & FB Pigou & CE hours & CE scc & Opt hours & Opt taul & Opt tauf & Opt scc \\ 
		\hline 
		<1 & 1.192 & 0.99326 & 1.2275 & 1.0056 & 1.192 & 0.049979 & 0.99326 & 0.99326 \\ 
		Bop & 0.13601 & 0.99959 & 0.13811 & 1.0311 & 0.13601 & 0.049979 & 0.99959 & 0.99959 \\ 
		log & 0.36434 & 0.99853 & 0.37243 & 1.0211 & 0.36434 & 0.049979 & 0.99853 & 0.99853 \\ 
		\hline 
	\end{tabular}
\end{table}



Table 1 to 3 compare the efficient allocation to an allocation resulting in the competitive equilibrium when the environmental tax is set to equal the social cost of carbon in the efficient allocation. The rationale being that without any further distortions setting environmental taxes to the social cost of carbon implements the efficient allocation. The last four columns of each table show hours worked, the optimal policy and the social cost of carbon in equilibrium resulting in the Ramsey planner allocation. 

Table \ref{tab:lin_lst} reveals that indeed, setting the corrective tax equal to the social cost of carbon under the social planner implements the first-best allocation when lump-sum transfers are available. The optimal policy chooses zero income taxes. 

The picture changes once no lump-sum transfers are available, compare table \ref{tab:lin_nolst}. In the competitive equilibrium setting the environmental tax to the social costs of carbon under the social planner results in inefficiently high hours worked for all values of $\theta$ considered. The optimal policy is to set a positive income tax rate; the optimal income tax code is progressive. When the substitution effect outweighs the income effect, i.e., $\theta<1$, then the optimal allocation results in inefficiently \textit{low} hours worked. When the income effect is at least as strong than the substitution effect, that is $\theta\geq 1$, then hours worked remain inefficiently high under the optimal policy. 

Interestingly, when the planner transfers environmental tax revenues through the income tax scheme, table \ref{tab:base}, then the efficient allocation is attainable for all values of $\theta$ considered through a progressive tax scheme. 

Only when the Ramsey planner can implement the efficient level of work, the environmental tax is set to equal the social cost of carbon.   




\textbf{In a nutshell}
\begin{itemize}
	\item hours worked without transfers are always too low even if efficient tax rate is chosen
	\item when hours are not efficient, then the environmental tax does not match the social cost of carbon
	\item when revenues are transferred through the income tax, the planner can implement the efficient allocation with the help of a progressive income tax \textit{(interesting!)}
	\item with $\theta<\frac{\varepsilon}{\varepsilon-s}$ optimal hours worked reduce, otherwise the income effect is too strong and hours worked increase! 
	Nevertheless, the allocation in LF without lump sum transfers always features too high hours worked. 
	\item why does the optimal policy with taul but no lump-sum transfers not implement the efficient level? \ar income taxes are not a measure to implement the efficient allocation; only similar when income and substitution effect cancel. Too high when income effect dominates, too low when substitution effect dominates.
 \ar general consumption tax should neither be able to implement efficient allocation! 
 %\item when there is no income tax, the optimal policy is to set the efficient dirty labour share (compare table \ref{tab:lin_nolst_notaul}). Labor supply is always too high but the optimal tax exceeds the social cost of carbon
\end{itemize}


\input{impo_modeL_FinN}
%\input{simplemodeL_Fisc}
%\section{Theoretical Results}\label{sec:theory}
\begin{itemize}
	\item laissez faire
	\item optimal policy results
\end{itemize}

In this section, I discuss the main theoretical results. 

The laissez-faire economy does violate the emission target. 

\paragraph{Dirty output}

Dirty output grows as
\begin{align*}
	\frac{Y_d'}{Y_d}=\left(\frac{p_d'}{p_d}\right)^{\frac{\alpha-(1-\alpha)(1-\varepsilon)}{1-\alpha}}\frac{A_d'}{A_d}\frac{H'}{H}
\end{align*}
When goods are substitutes, $\varepsilon>1$, the dirty good's price reduces labour supply in this sector. When goods are complements, labour supply in the dirty sector increases as the dirty goods price rises...

growth in the dirty sector falls with inflation in this sector: $\alpha -(1-\alpha)(1-\varepsilon)>0$, then the growth in machines exceeds 
\subsection{Results}
As a result, the percentage change in labour input goods by sectors are equivalent. This together with prices being independent of skill supply implies that the output ratio of sectors is unaffected by tax progressivity.
To see this write:
\begin{align}
	\frac{d\left(\frac{Y_d}{Y_s}\right)}{d \tau_l}=\frac{Y_d}{Y_c}\left(\frac{\frac{dY_d}{Y_d}}{d \tau_l}-\frac{\frac{dY_c}{Y_c}}{d \tau_l}\right)=0
\end{align}
and observe that the percentage change in sector output is homogeneous. 
\begin{align}
	\frac{1}{Y_d}\frac{dY_d}{d \tau_l}= \frac{1}{L_d}\frac{d L_d}{d \tau_l}=\frac{1}{H}\frac{d H}{d \tau_l}\ \text{and} \ \frac{1}{Y_c}\frac{dY_c}{d \tau_l}= \frac{1}{L_c}\frac{d L_c}{d \tau_l}=\frac{1}{H}\frac{d H}{d \tau_l}.
\end{align}






\begin{prop}[Effect of $\tau_l$ on output ratio]
	In the representative agent model with log utility and no disposal of government revenues, tax progressivity does not affect the equilibrium ratio of sector production. Only total output reduces as progressivity rises. \tr{directly obvious from seeing that ratio is constant!}
\end{prop}

\paragraph{Welfare}
%\section{Theoretic Results}
\subsection{An emission target calls for a reduction policy under likely parameter values}
\subsection{Tax progressivity affects the composition of total output}
In the model, tax progressivity affects the innovation decision due to heterogeneous effects on skill supply. 
The optimal ratio of skills supplied by the household is
\begin{align}
\frac{h_{ht}}{h_{lt}}=\left(\frac{w_{ht}}{w_{lt}}\right)^\frac{1-\tau_{lt}}{\tau_{lt}+\sigma}.
\end{align}
The semi-elasticity of the ratio of aggregate skill supply, defined as $\frac{H_h}{H_l}:=\frac{z_hh_h}{z_lh_l}$, in response to a change in tax progressivity is then given by
\begin{align}
\frac{d\log\left(\frac{H_h}{H_l}\right)}{d\tau_l}=-\frac{1+\sigma}{(\tau_l+\sigma)^2}\log\left(\frac{w_{h}}{w_l}\right). \end{align}
The direct effect, with fixed prices is negative. 
The term negative given a positive wage premium for high skill labour. Hence, a higher tax progressivity implies a decline in the relative supply of high skill labour. 

\paragraph{Effect on the externality }

\subsection{Growth in the dirty sector has to stop}
otherwise, price increases to infinity and labour input falls.<- potentially relevant stuff
\section{Calibration}\label{sec:calib}
%\input{modeL_Fisc}
%\section{Quantitative results}\label{sec:simul}

\begin{comment}
Interesting quantitative results

\begin{itemize}
	\item comparison laissez faire in 2050 to optimal policy (net-zero emissions starting in 2050)
	\item present ratios and variables that are constant 
	\item How?: 
	\begin{enumerate}
	\item calculate values of endogenous and predetermined variables starting from today
	\item apply growth rates in laissez-faire and in optimal SS (this is static)\ar simulate the economy/ should be there already
	\item dynamics: shooting algorithm! / relaxation (pertubation (= approximation) to get starting values)
	\end{enumerate}
\end{itemize}
\end{comment}

\begin{comment}
\paragraph{The effect of a higher disutility of high skill labour}

With a higher disutility in high skill labour, the share of the dirty good in production rises. 

\begin{figure}[h!!]
\includegraphics[width=1\textwidth]{../codding_model/Own/figures/Rep_agent/Yd_Yc_ratio_periods10_eppsilon0.40_zeta1.40_Ad08_Ac04_thetac0.70_thetad0.56_fullDisp0_HetGrowth1_tauul0.181_util0.png}
\caption{Effect of the scarcity of labour on the output ratio}
\end{figure}
\end{comment}

In this section, I compare the evolution of the economy under the laissez-faire calibration to the evolution under the optimal policy. First, I show results without the emission target. Second, results with emission target are discussed.

The comparison of the optimal to the laissez-faire allocation in the scenario without emission target shows that there are no motives for the government to intervene. The optimal policy is to set as a flat tax and to generate no revenues. 

\begin{figure}[h!!]
	\centering
	\caption{Optimal Policy }\label{fig:optpol}
	\begin{minipage}[]{0.32\textwidth}
		\centering{\footnotesize{(a) Without target }}
		%	\captionsetup{width=.45\linewidth}
		\includegraphics[width=1\textwidth]{../codding_model/Own/figures/Rep_agent/staticRam_LF_separate_tauul_periods59_eppsilon4.00_zeta1.40_Ad08_Ac04_thetac0.70_thetad0.56_HetGrowth1_tauul0.181_util0_withtarget0_lgd0.png}
	\end{minipage}
	\begin{minipage}[]{0.32\textwidth}
	\centering{\footnotesize{(b) With target $\varepsilon>1$ }}
	%	\captionsetup{width=.45\linewidth}
	\includegraphics[width=1\textwidth]{../codding_model/Own/figures/Rep_agent/staticRam_LF_separate_tauul_periods59_eppsilon4.00_zeta1.40_Ad08_Ac04_thetac0.70_thetad0.56_HetGrowth1_tauul0.181_util0_withtarget1_lgd0.png}
\end{minipage}
\begin{minipage}[]{0.32\textwidth}
	\centering{\footnotesize{(c) With target $\varepsilon<1$ }}
	%	\captionsetup{width=.45\linewidth}
	\includegraphics[width=1\textwidth]{../codding_model/Own/figures/Rep_agent/staticRam_LF_separate_tauul_periods59_eppsilon0.40_zeta1.40_Ad08_Ac04_thetac0.70_thetad0.56_HetGrowth1_tauul0.181_util0_withtarget1_lgd0.png}
\end{minipage}
\end{figure}

\begin{figure}[h!!]
	\centering
	\caption{Optimal versus laissez-faire allocation: No emission target, substitutes }\label{fig:optallo_subs}
	\begin{minipage}[]{0.32\textwidth}
		\centering{\footnotesize{(a) Consumption }}
		%	\captionsetup{width=.45\linewidth}
		\includegraphics[width=1\textwidth]{../codding_model/Own/figures/Rep_agent/staticRam_LF_separate_c_periods59_eppsilon4.00_zeta1.40_Ad08_Ac04_thetac0.70_thetad0.56_HetGrowth1_tauul0.181_util0_withtarget0_lgd1.png}
	\end{minipage}
	\begin{minipage}[]{0.32\textwidth}
		\centering{\footnotesize{(b) High skill supply }}
		%	\captionsetup{width=.45\linewidth}
		\includegraphics[width=1\textwidth]{../codding_model/Own/figures/Rep_agent/staticRam_LF_separate_hh_periods59_eppsilon4.00_zeta1.40_Ad08_Ac04_thetac0.70_thetad0.56_HetGrowth1_tauul0.181_util0_withtarget0_lgd0.png}
	\end{minipage}
	\begin{minipage}[]{0.32\textwidth}
		\centering{\footnotesize{(c) Low skill supply}}
		%	\captionsetup{width=.45\linewidth}
		\includegraphics[width=1\textwidth]{../codding_model/Own/figures/Rep_agent/staticRam_LF_separate_hl_periods59_eppsilon4.00_zeta1.40_Ad08_Ac04_thetac0.70_thetad0.56_HetGrowth1_tauul0.181_util0_withtarget0_lgd0.png}
	\end{minipage}
\begin{minipage}[]{0.32\textwidth}
\centering{\footnotesize{(d) clean output }}
%	\captionsetup{width=.45\linewidth}
\includegraphics[width=1\textwidth]{../codding_model/Own/figures/Rep_agent/staticRam_LF_separate_yc_periods59_eppsilon4.00_zeta1.40_Ad08_Ac04_thetac0.70_thetad0.56_HetGrowth1_tauul0.181_util0_withtarget0_lgd0.png}
\end{minipage}
\begin{minipage}[]{0.32\textwidth}
\centering{\footnotesize{(e) dirty output }}
%	\captionsetup{width=.45\linewidth}
\includegraphics[width=1\textwidth]{../codding_model/Own/figures/Rep_agent/staticRam_LF_separate_yd_periods59_eppsilon4.00_zeta1.40_Ad08_Ac04_thetac0.70_thetad0.56_HetGrowth1_tauul0.181_util0_withtarget0_lgd0.png}
\end{minipage}
\begin{minipage}[]{0.32\textwidth}
\centering{\footnotesize{(f) machines dirty}}
%	\captionsetup{width=.45\linewidth}
\includegraphics[width=1\textwidth]{../codding_model/Own/figures/Rep_agent/staticRam_LF_separate_xd_periods59_eppsilon4.00_zeta1.40_Ad08_Ac04_thetac0.70_thetad0.56_HetGrowth1_tauul0.181_util0_withtarget0_lgd0.png}
\end{minipage}
\begin{minipage}[]{0.32\textwidth}
	\centering{\footnotesize{(f) machines clean}}
	%	\captionsetup{width=.45\linewidth}
	\includegraphics[width=1\textwidth]{../codding_model/Own/figures/Rep_agent/staticRam_LF_separate_xc_periods59_eppsilon4.00_zeta1.40_Ad08_Ac04_thetac0.70_thetad0.56_HetGrowth1_tauul0.181_util0_withtarget0_lgd0.png}
\end{minipage}
\begin{minipage}[]{0.32\textwidth}
	\centering{\footnotesize{(g) Output ratio $y_d/y_c$}}
	%	\captionsetup{width=.45\linewidth}
	\includegraphics[width=1\textwidth]{../codding_model/Own/figures/Rep_agent/staticRam_LF_separate_ydyc_periods59_eppsilon4.00_zeta1.40_Ad08_Ac04_thetac0.70_thetad0.56_HetGrowth1_tauul0.181_util0_withtarget0_lgd0.png}
\end{minipage}
\end{figure}

\begin{figure}[h!!]
	\centering
	\caption{Optimal versus laissez-faire allocation: No emission target, complements SOMETHING WRONG? LOWER WELFARE}\label{fig:optallo_comp}
	\begin{minipage}[]{0.32\textwidth}
		\centering{\footnotesize{(a) Consumption }}
		%	\captionsetup{width=.45\linewidth}
		\includegraphics[width=1\textwidth]{../codding_model/Own/figures/Rep_agent/staticRam_LF_separate_c_periods59_eppsilon0.40_zeta1.40_Ad08_Ac04_thetac0.70_thetad0.56_HetGrowth1_tauul0.181_util0_withtarget0_lgd1.png}
	\end{minipage}
	\begin{minipage}[]{0.32\textwidth}
		\centering{\footnotesize{(b) High skill supply }}
		%	\captionsetup{width=.45\linewidth}
		\includegraphics[width=1\textwidth]{../codding_model/Own/figures/Rep_agent/staticRam_LF_separate_hh_periods59_eppsilon0.40_zeta1.40_Ad08_Ac04_thetac0.70_thetad0.56_HetGrowth1_tauul0.181_util0_withtarget0_lgd0.png}
	\end{minipage}
	\begin{minipage}[]{0.32\textwidth}
		\centering{\footnotesize{(c) Low skill supply}}
		%	\captionsetup{width=.45\linewidth}
		\includegraphics[width=1\textwidth]{../codding_model/Own/figures/Rep_agent/staticRam_LF_separate_hl_periods59_eppsilon0.40_zeta1.40_Ad08_Ac04_thetac0.70_thetad0.56_HetGrowth1_tauul0.181_util0_withtarget0_lgd0.png}
	\end{minipage}
	\begin{minipage}[]{0.32\textwidth}
		\centering{\footnotesize{(d) clean output }}
		%	\captionsetup{width=.45\linewidth}
		\includegraphics[width=1\textwidth]{../codding_model/Own/figures/Rep_agent/staticRam_LF_separate_yc_periods59_eppsilon0.40_zeta1.40_Ad08_Ac04_thetac0.70_thetad0.56_HetGrowth1_tauul0.181_util0_withtarget0_lgd0.png}
	\end{minipage}
	\begin{minipage}[]{0.32\textwidth}
		\centering{\footnotesize{(e) dirty output }}
		%	\captionsetup{width=.45\linewidth}
		\includegraphics[width=1\textwidth]{../codding_model/Own/figures/Rep_agent/staticRam_LF_separate_yd_periods59_eppsilon0.40_zeta1.40_Ad08_Ac04_thetac0.70_thetad0.56_HetGrowth1_tauul0.181_util0_withtarget0_lgd0.png}
	\end{minipage}
	\begin{minipage}[]{0.32\textwidth}
		\centering{\footnotesize{(f) machines dirty}}
		%	\captionsetup{width=.45\linewidth}
		\includegraphics[width=1\textwidth]{../codding_model/Own/figures/Rep_agent/staticRam_LF_separate_xd_periods59_eppsilon0.40_zeta1.40_Ad08_Ac04_thetac0.70_thetad0.56_HetGrowth1_tauul0.181_util0_withtarget0_lgd0.png}
	\end{minipage}
	\begin{minipage}[]{0.32\textwidth}
		\centering{\footnotesize{(f) machines clean}}
		%	\captionsetup{width=.45\linewidth}
		\includegraphics[width=1\textwidth]{../codding_model/Own/figures/Rep_agent/staticRam_LF_separate_xc_periods59_eppsilon0.40_zeta1.40_Ad08_Ac04_thetac0.70_thetad0.56_HetGrowth1_tauul0.181_util0_withtarget0_lgd0.png}
	\end{minipage}
\begin{minipage}[]{0.32\textwidth}
\centering{\footnotesize{(g) Output ratio $y_d/y_c$}}
%	\captionsetup{width=.45\linewidth}
\includegraphics[width=1\textwidth]{../codding_model/Own/figures/Rep_agent/staticRam_LF_separate_ydyc_periods59_eppsilon0.40_zeta1.40_Ad08_Ac04_thetac0.70_thetad0.56_HetGrowth1_tauul0.181_util0_withtarget0_lgd0.png}
\end{minipage}
\begin{minipage}[]{0.32\textwidth}
	\centering{\footnotesize{(g) Welfare}}
	%	\captionsetup{width=.45\linewidth}
	\includegraphics[width=1\textwidth]{../codding_model/Own/figures/Rep_agent/staticRam_LF_separate_welfare_periods59_eppsilon0.40_zeta1.40_Ad08_Ac04_thetac0.70_thetad0.56_HetGrowth1_tauul0.181_util0_withtarget0_lgd0.png}
\end{minipage}
\end{figure}

\begin{figure}[h!!]
	\centering
	\caption{Optimal versus laissez-faire allocation: With emission target, complements }\label{fig:optallo_comp_target}
	\begin{minipage}[]{0.32\textwidth}
		\centering{\footnotesize{(a) Consumption }}
		%	\captionsetup{width=.45\linewidth}
		\includegraphics[width=1\textwidth]{../codding_model/Own/figures/Rep_agent/staticRam_LF_separate_c_periods59_eppsilon0.40_zeta1.40_Ad08_Ac04_thetac0.70_thetad0.56_HetGrowth1_tauul0.181_util0_withtarget1_lgd1.png}
	\end{minipage}
	\begin{minipage}[]{0.32\textwidth}
		\centering{\footnotesize{(b) High skill supply }}
		%	\captionsetup{width=.45\linewidth}
		\includegraphics[width=1\textwidth]{../codding_model/Own/figures/Rep_agent/staticRam_LF_separate_hh_periods59_eppsilon0.40_zeta1.40_Ad08_Ac04_thetac0.70_thetad0.56_HetGrowth1_tauul0.181_util0_withtarget1_lgd0.png}
	\end{minipage}
	\begin{minipage}[]{0.32\textwidth}
		\centering{\footnotesize{(c) Low skill supply}}
		%	\captionsetup{width=.45\linewidth}
		\includegraphics[width=1\textwidth]{../codding_model/Own/figures/Rep_agent/staticRam_LF_separate_hl_periods59_eppsilon0.40_zeta1.40_Ad08_Ac04_thetac0.70_thetad0.56_HetGrowth1_tauul0.181_util0_withtarget1_lgd0.png}
	\end{minipage}
	\begin{minipage}[]{0.32\textwidth}
		\centering{\footnotesize{(d) clean output }}
		%	\captionsetup{width=.45\linewidth}
		\includegraphics[width=1\textwidth]{../codding_model/Own/figures/Rep_agent/staticRam_LF_separate_yc_periods59_eppsilon0.40_zeta1.40_Ad08_Ac04_thetac0.70_thetad0.56_HetGrowth1_tauul0.181_util0_withtarget1_lgd0.png}
	\end{minipage}
	\begin{minipage}[]{0.32\textwidth}
		\centering{\footnotesize{(e) dirty output }}
		%	\captionsetup{width=.45\linewidth}
		\includegraphics[width=1\textwidth]{../codding_model/Own/figures/Rep_agent/staticRam_LF_separate_yd_periods59_eppsilon0.40_zeta1.40_Ad08_Ac04_thetac0.70_thetad0.56_HetGrowth1_tauul0.181_util0_withtarget1_lgd0.png}
	\end{minipage}
	\begin{minipage}[]{0.32\textwidth}
		\centering{\footnotesize{(f) machines dirty}}
		%	\captionsetup{width=.45\linewidth}
		\includegraphics[width=1\textwidth]{../codding_model/Own/figures/Rep_agent/staticRam_LF_separate_xd_periods59_eppsilon0.40_zeta1.40_Ad08_Ac04_thetac0.70_thetad0.56_HetGrowth1_tauul0.181_util0_withtarget1_lgd0.png}
	\end{minipage}
	\begin{minipage}[]{0.32\textwidth}
		\centering{\footnotesize{(f) machines clean}}
		%	\captionsetup{width=.45\linewidth}
		\includegraphics[width=1\textwidth]{../codding_model/Own/figures/Rep_agent/staticRam_LF_separate_xc_periods59_eppsilon0.40_zeta1.40_Ad08_Ac04_thetac0.70_thetad0.56_HetGrowth1_tauul0.181_util0_withtarget1_lgd0.png}
	\end{minipage}
	\begin{minipage}[]{0.32\textwidth}
		\centering{\footnotesize{(g) Output ratio $y_d/y_c$}}
		%	\captionsetup{width=.45\linewidth}
		\includegraphics[width=1\textwidth]{../codding_model/Own/figures/Rep_agent/staticRam_LF_separate_ydyc_periods59_eppsilon0.40_zeta1.40_Ad08_Ac04_thetac0.70_thetad0.56_HetGrowth1_tauul0.181_util0_withtarget1_lgd0.png}
	\end{minipage}
\end{figure}


% only ramsey
\begin{figure}[h!!]
	\centering
	\caption{Optimal allocation: No emission target, complements }\label{fig:optallo_comp_onlyR}
	\begin{minipage}[]{0.32\textwidth}
		\centering{\footnotesize{(a) Consumption }}
		%	\captionsetup{width=.45\linewidth}
		\includegraphics[width=1\textwidth]{../codding_model/Own/figures/Rep_agent/staticonlyRam_separate_c_periods59_eppsilon0.40_zeta1.40_Ad08_Ac04_thetac0.70_thetad0.56_HetGrowth1_tauul0.181_util0_withtarget0_lgd1.png}
	\end{minipage}
	\begin{minipage}[]{0.32\textwidth}
		\centering{\footnotesize{(b) High skill supply }}
		%	\captionsetup{width=.45\linewidth}
		\includegraphics[width=1\textwidth]{../codding_model/Own/figures/Rep_agent/staticonlyRam_separate_hh_periods59_eppsilon0.40_zeta1.40_Ad08_Ac04_thetac0.70_thetad0.56_HetGrowth1_tauul0.181_util0_withtarget0_lgd0.png}
	\end{minipage}
	\begin{minipage}[]{0.32\textwidth}
		\centering{\footnotesize{(c) Low skill supply}}
		%	\captionsetup{width=.45\linewidth}
		\includegraphics[width=1\textwidth]{../codding_model/Own/figures/Rep_agent/staticonlyRam_separate_hl_periods59_eppsilon0.40_zeta1.40_Ad08_Ac04_thetac0.70_thetad0.56_HetGrowth1_tauul0.181_util0_withtarget0_lgd0.png}
	\end{minipage}
	\begin{minipage}[]{0.32\textwidth}
		\centering{\footnotesize{(d) clean output }}
		%	\captionsetup{width=.45\linewidth}
		\includegraphics[width=1\textwidth]{../codding_model/Own/figures/Rep_agent/staticonlyRam_separate_yc_periods59_eppsilon0.40_zeta1.40_Ad08_Ac04_thetac0.70_thetad0.56_HetGrowth1_tauul0.181_util0_withtarget0_lgd0.png}
	\end{minipage}
	\begin{minipage}[]{0.32\textwidth}
		\centering{\footnotesize{(e) dirty output }}
		%	\captionsetup{width=.45\linewidth}
		\includegraphics[width=1\textwidth]{../codding_model/Own/figures/Rep_agent/staticonlyRam_separate_yd_periods59_eppsilon0.40_zeta1.40_Ad08_Ac04_thetac0.70_thetad0.56_HetGrowth1_tauul0.181_util0_withtarget0_lgd0.png}
	\end{minipage}
	\begin{minipage}[]{0.32\textwidth}
		\centering{\footnotesize{(f) machines dirty}}
		%	\captionsetup{width=.45\linewidth}
		\includegraphics[width=1\textwidth]{../codding_model/Own/figures/Rep_agent/staticonlyRam_separate_xd_periods59_eppsilon0.40_zeta1.40_Ad08_Ac04_thetac0.70_thetad0.56_HetGrowth1_tauul0.181_util0_withtarget0_lgd0.png}
	\end{minipage}
	\begin{minipage}[]{0.32\textwidth}
		\centering{\footnotesize{(f) machines clean}}
		%	\captionsetup{width=.45\linewidth}
		\includegraphics[width=1\textwidth]{../codding_model/Own/figures/Rep_agent/staticonlyRam_separate_xc_periods59_eppsilon0.40_zeta1.40_Ad08_Ac04_thetac0.70_thetad0.56_HetGrowth1_tauul0.181_util0_withtarget0_lgd0.png}
	\end{minipage}
	\begin{minipage}[]{0.32\textwidth}
		\centering{\footnotesize{(g) Output ratio $y_d/y_c$}}
		%	\captionsetup{width=.45\linewidth}
		\includegraphics[width=1\textwidth]{../codding_model/Own/figures/Rep_agent/staticonlyRam_separate_ydyc_periods59_eppsilon0.40_zeta1.40_Ad08_Ac04_thetac0.70_thetad0.56_HetGrowth1_tauul0.181_util0_withtarget0_lgd0.png}
	\end{minipage}
\end{figure}

\begin{figure}[h!!]
	\centering
	\caption{Optimal allocation: With emission target, complements }\label{fig:optallo_comp_onlyR_target}
	\begin{minipage}[]{0.32\textwidth}
		\centering{\footnotesize{(a) Consumption }}
		%	\captionsetup{width=.45\linewidth}
		\includegraphics[width=1\textwidth]{../codding_model/Own/figures/Rep_agent/staticonlyRam_separate_c_periods59_eppsilon0.40_zeta1.40_Ad08_Ac04_thetac0.70_thetad0.56_HetGrowth1_tauul0.181_util0_withtarget1_lgd1.png}
	\end{minipage}
	\begin{minipage}[]{0.32\textwidth}
		\centering{\footnotesize{(b) High skill supply }}
		%	\captionsetup{width=.45\linewidth}
		\includegraphics[width=1\textwidth]{../codding_model/Own/figures/Rep_agent/staticonlyRam_separate_hh_periods59_eppsilon0.40_zeta1.40_Ad08_Ac04_thetac0.70_thetad0.56_HetGrowth1_tauul0.181_util0_withtarget1_lgd0.png}
	\end{minipage}
	\begin{minipage}[]{0.32\textwidth}
		\centering{\footnotesize{(c) Low skill supply}}
		%	\captionsetup{width=.45\linewidth}
		\includegraphics[width=1\textwidth]{../codding_model/Own/figures/Rep_agent/staticonlyRam_separate_hl_periods59_eppsilon0.40_zeta1.40_Ad08_Ac04_thetac0.70_thetad0.56_HetGrowth1_tauul0.181_util0_withtarget1_lgd0.png}
	\end{minipage}
	\begin{minipage}[]{0.32\textwidth}
		\centering{\footnotesize{(d) clean output }}
		%	\captionsetup{width=.45\linewidth}
		\includegraphics[width=1\textwidth]{../codding_model/Own/figures/Rep_agent/staticonlyRam_separate_yc_periods59_eppsilon0.40_zeta1.40_Ad08_Ac04_thetac0.70_thetad0.56_HetGrowth1_tauul0.181_util0_withtarget1_lgd0.png}
	\end{minipage}
	\begin{minipage}[]{0.32\textwidth}
		\centering{\footnotesize{(e) dirty output }}
		%	\captionsetup{width=.45\linewidth}
		\includegraphics[width=1\textwidth]{../codding_model/Own/figures/Rep_agent/staticonlyRam_separate_yd_periods59_eppsilon0.40_zeta1.40_Ad08_Ac04_thetac0.70_thetad0.56_HetGrowth1_tauul0.181_util0_withtarget1_lgd0.png}
	\end{minipage}
	\begin{minipage}[]{0.32\textwidth}
		\centering{\footnotesize{(f) machines dirty}}
		%	\captionsetup{width=.45\linewidth}
		\includegraphics[width=1\textwidth]{../codding_model/Own/figures/Rep_agent/staticonlyRam_separate_xd_periods59_eppsilon0.40_zeta1.40_Ad08_Ac04_thetac0.70_thetad0.56_HetGrowth1_tauul0.181_util0_withtarget1_lgd0.png}
	\end{minipage}
	\begin{minipage}[]{0.32\textwidth}
		\centering{\footnotesize{(f) machines clean}}
		%	\captionsetup{width=.45\linewidth}
		\includegraphics[width=1\textwidth]{../codding_model/Own/figures/Rep_agent/staticonlyRam_separate_xc_periods59_eppsilon0.40_zeta1.40_Ad08_Ac04_thetac0.70_thetad0.56_HetGrowth1_tauul0.181_util0_withtarget1_lgd0.png}
	\end{minipage}
	\begin{minipage}[]{0.32\textwidth}
		\centering{\footnotesize{(g) Output ratio $y_d/y_c$}}
		%	\captionsetup{width=.45\linewidth}
		\includegraphics[width=1\textwidth]{../codding_model/Own/figures/Rep_agent/staticonlyRam_separate_ydyc_periods59_eppsilon0.40_zeta1.40_Ad08_Ac04_thetac0.70_thetad0.56_HetGrowth1_tauul0.181_util0_withtarget1_lgd0.png}
	\end{minipage}
\end{figure}

\begin{figure}[h!!]
	\centering
	\caption{Optimal allocation: With emission target, substitutes }\label{fig:optallo_subst_onlyR_target}
	\begin{minipage}[]{0.32\textwidth}
		\centering{\footnotesize{(a) Consumption }}
		%	\captionsetup{width=.45\linewidth}
		\includegraphics[width=1\textwidth]{../codding_model/Own/figures/Rep_agent/staticonlyRam_separate_c_periods59_eppsilon4.00_zeta1.40_Ad08_Ac04_thetac0.70_thetad0.56_HetGrowth1_tauul0.181_util0_withtarget1_lgd1.png}
	\end{minipage}
	\begin{minipage}[]{0.32\textwidth}
		\centering{\footnotesize{(b) High skill supply }}
		%	\captionsetup{width=.45\linewidth}
		\includegraphics[width=1\textwidth]{../codding_model/Own/figures/Rep_agent/staticonlyRam_separate_hh_periods59_eppsilon4.00_zeta1.40_Ad08_Ac04_thetac0.70_thetad0.56_HetGrowth1_tauul0.181_util0_withtarget1_lgd0.png}
	\end{minipage}
	\begin{minipage}[]{0.32\textwidth}
		\centering{\footnotesize{(c) Low skill supply}}
		%	\captionsetup{width=.45\linewidth}
		\includegraphics[width=1\textwidth]{../codding_model/Own/figures/Rep_agent/staticonlyRam_separate_hl_periods59_eppsilon4.00_zeta1.40_Ad08_Ac04_thetac0.70_thetad0.56_HetGrowth1_tauul0.181_util0_withtarget1_lgd0.png}
	\end{minipage}
	\begin{minipage}[]{0.32\textwidth}
		\centering{\footnotesize{(d) clean output }}
		%	\captionsetup{width=.45\linewidth}
		\includegraphics[width=1\textwidth]{../codding_model/Own/figures/Rep_agent/staticonlyRam_separate_yc_periods59_eppsilon4.00_zeta1.40_Ad08_Ac04_thetac0.70_thetad0.56_HetGrowth1_tauul0.181_util0_withtarget1_lgd0.png}
	\end{minipage}
	\begin{minipage}[]{0.32\textwidth}
		\centering{\footnotesize{(e) dirty output }}
		%	\captionsetup{width=.45\linewidth}
		\includegraphics[width=1\textwidth]{../codding_model/Own/figures/Rep_agent/staticonlyRam_separate_yd_periods59_eppsilon4.00_zeta1.40_Ad08_Ac04_thetac0.70_thetad0.56_HetGrowth1_tauul0.181_util0_withtarget1_lgd0.png}
	\end{minipage}
	\begin{minipage}[]{0.32\textwidth}
		\centering{\footnotesize{(f) machines dirty}}
		%	\captionsetup{width=.45\linewidth}
		\includegraphics[width=1\textwidth]{../codding_model/Own/figures/Rep_agent/staticonlyRam_separate_xd_periods59_eppsilon4.00_zeta1.40_Ad08_Ac04_thetac0.70_thetad0.56_HetGrowth1_tauul0.181_util0_withtarget1_lgd0.png}
	\end{minipage}
	\begin{minipage}[]{0.32\textwidth}
		\centering{\footnotesize{(f) machines clean}}
		%	\captionsetup{width=.45\linewidth}
		\includegraphics[width=1\textwidth]{../codding_model/Own/figures/Rep_agent/staticonlyRam_separate_xc_periods59_eppsilon4.00_zeta1.40_Ad08_Ac04_thetac0.70_thetad0.56_HetGrowth1_tauul0.181_util0_withtarget1_lgd0.png}
	\end{minipage}
	\begin{minipage}[]{0.32\textwidth}
		\centering{\footnotesize{(g) Output ratio $y_d/y_c$}}
		%	\captionsetup{width=.45\linewidth}
		\includegraphics[width=1\textwidth]{../codding_model/Own/figures/Rep_agent/staticonlyRam_separate_ydyc_periods59_eppsilon4.00_zeta1.40_Ad08_Ac04_thetac0.70_thetad0.56_HetGrowth1_tauul0.181_util0_withtarget1_lgd0.png}
	\end{minipage}
\end{figure}
%\section{Quantitative Results}

In this section, I, first, discuss the quantitative results.
In the main 

\subsection{Main results: Optimal policy to reach emission targets}
\begin{figure}[h!!]
	\centering
	\caption{Optimal Policy }\label{fig:optPol}
	\begin{minipage}[]{0.4\textwidth}
		\centering{\footnotesize{(a) Income tax progressivity, $\tau_{lt}$}}
		%	\captionsetup{width=.45\linewidth}
		\includegraphics[width=1\textwidth]{../../codding_model/own_basedOnFried/optimalPol_elastS_DisuSci/figures/all_1705/Single_OPT_T_NoTaus_taul_spillover0_sep1_etaa1.00.png}
	\end{minipage}
\begin{minipage}[]{0.1\textwidth}
\
\end{minipage}
	\begin{minipage}[]{0.4\textwidth}
		\centering{\footnotesize{(b) Fossil tax, $\tau_{ft}$ }}
		%	\captionsetup{width=.45\linewidth}
		\includegraphics[width=1\textwidth]{../../codding_model/own_basedOnFried/optimalPol_elastS_DisuSci/figures/all_1705/Single_OPT_T_NoTaus_tauf_spillover0_sep1_etaa1.00.png}
	\end{minipage}
\end{figure} 
\tr{\textbf{Is the timing correct?}}
To optimally meet the IPCCs suggested emission target, the optimal income tax is progressive. As the emission target is less strict, between 2030 to 2045, optimal income tax progressivity is around $\tau_{lt}=0.01$. As the emission target jumps to net-zero emissions, optimal tax progressivity accelerates by an order of magnitude. Over time, tax progressivity increases.  

The optimal fossil tax displays a similar step pattern as the tax progressivity. Between 2030
\subsection{Main results: Optimal allocation}
\begin{figure}[h!!]
	\centering
	\caption{Optimal Policy }\label{fig:optAll}
	\begin{minipage}[]{0.32\textwidth}
		\centering{\footnotesize{(a) Growth fossil sector}}
		%	\captionsetup{width=.45\linewidth}
		\includegraphics[width=1\textwidth]{../../codding_model/own_basedOnFried/optimalPol_elastS_DisuSci/figures/all_1705/Single_OPT_T_NoTaus_Af_spillover0_sep1_etaa1.00.png}
	\end{minipage}
	\begin{minipage}[]{0.32\textwidth}
		\centering{\footnotesize{(b) Growth green sector }}
		%	\captionsetup{width=.45\linewidth}
		\includegraphics[width=1\textwidth]{../../codding_model/own_basedOnFried/optimalPol_elastS_DisuSci/figures/all_1705/Single_OPT_T_NoTaus_Ag_spillover0_sep1_etaa1.00.png}
	\end{minipage}
\begin{minipage}[]{0.32\textwidth}
	\centering{\footnotesize{(c) Growth neutral sector}}
	%	\captionsetup{width=.45\linewidth}
	\includegraphics[width=1\textwidth]{../../codding_model/own_basedOnFried/optimalPol_elastS_DisuSci/figures/all_1705/Single_OPT_T_NoTaus_An_spillover0_sep1_etaa1.00.png}
\end{minipage}
	\begin{minipage}[]{0.32\textwidth}
	\centering{\footnotesize{(d) Labour fossil sector}}
	%	\captionsetup{width=.45\linewidth}
	\includegraphics[width=1\textwidth]{../../codding_model/own_basedOnFried/optimalPol_elastS_DisuSci/figures/all_1705/Single_OPT_T_NoTaus_Lf_spillover0_sep1_etaa1.00.png}
\end{minipage}
\begin{minipage}[]{0.32\textwidth}
	\centering{\footnotesize{(e) Low-skilled labour }}
	%	\captionsetup{width=.45\linewidth}
	\includegraphics[width=1\textwidth]{../../codding_model/own_basedOnFried/optimalPol_elastS_DisuSci/figures/all_1705/Single_OPT_T_NoTaus_hl_spillover0_sep1_etaa1.00.png}
\end{minipage}
\begin{minipage}[]{0.32\textwidth}
	\centering{\footnotesize{(f) High-skilled labour}}
	%	\captionsetup{width=.45\linewidth}
	\includegraphics[width=1\textwidth]{../../codding_model/own_basedOnFried/optimalPol_elastS_DisuSci/figures/all_1705/Single_OPT_T_NoTaus_hh_spillover0_sep1_etaa1.00.png}
\end{minipage}
	\begin{minipage}[]{0.32\textwidth}
	\centering{\footnotesize{(d) Consumption}}
	%	\captionsetup{width=.45\linewidth}
	\includegraphics[width=1\textwidth]{../../codding_model/own_basedOnFried/optimalPol_elastS_DisuSci/figures/all_1705/Single_OPT_T_NoTaus_C_spillover0_sep1_etaa1.00.png}
\end{minipage}
\begin{minipage}[]{0.32\textwidth}
	\centering{\footnotesize{(e) Social Welfare}}
	%	\captionsetup{width=.45\linewidth}
	\includegraphics[width=1\textwidth]{../../codding_model/own_basedOnFried/optimalPol_elastS_DisuSci/figures/all_1705/Single_OPT_T_NoTaus_SWF_spillover0_sep1_etaa1.00.png}
\end{minipage}
\begin{minipage}[]{0.32\textwidth}
	\centering{\footnotesize{(f) Emissions}}
	%	\captionsetup{width=.45\linewidth}
	\includegraphics[width=1\textwidth]{../../codding_model/own_basedOnFried/optimalPol_elastS_DisuSci/figures/all_1705/Single_OPT_T_NoTaus_Emnet_spillover0_sep1_etaa1.00.png}
\end{minipage}
\end{figure} 
Under the assumptio
\section{Quantitative results}\label{sec:res}

In this section, I present and discuss the quantitative results. 
First, in Section \ref{subsec:exp}, I use the model to learn (i) how a constant carbon tax affects the economy, (ii) how it interacts with a tax on labor, and (iii) how high a carbon tax is necessary to meet emission limits.
Second, I ask how the government can optimally satisfy the emission limit in Section \ref{subsec:mr} by jointly choosing the progressivity of the income tax scheme and the carbon tax. 

%I focus on analyzing the mechanisms and welfare benefits from integrating the income tax scheme into the environmental policy. I also discuss the costs of not using lump-sum transfers.

\subsection{Results }
This section depicts the environmental tax which is required to meet the emission limits and the evolution of key variables under distinct policy regimes. The income tax scheme is kept fixed at its calibrated levels: $\tau_{\iota t}=0.181$, $\lambda=0.43$.
I consider the following regimes: first, environmental tax revenues are redistributed lump sum to households. Second, the government runs a consolidated budget and environmental tax revenues are redistributed via the income tax scheme; that is, $\lambda$ adjusts. Third, environmental tax revenues are recycled as subsidies to the green sector. Fourth, the government consumes environmental tax revenues. 



\subsection{Optimal Policy}\label{subsec:mr}


This section seeks to answer the question how a benevolent planner optimally attains the emission limit. After showing the results in section \ref{sec:optres}, I discuss the intention behind the optimal policy in section \ref{subsec:dis}. 
%This section depicts results on the optimal policy followed by the implied allocation in the benchmark model where environmental tax revenues are redistributed via the income tax scheme. 

\subsubsection{Results}\label{sec:optres}
To meet the emission limits suggested by the IPCC, the optimal policy is to tax labor until 2044; see panel (a) in figure \ref{fig:optPol}. The labor tax becomes a subsidy from 2045 onward. 
\begin{figure}[h!!]
	\centering
	\caption{Optimal Policy }\label{fig:optPol}
	\begin{subfigure}{0.4\textwidth}
		\caption{Average marginal income tax rate }
		%	\captionsetup{width=.45\linewidth}
		\includegraphics[width=1\textwidth]{../../codding_model/own_basedOnFried/optimalPol_010922_revision/figures/all_13Sept22_Tplus30/dTaulAv_OPT_T_NoTaus_COMPtaul_regime4_spillover0_knspil0_noskill0_sep0_xgrowth0_PV1_etaa0.79_lgd0.png}
	\end{subfigure}
\begin{minipage}[]{0.1\textwidth}
	\
\end{minipage}
	\begin{subfigure}{0.4\textwidth}
		\caption{Tax per ton of carbon in 2022 US\$ }
		%	\captionsetup{width=.45\linewidth}
		\includegraphics[width=1\textwidth]{../../codding_model/own_basedOnFried/optimalPol_010922_revision/figures/all_13Sept22_Tplus30/Single_periods12_OPT_T_NoTaus_Tauf_regime4_spillover0_knspil0_noskill0_sep0_xgrowth0_extern0_PV1_sizeequ0_GOV0_etaa0.79.png}
	\end{subfigure}
\floatfoot{Notes: \footnotesize{The x-axis indicates the first year of the 5 year period to which the variable value corresponds. }}
\end{figure} 
 The optimal carbon tax increases over time and jumps to a higher level when the net-zero emission limit is introduced in 2050.
In 2020, the carbon tax equals US\$987 and rises steadily to US\$1,325 in 2045.  As the emission limit declines to net-zero in 2050, the tax rapidly surges to US\$2,833 and gradually increases afterwards reaching US\$3,2781 in 2075. 
\paragraph{Efficient and optimal allocation}\label{subsec:notaul}

Figure \ref{fig:optAll_percLf_dyn} depicts the adjustments of key variables under the first-best (efficient) and the second-best (optimal) policy relative to the laissez-faire allocation.\footnote{\ I formulate the social planner's problem in appendix section \ref{app:sp_prob}.  Figure \ref{fig:LF} in Appendix \ref{app:quant_res_opt} shows the laissez-faire, the efficient, and the optimal allocation in levels.} 
The efficient allocation can be perceived as the allocation the Ramsey planner seeks to implement. However, she may not be able to achieve the efficient allocation due to the reliance on a limit number of tax instruments.

\begin{figure}[h!!!]
	\centering
	\caption{Efficient and optimal allocation in deviation from laissez-faire	}\label{fig:optAll_percLf_dyn}
	\begin{subfigure}[]{0.4\textwidth}
		\caption{Consumption}
		%	\captionsetup{width=.45\linewidth}
		\includegraphics[width=1\textwidth]{../../codding_model/own_basedOnFried/optimalPol_010922_revision/figures/all_13Sept22_Tplus30/C_PercentageLFDyn_Target_regime4_knspil0_spillover0_noskill0_sep0_xgrowth0_PV1_etaa0.79_lgd1.png}
	\end{subfigure}
	\begin{subfigure}[]{0.4\textwidth}
		\caption{High-skill hours worked}
		%	\captionsetup{width=.45\linewidth}
		\includegraphics[width=1\textwidth]{../../codding_model/own_basedOnFried/optimalPol_010922_revision/figures/all_13Sept22_Tplus30/hh_PercentageLFDyn_Target_regime4_knspil0_spillover0_noskill0_sep0_xgrowth0_PV1_etaa0.79_lgd0.png}
	\end{subfigure}
	\begin{subfigure}[]{0.4\textwidth}
		\caption{Low-skill hours worked}
		%	\captionsetup{width=.45\linewidth}
		\includegraphics[width=1\textwidth]{../../codding_model/own_basedOnFried/optimalPol_010922_revision/figures/all_13Sept22_Tplus30/hl_PercentageLFDyn_Target_regime4_knspil0_spillover0_noskill0_sep0_xgrowth0_PV1_etaa0.79_lgd0.png}
	\end{subfigure}
	\begin{subfigure}[]{0.4\textwidth}
		\caption{Fossil scientists}
		%	\captionsetup{width=.45\linewidth}
		\includegraphics[width=1\textwidth]{../../codding_model/own_basedOnFried/optimalPol_010922_revision/figures/all_13Sept22_Tplus30/sff_PercentageLFDyn_Target_regime4_knspil0_spillover0_noskill0_sep0_xgrowth0_PV1_etaa0.79_lgd0.png}
	\end{subfigure}
%	\begin{subfigure}[]{0.4\textwidth}
%		\caption{Non-energy scientists}
%		%	\captionsetup{width=.45\linewidth}
%		\includegraphics[width=1\textwidth]{../../codding_model/own_basedOnFried/optimalPol_010922_revision/figures/all_13Sept22_Tplus30/sn_PercentageLFDyn_Target_regime4_knspil0_spillover0_noskill0_sep0_xgrowth0_PV1_etaa0.79_lgd0.png}
%	\end{subfigure}
	\begin{subfigure}[]{0.4\textwidth}
		\caption{Green scientists}
		%	\captionsetup{width=.45\linewidth}
		\includegraphics[width=1\textwidth]{../../codding_model/own_basedOnFried/optimalPol_010922_revision/figures/all_13Sept22_Tplus30/sg_PercentageLFDyn_Target_regime4_knspil0_spillover0_noskill0_sep0_xgrowth0_PV1_etaa0.79_lgd0.png}
	\end{subfigure}
	\begin{subfigure}[]{0.4\textwidth}
		\caption{Green-to-fossil output}
		%	\captionsetup{width=.45\linewidth}
		\includegraphics[width=1\textwidth]{../../codding_model/own_basedOnFried/optimalPol_010922_revision/figures/all_13Sept22_Tplus30/GFF_PercentageLFDyn_Target_regime4_knspil0_spillover0_noskill0_sep0_xgrowth0_PV1_etaa0.79_lgd0.png}
	\end{subfigure}
%	\begin{subfigure}[]{0.4\textwidth}
%		\caption{Energy share in GDP}
%		%	\captionsetup{width=.45\linewidth}
%		\includegraphics[width=1\textwidth]{../../codding_model/own_basedOnFried/optimalPol_010922_revision/figures/all_13Sept22_Tplus30/EY_PercentageLFDyn_Target_regime4_knspil0_spillover0_noskill0_sep0_xgrowth0_PV1_etaa0.79_lgd0.png}
%	\end{subfigure}
	\floatfoot{Notes: \footnotesize{ The figure shows the percentage deviation of the allocation resulting under the integrated policy, that is, the progressivity of the income tax can be chosen (the black solid graph), the optimal allocation in the separate regime when no progressive income tax is available (the gray dashed graph), and the efficient allocation (the orange dotted graph), relative to the laissez-faire allocation. 
			Panel (d) shows aggregate growth where the variable value in $t$ refers to the growth rate from  period $t$ to period $t+1$. Hence, from 2045 to 2050 growth reduces significantly, since in 2050 the net emission limit has to be satisfied. Panel (f) shows the level of the carbon tax under the two policy regimes considered.
	}}
\end{figure} 
%
%\clearpage
%%
% Labor supply
The social planner attains the emission limit while increasing consumption and decreasing labor supply; panels (a) to (c) in Figure \ref{fig:optAll_percLf_dyn}. This allocation is achieved by more research in all sectors. 

Under the optimal policy, in contrast, consumption reduces relative to the laissez-faire allocation since the government lacks sufficient instruments to raise research while meeting the emission limit. In the non-energy and the fossil sector, the number of scientists even reduces relative to the laissez-faire allocation.

The social planner can sustain  high growth rates and simultaneously meet the emission limit by choosing a lower energy share to GDP and a higher ratio of green to fossil energy. 
In the competitive economy, a rise in fossil research has to be regulated via demand. Hence, a trade-off between research and emission mitigation occurs, and implementing the emission limit is costly in terms of R\&D investment and growth. 

%%%%%%%%%%%%%%%%%%%%%%%%%%%%%%%%%%%%%%%%%%%%%%%%%%%%%%%%%%%%%%%%%%%%%%%%%%%%%%%%%%%%
%% DISCUSSION 
%%%%%%%%%%%%%%%%%%%%%%%%%%%%%%%%%%%%%%%%%%%%%%%%%%%%%%%%%%%%%%%%%%%%%%%%%%%%%%%%%%%%

\subsubsection{Discussion}\label{subsec:dis}

 What explains the optimal policy? To answer this question, I look at how the optimal allocation with labor income tax differs from the optimal allocation when no income tax is available. 
 The subsequent section looks at a counterfactual experiment where only the optimal carbon tax is implemented. Comparing the resulting allocation to the optimal allocation sheds light on the specific role of the income tax. 


	
\paragraph{Comparison to separate policy regime}

Figure \ref{fig:opt_TLs} shows percentage deviations of the allocation under the policy regime with income tax, henceforth \textit{integrated} regime, relative to a \textit{separate} policy regime where no labor income tax can be chosen; i.e., $\tau_{\iota t}=0$.  
Panel (a) depicts the carbon tax which is smaller under the integrated policy regime up to 2045. Afterwards, the carbon tax is higher than in the separate regime. The higher carbon tax falls together with a subsidy on labor.
Labor income taxes and carbon taxes act as substitutes. 

%The gain of the integrated policy is higher non-energy productivity  during initial periods until 2060  and higher green growth during the net-zero emission periods (panels (c) to (e)).

Utility gains of the use of an income tax occur during the initial periods up to 2035; see panel (b). 
The economy profits from more leisure while technology growth from more research in the fossil and non-energy sector keep consumption high, see panels (c) and (d). 
The costs of the integrated policy is borne in future periods. To achieve the higher technology levels, the planner accepts less green growth and a smaller green-to-fossil ratio today which result in lower utility levels tomorrow. %
\begin{figure}[h!!!]
	\centering
	\caption{Deviation from optimal policy with only a carbon tax}\label{fig:opt_TLs}
	\begin{subfigure}{0.4\textwidth}
		\caption{ Carbon tax}
		%	\captionsetup{width=.45\linewidth}
		\includegraphics[width=1\textwidth]{../../codding_model/own_basedOnFried/optimalPol_010922_revision/figures/all_13Sept22_Tplus30/Tauf_OPT_COMPtaulPer_regime4_spillover0_knspil0_noskill0_sep0_xgrowth0_PV1_etaa0.79.png}
	\end{subfigure}
	\begin{subfigure}{0.4\textwidth}
	\caption{Period utility}
	%	\captionsetup{width=.45\linewidth}
	\includegraphics[width=1\textwidth]{../../codding_model/own_basedOnFried/optimalPol_010922_revision/figures/all_13Sept22_Tplus30/SWF_OPT_COMPtaulPer_regime4_spillover0_knspil0_noskill0_sep0_xgrowth0_PV1_etaa0.79.png}
	\end{subfigure}
	\begin{subfigure}{0.4\textwidth}
		\caption{Fossil scientists}
		%	\captionsetup{width=.45\linewidth}
		\includegraphics[width=1\textwidth]{../../codding_model/own_basedOnFried/optimalPol_010922_revision/figures/all_13Sept22_Tplus30/sff_OPT_COMPtaulPer_regime4_spillover0_knspil0_noskill0_sep0_xgrowth0_PV1_etaa0.79.png}
	\end{subfigure}
	\begin{subfigure}{0.4\textwidth}
		\caption{Non-energy scientists}
		%	\captionsetup{width=.45\linewidth}
		\includegraphics[width=1\textwidth]{../../codding_model/own_basedOnFried/optimalPol_010922_revision/figures/all_13Sept22_Tplus30/sn_OPT_COMPtaulPer_regime4_spillover0_knspil0_noskill0_sep0_xgrowth0_PV1_etaa0.79.png}
	\end{subfigure}
	\begin{subfigure}{0.4\textwidth}
	\caption{Green scientists}
	%	\captionsetup{width=.45\linewidth}
	\includegraphics[width=1\textwidth]{../../codding_model/own_basedOnFried/optimalPol_010922_revision/figures/all_13Sept22_Tplus30/sg_OPT_COMPtaulPer_regime4_spillover0_knspil0_noskill0_sep0_xgrowth0_PV1_etaa0.79.png}
\end{subfigure}
\begin{subfigure}{0.4\textwidth}
\caption{Green-to-fossil output}
%	\captionsetup{width=.45\linewidth}
\includegraphics[width=1\textwidth]{../../codding_model/own_basedOnFried/optimalPol_010922_revision/figures/all_13Sept22_Tplus30/GFF_OPT_COMPtaulPer_regime4_spillover0_knspil0_noskill0_sep0_xgrowth0_PV1_etaa0.79.png}
\end{subfigure}
	\floatfoot{Notes: \footnotesize{ Graphs show the percentage deviations of the variable under the integrated policy regime where the planner can choose income tax progressivity and the separate regime where the income tax scheme is non-distortive, $\tau_{\iota t}=0$. }}
\end{figure} 

\paragraph{Knowledge spillovers}
The driving model feature of this result are knowledge spillovers. Absent knowledge spillovers, the optimal income tax scheme is regressive throughout which boosts labor supply in general and raises the high-to-low skill ratio supplied.\footnote{\ Consider Figure \ref{fig:opt_TLs_noKN} in Appendix \ref{app:TLS} which shows the results in a model without knowledge spillovers.} Green growth is higher, and fossil and non-energy growth decline in all periods relative to the separate policy regime.\footnote{\  The optimal income tax is not regressive due to the compositional effect but rather due to the level effect on economic activity. With homogeneous skills, there is no compositional effect of income tax progressivity. Nevertheless, the optimal policy features regressive income taxes to boost labor supply in combination with higher carbon taxes. This policy allows to redirect research and production towards the green sector while the regressive income tax keeps working hours high. Figure \ref{fig:opt_TLs_noknow_homoskill} in Appendix \ref{app:TLS} shows the results. 
}. This policy achieves a stronger composition towards green production and growth through the higher carbon tax, while the regressive income tax mitigates the reductive effect of the higher carbon tax on consumption. 

Thus, knowledge spillovers render it advantageous to initially boost technology growth in fossil energy and non-energy instead of pushing towards green growth directly.
The gains from knowledge spillovers arise from decreasing returns to research within sectors. A smoother distribution of scientists across sectors makes them more productive. Knowledge spillovers to the green sector from conventional ones then mitigate the costs of the net-zero emission limit. 

In line with this argument, the social planner lowers fossil research relative to the laissez-faire allocation in all periods; compare Figure \ref{fig:optAll_percLf_dyn_nokn}. Instead, non-energy and green research increase more. As a result to decreasing returns to research, the rise in consumption attained under the efficient relative to the laissez-faire allocation is muted. In addition, efficient hours worked increase over time.  Hence, absent knowledge spillovers, the emission limit implies such a reduction in consumption, that the income effect dominates the substitution effect on labor. 

To direct research to the green sector, the Ramsey planner sets a higher carbon tax. However, this reduces the wage rate discouraging work effort. The regressive income tax counters this effect by subsidizing labor. The next section focuses on the effect of the labor income tax. 

\begin{figure}[h!!!]
	\centering
	\caption{Deviation under optimal policy with and without optimal labor income tax}\label{fig:opt_Count}
	\begin{subfigure}{0.4\textwidth}
		\caption{Fossil production}
		%	\captionsetup{width=.45\linewidth}
		\includegraphics[width=1\textwidth]{../../codding_model/own_basedOnFried/optimalPol_010922_revision/figures/all_13Sept22_Tplus30/CountTAUFPerDif_Opt_target_F_nsk0_xgr0_knspil0_regime4_spillover0_sep0_extern0_PV1_etaa0.79.png}
	\end{subfigure}	
	\begin{subfigure}{0.4\textwidth}
		\caption{Period utility}
		%	\captionsetup{width=.45\linewidth}
		\includegraphics[width=1\textwidth]{../../codding_model/own_basedOnFried/optimalPol_010922_revision/figures/all_13Sept22_Tplus30/CountTAUFPerDif_Opt_target_SWF_nsk0_xgr0_knspil0_regime4_spillover0_sep0_extern0_PV1_etaa0.79.png}
	\end{subfigure}	
	\begin{subfigure}{0.4\textwidth}
		\caption{High-skilled hours worked}
		%	\captionsetup{width=.45\linewidth}
		\includegraphics[width=1\textwidth]{../../codding_model/own_basedOnFried/optimalPol_010922_revision/figures/all_13Sept22_Tplus30/CountTAUFPerDif_Opt_target_hh_nsk0_xgr0_knspil0_regime4_spillover0_sep0_extern0_PV1_etaa0.79.png}
	\end{subfigure}	
	\begin{subfigure}{0.4\textwidth}
		\caption{Low-skilled hours worked}
		%	\captionsetup{width=.45\linewidth}
		\includegraphics[width=1\textwidth]{../../codding_model/own_basedOnFried/optimalPol_010922_revision/figures/all_13Sept22_Tplus30/CountTAUFPerDif_Opt_target_hl_nsk0_xgr0_knspil0_regime4_spillover0_sep0_extern0_PV1_etaa0.79.png}
	\end{subfigure}
	\begin{subfigure}{0.4\textwidth}
		\caption{Fossil scientists}
		%	\captionsetup{width=.45\linewidth}
		\includegraphics[width=1\textwidth]{../../codding_model/own_basedOnFried/optimalPol_010922_revision/figures/all_13Sept22_Tplus30/CountTAUFPerDif_Opt_target_sff_nsk0_xgr0_knspil0_regime4_spillover0_sep0_extern0_PV1_etaa0.79.png}
	\end{subfigure}
	\begin{subfigure}{0.4\textwidth}
		\caption{Green scientists}
		%	\captionsetup{width=.45\linewidth}
		\includegraphics[width=1\textwidth]{../../codding_model/own_basedOnFried/optimalPol_010922_revision/figures/all_13Sept22_Tplus30/CountTAUFPerDif_Opt_target_sg_nsk0_xgr0_knspil0_regime4_spillover0_sep0_extern0_PV1_etaa0.79.png}
	\end{subfigure}
	\begin{subfigure}{0.4\textwidth}
		\caption{Non-energy scientists}
		%	\captionsetup{width=.45\linewidth}
		\includegraphics[width=1\textwidth]{../../codding_model/own_basedOnFried/optimalPol_010922_revision/figures/all_13Sept22_Tplus30/CountTAUFPerDif_Opt_target_sn_nsk0_xgr0_knspil0_regime4_spillover0_sep0_extern0_PV1_etaa0.79.png}
	\end{subfigure}
	\begin{subfigure}{0.4\textwidth}
	\caption{Green-to-fossil output}
	%	\captionsetup{width=.45\linewidth}
	\includegraphics[width=1\textwidth]{../../codding_model/own_basedOnFried/optimalPol_010922_revision/figures/all_13Sept22_Tplus30/CountTAUFPerDif_Opt_target_GFF_nsk0_xgr0_knspil0_regime4_spillover0_sep0_extern0_PV1_etaa0.79.png}
\end{subfigure}
\end{figure}
\paragraph{Counterfactual policy: only optimal carbon tax}
%\tr{\ar Contribution of labor tax to allocation}
In contrast to the previous section, this one focuses on the contribution of the labor income tax to the optimal allocation.
Figure \ref{fig:opt_Count} presents percentage deviations under the integrated optimal policy relative to the allocation where only the optimal carbon tax is implemented. The difference can be perceived as the effect of the labor income tax, assuming the carbon tax is implemented first.\footnote{\ Hence, interactions between the carbon and the labor tax are assigned to the effect of the labor tax.}

 The labor income tax contributes to lowering fossil production. It raises fossil production once it becomes a subsidy in 2040 (panel (a)). The tax on labor raises period utility in early years and decreases it during the net-zero emission period (panel (b)). The rise in utility stems from more leisure. In contrast to the overall effect of the integrated policy, the labor income tax raises green growth in initial periods. This follows from a reduction in non-energy research since the energy good becomes more expensive. More scientists are allocated to the energy sector of which a higher number is directed to the fossil sector.  The regressive tax implies a reduction in period utility and a rise in fossil production.
 
 In a nutshell, the labor income tax, first, serves to implement the emission limit by reducing economic activity (the green-to-fossil energy ratio even declines (panel (h))). It becomes necessary to reduce work effort since the smaller carbon tax induces a smaller green-to-fossil energy ratio. In addition, the wage rate reduces less so that the carbon tax implies a smaller reduction in labor supply. Second, the regressive income tax, in contrast, serves to increase labor supply replicating the efficient allocation. Furthermore, non-energy research and the green-to-fossil ratio rise since high-to-low skill labor supply increases. 
 
   
\clearpage
%\section{Quantitative Results}

In this section, I, first, discuss the quantitative results.
Subsection AA presents the optimal allocation and policy given the emission target. Subsection BB discusses the results. In particular, I focus on understanding the role and importance of tax progressivity. 

\subsection{Main results}
\begin{figure}[h!!]
	\centering
	\caption{Optimal Policy }\label{fig:optPol}
	\begin{minipage}[]{0.4\textwidth}
		\centering{\footnotesize{(a) Income tax progressivity, $\tau_{lt}$}}
		%	\captionsetup{width=.45\linewidth}
		\includegraphics[width=1\textwidth]{../../codding_model/own_basedOnFried/optimalPol_elastS_DisuSci/figures/all_1705/Single_OPT_T_NoTaus_taul_spillover0_sep1_BN0_ineq0_etaa0.79.png}
	\end{minipage}
\begin{minipage}[]{0.1\textwidth}
\
\end{minipage}
	\begin{minipage}[]{0.4\textwidth}
		\centering{\footnotesize{(b) Fossil tax, $\tau_{ft}$ }}
		%	\captionsetup{width=.45\linewidth}
		\includegraphics[width=1\textwidth]{../../codding_model/own_basedOnFried/optimalPol_elastS_DisuSci/figures/all_1705/Single_OPT_T_NoTaus_tauf_spillover0_sep1_BN0_ineq0_etaa0.79.png}
	\end{minipage}
\end{figure} 
%\begin{figure}[h!!]
%	\centering
%	\caption{Optimal Policy }\label{fig:optPol}
%	\begin{minipage}[]{0.4\textwidth}
%		\centering{\footnotesize{(a) Income tax progressivity, $\tau_{lt}$}}
%		%	\captionsetup{width=.45\linewidth}
%		\includegraphics[width=1\textwidth]{../../codding_model/own_basedOnFried/optimalPol_elastS_DisuSci/figures/all_1705/Single_OPT_T_NoTaus_taul_spillover0_sep1_BN1_ineq0_etaa0.79.png}
%	\end{minipage}
%	\begin{minipage}[]{0.1\textwidth}
%		\
%	\end{minipage}
%	\begin{minipage}[]{0.4\textwidth}
%		\centering{\footnotesize{(b) Fossil tax, $\tau_{ft}$ }}
%		%	\captionsetup{width=.45\linewidth}
%		\includegraphics[width=1\textwidth]{../../codding_model/own_basedOnFried/optimalPol_elastS_DisuSci/figures/all_1705/Single_OPT_T_NoTaus_tauf_spillover0_sep1_BN1_ineq0_etaa0.79.png}
%	\end{minipage}
%\end{figure}
To optimally meet the IPCCs suggested emission target, the optimal income tax is progressive. As the emission target is less strict, between 2030 to 2045, optimal income tax progressivity is around $\tau_{lt}=0.02$. As the emission target jumps to net-zero emissions in 2050, optimal tax progressivity accelerates to above 0.08 and gradually increases in the subsequent years to around 0.09. This is approximately half the size found for the US in \cite{Heathcote2017OptimalFramework}: $\tau_{l}=0.181$. 
In the period without emission target from 2020 to 2030, the optimal income tax is regressive.

The optimal fossil tax displays a similar step pattern as income tax progressivity. From 2020 to the beginning of 2030, it is negative. It jumps to around 50\% as the emission target is to reduce emissions by 50\% relative to 2019 emissions. As the emission target rises  to net-zero emissions in 2050, the optimal tax on fossil sales is close to 95\%. 

\begin{figure}[h!!]
	\centering
	\caption{Optimal Policy }\label{fig:optAll}
	\begin{minipage}[]{0.32\textwidth}
		\centering{\footnotesize{(a) Growth fossil sector}}
		%	\captionsetup{width=.45\linewidth}
		\includegraphics[width=1\textwidth]{../../codding_model/own_basedOnFried/optimalPol_elastS_DisuSci/figures/all_1705/Single_OPT_T_NoTaus_Af_spillover0_sep1_BN0_ineq0_etaa0.79.png}
	\end{minipage}
	\begin{minipage}[]{0.32\textwidth}
		\centering{\footnotesize{(b) Growth green sector }}
		%	\captionsetup{width=.45\linewidth}
		\includegraphics[width=1\textwidth]{../../codding_model/own_basedOnFried/optimalPol_elastS_DisuSci/figures/all_1705/Single_OPT_T_NoTaus_Ag_spillover0_sep1_BN0_ineq0_etaa0.79.png}
	\end{minipage}
\begin{minipage}[]{0.32\textwidth}
	\centering{\footnotesize{(c) Growth neutral sector}}
	%	\captionsetup{width=.45\linewidth}
	\includegraphics[width=1\textwidth]{../../codding_model/own_basedOnFried/optimalPol_elastS_DisuSci/figures/all_1705/Single_OPT_T_NoTaus_An_spillover0_sep1_BN0_ineq0_etaa0.79.png}
\end{minipage}
	\begin{minipage}[]{0.32\textwidth}
	\centering{\footnotesize{(d) Labour fossil sector}}
	%	\captionsetup{width=.45\linewidth}
	\includegraphics[width=1\textwidth]{../../codding_model/own_basedOnFried/optimalPol_elastS_DisuSci/figures/all_1705/Single_OPT_T_NoTaus_Lf_spillover0_sep1_BN0_ineq0_etaa0.79.png}
\end{minipage}
\begin{minipage}[]{0.32\textwidth}
	\centering{\footnotesize{(e) Low-skilled labour }}
	%	\captionsetup{width=.45\linewidth}
	\includegraphics[width=1\textwidth]{../../codding_model/own_basedOnFried/optimalPol_elastS_DisuSci/figures/all_1705/Single_OPT_T_NoTaus_hl_spillover0_sep1_BN0_ineq0_etaa0.79.png}
\end{minipage}
\begin{minipage}[]{0.32\textwidth}
	\centering{\footnotesize{(f) High-skilled labour}}
	%	\captionsetup{width=.45\linewidth}
	\includegraphics[width=1\textwidth]{../../codding_model/own_basedOnFried/optimalPol_elastS_DisuSci/figures/all_1705/Single_OPT_T_NoTaus_hh_spillover0_sep1_BN0_ineq0_etaa0.79.png}
\end{minipage}
	\begin{minipage}[]{0.32\textwidth}
	\centering{\footnotesize{(d) Consumption}}
	%	\captionsetup{width=.45\linewidth}
	\includegraphics[width=1\textwidth]{../../codding_model/own_basedOnFried/optimalPol_elastS_DisuSci/figures/all_1705/Single_OPT_T_NoTaus_C_spillover0_sep1_BN0_ineq0_etaa0.79.png}
\end{minipage}
\begin{minipage}[]{0.32\textwidth}
	\centering{\footnotesize{(e) Social Welfare}}
	%	\captionsetup{width=.45\linewidth}
	\includegraphics[width=1\textwidth]{../../codding_model/own_basedOnFried/optimalPol_elastS_DisuSci/figures/all_1705/Single_OPT_T_NoTaus_SWF_spillover0_sep1_BN0_ineq0_etaa0.79.png}
\end{minipage}
\begin{minipage}[]{0.32\textwidth}
	\centering{\footnotesize{(f) Emissions}}
	%	\captionsetup{width=.45\linewidth}
	\includegraphics[width=1\textwidth]{../../codding_model/own_basedOnFried/optimalPol_elastS_DisuSci/figures/all_1705/Single_OPT_T_NoTaus_Emnet_spillover0_sep1_BN0_ineq0_etaa0.79.png}
\end{minipage}
\end{figure} 

\subsection{Discussion}
To study the role of income tax progressivity, I compare the optimal policy and allocation in the full model to a  model where no labour income tax is available.

The first thing to note is that a fossil tax suffices to meet the emission target, panel (a) in figure \ref{fig:Compno_taul}. The fossil tax is slightly higher in periods with emission target, compare panel (b).
The advantage from relying on labour income taxes to meet the emission target stems from a higher utility from leisure especially from high-skilled workers which outweighs lower consumption levels, compare panels (c) to (e). In fact, the rise in social welfare arises from the periods with net-zero emission target as shown by panel (f) which compares social welfare levels across policy regimes. 

Income tax progressivity increases social welfare in the net-zero emission world, as households work inefficiently high hours absent an income tax. Higher hours worked result in too high labour effort and research in the green and non-energy sector. The allocation in the fossil sector remains unchanged to meet the emission target; compare panels (g) to (j). 

Although consumption rises due to the higher work effort when there is no income tax, the gains from labour effort are diminished due to the cap on fossil energy. Since green and fossil energy are no perfect substitutes, the economy cannot profit as much from the rise in green energy. HYPOTHESIS: WITH ENERGY SOURCES BEING BETTER COMPLEMENTS, WORK EFFORTS WOULD BE MORE FRUITEFUL. The muted effect of green energy on total energy output is intensified when considering total output where input goods are complements. 

\begin{figure}[h!!]
	\centering
	\caption{Optimal Policy }\label{fig:Compno_taul}
			\begin{minipage}[]{0.32\textwidth}
		\centering{\footnotesize{(a) Emissions}}
		%	\captionsetup{width=.45\linewidth}
		\includegraphics[width=1\textwidth]{../../codding_model/own_basedOnFried/optimalPol_elastS_DisuSci/figures/all_1705/comp_notaul_OPT_T_NoTaus_Emnet_spillover0_sep1_BN0_ineq0_etaa0.79_lgd1.png}
	\end{minipage}
		\begin{minipage}[]{0.32\textwidth}
		\centering{\footnotesize{(b) Fossil tax}}
		%	\captionsetup{width=.45\linewidth}
		\includegraphics[width=1\textwidth]{../../codding_model/own_basedOnFried/optimalPol_elastS_DisuSci/figures/all_1705/comp_notaul_OPT_T_NoTaus_tauf_spillover0_sep1_BN0_ineq0_etaa0.79.png}
	\end{minipage}
	\begin{minipage}[]{0.32\textwidth}
		\centering{\footnotesize{(c) Consumption}}
		%	\captionsetup{width=.45\linewidth}
		\includegraphics[width=1\textwidth]{../../codding_model/own_basedOnFried/optimalPol_elastS_DisuSci/figures/all_1705/comp_notaul_OPT_T_NoTaus_C_spillover0_sep1_BN0_ineq0_etaa0.79.png}
	\end{minipage}
	\begin{minipage}[]{0.32\textwidth}
		\centering{\footnotesize{\ \\(d) High skill }}
		%	\captionsetup{width=.45\linewidth}
		\includegraphics[width=1\textwidth]{../../codding_model/own_basedOnFried/optimalPol_elastS_DisuSci/figures/all_1705/comp_notaul_OPT_T_NoTaus_hh_spillover0_sep1_BN0_ineq0_etaa0.79.png}
	\end{minipage}
	\begin{minipage}[]{0.32\textwidth}
		\centering{\footnotesize{\ \\(e) Low skill}}
		%	\captionsetup{width=.45\linewidth}
		\includegraphics[width=1\textwidth]{../../codding_model/own_basedOnFried/optimalPol_elastS_DisuSci/figures/all_1705/comp_notaul_OPT_T_NoTaus_hl_spillover0_sep1_BN0_ineq0_etaa0.79.png}
	\end{minipage}
	\begin{minipage}[]{0.32\textwidth}
	\centering{\footnotesize{\ \\(f) Social welfare}}
	%	\captionsetup{width=.45\linewidth}
	\includegraphics[width=1\textwidth]{../../codding_model/own_basedOnFried/optimalPol_elastS_DisuSci/figures/all_1705/comp_notaul_OPT_T_NoTaus_SWF_spillover0_sep1_BN0_ineq0_etaa0.79.png}
\end{minipage}
	\begin{minipage}[]{0.32\textwidth}
		\centering{\footnotesize{\ \\(g) Labour fossil}}
		%	\captionsetup{width=.45\linewidth}
		\includegraphics[width=1\textwidth]{../../codding_model/own_basedOnFried/optimalPol_elastS_DisuSci/figures/all_1705/comp_notaul_OPT_T_NoTaus_Lf_spillover0_sep1_BN0_ineq0_etaa0.79.png}
	\end{minipage}
	\begin{minipage}[]{0.32\textwidth}
		\centering{\footnotesize{\ \\(h) Labour green}}
		%	\captionsetup{width=.45\linewidth}
		\includegraphics[width=1\textwidth]{../../codding_model/own_basedOnFried/optimalPol_elastS_DisuSci/figures/all_1705/comp_notaul_OPT_T_NoTaus_Lg_spillover0_sep1_BN0_ineq0_etaa0.79.png}
	\end{minipage}
\begin{minipage}[]{0.32\textwidth}
	\centering{\footnotesize{\ \\(i) Labour non-energy}}
	%	\captionsetup{width=.45\linewidth}
	\includegraphics[width=1\textwidth]{../../codding_model/own_basedOnFried/optimalPol_elastS_DisuSci/figures/all_1705/comp_notaul_OPT_T_NoTaus_Ln_spillover0_sep1_BN0_ineq0_etaa0.79.png}
\end{minipage}
\begin{minipage}[]{0.32\textwidth}
	\centering{\footnotesize{\ \\(j) Research fossil}}
	%	\captionsetup{width=.45\linewidth}
	\includegraphics[width=1\textwidth]{../../codding_model/own_basedOnFried/optimalPol_elastS_DisuSci/figures/all_1705/comp_notaul_OPT_T_NoTaus_sff_spillover0_sep1_BN0_ineq0_etaa0.79.png}
\end{minipage}
	\begin{minipage}[]{0.32\textwidth}
		\centering{\footnotesize{\ \\(k) Green research}}
		%	\captionsetup{width=.45\linewidth}
		\includegraphics[width=1\textwidth]{../../codding_model/own_basedOnFried/optimalPol_elastS_DisuSci/figures/all_1705/comp_notaul_OPT_T_NoTaus_sg_spillover0_sep1_BN0_ineq0_etaa0.79.png}
	\end{minipage}
\begin{minipage}[]{0.32\textwidth}
	\centering{\footnotesize{\ \\(l) Non-energy research }}
	%	\captionsetup{width=.45\linewidth}
	\includegraphics[width=1\textwidth]{../../codding_model/own_basedOnFried/optimalPol_elastS_DisuSci/figures/all_1705/comp_notaul_OPT_T_NoTaus_sn_spillover0_sep1_BN0_ineq0_etaa0.79.png}
\end{minipage}

\begin{minipage}[]{0.32\textwidth}
	\centering{\footnotesize{\ \\(m) Fossil output}}
	%	\captionsetup{width=.45\linewidth}
	\includegraphics[width=1\textwidth]{../../codding_model/own_basedOnFried/optimalPol_elastS_DisuSci/figures/all_1705/comp_notaul_OPT_T_NoTaus_F_spillover0_sep1_BN0_ineq0_etaa0.79.png}
\end{minipage}
\begin{minipage}[]{0.32\textwidth}
	\centering{\footnotesize{\ \\(n) Green output}}
	%	\captionsetup{width=.45\linewidth}
	\includegraphics[width=1\textwidth]{../../codding_model/own_basedOnFried/optimalPol_elastS_DisuSci/figures/all_1705/comp_notaul_OPT_T_NoTaus_G_spillover0_sep1_BN0_ineq0_etaa0.79.png}
\end{minipage}
\begin{minipage}[]{0.32\textwidth}
	\centering{\footnotesize{\ \\(o) Non-energy output}}
	%	\captionsetup{width=.45\linewidth}
	\includegraphics[width=1\textwidth]{../../codding_model/own_basedOnFried/optimalPol_elastS_DisuSci/figures/all_1705/comp_notaul_OPT_T_NoTaus_N_spillover0_sep1_BN0_ineq0_etaa0.79.png}
\end{minipage}
\end{figure} 
\begin{figure}

\begin{minipage}[]{0.32\textwidth}
	\centering{\footnotesize{\ \\(p) Energy output}}
	%	\captionsetup{width=.45\linewidth}
	\includegraphics[width=1\textwidth]{../../codding_model/own_basedOnFried/optimalPol_elastS_DisuSci/figures/all_1705/comp_notaul_OPT_T_NoTaus_E_spillover0_sep1_BN0_ineq0_etaa0.79.png}
\end{minipage}
\begin{minipage}[]{0.32\textwidth}
	\centering{\footnotesize{\ \\(q) Final output}}
	%	\captionsetup{width=.45\linewidth}
	\includegraphics[width=1\textwidth]{../../codding_model/own_basedOnFried/optimalPol_elastS_DisuSci/figures/all_1705/comp_notaul_OPT_T_NoTaus_Y_spillover0_sep1_BN0_ineq0_etaa0.79.png}
\end{minipage}
\end{figure}

Another central aspect of the paper is the importance of inequality for the optimal environmental policy. How does household heterogeneity in labour supply shape the optimal environmental policy? First, I hypothesise that the skill bias of the green sector makes a less progressive income tax optimal. 
\subsection{Sensitivity}
In this subsection, I discuss results under counterfactual parameter values to elicit the robustness of the main result: the preference of progressive labour taxation above higher fossil taxes. 
First, the productivity gap between sectors might be driving the results. Second, how do results change as within sector spillovers of research are positive? Third, I study the results of a more general specification of the utility function where income affects the choice of hours worked \citep{Boppart2019LaborPerspectiveb, Bick2018HowImplications}. 
\section{Conclusion}\label{sec:con}
Some scholars argue that  reductive policies are necessary to handle environmental limits \citep{Schor2005SustainableReductionb, VanVuuren2018AlternativeTechnologies, Bertram2018TargetedScenarios}, and the question has been raised whether consumption is too high \citep{Arrow2004AreMuch}. On the other hand, the focus of environmental policy discussions in economics rests on corrective environmental taxation. In the light of tightening environmental limits \citep{Rockstrom2009AHumanity, IPCC2022}, I study whether labor income taxes - as a reductive policy tool - can help mitigate externalities. 

In the analytical part of the paper, I show in a simple model that labor income taxes are  progressive as part of the optimal environmental policy. %The model does not feature inequality.
% Quantitative results
% baseline model
When environmental tax revenues are not redistributed lump sum, labor supply is inefficiently high. Then, income taxes serve to diminish hours worked closer to the efficient level. The result prevails absent income inequality.


% quantitative
In the second part of the paper, I analyze in a quantitative model with skill heterogeneity and endogenous growth whether the optimal labor income tax remains progressive. Again, there are no equity concerns, but workers are perfectly ensured against income differences. 
The optimal income tax is progressive to reduce inefficiently high hours worked. The quantitative model reveals that income taxes also serve as a substitute for corrective taxes. Knowledge spillovers from the non-energy sector render environmental taxes especially costly. 
Fossil taxes make energy relatively more expensive which directs research from non-energy to energy sectors. As the non-energy sector features the most research processes it is especially important for aggregate technology and knowledge spillovers. Using income taxes instead of fossil taxes to lower emissions allows to direct more research to the non-energy sector and to profit from knowledge spillovers.
In sum, however, the reduction in labor supply outweighs the positive effect on growth and consumption decreases compared to a scenario where no income tax is used. 

In the quantitative setting, the income tax affects the economic structure through two channels. First, because the fossil sector is comparably labor intense, a reduction in labor supply favors the green sector. This mechanism makes a higher tax progressivity optimal. However, the effect vanishes in equilibrium due to endogenous growth.
Second, a skill-recomposition channel makes green energy production more costly compared to fossil production. This effect arises from a skill bias in the green sector and high-skill labor being more responsive to income taxation. 
The second channel dominates the recomposing effect of  income tax progressivity in equilibrium. A market size effect amplifies the skill-recomposition channel directing research to the fossil sector. 

%Initially, the intention not to harm growth too much makes a lower progressivity optimal. As growth in the fossil sector accelerates due to the dynamic structure of endogenous growth too low progressive income taxes conflict with meeting the emission limit. As a result, optimal progressivity increases over time.
%The optimal path of income tax progressivity is decreasing, a feature mainly driven by endogenous growth. As a result, the optimal income tax progressivity and the optimal fossil tax seem to behave like substitutes in the quantitative model. 

%Skill heterogeneity depresses optimal tax progressivity due to the adverse recomposing effect of a lower high-to-low skill labor supply on the green-to-fossil energy ratio. A higher corrective tax is required to meet emission limits when there is only one skill type: with only one skill the supply of fossil-specific inputs increases thereby violating the emission limit.

%% lump-sum transfers
%When environmental tax revenues are redistributed lump-sum, the motive to use labor income taxes to deal with inefficiently high labor supply vanishes. Instead, income taxes serve to boost growth as long as this does not conflict with meeting emission limits. Therefore, they are regressive. 
%\tr{not true! it is rather that the more in research is not worth it given the dynamics! and decreasing utility gains}
%However, the regressivity decreases since more labor supply causes more emissions especially the more progressed the technology. With only a labor income tax as a tool to raise growth, accelerating technology growth is not feasible as it is concomitant with more production and emissions. 

% extensions
In an extension, I am planning to give the Ramsey planner the opportunity to limit working hours directly. The literature advocating a reduction in consumption levels \citep[e.g.,][]{Schor2005SustainableReductionb} proposes a restriction of hours worked as policy instrument to lower the consumption of resources.
Even though advocated in the literature, there is evidence for political difficulties in reducing working hours. In 2020, the French Citizens' Convention on Climate voted against reducing working hours as a measure to handle climate change. Potentially, ignorance about economic consequences is an explanation. The extension would serve to better understand economic consequences. 



\begin{comment}
\paragraph{Extension: What if the low skilled get a higher share \ar they reduce even less \ar more fossil input supply}

Redistribution to households with a higher marginal propensity to consume emissions counteracts the externality. This effect is amplified by a market size effect  of dirty goods. 

content...
\end{comment}

% I plan to discuss results under counterfactual parameter values to elicit the robustness of the main result: the preference of progressive labour taxation above higher fossil taxes. 
%First, the productivity gap between sectors might be driving the results. Second, I will abstract from endogenous growth to learn about the labour-supply-innovation channel as a driver of the optimal policy. Finally, I plan to study how results change as returns to research are increasing within sector. 
%Due to the endogeneity of technological growth in the model, the reduction in work effort fosters less research especially in the non-energy sector.  %However, more hours worked in the Ramsey model fostering research would violate the emission target. As a result, growth in technology and in consumption is inefficiently low in order to meet the emission target. 

\begin{comment}
To shed more light on the main findings, I plan conduct several additional quantitative experiments. First, I want to reduce the size of the emission target, second, I allow for a longer time frame until net-zero emissions have to be reached. The IPCC report states that for a temperature target of 2°C net-zero emissions have to be reached by 2070 only. How does this laxer target affect the importance of labour income taxes. Given the wider time frame, the green sector might be able to catch up and growth could continue. Finally, how does a change in spillovers shape the result? % \textit{(Question: I guess that substitutability is key here! Growth in green implies growths in fossil when goods are no perfect substitutes! )}
content...

%Another central aspect of the paper is the importance of inequality for the optimal environmental policy. How does household heterogeneity in labour supply shape the optimal environmental policy? First, I hypothesise that the skill bias of the green sector makes a less progressive income tax optimal. 
One main result of the paper is reduction of consumption and work effort as an optimal policy. So far, I have assumed that households are passive and preferences are fixed; there is no trade-off between environmental quality and consumption from a household perspective.
In an extension to the baseline model, I plan to depart from the representative agent assumption and explicitly model household heterogeneity. This setting allows to capture a change in household behaviour: A share of households is willing to voluntarily reduce consumption. I provide evidence for such behaviour using a representative Dutch dataset. More than 50\% of households are willing to reduce consumption in order to help the economy. Importantly, these households have a higher likelihood to work in the green sector. How does such a change in behaviour affect the optimal policy? Given the additional reduction in green-specific labour supply, the planner might find it optimal to set a more regressive tax to booster green production and research.    

\end{comment}

%However, data suggests, that households do care, and they express a willingness to reduce consumption.\footnote{\ The data I have studied comes from the Liss Panel, a representative sample of Dutch households, more than 50\% of participants indicate a readiness to change their behaviour to help the environment.} I want to study the effect of such behavioural  change on the optimal policy. Interestingly, households in high-skill jobs are more likely to declare their willingness to reduce. This linkage may intensify the trade-off between reduction and green labour supply. 


%1) BN and inequality
%2) preferences for labour
\begin{comment}
Preferences and the trade-off between leisure and consumption determining household behaviour seem to be key to the results. As argued by \cite{Boppart2019labourPerspectiveb}, the intensive margin of hours worked have been falling steadily over the last 130 years. They argue for the consistency of preferences which feature a slightly higher income effect than substitution effect. In the current model with log-utility and representative family framework,  the substitution effects offset each other. With the preferences suggested in \cite{Boppart2019labourPerspectiveb}, growth would affect hours worked, assumably changing the optimal policy. It could, for instance, be the case, that growth has to be slowed down even more, to prevent too high work efforts and consumption levels. % high-income, high-skill households might increase their labour supply with growth. 

content...
\end{comment}



%Finally, endogenising growth constitutes another interesting trade-off when the impact of fiscal policy is skill specific. 
%As regards growth, it seems reasonable to consider growth as a change in the substitutability of dirty and clean goods in the final consumption good. As it stands now, growth in the dirty sector results in emission growth, ceteris paribus. Growth might instead be associated with a more efficient use of dirty energy sources, so that more output can be generated at lower emissions.
%
%Think about effects of government using revenues for other consumption. Then reducing demand will diminish demand for the final good. 
%Broadly speaking, there are two channels through which distortionary labour taxation affects emissions. First, by affecting households' labour supply decision (efficiency channel) and second in a mechanical way by changing households disposable income. The latter effect cancels out when tax revenues are used by the government to consume the final output good. Allowing the government to recycle revenues in a different way than for final good consumption uncloses another instrument to reduce emissions. 

%Further ideas for extensions: include behavioural aspects: a voluntary reduction in demand, and a lower disutility from working in the green sector.
\begin{comment}
\paragraph{Ways forward}
How to introduce compositional effects:
\begin{enumerate}
	\item 	Utility function: With substitution and income effect not canceling (u(c)=$\frac{c^{1-\gamma}}{1-\gamma},\ \gamma\neq 1$), the wage rate might play a role, depends on GE effects.
	\item endogenising skill supply (rep agent chooses how much skill to supply, but this he already does... / might need to introduce structure as in HSV)
	\item government revenues are not used for final good consumption. Instead,  disposed of/ used for sth useful (this could be an extension and contribute to benefits of progressivity) THINK THIS ONLY CHANGES THE LEVEL TOO!
\end{enumerate}
\paragraph{Point 1 above}
change the utility function in the code to see what happens, if $\frac{Y_d}{Y_c}$ is constant in particular 
\paragraph{Point 3 above}
\textcolor{blue}{2) Government consumption wasted}
Letting the government not consume the final output good may alter the result. 
Now, the aggregate price level is determined endogenously as the goods market does not clear by Walras' law. 

In the equilibrium equations, I drop $p_t=1$ and use goods market clearing instead\\ $Y=c+\psi (x_c+x_d)$.

Blödsinn, only changes level

content...
\end{comment}
%
\section{Going forward: 12 August 22}
\begin{itemize}
	\item check CEV calculation and report results
	\item results without PV to understand dynamics!
	\item include results in other policy regimes\\
	Finally, in section \ref{subsec:comp_lumpsum}, I turn to analyze the optimal allocation under the alternative policy regimes: redistribution of environmental tax revenues via (1) lump-sum transfers and (2) the income tax scheme. 
	%
	\item \tr{integrated policy has an advantage even absent externality when there is endogenous growth!}
	\item why is there the strong deviation in the green to fossil energy ratio in the non-skill version with endogenous growth \ar has to stem from the higher environmental tax? \checkmark 
	\item \tr{Idea: could be that endogenous growth intensifies shift in skill ratio \ar if so, would expect a smaller reaction of skill supply in xgr model (but then the policy is differen!) Would need experiment with same policy; counterfactual}
	\item also why is there a higher environmental tax in the non-skill model? Why is it needed? \checkmark
\end{itemize}
\textbf{extensions:}
\begin{itemize}
	\item other derivation of emission target not reduction by 50\% but weighted by what the country contributes, for instance
	
	\item which  regime is best for equity measured in terms of utility, wages?lump-sum transfers or additional progressive taxes? \ar Evaluate by looking at high and low skill wages. 
	\item think about \textbf{involuntary unemployment} \ar shouldn't there be gains from an overall reduction in labor supply in terms of involuntary unemployment? As households want to work less?
	\item empirical studies motivated through this paper's results?
	\item  What if there is a pre-existing income tax but not gov funding constraint? could still find an increase in labor income tax if optimal level is above initial level
	\item  look at a policy where income taxes are used to fund government spending \ar then this would generate more gov. revenues
	\item commment on whether there is a double dividend from using income taxes
	\item add inequality and heterogenous consumption bundles to model and ask about changes in the distribution of income
\end{itemize} 
\subsection{Sensitivity}
I will now briefly discuss sensitivity analyses to the quantitative exercise. 
\subsubsection{Wage elasticity of labor}

Recent papers have examined the wage elasticity of labor. \cite{Boppart2019LaborPerspectiveb} present evidence that hours worked per worker have been falling steadily over time 

\subsubsection{Research subsidy}
but finding should be similar to version without endogenous growth
\subsubsection{Changing emission limits calculation}
The \textit{equal-per-capita} approach is favorable for population high countries like the US. Therefore, in the sensitivity analysis, I rerun the model where US emission limits follow from \textit{Equal cumulative per capita} approach, where countries with historically  high emissions per capita mitigate more.
% \textit{constant-emission-ratio} approach which is achieved by all countries reducing emissions by 50\% in the 2030s relative to 2019. This principle limits US emissions to 2.309Gt in the 2030s. THIS APPROACH IS EVEN MORE FAVOURABLE  TO THE US BECAUSE CURRENTLY THEY EMIT A HIGHER SHARE PER CAPITA THAN THE REST OF THE WORLD. 

\subsubsection{Technology gap}

\begin{comment}
\section{Model}\label{app:model}

content...


\subsection{Solving the model}
%Demand for intermediate goods determines the price ratio, $\frac{p_g}{p_f}$ in equilibrium, eq. \eqref{eq:ana_dem_fin}. 
%Intermediate good market clearing, i.e. substituting intermediate good production functions, eq. \eqref{eq:ana_prod_F} and \eqref{eq:ana_prod_G}, in \eqref{eq:ana_dem_fin} yields
%\begin{align}
%\frac{p_g}{p_f}=\frac{A_f}{Ag}\frac{s}{1-s}\frac{1-\varepsilon}{\varepsilon},
%\end{align}
%where I used that $s=\frac{L_F}{h}$ and the labor market clearing condition. Solving the price definition of the final goods price, eq. \eqref{eq:ana_pr} for $p_g$ as a function of $\frac{p_g}{p_f}$ and substituting the previous expression for the price ratio gives the price of the clean good in equilibrium as a function of labor shares:
%\begin{align}
%p_g=(1-\varepsilon)\left(\frac{A_f}{A_g}\right)^\varepsilon\left(\frac{s}{1-s}\right)^{1-\varepsilon}.
%\end{align}
%The equilibrium price paid for the clean good 



\subsection{Inefficiency in the wage rate with externality}

The wage rate in the competitive equilibrium is below the marginal product of hours worked. 
The reason is that the lower labor share in the dirty sector has to be sustained by a tax on dirty production. 
Otherwise, market forces would equilibrate the marginal product of labor in both sectors and $\varepsilon=s$. Yet, this disregards the negative externality of dirty production. 

In the competitive equilibrium, hence, the environmental tax serves to sustain the wedge between marginal products of labor across intermediate sectors: it is higher in the dirty and lower in the green sector.

This comes at the cost of lowering the aggregate wage rate in the economy to the marginal product of labor in the green sector. 
Nevertheless, the marginal product of labor in the fossil sector is higher. As a result, the aggregate wage rate, $w$, falls short of the aggregate marginal product of labor in the economy. This mechanism on its own renders labor supply inefficiently low in the competitive economy. However, as shown in proof \ref{prop:1}, the equilibrium hours supplied are inefficiently high in the competitve allocation. 


When the effect of a decreased wage  en gros reduces labor supply, it makes it more costly for the government to generate funds due to a smaller tax base for the income tax. This is the mechanism pointed to by the double dividend literature.


\end{comment}
\clearpage
\appendix
\section{Derivations and proofs}\label{app:derivations}

\subsection{Theory results \ref{sec:mod_an}}
\subsubsection{Useful relations in the simple model}\label{app:dervs_use}
\begin{align*}
\frac{\partial Gov}{\partial s}=\frac{\partial Y}{\partial F}\frac{\partial F}{\partial s}+\frac{\partial Y}{\partial G}\frac{\partial G}{\partial s}-\frac{\partial C}{\partial s}\\
\frac{\partial Gov}{\partial H}=\frac{\partial Y}{\partial F}\frac{\partial F}{\partial s}+\frac{\partial Y}{\partial G}\frac{\partial G}{\partial s}-\frac{\partial C}{\partial H}\\
\frac{\partial Gov}{\partial s}=\frac{\partial Y}{\partial s}-\frac{\partial C}{\partial s}\\
\frac{\partial Gov}{\partial s}=p_f F \frac{\partial \tau_F}{\partial s}+\tau_F F \frac{\partial p_f}{\partial s}+\tau_F p_f \frac{\partial F}{\partial s}\\
%\frac{\partial \tau_F}{\partial s}= -\frac{1-\varepsilon}{\varepsilon}\frac{1}{(1-s)^2}, \\
%\frac{\partial \tau_F}{\partial s}=p_f\frac{1-\varepsilon}{1-\tau_F}\frac{\partial \tau_F}{\partial s}\\
\frac{\partial F}{\partial s}=\frac{F}{s}
\\
\frp{Y}{H}= \frp{Y}{s}\frac{s}{H}+\frp{Y}{G}\frp{G}{Lg}
\\
\frp{G}{H}=-\frac{(1-s)}{H}\frp{G}{s}
\\\frp{G}{s}=-H\frp{G}{L_G}\\
\frp{F}{H}=\frac{s}{H}\frp{F}{s}\\
\frp{F}{s}=H\frp{F}{L_F}
\end{align*}

\subsubsection{Reduction in dirty labor share is efficient}
\begin{proof}
	With a negative externality of dirty production it has to hold that 
	\begin{align}
	\frp{Y}{F}\frp{F}{s}>-\frp{Y}{G}\frp{G}{s},
	\end{align}
	which can be rewritten to 
	\begin{align}\label{eq:mpl_eff}
	\frp{Y}{L_F}>\frp{Y}{L_G}. 
	\end{align}
	In the efficient allocation absent externality, marginal products of dirty and green labor are equalized. 
	Under decreasing returns to scale it holds that the left-hand side is decreasing in $L_F$ and the right-hand side of equation \ref{eq:mpl_eff} is decreasing in $L_G$. Hence, the adjustment to satisfy equation \ref{eq:mpl_eff} relative to the efficient allocation without externality requires a decrease in $L_F$ and/or a rise in $L_G$  .
	This reallocation is achieved by reducing $s$, since $L_F=sH$ and $L_G=(1-s)H$.	
\end{proof}


\begin{comment}
content...
\paragraph{If a reduction in dirty labor share is efficient, then the aggregate production function features decreasing returns to scale in labor}
\begin{proof}
	\textit{The proof rest on the assumption that returns to scale are symmetric across dirty and clean production; either both decreasing or both are non-decreasing.}
It holds by assumption that $s_{FB,E>0}<s_{FB,E=0}$, where $E>0$ indicates that the externality is active. 
Assume by contradiction that the aggregate production function features non-decreasing returns to scale. This implies that:
\begin{align}
\left. \frp{Y}{L_F} \right|_{s_{FB,E>0}}\leq \left. \frp{Y}{L_F} \right|_{s_{FB,E=0}},\\
\left. \frp{Y}{L_G} \right|_{s_{FB,E>0}}\geq \left. \frp{Y}{L_G} \right|_{s_{FB,E=0}}.
\end{align}
When there is no externality, the efficient allocation is characterized by
\begin{align}
\left. \frp{Y}{L_F} \right|_{s_{FB,E=0}}= \left. \frp{Y}{L_G} \right|_{s_{FB,E=0}}.
\end{align}
Using the inequalities above yields
\begin{align}
\left. \frp{Y}{L_F} \right|_{s_{FB,E>0}}\leq \left. \frp{Y}{L_G} \right|_{s_{FB,E>0}}.
\end{align}
This contradicts the optimality condition which requires 
\begin{align}
\left. \frp{Y}{L_F} \right|_{s_{FB,E>0}}> \left. \frp{Y}{L_G} \right|_{s_{FB,E>0}}.
\end{align}
Hence, when a reduction in the dirty labor share is efficient, then the aggregate production function features decreasing returns to scale in both labor input goods. 
\end{proof}
\end{comment}

\subsubsection{The social cost of pollution and the Pigouvian tax rate}\label{app:scp}

The social cost of pollution in my model is defined as the marginal price the representative household is willing to pay for a marginal reduction in dirty production. That is, the household maximises over dirty production for which a market exists.

The household's problem is determined as
\begin{align}
\underset{C,H,F}{\max} U(C,H,F)-\mu \left(C+\tilde{p}_FF-Y(H)\right).
\end{align}
Where $\mu$ is the Lagrange multiplier. Taking the derivative with respect to dirty production  and with respect to consumption yields
\begin{align}
U_F=\mu \tilde{p}_F,\\
U_C=\mu.
\end{align}
Substituting the Lagrange multiplier gives the negative of the equilibrium price the household is willing to pay for a reduction in dirty prodction: $\tilde{p}_F=\frac{U_F}{U_C}$. Since the environmental tax in the model is a percentage of revenues, the price producers pay per unit of dirty production is $\tau_F p_F$. Thus, the social cost of pollution to be deducted from to producers' revenues in percent is $\tau^{Pigou}=\frac{-U_F}{U_Cp_F}$.


\subsubsection{With a positive environmental tax, the wage rate in the competitive equilibrium is below the marginal product of labor}\label{app:wageMPL}

The aggregate marginal product of labor is defined as
\begin{align}
MPL&= \frp{Y}{H}.
\end{align}
This expression can be rewritten using relations of derivatives summarized in \ref{app:dervs_use} as follows.
\begin{align}
&= \frp{Y}{F}\frp{Y}{H}+\frp{Y}{G}\frp{G}{H}\\
&= \frp{Y}{F}\frp{F}{L_F}s+\frp{Y}{G}\frp{G}{L_G}(1-s)\\
&= \frp{Y}{G}\frp{G}{L_G}+ s\left(\frp{Y}{F}\frp{F}{L_F}-\frp{Y}{G}\frp{G}{L_G}\right).\label{eq:mpl_opt}
\end{align}
The term in brackets is positive under the optimal policy as can be seen from the first order condition with respect to $s$, equation \ref{eq:sbs}:
\begin{align}
\frp{Y}{F}\frp{F}{L_F}-\frp{Y}{G}\frp{G}{L_G}=\frac{1}{H}\left(\frp{Y}{F}\frp{F}{s}+\frp{Y}{G}\frp{G}{s}\right)=\frac{1}{H}\left(\frac{-U_F\frp{F}{s}}{U_C}\right)>0.
\end{align}
The inequality holds since the externality of polluting production is negative. %, above expression is positive.
%Therefore, the marginal product of labor in the efficient allocation equals
Now note that the first summand in equation \ref{eq:mpl_opt} is the competitive wage rate.  Hence $w<MPL$.

The gap between the wage rate and the marginal product of labor equals the gap between the marginal products of labor across sectors times the relative size of the dirty sector. 

\subsubsection{Sufficiency of the environmental tax when environmental tax revenues are redistributed lump sum}\label{app:incometax0}

Noticing that $\frac{\partial Y}{\partial H}= \frac{\partial Y}{\partial s}\frac{s}{H}-\frac{\partial Y}{\partial G}\frac{\partial G}{\partial s}\frac{1}{H}$ and that $\frac{\partial F}{\partial H}=\frac{\partial F}{\partial s}\frac{s}{H}$, and substituting equation \ref{eq:sbs} in equation \ref{eq:sbh} yields
\begin{align}\label{eq:pigou}
-U_C \frac{\partial Y}{\partial G}\frp{G}{L_G}=-U_H.
\end{align}
Hence, if the environmental tax is set to guarantee that condition \ref{eq:sbh} holds, then optimal hours worked only trade-off the disutility from labor and the utility from more consumption when environmental tax revenues are redistributed lump-sum.

Equation \ref{eq:pigou} also holds for the social planner allocation simplifying the second first order condition, equation \ref{eq:fbh}.


Substituting $U_H$ from household optimality, equation \ref{eq:hsup}, and the clean sectors' profit maximizing condition from equations \ref{eq:profmax} yields
\begin{align}
1=1-\tau^*_\iota.
\end{align}
Hence, $\tau^*_\iota =0$ from which follows that $\lambda =1$ so that the income tax scheme is a flat tax rate equal to zero; the labor income tax is not used in optimum.

%\subsubsection{Simplifying social planner's first order conditions}
%
%The social planner's first order condition on labor can be rewritten as in the previous section to
%\begin{align}
%-U_H=U_C\frac{\partial Y}{\partial G}\frp{G}{L_G}
%\end{align}
\subsubsection{Proof proposition \ref{prop:1}: Absent lump-sum transfers, hours are inefficiently high under decreasing returns to scale}\label{app:nolumpsum_hourshigh}
\begin{proof}\textit{Absent lump-sum transfers, hours are inefficiently high when the environmental tax implements efficient share of dirty production and the aggregate production function features decreasing returns to scale in labor inputs.}
	
	This proof proceeds by contradiction. 
	Assume by contradiction that $H^*\leq H_{FB}$. 
	It has to hold that 
	\begin{align}
	-U_H^*\leq -U_{H,FB}.
	\end{align} 
	
	Substituting the households' optimal labor supply and the social planner's first order condition for hours, equation \ref{eq:fbh_simp} yields
	\begin{align}\label{eq:prH}
	U_C^*w^* \leq U_{C,FB}\frp{Y_{FB}}{G_{FB}}\frp{G_{FB}}{L_{G,FB}}.
	\end{align}
	
	Rewriting equation \ref{eq:prH} above yields
	\begin{align}
	\frac{U_C^*}{U_{C,FB}}\leq \frac{\frp{Y_{FB}}{G_{FB}}\frp{G_{FB}}{L_{G,FB}}}{\frp{Y^*}{G^*}\frp{G^*}{L^*_{G}}},
	\end{align}
	where I replaced $w^*=\frp{Y^*}{G^*}\frp{G^*}{L^*_{G}}$.
	
	By assumption $s^*=s_{FB}$, $H^*\leq H_{FB}$, and the aggregate production function is increasing in its inputs. It follows that output is higher in the efficient allocation $Y_{FB}\geq Y^*$ and hence $C^*<C_{FB}$, since $Gov>0$ in the competitive equilibrium. By additive separability of the utility function and strict concavity with respect to consumption, we have that $\frac{U_C^*}{U_{C,FB}}>1$.
	
	Now note that $H^*\leq H_{FB}$ implies  $L_G^*\leq L_{G,FB}$, since the dirty labor share is equal. Under decreasing returns to scale of aggregate production to clean labor, it holds that the right-hand side is below or equal unity.Thus,
	\begin{align}
	\frac{U_C^*}{U_{C,FB}}>1\geq \frac{\frp{Y_{FB}}{G_{FB}}\frp{G_{FB}}{L_{G,FB}}}{\frp{Y^*}{G^*}\frp{G^*}{L^*_{G}}}. 
	\end{align}
	A contradiction to the assumption that $H^*\leq H_{FB}$. Hence, it has to hold that $H^*>H_{FB}$. 
\end{proof}


\subsubsection{Derivation $\tau_F^*$ without lump-sum transfers}\label{app:reiv_tauf}
	
Divide the Ramsey planner's first order condition with respect to $s$, equation \ref{eq:sbs}, by $U_C$ and $\frp{Y}{F}\frp{F}{s}$. Solving for $1+\frac{\frac{\partial Y}{\partial G}\frac{\partial G}{\partial s}}{\frac{\partial Y}{\partial F}\frac{\partial F}{\partial s}}$, which equals $\tau_F$, yields the desired result:

\begin{align}
\tau_{F}=SCC + \frp{Gov}{s}.
\end{align}

\begin{comment}
The latter summand can be rewritten to 
\begin{align}
\frp{Gov}{s}= \frp{Y}{s}+H^2 \frp{\left(\frp{Y}{L_G}\right)}{L_G}.
\end{align}
Where under decreasing returns to scale the second summand is negative and the first is positive. \textit{To be continued.} 

content...
\end{comment}
\subsubsection{Derivation $\tau_l$ without lump-sum transfers }\label{app:subsub_nltaul}

\begin{proof}\textit{Absent lump-sum transfers, the optimal income tax scheme is progressive}
Following similar steps as in section \ref{app:incometax0}, the optimal labor income tax progressivity parameter is given by
\begin{align}
\tau_{\iota}^*=\frac{\frac{s}{H}\frac{\partial Gov}{\partial s}- \frac{\partial Gov}{\partial H}}{\frac{\partial Y}{\partial G}\frac{\partial G}{\partial s}\frac{1}{H}}.
\end{align}

	Using the market clearing condition for final output to replace government spending and noticing the relations of derivatives with respect to aggregate labor supply and the dirty labor share, one can write above expression as
	\begin{align}
	\tau_{\iota}=1-\frac{H\frp{C}{H}-s\frp{C}{s}}{wH}.
	\end{align}
	Substituting $\frp{C}{H}=H\frp{w}{H}+w$ and $\frp{C}{s}=H\frp{w}{s}$ from the household's budget constraint gives
	\begin{align}
	\tau_{\iota}=\frac{s}{w}\frp{w}{s}-\frac{H}{w}\frp{w}{H}.
	\end{align}
In a next step, I explicitly solve for $\frp{w}{s}$ and $\frp{w}{H}$, where I use that $w=\frp{Y}{G}\frp{G}{L_G}$ in equilibrium.

\begin{align}
\frp{w}{H}=\left(\frp{G}{L_G}\right)^2\frac{\partial^2Y}{\partial G^2}(1-s)+\frp{Y}{G}\frac{\partial ^2G}{\partial L_G^2}(1-s)+\frp{G}{L_G}\frac{\partial^2 Y}{\partial G \partial F}s\\
%%%%
\frp{w}{s}= \left(\frp{G}{L_G}\right)^2\frac{\partial ^2Y}{\partial G^2}(-H)+\frp{G}{L_G}\frac{\partial ^2Y}{\partial G \partial F}H+\frp{Y}{G}\frac{\partial ^2 G}{\partial L_G^2}(-H)
\end{align}
substituting derivatives and canceling terms yields:
\begin{align}
\tau_\iota= -\frac{H}{w}\frp{\left(\frp{Y}{L_G}\right)}{L_G}.=-\frac{H}{w}\left(\left(\frp{G}{L_G}\right)^2\frac{\partial ^2Y}{\partial G^2}+\frp{Y}{G}\frac{\partial ^2G}{\partial L_G ^2}\right).
\end{align}
Under the assumption of decreasing returns to scale of aggregate production with respect to green labor the term in brackets is negative, and it holds that $\tau_\iota >0$ and the optimal income tax rate is progressive. 

For intuition, note that the right-hand side of the previous expression equals the partial derivative of the wage rate with respect to the dirty labor share under the assumption that dirty production is fixed divided by the wage rate:
\begin{align}
\tau^*_\iota =\left. \frac{1}{w}\frp{w}{s} \right|_{F=\bar{F}}.
\end{align}
%Since the presence of the environmental tax artificially increases labor in the green sector depressing the wage rate (under the assumption of decreasing returns to scale), the wage rate rises by a reduction of the green labor share. 

The equation makes clear that environmental taxation and the labor income tax are complements. When the environmental tax rises, thereby increasing the share of labor allocated to the green sector, the marginal product of green labor decreases further. A marginal reduction in the green labor share would increase the wage rate more the higher the green labor share, hence, the optimal labor tax progressivity increases with the environmental tax. 
Secondly, the wage rate decreases with $\tau_F$ which as well inflates the optimal labor tax progressivity. 
	\end{proof}

\subsubsection{Proof proposition: Infeasibility of efficient allocation}\label{app:ineff}
\begin{proof}\textit{The efficient allocation is infeasible (under the assumption of constant or decreasing returns to scale)}
	To prove this claim, I assume that the government chooses the optimal policy; which is the highest social welfare the Ramsey planner can achieve. I show that the optimal policy does not satisfy the social planner's allocation. Since the social planner could have chosen the Ramsey planner's allocation  but did not, it follows that the social planner's allocation features a higher social welfare.
	
	For the optimal allocation to be efficient, it must be the case that $s^*=s_{FB}$, (i) $C^*=C_{FB}$, and (ii) $H^*=H_{FB}$. I show that, under the assumption that $s^*=s_{FB}$, either (i) or (ii) can hold at a time by demonstrating that assuming (i) violates (ii) and vice versa.
	
	
	
	%\begin{lemma}\textit{$\tau_F=0$ is not optimal}
	%When $\tau_F=0$ then $Gov=0$ and $\frp{Gov}{s}=0$. Furthermore, market forces then imply that the marginal products of labor are equal so that $\frp{Y}{F}\frp{F}{s}=-\frp{Y}{G}\frp{G}{s}$. Substituting this in equation \ref{eq:sbs} yields
	%\begin{align}
	%0=-U_F\frp{F}{s}>0,
	%\end{align}
	%a contradiction. 
	%\end{lemma}
	%
	%\textit{(i) Assume $C^*=C_{FB}$ and $s^*=s_{FB}$:}
	%Since $\tau_F\neq0$, it follows that $Gov>0$ and hence $Y^*=C^*+Gov>Y_{FB}$. Since the allocation of labor is the same in the efficient and the optimal allocation and output is rising in labor, it follows that $H^*>H_{FB}$. 
	%\tr{Missing: if $\tau_F<0$ then $Gov<0$ }
	
	If $s^*=s_{E>0,FB}<s_{E=0,FB}$ then it must be the case that the environmental tax is positive to sustain a gap between marginal productivities in the dirty and the clean sector: $\tau_F>0$. Then, $Gov=\tau_Fp_fF>0$. 
	First assume that (i) holds true: $C^*=C_{FB}$. From the good's market clearing condition and resource constraint of the social planner's problem it follows that
	$Y^*-Gov=C^*=C_{FB}=Y_{FB}$, due to  $Gov>0$ we have that $Y^*>Y_{FB}$. Since hours are the only production input, positively affect output, and $s^*=s_{FB}$ the higher output in the optimal allocation implies that $H^*>H_{FB}$. A violation of condition (ii). 
	
	Assume now that condition (ii) holds: $H^*=H_{FB}$. Since $s^*=s_{FB}$ by assumption it holds that $Y^*=Y_{FB}$ and, by the same argument as before: $Gov>0$. Thus, by the resource and market clearing condition: $C_{FB}=Y_{FB}>Y^*-Gov=C^*$. When labor supply is efficient, then consumption is inefficiently low; condition (i) is violated. 
	
	\begin{comment} (Proof building on first order conditions)
	Assume, 
	The social planner's first order condition on labor supply can be written as
	\begin{align}
	-U_{H, FB}=U_{C, FB}\frp{Y}{G}_{FB}\frp{G}{L_G}_{FB}
	\end{align}
	and optimal labor supply is determined by
	\begin{align}
	-U^*_{H}&=U^*_C(1-\tau_\iota)w
	\end{align}
	Equalizing yields
	\begin{align}
	U_C^*(1-\tau_\iota)w=U_{C,FB}\frp{Y}{G}_{FB}\frp{G}{L_G}_{FB},
	\end{align}
	a condition for optimal labor supply to be efficient. 
	
	In the following, I demonstrate that (i) assuming $C^*=C_{FB}$ violates the condition above and $H^*\neq H_{FB}$ and that (ii) assuming $H^*=H_{FB}$ results in $C^*<C_{FB}$. 
	
	\textit{(i) Assume $C^*=C_{FB}$:}
	then
	\begin{align}
	(1-\tau_\iota)w=\frp{Y}{G}_{FB}\frp{G}{L_G}_{FB}.
	\end{align}
	Assume by contradiction that $H^*=H_{FB}$, since $s^*=s_{FB}$ by assumption, it follows that $w=\frp{Y}{G}_{FB}\frp{G}{L_G}_{FB}$. 
	Since $\tau_\iota\neq 0$ under constant or decreasing returns to scale, it holds that $H^*<H_{FB}$, a contradiction. 
	
	
	%Hence,
	%\begin{align}
	%(1-\tau_\iota)w<\frp{Y}{G}_{FB}\frp{G}{L_G}_{FB}.
	%\end{align}
	%
	%Labor supply in the competitive equilibrium is lower than in the efficient allocation when consumption is equal under the optimal policy. WHY?
	%It follows, that optimal labor supply does not equal its efficient counterpart when optimal consumption is efficient.
	
	\textit{(ii) Assume $H^*=H_{FB}$:} 
	It follows that 
	\begin{align}
	\frac{U_C^*}{U_{C,FB}}=\frac{\frp{Y}{G}_{FB}\frp{G}{L_G}_{FB}}{w}\frac{1}{1-\tau_\iota}=\frac{1}{1-\tau_\iota}>1.
	\end{align}
	From concavity of the utility function it follows that $C^*<C_{FB}$. 
	
	content...
	\end{comment}
\end{proof}

\subsubsection{Proofs proposition \ref{prop:3}}\label{app:proofintegrated}
\begin{proof} \textit{The optimal income tax scheme is progressive}\\ % if the optimal environmental tax is positive.}\\
	Under the new policy, the household's labor supply is determined by
	\begin{align}
	-U_H=\frac{U_C (1-\tau_{\iota})(wH+\tau_F p_fF)}{H}.
	\end{align}
	Expressing the derivatives in the Ramsey planner's first order condition with respect to hours as derivatives with respect to the dirty labor share, $s$, and substituting the first order condition with respect to $s$ yields:
	\begin{align}
	U_C \frp{Y}{G}\frp{G}{L_G}=-U_H.
	\end{align}
	%This equation is equivalent to the social planner's first order condition on hours, equation \ref{eq:fbh}. The optimal policy is to choose
	%\tr{Does this give a hint to why inefficiency without redistribution? The Ramsey planner's foc and household optimality always coincide. But, when Gov does not cancel the two do not coincide! ? the two do not coincide, Bcs consumption is too low so that $U_C$ too high which increases}
	Noticing that $\frp{Y}{G}\frp{G}{L_G}=w$ and replacing household's labor supply condition gives
	\begin{align}
	& w=\frac{(1-\tau_\iota)Y}{H}\\
	\Leftrightarrow\ & \tau_\iota=1-\frac{wH}{Y}. 
	\end{align} 
	Since $Y=C=wH+\tau_Fp_fF$ from the market clearing and household budget constraint, it follows that $wH<Y$ whenever $\tau^*_F>0$. Hence, $\tau_F^*>0$ implies $\tau^*_{\iota}>0$.
	%
	%Observe that $Y\geq MPL \times H$, where $MPL$ stands in for the marginal product of labor, if the aggregate production function features decreasing or constant returns to scale. Under such a production function one can rewrite the last expression as
	%\begin{align}
	%\tau_{\iota}=1-\frac{wH}{Y}\geq 1-\frac{w H}{MPL \times H}
	%\end{align}
	% Note further that the marginal product of labor exceeds the wage rate whenever the environmental tax is different from zero; compare the disucssion in subsection \ref{subsec:Rams}. It follows that the right-hand side is positive, hence
	%\begin{align}
	%\tau_{\iota}>0,
	%\end{align}
	The optimal tax scheme is progressive.
\end{proof}

\begin{proof}\textit{The optimal allocation is efficient}
	
	The idea of this proof is to show that the efficient allocation is attainable for the Ramsey planner. Since the social planner could implement any competitive allocation (which necessarily satisfies the resource constraint) and has the same objective function, the efficient allocation maximizes the Ramsey problem. 
	
	To show that the efficient allocation is feasible, I assume that $s^*=s_{FB}$. Showing that $H^*=H_{FB}$ and $C^*=C_{FB}$ are a solution to the Ramsey problem, proves that the optimal policy implements the efficient allocation for two reasons. First, by the argument in the previous paragraph any competitive allocation is a potential candidate solution to the social planner's problem and the social planner has the same objective function. Second, due to strict concavity of the utility and strict monotonicity of the production function \textit{(so that more input means more output)}, the solution is also unique.
	
	When $H^*=H_{FB}$ then $C^*=C_{FB}$ since $s^*=s_{FB}$ by assumption. It now show that under this allocation optimal labor supply, indeed, is efficient, that is:
	\begin{align}
	U_C^*\frp{Y^*}{G^*}\frp{G^*}{L^*_{G}} = U_{C,FB}\frp{Y_{FB}}{G_{FB}}\frp{G_{FB}}{L_{G,FB}}.
	\end{align}
	
	From the assumed allocation it follows that $U_C^*=U_{C,FB}$ and $\frp{Y_{FB}}{G_{FB}}\frp{G_{FB}}{L_{G,FB}}=\frp{Y^*}{G^*}\frp{G^*}{L^*_{G}}$ and above condition is satisfied. 
	
	It remains to show that under the assumed allocation, $s^*=s_{FB}$ holds true. Since $Gov=0$ the Ramsey planner's first order condition with respect to $s$ equals that of the social planner. Since production and marginal utilities in the optimal allocation equal their counterparts in the efficient allocation, it has to holds that $\tau_F^*$ implements $s^*=s_{FB}$.  
	%Second, efficiency of labor supply, i.e., $H^*=H_{FB}$, as the only solution of the Ramsey planner's problem follows from demonstrating that both (i) $H^*>H_{FB}$ and (ii) $H^*<H_{FB}$ result in a contradiction under the assumption that $s^*=s_{FB}$.
	
	%Assume by contradiction that (i), $H^*>H_{FB}$. 
\end{proof}


%
\section{analytic Model with  functional forms}
\subsection{Competitive equilibrium in simple model}

\begin{align}
\text{Utility}\hspace{5mm}& \frac{C_t^{1-\theta}-1}{1-\theta}-\chi \frac{h_t^{1+\sigma}}{1+\sigma}-\varphi(\omega F)^\eta\\
\text{Budget}\hspace{5mm}& C_t = \lambda_t(w_th_t)^{1-\tau_{\iota t}}\\
\text{optimality HH}\hspace{5mm}& h^{\sigma+\tau_{\iota t}+\theta(1-\tau_{\iota t})}=\lambda_t^{1-\theta}(1-\tau_{\iota t})w_t^{(1-\tau_{\iota t})(1-\theta)}\\
\text{Final Production}\hspace{5mm}&Y=F^{\varepsilon_y}G^{1-
	\varepsilon_y}\\ %\left[F^\frac{\varepsilon_y-1}{\varepsilon_y}+G^\frac{\varepsilon_y-1}{\varepsilon_y}\right]^\frac{\varepsilon_y}{\varepsilon_y-1}\\
%\text{price}\hspace{5mm}&1=p_y= \left(\frac{p_f}{\varepsilon_y}\right)^{\varepsilon_y}\left(\frac{p_g}{1-\varepsilon_y}\right)^{1-\varepsilon_y}\label{eq:ana_pr}\\
\text{Demand clean good}\hspace{5mm}&p_g=(1-\varepsilon)\left(\frac{F}{G}\right)^\varepsilon\label{eq:ana_dem_clean}\\
\text{Demand clean good}\hspace{5mm}&p_f=\varepsilon\left(\frac{F}{G}\right)^{\varepsilon-1}\label{eq:ana_dem_dirty}\\
\text{Production F and G}\hspace{5mm}&F=A_fL_F\label{eq:ana_prod_F}\\\
& G=A_gL_G\label{eq:ana_prod_G}\\
\text{labour demand}\hspace{5mm}& w=p_f(1-\tau_{ft})A_f\\
& w=p_gA_g\\
\text{technology}\hspace{5mm}&A_{ft+1}=(1+\nu_f)A_{ft}\\
&A_{gt+1}=(1+\nu_g)A_{gt}\\
\text{Government}\hspace{5mm}&T=\tau_{f}p_fF
\\
\text{Balanced income tax revenues}\hspace{5mm}&\lambda_t=\frac{w_t h_t}{(w_t h_t)^{1-\tau_{\iota t}}}\\
&E_{net}=\omega F-\delta
\end{align}
\subsubsection{Derivation expression for $h^{FB}$}
Rewriting equation \ref{eq:fbh}, the efficient amount of hours worked can be indirectly expressed as:
\begin{align}
h^{FB}=\frac{1}{\chi^\frac{1}{\sigma}}\left(w_{eff}^{1-\theta}-\frac{dE}{dF}A_f s^{FB}\left(h^{FB}\right)^\theta \right)^\frac{1}{\sigma+\theta}.\label{eq:heff_1}
\end{align}

Note that an explicit expression for $h^{FB}$ follows from equation \ref{eq:fbs} when there is an externality and $\frac{dE}{dF}\neq 0$. Then 

\begin{align}
h^{FB}= \left(\frac{\varepsilon(1-s)-s(1-\varepsilon)}{s(1-s)}\frac{w_{eff}^{1-\theta}}{\frac{dE}{dF}A_f }\right)^\frac{1}{\theta}
\end{align}
and the result follows from substituting the last expression in expression  \ref{eq:heff_1}.

\subsection{Proof: Hours worked with only the efficient share of dirty labor are inefficiently high}

\begin{proof}
	The proof proceeds in two steps. First, I show that the share of labor allocated to the dirty sector is smaller than its efficient level absent externality which is $s=\varepsilon$.
	In the second step, I show that even if the environmental tax is set to the tax which replicates the efficient share of dirty labor, hours worked, denoted by $h_{CE, s^{eff}}$, exceed their efficient level, $h_{FB}$, when neither lump-sum transfers no labour income taxes are available. 
	
	First note that the share of dirty labor is fully determined by the environmental tax. The environmental tax is set to implement a gap between the marginal product of labor between the clean and the dirty sector. The relation follows from labor market clearing and intermediate goods market clearing 
	\begin{align}
	\tau_f = \frac{\varepsilon-s}{(1-s)\varepsilon}\label{eq:tauf}
	\end{align}
	
	\textbf{Step 1:} $\frac{dE}{dF}>0$ \ar $\varepsilon>s$\\
	Rewriting equation \ref{eq:fbs} yields
	\begin{align}
	\frac{\varepsilon(1-s)-s(1-\varepsilon)}{s(1-s)}=\frac{dE}{dF}A_fh^\theta w_{FB}^{1-\theta}.
	\end{align}
	When the externality is negative, i.e., $\frac{dE}{dF}>0$, then the right-hand side is positive.
	Since $s\in(0,1)$ - due to both intermediate goods being necessary to produce the final good and zero consumption is not a solution - the left-hand side is positive when
	\begin{align}
	\varepsilon(1-s)-s(1-\varepsilon)>0,
	\end{align}
	which holds true if and only if $\varepsilon>s$.
	
	\textbf{Step 2:} $\varepsilon>s$ \ar $h_{CE, s^{eff}}>h_{FB}$\\
	I prove the claim by evoking a contradiction to the assumption that $h_{CE, s^{eff}}\leq h_{FB}$. Using equation \ref{eq:hopt} and \ref{eq:heff} the expression becomes
	
	\begin{align}
	&\left(\frac{w^{1-\theta}}{\chi}\right)^{\frac{1}{\sigma+\theta}}\leq \left(\frac{w_{FB}^{1-\theta}}{\chi}\frac{1-\varepsilon}{1-s}\right)^\frac{1}{\sigma+\theta}
	\\
	\Leftrightarrow&\left(\frac{w}{w_{FB}}\right)^{\frac{1-\theta}{\sigma+\theta}}\leq \left(\frac{1-\varepsilon}{1-s}\right)^\frac{1}{\sigma+\theta}
	\end{align}
	
	Note that the ratio of the wage in the competitive economy to the marginal product of labor is $\frac{w}{w_{FB}}=\frac{1-\varepsilon}{1-s}$, which follows from equation \ref{eq:compw} and the definition of $w_{FB}$ under the assumption that the dirty labor share is set to the first best equivalent. Substituting this in the previous equation and rearranging terms yields
	\begin{align}
	\left(\frac{1}{1-\varepsilon}\right)^\frac{\theta}{\sigma+\theta}\leq \left(\frac{1}{1-s}\right)^\frac{\theta}{\sigma+\theta}
	\end{align}
	which holds true whenever
	\begin{align}
	\varepsilon<s.
	\end{align}
	This contradicts $s<\varepsilon$ which has been shown to hold in presence of a negative externality in the dirty sector in step 1. 
\end{proof}

\textit{Intuition:} the fact that the wage rate in the competitive equilibrium understates the marginal product of labor  depresses labor supply due to a substituion effect. When the income effect is more pronounced, that is, $\theta>1$, the low wage rate increases labor supply above the efficient level. When the substitution effect is stronger when $\theta<1$, then the distortion in the wage rate decreases labor supply. 
The neglected contribution to the externality by households in the competitive equilibrium makes hours inefficiently high irrespective of parameter values. 
Hence, with $\theta<1$ the overall distortion in hours worked is mitigated has households the stronger substitution effect offsets part of the inefficient high labor supply due to the neglect of the externality. ¸

In the model, it does so by exactly the same amount as hours contribute to the externality. 
than compensated for by the neglect of the negative effect of hours on the externality due to the concave curvature of the utility function, $\theta>0$. If $\theta=0$, then the two effects would exactly offset. 



\subsection{Proof: lump-sum transfers restore the efficient allocation}
\begin{proof}\label{pr:lst_eff}
	To establish that lump-sum transfers of environmental tax revenues restore the efficient allocation, I first derive the size of lump-sum transfers which implement the efficient level of hours worked given that the efficient dirty labor share is established by choice of the environmental tax. In a second step, I show that this level of transfers coincides with the revenues from the environmental tax when this is set to implement the efficient dirty labor share. 
	These two steps prove that the efficient amount of hours results from lump-sum transferring environmental tax revenues. Finally, I show that the resulting level of consumption is efficient. This completes the proof.
	
	
	
	\textbf{Step 1:} Solve for transfers which implement efficient level of hours\\
	I equalize equation \ref{eq:heff} and \ref{eq:hopt} setting the income tax progressivity, $\tau_{\iota}$, to zero.
	\begin{align}
	\left(\frac{w^{1-\theta}\left(1+\frac{T^*}{wh}\right)^{-\theta}}{\chi}\right)^\frac{1}{\sigma+\theta}=\left(\frac{w_{FB}^{1-\theta}}{\chi}\frac{1-\varepsilon}{1-s}\right)^\frac{1}{\sigma+\theta}.
	\end{align}
	Using the relation of $w_{FB}$ and $w$ established in the previous proof, I can rewrite the right-hand side
	\begin{align}
	&\left(\frac{w^{1-\theta}\left(1+\frac{T^*}{wh}\right)^{-\theta}}{\chi}\right)^\frac{1}{\sigma+\theta}=\left(\frac{w^{1-\theta}}{\chi}\left(\frac{1-s}{1-\varepsilon}\right)^{1-\theta}\frac{1-\varepsilon}{1-s}\right)^\frac{1}{\sigma+\theta}.
	\end{align}
	This step is instructive in showing that transfers will correct for the two inefficiencies in the competitive economy: (i) the too low wage rate captured by the term $\left(\frac{1-s}{1-\varepsilon}\right)^{1-\theta}$, and (ii) the neglect of the effect of hours worked on the externality, captures by $\frac{1-\varepsilon}{1-s}<1$. 
	
	Solving for transfers yields
	\begin{align}
	T^* = \left(\frac{\varepsilon-s}{1-\varepsilon}\right)wh_{FB},
	\end{align}
	
	\textbf{Step 2:} $T^*=\tau_f^*p_fF$\\
	The environmental tax in equilibrium is determined by equation \ref{eq:tauf}: $\tau_f = \frac{\varepsilon-s}{(1-s)\varepsilon}$. Free labor movement enforcing a unique wage rate implies that $p_f=p_g\frac{A_g}{(1-\tau_f)A_f}$. The price for the clean good, $p_g$, in equilibrium, balances clean demand and wages paid: $p_g=\varepsilon^\varepsilon(1-\varepsilon)^{1-\varepsilon}\left(\frac{(1-\tau_f)A_f}{A_g}\right)^\varepsilon$. Dirty output, $F$, is given by $F=A_fsh$. 
	Substituting these expressions in the expression for $T^*$ above and observing that $w=(1-\varepsilon)\left(\frac{A_f}{A_g}\frac{s}{1-s}\right)^\varepsilon A_g$ yields the result.
	
	\textbf{Step 3: } Consumption under $T^*$ is efficient\\
	Trivially, as market clearing has to hold in the competitive equilibrium, it follows that 
	\begin{align}
	C^*=\left(A_f s^*\right)^\varepsilon\left(A_g(1-s^*)\right)^{1-\varepsilon}h^*
	\end{align} 
	Since $T^*$ and $\tau_f^*$ have been set to establish the efficient dirty labor share, $s^*=s_{FB}$ and the efficient level of hours worked, $h^*=h_{FB}$, it follows that $C^*=C_{FB}$. This completes the proof.
\end{proof}

\subsection{Infeasibility of efficient allocation if environmental tax revenues are consumed by the government}

\begin{proof}
	The proof proceeds by construction. First, I assume that the efficient dirty labor share has been implemented and that consumption equals the efficient level. Solving for the competitive level of hours shows that they exceed the efficient level of hours.  Hence, the efficient allocation is not feasible when the government consumes environmental tax revenues since work effort has to be inefficiently high to sustain the first-best level of consumption. 
	
	Working hours to support the efficient level of consumption are given by the market clearing condition  (which holds true with and without income tax scheme)
	\begin{align}
	C_{FB} = \left(A_f s_{FB}\right)^\varepsilon\left(A_g(1-s_{FB})\right)^{1-\varepsilon}h-\tau_f^*p_f^*A_fs_{FB}h
	\end{align}
	Since $s^*=s_{FB}$ by assumption, substituting equilibrium expressions for $\tau_f^*$ and $p_f^*$ used in proof \ref{pr:lst_eff} and solving for $h$ yields
	\begin{align}
	h=\frac{C_{FB}}{w}.
	\end{align}
	Substitution of consumption from the first best allocation, $C_{FB}=\left(A_f s_{FB}\right)^\varepsilon\left(A_g(1-s_{FB})\right)^{1-\varepsilon}h_{FB}$, gives
	\begin{align}
	\frac{h}{h_{FB}}=\frac{w_{FB}}{w}.
	\end{align}
	Since $\varepsilon>s$ the right-hand side, $\frac{w_{FB}}{w}=\frac{1-s}{1-\varepsilon}$, is above unity. Hence, $h>h_{FB}$. 
\end{proof}

\subsection{Proof: redistributing environmental tax revenues through the non-linear income tax scheme restores the efficient allocation. The income tax scheme to support the efficient allocation is progressive.}

\begin{proof}
	Hours under the non-linear tax-scheme policy become
	\begin{align}
	h=\left(\frac{(1-\tau_{\iota})w^{1-\theta}(1+\tau_f p_f\frac{F}{hw})^{1-\theta}}{\chi}\right)^\frac{1}{\sigma+\theta}.
	\end{align}
	
	I assume that the dirty labor share is set to the efficient level, $s=s_{FB}$. This determines $\tau^*_f$ and $p^*_f$.
	It has to be shown that
	\begin{align}
	\left(\frac{(1-\tau_{\iota})w^{1-\theta}\left(1+\tau^*_f p^*_f\frac{F_{FB}}{h_{FB}w}\right)^{1-\theta}}{\chi}\right)^\frac{1}{\sigma+\theta}=\left(\frac{w^{1-\theta}}{\chi}\left(\frac{1-s}{1-\varepsilon}\right)^{1-\theta}\frac{1-\varepsilon}{1-s}\right)^\frac{1}{\sigma+\theta}
	\end{align}
	From proof \ref{pr:lst_eff} step 2 we know that $\frac{\tau_f^*p_fF}{(wh_{FB})}=\left(\frac{\varepsilon-s}{1-\varepsilon}\right)$ and hence $\left(1+\frac{\tau_f^*p_fF}{wh_{FB}}\right)^{-\theta}=\left(\frac{1-s}{1-\varepsilon}\right)^{-\theta}$ and above condition simplifies to
	\begin{align}
	(1-\tau_{\iota})\frac{\varepsilon-s}{1-\varepsilon}=1.
	\end{align}
	Rearranging terms yields
	\begin{align}
	\tau_{\iota}= \frac{\varepsilon-s}{1-s}.
	\end{align}
	Since $\frac{\varepsilon-s}{1-s}\in(0,1)$ when there is a negative externality from dirty production, the optimal tax scheme, which implements the efficient allocation, exists and is progressive. \textit{Note that 1 is an upper bound on the tax scheme progressivity parameter as otherwise the marginal returns to labor would be decreasing in hours worked. It is positive since $s<\varepsilon$.}
\end{proof}

\begin{comment}
\subsection{Numeric results in simple model}
\begin{table}[h!!]
	\caption{Linear tax scheme and lump-sum transfers}\label{tab:lin_lst}
	\begin{tabular}{lllllllll}
		Thetaa & FB hours & FB Pigou & CE hours & CE scc & Opt hours & Opt taul & Opt tauf & Opt scc \\ 
		\hline 
		<1 & 1.192 & 0.99326 & 1.192 & 0.99326 & 1.192 & -3.7748e-15 & 0.99326 & 0.99326 \\ 
		Bop & 0.13601 & 0.99959 & 0.13601 & 0.99959 & 0.13601 & -3.7748e-15 & 0.99959 & 0.99959 \\ 
		log & 0.36434 & 0.99853 & 0.36434 & 0.99853 & 0.36434 & -3.7748e-15 & 0.99853 & 0.99853 \\ 
		\hline 
	\end{tabular}
\end{table}
\begin{table}
	\caption{Linear tax scheme, env. tax revenues not transferred lump-sum}\label{tab:lin_nolst}
	\begin{tabular}{lllllllll}
		Thetaa & FB hours & FB Pigou & CE hours & CE scc & Opt hours & Opt taul & Opt tauf & Opt scc \\ 
		\hline 
		<1 & 1.192 & 0.99326 & 1.2061 & 0.97804 & 1.1706 & 0.049876 & 0.9934 & 0.94584 \\ 
		Bop & 0.13601 & 0.99959 & 0.14026 & 0.96001 & 0.13808 & 0.049876 & 0.99958 & 0.94766 \\ 
		log & 0.36434 & 0.99853 & 0.37243 & 0.97015 & 0.36435 & 0.049876 & 0.99853 & 0.94804 \\ 
		\hline 
	\end{tabular}
\end{table}
\begin{table}[h!!]
	\caption{Baseline model env. revenues transferred via income tax scheme ($\lambda$)}\label{tab:base}
	\begin{tabular}{lllllllll}
		Thetaa & FB hours & FB Pigou & CE hours & CE scc & Opt hours & Opt taul & Opt tauf & Opt scc \\ 
		\hline 
		<1 & 1.192 & 0.99326 & 1.2275 & 1.0056 & 1.192 & 0.049979 & 0.99326 & 0.99326 \\ 
		Bop & 0.13601 & 0.99959 & 0.13811 & 1.0311 & 0.13601 & 0.049979 & 0.99959 & 0.99959 \\ 
		log & 0.36434 & 0.99853 & 0.37243 & 1.0211 & 0.36434 & 0.049979 & 0.99853 & 0.99853 \\ 
		\hline 
	\end{tabular}
\end{table}



Table 1 to 3 compare the efficient allocation to an allocation resulting in the competitive equilibrium when the environmental tax is set to equal the social cost of carbon in the efficient allocation. The rationale being that without any further distortions setting environmental taxes to the social cost of carbon implements the efficient allocation. The last four columns of each table show hours worked, the optimal policy and the social cost of carbon in equilibrium resulting in the Ramsey planner allocation. 

Table \ref{tab:lin_lst} reveals that indeed, setting the corrective tax equal to the social cost of carbon under the social planner implements the first-best allocation when lump-sum transfers are available. The optimal policy chooses zero income taxes. 

The picture changes once no lump-sum transfers are available, compare table \ref{tab:lin_nolst}. In the competitive equilibrium setting the environmental tax to the social costs of carbon under the social planner results in inefficiently high hours worked for all values of $\theta$ considered; compare the columns showing the allocation resulting in the competitive equilibrium when only the efficient dirty share is implemented. 
Theoretically, the labor income tax can be used to establish the 
efficient level of hours worked given that the dirty labor share is efficient. However, since
environmental tax revenues are not redistributed lump-sum, household consumption is lower than under the social planner and the efficient level of hours worked and the efficient dirty labor share feature a lower social welfare in the competitive equilibrium. In other words, a further reduction in labor is too costly in terms of consumption and the optimal labor tax is lower than what would implement efficient hours. \textit{This might change when the household derives utility from government consumption.}

The optimal policy is to set a positive income tax rate; the optimal income tax code is progressive. When the substitution effect outweighs the income effect, i.e., $\theta<1$, then the optimal allocation results in inefficiently \textit{low} hours worked. When the income effect is at least as strong than the substitution effect, that is $\theta\geq 1$, then hours worked remain inefficiently high under the optimal policy. 

Interestingly, when the planner transfers environmental tax revenues through the income tax scheme, table \ref{tab:base}, then the efficient allocation is attainable for all values of $\theta$ considered through a progressive tax scheme. 

Only when the Ramsey planner can implement the efficient level of work, the environmental tax is set to equal the social cost of carbon.   




\textbf{In a nutshell}
\begin{itemize}
	\item hours worked without transfers are always too low even if efficient tax rate is chosen
	\item when hours are not efficient, then the environmental tax does not match the social cost of carbon
	\item when revenues are transferred through the income tax, the planner can implement the efficient allocation with the help of a progressive income tax \textit{(interesting!)}
	\item with $\theta<\frac{\varepsilon}{\varepsilon-s}$ optimal hours worked reduce, otherwise the income effect is too strong and hours worked increase! 
	Nevertheless, the allocation in LF without lump sum transfers always features too high hours worked. 
	\item why does the optimal policy with taul but no lump-sum transfers not implement the efficient level? \ar income taxes are not a measure to implement the efficient allocation; only similar when income and substitution effect cancel. Too high when income effect dominates, too low when substitution effect dominates.
	\ar general consumption tax should neither be able to implement efficient allocation! 
	%\item when there is no income tax, the optimal policy is to set the efficient dirty labour share (compare table \ref{tab:lin_nolst_notaul}). Labor supply is always too high but the optimal tax exceeds the social cost of carbon
\end{itemize}

\end{comment}
%\begin{comment}


%	content...
%\end{comment}

\section{Quantitative model}\label{app:quant_mod}
\subsection{Model equations}
\vspace{-2mm}
%Assume interior solutions for labor and research supply.
\begin{align}
\text{\textbf{Household}}&\nonumber\\ \text{period utility}\vspace{4mm}&  \log(C_t)-\chi\frac{z_hh_{ht}^{1+\sigma}+(1-z_h)h_{lt}^{1+\sigma}}{1+\sigma}-\chi_s\frac{S_t^{1+\sigma}}{{1+\sigma}} \nonumber %when z is also to the power of 1+sigma than, the higher zh the lower hours supplied! Not reasonable
\\
\text{Budget}\ \vspace{4mm}& C_t=z_h \lambda_t (w_{ht}h_{ht})^{1-\tau_{\iota t}}+(1-z_h) \lambda_t (w_{lt}h_{lt})^{1-\tau_{\iota t}}+T_{lst}\nonumber\\ %\\
\text{Optimality}\ \vspace{4mm}
& C_t^{-1}= \mu_tp_{t}\nonumber\\
& \chi h_{ht}^{\sigma}=\mu_t \lambda_t(1-\tau_{\iota t})w_{ht}^{1-\tau_{\iota t}}h_{ht}^{-\tau_{\iota t}}-\gamma_{ht}/z_h\nonumber\\
& \chi h_{lt}^{\sigma}=\mu_t \lambda_t(1-\tau_{\iota t})w_{lt}^{1-\tau_{\iota t}}h_{lt}^{-\tau_{\iota t}}-\gamma_{lt}/(1-z_h)\nonumber\\
%&( h_{st})^{\sigma}=\mu_t \lambda_t(1-\tau_{\iota t})w_{st}^{1-\tau_{\iota t}}h_{st}^{-\tau_{\iota t}}\\
%\Rightarrow\ \ & \frac{h_{ht}}{h_{lt}}=\left(\frac{w_{ht}}{w_{lt}}\right)^{\frac{1-\tau_{\iota t}}{{\sigma+\tau_{\iota t}}}}\ \text{(Interior solution)}\\
& \chi_s S_{t}^{\sigma_s} =\mu_t w_{st}-\gamma_{st} \nonumber\\ %\text{(scientist income confiscated by gov.)}\\
& \text{where $\gamma_{ht}, \gamma_{lt}$, and $\gamma_{sJt}$ are Lagrange multipliers}\nonumber \\ 
& \text{on the inequality constraints}\nonumber \\ 
& \text{with respect to time endowment.}\nonumber\\
\text{\textbf{Final good and Energy producers}}&\nonumber\\
\text{Optimality}\ \vspace{4mm}&
%\frac{E_t}{N_t}=\frac{\delta_y}{(1-\delta_y)}\left(\frac{p_{Nt}}{p_{Et}}\right)^{\varepsilon_y}\\
 p_{t}\delta_y^\frac{1}{\varepsilon_y}Y_t^\frac{1}{\varepsilon_y}E_t^{-\frac{1}{\varepsilon_y}}=p_{Et}\nonumber\\
& p_{t}(1-\delta_y)^\frac{1}{\varepsilon_y}Y_t^\frac{1}{\varepsilon_y}N_t^{-\frac{1}{\varepsilon_y}}=p_{Nt}\nonumber\\
&
p_{Et}E_t^\frac{1}{\varepsilon_e}{F}_t^{-\frac{1}{\varepsilon_e}}=p_{{F}t}+\tau_{Ft}\nonumber\\
& p_{Et}E_t^\frac{1}{\varepsilon_e}G_t^{-\frac{1}{\varepsilon_e}}=p_{Gt}\nonumber\\
%& p_{\tilde{F}t}\tilde{F}_t^\frac{1}{\varepsilon_f}\deltA_fF_t^{-\frac{1}{\varepsilon_f}}=p_{Ft}+\tau_{F}\\
%& p_{\tilde{F}t}\tilde{F}_t^\frac{1}{\varepsilon_f}(1-\deltA_f)O_t^{-\frac{1}{\varepsilon_f}}=p_{Ot}+\tau_{o}
%\\
\text{Definitions prices}\ \vspace{4mm}&
p_{t}= \left[\delta_yp_{Et}^{1-\varepsilon_y}+(1-\delta_y)p_{Nt}^{1-\varepsilon_y}\right]^\frac{1}{1-\varepsilon_y}\nonumber\\
& p_{Et}= \left[(p_{{F}t}+\tau_{Ft})^{1-\varepsilon_e}+p_{Gt}^{1-\varepsilon_y}\right]^\frac{1}{1-\varepsilon_e}\nonumber\\
%& p_{\tilde{F}t}= \left[\deltA_f^{\varepsilon_f}(p_{Ft}+\tau_{F})^{1-\varepsilon_f}+(1-\deltA_f)^{\varepsilon_f}(p_{nt}+\tau_{o})^{1-\varepsilon_f}\right]^\frac{1}{1-\varepsilon_f}
%\\
\text{Production}\ \vspace{4mm}& 
Y_t=\left(\delta_y^\frac{1}{\varepsilon_y}E_t^\frac{\varepsilon_y-1}{\varepsilon_y}+(1-\delta_y)^\frac{1}{\varepsilon_y}N_t^\frac{\varepsilon_y-1}{\varepsilon_y}\right)^\frac{\varepsilon_y}{\varepsilon_y-1}\label{mod:y}\\
&E_t=\left({F}_t^\frac{\varepsilon_e-1}{\varepsilon_e}+G_t^\frac{\varepsilon_e-1}{\varepsilon_e}\right)^\frac{\varepsilon_e}{\varepsilon_e-1}\label{mod:E}\\
%&\tilde{F}_t=\left(\deltA_fF_t^\frac{\varepsilon_f-1}{\varepsilon_f}+(1-\deltA_f)O_t^\frac{\varepsilon_f-1}{\varepsilon_f}\right)^\frac{\varepsilon_f}{\varepsilon_f-1}\\
\text{\textbf{Intermediate good producers}}&\nonumber\\
\text{Production}\ \vspace{4mm}& F_t= x_{Ft}^{\alpha_F}\left(A_{Ft}L_{Ft}\right)^{1-\alpha_F}% (\alpha_F(p_{Ft}(1-\tau_{Ft})))^\frac{ \alpha_F}{1-\alpha_F}A_{Ft}L_{Ft}
\label{mod:F}\\
&N_t= x_{Nt}^{\alpha_N}\left(A_{Nt}L_{Nt}\right)^{1-\alpha_N}%(\alpha_Np_{Nt})^\frac{ \alpha_N}{1-\alpha_N}A_{Nt}L_{Nt}
\label{mod:N}\\
&G_t=  x_{Gt}^{\alpha_G}\left(A_{Gt}L_{Gt}\right)^{1-\alpha_G} %(\alpha_Gp_{Gt})^\frac{ \alpha_G}{1-\alpha_G}A_{Gt}L_{Gt}
\label{mod:G}\\
\text{Labour demand}\ \vspace{4mm}%\label{eq:lab_demand}\\
& w_{lFt}=(p_{Ft})^\frac{1}{1-\alpha_F}(1-\alpha_F)(\alpha_F)^\frac{\alpha_F}{1-\alpha_F}A_{Ft}\nonumber\\
& w_{lNt}=p_{Nt}^\frac{1}{1-\alpha_N}(1-\alpha_N)(\alpha_N)^\frac{\alpha_N}{1-\alpha_N}A_{Nt}\nonumber\\
& w_{lGt}=(p_{Gt})^\frac{1}{1-\alpha_G}(1-\alpha_G)(\alpha_G)^\frac{\alpha_G}{1-\alpha_G}A_{Gt}\nonumber
\\
\text{Machine demand}\ \vspace{4mm}
&x_{Fit}= \left(\alpha_F p_{Ft}\right)^\frac{1}{1-\alpha_F}L_{Ft}A_{Fit}\nonumber\\
&x_{Nit}= \left(\alpha_N p_{Nt}\right)^\frac{1}{1-\alpha_N}L_{Nt}A_{Nit}\nonumber\\
&x_{Git}= \left(\alpha_G p_{Gt}\right)^\frac{1}{1-\alpha_G}L_{Gt}A_{Git}\nonumber
\\
\text{\textbf{Labour producers}}&\nonumber
\\
\text{Production}\ \vspace{4mm}& L_{Ft}=h_{hFt}^{\theta_{F}}h_{lFt}^{1-\theta_{F}}\label{mod:LF}\\
& L_{Nt}=h_{hNt}^{\theta_{N}}h_{lNt}^{1-\theta_{N}}\label{mod:LN}\\
& L_{Gt}=h_{hGt}^{\theta_{g}}h_{lGt}^{1-\theta_{G}}\label{mod:LG}\\
\text{Optimality}\ \vspace{4mm}& h_{hFt}= \theta_{F}L_{Ft}\frac{w_{lFt}}{w_{ht}}\nonumber\\ %\label{eq:opt_lab_pro}\\
& h_{hNt}= \theta_{N}L_{Nt}\frac{w_{lNt}}{w_{ht}}\nonumber\\
& h_{hGt}= \theta_{G}L_{Gt}\frac{w_{lGt}}{w_{ht}}\nonumber\\
& h_{lFt}= (1-\theta_{F})L_{Ft}\frac{w_{lFt}}{w_{lt}}\nonumber\\%\label{eq:opt_lab_pro_low}\\
& h_{lNt}= (1-\theta_{N}) L_{Nt}\frac{w_{lNt}}{w_{lt}}\nonumber\\
& h_{lGt}= (1-\theta_{G}) L_{Gt}\frac{w_{lGt}}{w_{lt}}\nonumber\\
%\end{align}
%
%\begin{align}
\text{\textbf{Machine producers}}\nonumber\\
\text{Price setting}\ \vspace{4mm}&p_{xFit}=\frac{1}{\alpha_F(1+\zeta_F)}\nonumber\\
&p_{xNit}=\frac{1}{\alpha_N(1+\zeta_N)}\nonumber\\
&p_{xGit}=\frac{1}{\alpha_G(1+\zeta_G)}
\nonumber\\ 
\text{Demand Scientists}\ \vspace{4mm}&
w_{st}=\frac{\eta \gamma A_{Ft-1}^{1-\phi}A_{t-1}^{\phi}\left(\frac{s_{Ft}}{\rho_F}\right)^{\eta}p_{Ft}F_t}{\frac{1}{1-\alpha_F}s_{Ft}A_{Ft}}\nonumber\\
&
w_{st}=\frac{\eta \gamma  A_{Nt-1}^{1-\phi}A_{t-1}^{\phi}\left(\frac{S_{Nt}}{\rho_N}\right)^{\eta}p_{Nt}N_t}{\frac{1}{1-\alpha_N}s_{Nt}A_{Nt}}\nonumber\\
&
w_{st}=\frac{\eta \gamma  A_{Gt-1}^{1-\phi}A_{t-1}^{\phi}\left(\frac{s_{gt}}{\rho_G}\right)^{\eta}p_{Gt}G_t}{\frac{1}{1-\alpha_G}s_{Gt}A_{Gt}}\nonumber\\
\text{Innovation}\ \vspace{4mm}&
A_{Fit}=A_{Ft-1}\left(1+\gamma\left(\frac{s_{Fit}}{\rho_F}\right)^{\eta}\left(\frac{A_{t-1}}{A_{Ft-1}}\right)^{\phi}\right)\nonumber \\
&
A_{Nit}=A_{Nt-1}\left(1+\gamma\left(\frac{s_{Nit}}{\rho_N}\right)^{\eta}\left(\frac{A_{t-1}}{A_{Nt-1}}\right)^{\phi}\right)\nonumber \\
&
A_{Git}=A_{Gt-1}\left(1+\gamma\left(\frac{s_{Git}}{\rho_G}\right)^{\eta}\left(\frac{A_{t-1}}{A_{Gt-1}}\right)^{\phi}\right)\nonumber \\
%\text{Demand Scientists}\ \vspace{4mm}&
%w_{sft}=\frac{\eta \gamma \alpha_F A_{ft-1}\left(\frac{S_{ft}}{\rho_f}\right)^{\eta}p_{Ft}F_t}{\frac{1}{1-\alpha_F}S_{ft}A_{ft}}\label{eq:demand_sc}\\
%&
%w_{snt}=\frac{\eta \gamma \alpha_N A_{nt-1}\left(\frac{S_{nt}}{\rho_n}\right)^{\eta}p_{Nt}N_t}{\frac{1}{1-\alpha_N}S_{nt}A_{nt}}\\
%&
%w_{sgt}=\frac{\eta \gamma \alpha_G A_{gt-1}\left(\frac{S_{gt}}{\rho_g}\right)^{\eta}p_{Gt}G_t}{\frac{1}{1-\alpha_G}S_{gt}A_{gt}}\\
%\text{Innovation}\ \vspace{4mm}&
%A_{fit}=A_{ft-1}\left(1+\gamma\left(\frac{S_{fit}}{\rho_f}\right)^{\eta}\right) \\
%&
%A_{nit}=A_{nt-1}\left(1+\gamma\left(\frac{S_{nit}}{\rho_n}\right)^{\eta}\right) \\
%&
%A_{git}=A_{gt-1}\left(1+\gamma\left(\frac{S_{git}}{\rho_g}\right)^{\eta}\right) \\
%&A_t=\max\{A_{nt}, A_{ft}, A_{gt}\}\\`
%DROP THE FOLLOWING AS IT DOES NOT GROW AT A CONSTANT RATE IF SHARES ARE CHANGING! &A_t= \frac{\rho_fA_{ft}+\rho_gA_{gt}+\rho_nA_{nt}}{\rho_f+\rho_n+\rho_g}\\
\text{\textbf{Government}}&\nonumber\\
&0=z_h(w_{ht}h_{ht}-\lambda_t(w_{ht}h_{ht})^{(1-\tau_{\iota t})}) +(1-z_h)(w_{lt}h_{lt}-\lambda_t(w_{lt}h_{lt})^{(1-\tau_{\iota t})})\nonumber\\
&T_{\pi t}= -\int_{0}^{1} \left(p_{xFit}\zeta_{Ft}x_{Fit}+p_{xGit}\zeta_{Gt}x_{Git}+ p_{xNit}\zeta_{Nt}x_{Nit}\right)di\nonumber\\ &\hspace{10mm}+\int_{0}^{1}\left(\pi_{Fit}+\pi_{Git}+\pi_{Nit}\right)di = -w_{st}S_t\nonumber\\
\text{with}\ \vspace{4mm}&\zeta_{Jt}=\frac{1-\alpha_J}{\alpha_J}\ for J\in\{F,G,N\}\nonumber \\
& T_{lst}=\tau_{Ft}F_t\nonumber\\
%\text{simplified:}\ \vspace{4mm} & Gov_{\iota t}= z_h(w_{ht}h_{ht}-\lambda_t(w_{ht}h_{ht})^{(1-\tau_{\iota t})})\nonumber\\&+(1-z_h)(w_{lt}h_{lt}-\lambda_t(w_{lt}h_{lt})^{(1-\tau_{\iota t})}) +\tau_{Ft}F_t
%\nonumber\\
%&Gov_{Ft}=\tau_{Ft}F_t\hspace{2mm} or\hspace{2mm} \tau_{Gt}p_{Gt}G_t=\tau_{Ft}F_t \hspace{2mm} or \hspace{2mm} T_{ls, t}=\tau_{Ft}F_t\hspace{2mm} or \hspace{2mm} T_{\iota t}=\tau_{Ft}F_t\nonumber \\
\text{\textbf{Markets}}&\nonumber\\
&h_{hFt}+h_{hNt}+h_{hGt}=z_{h} h_{ht}\nonumber\\
&h_{lFt}+h_{lNt}+h_{lGt}=(1-z_h) h_{lt}\nonumber\\
&s_{Ft}+s_{Nt}+s_{Gt}=S_{t}\nonumber\\
&C_{t}+\int_{0}^{1} \left(x_{Fit}+x_{Nit}+x_{Git}\right)d_i=Y_t\nonumber
\end{align}

\subsection{Numerical appendix}\label{app:PV}

Since I cannot solve explicitly for the optimal policy over an infinite horizon, I truncate the problem after period $T$. 
In the literature, utility in periods after $T$ are approximated under the assumption that policy variables are fixed, and the economy reaches a balanced growth path \citep{Barrage2019OptimalPolicy, Jones1993OptimalGrowth}. However, assuming a constant carbon tax would most likely violate the emission limit since the model is designed to reflect market forces describing an economy with green and fossil sectors operating in equilibrium. 


I motivate the design of the continuation value by pretending the planner would hand over the economy to a successor after period $T$. A continuation value, $PV$, in the objective function captures that the planner cares about utility after period $T$. 
This set-up accounts for concerns about economic well-being of future generations in a similar vein than the sustainability criterion proposed by the \cite{UNSUS} by attaching some value to the final technology level.\footnote{\ The sustainable development criterion reads "\textit{[...] to ensure that it meets the needs of the present without comprising the ability of future generations to meet their own needs.
	}" (p.24). This is a vague definition.  \cite{Dasgupta2021} p.(332) interprets this criterion as meaning: 
	"\textit{[...] each generation should bequeath to its successor at least as large a productive base as it had inherited from its predecessor. }". 
	However, this cannot be used to derive a sensible condition on the optimization in the present setting, since there is no negative growth and technology is the only asset bequeathed to future generations. Thus,
	successors will always have at least as much productive resources as predecessors left. The relation to the future is instead approximated by a future potential to derive utility from consumption given bequeathed technology levels. Natural needs of the future are accounted for through the emission limit. } I approximate the value of future technology levels by assuming constant growth rates.  
To mitigate concerns that the choice of the continuation value drives the results, I experiment with the exact value of explicit optimization periods. I truncate the problem once explicitly adding a further period leaves the optimal allocation numerically unchanged. That is the case after $T=42$, or 210 years. %Then the 
The planner's objective function becomes: 
\begin{align*}
\underset{\{\tau_{Ft}\}_{t=0}^{\infty},\{\tau_{\iota t}\}_{t=0}^{\infty}}{\max}&\sum_{t=0}^{\infty}\beta^t u(C_{t}, h_{ht}, h_{lt}, S_t)
+PV.
\end{align*}

In more detail, I define the continuation value as the consumption utility over the infinite horizon starting from the last explicit maximization period:
\begin{align*}
PV=\sum_{s=T+1}^{\infty} \beta^{s}u(C_s, h_{hs}, h_{ls}, S_{s}).
\end{align*}
I make three simplifying assumptions to substitute the infinite sum in the optimization problem. First, 
I assume that the private consumption share in period $c_s$, with $C_s=c_sY_s$, is constant from period $T+1$ onward.  Then, consumption grows at the same rate as output. 
Second, as an approximation to future growth, I assume the economy grows at the same rate as in the last explicit optimization period. Third, hours of workers and scientists remain at their value in the last explicit optimization period. %\footnote{\ Note that I am not making the assumption that the future economy is best described by the continuation value; it rather serves as a proxy to measure the value of passed technology levels. } 
Let $\gamma_{yT}=\frac{Y_{T}}{Y_{T-1}}-1$. Under above assumptions, I can rewrite future consumption as $C_s=(1+\gamma_{yT})^{s-T}C_{T}$.
Given the functional form
\begin{align*}
u(C_s)= \frac{C_s^{1-\theta}}{1-\theta},
\end{align*}
the continuation value reduces to
\begin{align*}
PV= \beta^{T}\left(\frac{1}{1-\beta (1+\gamma_{yT})^{1-\theta}}\frac{C_{T}^{1-\theta}}{1-\theta}+ \frac{1}{1-\beta}u(h_{hT}, h_{lT}, S_T)\right).
\end{align*}
Where $u(C_s, h_{hs}, h_{ls}, S_{s})=u(c_s)+u(h_{hT}, h_{lT}, S_T)$.
When $\theta\underset{\lim}{\rightarrow} 1$,  the first summand in brackets becomes
\begin{align*}
\frac{1}{1-\beta}\log(C_{T}).
\end{align*}

%\tr{in the version with externality in utility function, the continuation value is only with respect to consumption and hours of workers and scientists. }
%

\subsection{Social planner allocation}\label{app:sp_prob} The solution to the social planner's problem is defined as an allocation \\ $\{h_{hFt}, h_{hGt}, h_{hNt}, h_{lFt}, h_{lGt}, h_{lNt}, x_{Nt}, x_{Ft}, x_{Gt}, C_t,  h_{ht}, h_{lt}, s_{Ft}, s_{Gt}, s_{Nt} \}$ for each period which maximizes social welfare, $\sum_{t=0}^{T}\beta^t u(C_{t}, h_{ht}, h_{lt}, S_{t})+ CV$, subject to the emission limit and feasibility. 
%\begin{align*} &\sum_{t=0}^{T}\beta^t u(C_{t}, h_{ht}, h_{lt}, S_{t})+ CV\\
%s.t.\ &  \omega F_{t} -\delta \leq \Omega_t,\\
%&C_t+x_{Ft}+x_{Gt}+x_{Nt}=Y_t,\\
%&\text{Law of Motion of technologies}\\
%&z_h h_{ht}=h_{hFt}+h_{hGt}+h_{hNt},\\
%&(1-z_h) h_{lt}=h_{lFt}+h_{lGt}+h_{lNt},\\
%&S_{t}=s_{Ft}+s_{Gt}+s_{Nt},\\
%&\text{time endowment scientists and workers}\\
%\end{align*}
%Production of $Y_t$ is defined by the eq.s describing production in the model: eq. \eqref{mod:y} to \ref{mod:LG}. 
It holds that $x_{Jt}=\int_{0}^{1}x_{Jit}di$.

\section{Quantitative results}\label{app:quant_res}

\subsection{Results: Policy experiments}\label{app:polexp}


\subsubsection{A constant carbon tax}\label{app:polexp_cc}

%Figure \ref{fig:Efftaul_nsk0_xgr0_know_app} demonstrates the effect of a progressive income tax of $\tau_{\iota}=0.181$ when a constant carbon tax of 185\$ per ton of CO$_2$ is levied on fossil purchases. 
%The compositional effect of a progressive income tax originates from the asymmetric reaction of high- and low-skill workers. High skill workers reduce their labor supply more, as the after-tax wage declines in response to a higher tax progressivity. Since they profit from less leisure initially, they require a higher wage rate to be compensated for the marginal unit of labor due to the concavity of the utility function in leisure. \tr{The effect of a marginal increase in income tax progressivity intensifies with the level of pre-tax income.}
%
%Since the green sector relies more heavily on high-skill labor, the green-specific input good becomes more expensive. Similarly, the non-energy good uses a comparably lower share of high-skill labor, and its production becomes cheaper. Therefore, energy producers demand more fossil goods and final good producers demand more non-energy goods; consider panels (f) and (g). The shift in demand by final good producers is smaller because their ability to substitute between energy and non-energy goods is more limited.  
%
%Research responds to the shift in demand. First, non-energy research becomes less profitable, since the price of non-energy goods falls while the amount of non-energy goods does not rise sufficiently. As a result, research is directed towards the energy sector. The share of non-energy researchers reduces, albeit minimally; see panel (e). Focusing solely on the allocation of researchers between the fossil and green sector, the relatively higher supply of low-skill labor rises the market size of the fossil good. Since intermediate energy goods are sufficiently substitutable, research shifts from green to the fossil (panel (a)). Nevertheless, due to the reallocation of research to energy goods in general, the green sector, too, sees a rise in researchers (panel (b)). Overall, the share of green to fossil research rises (panel (d)). The reallocation of research towards fossil goods intensifies the rise in green-to-fossil energy. The reallocation of research towards the energy sector diminishes the decline in the energy share. 
%
%The dynamics of the effect of progressive income taxes arise from a continuously intensified reduction in the high-to-low-skill ratio (panel (i)). It occurs despite a rising skill premium and is present in a model with exogenous growth, no knowledge spillovers, and equal labor shares.
%The rationale behind this observation is that a constant carbon tax entails a transition to green production, which increases demand for high-skill labor. The higher hours of high skill workers make the same rise in income tax progressivity more effective. 
%
%The carbon tax induces a transition to green energy until growth in the green sector does no longer change the marginal productivity of labor across sectors and the economy is on a balanced growth path. 

\begin{figure}[h!!]
	\centering
	\caption{Effect of a constant carbon tax: additional variables }\label{fig:Leveltauf_nsk0_xgr0_add}
	\begin{minipage}[]{0.32\textwidth}
		\centering{\footnotesize{(a) Fossil scientists }}
		%	\captionsetup{width=.45\linewidth}
		\includegraphics[width=1\textwidth]{../../codding_model/own_basedOnFried/optimalPol_010922_revision/figures/all_13Sept22/PerdifNoTauf_regime0_CompTaul_sff_spillover0_nsk0_xgr0_knspil0_sep0_LFlimit0_emsbase0_countec0_GovRev0_etaa0.79_lgd1.png}
	\end{minipage}
	\begin{minipage}[]{0.32\textwidth}
		\centering{\footnotesize{(b) Green scientists }}
		%	\captionsetup{width=.45\linewidth}
		\includegraphics[width=1\textwidth]{../../codding_model/own_basedOnFried/optimalPol_010922_revision/figures/all_13Sept22/PerdifNoTauf_regime0_CompTaul_sg_spillover0_nsk0_xgr0_knspil0_sep0_LFlimit0_emsbase0_countec0_GovRev0_etaa0.79_lgd0.png}
	\end{minipage}
\begin{minipage}[]{0.32\textwidth}
\centering{\footnotesize{(c) Non-energy scientists }}
%	\captionsetup{width=.45\linewidth}
\includegraphics[width=1\textwidth]{../../codding_model/own_basedOnFried/optimalPol_010922_revision/figures/all_13Sept22/PerdifNoTauf_regime0_CompTaul_sn_spillover0_nsk0_xgr0_knspil0_sep0_LFlimit0_emsbase0_countec0_GovRev0_etaa0.79_lgd0.png}
\end{minipage}	
\begin{minipage}[]{0.32\textwidth}
	\centering{\footnotesize{(d) Non-energy scientists share}}
	%	\captionsetup{width=.45\linewidth}
	\includegraphics[width=1\textwidth]{../../codding_model/own_basedOnFried/optimalPol_010922_revision/figures/all_13Sept22/PerdifNoTauf_regime0_CompTaul_snS_spillover0_nsk0_xgr0_knspil0_sep0_LFlimit0_emsbase0_countec0_GovRev0_etaa0.79_lgd0.png}
\end{minipage}
\begin{minipage}[]{0.32\textwidth}
\centering{\footnotesize{(e) Green-to-fossil ratio }}
%	\captionsetup{width=.45\linewidth}
\includegraphics[width=1\textwidth]{../../codding_model/own_basedOnFried/optimalPol_010922_revision/figures/all_13Sept22/PerdifNoTauf_regime0_CompTaul_GFF_spillover0_nsk0_xgr0_knspil0_sep0_LFlimit0_emsbase0_countec0_GovRev0_etaa0.79_lgd0.png}
\end{minipage}
\begin{minipage}[]{0.32\textwidth}
\centering{\footnotesize{(f) High-to-low skill ratio }}
%	\captionsetup{width=.45\linewidth}
\includegraphics[width=1\textwidth]{../../codding_model/own_basedOnFried/optimalPol_010922_revision/figures/all_13Sept22/PerdifNoTauf_regime0_CompTaul_hhhl_spillover0_nsk0_xgr0_knspil0_sep0_LFlimit0_emsbase0_countec0_GovRev0_etaa0.79_lgd0.png}
\end{minipage}
\begin{minipage}[]{0.32\textwidth}
	\centering{\footnotesize{(g) High-to-low-skill ratio in levels, $\tau_{\iota}=0.181$}}
	%	\captionsetup{width=.45\linewidth}
	\includegraphics[width=1\textwidth]{../../codding_model/own_basedOnFried/optimalPol_010922_revision/figures/all_13Sept22/LevTaufNoTauf_TaulCalib_regime0_hhhl_spillover0_nsk0_xgr0_knspil0_sep0_LFlimit0_emsbase0_countec0_GovRev0_etaa0.79_lgd1.png}
\end{minipage}
\begin{minipage}[]{0.32\textwidth}
\centering{\footnotesize{(h) Green-to-fossil ratio in levels, $\tau_{\iota}=0.181$}}
%	\captionsetup{width=.45\linewidth}
\includegraphics[width=1\textwidth]{../../codding_model/own_basedOnFried/optimalPol_010922_revision/figures/all_13Sept22/LevTaufNoTauf_TaulCalib_regime0_GFF_spillover0_nsk0_xgr0_knspil0_sep0_LFlimit0_emsbase0_countec0_GovRev0_etaa0.79_lgd0.png}
\end{minipage}
\floatfoot{Notes:{ The graphs show the percentage difference between the business-as-usual policy without carbon tax and with carbon tax in the economy with and without progressive income tax by the solid and dash graphs, respectively.  Except for panels (g) and (h) which show variables in levels with labor income tax progressivity fixed at $\tau_{\iota}=0.181$ with and without the constant carbon tax by the solid and dashed graphs, respectively. }}
\end{figure}

\clearpage
\thispagestyle{empty}
 \begin{figure}[h!!]
	\centering
	\caption{Effect of a constant carbon tax in model variations }\label{fig:Leveltauf_nsk0_xgr0_noknow}	
	\begin{subfigure}{0.75\textwidth}
		\caption{No knowledge spillovers}
	\begin{minipage}[]{0.32\textwidth}
	\centering{\footnotesize{(a) Net emissions}}
	%	\captionsetup{width=.45\linewidth}
	\includegraphics[width=1\textwidth]{../../codding_model/own_basedOnFried/optimalPol_010922_revision/figures/all_13Sept22/CompTauf_bytaul_Reg0_Emnet_spillover0_nsk0_xgr0_knspil1_sep0_LFlimit0_emsbase0_countec0_GovRev0_etaa0.79_lgd1.png}
\end{minipage}	
\begin{minipage}[]{0.32\textwidth}
	\centering{\footnotesize{(b) Fossil }}
	%	\captionsetup{width=.45\linewidth}
	\includegraphics[width=1\textwidth]{../../codding_model/own_basedOnFried/optimalPol_010922_revision/figures/all_13Sept22/PerdifNoTauf_regime0_CompTaul_F_spillover0_nsk0_xgr0_knspil1_sep0_LFlimit0_emsbase0_countec0_GovRev0_etaa0.79_lgd1.png}
\end{minipage}
\begin{minipage}[]{0.32\textwidth}
	\centering{\footnotesize{(c) Energy share to GDP}}
	%	\captionsetup{width=.45\linewidth}
	\includegraphics[width=1\textwidth]{../../codding_model/own_basedOnFried/optimalPol_010922_revision/figures/all_13Sept22/PerdifNoTauf_regime0_CompTaul_EY_spillover0_nsk0_xgr0_knspil1_sep0_LFlimit0_emsbase0_countec0_GovRev0_etaa0.79_lgd0.png}
\end{minipage}
\begin{minipage}[]{0.32\textwidth}
\centering{\footnotesize{(d) Non-energy scientists share}}
%	\captionsetup{width=.45\linewidth}
\includegraphics[width=1\textwidth]{../../codding_model/own_basedOnFried/optimalPol_010922_revision/figures/all_13Sept22/PerdifNoTauf_regime0_CompTaul_snS_spillover0_nsk0_xgr0_knspil1_sep0_LFlimit0_emsbase0_countec0_GovRev0_etaa0.79_lgd0.png}
\end{minipage}
\begin{minipage}[]{0.32\textwidth}
\centering{\footnotesize{(e) Non-energy scientists \\ \  }}
%	\captionsetup{width=.45\linewidth}
\includegraphics[width=1\textwidth]{../../codding_model/own_basedOnFried/optimalPol_010922_revision/figures/all_13Sept22/PerdifNoTauf_regime0_CompTaul_sn_spillover0_nsk0_xgr0_knspil1_sep0_LFlimit0_emsbase0_countec0_GovRev0_etaa0.79_lgd0.png}
\end{minipage}
	\end{subfigure}
		
		\begin{subfigure}{0.75\textwidth}
			\caption{Equal labor shares }
			\begin{minipage}[]{0.32\textwidth}
				\centering{\footnotesize{(a) Net emissions}}
				%	\captionsetup{width=.45\linewidth}
				\includegraphics[width=1\textwidth]{../../codding_model/own_basedOnFried/optimalPol_010922_revision/figures/all_13Sept22/CompTauf_bytaul_Equlab_Reg0_Emnet_spillover0_nsk0_xgr0_knspil0_sep0_LFlimit0_emsbase0_countec0_GovRev0_etaa0.79_lgd1.png}
			\end{minipage}	
			\begin{minipage}[]{0.32\textwidth}
				\centering{\footnotesize{(b) Fossil }}
				%	\captionsetup{width=.45\linewidth}
				\includegraphics[width=1\textwidth]{../../codding_model/own_basedOnFried/optimalPol_010922_revision/figures/all_13Sept22/PerdifNoTauf_Equlab_regime0_CompTaul_F_spillover0_nsk0_xgr0_knspil0_sep0_LFlimit0_emsbase0_countec0_GovRev0_etaa0.79_lgd1.png}
			\end{minipage}
			\begin{minipage}[]{0.32\textwidth}
				\centering{\footnotesize{(c) Energy share to GDP}}
				%	\captionsetup{width=.45\linewidth}
				\includegraphics[width=1\textwidth]{../../codding_model/own_basedOnFried/optimalPol_010922_revision/figures/all_13Sept22/PerdifNoTauf_Equlab_regime0_CompTaul_EY_spillover0_nsk0_xgr0_knspil0_sep0_LFlimit0_emsbase0_countec0_GovRev0_etaa0.79_lgd0.png}
			\end{minipage}
		\begin{minipage}[]{0.32\textwidth}
		\centering{\footnotesize{(d) Non-energy scientists share}}
		%	\captionsetup{width=.45\linewidth}
		\includegraphics[width=1\textwidth]{../../codding_model/own_basedOnFried/optimalPol_010922_revision/figures/all_13Sept22/PerdifNoTauf_Equlab_regime0_CompTaul_snS_spillover0_nsk0_xgr0_knspil0_sep0_LFlimit0_emsbase0_countec0_GovRev0_etaa0.79_lgd0.png}
	\end{minipage}
\begin{minipage}[]{0.32\textwidth}
\centering{\footnotesize{(e) Non-energy scientists \\ \ }}
%	\captionsetup{width=.45\linewidth}
\includegraphics[width=1\textwidth]{../../codding_model/own_basedOnFried/optimalPol_010922_revision/figures/all_13Sept22/PerdifNoTauf_Equlab_regime0_CompTaul_sn_spillover0_nsk0_xgr0_knspil0_sep0_LFlimit0_emsbase0_countec0_GovRev0_etaa0.79_lgd0.png}
\end{minipage}
		\end{subfigure} 
		
	\begin{subfigure}{0.75\textwidth}
	\caption{Equal labor shares and no knowledge spillovers}
	\begin{minipage}[]{0.32\textwidth}
		\centering{\footnotesize{(a) Net emissions}}
		%	\captionsetup{width=.45\linewidth}
		\includegraphics[width=1\textwidth]{../../codding_model/own_basedOnFried/optimalPol_010922_revision/figures/all_13Sept22/CompTauf_bytaul_Equlab_Reg0_Emnet_spillover0_nsk0_xgr0_knspil1_sep0_LFlimit0_emsbase0_countec0_GovRev0_etaa0.79_lgd1.png}
	\end{minipage}	
	\begin{minipage}[]{0.32\textwidth}
		\centering{\footnotesize{(b) Fossil }}
		%	\captionsetup{width=.45\linewidth}
		\includegraphics[width=1\textwidth]{../../codding_model/own_basedOnFried/optimalPol_010922_revision/figures/all_13Sept22/PerdifNoTauf_Equlab_regime0_CompTaul_F_spillover0_nsk0_xgr0_knspil1_sep0_LFlimit0_emsbase0_countec0_GovRev0_etaa0.79_lgd1.png}
	\end{minipage}
	\begin{minipage}[]{0.32\textwidth}
		\centering{\footnotesize{(c) Energy share to GDP}}
		%	\captionsetup{width=.45\linewidth}
		\includegraphics[width=1\textwidth]{../../codding_model/own_basedOnFried/optimalPol_010922_revision/figures/all_13Sept22/PerdifNoTauf_Equlab_regime0_CompTaul_EY_spillover0_nsk0_xgr0_knspil1_sep0_LFlimit0_emsbase0_countec0_GovRev0_etaa0.79_lgd0.png}
	\end{minipage}
\begin{minipage}[]{0.32\textwidth}
\centering{\footnotesize{(d) Non-energy scientists share}}
%	\captionsetup{width=.45\linewidth}
\includegraphics[width=1\textwidth]{../../codding_model/own_basedOnFried/optimalPol_010922_revision/figures/all_13Sept22/PerdifNoTauf_Equlab_regime0_CompTaul_snS_spillover0_nsk0_xgr0_knspil1_sep0_LFlimit0_emsbase0_countec0_GovRev0_etaa0.79_lgd0.png}
\end{minipage}
\begin{minipage}[]{0.32\textwidth}
\centering{\footnotesize{(e) Non-energy scientists \\ \ }}
%	\captionsetup{width=.45\linewidth}
\includegraphics[width=1\textwidth]{../../codding_model/own_basedOnFried/optimalPol_010922_revision/figures/all_13Sept22/PerdifNoTauf_Equlab_regime0_CompTaul_sn_spillover0_nsk0_xgr0_knspil1_sep0_LFlimit0_emsbase0_countec0_GovRev0_etaa0.79_lgd0.png}
\end{minipage}
\end{subfigure}		
\end{figure} 
\clearpage

\begin{figure}[h!!]
	\centering
	\caption{Effect of a progressive income tax, $\tau_{\iota t}=0.181$, with a constant carbon tax}		\label{fig:Efftaul_nsk0_xgr0_know_app}		
	\begin{minipage}[]{0.32\textwidth}
		\centering{\footnotesize{(a) Fossil research}}
		%	\captionsetup{width=.45\linewidth}
		\includegraphics[width=1\textwidth]{../../codding_model/own_basedOnFried/optimalPol_010922_revision/figures/all_13Sept22/CompTaufPER_bytaul_Reg0_sff_spillover0_nsk0_xgr0_knspil0_sep0_LFlimit0_emsbase0_countec0_GovRev0_etaa0.79_lgd0.png}
	\end{minipage}	
	\begin{minipage}[]{0.32\textwidth}
		\centering{\footnotesize{(b) Green research}}
		%	\captionsetup{width=.45\linewidth}
		\includegraphics[width=1\textwidth]{../../codding_model/own_basedOnFried/optimalPol_010922_revision/figures/all_13Sept22/CompTaufPER_bytaul_Reg0_sg_spillover0_nsk0_xgr0_knspil0_sep0_LFlimit0_emsbase0_countec0_GovRev0_etaa0.79_lgd0.png}
		%CompTaul_LFBAUPer_Reg0_sg_spillover0_nsk0_xgr0_knspil0_sep0_countec0_GovRev0_etaa0.79
	\end{minipage}
	\begin{minipage}[]{0.32\textwidth}
		\centering{\footnotesize{(c) Non-energy research share}}
		%	\captionsetup{width=.45\linewidth}
		\includegraphics[width=1\textwidth]{../../codding_model/own_basedOnFried/optimalPol_010922_revision/figures/all_13Sept22/CompTaufPER_bytaul_Reg0_snS_spillover0_nsk0_xgr0_knspil0_sep0_LFlimit0_emsbase0_countec0_GovRev0_etaa0.79_lgd0.png}
	\end{minipage}
	\begin{minipage}[]{0.32\textwidth}
		\centering{\footnotesize{(d) Green-to-fossil research}}
		%	\captionsetup{width=.45\linewidth}
		\includegraphics[width=1\textwidth]{../../codding_model/own_basedOnFried/optimalPol_010922_revision/figures/all_13Sept22/CompTaufPER_bytaul_Reg0_sgsff_spillover0_nsk0_xgr0_knspil0_sep0_LFlimit0_emsbase0_countec0_GovRev0_etaa0.79_lgd0.png}
	\end{minipage}
	\begin{minipage}[]{0.32\textwidth}
		\centering{\footnotesize{(e)Share non-energy scientists}}
		%	\captionsetup{width=.45\linewidth}
		\includegraphics[width=1\textwidth]{../../codding_model/own_basedOnFried/optimalPol_010922_revision/figures/all_13Sept22/CompTaufPER_bytaul_Reg0_snS_spillover0_nsk0_xgr0_knspil0_sep0_LFlimit0_emsbase0_countec0_GovRev0_etaa0.79_lgd0.png}
	\end{minipage}
	%	\begin{minipage}[]{0.32\textwidth}
	%		\centering{\footnotesize{(f) Green-to-fossil output}}
	%		%	\captionsetup{width=.45\linewidth}
	%		\includegraphics[width=1\textwidth]{../../codding_model/own_basedOnFried/optimalPol_010922_revision/figures/all_13Sept22/CompTaufPER_bytaul_Reg0_GFF_spillover0_nsk0_xgr0_knspil0_sep0_LFlimit0_emsbase0_countec0_GovRev0_etaa0.79_lgd0.png}
	%	\end{minipage}
	%	\begin{minipage}[]{0.32\textwidth}
	%		\centering{\footnotesize{(g)Energy to GDP}}
	%		%	\captionsetup{width=.45\linewidth}
	%		\includegraphics[width=1\textwidth]{../../codding_model/own_basedOnFried/optimalPol_010922_revision/figures/all_13Sept22/CompTaufPER_bytaul_Reg0_EY_spillover0_nsk0_xgr0_knspil0_sep0_LFlimit0_emsbase0_countec0_GovRev0_etaa0.79_lgd0.png}
	%	\end{minipage}
	\begin{minipage}[]{0.32\textwidth}
		\centering{\footnotesize{(f) Non-energy goods}}
		%	\captionsetup{width=.45\linewidth}
		\includegraphics[width=1\textwidth]{../../codding_model/own_basedOnFried/optimalPol_010922_revision/figures/all_13Sept22/CompTaufPER_bytaul_Reg0_N_spillover0_nsk0_xgr0_knspil0_sep0_LFlimit0_emsbase0_countec0_GovRev0_etaa0.79_lgd0.png}
	\end{minipage}
	%\begin{minipage}[]{0.32\textwidth}
	%	\centering{\footnotesize{(i) High-to- low-skill ratio}}
	%	%	\captionsetup{width=.45\linewidth}
	%	\includegraphics[width=1\textwidth]{../../codding_model/own_basedOnFried/optimalPol_010922_revision/figures/all_13Sept22/CompTaufPER_bytaul_Reg0_hhhl_spillover0_nsk0_xgr0_knspil0_sep0_LFlimit0_emsbase0_countec0_GovRev0_etaa0.79_lgd0.png}
	%\end{minipage}
	%\begin{minipage}[]{0.32\textwidth}
	%	\centering{\footnotesize{(j) High-skill hours in levels}}
	%	%	\captionsetup{width=.45\linewidth}
	%	\includegraphics[width=1\textwidth]{../../codding_model/own_basedOnFried/optimalPol_010922_revision/figures/all_13Sept22/LevTaufNoTauf_TaulCalib_Equlab_regime0_hh_spillover0_nsk0_xgr1_knspil1_sep0_LFlimit0_emsbase0_countec0_GovRev0_etaa0.79_lgd1.png}
	%\end{minipage}	
\end{figure}


\clearpage
\subsubsection{Necessary carbon tax to meet emission limist}\label{app:neccab}
 \begin{figure}[h!!]
	\centering
	\caption{Effect of carbon tax relative to BAU with and without progressive income tax }\label{fig:Limit_nsk0_xgr0_know_app}		
	\begin{minipage}[]{0.32\textwidth}
		\centering{\footnotesize{(a) Green growth}}
		%	\captionsetup{width=.45\linewidth}
		\includegraphics[width=1\textwidth]{../../codding_model/own_basedOnFried/optimalPol_010922_revision/figures/all_13Sept22/CompTauf_bytaul_Reg0_gAg_spillover0_nsk0_xgr0_knspil0_sep0_LFlimit1_emsbase0_countec0_GovRev0_etaa0.79_lgd0.png}
	\end{minipage}	
	\begin{minipage}[]{0.32\textwidth}
		\centering{\footnotesize{(b) Fossil research }}
	%	%	\captionsetup{width=.45\linewidth}
		\includegraphics[width=1\textwidth]{../../codding_model/own_basedOnFried/optimalPol_010922_revision/figures/all_13Sept22/CompTauf_bytaul_Reg0_sff_spillover0_nsk0_xgr0_knspil0_sep0_LFlimit1_emsbase0_countec0_GovRev0_etaa0.79_lgd1.png}
	\end{minipage}	
	\begin{minipage}[]{0.32\textwidth}
		\centering{\footnotesize{(c) Green research}}
		%	\captionsetup{width=.45\linewidth}
		\includegraphics[width=1\textwidth]{../../codding_model/own_basedOnFried/optimalPol_010922_revision/figures/all_13Sept22/CompTauf_bytaul_Reg0_sg_spillover0_nsk0_xgr0_knspil0_sep0_LFlimit1_emsbase0_countec0_GovRev0_etaa0.79_lgd1.png}
	\end{minipage}			
	\begin{minipage}[]{0.32\textwidth}
		\centering{\footnotesize{(d) Non-energy research}}
		%	\captionsetup{width=.45\linewidth}
		\includegraphics[width=1\textwidth]{../../codding_model/own_basedOnFried/optimalPol_010922_revision/figures/all_13Sept22/CompTauf_bytaul_Reg0_sn_spillover0_nsk0_xgr0_knspil0_sep0_LFlimit1_emsbase0_countec0_GovRev0_etaa0.79_lgd1.png}
	\end{minipage}	
	\begin{minipage}[]{0.32\textwidth}
		\centering{\footnotesize{(e) Deviation Green-to-fossil ratio}}
		%	\captionsetup{width=.45\linewidth}
		\includegraphics[width=1\textwidth]{../../codding_model/own_basedOnFried/optimalPol_010922_revision/figures/all_13Sept22/CompTaufPER_bytaul_Reg0_GFF_spillover0_nsk0_xgr0_knspil0_sep0_LFlimit1_emsbase0_countec0_GovRev0_etaa0.79_lgd0.png}
	\end{minipage}		
	\begin{minipage}[]{0.32\textwidth}
		\centering{\footnotesize{(f) Deviation energy to GDP}}
		%	\captionsetup{width=.45\linewidth}
		\includegraphics[width=1\textwidth]{../../codding_model/own_basedOnFried/optimalPol_010922_revision/figures/all_13Sept22/CompTaufPER_bytaul_Reg0_EY_spillover0_nsk0_xgr0_knspil0_sep0_LFlimit1_emsbase0_countec0_GovRev0_etaa0.79_lgd0.png}
	\end{minipage}	
\floatfoot{Notes: \footnotesize{Panels (a) to (d) show allocations in levels under the necessary carbon tax by preexisting tax progressivity. 
Panels (e) to (f) show deviations in allocations induced by the joint policy, i.e. carbon tax and progressive income tax, relative to the scenario without progressive income tax.}}
\end{figure} 

\begin{figure}[h!!]
	\centering
	\caption{\footnotesize{ Deviation in carbon tax: no knowledge spillovers, equal labor shares, homogeneous skills, and exogenous growth}}\label{fig:zeromod_tauf}
	\begin{minipage}[]{0.32\textwidth}
		\centering
		%\caption{\footnotesize{ Deviation in carbon tax: no knowledge spillovers, equal labor shares, homogeneous skills, and exogenous growth}}
		%%	\captionsetup{width=.45\linewidth}
		\includegraphics[width=1\textwidth]{../../codding_model/own_basedOnFried/optimalPol_010922_revision/figures/all_13Sept22/CompTaufPER_bytaul_Equlab_Reg0_tauf_spillover0_nsk1_xgr1_knspil1_sep0_LFlimit1_emsbase0_countec0_GovRev0_etaa0.79_lgd0.png} \end{minipage}		
\end{figure} 


\subsubsection{Effect of model features on the necessary carbon tax}\label{app:eff_feat_exp}

\paragraph{Role of heterogeneous labor shares}
Heterogeneous capital shares increase the necessary fossil tax to meet emission limits since a positive supply side effect is muted. 
With equal capital shares, the necessary fossil tax is almost half as high as in the benchmark model during the net-zero emission period. 
As the fossil sector lowers demand for labor, the green sector profits from a higher labor supply, yet, less so under a lower labor share.  

With equal labor shares the fossil tax in presence of a progressive income tax is, however, closer to the required fossil tax absent progressive labor taxation.  The reason is that the compositional effect of the  progressive labor tax on skill supply is less devastating for green production when the green sector relies less on labor. Then, a bigger reduction in the fossil tax is possible while meeting emission limits. 


\begin{figure}[h!!]
	\centering
	\caption{Necessary carbon tax with and without progressive income tax  }\label{fig:Limit_nsk0_xgr0_eual}	
	\begin{subfigure}{0.7\textwidth}
		\caption{Equal labor shares}
		\begin{minipage}[]{0.45\textwidth}
			\centering{\footnotesize{(a) Carbon tax}}
			%	\captionsetup{width=.45\linewidth}
			\includegraphics[width=1\textwidth]{../../codding_model/own_basedOnFried/optimalPol_010922_revision/figures/all_13Sept22/CompTauf_bytaul_Equlab_Reg0_tauf_spillover0_nsk0_xgr0_knspil0_sep0_LFlimit1_emsbase0_countec0_GovRev0_etaa0.79_lgd1.png}
		\end{minipage}	
		\begin{minipage}[]{0.45\textwidth}
			\centering{\footnotesize{(b) Deviation in carbon tax}}
			%	\captionsetup{width=.45\linewidth}
			\includegraphics[width=1\textwidth]{../../codding_model/own_basedOnFried/optimalPol_010922_revision/figures/all_13Sept22/CompTaufPER_bytaul_Equlab_Reg0_tauf_spillover0_nsk0_xgr0_knspil0_sep0_LFlimit1_emsbase0_countec0_GovRev0_etaa0.79_lgd0.png} 
		\end{minipage}	
	\end{subfigure}
	
	\begin{subfigure}{0.7\textwidth}
		\caption{Homogeneous skills}
		\begin{minipage}[]{0.45\textwidth}
			\centering{\footnotesize{(a) Carbon tax}}
			%	\captionsetup{width=.45\linewidth}
			\includegraphics[width=1\textwidth]{../../codding_model/own_basedOnFried/optimalPol_010922_revision/figures/all_13Sept22/CompTauf_bytaul_Reg0_tauf_spillover0_nsk1_xgr0_knspil0_sep0_LFlimit1_emsbase0_countec0_GovRev0_etaa0.79_lgd1.png}
		\end{minipage}	
		\begin{minipage}[]{0.45\textwidth}
			\centering{\footnotesize{(b) Deviation in carbon tax}}
			%	\captionsetup{width=.45\linewidth}
			\includegraphics[width=1\textwidth]{../../codding_model/own_basedOnFried/optimalPol_010922_revision/figures/all_13Sept22/CompTaufPER_bytaul_Reg0_tauf_spillover0_nsk1_xgr0_knspil0_sep0_LFlimit1_emsbase0_countec0_GovRev0_etaa0.79_lgd0.png} 
		\end{minipage}	
	\end{subfigure}		
	
	\begin{subfigure}{0.7\textwidth}
		\caption{Exogenous growth}
		\begin{minipage}[]{0.45\textwidth}
			\centering{\footnotesize{(a) Carbon tax}}
			%	\captionsetup{width=.45\linewidth}
			\includegraphics[width=1\textwidth]{../../codding_model/own_basedOnFried/optimalPol_010922_revision/figures/all_13Sept22/CompTauf_bytaul_Reg0_tauf_spillover0_nsk1_xgr0_knspil0_sep0_LFlimit1_emsbase0_countec0_GovRev0_etaa0.79_lgd1.png}
		\end{minipage}	
		\begin{minipage}[]{0.45\textwidth}
			\centering{\footnotesize{(b) Deviation in carbon tax}}
			%	\captionsetup{width=.45\linewidth}
			\includegraphics[width=1\textwidth]{../../codding_model/own_basedOnFried/optimalPol_010922_revision/figures/all_13Sept22/CompTaufPER_bytaul_Reg0_tauf_spillover0_nsk0_xgr1_knspil0_sep0_LFlimit1_emsbase0_countec0_GovRev0_etaa0.79_lgd0.png} 
		\end{minipage}	
	\end{subfigure}

\begin{subfigure}{0.7\textwidth}
	\caption{Equal labor shares and no knowledge spillovers}
	\begin{minipage}[]{0.45\textwidth}
		\centering{\footnotesize{(a) Carbon tax}}
		%	\captionsetup{width=.45\linewidth}
		\includegraphics[width=1\textwidth]{../../codding_model/own_basedOnFried/optimalPol_010922_revision/figures/all_13Sept22/CompTauf_bytaul_Equlab_Reg0_tauf_spillover0_nsk0_xgr0_knspil1_sep0_LFlimit1_emsbase0_countec0_GovRev0_etaa0.79_lgd1.png}
	\end{minipage}	
	\begin{minipage}[]{0.45\textwidth}
		\centering{\footnotesize{(b) Deviation in carbon tax}}
		%	\captionsetup{width=.45\linewidth}
		\includegraphics[width=1\textwidth]{../../codding_model/own_basedOnFried/optimalPol_010922_revision/figures/all_13Sept22/CompTaufPER_bytaul_Equlab_Reg0_tauf_spillover0_nsk0_xgr0_knspil1_sep0_LFlimit1_emsbase0_countec0_GovRev0_etaa0.79_lgd0.png} 
	\end{minipage}	
\end{subfigure}
	
\end{figure}
\paragraph{Role of heterogeneous skills}

The required fossil tax to meet emission limits is higher when skills are homogeneous. The reason is that the resources for fossil production increase.
When skills are homogeneous, labor income tax progressivity has no compositional effect on the economic structure. One might expect that, therefore, that a stronger reduction in the fossil tax is admissible. However, the reduction in the carbon tax is smaller when skills are homogeneous. 

One explanation is that with homogeneous skills, similar to heterogeneous capital shares, there is no supply side effect triggered by the fossil tax. Instead, labor supply is unaffected. With heterogeneous skills, however, the fossil tax has a compositional effect on labor supply which makes green production less costly. A reduction in the fossil tax does not harm the green to fossil ratio as much when skills are heterogeneous. 

\paragraph{Role of endogenous growth}
In line with \cite{Fried2018ClimateAnalysis}, I find that a smaller fossil tax is required when growth is endogenous because the shift in research intensifies the effect of the fossil tax.  




%\subsubsection{Optimal net emissions in model variations}
%\begin{figure}[h!!]
%	\centering
%	\caption{Optimal Policy }\label{fig:EmsSP}
%	\begin{minipage}[]{0.32\textwidth}
%		\centering{\footnotesize{(a) Benchmark model}}
%		%	\captionsetup{width=.45\linewidth}
%		\includegraphics[width=1\textwidth]{../../codding_model/own_basedOnFried/optimalPol_010922_revision/figures/all_13Sept22_Tplus30/Emnet_CompEff_Target_onlyeff_spillover0_knspil0_noskill0_sep0_xgrowth0_countec0_PV1_etaa0.79_lgd0.png}
%	\end{minipage}
%	\begin{minipage}[]{0.32\textwidth}
%		\centering{\footnotesize{(b)Exogenous growth}}
%		%	\captionsetup{width=.45\linewidth}
%		\includegraphics[width=1\textwidth]{../../codding_model/own_basedOnFried/optimalPol_010922_revision/figures/all_13Sept22_Tplus30/Emnet_CompEff_Target_onlyeff_spillover0_knspil0_noskill0_sep0_xgrowth1_countec0_PV1_etaa0.79_lgd0.png}
%	\end{minipage}
%	\begin{minipage}[]{0.32\textwidth}
%		\centering{\footnotesize{(c) No knowledge spillovers}}
%		%	\captionsetup{width=.45\linewidth}
%		\includegraphics[width=1\textwidth]{../../codding_model/own_basedOnFried/optimalPol_010922_revision/figures/all_13Sept22_Tplus30/Emnet_CompEff_Target_onlyeff_spillover0_knspil1_noskill0_sep0_xgrowth0_countec0_PV1_etaa0.79_lgd0.png}
%	\end{minipage}
%\begin{minipage}[]{0.32\textwidth}
%\centering{\footnotesize{(c) No knowledge spillovers, exogenous growth}}
%%	\captionsetup{width=.45\linewidth}
%\includegraphics[width=1\textwidth]{../../codding_model/own_basedOnFried/optimalPol_010922_revision/figures/all_13Sept22_Tplus30/Emnet_CompEff_Target_onlyeff_spillover0_knspil1_noskill0_sep0_xgrowth1_countec0_PV1_etaa0.79_lgd0.png}
%\end{minipage}
%\end{figure} 

\clearpage
\subsection{Effect of policy regime}\label{app:pol_regimes}

The analytical section has highlighted the importance of how environmental tax revenues are used. In the main part, I focus on the policy regime where environmental tax revenues are redistributed through the income tax scheme; the government runs a consolidated budget. 
The present section is dedicated to discussing the necessary carbon tax under alternative redistribution systems: lump-sum redistribution and recycling environmental tax revenues as subsidies to the green sector. 

The progressive income tax is kept constant at its calibrated value, $\tau_{\iota}=0.181$. Figure \ref{fig:regs} depicts the results.
Using environmental tax revenues to subsidize the green sector achieves the highest green-to-fossil energy ratio at the smallest carbon tax, panels (a) and (b). Interestingly, non-energy growth is also highest in the scenario with green subsidies. The reason is that the energy to non-energy price is smaller due to a lower carbon tax. On the one hand, this implies a higher energy share in final production, panel (c). On the other hand, this directs research to non-energy goods, and non-energy growth follows a higher trajectory. In sum, when carbon tax revenues are used to subsidize green producers, the economy can attain the emission limit at a higher energy share but at an advantageous green-to-fossil energy ratio. 

Hours worked of both skill types are similar in the model under the consolidated budget and the green-subsidy regime. With lump-sum transfers, hours worked are lower for both types. Hence, there does not seem to be a work-increasing effect of green subsidies through a higher labor demand. In fact, the typical labor good of the green sector, high-skill labor, is smaller under the regime with green subsidies than under the consolidated budget one. 

\begin{figure}[h!!]
	\caption{Necessary carbon tax and allocation by policy regime in levels}\label{fig:regs}
	\begin{minipage}[]{0.32\textwidth}
		\centering{\footnotesize{(a) Tax per ton of carbon in \$US}}
		%	\captionsetup{width=.45\linewidth}
		\includegraphics[width=1\textwidth]{../../codding_model/own_basedOnFried/optimalPol_010922_revision/figures/all_13Sept22/CompRed_TaulCalib_Tauf_spillover0_knspil0_nsk0_xgr0_sep0_LFlimit1_emsbase0_countec0_GovRev0_etaa0.79_lgd1.png}
	\end{minipage}
\begin{minipage}[]{0.32\textwidth}
\centering{\footnotesize{(b) Green-to-fossil energy ratio}}
%	\captionsetup{width=.45\linewidth}
\includegraphics[width=1\textwidth]{../../codding_model/own_basedOnFried/optimalPol_010922_revision/figures/all_13Sept22/CompRed_TaulCalib_GFF_spillover0_knspil0_nsk0_xgr0_sep0_LFlimit1_emsbase0_countec0_GovRev0_etaa0.79_lgd0.png}
\end{minipage}
\begin{minipage}[]{0.32\textwidth}
	\centering{\footnotesize{(c) Energy to GDP}}
	%	\captionsetup{width=.45\linewidth}
	\includegraphics[width=1\textwidth]{../../codding_model/own_basedOnFried/optimalPol_010922_revision/figures/all_13Sept22/CompRed_TaulCalib_EY_spillover0_knspil0_nsk0_xgr0_sep0_LFlimit1_emsbase0_countec0_GovRev0_etaa0.79_lgd0.png}
\end{minipage}
\begin{minipage}[]{0.32\textwidth}
	\centering{\footnotesize{(d) Fossil growth}}
	%	\captionsetup{width=.45\linewidth}
	\includegraphics[width=1\textwidth]{../../codding_model/own_basedOnFried/optimalPol_010922_revision/figures/all_13Sept22/CompRed_TaulCalib_gAf_spillover0_knspil0_nsk0_xgr0_sep0_LFlimit1_emsbase0_countec0_GovRev0_etaa0.79_lgd0.png}
\end{minipage}
\begin{minipage}[]{0.32\textwidth}
	\centering{\footnotesize{(e) Non-energy growth}}
	%	\captionsetup{width=.45\linewidth}
	\includegraphics[width=1\textwidth]{../../codding_model/own_basedOnFried/optimalPol_010922_revision/figures/all_13Sept22/CompRed_TaulCalib_gAn_spillover0_knspil0_nsk0_xgr0_sep0_LFlimit1_emsbase0_countec0_GovRev0_etaa0.79_lgd0.png}
\end{minipage}
\begin{minipage}[]{0.32\textwidth}
\centering{\footnotesize{(f) Green growth}}
%	\captionsetup{width=.45\linewidth}
\includegraphics[width=1\textwidth]{../../codding_model/own_basedOnFried/optimalPol_010922_revision/figures/all_13Sept22/CompRed_TaulCalib_gAg_spillover0_knspil0_nsk0_xgr0_sep0_LFlimit1_emsbase0_countec0_GovRev0_etaa0.79_lgd0.png}
\end{minipage}
\begin{minipage}[]{0.32\textwidth}
\centering{\footnotesize{(g) High-skill hours worked}}
%	\captionsetup{width=.45\linewidth}
\includegraphics[width=1\textwidth]{../../codding_model/own_basedOnFried/optimalPol_010922_revision/figures/all_13Sept22/CompRed_TaulCalib_hh_spillover0_knspil0_nsk0_xgr0_sep0_LFlimit1_emsbase0_countec0_GovRev0_etaa0.79_lgd0.png}
\end{minipage}
\begin{minipage}[]{0.32\textwidth}
\centering{\footnotesize{(h) Low-skill hours worked}}
%	\captionsetup{width=.45\linewidth}
\includegraphics[width=1\textwidth]{../../codding_model/own_basedOnFried/optimalPol_010922_revision/figures/all_13Sept22/CompRed_TaulCalib_hl_spillover0_knspil0_nsk0_xgr0_sep0_LFlimit1_emsbase0_countec0_GovRev0_etaa0.79_lgd0.png}
\end{minipage}
%
%\begin{subfigure}{0.9\textwidth}
%	\centering
%	\caption{ $\tau_\iota=0$}
%	\begin{minipage}[]{0.32\textwidth}
%		\centering{\footnotesize{Carbon tax}}
%		%	\captionsetup{width=.45\linewidth}
%		\includegraphics[width=1\textwidth]{../../codding_model/own_basedOnFried/optimalPol_010922_revision/figures/all_13Sept22/CompRed_Taul0_Tauf_spillover0_knspil0_nsk0_xgr0_sep0_LFlimit1_emsbase0_countec0_GovRev0_etaa0.79_lgd1.png}
%	\end{minipage}
%	\begin{minipage}[]{0.32\textwidth}
%		\centering{\footnotesize{Green-to-fossil energy ratio}}
%		%	\captionsetup{width=.45\linewidth}
%		\includegraphics[width=1\textwidth]{../../codding_model/own_basedOnFried/optimalPol_010922_revision/figures/all_13Sept22/CompRed_Taul0_GFF_spillover0_knspil0_nsk0_xgr0_sep0_LFlimit1_emsbase0_countec0_GovRev0_etaa0.79_lgd0.png}
%	\end{minipage}
%	\begin{minipage}[]{0.32\textwidth}
%		\centering{\footnotesize{Fossil growth}}
%		%	\captionsetup{width=.45\linewidth}
%		\includegraphics[width=1\textwidth]{../../codding_model/own_basedOnFried/optimalPol_010922_revision/figures/all_13Sept22/CompRed_Taul0_gAf_spillover0_knspil0_nsk0_xgr0_sep0_LFlimit1_emsbase0_countec0_GovRev0_etaa0.79_lgd0.png}
%	\end{minipage}
%	\begin{minipage}[]{0.32\textwidth}
%		\centering{\footnotesize{Non-energy growth}}
%		%	\captionsetup{width=.45\linewidth}
%		\includegraphics[width=1\textwidth]{../../codding_model/own_basedOnFried/optimalPol_010922_revision/figures/all_13Sept22/CompRed_Taul0_gAn_spillover0_knspil0_nsk0_xgr0_sep0_LFlimit1_emsbase0_countec0_GovRev0_etaa0.79_lgd0.png}
%	\end{minipage}
%	\begin{minipage}[]{0.32\textwidth}
%		\centering{\footnotesize{Green growth}}
%		%	\captionsetup{width=.45\linewidth}
%		\includegraphics[width=1\textwidth]{../../codding_model/own_basedOnFried/optimalPol_010922_revision/figures/all_13Sept22/CompRed_Taul0_gAg_spillover0_knspil0_nsk0_xgr0_sep0_LFlimit1_emsbase0_countec0_GovRev0_etaa0.79_lgd0.png}
%	\end{minipage}
%\end{subfigure}
\end{figure} 


\clearpage



\subsection{Optimal policy}\label{app:quant_res_opt}


\begin{figure}[h!!]
	\centering
	\caption{Allocation in levels }\label{fig:LF}	
	\begin{subfigure}{1\textwidth}		
		\caption{Laissez-faire and optimal allocation}
	\begin{subfigure}[]{0.32\textwidth}
		\centering{\footnotesize{(a) Consumption}}
		%	\captionsetup{width=.45\linewidth}
		\includegraphics[width=1\textwidth]{../../codding_model/own_basedOnFried/optimalPol_010922_revision/figures/all_13Sept22_Tplus30/C_LFCompOPT_T_NoTaus_regime4_spillover0_noskill0_sep0_xgrowth0_PV1_etaa0.79_lgd1.png}
	\end{subfigure}	
	\begin{subfigure}[]{0.32\textwidth}
	\centering{\footnotesize{(b) High-skill hours worked}}
	%	\captionsetup{width=.45\linewidth}
	\includegraphics[width=1\textwidth]{../../codding_model/own_basedOnFried/optimalPol_010922_revision/figures/all_13Sept22_Tplus30/hh_LFCompOPT_T_NoTaus_regime4_spillover0_noskill0_sep0_xgrowth0_PV1_etaa0.79_lgd0.png}
\end{subfigure}	
\begin{subfigure}[]{0.32\textwidth}
\centering{\footnotesize{(c) Low-skill hours worked}}
%	\captionsetup{width=.45\linewidth}
\includegraphics[width=1\textwidth]{../../codding_model/own_basedOnFried/optimalPol_010922_revision/figures/all_13Sept22_Tplus30/hl_LFCompOPT_T_NoTaus_regime4_spillover0_noskill0_sep0_xgrowth0_PV1_etaa0.79_lgd0.png}
\end{subfigure}	
%	\begin{subfigure}[]{0.32\textwidth}
%	\centering{\footnotesize{(d) Fossil scientists}}
%	%	\captionsetup{width=.45\linewidth}
%	\includegraphics[width=1\textwidth]{../../codding_model/own_basedOnFried/optimalPol_010922_revision/figures/all_13Sept22_Tplus30/sff_LFCompOPT_T_NoTaus_regime4_spillover0_noskill0_sep0_xgrowth0_PV1_etaa0.79_lgd1.png}
%\end{subfigure}	
%\begin{subfigure}[]{0.32\textwidth}
%	\centering{\footnotesize{(e) Non-energy scientists}}
%	%	\captionsetup{width=.45\linewidth}
%	\includegraphics[width=1\textwidth]{../../codding_model/own_basedOnFried/optimalPol_010922_revision/figures/all_13Sept22_Tplus30/sn_LFCompOPT_T_NoTaus_regime4_spillover0_noskill0_sep0_xgrowth0_PV1_etaa0.79_lgd0.png}
%\end{subfigure}	
%\begin{subfigure}[]{0.32\textwidth}
%	\centering{\footnotesize{(f) Green scientists}}
%	%	\captionsetup{width=.45\linewidth}
%	\includegraphics[width=1\textwidth]{../../codding_model/own_basedOnFried/optimalPol_010922_revision/figures/all_13Sept22_Tplus30/sg_LFCompOPT_T_NoTaus_regime4_spillover0_noskill0_sep0_xgrowth0_PV1_etaa0.79_lgd0.png}
%\end{subfigure}
	\end{subfigure}	
	\begin{subfigure}{1\textwidth}		
	\caption{Laissez-faire and social planner allocation}
	\begin{subfigure}[]{0.32\textwidth}
		\centering{\footnotesize{(a) Consumption}}
		%	\captionsetup{width=.45\linewidth}
		\includegraphics[width=1\textwidth]{../../codding_model/own_basedOnFried/optimalPol_010922_revision/figures/all_13Sept22_Tplus30/C_LFCompSP_T_regime4_knspil0_spillover0_noskill0_sep0_xgrowth0_PV1_etaa0.79_lgd1.png}
	\end{subfigure}	
	\begin{subfigure}[]{0.32\textwidth}
		\centering{\footnotesize{(b) High-skill hours worked}}
		%	\captionsetup{width=.45\linewidth}
		\includegraphics[width=1\textwidth]{../../codding_model/own_basedOnFried/optimalPol_010922_revision/figures/all_13Sept22_Tplus30/hh_LFCompSP_T_regime4_knspil0_spillover0_noskill0_sep0_xgrowth0_PV1_etaa0.79_lgd0.png}
	\end{subfigure}	
	\begin{subfigure}[]{0.32\textwidth}
		\centering{\footnotesize{(c) Low-skill hours worked}}
		%	\captionsetup{width=.45\linewidth}
		\includegraphics[width=1\textwidth]{../../codding_model/own_basedOnFried/optimalPol_010922_revision/figures/all_13Sept22_Tplus30/hl_LFCompSP_T_regime4_knspil0_spillover0_noskill0_sep0_xgrowth0_PV1_etaa0.79_lgd0.png}
	\end{subfigure}
%	\begin{subfigure}[]{0.32\textwidth}
%	\centering{\footnotesize{(f) Fossil scientists}}
%	%	\captionsetup{width=.45\linewidth}
%	\includegraphics[width=1\textwidth]{../../codding_model/own_basedOnFried/optimalPol_010922_revision/figures/all_13Sept22_Tplus30/sff_LFCompSP_T_regime4_knspil0_spillover0_noskill0_sep0_xgrowth0_PV1_etaa0.79_lgd1.png}
%\end{subfigure}	
%\begin{subfigure}[]{0.32\textwidth}
%	\centering{\footnotesize{(e) Non-energy scientists}}
%	%	\captionsetup{width=.45\linewidth}
%	\includegraphics[width=1\textwidth]{../../codding_model/own_basedOnFried/optimalPol_010922_revision/figures/all_13Sept22_Tplus30/sn_LFCompSP_T_regime4_knspil0_spillover0_noskill0_sep0_xgrowth0_PV1_etaa0.79_lgd0.png}
%\end{subfigure}	
%\begin{subfigure}[]{0.32\textwidth}
%	\centering{\footnotesize{(f) Green scientists}}
%	%	\captionsetup{width=.45\linewidth}
%	\includegraphics[width=1\textwidth]{../../codding_model/own_basedOnFried/optimalPol_010922_revision/figures/all_13Sept22_Tplus30/sg_LFCompSP_T_regime4_knspil0_spillover0_noskill0_sep0_xgrowth0_PV1_etaa0.79_lgd0.png}
%\end{subfigure}	
\end{subfigure}	
\end{figure}
\clearpage
\thispagestyle{empty}
\begin{figure}[h!!!]
	\centering
	\caption{Efficient and optimal allocation in deviation from laissez-faire without knowledge spillovers	}\label{fig:optAll_percLf_dyn_nokn}
	\begin{subfigure}[]{0.4\textwidth}
		\centering{\footnotesize{(a) Consumption\\ \ }}
		%	\captionsetup{width=.45\linewidth}
		\includegraphics[width=1\textwidth]{../../codding_model/own_basedOnFried/optimalPol_010922_revision/figures/all_13Sept22_Tplus30/C_PercentageLFDyn_Target_regime4_knspil1_spillover0_noskill0_sep0_xgrowth0_PV1_etaa0.79_lgd1.png}
	\end{subfigure}\begin{minipage}[]{0.1\textwidth}
	\
\end{minipage}
	\begin{subfigure}[]{0.4\textwidth}
		\centering{\footnotesize{(b) Average hours worked\\ \  }}
		%	\captionsetup{width=.45\linewidth}
		\includegraphics[width=1\textwidth]{../../codding_model/own_basedOnFried/optimalPol_010922_revision/figures/all_13Sept22_Tplus30/Hagg_PercentageLFDyn_Target_regime4_knspil1_spillover0_noskill0_sep0_xgrowth0_PV1_etaa0.79_lgd0.png}
	\end{subfigure}
\begin{minipage}[]{0.1\textwidth}
	\
\end{minipage}
	\begin{subfigure}[]{0.4\textwidth}
		\centering{\footnotesize{\ \\(d) Fossil scientists\\ \ }}
		%	\captionsetup{width=.45\linewidth}
		\includegraphics[width=1\textwidth]{../../codding_model/own_basedOnFried/optimalPol_010922_revision/figures/all_13Sept22_Tplus30/sff_PercentageLFDyn_Target_regime4_knspil1_spillover0_noskill0_sep0_xgrowth0_PV1_etaa0.79_lgd0.png}
	\end{subfigure}
	\begin{subfigure}[]{0.4\textwidth}
		\centering{\footnotesize{\ \\(e) Non-energy scientists\\ \ }}
		%	\captionsetup{width=.45\linewidth}
		\includegraphics[width=1\textwidth]{../../codding_model/own_basedOnFried/optimalPol_010922_revision/figures/all_13Sept22_Tplus30/sn_PercentageLFDyn_Target_regime4_knspil1_spillover0_noskill0_sep0_xgrowth0_PV1_etaa0.79_lgd0.png}
	\end{subfigure}\begin{minipage}[]{0.1\textwidth}
	\
\end{minipage}
	\begin{subfigure}[]{0.4\textwidth}
		\centering{\footnotesize{\ \\(f) Green scientists \\ \   }}
		%	\captionsetup{width=.45\linewidth}
		\includegraphics[width=1\textwidth]{../../codding_model/own_basedOnFried/optimalPol_010922_revision/figures/all_13Sept22_Tplus30/sg_PercentageLFDyn_Target_regime4_knspil1_spillover0_noskill0_sep0_xgrowth0_PV1_etaa0.79_lgd0.png}
	\end{subfigure}
	\begin{subfigure}[]{0.4\textwidth}
		\centering{\footnotesize{\ \\(g) Green-to-fossil output \\ \   }}
		%	\captionsetup{width=.45\linewidth}
		\includegraphics[width=1\textwidth]{../../codding_model/own_basedOnFried/optimalPol_010922_revision/figures/all_13Sept22_Tplus30/GFF_PercentageLFDyn_Target_regime4_knspil1_spillover0_noskill0_sep0_xgrowth0_PV1_etaa0.79_lgd0.png}
	\end{subfigure}\begin{minipage}[]{0.1\textwidth}
	\
\end{minipage}
	\begin{subfigure}[]{0.4\textwidth}
		\centering{\footnotesize{\ \\(h) Energy share in GDP \\ \   }}
		%	\captionsetup{width=.45\linewidth}
		\includegraphics[width=1\textwidth]{../../codding_model/own_basedOnFried/optimalPol_010922_revision/figures/all_13Sept22_Tplus30/EY_PercentageLFDyn_Target_regime4_knspil1_spillover0_noskill0_sep0_xgrowth0_PV1_etaa0.79_lgd0.png}
	\end{subfigure}
\end{figure} 

\clearpage
\begin{figure}[h!!]
	\centering
	\caption{Allocations in levels; no knowledge spillovers }\label{fig:LF_noKN}	
	\begin{subfigure}{1\textwidth}		
		\caption{Laissez-faire and optimal allocation}
		\begin{subfigure}[]{0.32\textwidth}
			\centering{\footnotesize{(a) Consumption}}
			%	\captionsetup{width=.45\linewidth}
			\includegraphics[width=1\textwidth]{../../codding_model/own_basedOnFried/optimalPol_010922_revision/figures/all_13Sept22_Tplus30/C_LFCompOPT_T_NoTaus_regime4_spillover0_noskill0_sep0_xgrowth0_PV1_etaa0.79_lgd1.png}
		\end{subfigure}	
		\begin{subfigure}[]{0.32\textwidth}
			\centering{\footnotesize{(b) High-skill hours worked}}
			%	\captionsetup{width=.45\linewidth}
			\includegraphics[width=1\textwidth]{../../codding_model/own_basedOnFried/optimalPol_010922_revision/figures/all_13Sept22_Tplus30/hh_LFCompOPT_T_NoTaus_regime4_spillover0_noskill0_sep0_xgrowth0_PV1_etaa0.79_lgd0.png}
		\end{subfigure}	
		\begin{subfigure}[]{0.32\textwidth}
			\centering{\footnotesize{(c) Low-skill hours worked}}
			%	\captionsetup{width=.45\linewidth}
			\includegraphics[width=1\textwidth]{../../codding_model/own_basedOnFried/optimalPol_010922_revision/figures/all_13Sept22_Tplus30/hl_LFCompOPT_T_NoTaus_regime4_spillover0_noskill0_sep0_xgrowth0_PV1_etaa0.79_lgd0.png}
		\end{subfigure}	
		\begin{subfigure}[]{0.32\textwidth}
			\centering{\footnotesize{(d) Fossil scientists}}
			%	\captionsetup{width=.45\linewidth}
			\includegraphics[width=1\textwidth]{../../codding_model/own_basedOnFried/optimalPol_010922_revision/figures/all_13Sept22_Tplus30/sff_LFCompOPT_T_NoTaus_regime4_spillover0_noskill0_sep0_xgrowth0_PV1_etaa0.79_lgd1.png}
		\end{subfigure}	
		\begin{subfigure}[]{0.32\textwidth}
			\centering{\footnotesize{(e) Non-energy scientists}}
			%	\captionsetup{width=.45\linewidth}
			\includegraphics[width=1\textwidth]{../../codding_model/own_basedOnFried/optimalPol_010922_revision/figures/all_13Sept22_Tplus30/sn_LFCompOPT_T_NoTaus_regime4_spillover0_noskill0_sep0_xgrowth0_PV1_etaa0.79_lgd0.png}
		\end{subfigure}	
		\begin{subfigure}[]{0.32\textwidth}
			\centering{\footnotesize{(f) Green scientists}}
			%	\captionsetup{width=.45\linewidth}
			\includegraphics[width=1\textwidth]{../../codding_model/own_basedOnFried/optimalPol_010922_revision/figures/all_13Sept22_Tplus30/sg_LFCompOPT_T_NoTaus_regime4_spillover0_noskill0_sep0_xgrowth0_PV1_etaa0.79_lgd0.png}
		\end{subfigure}
	\end{subfigure}	
	\begin{subfigure}{1\textwidth}		
		\caption{Laissez-faire and social planner allocation}
		\begin{subfigure}[]{0.32\textwidth}
			\centering{\footnotesize{(a) Consumption}}
			%	\captionsetup{width=.45\linewidth}
			\includegraphics[width=1\textwidth]{../../codding_model/own_basedOnFried/optimalPol_010922_revision/figures/all_13Sept22_Tplus30/C_LFCompSP_T_regime4_knspil1_spillover0_noskill0_sep0_xgrowth0_PV1_etaa0.79_lgd1.png}
		\end{subfigure}	
		\begin{subfigure}[]{0.32\textwidth}
			\centering{\footnotesize{(b) High-skill hours worked}}
			%	\captionsetup{width=.45\linewidth}
			\includegraphics[width=1\textwidth]{../../codding_model/own_basedOnFried/optimalPol_010922_revision/figures/all_13Sept22_Tplus30/hh_LFCompSP_T_regime4_knspil1_spillover0_noskill0_sep0_xgrowth0_PV1_etaa0.79_lgd0.png}
		\end{subfigure}	
		\begin{subfigure}[]{0.32\textwidth}
			\centering{\footnotesize{(c) Low-skill hours worked}}
			%	\captionsetup{width=.45\linewidth}
			\includegraphics[width=1\textwidth]{../../codding_model/own_basedOnFried/optimalPol_010922_revision/figures/all_13Sept22_Tplus30/hl_LFCompSP_T_regime4_knspil1_spillover0_noskill0_sep0_xgrowth0_PV1_etaa0.79_lgd0.png}
		\end{subfigure}
		\begin{subfigure}[]{0.32\textwidth}
			\centering{\footnotesize{(f) Fossil scientists}}
			%	\captionsetup{width=.45\linewidth}
			\includegraphics[width=1\textwidth]{../../codding_model/own_basedOnFried/optimalPol_010922_revision/figures/all_13Sept22_Tplus30/sff_LFCompSP_T_regime4_knspil1_spillover0_noskill0_sep0_xgrowth0_PV1_etaa0.79_lgd1.png}
		\end{subfigure}	
		\begin{subfigure}[]{0.32\textwidth}
			\centering{\footnotesize{(e) Non-energy scientists}}
			%	\captionsetup{width=.45\linewidth}
			\includegraphics[width=1\textwidth]{../../codding_model/own_basedOnFried/optimalPol_010922_revision/figures/all_13Sept22_Tplus30/sn_LFCompSP_T_regime4_knspil1_spillover0_noskill0_sep0_xgrowth0_PV1_etaa0.79_lgd0.png}
		\end{subfigure}	
		\begin{subfigure}[]{0.32\textwidth}
			\centering{\footnotesize{(f) Green scientists}}
			%	\captionsetup{width=.45\linewidth}
			\includegraphics[width=1\textwidth]{../../codding_model/own_basedOnFried/optimalPol_010922_revision/figures/all_13Sept22_Tplus30/sg_LFCompSP_T_regime4_knspil1_spillover0_noskill0_sep0_xgrowth0_PV1_etaa0.79_lgd0.png}
		\end{subfigure}	
	\end{subfigure}	
\end{figure}
\clearpage
%\subsubsection{Comparison combined and carbon-tax-only policy regime }
%\begin{figure}[h!!!]
%	\centering
%	\caption{Deviation from optimal policy with only a carbon tax; additional variables}\label{fig:opt_TLs_add}
%	\begin{subfigure}{0.32\textwidth}
%		\caption{Fossil technology}
%		%	\captionsetup{width=.45\linewidth}
%		\includegraphics[width=1\textwidth]{../../codding_model/own_basedOnFried/optimalPol_010922_revision/figures/all_13Sept22_Tplus30/Af_OPT_COMPtaulPer_regime4_spillover0_knspil0_noskill0_sep0_xgrowth0_PV1_etaa0.79.png}
%	\end{subfigure}
%	\begin{subfigure}{0.32\textwidth}
%	\caption{Green technology}
%	%	\captionsetup{width=.45\linewidth}
%	\includegraphics[width=1\textwidth]{../../codding_model/own_basedOnFried/optimalPol_010922_revision/figures/all_13Sept22_Tplus30/Ag_OPT_COMPtaulPer_regime4_spillover0_knspil0_noskill0_sep0_xgrowth0_PV1_etaa0.79.png}
%\end{subfigure}
%\begin{subfigure}{0.32\textwidth}
%\caption{Fossil growth}
%%	\captionsetup{width=.45\linewidth}
%\includegraphics[width=1\textwidth]{../../codding_model/own_basedOnFried/optimalPol_010922_revision/figures/all_13Sept22_Tplus30/gAf_OPT_COMPtaulPer_regime4_spillover0_knspil0_noskill0_sep0_xgrowth0_PV1_etaa0.79.png}
%\end{subfigure}
%\begin{subfigure}{0.32\textwidth}
%	\caption{Green growth}
%	%	\captionsetup{width=.45\linewidth}
%	\includegraphics[width=1\textwidth]{../../codding_model/own_basedOnFried/optimalPol_010922_revision/figures/all_13Sept22_Tplus30/gAg_OPT_COMPtaulPer_regime4_spillover0_knspil0_noskill0_sep0_xgrowth0_PV1_etaa0.79.png}
%\end{subfigure}
%\begin{subfigure}{0.32\textwidth}
%	\caption{Non-energy growth}
%	%	\captionsetup{width=.45\linewidth}
%	\includegraphics[width=1\textwidth]{../../codding_model/own_basedOnFried/optimalPol_010922_revision/figures/all_13Sept22_Tplus30/gAn_OPT_COMPtaulPer_regime4_spillover0_knspil0_noskill0_sep0_xgrowth0_PV1_etaa0.79.png}
%\end{subfigure}
%\end{figure}
\subsubsection{Results without knowledge spillovers}\label{app:TLS}


\begin{figure}[h!!!]
	\centering
	\caption{Deviation from optimal policy with only a carbon tax; No knowledge spillovers}\label{fig:opt_TLs_noKN}
		\begin{subfigure}{0.32\textwidth}
		\caption{Av. marginal income tax rate }
		%	\captionsetup{width=.45\linewidth}
		\includegraphics[width=1\textwidth]{../../codding_model/own_basedOnFried/optimalPol_010922_revision/figures/all_13Sept22_Tplus30/dTaulAv_OPT_T_NoTaus_COMPtaul_regime4_spillover0_knspil1_noskill0_sep0_xgrowth0_PV1_etaa0.79_lgd0.png}
	\end{subfigure}
	\begin{subfigure}{0.32\textwidth}
		\caption{Carbon tax}
		%	\captionsetup{width=.45\linewidth}
		\includegraphics[width=1\textwidth]{../../codding_model/own_basedOnFried/optimalPol_010922_revision/figures/all_13Sept22_Tplus30/Tauf_OPT_COMPtaulPer_regime4_spillover0_knspil1_noskill0_sep0_xgrowth0_PV1_etaa0.79.png}
	\end{subfigure}
	\begin{subfigure}{0.32\textwidth}
		\caption{Period utility}
		%	\captionsetup{width=.45\linewidth}
		\includegraphics[width=1\textwidth]{../../codding_model/own_basedOnFried/optimalPol_010922_revision/figures/all_13Sept22_Tplus30/SWF_OPT_COMPtaulPer_regime4_spillover0_knspil1_noskill0_sep0_xgrowth0_PV1_etaa0.79.png}
	\end{subfigure}	
	\begin{subfigure}{0.32\textwidth}
		\caption{Fossil scientists}
		%	\captionsetup{width=.45\linewidth}
		\includegraphics[width=1\textwidth]{../../codding_model/own_basedOnFried/optimalPol_010922_revision/figures/all_13Sept22_Tplus30/sff_OPT_COMPtaulPer_regime4_spillover0_knspil1_noskill0_sep0_xgrowth0_PV1_etaa0.79.png}
	\end{subfigure}
	\begin{subfigure}{0.32\textwidth}
		\caption{Green scientists}
		%	\captionsetup{width=.45\linewidth}
		\includegraphics[width=1\textwidth]{../../codding_model/own_basedOnFried/optimalPol_010922_revision/figures/all_13Sept22_Tplus30/sg_OPT_COMPtaulPer_regime4_spillover0_knspil1_noskill0_sep0_xgrowth0_PV1_etaa0.79.png}
	\end{subfigure}
%	\begin{subfigure}{0.4\textwidth}
%		\caption{Non-energy scientists}
%		%	\captionsetup{width=.45\linewidth}
%		\includegraphics[width=1\textwidth]{../../codding_model/own_basedOnFried/optimalPol_010922_revision/figures/all_13Sept22_Tplus30/sn_OPT_COMPtaulPer_regime4_spillover0_knspil1_noskill0_sep0_xgrowth0_PV1_etaa0.79.png}
%	\end{subfigure}
	\begin{subfigure}{0.32\textwidth}
		\caption{Green-to-fossil output}
		%	\captionsetup{width=.45\linewidth}
		\includegraphics[width=1\textwidth]{../../codding_model/own_basedOnFried/optimalPol_010922_revision/figures/all_13Sept22_Tplus30/GFF_OPT_COMPtaulPer_regime4_spillover0_knspil1_noskill0_sep0_xgrowth0_PV1_etaa0.79.png}
	\end{subfigure}
	\floatfoot{Notes: \footnotesize{ Graphs show the percentage deviations of the variable under the combined policy regime where the planner can choose income tax progressivity and the carbon-tax-only regime without income tax, $\tau_{\iota t}=0$. }}
\end{figure} 

\clearpage

\begin{figure}[h!!!]
	\centering
	\caption{Effect of combined policy regime in model with lump-sum transfers, no knowledge spillovers, and homogeneous skills }\label{fig:opt_TLs_noknow_homoskill}
	\begin{subfigure}{0.32\textwidth}
		\caption{Labor income tax progressivity, $\tau_{\iota t}$}
		%	\captionsetup{width=.45\linewidth}
		\includegraphics[width=1\textwidth]{../../codding_model/own_basedOnFried/optimalPol_010922_revision/figures/all_13Sept22_Tplus30/taul_OPT_COMPtaul_regime4_spillover0_knspil1_noskill1_sep0_xgrowth0_PV1_etaa0.79_lgd0.png}
	\end{subfigure}
	\begin{subfigure}[]{0.32\textwidth}
		\caption{Carbon tax}
		%	\captionsetup{width=.45\linewidth}
		\includegraphics[width=1\textwidth]{../../codding_model/own_basedOnFried/optimalPol_010922_revision/figures/all_13Sept22_Tplus30/tauf_OPT_COMPtaulPer_regime4_spillover0_knspil1_noskill1_sep0_xgrowth0_PV1_etaa0.79.png}
	\end{subfigure}
	\begin{subfigure}[]{0.32\textwidth}
		\caption{Fossil growth }
		%	\captionsetup{width=.45\linewidth}
		\includegraphics[width=1\textwidth]{../../codding_model/own_basedOnFried/optimalPol_010922_revision/figures/all_13Sept22_Tplus30/gAf_OPT_COMPtaulPer_regime4_spillover0_knspil1_noskill1_sep0_xgrowth0_PV1_etaa0.79.png}
	\end{subfigure}
	\begin{subfigure}[]{0.32\textwidth}
		\caption{Green growth }
		%	\captionsetup{width=.45\linewidth}
		\includegraphics[width=1\textwidth]{../../codding_model/own_basedOnFried/optimalPol_010922_revision/figures/all_13Sept22_Tplus30/gAg_OPT_COMPtaulPer_regime4_spillover0_knspil1_noskill1_sep0_xgrowth0_PV1_etaa0.79.png}
	\end{subfigure}
	\begin{subfigure}[]{0.32\textwidth}
		\caption{ Non-energy growth }
		%	\captionsetup{width=.45\linewidth}
		\includegraphics[width=1\textwidth]{../../codding_model/own_basedOnFried/optimalPol_010922_revision/figures/all_13Sept22_Tplus30/gAn_OPT_COMPtaulPer_regime4_spillover0_knspil1_noskill1_sep0_xgrowth0_PV1_etaa0.79.png}
	\end{subfigure}
	\begin{subfigure}[]{0.32\textwidth}
		\caption{Period utility }
		%	\captionsetup{width=.45\linewidth}
		\includegraphics[width=1\textwidth]{../../codding_model/own_basedOnFried/optimalPol_010922_revision/figures/all_13Sept22_Tplus30/SWF_OPT_COMPtaulPer_regime4_spillover0_knspil1_noskill1_sep0_xgrowth0_PV1_etaa0.79.png}
	\end{subfigure}
	\begin{subfigure}[]{0.32\textwidth}
		\caption{ Hours worked }
		%	\captionsetup{width=.45\linewidth}
		\includegraphics[width=1\textwidth]{../../codding_model/own_basedOnFried/optimalPol_010922_revision/figures/all_13Sept22_Tplus30/hh_OPT_COMPtaulPer_regime4_spillover0_knspil1_noskill1_sep0_xgrowth0_PV1_etaa0.79.png}
	\end{subfigure}
\begin{subfigure}[]{0.32\textwidth}
\caption{Consumption }
%	\captionsetup{width=.45\linewidth}
\includegraphics[width=1\textwidth]{../../codding_model/own_basedOnFried/optimalPol_010922_revision/figures/all_13Sept22_Tplus30/C_OPT_COMPtaulPer_regime4_spillover0_knspil1_noskill1_sep0_xgrowth0_PV1_etaa0.79.png}
\end{subfigure}
	\floatfoot{Notes: \footnotesize{ Except for Panel (a), all graphs show the percentage deviation of the variable under the combined policy regime where the planner can choose income tax progressivity and the carbon-tax-only regime where the income tax scheme is non-distortive, $\tau_{\iota t}=0$. Panel (a) shows the income tax progressivity by regime.}}
\end{figure} 

\thispagestyle{empty}
\begin{figure}[h!!!]
	\centering
	\caption{Deviation under optimal policy with and without optimal labor income tax; homogeneous skills}\label{fig:opt_Count_homskill}
	\begin{subfigure}{0.4\textwidth}
		\caption{Average marginal income tax rate }
		%	\captionsetup{width=.45\linewidth}
		\includegraphics[width=1\textwidth]{../../codding_model/own_basedOnFried/optimalPol_010922_revision/figures/all_13Sept22_Tplus30/dTaulAv_OPT_T_NoTaus_COMPtaul_regime4_spillover0_knspil0_noskill1_sep0_xgrowth0_PV1_etaa0.79_lgd0.png}
	\end{subfigure}
	\begin{subfigure}{0.4\textwidth}
	\caption{Tax per ton of carbon in 2022 US\$}
	%	\captionsetup{width=.45\linewidth}
	\includegraphics[width=1\textwidth]{../../codding_model/own_basedOnFried/optimalPol_010922_revision/figures/all_13Sept22_Tplus30/Tauf_OPT_T_NoTaus_COMPtaul_regime4_spillover0_knspil0_noskill1_sep0_xgrowth0_PV1_etaa0.79_lgd0.png}
\end{subfigure}
	\begin{subfigure}{0.4\textwidth}
		\caption{Fossil production}
		%	\captionsetup{width=.45\linewidth}
		\includegraphics[width=1\textwidth]{../../codding_model/own_basedOnFried/optimalPol_010922_revision/figures/all_13Sept22_Tplus30/CountTAUFPerDif_Opt_target_F_nsk1_xgr0_knspil0_regime4_spillover0_sep0_extern0_PV1_etaa0.79.png}
	\end{subfigure}	
	\begin{subfigure}{0.4\textwidth}
		\caption{Average hours worked}
		%	\captionsetup{width=.45\linewidth}
		\includegraphics[width=1\textwidth]{../../codding_model/own_basedOnFried/optimalPol_010922_revision/figures/all_13Sept22_Tplus30/CountTAUFPerDif_Opt_target_hh_nsk1_xgr0_knspil0_regime4_spillover0_sep0_extern0_PV1_etaa0.79.png}
	\end{subfigure}
	\begin{subfigure}{0.4\textwidth}
		\caption{Fossil scientists}
		%	\captionsetup{width=.45\linewidth}
		\includegraphics[width=1\textwidth]{../../codding_model/own_basedOnFried/optimalPol_010922_revision/figures/all_13Sept22_Tplus30/CountTAUFPerDif_Opt_target_sff_nsk1_xgr0_knspil0_regime4_spillover0_sep0_extern0_PV1_etaa0.79.png}
	\end{subfigure}
	\begin{subfigure}{0.4\textwidth}
		\caption{Green scientists}
		%	\captionsetup{width=.45\linewidth}
		\includegraphics[width=1\textwidth]{../../codding_model/own_basedOnFried/optimalPol_010922_revision/figures/all_13Sept22_Tplus30/CountTAUFPerDif_Opt_target_sg_nsk1_xgr0_knspil0_regime4_spillover0_sep0_extern0_PV1_etaa0.79.png}
	\end{subfigure}
	\begin{subfigure}{0.4\textwidth}
		\caption{Non-energy scientists}
		%	\captionsetup{width=.45\linewidth}
		\includegraphics[width=1\textwidth]{../../codding_model/own_basedOnFried/optimalPol_010922_revision/figures/all_13Sept22_Tplus30/CountTAUFPerDif_Opt_target_sn_nsk1_xgr0_knspil0_regime4_spillover0_sep0_extern0_PV1_etaa0.79.png}
	\end{subfigure}
	\begin{subfigure}{0.4\textwidth}
		\caption{Green-to-fossil output}
		%	\captionsetup{width=.45\linewidth}
		\includegraphics[width=1\textwidth]{../../codding_model/own_basedOnFried/optimalPol_010922_revision/figures/all_13Sept22_Tplus30/CountTAUFPerDif_Opt_target_GFF_nsk1_xgr0_knspil0_regime4_spillover0_sep0_extern0_PV1_etaa0.79.png}
	\end{subfigure}
\end{figure}


%\section{Degrowth and end to growth in the literature}
\begin{itemize}
	\item \cite{Dasgupta2021}\ar impossibility to grow indefinitely \ar need to reduce to not surpass safe operating space
	\item \cite{Schor2005SustainableReduction}
	\item \cite{VanVuuren2018AlternativeTechnologies}: limit to carbon capture and storage technologies; if output growth requires fossil energy, than infinite growth would need infinite storage; to reduce dependence on this technology beneficial to reduce demand 
	\item \cite{Bertram2018TargetedScenarios}: reduction in demand to simultaneously meet emission targets and sustainability goals (global acceptability)\ar reduction of energy demand alleviates competition between reaching emission limits and sustainability goals (zero hunger, affordability of energy); to lower sustainability risk;
	\\   change demand as a parameter in  model; motivation: taking global inequality into account alternative measures (mitigation policies) to carbon taxes  become optimal. These include lifestyle changes \textbf{in addition to sector-specific carbon taxes!} (25\% lower energy demand and -20\% lower demand for agricultural products )
\end{itemize}

In his review, \cite{Dasgupta2021} attempts to construct an economics of biodiversity. By taking nature's maintenance services, that is, services without which human activity and live would not be possible, as a constituent of total factor productivity, economic growth is no longer disconnected from planetary boundaries (p.137).\footnote{\ The term \textit{planetary boundaries} has been coined by \cite{Rockstrom2009AHumanity} who use it to refer to a state of nature in which humans can safely exist.} 
Papers on environmental economics acknowledge that natural conditions are important to production by assuming tipping points \citep{Acemoglu2012TheChange} or rising temperatures affecting output as in \textit{cite Nordhaus1994, Stern 2006} \cite{Barrage2019OptimalPolicy}.
In addition, \cite{Dasgupta2021} accounts for the waste resulting from production and consumption which again requires nature for 

\cite{Dasgupta2021} argues that infinite GDP growth is impossible given planetary boundaries (p. 47), i.e., the safe space for humans to exist, and that waste which degrades the environment is always positively related to output. 

My project relates by looking at one particular boundary: the one on carbon emissions through a limited carbon budget postulated in the natural sciences \citep{IPCC2022, Rockstrom2009AHumanity}. The model is designed to allow for a stop to technological and consumption growth. 

 
%\appendix
\section{Appendix}
\subsection{Growth and the Environment}
It is a vibrant debate whether technological process will result in a production technology that is perfectly clean in that it does not exert any environmental externality. 
\begin{itemize}
	%\item \underline{Extensions to technology in \cite{Acemoglu2012TheChange} }
	%\begin{itemize}
	\item \underline{externality of ``clean'' sector} \citep[see also][]{Dasgupta2021, Brock2005ChapterEmpirics}
	\begin{itemize}
		\item[-] renewable/ non-fossil fuels \ar externalities in production process are present e.g. production of solar panels uses toxic inputs \citep{Yue2014DomesticAnalysis}; non-fossil fuel nitrogen generation (e.g., biomass burning to clear land) important ($\approx$ 50\%) \citep{Song2021ImportantEmissions}; low but chronical levels of nitrogen cause species extinctions \citep{Clark2008LossGrasslands}
		\item[-] waste (after use) \ar depends on recycling technology %\ar recycling system for solar panels not profitable enough today
		%	\item[-] substitutability of nature in production (input sources eg. fossil vs. non-fossil fuels)
		%\end{itemize}
		%\item Irreversibilities already before thresholds are hit (e.g. species extinction)
		
	\end{itemize}
	%\item greenhouse gases: Carbon dioxide $CO_2$ (vast majority), Nitrous oxide $N_2O$, methane $CH_4$
	%\item stock of nature globally determined
	\item \underline{parallel positive trend in demand} (population growth, rebound effect) that outperforms increase in clean technology growth \small{(no long-run issue if perfectly clean technology exists)}
	\item \normalsize{\underline{objective function}:} \cite{Arrow2004AreMuch}(Journal of Economic Perspectives) \ar using a sustainability measure they provide evidence that consumption is too high (= not leaving enough natural resources for future generations)
	\item \underline{risk, ambiguity}
	\item if have to meet climate target in short run, might need to lower production to do so; or it might be better in terms of inequality?
\end{itemize}<- Important notes on nature and on Balanced growth path in model based on fried
%\section{Guideline Computations}
\begin{enumerate}
	\item calibrate initial situation to data, using observed tax rates (this would be a competitive equilibrium with taxes as given)
	\item find BGP; BGP exists when $c^*_s+c_n^*\geq \bar{c}$; that is optimal allocation without penalty satisfies basic needs; from this point onwards there are no reallocations across sectors and sectors grow at a constant rate, this is equivalent to the solution of the problem without penalty term \ar \textbf{Need to solve Ramsey problem for BGP absent penalty term}
	Could also solve for the BGP in Ramsey model numerically: get model equations and set growth corrected variables to constant values\\
	Follow \cite{Jones1993OptimalGrowth}:
	\begin{enumerate}
		\item fix assumed SS tax rates and transfers relative to output
		\item calculate ss values of consumption/output, other variables relative to output (constant in ss)
		\item make end corrections to Ramsey problem which is explicitly solved up to period T given the values from point 1 and 2 above
		\item iterate until guess in 1 matches with solution for ss value 
	\end{enumerate}
end corrections are derived analytically.
	\item for competitive equilibrium follow \cite{Acemoglu2008CapitalGrowth}: 
	\begin{enumerate}
		\item analytically or numerically calculate BGP values
		\item initial values and parameters match to data (including tax rates and transfers)
		\item use shooting (or relaxation) algorithm to find solution, i.e. sequence of allocations that solve two boundary value problem: shooting algorithm finds initial conditions that s.t. ss values are matched. 
		\item proof uniqueness of transition path? 
	\end{enumerate}
	
	they write \begin{quote}
		The previous subsection demonstrated that there exists a unique CGP with  nonbalanced  sectoral  growth;  that  is,  there  is  aggregate  output growth at a constant rate together with differential sectoral growth and reallocation of factors of production across sectors. We now investigate whether the competitive equilibrium will approach the CGP. 
	\end{quote} 
\ar From where my economy starts today with calibrated ws, and taxes (a constant growth path), does it converge to a new constant growth path where ws is higher? 
\item in my model growth rates of sectoral consumption are not constant over time! Households reallocate shares as they get richer
\end{enumerate}

Simple version without growth
\begin{enumerate}
	\item economy is in SS today, as households income does not change their consumption is fixed, period t=0
	\item then in period t=1 ws rises (1)\ar what is the optimal policy when ws rises starting from calibrated tax rates; (2)\ar what is the new ss and how does the economy converge?
	\item[\ar] I know initial and end conditions, the shock system is also known a priori, then use shooting or relaxation algorithm to calculate transition
\end{enumerate}
%-------------------------------------
\clearpage
\bibliography{../../../bib_2_0}
\addcontentsline{toc}{section}{References}
\end{document}