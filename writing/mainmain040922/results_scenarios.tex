\subsection{Results 1}\label{subsec:exp}

This section serves to 
We are now equipped to study how a carbon tax  equal to the social costs of carbon of 185\$, as found by the \textit{Resources for the Future} (RfF), an independent research institution, affects the economy. In particular, I am interested in whether a progressive income tax scheme shapes the effectiveness of a carbon tax. 
 Comparing the level of emissions resulting under the constant carbon tax to the emission limits suggested by the IPCC reveals that a stronger reduction in emissions is necessary. Therefore, I calculate a necessary environmental tax to satisfy the US emission limit. A progressive income tax lowers the required environmental tax by between 10 and 7\%. Furthermore, the allocation is characterized by higher technology growth in the fossil and non-energy sector. 


\subsubsection{Effect of a constant carbon tax}
% \begin{itemize}
% 	\item  emissions continue to rise due to knowledge spillovers \checkmark
% 	\item  effect on research (non-energy research) \checkmark
% 	\item taul has level effect on emissions, interaction: compositional effect but small (but could become relevant through disadvantageous effect of tauf on non-energy research) \checkmark 	
% \end{itemize}

\begin{figure}[h!!]
	\centering
	\caption{Effect of a constant carbon tax equal to 185\$ per ton of carbon }\label{fig:Leveltauf_nsk0_xgr0_know}		
\begin{minipage}[]{0.32\textwidth}
	\centering{\footnotesize{(a) Net emissions}}
	%	\captionsetup{width=.45\linewidth}
	\includegraphics[width=1\textwidth]{../../codding_model/own_basedOnFried/optimalPol_010922_revision/figures/all_13Sept22/CompTauf_bytaul_Reg0_Emnet_spillover0_nsk0_xgr0_knspil0_sep0_LFlimit0_emsbase0_countec0_GovRev0_etaa0.79_lgd1.png}
\end{minipage}	
\begin{minipage}[]{0.32\textwidth}
\centering{\footnotesize{(b) Fossil }}
%	\captionsetup{width=.45\linewidth}
\includegraphics[width=1\textwidth]{../../codding_model/own_basedOnFried/optimalPol_010922_revision/figures/all_13Sept22/PerdifNoTauf_regime0_CompTaul_F_spillover0_nsk0_xgr0_knspil0_sep0_LFlimit0_emsbase0_countec0_GovRev0_etaa0.79_lgd1.png}
\end{minipage}
\begin{minipage}[]{0.32\textwidth}
\centering{\footnotesize{(c) Energy share to GDP}}
%	\captionsetup{width=.45\linewidth}
\includegraphics[width=1\textwidth]{../../codding_model/own_basedOnFried/optimalPol_010922_revision/figures/all_13Sept22/PerdifNoTauf_regime0_CompTaul_EY_spillover0_nsk0_xgr0_knspil0_sep0_LFlimit0_emsbase0_countec0_GovRev0_etaa0.79_lgd1.png}
\end{minipage}

\floatfoot{Notes:{ Panel (a) shows levels of net emissions under a constant carbon tax equal to 185\$ for a world with progressive income taxation at $\tau_l=0.181$, the solid graph, and without progressive income taxation, $\tau_l=0$. The thin dotted graph shows the emission limit suggested by the IPCC. Panels (b) and (c) show the percentage difference between the business-as-usual policy without carbon tax and with carbon tax in the economy with and without progressive income tax by the solid and dash graphs, respectively. }}
\end{figure} 
 The carbon tax equal to 185\$ diminishes emissions by around 42.4\% initially relative to the business-as-usual policy without carbon tax. As energy producers face a higher price for fossil energy, they lower demand for fossil and rise demand for green energy. The lower demand for fossil reduces the demand for labor from the fossil sector. As a result more labor is available for green energy production, and green supply increases. The price of green goods may fall despite the higher demand. This supply-side mechanism facilitates a transition to green energy; however, it is muted due to a low labor share in the green sector and heterogeneity in labor input goods.
 As the market size of the green sector increases, profitability of research in the green sector rises. In contrast, research in the fossil sector falls. 
 
 As a side effect of the fossil tax, research in the non-energy sector also reduces. The reason is that a higher price of the energy good lowers demand for non-energy goods as the two are complements. Revenues of  non-energy goods producers decline. As a result, research in this sector becomes less profitable.\footnote{\ In line with the theory on directed technical change, a price effect dominates the direction of research when goods are complements \citep{Hemous2021DirectedEconomics}.} 
 

% EFFECT OF TAUL: income tax progressivity does not change growth in the non-energy sector,\tr{ since the level effect is absorbed by price changes} CHECK THIS. A compositional effect arises when skills are heterogeneous. In equilibrium, high-skill workers work more hours due to a higher wage rate. Therefore, the marginal gains from leisure are higher for this type. When the tax scheme becomes more progressive, high-skill workers are more responsive and curb their labor supply more.\footnote{\ They have to be compensated by a higher wage rate for a further reduction of leisure. The additional unit of leisure is more valuable to them.}  A more progressive tax scheme makes green labor more expensive and green production reduces more than fossil production. More research in the fossil sector amplifies this compositional effect. 

 \textit{Annotation (SEP=0):  a progressive income tax raises research in fossil and in green \ar in energy goods, and lowers research in non-energy (the progressive tax makes energy goods more expensive which are more skill intense, being complements, research for energy becomes more profitable) But the effect ist small. This shift in innovation counters a reduction in the energy to output share (energy is less labor intense making labor more expensive it goes to the non-energy sector plus a reduction in skills; Q is this reduction also present in model with equal cap shares?); Furthermore, a prgressive income tax lowers the green to fossil ratio, intensified by response in research.  }
 
 
  The effect of the carbon tax is by and large unaffected by the value of income tax progressvity.
 % The stronger reduction of net emissions under a scenario with progressive income tax follows mechanically from a lower status-quo level of gros emissions due to the constant size of sinks.\footnote{\ A smaller initial level of fossil production inflates the percentage change in net emissions. }
   Except for the reaction of labor supply which explains the slightly smaller reduction in fossil production under a progressive income tax depicted in panel (a). The rationale is as follows.
    The fossil tax changes the skill premium since demand for the green-specific high skill labor increases. High-skill hours in equilibrium rise, while hours of low-skilled workers reduce.  A progressive income tax scheme mutes the effect of changes in the wage rate, that is, the pre-tax labor income, on labor supply. The reason is that the elasticity of after-tax labor income with respect to pre-tax labor income reduces. This diminishes the supply response in the skill ratio to the carbon tax. Therefore, the reduction in fossil production is muted in presence of a progressive income tax scheme. Research responds to the smaller supply of the skill ratio, and the rise in green research is muted while it is higher in the fossil sector.  \tr{Look at comparison between tauf}. Furthermore, a smaller reduction in low-skill supply makes non-energy goods, which are more low-skill intense than energy goods, cheaper. The lower price of non-energy goods and its complementarity to energy goods (so that demand for energy remains high) explains a stronger reduction in non-energy research when the income tax scheme is progressive.\footnote{\ The price for non-energy research also reduces more in equilibrium. The reduction in non-energy research is demand-side driven.} Energy research reduces less under a progressive income tax scheme. Yet, the effect is so small that there is no visible heterogeneity in the effect of the carbon tax on the energy share in final good production (panel (c)). Consequently, the presence of a progressive income tax scheme counters the intention to lower fossil and energy production when skills are heterogeneous.
   
Overall, the smaller level in production induced by the tax on labor helps to reduce net emissions below the level without income tax.
 
 %%%%%%%%%%%%%%%%%%%%%%%%%%%%%%%%%%%%%
 %% knowledge spillovers
 %%%%%%%%%%%%%%%%%%%%%%%%%%%%%%%%%%%%%
 Over time the effectiveness of the carbon tax to lower fossil production declines from 37\% to 35.5\%. There are two model features explaining this result: knowledge spillovers and heterogeneity in labor shares. 
 Consider first the effect of knowledge spillovers.
  Initially, the fossil tax reduces research in the fossil sector, however, as green technology advances, research in the fossil sector gains in profitability, and fossil research resurges. It is not only that a constant amount of researchers becomes more productive but also a change in the equilibrium level of fossil researchers which intensifies the effect of knowledge spillovers. This mechanism explains the quick increase in emissions under a constant fossil tax by approximately 1.5. Therefore, when knowledge spillovers are strong, reducing emissions to a net-zero emission limit requires a continues intensification of environmental intervention. In its extreme, growth might have to stop eventually in order to prevent the fossil sector from growing too much.\footnote{\ Figure \ref{fig:Leveltauf_nsk0_xgr0_noknow} shows the effect of a constant carbon tax in a model without knowledge spillovers, $\phi=0$. The rise in net emissions over time is muted and a constant carbon tax becomes more effective over time. \textbf{Absent knowledge spillovers more research is allocated to the green and non-energy sector} Check}
 
 
%%%%%%%%%%%%%%%%%%%%%%%%%%%%%%%%%%%%%%%%%%%
% \paragraph{Role of heterogeneous labor shares}
%%%%%%%%%%%%%%%%%%%%%%%%%%%%%%%%%%%%%%%%%%%%%%
 Second, the green sector has the smallest labor share. Labor is more important in the fossil and most important in the non-energy sector. This heterogeneity lowers the effectiveness of the fossil tax. 
 Consider a situation in which the green and the fossil sector both use the same share of labor. Then, a reduction in demand for labor in the fossil sector eases labor costs for the green sector. When, however, the green sector only uses a small share of labor, the higher labor supply does not lower green production costs as much, and the green good remains more expensive. The share of green energy and labor rises less. Figure \ref{fig:Leveltauf_nsk0_xgr0_equal_know} in the appendix displays the behavior of key variables in a model with homogeneous labor shares across sectors, i.e., $\alpha_g=\alpha_f=\alpha_n$. 

When labor shares are equal, the effectiveness of the carbon tax increases over time. 
Absent a carbon tax, a smaller labor share in the green sector raises fossil production since, as the fossil sector becomes more productive, the marginal product of labor in the fossil sector increases more and labor transitions to the fossil sector. The carbon tax counters this mechanism. \textbf{Is it the BAU allocation that changes or the policy one?} % (Note that consequently the economy is not on a bgp in BAU)

Endogenous growth intensifies this adverse effect of heterogeneous capital shares: as the market size of green goods is depressed, the fossil tax does not boost green research as much as with equal capital shares. In sum, heterogeneous labor shares imply an increasing path of emissions over time under the fossil tax. %The rising path follows from the faster increase in the marginal product of labor in the fossil sector. 
This feature of the economy also calls for a carbon tax increasing over time.
 
 
 Absent knowledge spillovers and with equal labor shares, the constant carbon tax suffices to lower emissions over time. Then, endogenous growth directs research away from the fossil sector so that emissions reducer over time under a constant carbon tax. 
 This finding replicates the result in \cite{Acemoglu2012TheChange} who abstract from knowledge spillovers and heterogeneity in labor shares and conclude that when dirty and clean goods are sufficient substitutes, a temporary intervention suffices to prevent too high pollution. In the present model, knowledge spillovers and heterogeneous capital shares 
 call for an ever increasing carbon tax even though the green and the energy good are sufficiently substitutable. Figure \ref{fig:Leveltauf_nsk0_xgr0_equal_noknow} in appendix section \ref{app:polexp}  visualizes this result: Without knowledge spillovers and with equal labor shares, the path of emissions is declining over time and the effectiveness of the carbon tax to lower fossil production increases over time. 
 
% \begin{comment}
% \paragraph{Role of heterogeneous skills}
% When skills are heterogeneous, emissions diminish because availability of the labor input good reduces. The effect of the fossil tax is similar to the model with homogeneous skills, yet, the fossil tax now affects labor.\footnote{\ With homogeneous skills, the equilibrium level of labor is unaffected by the fossil tax due to the log-utility of consumption implying that income and substitution effects of the wage rate cancel.}
% With two types of skills and heterogeneous high-skill share, the fossil tax has a diminishing effect on low-skill hours in equilibrium and raises high-skill hours. The reason is that a higher demand for green produce translates into a higher demand for the green-specific labor good which is high-skill biased. 
% 
% The effect on working hours through the fossil tax is affected by the presence of a progressive income tax scheme. The higher income tax progressivity, the smaller the effect of a change in the wage rate on labor supply, since the elasticity of post-tax labor income with respect to pre-tax labor income falls. 
% This asymmetry has an effect on research. A smaller reduction in low-skill supply makes non-energy goods, which are more low-skill intense than energy goods, cheaper. The lower price of non-energy goods and its complementarity to energy goods (so that demand for energy remains high) explains a stronger reduction in non-energy research when the income tax scheme is progressive.\footnote{\ The price for non-energy research also reduces more in equilibrium. The reduction in non-energy research is demand-side driven.} Energy research reduces less under a progressive income tax scheme. In addition, since a relatively higher low-skill supply boosts the market share of fossil producers, fossil research diminishes less than green research in presence of a progressive income tax. 
% Consequently, the presence of a progressive income tax scheme counters the intention to lower fossil and energy production when skills are heterogeneous. Nevertheless, is numerically negligible. 
%  
%  content...
%  \end{comment}
  
  
 \subsubsection{Meeting the emission limit}
 
 The previous section made apparent that, first, the carbon tax suggested by the RfF does not reduce emissions enough to meet the IPCC's emission limit, and, second, that dynamics of the model call for a dynamic carbon tax. In this section, I calculate the necessary carbon tax to meet the net-zero emission limit. I compare the resulting tax and allocations for the policy regimes with and without progressive income tax. 
 

 \begin{figure}[h!!]
 	\centering
 	\caption{Necessary carbon tax with and without progressive income tax  }\label{fig:Limit_nsk0_xgr0_know}		
 	\begin{minipage}[]{0.32\textwidth}
 		\centering{\footnotesize{(a) Carbon tax}}
 		%	\captionsetup{width=.45\linewidth}
 		\includegraphics[width=1\textwidth]{../../codding_model/own_basedOnFried/optimalPol_010922_revision/figures/all_13Sept22/CompTauf_bytaul_Reg0_tauf_spillover0_nsk0_xgr0_knspil0_sep0_LFlimit1_emsbase0_countec0_GovRev0_etaa0.79_lgd1.png}
 	\end{minipage}		\begin{minipage}[]{0.32\textwidth}
 	\centering{\footnotesize{(b) Deviation in carbon tax}}
 	%	\captionsetup{width=.45\linewidth}
 	\includegraphics[width=1\textwidth]{../../codding_model/own_basedOnFried/optimalPol_010922_revision/figures/all_13Sept22/CompTaufPER_bytaul_Reg0_tauf_spillover0_nsk0_xgr0_knspil0_sep0_LFlimit1_emsbase0_countec0_GovRev0_etaa0.79_lgd0.png} \end{minipage}		
%\begin{minipage}[]{0.32\textwidth}
%	\centering{\footnotesize{(c) Fossil research }}
%	%	\captionsetup{width=.45\linewidth}
%	\includegraphics[width=1\textwidth]{../../codding_model/own_basedOnFried/optimalPol_010922_revision/figures/all_13Sept22/CompTauf_bytaul_Reg0_sff_spillover0_nsk0_xgr0_knspil0_sep0_LFlimit1_emsbase0_countec0_GovRev0_etaa0.79_lgd1.png}
%\end{minipage}	
\begin{minipage}[]{0.32\textwidth}
\centering{\footnotesize{(c) Green-to-fossil energy ratio}}
%	\captionsetup{width=.45\linewidth}
\includegraphics[width=1\textwidth]{../../codding_model/own_basedOnFried/optimalPol_010922_revision/figures/all_13Sept22/CompTauf_bytaul_Reg0_GFF_spillover0_nsk0_xgr0_knspil0_sep0_LFlimit1_emsbase0_countec0_GovRev0_etaa0.79_lgd0.png}
\end{minipage}		
%\begin{minipage}[]{0.32\textwidth}
%\centering{\footnotesize{(d) Energy to GDP}}
%%	\captionsetup{width=.45\linewidth}
%\includegraphics[width=1\textwidth]{../../codding_model/own_basedOnFried/optimalPol_010922_revision/figures/all_13Sept22/CompTauf_bytaul_Reg0_EY_spillover0_nsk0_xgr0_knspil0_sep0_LFlimit1_emsbase0_countec0_GovRev0_etaa0.79_lgd0.png}
%\end{minipage}	
\begin{minipage}[]{0.32\textwidth}
\centering{\footnotesize{(e) Fossil growth}}
%	\captionsetup{width=.45\linewidth}
\includegraphics[width=1\textwidth]{../../codding_model/own_basedOnFried/optimalPol_010922_revision/figures/all_13Sept22/CompTauf_bytaul_Reg0_gAf_spillover0_nsk0_xgr0_knspil0_sep0_LFlimit1_emsbase0_countec0_GovRev0_etaa0.79_lgd0.png}
\end{minipage}			
\begin{minipage}[]{0.32\textwidth}
\centering{\footnotesize{(f) Non-energy growth}}
%	\captionsetup{width=.45\linewidth}
\includegraphics[width=1\textwidth]{../../codding_model/own_basedOnFried/optimalPol_010922_revision/figures/all_13Sept22/CompTauf_bytaul_Reg0_gAn_spillover0_nsk0_xgr0_knspil0_sep0_LFlimit1_emsbase0_countec0_GovRev0_etaa0.79_lgd0.png}
\end{minipage}			
\begin{minipage}[]{0.32\textwidth}
\centering{\footnotesize{(f) Green growth}}
%	\captionsetup{width=.45\linewidth}
\includegraphics[width=1\textwidth]{../../codding_model/own_basedOnFried/optimalPol_010922_revision/figures/all_13Sept22/CompTauf_bytaul_Reg0_gAg_spillover0_nsk0_xgr0_knspil0_sep0_LFlimit1_emsbase0_countec0_GovRev0_etaa0.79_lgd0.png}
\end{minipage}		%\begin{minipage}[]{0.32\textwidth}
%\centering{\footnotesize{(b) Deviation in carbon tax}}
%%	\captionsetup{width=.45\linewidth}
%\includegraphics[width=1\textwidth]{../../codding_model/own_basedOnFried/optimalPol_010922_revision/figures/all_13Sept22/CompTaufPER_bytaul_Equlab_Reg0_tauf_spillover0_nsk1_xgr1_knspil1_sep0_LFlimit1_emsbase0_countec0_GovRev0_etaa0.79_lgd0.png} \end{minipage}		
 	\floatfoot{Notes:{ }}
 \end{figure} 
 
Figure \ref{fig:Limit_nsk0_xgr0_know} shows the necessary carbon tax and differences in allocations across progressivity scenarios. 
The necessary carbon tax is lower when labor is taxed, panel (a). The deviation of carbon tax reaches -10\% in initial periods but diminishes over time to approximately -7.25\%, panel (b).
As we will see below, the smaller deviation is not only due to the smaller effectiveness of carbon taxes in the light of knowledge spillovers and a small labor share in the green sector but also driven by the tightness of the emission limit itself.\footnote{\ Abstracting from all model features discussed earlier, the carbon tax approaches the one without progressive income tax, too. } Then, the reductive effects of labor tax do not allow for a lower carbon tax. 
 The compositional effect of progressive income taxes, in contrast, remains relatively constant over time. Hence, it is not more adverse effects of a progressive income tax which explains the smaller reduction in carbon taxes. Over time, the economy can profit less from the reductive effect of progressive income taxes to meet emission limits. 

% compositional effect
Panels (c) depicts the ratio of green to fossil output and the energy share of GDP. The joint policy with progressive income tax implies a substantially smaller green-to-fossil ratio and a higher energy share.  
A carbon tax makes green goods relatively cheaper and energy goods more expensive. Endogenous growth contributes to the rise in green energy share. A market size effect directs research to the green sector, but less so the smaller the carbon tax. In contrast, a smaller carbon tax results in a higher share of non-energy scientists due to the complementarity of energy and non-energy goods. Nevertheless, in equilibrium, the energy to GDP ratio falls with the carbon tax. 
%First, as explained earlier, a carbon tax induces more green and less non-energy research (the latter being due to the higher price of energy goods when carbon is taxed). 
The progressive income tax contributes to the lower green-to-fossil output ratio but diminishes the energy-to-GDP ratio  by changing the supply of high-skill relative to low-skill labor. Figure \ref{fig:Efftaul_nsk0_xgr0_know_app} in the appendix shows that with a progressive income tax the share of green-to-fossil research is slightly lower. This contributes to a smaller green energy share. Furthermore, a progressive income tax implies a smaller energy share. The reason is that non-energy goods are less skill intense than energy goods. As the ratio of high-to-low skill falls, non-energy goods become cheaper. This effect counteracts the use of fossil energy. Yet, the effect of income tax progressivity is small. 
%\textbf{check if this result remains with equal capital shares: it does: but the energy share is increasing with equal labor shares since productivity rise in non-energy does not redirect labor. Hence, a reduction in high-skill supply has the same qualitative effect on the energy ratio. But it seems to be smaller... } 

 
 % effect on growth
In addition to the compositional effects the presence of a progressive income tax affects growth rates. 
Under the joint policy, the economy sees higher technology growth in the fossil and the non-energy sector. 
Again, changes in growth rates are mainly induced by the smaller carbon tax as the effect of progressive income taxes is small. Qualitatively, the progressive income tax counters higher non-energy research by making energy goods less expensive. On the other hand, it contributes to the reduction in the green-to-fossil research ratio through a market size mechanism.  In sum, the higher amount of energy scientists outweighs the recomposition towards fossil research so that green research increases in presence of a progressive income tax. Again, effects are small. 

To summarize, a progressive income tax has an adverse effect on the green energy share directly and indirectly through its impact on the effectiveness of the carbon tax.
Then again, the results suggests that under a progressive income tax the emission limit can be reached at higher technology growth rates. 
The following section serves to answer the question whether some substitution of  carbon taxes with progressive income taxes is indeed optimal. 

 In the following, I will briefly point to the role of model features in shaping the necessary carbon tax. 
% \begin{itemize}
% 	\item dynamics: I expect rising over time due to capital ratio heterogeneity and knowledge spillovers
% 	\item role of taul as a level effect versus compositional effect
% 	\item advantages under policy tuple progressive income tax and lower fossil tax as opposed to only fossil tax?
% \end{itemize}

\paragraph{The role of knowledge spillovers}

Knowledge spillovers render a higher fossil tax in all periods and a more aggressive environmental policy over time necessary.
This underlines the previous observation that emissions rise faster especially in future periods when knowledge spills from the green to the fossil sector. 

%On the other hand, knowledge spillovers mitigate the adverse compositional effect of progressive income taxes on the green-to-fossil energy share.
%COMMENT: THIS IS TRUE BUT THERE IS NO EVIDENCE FOR AN IMPACT ON TAUF

 %\textbf{comparison effect taul with and without knowledge spillovers}%As a result, the required fossil tax deviates more from the fossil tax when no progressive income tax is present. 
 

Secondly, the combination of a progressive income tax and a lower carbon tax leads to a smaller decline in green growth in future periods when knowledge spills over, and fossil growth rises less. As a result, the green-to-fossil energy ratio falls less. \textbf{Knowledge spillovers, therefore, lower the costs of a combined environmental policy consisting of a progressive income tax and a carbon tax. } 

As knowledge spillovers ameliorate the costs of a combined policy in future periods due to higher green growth, the required carbon tax can be set lower relative to the model without progressive tax. In other words, due to the better green-to-fossil energy ratio resulting from knowledge spillovers, a lower carbon tax can be set in future periods. 

The energy-to-output ratio exceeds its counterpart under the scenario with only a carbon tax approximately by the same percentage in early periods; later, however, it is higher due to the lower carbon tax. 
 
\footnote{\ 
It is not that knowledge spillovers make a fossil tax more costly in terms of technology growth in fossil or non-energy. In fact, growth in these two sectors is closer to the allocation with a combined policy.   Non-energy growth is higher because non-energy research rises more as the non-energy good is relatively more expensive absent knowledge spillovers. Fossil growth faster as it is hampered by knowledge spillovers as leading sector.}

This section preempts some of the optimal policy results which I will discuss in section \ref{sec:res}: progressive income taxes are optimally used to substitute for carbon taxes due to knowledge spillovers. The gains of this policy higher fossil and non-energy growth, are reaped in earlier periods, but only when the costs of lower green growth in the future is mitigated by knowledge spillovers, the Ramsey planner finds such a policy optimal. 
 \begin{figure}[h!!]
	\centering
	\caption{Comparison in deviations with and without knowledge spillovers \tr{Add no knowledge spillovers in graph.} }\label{fig:Limit_nsk0_xgr0_know_Devs}		
	\begin{minipage}[]{0.32\textwidth}
		\centering{\footnotesize{(a) Green-to-fossil energy ratio}}
		%	\captionsetup{width=.45\linewidth}
		\includegraphics[width=1\textwidth]{../../codding_model/own_basedOnFried/optimalPol_010922_revision/figures/all_13Sept22/CompTaufPER_bytaul_KN_Reg0_GFF_spillover0_nsk0_xgr0_knspil0_sep0_LFlimit1_emsbase0_countec0_GovRev0_etaa0.79_lgd1.png}
	 \end{minipage}	
	\begin{minipage}[]{0.32\textwidth}
		\centering{\footnotesize{(b) Fossil growth}}
		%	\captionsetup{width=.45\linewidth}
		\includegraphics[width=1\textwidth]{../../codding_model/own_basedOnFried/optimalPol_010922_revision/figures/all_13Sept22/CompTaufPER_bytaul_KN_Reg0_gAf_spillover0_nsk0_xgr0_knspil0_sep0_LFlimit1_emsbase0_countec0_GovRev0_etaa0.79_lgd0.png} 
	\end{minipage}	
	\begin{minipage}[]{0.32\textwidth}
		\centering{\footnotesize{(e) Non-energy growth}}
		%	\captionsetup{width=.45\linewidth}
		\includegraphics[width=1\textwidth]{../../codding_model/own_basedOnFried/optimalPol_010922_revision/figures/all_13Sept22/CompTaufPER_bytaul_KN_Reg0_gAn_spillover0_nsk0_xgr0_knspil0_sep0_LFlimit1_emsbase0_countec0_GovRev0_etaa0.79_lgd0.png} \end{minipage}	
	\begin{minipage}[]{0.32\textwidth}
		\centering{\footnotesize{(e) Green growth}}
		%	\captionsetup{width=.45\linewidth}
		\includegraphics[width=1\textwidth]{../../codding_model/own_basedOnFried/optimalPol_010922_revision/figures/all_13Sept22/CompTaufPER_bytaul_KN_Reg0_gAg_spillover0_nsk0_xgr0_knspil0_sep0_LFlimit1_emsbase0_countec0_GovRev0_etaa0.79_lgd0.png} \end{minipage}		
\end{figure}

I discuss the effects of heterogeneous skills, heterogeneous labor shares, and endogenous skill in appendix \ref{app:eff_feat_exp}. 
%Absent knowledge spillovers, growth in the non-energy sector is indeed higher than absent a progressive tax, but it becomes smaller when a more aggressive fossil tax is required. Then, the reductive effect of the progressive income tax does not suffice to diminish the fossil tax in a way that non-energy research increases. Then, the progressive income tax in addition to the fossil tax both lower non-energy research and growth. With knowledge spillovers, non-energy research and growth are higher; a positive spillover effect of fossil growth. The green sector, too, profits from knowledge spillovers from the fossil sector. Green growth eventually is higher under the policy with progressive 

%However, overall, the benefits of a lower fossil tax in presence of a higher tax progressivity come at the cost of lower consumption and a smaller green-to-fossil ratio and a higher energy share. Yet, the latter two are acceptable due to a smaller level of production. 
