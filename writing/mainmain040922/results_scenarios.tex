\subsection{Results 1}\label{subsec:exp}

We are now equipped to 
How does a carbon tax  equal to the social costs of carbon of 185\$, as found by the \textit{Resources for the Future} (RfF), an independent research institution, affect the economy? In particular I am interested in whether a progressive income tax scheme shapes the effectiveness of a carbon tax. 

 I compare the resulting level of emissions to the emission limits and find that a stronger reduction in emissions is necessary. Therefore, I calculate a necessary environmental tax to satisfy the US emission limit. A progressive income tax lowers the required environmental tax by between 10 and 7\%. Furthermore, the allocation is characterized by higher technology growth in the fossil and non-energy sector. 


\subsubsection{Effect of a constant carbon tax}
% \begin{itemize}
% 	\item  emissions continue to rise due to knowledge spillovers \checkmark
% 	\item  effect on research (non-energy research) \checkmark
% 	\item taul has level effect on emissions, interaction: compositional effect but small (but could become relevant through disadvantageous effect of tauf on non-energy research) \checkmark 	
% \end{itemize}

\begin{figure}[h!!]
	\centering
	\caption{A constant carbon tax equal to social cost of carbon }\label{fig:Leveltauf_nsk0_xgr0_know}		
\begin{minipage}[]{0.32\textwidth}
	\centering{\footnotesize{(a) Net emissions}}
	%	\captionsetup{width=.45\linewidth}
	\includegraphics[width=1\textwidth]{../../codding_model/own_basedOnFried/optimalPol_010922_revision/figures/all_13Sept22/CompTauf_bytaul_Reg0_Emnet_spillover0_nsk0_xgr0_knspil0_sep0_LFlimit0_emsbase0_countec0_GovRev0_etaa0.79_lgd1.png}
\end{minipage}	
\begin{minipage}[]{0.32\textwidth}
\centering{\footnotesize{(b) Fossil }}
%	\captionsetup{width=.45\linewidth}
\includegraphics[width=1\textwidth]{../../codding_model/own_basedOnFried/optimalPol_010922_revision/figures/all_13Sept22/PerdifNoTauf_regime0_CompTaul_F_spillover0_nsk0_xgr0_knspil0_sep0_LFlimit0_emsbase0_countec0_GovRev0_etaa0.79_lgd1.png}
\end{minipage}
\begin{minipage}[]{0.32\textwidth}
\centering{\footnotesize{(c) Energy share to GDP}}
%	\captionsetup{width=.45\linewidth}
\includegraphics[width=1\textwidth]{../../codding_model/own_basedOnFried/optimalPol_010922_revision/figures/all_13Sept22/PerdifNoTauf_regime0_CompTaul_EY_spillover0_nsk0_xgr0_knspil0_sep0_LFlimit0_emsbase0_countec0_GovRev0_etaa0.79_lgd1.png}
\end{minipage}

\floatfoot{Notes:{ Panel (a) shows levels of net emissions under a constant carbon tax equal to 185\$ for a world with progressive income taxation at $\tau_l=0.181$, the solid graph, and without progressive income taxation, $\tau_l=0$. The thin dotted graph shows the emission limit suggested by the IPCC. Panels (b) and (c) show the percentage difference between the business-as-usual policy without carbon tax and with carbon tax in the economy with and without progressive income tax by the solid and dash graphs, respectively. }}
\end{figure} 
 The carbon tax equal to 185\$ diminishes emissions by around 42.4\% initially relative to the business-as-usual policy without carbon tax. As energy producers face a higher price for fossil energy, they lower demand for fossil and rise demand for green energy. The lower demand for fossil reduces the demand for labor from the fossil sector. As a result more labor is available for green energy production, and green supply increases. The price of green goods may fall despite the higher demand. This supply-side mechanism facilitates a transition to green energy; however, it is muted due to a low labor share in the green sector and heterogeneity in labor input goods.
 As the market size of the green sector increases, profitability of research in the green sector rises. In contrast, research in the fossil sector falls. 
 
 As a side effect of the fossil tax, research in the non-energy sector also reduces. The reason is that a higher price of the energy good lowers demand for non-energy goods as the two are complements. Revenues of  non-energy goods producers decline. As a result, research in this sector becomes less profitable.\footnote{\ In line with the theory on directed technical change, a price effect dominates the direction of research when goods are complements \citep{Hemous2021DirectedEconomics}.} 
 

% EFFECT OF TAUL: income tax progressivity does not change growth in the non-energy sector,\tr{ since the level effect is absorbed by price changes} CHECK THIS. A compositional effect arises when skills are heterogeneous. In equilibrium, high-skill workers work more hours due to a higher wage rate. Therefore, the marginal gains from leisure are higher for this type. When the tax scheme becomes more progressive, high-skill workers are more responsive and curb their labor supply more.\footnote{\ They have to be compensated by a higher wage rate for a further reduction of leisure. The additional unit of leisure is more valuable to them.}  A more progressive tax scheme makes green labor more expensive and green production reduces more than fossil production. More research in the fossil sector amplifies this compositional effect. 

 \textit{Annotation (SEP=0):  a progressive income tax raises research in fossil and in green \ar in energy goods, and lowers research in non-energy (the progressive tax makes energy goods more expensive which are more skill intense, being complements, research for energy becomes more profitable) But the effect ist small. This shift in innovation counters a reduction in the energy to output share (energy is less labor intense making labor more expensive it goes to the non-energy sector plus a reduction in skills; Q is this reduction also present in model with equal cap shares?); Furthermore, a prgressive income tax lowers the green to fossil ratio, intensified by response in research.  }
 
 
  The effect of the carbon tax is by and large unaffected by the value of income tax progressvity.
 % The stronger reduction of net emissions under a scenario with progressive income tax follows mechanically from a lower status-quo level of gros emissions due to the constant size of sinks.\footnote{\ A smaller initial level of fossil production inflates the percentage change in net emissions. }
   Except for the reaction of labor supply which explains the slightly smaller reduction in fossil production under a progressive income tax depicted in panel (a). The rationale is as follows.
    The fossil tax changes the skill premium since demand for the green-specific high skill labor increases. High-skill hours in equilibrium rise, while hours of low-skilled workers reduce.  A progressive income tax scheme mutes the effect of changes in the wage rate, that is, the pre-tax labor income, on labor supply. The reason is that the elasticity of after-tax labor income with respect to pre-tax labor income reduces. This diminishes the supply response in the skill ratio to the carbon tax. Therefore, the reduction in fossil production is muted in presence of a progressive income tax scheme. Research responds to the smaller supply of the skill ratio, and the rise in green research is muted while it is higher in the fossil sector.  \tr{Look at comparison between tauf}
   
    Nevertheless, despite the compositional effect of tax progressivity, the energy-to-output ratio changes similarly (panel (c)). Overall, the smaller level in production helps to reduce net emissions below the level without income tax.
 
 \paragraph{Knowledge spillovers}
 Knowledge spillovers diminish the effectiveness of the fossil tax to lower emissions over time. Initially, the fossil tax reduces research in the fossil sector, however, as green technology advances, research in the fossil sector gains in profitability, and fossil research resurges. It is not only that a constant amount of researchers becomes more productive but also a change in the equilibrium level of fossil researchers which intensifies the effect of knowledge spillovers. This mechanism explains the quick increase in emissions under a constant fossil tax. Therefore, when knowledge spillovers are strong, reducing emissions to a net-zero emission limit requires a continues intensification of environmental intervention. 
 
 When knowledge spillovers are strong, growth might have to stop eventually in order to prevent the fossil sector from growing to much. 
 
 \cite{Acemoglu2012TheChange} abstract from knowledge spillovers, heterogeneous capital shares, and conclude that when dirty and clean goods are sufficient substitutes, a temporary intervention suffices. 
 
 
 \begin{figure}[h!!]
 	\centering
 	\caption{A constant carbon tax equal to social cost of carbon }\label{fig:Leveltauf_nsk0_xgr0_noknow_notaul}		
 	\begin{minipage}[]{0.32\textwidth}
 		\centering{\footnotesize{(a) Net emissions}}
 		%	\captionsetup{width=.45\linewidth}
 		\includegraphics[width=1\textwidth]{../../codding_model/own_basedOnFried/optimalPol_010922_revision/figures/all_13Sept22/CompTauf_bytaul_Reg0_Emnet_spillover0_nsk0_xgr0_knspil1_sep0_LFlimit0_emsbase0_countec0_GovRev0_etaa0.79_lgd1.png}
 	\end{minipage}	
 	\begin{minipage}[]{0.32\textwidth}
 		\centering{\footnotesize{(b) Fossil }}
 		%	\captionsetup{width=.45\linewidth}
 		\includegraphics[width=1\textwidth]{../../codding_model/own_basedOnFried/optimalPol_010922_revision/figures/all_13Sept22/PerdifNoTauf_regime0_CompTaul_F_spillover0_nsk0_xgr0_knspil1_sep0_LFlimit0_emsbase0_countec0_GovRev0_etaa0.79_lgd1.png}
 	\end{minipage}
 	\begin{minipage}[]{0.32\textwidth}
 		\centering{\footnotesize{(c) Energy share to GDP}}
 		%	\captionsetup{width=.45\linewidth}
 		\includegraphics[width=1\textwidth]{../../codding_model/own_basedOnFried/optimalPol_010922_revision/figures/all_13Sept22/PerdifNoTauf_regime0_CompTaul_EY_spillover0_nsk0_xgr0_knspil1_sep0_LFlimit0_emsbase0_countec0_GovRev0_etaa0.79_lgd1.png}
 	\end{minipage}
 	
 	\floatfoot{Notes:{ Panel (a) shows levels of net emissions under a constant carbon tax equal to 185\$ for a world with progressive income taxation at $\tau_l=0.181$, the solid graph, and without progressive income taxation, $\tau_l=0$. The thin dotted graph shows the emission limit suggested by the IPCC. Panels (b) and (c) show the percentage difference between the business-as-usual policy without carbon tax and with carbon tax in the economy with and without progressive income tax by the solid and dash graphs, respectively. }}
 \end{figure} 
 \paragraph{Role of heterogeneous labor shares}
 Under the baseline calibration, capital shares are heterogeneous across sectors. The green sector has the smallest labor share. Labor is more important in the fossil and most important in the non-energy sector. This heterogeneity lowers the effectiveness of the fossil tax. The reduction in demand for labor from the fossil sector does not lower production costs for the green sector as much. Consequently, the green good remains more expensive, the share of green energy and labor rises less. 

Over time, a smaller labor share in the green sector raises fossil production since, as the fossil sector becomes more productive, the marginal product of labor in the fossil sector increases more and labor transitions to the fossil sector.  % (Note that consequently the economy is not on a bgp in BAU)

Endogenous growth intensifies this adverse effect of heterogeneous capital shares: as the market size of green goods is depressed, the fossil tax does not boost green research as much as with equal capital shares. In sum, heterogeneous labor shares imply an increasing path of emissions over time under the fossil tax. %The rising path follows from the faster increase in the marginal product of labor in the fossil sector. 
This feature of the economy also calls for a carbon tax increasing over time.
 
 \paragraph{Role of heterogeneous skills}
 When skills are heterogeneous, emissions diminish because availability of the labor input good reduces. The effect of the fossil tax is similar to the model with homogeneous skills, yet, the fossil tax now affects labor.\footnote{\ With homogeneous skills, the equilibrium level of labor is unaffected by the fossil tax due to the log-utility of consumption implying that income and substitution effects of the wage rate cancel.}
 With two types of skills and heterogeneous high-skill share, the fossil tax has a diminishing effect on low-skill hours in equilibrium and raises high-skill hours. The reason is that a higher demand for green produce translates into a higher demand for the green-specific labor good which is high-skill biased. 
 
 The effect on working hours through the fossil tax is affected by the presence of a progressive income tax scheme. The higher income tax progressivity, the smaller the effect of a change in the wage rate on labor supply, since the elasticity of post-tax labor income with respect to pre-tax labor income falls. 
 This asymmetry has an effect on research. A smaller reduction in low-skill supply makes non-energy goods, which are more low-skill intense than energy goods, cheaper. The lower price of non-energy goods and its complementarity to energy goods (so that demand for energy remains high) explains a stronger reduction in non-energy research when the income tax scheme is progressive.\footnote{\ The price for non-energy research also reduces more in equilibrium. The reduction in non-energy research is demand-side driven.} Energy research reduces less under a progressive income tax scheme. Since a relatively higher low-skill supply boosts the market share of fossil producers, fossil research diminishes less than green research in presence of a progressive income tax. 
 Consequently, the presence of a progressive income tax scheme counters the intention to lower fossil and energy production when skills are heterogeneous. Nevertheless, is numerically negligible. 
  
 \subsubsection{Meeting the emission limit}
 Panel (a) in figure \ref{fig:Leveltauf_nsk0_xgr0_noknow_notaul} in the previous section reveals that a further reduction in emissions is required to satisfy the emission limit. Furthermore, the rising fossil production due to knowledge spillovers and heterogeneous labor shares call for a dynamic environmental policy to permanently keep emissions at net-zero. 
 
 \begin{itemize}
 	\item dynamics: I expect rising over time due to capital ratio heterogeneity and knowledge spillovers
 	\item role of taul as a level effect versus compositional effect
 	\item advantages under policy tuple progressive income tax and lower fossil tax as opposed to only fossil tax?
 \end{itemize}

\paragraph{The role of knowledge spillovers}
Knwoledge spillovers render a higher fossil tax and a more aggressive environmental policy over time necessary. 
More interestingly,firstly, they affect how progressive income taxes shape the required fossil tax. Knowledge spillovers mitigate the adverse compositional effect of progressive income taxes on the green-to-fossil energy share. %As a result, the required fossil tax deviates more from the fossil tax when no progressive income tax is present. 

Secondly, knowledge spillovers make a fossil tax more costly in terms of technology growth in all sectors.  A lower fossil tax acceptable under a higher labor tax progressivity allows for higher growth in all sectors. 
Absent knowledge spillovers growth in the non-energy sector is indeed higher than absent a progressive tax, but it becomes smaller when a more aggressive fossil tax is required. Then, the reductive effect of the progressive income tax does not suffice to diminish the fossil tax in a way that non-energy research increases. Then, the progressive income tax in addition to the fossil tax both lower non-energy research and growth. With knowledge spillovers, non-energy research and growth are higher; a positive spillover effect of fossil growth. The green sector, too, profits from knowledge spillovers from the fossil sector. Green growth eventually is higher under the policy with progressive 

However, overall, the benefits of a lower fossil tax in presence of a higher tax progressivity come at the cost of lower consumption and a smaller green-to-fossil ratio and a higher energy share. Yet, the latter two are acceptable due to a smaller level of production. 

\paragraph{Role of heterogeneous labor shares}
Heterogenous capital shares increase the necessary fossil tax to meet emission limits since a positive supply side effect is muted. 
With equal capital shares the necessary fossil tax is almost halfs during the net-zero emission period. 
As the fossil sector lowers demand for labor, the green sector profits from a higher labor supply, yet, less so under a lower labor share.  


With equal capital shares the fossil tax in presence of a progressive income tax is, however, closer to the required fossil tax absent progressive labor taxation.  The reason is that the compositional effect of the  progressive labor tax on skill supply is less devastating for green production when the green sector relies less on labor. Then, a bigger reduction in the fossil tax is possible while meeting emission limits. 

\paragraph{Role of heterogeneous skills}

The reuired fossil tax to meet emisison limits is higher when skills are homogeneous. The reason is that the resources for fossil production increase.
When skills are homogeneous, labor income tax progressivity has no compositional effect on the economic structure. One might expect that, therefore, a stronger reduction in the fossil tax is admissible. 

One explanation is that with homogeneous skills, similar to heterogeneous capital shares, there is no supply side effect when skills are homogeneous triggered by the fossil tax. Instead, labor supply is unaffected. With heterogeneous skills, however, the fossil tax has a compositional effect on labor supply which makes green production less costly. A reduction in the fossil tax does not harm the green to fossil ratio as much when skills are heterogeneous. 

\paragraph{Role of endogenous growth}
In line with \cite{Fried2018ClimateAnalysis}, I find that a smaller fossil tax is required when growth is endogenous because the shift in research intensifies the effect of the fossil tax.  
