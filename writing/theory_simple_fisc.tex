\section{Theoretical Results}\label{sec:theory}
\begin{itemize}
	\item laissez faire
	\item optimal policy results
\end{itemize}

In this section, I discuss the main theoretical results. 

\paragraph{Laissez-faire}
\paragraph{Dirty output}

Dirty output grows with the following rate
\begin{align*}
	\frac{Y_d'}{Y_d}=\left(\frac{p_d'}{p_d}\right)^{\frac{\alpha+(1-\alpha)(1-\varepsilon)}{1-\alpha}}\frac{A_d'}{A_d}\frac{H'}{H}.
\end{align*}
The effect of sector-specific inflation on output captures, on the one hand, the positive effect of a higher demand for machines when the price for the respective good grows. This raises the marginal product of labour so that the dirty sector demands more labour; captures by the $\frac{\alpha}{1-\alpha}$ in the exponent. In addition,  a rise in the dirty good's price increases the marginal profit the dirty sector generates from increasing labour input one-for-one, this is captured by the exponent of 1. On the other hand,  the rise in the dirty good's price lowers demand for dirty output by final goods producers, even more so the more goods are substitutes, captured by the $\varepsilon$. The elasticity of dirty output with respect to dirty sector inflation is positive, when the reduction in demand by final good producers, $\varepsilon$,  is smaller than the rise in the marginal profit of labour, $\frac{1}{1-\alpha}$. 
All else equal, a rise in dirty productivity increases dirty output, as does a rise in aggregate disutility-weighted labour supply.

Since prices in this simple model are a function of total factor productivity only, the government can only affect dirty output growth through total labour supply by households; the output ratio which is irresponsive to progressivity of the tax schedule.%The percentage change in labour input goods by sectors are equivalent. This together with prices being independent of skill supply implies that the output ratio of sectors is unaffected by tax progressivity.
\footnote{\ This is directly obvious from the ratio demand by final good producers, which is only a function of the price ratio. }


	\begin{comment}
	To see this write:
\begin{align}
	\frac{d\left(\frac{Y_d}{Y_c}\right)}{d \tau_l}=\frac{Y_d}{Y_c}\left(\frac{\frac{dY_d}{Y_d}}{d \tau_l}-\frac{\frac{dY_c}{Y_c}}{d \tau_l}\right)=0
\end{align}
and observe that the percentage change in sector output is homogeneous. 
\begin{align}
	\frac{1}{Y_d}\frac{dY_d}{d \tau_l}= \frac{1}{L_d}\frac{d L_d}{d \tau_l}=\frac{1}{H}\frac{d H}{d \tau_l}\ \text{and} \ \frac{1}{Y_c}\frac{dY_c}{d \tau_l}= \frac{1}{L_c}\frac{d L_c}{d \tau_l}=\frac{1}{H}\frac{d H}{d \tau_l}.
\end{align}
\textbf{}
content...
\end{comment}






\begin{prop}[Effect of $\tau_l$ on dirty output]
	In the representative agent model with log utility, tax progressivity does not affect the equilibrium ratio of sector production. Fiscal policy can only lower dirty output by reducing aggregate output.
\end{prop}

