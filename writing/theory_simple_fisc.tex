\section{Theoretical Results}\label{sec:theory}
\begin{itemize}
	\item laissez faire
	\item optimal policy results
\end{itemize}

In this section, I discuss the main theoretical results. 

The laissez-faire economy does violate the emission target. 

\paragraph{Dirty output}

Dirty output grows as
\begin{align*}
	\frac{Y_d'}{Y_d}=\left(\frac{p_d'}{p_d}\right)^{\frac{\alpha-(1-\alpha)(1-\varepsilon)}{1-\alpha}}\frac{A_d'}{A_d}\frac{H'}{H}
\end{align*}
When goods are substitutes, $\varepsilon>1$, the dirty good's price reduces labour supply in this sector. When goods are complements, labour supply in the dirty sector increases as the dirty goods price rises...

growth in the dirty sector falls with inflation in this sector: $\alpha -(1-\alpha)(1-\varepsilon)>0$, then the growth in machines exceeds 
\subsection{Results}
As a result, the percentage change in labour input goods by sectors are equivalent. This together with prices being independent of skill supply implies that the output ratio of sectors is unaffected by tax progressivity.
To see this write:
\begin{align}
	\frac{d\left(\frac{Y_d}{Y_s}\right)}{d \tau_l}=\frac{Y_d}{Y_c}\left(\frac{\frac{dY_d}{Y_d}}{d \tau_l}-\frac{\frac{dY_c}{Y_c}}{d \tau_l}\right)=0
\end{align}
and observe that the percentage change in sector output is homogeneous. 
\begin{align}
	\frac{1}{Y_d}\frac{dY_d}{d \tau_l}= \frac{1}{L_d}\frac{d L_d}{d \tau_l}=\frac{1}{H}\frac{d H}{d \tau_l}\ \text{and} \ \frac{1}{Y_c}\frac{dY_c}{d \tau_l}= \frac{1}{L_c}\frac{d L_c}{d \tau_l}=\frac{1}{H}\frac{d H}{d \tau_l}.
\end{align}






\begin{prop}[Effect of $\tau_l$ on output ratio]
	In the representative agent model with log utility and no disposal of government revenues, tax progressivity does not affect the equilibrium ratio of sector production. Only total output reduces as progressivity rises. \tr{directly obvious from seeing that ratio is constant!}
\end{prop}

\paragraph{Welfare}