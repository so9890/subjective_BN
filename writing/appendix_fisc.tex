\appendix
\section{Appendix}
\subsection{Growth and the Environment}
It is a vibrant debate whether technological process will result in a production technology that is perfectly clean in that it does not exert any environmental externality. 
\begin{itemize}
	%\item \underline{Extensions to technology in \cite{Acemoglu2012TheChange} }
	%\begin{itemize}
	\item \underline{externality of ``clean'' sector} \citep[see also][]{Dasgupta2021, Brock2005ChapterEmpirics}
	\begin{itemize}
		\item[-] renewable/ non-fossil fuels \ar externalities in production process are present e.g. production of solar panels uses toxic inputs \citep{Yue2014DomesticAnalysis}; non-fossil fuel nitrogen generation (e.g., biomass burning to clear land) important ($\approx$ 50\%) \citep{Song2021ImportantEmissions}; low but chronical levels of nitrogen cause species extinctions \citep{Clark2008LossGrasslands}
		\item[-] waste (after use) \ar depends on recycling technology %\ar recycling system for solar panels not profitable enough today
		%	\item[-] substitutability of nature in production (input sources eg. fossil vs. non-fossil fuels)
		%\end{itemize}
		%\item Irreversibilities already before thresholds are hit (e.g. species extinction)
		
	\end{itemize}
	%\item greenhouse gases: Carbon dioxide $CO_2$ (vast majority), Nitrous oxide $N_2O$, methane $CH_4$
	%\item stock of nature globally determined
	\item \underline{parallel positive trend in demand} (population growth, rebound effect) that outperforms increase in clean technology growth \small{(no long-run issue if perfectly clean technology exists)}
	\item \normalsize{\underline{objective function}:} \cite{Arrow2004AreMuch}(Journal of Economic Perspectives) \ar using a sustainability measure they provide evidence that consumption is too high (= not leaving enough natural resources for future generations)
	\item \underline{risk, ambiguity}
	\item if have to meet climate target in short run, might need to lower production to do so; or it might be better in terms of inequality?
\end{itemize}

\subsection{Greenhouse gas emissions and the Paris Agreement}

Two alternatives exist to specifiy the relation between the environment and production: (i) a broad approach considering natures use as a sink and as a resource, and all relevant pollutants. 
In order to determine \textit{relevant}, I refer to the planetary boundaries discussed in \cite{Rockstrom2009AHumanity}. (ii) a more specific approach that focuses on greenhouse gas emissions in particular which allows to draw on emission goals specified by country. Paris agreement goal: `'\textit{Climate neutral world by the mid-century}'' (source \url{https://unfccc.int/process-and-meetings/the-paris-agreement/the-paris-agreement}). In 2020 countries had to submit plansfor a \textit{long-term low ghg emissions} (LT-LEDS) where long term means mid-century (I assume). In the EU member states have submitted \textit{integrated national-energy and climate plans} (NECPS) (source \url{https://ec.europa.eu/info/energy-climate-change-environment/implementation-eu-countries/energy-and-climate-governance-and-reporting/national-long-term-strategies_en}). According to this source, emissions occur in the following fields: 
\textit{emission reductions and enhancements of removals in individual sectors, including \textbf{electricity, industry, transport, the heating and cooling and buildings sector (residential and tertiary), agriculture, waste and land use, land-use change and forestry (LULUCF)}}; the website also contains documents on country specific plans and actions

\subsection{Modelling choice: external emission target}
There is a multitude of uncertainties associated with quantitative models relating production and nature. 
In the Paris Agreement states have agreed on a legally binding maximum increase of temperature to 1.5° over pre-industrial levels. (CHECK). The global community seeks to be \textit{climate-neutral in 2050}. Complex physical mechanisms determine the interrelation between greenhouse gas emissions and temperature rises, for example, non-convexities, interaction with other geo processes and irreversibilities. 
Since scientists better qualified to assess these complexities and risks, I do not model the interaction of emissions and the climate explicitly. Instead, I draw from the scientific literature to specify emission targets. These enter as constraints into the objective function of the government. (uncertainties include: regeneration rates of the environment, e.g. carbon sinks, and how emissions translate into temperature.)
However, an important uncertainty remains: what degree of emission reduction can be achieved by technological progress in the specified time frame? I use different specifications of technological possibilities: (i) a scenario where technological progress is possible to reduce emissions to zero, (ii) a scenario with medium reduction possibility until mid-century, (iii)