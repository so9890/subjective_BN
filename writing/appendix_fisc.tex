\appendix
\section{Appendix}
\subsection{Growth and the Environment}
It is a vibrant debate whether technological process will result in a production technology that is perfectly clean in that it does not exert any environmental externality. 
\begin{itemize}
	%\item \underline{Extensions to technology in \cite{Acemoglu2012TheChange} }
	%\begin{itemize}
	\item \underline{externality of ``clean'' sector} \citep[see also][]{Dasgupta2021, Brock2005ChapterEmpirics}
	\begin{itemize}
		\item[-] renewable/ non-fossil fuels \ar externalities in production process are present e.g. production of solar panels uses toxic inputs \citep{Yue2014DomesticAnalysis}; non-fossil fuel nitrogen generation (e.g., biomass burning to clear land) important ($\approx$ 50\%) \citep{Song2021ImportantEmissions}; low but chronical levels of nitrogen cause species extinctions \citep{Clark2008LossGrasslands}
		\item[-] waste (after use) \ar depends on recycling technology %\ar recycling system for solar panels not profitable enough today
		%	\item[-] substitutability of nature in production (input sources eg. fossil vs. non-fossil fuels)
		%\end{itemize}
		%\item Irreversibilities already before thresholds are hit (e.g. species extinction)
		
	\end{itemize}
	%\item greenhouse gases: Carbon dioxide $CO_2$ (vast majority), Nitrous oxide $N_2O$, methane $CH_4$
	%\item stock of nature globally determined
	\item \underline{parallel positive trend in demand} (population growth, rebound effect) that outperforms increase in clean technology growth \small{(no long-run issue if perfectly clean technology exists)}
	\item \normalsize{\underline{objective function}:} \cite{Arrow2004AreMuch}(Journal of Economic Perspectives) \ar using a sustainability measure they provide evidence that consumption is too high (= not leaving enough natural resources for future generations)
	\item \underline{risk, ambiguity}
	\item if have to meet climate target in short run, might need to lower production to do so; or it might be better in terms of inequality?
\end{itemize}

\subsection{Greenhouse gas emissions and the Paris Agreement}

Two alternatives exist to specifiy the relation between the environment and production: (i) a broad approach considering natures use as a sink and as a resource, and all relevant pollutants. 
In order to determine \textit{relevant}, I refer to the planetary boundaries discussed in \cite{Rockstrom2009AHumanity}. (ii) a more specific approach that focuses on greenhouse gas emissions in particular which allows to draw on emission goals specified by country. Paris agreement goal: `'\textit{Climate neutral world by the mid-century}'' (source \url{https://unfccc.int/process-and-meetings/the-paris-agreement/the-paris-agreement}). In 2020 countries had to submit plansfor a \textit{long-term low ghg emissions} (LT-LEDS) where long term means mid-century (I assume). In the EU member states have submitted \textit{integrated national-energy and climate plans} (NECPS) (source \url{https://ec.europa.eu/info/energy-climate-change-environment/implementation-eu-countries/energy-and-climate-governance-and-reporting/national-long-term-strategies_en}). According to this source, emissions occur in the following fields: 
\textit{emission reductions and enhancements of removals in individual sectors, including \textbf{electricity, industry, transport, the heating and cooling and buildings sector (residential and tertiary), agriculture, waste and land use, land-use change and forestry (LULUCF)}}; the website also contains documents on country specific plans and actions

\subsection{Modelling choice: external emission target}\label{app:emission_climate_targets}
There is a multitude of uncertainties shaping the relation of production, on the one hand, and nature and climate warming, on the other hand. These uncertainties relate to (i) the technological possibilities to reduce emissions in the future and (ii) the relation of emissions and the climate. 

On the other hand, in the Paris Agreement clear political goals have been formulated in 2015. Under this treaty, states have agreed on a legally binding maximum increase in temperature to well below 2°C, preferably 1.5° over pre-industrial levels, and the global community seeks to be climate-neutral in 2050 \url{https://unfccc.int/process-and-meetings/the-paris-agreement/the-paris-agreement}. 

\paragraph{Uncertainty 1): Emissions $\rightarrow$ temperature}
Carbon dioxide has been the focus of the literature integrating climate change and economic models \citep[such as,][]{Golosov2014OptimalEquilibrium,Barrage2019OptimalPolicy}. 
The (geo-)physical mechanisms which determine the interrelation between carbon emissions and temperature changes are highly uncertain and complex. For example, (1) there is no good understanding of the relation of CO2 and the climate as the temperature rises to certain limits, (2)  feedback of the Earth system, such as permafrost thawing, has to be taken into account, as well as (3) interactions of carbon with non-CO2 emissions, (\citep[][p.96, 2nd paragraph]{Rogelj2018MitigationDevelopment.}).  Uncertainty also surrounds the regeneration rate of the environment \citep{Acemoglu2012TheChange} and irreversibilities (CITE HAssler handbook chapter here). In a quantitative study on optimal environmental policies, hence, a lot of assumptions and simplifications have to be made. 

In chapter 2 of the
\textit{IPCC Special Report} \citep{Rogelj2018MitigationDevelopment.}, scientists quantify emission pathways to meet the 1.5°C goal of the Paris Agreement by carefully taking uncertainties and the complex geophysical processes into account. I use these limits on emissions as constraints to the government's objective function. This approach is clearly policy relevant, while at the same time reduces the need to make (geophysical) assumptions. Furthermore, it allows me to take other important non-CO2 emissions into account, too. (\textit{Look at the discussion of integrated assessment models in \cite{Hassler2016EnvironmentalMacroeconomics} for the advantages of integrating a simplified carbon cycle into macro models (\ar dynamics) })
%\tr{\ar In a nutshell, I take from these reports the emission reduction pathways. I do have to make assumptions on the possibilities of technological innovations. No carbon cycle needed but less assumptions have to be made.}

\paragraph{Uncertainty 2: Technological progress $\rightarrow$ emissions}
An important modelling uncertainty remains: what degree of emission reduction can be achieved by technological progress in the specified time frame? I use different specifications of technological possibilities: (i) a scenario where technological progress is sufficient to reduce emissions to zero until 2050 at current consumption levels per capita, (ii) and one where innovation steps are insufficient.