\section{Literature}


\subsection{Models}
\begin{itemize}
\item \cite{Bilbiie2012EndogenousCycles}
\begin{itemize}
\item a model with rep agent
\item investment in the form of stock 
\item innovation as a form of new products
\item one final good sector
\item monopolistic competition
\item homothetic preferences
\end{itemize}
\item \cite{Ravn2006DeepHabits}
\begin{itemize}
 \item habits over average previous consumption of specific good! not over total consumption
 \item rep agent 
 \item habits: marginal utility rises as habits rise \ar could look at what happens as habits are reduced! \ar marginal utility at given consumption level reduces!
 \item more is always better! Plus increases habits \ar I want: that more might not be better after some point
\end{itemize}
\item \cite{McKay2021LumpyPolicy}
\begin{itemize}
\item New Keynesian model with durable and non-durable consumption 
\end{itemize}
\item \cite{Acemoglu2012TheChange}
\begin{itemize}
\item endogenous growth
\item rep agent
\item single labour market
\end{itemize}
\item \cite{Heikkinen2015DegrowthConsumers}: macro model with voluntary reduction in consumption
\item \cite{Borissov2019CarbonDevelopment}: model labour sector in more detail: skill, sectors, and transition
\item \cite{Michaillat2015AggregateUnemployment, Auerbach2021InequalityEconomy} examples of models with economic slack. But both do not feature a satiation point of consumption. 
\end{itemize}

