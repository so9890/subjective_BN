\section{Literature}


\subsection{Models}
\begin{itemize}
\item \cite{Bilbiie2012EndogenousCycles}
\begin{itemize}
\item a model with rep agent
\item investment in the form of stock 
\item innovation as a form of new products
\item one final good sector
\item monopolistic competition
\item homothetic preferences
\end{itemize}
\item \cite{Ravn2006DeepHabits}
\begin{itemize}
 \item habits over average previous consumption of specific good! not over total consumption
 \item rep agent 
 \item habits: marginal utility rises as habits rise \ar could look at what happens as habits are reduced! \ar marginal utility at given consumption level reduces!
 \item more is always better! Plus increases habits \ar I want: that more might not be better after some point
\end{itemize}
\item \cite{McKay2021LumpyPolicy}
\begin{itemize}
\item New Keynesian model with durable and non-durable consumption 
\end{itemize}
\item \cite{Acemoglu2012TheChange}
\begin{itemize}
\item endogenous growth
\item rep agent
\item single labour market
\item no resource use in clean sector! ; abstracts from waste
\item disaster risk!: There is a lower bound on the quality of the environment 
\item environmental externality only affects Utility! So no chance for \textbf{environmental quality} to drive production to zero!
BUT there is a natural resource which is used in production; \textit{How do the two relate?} \ar when environmental quality affects regeneration of exhaustible resource, then there would be some connection, but there is no regeneration of the resource, I think
\item there is degradation of the environment through unsustainable production (only!) and 
\end{itemize}
Functional forms
\begin{align*}
S\in[0,\bar{S}],\ & \text{where}\ \bar{S}\ \text{is the quality of the environment without pollution;}\\
S_v=0 \Rightarrow S_t=0 \forall t\geq v,\ &  0 \ \text{is the point of no return.}\\
\underset{S\rightarrow0}{lim} U(C,S)=-\infty\ & \text{S=0 is a disaster!}\\
\underset{S\rightarrow0}{lim}\frac{\partial U(C,S)}{\partial S}=\infty\ &\\
S_{t+1}= -\xi Y_{dt}+(1+\delta)S_t& \\ 
\text{evolution of environmental quality:} & \text{ falls in dirty production; regeneration rate }\\
 \text{both are exponential relationships}\Rightarrow&\text{ smaller env. quality slower regeneration}\\ 
 &\text{ higher pollution, stronger degradation}
\end{align*}
The dirty sector uses an exploitable resource in the production process
\begin{align*}
Y_{dt}= R_t^{\alpha_2}L_{dt}^{1-\alpha}\int_{0}^{1}A_{dit}^{1-\alpha_1}x_{dit}^{\alpha_1}di
\end{align*}
$R_t$ is the exhaustible resource
\begin{align*}
Q_{t+1}=Q_t-R_t
\end{align*}
they look at a version where the resource is common property (water, air) or owned (Hotelling rule)
\item \cite{Heikkinen2015DegrowthConsumers}: macro model with voluntary reduction in consumption
\item \cite{Borissov2019CarbonDevelopment}: model labour sector in more detail: skill, sectors, and transition
\item \cite{Michaillat2015AggregateUnemployment, Auerbach2021InequalityEconomy} examples of models with economic slack. But both do not feature a satiation point of consumption. 
\end{itemize}

\subsection{Motivation}
\begin{itemize}
\item \cite{Schor2005SustainableReduction}
\begin{itemize}
	\item arguments against unlimited growth
	\begin{itemize}
\item hhh
	\end{itemize}
\end{itemize}
\item \cite{Dasgupta2021}
\begin{itemize}
\item emphasises the use of nature as a sink (stock) and as an input to production \ar can the two be combined?
\end{itemize}
\end{itemize}
