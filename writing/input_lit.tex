\section{Literature}


\subsection{Models}
\begin{itemize}
\item \cite{Bilbiie2012EndogenousCycles}
\begin{itemize}
\item a model with rep agent
\item investment in the form of stock 
\item innovation as a form of new products
\item one final good sector
\item monopolistic competition
\item homothetic preferences
\end{itemize}
\item \cite{Ravn2006DeepHabits}
\begin{itemize}
 \item habits over average previous consumption of specific good! not over total consumption
 \item rep agent 
 \item habits: marginal utility rises as the habit rises \ar could look at what happens as habits are reduced! \ar marginal utility at given consumption level reduces!
 \item more is always better! Plus increases habits \ar I want: that more might not be better after some point
\end{itemize}
\item \cite{McKay2021LumpyPolicy}
\begin{itemize}
\item New Keynesian model with durable and non-durable consumption 
\end{itemize}
\item \cite{Acemoglu2012TheChange}
\begin{itemize}
\item endogenous growth
\item rep agent
\item single labour market
\end{itemize}
\item \cite{Heikkinen2015DegrowthConsumers}
\item \cite{Borissov2019CarbonDevelopment} modeling labour sector in more detail: skill, sectors, and transition
\end{itemize}

\subsection{Empirical evidence}
\subsubsection{Labour market: skills and green sector}
TO READ: \cite{Bowen2018CharacterisingComposition} jobs that would benefit from a green transition; \\
\cite{Consoli2016DoCapital} do green and non-green jobs differ wrt skills? (US): green jobs use more intensively high-level cognitive and interpersonal skills compared to non-green jobs. Green occupations also exhibit higher levels of standard dimensions of human capital such as formal eductaion, work experience and on-the-job training. 

\subsubsection{Consumption cross-section }
MY CONTRIBUTION: allow for quality of good to matter for emission (I think this is missing in the literature...the data there only depends on product types)

\subsubsection{Consumption time series}
How does consumption on household level vary over time? Is there evidence for households deliberately reducing their consumption? 
\ar for literature on that consult. \cite{Heikkinen2015DegrowthConsumers} (Macro model with upper bound, refers to empirical literature on that)

\subsubsection{EXTENSION: Durables versus non-Durable}
\textbf{Which one accounts for more resource usage?\ar if durables then would make more sense to look at these}
\\
Durables: early replacement as a cause for overconsumption due to social drivers \citep{Hou2020FeelingsIntentions}
\\ 
Including durables in model to look at policies which target usage of durables...
