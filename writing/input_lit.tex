\section{Literature}

\subsection{Empirics}
\subsubsection{EXTENSION: Durables versus non-Durable}
Which one accounts for more resource usage?
\\
Durables: early replacement as a cause for overconsumption due to social drives \citep{Hou2020FeelingsIntentions}

\subsection{Models}
\begin{itemize}
\item \cite{Bilbiie2012EndogenousCycles}
\begin{itemize}
\item a model with rep agent
\item investment in the form of stock 
\item innovation as a form of new products
\item one final good sector
\item monopolistic competition
\item homothetic preferences
\end{itemize}
\item \cite{Ravn2006DeepHabits}
\begin{itemize}
 \item habits over average previous consumption of specific good! not over total consumption
 \item rep agent 
 \item habits: marginal utility rises as the habit rises \ar could look at what happens as habits are reduced! \ar marginal utility at given consumption level reduces!
 \item more is always better! Plus increases habits \ar I want: that more might not be better after some point
\end{itemize}
\item \cite{McKay2021LumpyPolicy}
\begin{itemize}
\item New Keynesian model with durable and non-durable consumption 
\end{itemize}
\item \cite{Acemoglu2012TheChange}
\end{itemize}