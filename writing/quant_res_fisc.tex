\section{Quantitative results}\label{sec:simul}

\begin{comment}
Interesting quantitative results

\begin{itemize}
	\item comparison laissez faire in 2050 to optimal policy (net-zero emissions starting in 2050)
	\item present ratios and variables that are constant 
	\item How?: 
	\begin{enumerate}
	\item calculate values of endogenous and predetermined variables starting from today
	\item apply growth rates in laissez-faire and in optimal SS (this is static)\ar simulate the economy/ should be there already
	\item dynamics: shooting algorithm! / relaxation (pertubation (= approximation) to get starting values)
	\end{enumerate}
\end{itemize}
\end{comment}

\begin{comment}
\paragraph{The effect of a higher disutility of high skill labour}

With a higher disutility in high skill labour, the share of the dirty good in production rises. 

\begin{figure}[h!!]
\includegraphics[width=1\textwidth]{../codding_model/Own/figures/Rep_agent/Yd_Yc_ratio_periods10_eppsilon0.40_zeta1.40_Ad08_Ac04_thetac0.70_thetad0.56_fullDisp0_HetGrowth1_tauul0.181_util0.png}
\caption{Effect of the scarcity of labour on the output ratio}
\end{figure}
\end{comment}

In this section, I compare the evolution of the economy under the laissez-faire calibration to the evolution under the optimal policy. First, I show results without the emission target. Second, results with emission target are discussed.

The comparison of the optimal to the laissez-faire allocation in the scenario without emission target shows that no government is optimal. While in the laiseez
\begin{minipage}{}
•
\end{minipage}•