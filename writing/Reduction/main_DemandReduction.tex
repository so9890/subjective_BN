\documentclass[12pt]{article}
\usepackage[utf8]{inputenc}
\usepackage{xcolor}
\usepackage{graphicx}
\usepackage{listings}
\usepackage{epstopdf}
\usepackage{etoc}
\usepackage{pdfpages}
\usepackage[capposition=top]{floatrow}
\usepackage{pdflscape} % landsacpe package
% set font to times
%\usepackage{mathptmx} % times!!! 
%\usepackage[T1]{fontenc}
\usepackage{amsmath}
\usepackage{soul}
\usepackage[left=2.5cm, right=2.5cm, top=2.5cm, bottom =2.5cm]{geometry}
\usepackage{natbib}
%\usepackage[natbibapa]{apacite}
%\usepackage{apacite}
%\bibliographystyle{apacite}
\bibliographystyle{apa}
%\renewcommand{\footnotesize}{\fontsize{10pt}{11pt}\selectfont}
\usepackage[onehalfspacing]{setspace}
\usepackage{listings}
\renewcommand{\figurename}{\textbf{Figure}}
\renewcommand{\hat}{\widehat}
\usepackage[bf]{caption}
\usepackage{tikz}
%\begin{comment}
%\usepackage[headsepline,footsepline]{scrlayer-scrpage} % has to come before package!!! otherwise option clash
%\usepackage{scrlayer-scrpage}
%\pagestyle{scrheadings} % kopfzeile/ fußzeile
%\clearpairofpagestyles
%\ohead{}
%\ihead{\textit{Redistribution, Demand and  Sustainable Production}}
%\cfoot{\thepage}
%\pagestyle{plain} % comment this one to have header
%\end{comment}
\allowdisplaybreaks
\usepackage{comment}
 \usepackage{siunitx}
  \usepackage{textcomp}
\definecolor{sonja}{cmyk}{0.9,0,0.3,0}
%\definecolor{purple}{model}{color-spec}
\usepackage{amssymb}
\newcommand{\ar}{$\Rightarrow$ \ }
\newcommand{\frp}[2]{\frac{\partial{#1}}{\partial{#2}}}
\newcommand{\tr}[1]{\textcolor{red}{#1}}
\newcommand{\vlt}[1]{\textcolor{violet}{#1}}
\newcommand{\bl}[1]{\textcolor{blue}{#1}}
\newcommand{\sn}[1]{\textcolor{sonja}{#1}}
%%% TIKZS
\usepackage{tikz}
\usetikzlibrary{mindmap,trees}
\usetikzlibrary{backgrounds}
\tikzstyle{every edge}=  [fill=orange]  
\usetikzlibrary{tikzmark}
\usetikzlibrary{decorations.markings}
\usepackage{tikz-cd}
\usetikzlibrary{arrows,calc,fit}
\tikzset{mainbox/.style={draw=sonja, text=black, fill=white, ellipse, rounded corners, thick, node distance=5em, text width=8em, text centered, minimum height=3.5em}}
\tikzset{mainboxbig/.style={draw=sonja, text=black, fill=white, ellipse, rounded corners, thick, node distance=5em, text width=13em, text centered, minimum height=3.5em}}
\tikzset{dummybox/.style={draw=none, text=black , rectangle, rounded corners, thick, node distance=4em, text width=20em, text centered, minimum height=3.5em}}
\tikzset{box/.style={draw , rectangle, rounded corners, thick, node distance=7em, text width=8em, text centered, minimum height=3.5em}}
\tikzset{container/.style={draw, rectangle, dashed, inner sep=2em}}
\tikzset{line/.style={draw, very thick, -latex'}}
\tikzset{    pil/.style={
		->,
		thick,
		shorten <=2pt,
		shorten >=2pt,}}
	
% other stuff
\newcommand{\innermid}{\nonscript\;\delimsize\vert\nonscript\;}
\newcommand{\activatebar}{%
	\begingroup\lccode`\~=`\|
	\lowercase{\endgroup\let~}\innermid 
	\mathcode`|=\string"8000
}
%\usepackage{biblatex}
%\addbibresource{bib_mt.bib}
\usepackage{ulem}
\title{Fiscal Policy, Inequality, and the Environment}
%\title{The Environment, Inequality, and Growth\\ \small{ optimal fiscal policy in an endogenous growth model with inequality and emission targets}}
\date{Sonja Dobkowitz\\ Bonn Graduate School of Economics\\ %University of Bonn\\
\vspace{1mm}
%Preliminary and incomplete\\
First version: January 9, 2022\\
This version: \today }
\usepackage{graphicx,caption}
\usepackage{hyperref}
\usepackage{minitoc}
\setcounter{secttocdepth}{5}
\usetikzlibrary{shapes.geometric}

% for tabular

%\usepackage{array}
\usepackage{makecell}
\usepackage{multirow}
\usepackage{bigdelim}

%propositions etc
\newtheorem{prop}{Proposition}
\newtheorem{corollary}{Corollary}[prop]
\newtheorem{lemma}[prop]{Lemma}

\renewenvironment{abstract}
{\small
	\list{}{
		\setlength{\leftmargin}{0.025\textwidth}%
		\setlength{\rightmargin}{\leftmargin}%
	}%
	\item\relax}
{\endlist}
\begin{document}
%	\includepdf[pages=-]{../titlepage.pdf}
	\maketitle
	\begin{comment}THIS ABSTRACT REFERS TO DEMAND REDUCTION POLICIES. whILE THE OTHER ONE BELOW ONLY MOTIVATES TO LOOK AT DISTORTIONARY INCOME TAXES AS AN ALTERNATIVE TO FOSSIL TAXES
	\begin{abstract}
		\begin{singlespacing}
			\textbf{Abstract \ }
			Natural scientists have identified a reduction of demand %for energy and land %thus, a change in lifestyle 
			as an important contributor to meeting global climate targets. However, a general equilibrium analysis of a demand reduction is missing.
			While as a direct effect environmental pollution declines through lowering production, it may have counteracting general equilibrium effects:
			A recomposition of relative skill supply shifts the economy away from green energy to fossil production. Directed technical change reacts by allocating research to dirty energy.
		\textbf{ I study the effect of a (voluntary) reduction in demand on the optimal environmental policy in a model accommodating directed technical change and skill heterogeneity.} When a voluntary reduction of demand is missing, the government can attain a reduction through distortionary income taxation. 

		\end{singlespacing}
		
		\end{abstract}
		\end{comment}
	\begin{abstract}
		\begin{comment}THIS ABSTRAC FOCUSES ON OPTIMAL INCOME TAXES BUT HAS A DIFFERENT TRADE OFF AND FOCUS: THE ENVIRONMENT´
		\end{comment}
		\begin{singlespacing}
			\textbf{Abstract \ }
			The latest IPCC report has identified an alarming gap between climate change policies and emission targets compatible with climate goals.\footnote{\ Countries' current emission targets reach 52-60 GtCO2-eq per year in 2030 and to 46-67 GtCO2-eq per year by 2050 instead of a 1.5 degree compatible limit of net-zero emissions in 2050. The estimated temperature rise induced by countries' policy targets is 2.4°C or 3.5°C by 2100. } Regulators face challenges when implementing carbon taxes, as demonstrated by the yellow vest movement in France 2018 or the recent cap on energy prices in reaction to rising fuel prices caused by the war in Ukraine.
			Distortionary income taxes constitute an alternative policy instrument. As a positive direct effect they lower the level of production as households diminish their labour supply. However, a recomposition of relative skill supply shifts the economy from green energy to fossil production. Directed technical change aggravates the shift by allocating research to dirty energy. 
			On the other hand, redistribution to poor households may boost aggregate demand for cleaner goods. 
			What is the environmentally optimal tax progressivity?
		\end{singlespacing}
	\end{abstract}
%\tableofcontents

%
\tableofcontents
\section{Introduction}
An equity-efficiency trade-off is central to the discussion of optimal labour income taxation and tax progressivity in the public finance literature.  The benefits of labour taxes and progressivity arise from redistribution. With concave utility specifications full redistribution is efficient. However, the optimal tax system does not feature full redistribution when labour supply is endogenous. Instead, redistribution is traded off against aggregate output as individuals reduce their labour supply and skill investment in response to labour income taxation. 

%\paragraph{An environmental perspective}

 Adding environmental externalities to the classical public finance framework changes the perception of efficiency costs. Instead of merely reducing welfare, direct benefits through a reduction of the externality arise by lowering output. 
Yet, in general, in the presence of corrective tax instruments and the assumption of a clean production alternative, %which direct production to a non-polluting alternative, however,
distortionary labour taxes are not employed as environmental policy instruments.\footnote{ NOTE: e.g Compare my paper on sustainable production and demand where there is a clean alternative available and no environmental disaster.} 
However, technological progress might not suffice to meet climate targets while leaving per capita consumption (growth) unchanged. This is especially relevant given the tight time frame foreseen by the Paris Agreement.\footnote{\ Under this treaty, states have agreed on limiting temperature rise to well below 2°C, preferably 1.5°C and to achieve climate neutrality by mid-century \url{https://unfccc.int/process-and-meetings/the-paris-agreement/the-paris-agreement}. } Indeed, in an report by the Intergovernmental Panel on Climate Change (IPCC)\footnote{\ A body of the United Nations established to assess the science related to climate change.},  \cite{Rogelj2018MitigationDevelopment.} advocate additional demand-side measures to lower demand for energy and greenhouse gas intense products (p.97 bottom) to meet agreed-on climate targets.\footnote{\tr{Notes: I am unsure if distortionary taxes would indeed arise as an optimal environmental policy tool when corrective taxes in both sectors are available. Maybe need some other argument, e.g. not available, why?; some political argument; inequality might be another if poor households consume a bigger polluting share; Changing already established policy measures less difficult than introducing new taxes?...I would make this framing dependent on the findings. If I focus on tax progressivity:  progressivity could lower consumption of consumption-rich households in particular which could contain some sort of advantage. A general consumption tax might serve as a baseline corrective tax and the explicit corrective tax accounts for the gap between the vat consumption tax and the social cost of the dirtier sector. }}

%\paragraph{Consumption reduction (degrowth!)}
% 
% Growth drag may result in papers which study the trade-off between consumption growth and the environment \citep{Acemoglu2012TheChange, Stokey1998AreGrowth, Jones2016LifeGrowth}. %\citep{Rockstrom2009AHumanity}\footnote{\ In their article, \cite{Rockstrom2009AHumanity} argue that several planetary boundaries have already been surpassed. Uncertainties about their interaction are another argument in favour of more cautious environmental policies. } such as biodiversity and nitrogen xxx.
%Taken this broader view on environmental pollution through economic activity, it becomes more questionable, if a perfectly clean technology exists or can ever be innovated.  

%In sum, the need to reduce environmental externalities rather quickly and the potential non-existence of perfectly green technologies could make it optimal to use classical fiscal tax instruments for environmental concerns through the efficiency channel. 

%\textit{So far: efficiency-equity channel; plus on efficiency side}
%\paragraph{Trade-offs: environmental externality}
 Consumption reduction in affluent countries has also been promoted by some (non-economic) scholars \citep{Schor2005SustainableReduction, Pullinger2014WorkingDesign, Arrow2004AreMuch}. But, the general equilibrium effects are less well understood.
While having an advantageous direct effect on the externality, counteracting indirect effects arise in a general equilibrium framework. Proponents of a reduction policy especially focus on consumption by the rich which consume a higher amount of natural resources.\footnote{\ Note Sonja: Abstracting from inequality, would it still be best to reduce consumption by the rich when the poor have a higher marginal propensity to consume dirty? \textit{Could be an important aspect in the model}. Not in the baseline, look at it in an extension...}
This concern could add to the benefits of tax progressivity.
However, targeting these households in particular for environmental reasons, in contrast, will lower the supply of high skilled labour.\footnote{\ The relation of labour tax progressivity and skill investment has been studied by \cite{Heathcote2017OptimalFramework}.} Yet, these skills are essentially important in greener sectors of the economy \citep{Consoli2016DoCapital}. As a result, dirty production becomes relatively cheaper and the dirty share of production rises. 

%The literature on optimal environmental policy knows the 
Endogenising the direction of technological progress might add to the adverse negative effects of a reduction policy:
Innovations might shift towards the more polluting good in response to a higher progressivity. The reduction in the cleaner sector's labour input good diminishes the relative market size of this sector. The market effect shifts innovation towards the bigger dirty sector. On the other hand, the higher price of cleaner products makes innovation in this sector more profitable, the price effect. 
One contribution of the paper is to discuss optimal fiscal policy in an endogenous growth setting. 


%\paragraph{Research question}
Given these counteracting forces, this paper seeks to answer whether a more progressive labour tax is indeed optimal for environmental reasons. 
%\tr{Why not use one corrective tax in each sector? Political argument? }

%\paragraph{Empirical motivation}
The key channel inducing adverse consequences of a reduction policy on the externality hinges on the skill-bias in the cleaner sector \textbf{and in the innovation sector}. % and that consumption-rich households are high skill households. \ar that they are richer is an equilibrium effect 
For the US, \cite{Consoli2016DoCapital} find that within broader occupational groups greener occupations are skill-biased: on average, the degree of non-routine tasks is higher as are formal education, work experience, and on-the-job training. I link this evidence to the type of skills employed in the dirty and the cleaner sector in the model. %\ar might be able to abstract from CRRA utility function and to focus on the skill accumulation dimension...
% \cite{Bowen2018CharacterisingComposition} estimate a relatively low difference in skills between jobs and argue that this gap can be closed quickly by on-the-job trainings. 
\cite{Borissov2019CarbonDevelopment}
build an endogeneous skill accumulation model incorporating the skill-bias in the green sector. They motivate their study by citing policy recommendation papers and \cite{Vona2018EnvironmentalExploration}: "\textit{green skills are closely related to the design, production, management and monitoring of technology and conclude that education emerges as a critical ingredient in the policy mix to promote sustainable economic growth}". Hence, skills are relevant in two levels for cleaner production: (1) for the innovation process of clean technologies (development), and (2) their management (operation). (\cite{Fried2018ClimateAnalysis} endogenises researchers!); \ar when introducing technological change reduction policies might become especially important. 

%\paragraph{Model}
I study the effects of labour-tax progressivity  on the environmental externality in a tractable general equilibrium model with directed technical change \tr{(might drop endogenous growth from the baseline model depending on findings, see Roadmap in section \ref{sec:rm})}. The model builds on \cite{Heathcote2017OptimalFramework} and \cite{Acemoglu2012TheChange}.
In contrast to \cite{Heathcote2017OptimalFramework}, I abstract from idiosyncratic risk and the life-cycle component to focus on the medium to long run and the environment. 
The planner still faces the trade-off between equity and skill accumulation. 

The government chooses tax instruments to maximise a 
social welfare function. But, the choice is constrained by emission targets consistent with the Paris Agreement. The advantage of this approach is that it suffers less from  model misspesifications due to  uncertainties about how emissions affect the environment. Furthermore, it is closely related to the current political debate. Compare appendix section \ref{app:emission_climate_targets} for a more in depth discussion of this aspect.  


An important modelling uncertainty remains: what degree of emission reduction can be achieved by technological progress in the specified time frame? I will give careful attention to this aspect in course of the paper: analytically and quantitatively.\footnote{\ In the quantitative part, I use different specifications of technological possibilities: (i) a scenario where technological progress is sufficient to reduce emissions to zero until 2050 at current consumption levels, (ii) and one where it is not. \textit{ Look at how others estimate innovation steps. Still questionable if can draw conclusion from past innovations to future possibilities of innovations. }}


Three different formulations of the planner's objective function seem interesting: (1) a Utilitarian social welfare function, (2) the minimisation of the variance of reductions in household utility (which is meant to proxy for social tension), and (3) an objective function to minimise the consumption reduction of the poorest in a Rawlsian spirit. \tr{(I will start to work with the Utilitarian version and only add (2) and (3) if necessary, i.e., tax progressivity is unaffected by the environmental constraint.)}


%\paragraph{Exercise}
The paper is divided into two parts: an analytical and a quantitative one.
In the first part, I derive  conditions which shape the relation of production and emissions so that fiscal policies are employed for environmental reasons in presence  of corrective taxes.  
In the second part, I examine the optimal policy and transitions to the new steady state from a realistically calibrated current state of the economy in a quantitative version of the model\footnote{\ For example, as in  \cite{Fried2018ClimateAnalysis} I add additional features to the innovation process such as spillovers across sectors and a third neutral sector of production.}.

% Two possible setups seem interesting. (1) The government maximises a \textit{(possibly pareto-weighted, utilitarian)} social welfare function. This version allows to account for other planetary boundaries potentially interacted with the dominant climate boundary and carbon emissions. (2)
%(\textit{\textbf{Note:}}(3) A reduction policy can also become optimal when climate policies as required to meet the targets are socially unfeasible. For example, poor households consume a higher share of polluting goods, reducing overall consumption through labour taxes, instead, could be more favourable in terms of inequality. )

I perform two quantitative experiments to scrutinise the contribution of directed technological change and skill heterogeneity on the optimal policy. First, I rerun the analysis in versions of the model where either or both channels are shut down.  Second, I introduce the optimal policy of the amended model, e.g. the  version without skill heterogeneity, into the model with skill heterogeneity to learn how this aspect shapes the effect of the optimal policy on the externality. %\footnote{\ \tr{Note: These quantitative exercises }}

%\paragraph{Sensitivity}
First, I adjust the model for different specifications of the skill accumulation process which is essential in shaping the adverse effects of reduction policies on the environment.\footnote{\ For example, skill can be modelled as generating intergenerational spill-overs within households as in \cite{Borissov2019CarbonDevelopment}.} Second, I allow for additional heterogeneity in preferences: some households voluntarily reduce their consumption.\footnote{\ \tr{Compare my other proposal on voluntary reduction.: Suggestive evidence for $\underline{a\ rise}$ in the share of households which voluntarily reduce their consumption (of new products) and working hours. Yet, voluntary reduction seems to be uncorrelated with skills.}} \textit{Heterogeneity in the marginal propensity to consume could affect the role of fiscal policy on the externality potentially reducing its effectiveness.}\footnote{\ Could also study special utility functions which incorporate the arguments put forward by advocates of reduction policies such as: habits, social preferences, or  spill-overs of leisure \citep[][\textit{to be read}]{Alesina2005WorkDifferent}. These aspects imply inefficient high levels of consumption and work effort even absent an environmental externality.  Hence, they will add to the benefits of reducing consumption. 
}
Third, instead of fiscal policies the government has a policy tool to lower the maximum hours worked per worker and period \citep[\textit{compare}][]{Alvarez-Cuadrado2007EnvyHours}. Such a policy is closer to what proponents of a reduction policy suggest. % Third, more evolved modelling of the innovation process as in \cite{Fried2018ClimateAnalysis} is introduced \textit{(add this to the main quantitative part)}. 
\tr{These are some ideas, will decide while working on the paper what extensions to focus on.}
\\

\noindent\rule[1ex]{\textwidth}{1pt}


\paragraph{The below is an addendum in case I want to add capital taxation...} \ 
\\

\noindent
\textit{below relatively close to \cite{Conesa2009TaxingAll}} 
Another prominent debate in the public finance literature centres on the optimal size of the capital tax. The optimality of zero capital taxation has been argued for in the seminal work by Judd (1985) and affirmed by others.  However, as discussed by \cite{Conesa2009TaxingAll}, the optimality of a zero capital tax hinges upon the endogeneity of labour supply. The capital tax rises when allowing for life-cycle aspects. In this setting, it is optimal to tax capital heavily and to accept a reduction in consumption at a modest reduction of hours worked for the benefit of better insurance and redistribution. Taking environmental externalities into account, the optimal capital tax may even be higher due to its advantageous effect on the externality.

\cite{Domeij2004OnTaxes} discuss a reduction of capital taxes in a model with idiosyncratic productivity shocks which imply  wealth inequality. Their focus rests on the distributional effect which arise in the transition from the benchmark to a lower capital tax. They find huge adverse distributional effects over the transition since a linear labour tax is less progressive.  (\textit{check this}) The distributional effects outweigh the benefits of a higher capital stock and output. 

\tr{(I would first want to look at labour tax progressivity as I can build on the tractable model by \cite{Heathcote2017OptimalFramework}.)}
\\

\noindent\rule[1ex]{\textwidth}{1pt}

\paragraph{Literature}
\begin{itemize}
\item Public finance literature on optimal labour tax (progressivity) or capital tax (advantage of labour tax progressivity: there is a tractable model already available; capital tax with the zero capital tax finding seems interesting and important though...But requires a different model with capital) \citep{Heathcote2017OptimalFramework, Conesa2009TaxingAll, Domeij2004OnTaxes}
\item Optimal environmental policy with and without directed technical change \citep{Acemoglu2012TheChange, Acemoglu2016TransitionTechnology, Fried2018ClimateAnalysis, Barrage2019OptimalPolicy, Golosov2014OptimalEquilibrium, Hassler2016EnvironmentalMacroeconomics}
\item potentially: limits to growth \citep{Stokey1998AreGrowth, Jones2016LifeGrowth, Arrow2004AreMuch}
\item building on the following literature: \begin{itemize}
\item skills and green production: empirical \citep{Consoli2016DoCapital, Bowen2018CharacterisingComposition, Borissov2019CarbonDevelopment}; growth model with skill accumulation \citep{Borissov2019CarbonDevelopment}
\item literature on the environment and overconsumption: (i) economics \citep{Dasgupta2021, Brock2005ChapterEmpirics, Arrow2004AreMuch, Cohen2019AnnualSubstitutable}, (ii) natural sciences \citep{ Rockstrom2009AHumanity, Rogelj2018MitigationDevelopment.}
\item literature advocating reduction policies \citep{Schor2005SustainableReduction, Pullinger2014WorkingDesign}
%Some (non-economist) scholars argue for the necessity to reduce current consumption growth in developed economies due to threats to planetary boundaries \citep{Arrow2004AreMuch, Schor2005SustainableReduction}. 

\end{itemize}
\end{itemize}

\section{\tr{Roadmap}}\label{sec:rm}

\paragraph{To do next 21/01/22}
\ \\
\textbf{write 10 page paper with main ingredients: well motivated, simple rep agent model, what is my innovation: tax instrument and targets, stylized results (no focus on calibration)} \ar to hand in to conferences in February
\begin{itemize}
	\item simple simple model: rep agent, externality in form of target
	\item motivate to look at linear taxes/(maybe tax progressivity later)
	\item two sectors with two different labour inputs
	\item think about emission and climate targets! 
	\item there is no corrective tax (what can the government attain with policies that are already well established?)
\end{itemize}
\paragraph{To do sometime }
\begin{itemize}
	\item read papers on consumption tax
	\item read \cite{Acemoglu2002DirectedChange} \ar how is technological progress modeled, how is model applied to empirical examples
	\item how to model skills? What is important empirically? On the job training? or occupational choice?
\end{itemize}
\begin{enumerate}
	\item[0.] where/how do skills become important for a transition to cleaner production? (a) in innovation itself (engineering skills), (b) in using the new technology (managerial skills) \citep{Vona2018EnvironmentalExploration}
\item answer research question in model without directed technical change, analytically and quantitatively. This is basically an extension to \cite{Heathcote2017OptimalFramework} to include (i) two sectors, (ii) a corrective tax, and (iii) a climate target in the planner's objective function. (note that HSV in baseline is a static model)
\\ further questions: 
\begin{itemize}
	\item \tr{are there public finance papers which look at the distribution of skills \textbf{across sectors}?}
\end{itemize}
\item add directed technical change \ar if leaves findings unchanged, write up as extension; if this changes finding importantly (e.g. qualitatively) then include into baseline model
\begin{itemize}
	\item distribution of skills and technological change will interact with each other... 
	\item have to read more about directed technical change and skill bias!
	\item references
\begin{itemize}
	\item Benabou 2002 (\ar no directed technical change) and 2005 \ar skill bias of technological change: technological change captured by the elasticity of substitution of skill factors; \textbf{could argue that when it comes to clean vs dirty production it is about innovations to reduce emissions}; \tr{sector specific technology is key}
	\item \cite{Acemoglu2002DirectedChange}
\end{itemize}
\end{itemize}
\end{enumerate}




\paragraph{Quantitative model}
Recent work has shown, that  higher tax progressivity is amplified in lowering inequality through a compression of the wage rate. (there is a second effect...). On the other hand, high skill labour is used in a higher share in green sectors \cite{Consoli2016DoCapital}. Therefore, progressivity implies a shift to dirty innovations and a higher externality. These channels constitute a new trade-off between inequality and climate change mitigation. 

% motivation from consumption reduction proponents
Furthermore, a higher tax progressivity reduces consumption at higher income levels (only if not smoothed by savings), production and externalities. Then again, high skill labour may be missing for green production. The reduction and recomposition mechanism counteract each other once accounting for skill heterogeneity. 

%\begin{itemize}
%	\item What is the optimal policy to achieve emission targets by 2050?
%	\item role for fiscal policy due to time frame and skill supply?
%	\item inefficient low supply of high skill labour \ar regressive tax optimal?
%	\item demand side: to counteract the negative effect of redistribution through lower skill supply? \ar demand side effect
%	\item voluntary reduction in consumption \ar even lower skill supply? \ar who are these households? (rich/ high low skill?)
%	\item non-monetary motive for scientists? 
%\end{itemize}

\subsection*{Comment: 31/03/22}
It seems difficult to solve the problem when no balanced growth path exist since fossil output has to be constant under the optimal policy. (But that is a constant growth rate, just that output ratios are not constant) 
Therefore,
\begin{enumerate}
	\item solve under assumption of constant growth rates \ar that would be the limit and determines the continuation value of the economy
	\item assume a planner who only cares about the transition to the net-zero emission economy \ar most important to satisfy voters today
	\begin{itemize}
		\item the objective function is the sum of transition periods (2020 to 2080), with the length of a period =5 years \ar 30 periods.
		\item using numeric method as in Barrage should be solvable; no continuation value; (later adding continuation value not a big deal)
		\item no need to make it stationary
		\item how does the economy evolve afterwards?
	\end{itemize}

\end{enumerate}
\subsection*{Comment: 25/03/2022}
In the models on endogenous growth, emissions positively depend on innovations in the dirty sector. Technology in this sector has to be perceived as more goods being produced each of which exerts the same level of the externality. This seems sensible, when these goods need the same input of emissions generating factors but can be produced with the same number of machines. Also sensible when thinking about waste which is on product level. But then again could think of progress as needing less input goods to achieve the same number of outputs, then should measure emissions by input factors and not output. This is captured by growth in the green sector.

Now, assuming emissions are proportional to output, and introducing the exogenous limit on emissions s.t. net-emissions have to equal zero from mid-century onward, then the assumption that on the BGP all technology ratios are constant implies that all growth has to stop. (also assuming here that carbon capture cannot grow without end). This seems very restrictive. Could divert from assumption of constant technology ratios on BGP: instead assuming a generalised BGP which allows for transitions across sectors.  On this GBGP fossil output has to remain at the same level (as an upper bound assume the one prescribed by the IPCC). Then growth in the fossil sector is zero (compatible with BGP, but technology ratios are not constant.) In fact, a BGP. 

Due to spill overs there might be room for ever growing corrective taxes to counter market forces. 
As the green sector growth, green energy becomes relatively cheaper and cheaper compared to fossil energy. This market effect could redirect production and innovation to green energy. 
The labour income tax could support by changing relative skill supply. 

\subsection*{Comment: 19/03/2022}
\begin{itemize}

	\item \textbf{Question 1 a)}: \textbf{how does the presence of a progressive tax (as calibrated) (or a demand target) change the optimal environmental policy?}\\ \ \\
%	\\ Steps
%	\begin{enumerate}
%		\item max swf st emission constraints; save optimal policy
%		\item max swf st emission constraint and demand constraint
%		\item compare optimal policies
%	\end{enumerate}
\textbf{Motivation:} How tax progressivity affects the externality: lowering demand on the one hand reduces emissions as output decreases; on the other hand, labour supply incentives change, potentially more so for high skilled than for low skilled workers.\\
In the literature on optimal environmental policies, positive labour income taxes lower the optimal environmental tax below the Pigouvian rate, due to efficiency costs. In this setting, however, the optimal environmental tax might be higher to counteract the positive effect of income tax progressivity on the externality. \\
Starting from \cite{Fried2018ClimateAnalysis}; set up:  model with rep agent, max social welfare function plus target;  the planner can choose corrective taxes the income tax is given exogenously
\begin{enumerate}
	\item amend model to incorporate income taxes and skill heterogeneity
\item set up her model to find the optimal environmental tax as done in \cite{Barrage2019OptimalPolicy}
\item 
\end{enumerate}
\item[\ar] check model implications in the data: is there heterogeneity in the effect of tax progressvity on the externality across countries which differ in their wage-hour elasticity. 

	\item Question 1 b): (Optimal policy) Is there the potential for the income tax code to be targeted at the externality even if a corrective tax is present and there is no demand target?
Why? Rather, a further reduction of tax progressivity might be optimal to increase high-skill labour supply. 
\ar \textbf{another trade-off between inequality and the externality}. Not only via the efficiency channel but due to redistribution, innovations will be directed towards the polluting sector. \\
Add exogenous demand target as suggested by natural scientists. What is the optimal policy in this case? Could also find that the environemntal tax is used to lower aggregate output when goods are complements! 

\item \textbf{Adding inequality}
On the other hand, the environmental tax could be lower than absent inequality as it increases wage dispersion. 

	\item Question 2 (This refers to \cite{Loebbing2019NationalChange}): reducing consumption bears the potential of social unrest. \ar add inequality. How would a social planner choose to meet the targets if he searches to minimise social tension/ impact on the poor/ utilitarian swf? 
	\\
	set up: two household types, max swf and targets; additional motive to avoid inequality
	\\
	Steps
	\begin{enumerate}
	\item what are the distributional effects of the policy found for question 1?
	\item How would the optimal planner set the optimal policy now? 
	\end{enumerate}
\item why not look at good specific taxes? \ar because (1) hit the poor (regressivity), (2) corresponds to corrective tax! \ar but then no need for income tax
\end{itemize}

New motivation: \\
Natural scientists have identified a reduction in demand for energy and land-intense products as key to meeting climate targets and sustainability goals jointly, or to diminish the reliance on risky carbon dioxide reduction technologies. More broadly, \cite{Arrow2004AreMuch} argue for the efficiency gains of lowering aggregate consumption through the use of public policy instruments. 
However, a dynamic general equilibrium analysis is missing. 

\paragraph{How to get to this?}
\ar Amend \cite{Fried2018ClimateAnalysis}. 
\begin{enumerate}
\item add income tax and elastic labour supply
\end{enumerate}
\subsection*{Comment: 16/03/2022}
\begin{itemize}
	\item this version: the planner maximises social welfare but is constrained by an emission target; the planner only has labour income taxes at its disposal; the economy can be represented by a representative agent
	\item way forward (quantitatively):
	\begin{enumerate}
		\item model rep agent as only supplying one skill \ar $\zeta=1$
			\item depart from log utility of consumption; use preferences with a slightly higher income effect than substitution effect as suggested by \cite{Boppart2019labourPerspectiveb}.
		\item  add a target on demand to the Ramsey problem\ar the planner not only has to meet emission targets but also a target on demand which is motivated by the natural sciences debate on climate change: 
		\begin{quote}Reduction policies alleviate the pressure to meet other sustainability goals \citep{Bertram2018TargetedScenarios}, they reduce the necessity to rely on \textit{carbon dioxide removal} technologies which are not without risk as they rely on  underground CO2 storage and compete with land needed for food production and biodiversity protection \citep{VanVuuren2018AlternativeTechnologies}.
		\end{quote}
	then, the planner might choose income taxes to meet the emission target even if corrective taxes are available.

		\item  introduce heterogeneity  (skills) 
		\item  introduce directed technical change 
		\item what if households want to lower consumption absent any policy intervention? How does this change the analysis?
	\end{enumerate}
	
\end{itemize}

\section{Introduction}
% this intro refers to the following setup:
% The government can only use labour taxes and corrective taxes are not available
% e.g what can the government achieve with common 

% Structure Intro
% 1. Motivation: (2) setting real world, (3) Why is the question interesting? (Tradeoff)

% 2. What I do: Contribution and main finding

% 3. Model (several layers)

% 4. Calibration

% 5. Main quantitative experiment and results
% 
% To do: 
%\tr{ (i) Connect paragraphs,
% (ii) guide reader, 
% (iii) make smooth }

\begin{comment}
\textcolor{violet}{Still to do:
\begin{itemize}
	\item possibilities to model technical change: substitutability of goods, growth in sector, innovation on substitutability versus consumption growth
\end{itemize}
}

content...
\end{comment}

%\paragraph{Classical use of fiscal instruments}
An equity-efficiency trade-off is central to the discussion of optimal labour income taxation and tax progressivity in the public finance literature.  The benefits of labour taxes and progressivity arise, inter alia, from redistribution. %and from generating government revenues. 
With concave utility specifications full redistribution is efficient. However, the optimal tax system does not feature full redistribution when labour supply is endogenous. Instead, redistribution is traded off against aggregate output as individuals reduce their labour supply and skill investment in response to labour income taxation. 

%\paragraph{Environmental Externality}
Adding environmental externalities to the classical public finance framework changes the perception of efficiency costs. Instead of merely reducing welfare, direct benefits through a reduction of the externality arise by lowering output. 
In theory, corrective, environmental taxes can establish the efficient allocation in a representative agent economy. Absent inequality, such a tax instrument is optimally set to the social cost of an externality. Originators then internalise these costs in addition to their private ones. However, governments face political difficulties in implementing such policy instruments.\footnote{\ Compare, for instance, the Yellow Vest movement in France in 2018.} On the other hand, scientific research has emphasised the urgency to act and highlighted the advantages of lowering demand for land and energy.
For example, reduction policies alleviate the pressure to meet other sustainability goals \citep{Bertram2018TargetedScenarios}, they reduce the necessity to rely on \textit{carbon dioxide removal} technologies which are not without risk as they rely on  underground CO2 storage and compete with land needed for food production and biodiversity protection \citep{VanVuuren2018AlternativeTechnologies}.
 Therefore, this paper shifts the focus of optimal environmental policies  to fiscal tax instruments as tools to lower demand and meet emission targets. What can be achieved in terms of climate targets and what are the costs?

%\paragraph{Trade-off/ Mechanisms}
 Consumption reduction in affluent countries has been promoted as an environmental policy \citep{Schor2005SustainableReduction, Pullinger2014WorkingDesign, Arrow2004AreMuch}. But, the general equilibrium effects are less well understood.
While having an advantageous direct effect on the externality, counteracting indirect effects may exist in a general equilibrium framework. Proponents of a reduction policy especially focus on consumption by the rich which consume a higher amount of natural resources.\footnote{\ There is a bunch of research on the consumption of resources by income groupd; see for instance \cite{Sager2019IncomeCurves}.} %\footnote{\ Note Sonja: Abstracting from inequality, would it still be best to reduce consumption by the rich when the poor have a higher marginal propensity to consume dirty? \textit{Could be an important aspect in the model}. Not in the baseline, look at it in an extension...}
This concern could add to the benefits of tax progressivity.
In contrast, targeting rich households in particular for environmental reasons will lower the supply of high skilled labour.\footnote{\ The relation of labour income tax progressivity and skill investment has been studied by \cite{Heathcote2017OptimalFramework}.} Yet, these skills are essentially important in greener sectors of the economy \citep{Consoli2016DoCapital}. As a result, dirty production becomes relatively cheaper and the dirty share of production rises. 
% I want to add endogenous innovation later

%\paragraph{Model}
% this version: With Rep agent
I build a tractable model which incorporates the key aspects sketched above. There are two sectors one of which emits pollutants: the dirty sector. Both clean and dirty goods are necessary inputs to the final consumption good. Sectors produce with a sector-specific labour input good. The labour input good in the clean sector contains a higher share of high-skilled labour. 
The economy behaves as if there was a representative household which provides high and low-skilled labour. The former exerts a higher utility cost for the household generating a wage premium for high-skilled labour. 

The government maximises social welfare from a Ramsey planner's perspective. However, it is constrained by an exogenous limit on emissions. The advantage of this approach is that it suffers less from  model misspesifications due to  uncertainties about how emissions affect the environment. Furthermore, it is closely related to the current political debate.\footnote{\ Compare appendix section \ref{app:emission_climate_targets} for a more in depth discussion of this aspect. } 

%\paragraph{Calibration}
I inform the exogenous emission limit by the  targets proposed in the 2018 report of the Intergovernmental Panel on Climate Change (IPCC)\footnote{\ A body of the United Nations established to assess the science related to climate change.},  \cite{Rogelj2018MitigationDevelopment.}. These targets are designed for states to comply with the Paris Agreement: global net greenhouse-gas emissions in 2030 shall equal 25-30 GtCO2e per year and zero in 2050 (p.95 in \cite{Rogelj2018MitigationDevelopment.}).%Indeed,  the agreement  foresees a tight time frame for emission reductions: climate neutrality should be achieved by mid-century.
\footnote{\ Under this treaty, states have agreed on limiting temperature rise to well below 2°C, preferably to 1.5°C, and to achieve climate neutrality by mid-century \url{https://unfccc.int/process-and-meetings/the-paris-agreement/the-paris-agreement}. }
%Compared to integrated climate assessment models, (CHECK DEFINITION) this approach requires less assumptions concerning the relation of emissions and the climate. What

Another important calibration choice is the substitutability of clean and dirty production in the final consumption good. I make the cautious assumption that goods are no perfect substitutes. In other words, there is always at least a small amount of dirty production necessary to produce the final consumption good. \tr{\cite{Cohen2019AnnualSubstitutable} discuss and estimate the substitutability of natural capital in production with a focus on energy. }

%\paragraph{Quantitative Exercise and Results}
The paper is divided into two parts: an analytical part where I derive propositions concerning the role of fiscal policy and a quantitative part which discusses the optimal policy and transitions. 

The main theoretical result is that in the laissez-faire economy, emissions grow without bound. Irrespective of whether the clean and dirty good are substitutes or compliments.

In the quantitative exercise I let a planner choose the optimal policy by maximizing a utilitarian social welfare function but it faces an constraint on emissions. I solve explicitly for the optimal policy in each period making the optimal tax progressivity time dependent. 


\paragraph{Literature}

The paper is related to 3 strands of literature. 

First, to the public finance literature.  \cite{Heathcote2017OptimalFramework} study optimal labour tax progressivity on 



Second, to the literature on optimal environmental policy. 

Third, to the literature on directed technical change. 
\textbf{HEMOUS and Olsen} discuss an endogenous growth model with heterogeneous labour input:
\begin{itemize}
	\item the wage premium is not constant on a BGP which they specify as stable if innovation occurs in both sectors
	\item hence: a non stable BGP is one where innovation does not occur in both sectors at some point
	\item need to allow for this option< when solving the model
	\item quality ladder model: each scientist after having chosen a sector of production, there is no congestion (each scientist works on one machine) legitimate due to within-sector spillovers
	\item on a BGP with equal growth the wage premium may grow (Result in \cite{Acemoglu2002DirectedChange}) 
	\item in \cite{Acemoglu2012TheChange} 
	\begin{itemize}
		\item as emissions are proportional to dirty output implicit assumption of a Leontief production function if there was energy (and also this as only source of emissions); 
		 \item endogenous labour (no sector-specific labour supply) \ar the more productive sector attracts more labour (the MPL is higher at an equal ratio so that more labour ends up in the more productive sector to have equal wages)
		 \item for substitutes innovation might be stuck in the more advanced market as the price effect (which directs innovation to the less productive market) is muted
		 \item[\ar] in \cite{Fried2018ClimateAnalysis} fossil and green energy are substitutes \ar stuck in fossil innovation; but non-energy goods and energy are complements \ar price effect strong; equalising effect
		 \item in LF stuck with dirty innovation, government can redirect innovation towards the clean sector until it catches up \ar policy intervention needer for a " sufficient amount of time" \ar might be missing in today's world
		 \item with DTC need postponing intervention problematic!
		 \item subsidies and corrective taxes needed to implement first best 
	\end{itemize}
\item \cite{Acemoglu2016TransitionTechnology} incremental (sector-specific) and radical (building on the leading technology irrespective of sector) innovation \ar cross-sectoral spillovers which were absent in \cite{Acemoglu2012TheChange}
\end{itemize}

%Finally, the paper is meant to add to the discussion on reduction versus recomposition policies as tools to reduce human impact on the environment. 

\paragraph{Outline}
The paper is structured as follows. In the next section, I define a simple model. % to analyse the role of fiscal taxes on the environment. 
Section \ref{sec:theory} discusses theoretical optimal policy results. Section \ref{sec:calib} argues for the plausibility of chosen parameter values. In section \ref{sec:simul}, I show dynamics of the economy under the laissez-faire and the optimal policy regimes. 


\section{Literature}


\subsection{Models}
\begin{itemize}
\item \cite{Bilbiie2012EndogenousCycles}
\begin{itemize}
\item a model with rep agent
\item investment in the form of stock 
\item innovation as a form of new products
\item one final good sector
\item monopolistic competition
\item homothetic preferences
\end{itemize}
\item \cite{Ravn2006DeepHabits}
\begin{itemize}
 \item habits over average previous consumption of specific good! not over total consumption
 \item rep agent 
 \item habits: marginal utility rises as habits rise \ar could look at what happens as habits are reduced! \ar marginal utility at given consumption level reduces!
 \item more is always better! Plus increases habits \ar I want: that more might not be better after some point
\end{itemize}
\item \cite{McKay2021LumpyPolicy}
\begin{itemize}
\item New Keynesian model with durable and non-durable consumption 
\end{itemize}
\item \cite{Acemoglu2012TheChange}
\begin{itemize}
\item endogenous growth
\item rep agent
\item single labour market
\item no resource use in clean sector! ; abstracts from waste
\item disaster risk!: There is a lower bound on the quality of the environment 
\item environmental externality only affects Utility! So no chance for \textbf{environmental quality} to drive production to zero!
BUT there is a natural resource which is used in production; \textit{How do the two relate?} \ar when environmental quality affects regeneration of exhaustible resource, then there would be some connection, but there is no regeneration of the resource, I think
\item there is degradation of the environment through unsustainable production (only!) and 
\end{itemize}
Functional forms
\begin{align*}
S\in[0,\bar{S}],\ & \text{where}\ \bar{S}\ \text{is the quality of the environment without pollution;}\\
S_v=0 \Rightarrow S_t=0 \forall t\geq v,\ &  0 \ \text{is the point of no return.}\\
\underset{S\rightarrow0}{lim} U(C,S)=-\infty\ & \text{S=0 is a disaster!}\\
\underset{S\rightarrow0}{lim}\frac{\partial U(C,S)}{\partial S}=\infty\ &\\
S_{t+1}= -\xi Y_{dt}+(1+\delta)S_t& \\ 
\text{evolution of environmental quality:} & \text{ falls in dirty production; regeneration rate }\\
 \text{both are exponential relationships}\Rightarrow&\text{ smaller env. quality slower regeneration}\\ 
 &\text{ higher pollution, stronger degradation}
\end{align*}
The dirty sector uses an exploitable resource in the production process
\begin{align*}
Y_{dt}= R_t^{\alpha_2}L_{dt}^{1-\alpha}\int_{0}^{1}A_{dit}^{1-\alpha_1}x_{dit}^{\alpha_1}di
\end{align*}
$R_t$ is the exhaustible resource
\begin{align*}
Q_{t+1}=Q_t-R_t
\end{align*}
they look at a version where the resource is common property (water, air) or owned (Hotelling rule)
\item \cite{Heikkinen2015DegrowthConsumers}: macro model with voluntary reduction in consumption
\item \cite{Borissov2019CarbonDevelopment}: model labour sector in more detail: skill, sectors, and transition
\item \cite{Michaillat2015AggregateUnemployment, Auerbach2021InequalityEconomy} examples of models with economic slack. But both do not feature a satiation point of consumption. 
\end{itemize}

\subsection{Motivation}
\begin{itemize}
\item \cite{Schor2005SustainableReduction}
\begin{itemize}
	\item arguments against unlimited growth
	\begin{itemize}
\item hhh
	\end{itemize}
\end{itemize}
\item \cite{Dasgupta2021}
\begin{itemize}
\item emphasises the use of nature as a sink (stock) and as an input to production \ar can the two be combined?
\end{itemize}
\end{itemize}

\section{Simple Model and Hypothesis}
Simple model with representative agent who provides two skills: high and low. I look at steady states: laissez faire versus the steady state under the optimal policy. 
The focus rests on matching emissions in the model to emission targets suggested by the IPCC report \citep{Rogelj2018MitigationDevelopment.}. 

\subsection{Model and Hypothesis}

\paragraph{Households}
% the rep agent
The economy behaves as if there was a representative household. The household chooses between high and low-skilled labour. The household experiences utility costs from skill accumulation. In equilibrium, the high-skilled labour receives a higher wage rate. 

\begin{align}
U=\underset{\{c_{t}\}_{t=0}^{\infty}, \{h_{lt}\}_{t=0}^{\infty}, \{h_{ht}\}_{t=0}^{\infty}}{max}&
\sum_{t=0}^{\infty}\beta^t u(c_{t}, h_{lt}, h_{ht}, h_{ht+1})\\
%U_{s}=\underset{\{c_{st}\}_{t=0}^{\infty}, \{h_{st}\}_{t=0}^{\infty}}{max}&\sum_{t=0}^{\infty}\beta^t u_s(c_{st}, h_{st}; S_t)\\
s.t.& \ \ c_{t}p_{t}=% (1-\tau_{lt})(h_{ht}w_{ht}+h_{lt}w_{lt})+T_t\\ 
\lambda \left(h_{ht}w_{ht}+h_{lt}w_{lt}\right)^{1-\tau}\\
\ & h_{ht}+h_{lt}\leq \bar{H}_t\\
\ & h_{st}\geq 0 \ \forall s\in \{l,h\}
\end{align}
The period utility function includes costs for on-the-job training which is required for high-skilled labour in the next period. 
\begin{align}
	u(c_{t}, h_{lt}, h_{ht}, h_{ht+1})&= %\frac{c_t^{1-\gamma}}{1-\gamma}
	\log(c_t)-\frac{(h_{lt}+\zeta h_{ht})^{1+\sigma}}{1+\sigma}%-v(h_{ht+1})%,\\
	%\text{where}\  & v(h_{ht+1})=\left\{\begin{array}{lll}\zeta& \hspace{2mm} \text{if} \hspace*{2mm}  h_{ht+1}> 0, &\\
%0  &\hspace{2mm}\text{if}\hspace{2mm}  h_{ht+1}= 0.
%	\end{array}
%	\right. 		
\end{align}
The positive parameter $\zeta$ implies a higher marginal disutility for high-skilled labour than unskilled labour. As a result, in equilibirum,  
high-skilled labour earns a premium to compensate the representative household for the higher disutility. When labour income gets taxed, the returns to learning reduce and skilled labour becomes scarcer on impact. 

The log-utility from consumption ensures balanced-growth-path compatibility of hours worked. However, this makes the reduction in hours supplied independent of the wage rate. Still, the extensive margin through learning should remain.

I define the variable $H_t=\zeta h_{ht}+h_{lt}$ which facilitates the derivation of results. 

\paragraph{Production}
There are two production sectors: a clean and a dirty one, indexed by $c$ and $d$. Sector specific goods are imperfect substitutes in final consumption good production.\footnote{\ This ensures that the dirty good is always produced and overall, production is never perfectly clean.}  
The final good producing sector is perfectly competitive:
$Y_t=\left(Y_{ct}^{\frac{\varepsilon-1}{\varepsilon}}+Y_{dt}^{\frac{\varepsilon-1}{\varepsilon}}\right)^\frac{\varepsilon}{\varepsilon-1}$. 
I take the composite good as the numeraire so that $\left[p_{dt}^{1-\varepsilon}+p_{ct}^{1-\varepsilon}\right]^{\frac{1}{1-\varepsilon}}=1$.

In both sectors, a unit mass of competitive firms $i$ produces an individual consumption good. All firms use machines, $x_{jit}$ and an intermediate labour good, $L_{jt}$ as input.\footnote{\ For now I abstract from a natural resource.} 
\begin{align*}
&Y_{dt}= L_{dt}^{1-\alpha_d}\int_{0}^{1}A_{dit}^{1-\alpha_d}x_{dit}^{\alpha_d} di,\ \hspace{2mm} Y_{ct}= L_{ct}^{1-\alpha_c}\int_{0}^{1}A_{cit}^{1-\alpha_c}x_{cit}^{\alpha_c} di.
\end{align*}

The labour input good is produced by a perfectly competitive and sector-specific labour industry according to: 
\begin{align}
L_{jt}=l_{jht}^{\theta_j}l_{jlt}^{1-\theta_j}, \ for \ j \in\{c,d\},
\end{align}
where $\theta_c>\theta_d$ so that the clean sectors labour input has a higher share of high-skilled labour. 


\paragraph{Machine producing firms}
A perfectly competitive sector produces machines, $x_{ijt}$, and sells them to final good firms in the respective sector at price $p_{ijt}$. It is assumed that the costs to produce one machine, $\psi$, are homogeneous across firms. It follows that $p_{ijt}=\psi$.


\paragraph{Technological progress}
Technological progress is exogenous:
\begin{align}
A_{ijt+1}=(1+\upsilon_{jt}) A_{ijt}\ for \ j \in\{c,d\}. 
\end{align}

\paragraph{Impossibility of reaching target in steady state with endogenous growth}
Note that with exogenous growth in each sector there is no possibility for the government to stop emissions from growing, since production of the dirty good is essential for the consumption good (no perfect substitution: $\varepsilon<\infty$). To meet the emission target, the government either needs to affect the growth rate in the economy; i.e., $\upsilon_j$ is a choice variable, or work and consumption need to be set to zero; or the emission target has to be defined in relative terms. The latter possibility contradicts the Paris Agreement which is concerned with absolute emissions.  
I therefore assume, that the government can change the growth rate.

The government chooses the growth rate in each sector, taking into account that research is constrained by an exogenous  amount of scientists
\begin{align}
\upsilon_{ct}+\upsilon_{dt}\leq\Upsilon
\end{align}
 
  
\paragraph{Government}

The government maximises social welfare but is constrained by meeting emission targets in line with the Paris Agreement. Furthermore, the government does not have corrective taxes at its disposal. Instead, only already established tax instruments: distortionary labour taxes (consumption taxes) are available. 

\begin{align}
\underset{\{\tau_{lt}, \upsilon_{ct}, \upsilon_{dt}\}_{t=0}^{\infty}}{max}&\sum_{t=0}^{\infty}\beta^t u(c_{t}, h_{ht}, h_{lt})\\
s.t.\ & \tau_{lt}(h_{ht}w_{ht}+h_{lt}w_{lt})=T_t\  \forall \ t\geq 0\\
& \underbrace{\kappa Y_{nt}}_{\text{emissions in t}} -\delta \leq E_t \  \forall \ t\geq 0\\
&\upsilon_{ct}+\upsilon_{dt}\leq\Upsilon\  \forall \ t\geq 0\\
& \text{behaviour of firms and households}
\end{align}

$E_t$ are flow emissions per year. The IPCC prescribes net-zero emissions starting from 2050 and in 2030 to $E_t= 25-30GtCO2e\ yr^{-1}$. The parameter $\delta$ captures the capacity of the environment to reduce emitted $CO2$ through sinks, such as forests and moors. Hence, in the net-zero steady state it has to hold that $Y_{nt}=\frac{\delta}{\kappa}\ \forall t\geq 30$ assuming that the analysis starts in 2020. 

\subsection{Hypothesised outcome}
How do I expect the optimal steady state to differ from the laissez-faire one? 
In the representative agent model, the government faces a trade-off  between efficiency and the externality. 
On the one hand, the distortionary labour tax reduces output and thereby the externality of production. On the other hand, it reduces utility from consumption.

Allowing for two skill types and a skill bias of the cleaner sector adds an additional layer to the effect of labour taxes on the environment. Instead of merely reducing output there is also a recomposition effect. 
In response to the labour tax, the household reduces its labour supply. Since the high-skilled labour earns a higher wage rate, unskilled labour becomes more attractive to the representative agent. The lower supply of skilled labour increases production costs of the cleaner sector. The price of the cleaner good increases. Hence, the share of clean to dirty output falls. This indirect recomposition effect counteracts the direct reduction of the externality. 

I hypothesise that under this assumption growth in the clean sector, too, will have to stop. Why? Consider that only the clean sector growths, then the price for clean goods has to fall so that the final good sector  continues demand the supply of the clean good. The price will be driven towards zero. Which cannot be an equilibrium solution since the clean sector would stop producing as costs exceed revenues (only if marginal production costs tend to zero but they don't as labour exerts disutility).

How can then be there a role for distortionary labour taxes if the government can choose growth rates? Maybe not. But once growth is endogenous? Maybe during the transition? 
Maybe because reducing labour supply is better in terms of utility? 
\subsection{Equilibrium conditions}

\begin{align*}
\text{Household}\ & 
\end{align*}
\subsection{Steady state}
All variables grow at a constant rate $\eta$
%\section{Theoretical Results}\label{sec:theory}
\begin{itemize}
	\item laissez faire
	\item optimal policy results
\end{itemize}

In this section, I discuss the main theoretical results. 

The laissez-faire economy does violate the emission target. 

\paragraph{Dirty output}

Dirty output grows as
\begin{align*}
	\frac{Y_d'}{Y_d}=\left(\frac{p_d'}{p_d}\right)^{\frac{\alpha-(1-\alpha)(1-\varepsilon)}{1-\alpha}}\frac{A_d'}{A_d}\frac{H'}{H}
\end{align*}
When goods are substitutes, $\varepsilon>1$, the dirty good's price reduces labour supply in this sector. When goods are complements, labour supply in the dirty sector increases as the dirty goods price rises...

growth in the dirty sector falls with inflation in this sector: $\alpha -(1-\alpha)(1-\varepsilon)>0$, then the growth in machines exceeds 
\subsection{Results}
As a result, the percentage change in labour input goods by sectors are equivalent. This together with prices being independent of skill supply implies that the output ratio of sectors is unaffected by tax progressivity.
To see this write:
\begin{align}
	\frac{d\left(\frac{Y_d}{Y_s}\right)}{d \tau_l}=\frac{Y_d}{Y_c}\left(\frac{\frac{dY_d}{Y_d}}{d \tau_l}-\frac{\frac{dY_c}{Y_c}}{d \tau_l}\right)=0
\end{align}
and observe that the percentage change in sector output is homogeneous. 
\begin{align}
	\frac{1}{Y_d}\frac{dY_d}{d \tau_l}= \frac{1}{L_d}\frac{d L_d}{d \tau_l}=\frac{1}{H}\frac{d H}{d \tau_l}\ \text{and} \ \frac{1}{Y_c}\frac{dY_c}{d \tau_l}= \frac{1}{L_c}\frac{d L_c}{d \tau_l}=\frac{1}{H}\frac{d H}{d \tau_l}.
\end{align}






\begin{prop}[Effect of $\tau_l$ on output ratio]
	In the representative agent model with log utility and no disposal of government revenues, tax progressivity does not affect the equilibrium ratio of sector production. Only total output reduces as progressivity rises. \tr{directly obvious from seeing that ratio is constant!}
\end{prop}

\paragraph{Welfare}
\section{Theoretic Results}
\subsection{An emission target calls for a reduction policy under likely parameter values}
\subsection{Tax progressivity affects the composition of total output}
In the model, tax progressivity affects the innovation decision due to heterogeneous effects on skill supply. 
The optimal ratio of skills supplied by the household is
\begin{align}
\frac{h_{ht}}{h_{lt}}=\left(\frac{w_{ht}}{w_{lt}}\right)^\frac{1-\tau_{lt}}{\tau_{lt}+\sigma}.
\end{align}
The semi-elasticity of the ratio of aggregate skill supply, defined as $\frac{H_h}{H_l}:=\frac{z_hh_h}{z_lh_l}$, in response to a change in tax progressivity is then given by
\begin{align}
\frac{d\log\left(\frac{H_h}{H_l}\right)}{d\tau_l}=-\frac{1+\sigma}{(\tau_l+\sigma)^2}\log\left(\frac{w_{h}}{w_l}\right). \end{align}
The direct effect, with fixed prices is negative. 
The term negative given a positive wage premium for high skill labour. Hence, a higher tax progressivity implies a decline in the relative supply of high skill labour. 

\paragraph{Effect on the externality }

\subsection{Growth in the dirty sector has to stop}
otherwise, price increases to infinity and labour input falls.
\section{Calibration}\label{sec:calib}
%\section{Model}

In this section, I spell out the tractable model. 
\subsection{Household}
There is a unit mass of households in the economy which differ in their skills and effective labour productivity. 


\section{Fiscal policy and the environment}

\section{Quantitative results}\label{sec:simul}

\begin{comment}
Interesting quantitative results

\begin{itemize}
	\item comparison laissez faire in 2050 to optimal policy (net-zero emissions starting in 2050)
	\item present ratios and variables that are constant 
	\item How?: 
	\begin{enumerate}
	\item calculate values of endogenous and predetermined variables starting from today
	\item apply growth rates in laissez-faire and in optimal SS (this is static)\ar simulate the economy/ should be there already
	\item dynamics: shooting algorithm! / relaxation (pertubation (= approximation) to get starting values)
	\end{enumerate}
\end{itemize}
\end{comment}

\begin{comment}
\paragraph{The effect of a higher disutility of high skill labour}

With a higher disutility in high skill labour, the share of the dirty good in production rises. 

\begin{figure}[h!!]
\includegraphics[width=1\textwidth]{../codding_model/Own/figures/Rep_agent/Yd_Yc_ratio_periods10_eppsilon0.40_zeta1.40_Ad08_Ac04_thetac0.70_thetad0.56_fullDisp0_HetGrowth1_tauul0.181_util0.png}
\caption{Effect of the scarcity of labour on the output ratio}
\end{figure}
\end{comment}

In this section, I compare the evolution of the economy under the laissez-faire calibration to the evolution under the optimal policy. First, I show results without the emission target. Second, results with emission target are discussed.

The comparison of the optimal to the laissez-faire allocation in the scenario without emission target shows that there are no motives for the government to intervene. The optimal policy is to set as a flat tax and to generate no revenues. 

\begin{figure}[h!!]
	\centering
	\caption{Optimal Policy }\label{fig:optpol}
	\begin{minipage}[]{0.32\textwidth}
		\centering{\footnotesize{(a) Without target }}
		%	\captionsetup{width=.45\linewidth}
		\includegraphics[width=1\textwidth]{../codding_model/Own/figures/Rep_agent/staticRam_LF_separate_tauul_periods59_eppsilon4.00_zeta1.40_Ad08_Ac04_thetac0.70_thetad0.56_HetGrowth1_tauul0.181_util0_withtarget0_lgd0.png}
	\end{minipage}
	\begin{minipage}[]{0.32\textwidth}
	\centering{\footnotesize{(b) With target $\varepsilon>1$ }}
	%	\captionsetup{width=.45\linewidth}
	\includegraphics[width=1\textwidth]{../codding_model/Own/figures/Rep_agent/staticRam_LF_separate_tauul_periods59_eppsilon4.00_zeta1.40_Ad08_Ac04_thetac0.70_thetad0.56_HetGrowth1_tauul0.181_util0_withtarget1_lgd0.png}
\end{minipage}
\begin{minipage}[]{0.32\textwidth}
	\centering{\footnotesize{(c) With target $\varepsilon<1$ }}
	%	\captionsetup{width=.45\linewidth}
	\includegraphics[width=1\textwidth]{../codding_model/Own/figures/Rep_agent/staticRam_LF_separate_tauul_periods59_eppsilon0.40_zeta1.40_Ad08_Ac04_thetac0.70_thetad0.56_HetGrowth1_tauul0.181_util0_withtarget1_lgd0.png}
\end{minipage}
\end{figure}

\begin{figure}[h!!]
	\centering
	\caption{Optimal versus laissez-faire allocation: No emission target, substitutes }\label{fig:optallo_subs}
	\begin{minipage}[]{0.32\textwidth}
		\centering{\footnotesize{(a) Consumption }}
		%	\captionsetup{width=.45\linewidth}
		\includegraphics[width=1\textwidth]{../codding_model/Own/figures/Rep_agent/staticRam_LF_separate_c_periods59_eppsilon4.00_zeta1.40_Ad08_Ac04_thetac0.70_thetad0.56_HetGrowth1_tauul0.181_util0_withtarget0_lgd1.png}
	\end{minipage}
	\begin{minipage}[]{0.32\textwidth}
		\centering{\footnotesize{(b) High skill supply }}
		%	\captionsetup{width=.45\linewidth}
		\includegraphics[width=1\textwidth]{../codding_model/Own/figures/Rep_agent/staticRam_LF_separate_hh_periods59_eppsilon4.00_zeta1.40_Ad08_Ac04_thetac0.70_thetad0.56_HetGrowth1_tauul0.181_util0_withtarget0_lgd0.png}
	\end{minipage}
	\begin{minipage}[]{0.32\textwidth}
		\centering{\footnotesize{(c) Low skill supply}}
		%	\captionsetup{width=.45\linewidth}
		\includegraphics[width=1\textwidth]{../codding_model/Own/figures/Rep_agent/staticRam_LF_separate_hl_periods59_eppsilon4.00_zeta1.40_Ad08_Ac04_thetac0.70_thetad0.56_HetGrowth1_tauul0.181_util0_withtarget0_lgd0.png}
	\end{minipage}
\begin{minipage}[]{0.32\textwidth}
\centering{\footnotesize{(d) clean output }}
%	\captionsetup{width=.45\linewidth}
\includegraphics[width=1\textwidth]{../codding_model/Own/figures/Rep_agent/staticRam_LF_separate_yc_periods59_eppsilon4.00_zeta1.40_Ad08_Ac04_thetac0.70_thetad0.56_HetGrowth1_tauul0.181_util0_withtarget0_lgd0.png}
\end{minipage}
\begin{minipage}[]{0.32\textwidth}
\centering{\footnotesize{(e) dirty output }}
%	\captionsetup{width=.45\linewidth}
\includegraphics[width=1\textwidth]{../codding_model/Own/figures/Rep_agent/staticRam_LF_separate_yd_periods59_eppsilon4.00_zeta1.40_Ad08_Ac04_thetac0.70_thetad0.56_HetGrowth1_tauul0.181_util0_withtarget0_lgd0.png}
\end{minipage}
\begin{minipage}[]{0.32\textwidth}
\centering{\footnotesize{(f) machines dirty}}
%	\captionsetup{width=.45\linewidth}
\includegraphics[width=1\textwidth]{../codding_model/Own/figures/Rep_agent/staticRam_LF_separate_xd_periods59_eppsilon4.00_zeta1.40_Ad08_Ac04_thetac0.70_thetad0.56_HetGrowth1_tauul0.181_util0_withtarget0_lgd0.png}
\end{minipage}
\begin{minipage}[]{0.32\textwidth}
	\centering{\footnotesize{(f) machines clean}}
	%	\captionsetup{width=.45\linewidth}
	\includegraphics[width=1\textwidth]{../codding_model/Own/figures/Rep_agent/staticRam_LF_separate_xc_periods59_eppsilon4.00_zeta1.40_Ad08_Ac04_thetac0.70_thetad0.56_HetGrowth1_tauul0.181_util0_withtarget0_lgd0.png}
\end{minipage}
\begin{minipage}[]{0.32\textwidth}
	\centering{\footnotesize{(g) Output ratio $y_d/y_c$}}
	%	\captionsetup{width=.45\linewidth}
	\includegraphics[width=1\textwidth]{../codding_model/Own/figures/Rep_agent/staticRam_LF_separate_ydyc_periods59_eppsilon4.00_zeta1.40_Ad08_Ac04_thetac0.70_thetad0.56_HetGrowth1_tauul0.181_util0_withtarget0_lgd0.png}
\end{minipage}
\end{figure}

\begin{figure}[h!!]
	\centering
	\caption{Optimal versus laissez-faire allocation: No emission target, complements SOMETHING WRONG? LOWER WELFARE}\label{fig:optallo_comp}
	\begin{minipage}[]{0.32\textwidth}
		\centering{\footnotesize{(a) Consumption }}
		%	\captionsetup{width=.45\linewidth}
		\includegraphics[width=1\textwidth]{../codding_model/Own/figures/Rep_agent/staticRam_LF_separate_c_periods59_eppsilon0.40_zeta1.40_Ad08_Ac04_thetac0.70_thetad0.56_HetGrowth1_tauul0.181_util0_withtarget0_lgd1.png}
	\end{minipage}
	\begin{minipage}[]{0.32\textwidth}
		\centering{\footnotesize{(b) High skill supply }}
		%	\captionsetup{width=.45\linewidth}
		\includegraphics[width=1\textwidth]{../codding_model/Own/figures/Rep_agent/staticRam_LF_separate_hh_periods59_eppsilon0.40_zeta1.40_Ad08_Ac04_thetac0.70_thetad0.56_HetGrowth1_tauul0.181_util0_withtarget0_lgd0.png}
	\end{minipage}
	\begin{minipage}[]{0.32\textwidth}
		\centering{\footnotesize{(c) Low skill supply}}
		%	\captionsetup{width=.45\linewidth}
		\includegraphics[width=1\textwidth]{../codding_model/Own/figures/Rep_agent/staticRam_LF_separate_hl_periods59_eppsilon0.40_zeta1.40_Ad08_Ac04_thetac0.70_thetad0.56_HetGrowth1_tauul0.181_util0_withtarget0_lgd0.png}
	\end{minipage}
	\begin{minipage}[]{0.32\textwidth}
		\centering{\footnotesize{(d) clean output }}
		%	\captionsetup{width=.45\linewidth}
		\includegraphics[width=1\textwidth]{../codding_model/Own/figures/Rep_agent/staticRam_LF_separate_yc_periods59_eppsilon0.40_zeta1.40_Ad08_Ac04_thetac0.70_thetad0.56_HetGrowth1_tauul0.181_util0_withtarget0_lgd0.png}
	\end{minipage}
	\begin{minipage}[]{0.32\textwidth}
		\centering{\footnotesize{(e) dirty output }}
		%	\captionsetup{width=.45\linewidth}
		\includegraphics[width=1\textwidth]{../codding_model/Own/figures/Rep_agent/staticRam_LF_separate_yd_periods59_eppsilon0.40_zeta1.40_Ad08_Ac04_thetac0.70_thetad0.56_HetGrowth1_tauul0.181_util0_withtarget0_lgd0.png}
	\end{minipage}
	\begin{minipage}[]{0.32\textwidth}
		\centering{\footnotesize{(f) machines dirty}}
		%	\captionsetup{width=.45\linewidth}
		\includegraphics[width=1\textwidth]{../codding_model/Own/figures/Rep_agent/staticRam_LF_separate_xd_periods59_eppsilon0.40_zeta1.40_Ad08_Ac04_thetac0.70_thetad0.56_HetGrowth1_tauul0.181_util0_withtarget0_lgd0.png}
	\end{minipage}
	\begin{minipage}[]{0.32\textwidth}
		\centering{\footnotesize{(f) machines clean}}
		%	\captionsetup{width=.45\linewidth}
		\includegraphics[width=1\textwidth]{../codding_model/Own/figures/Rep_agent/staticRam_LF_separate_xc_periods59_eppsilon0.40_zeta1.40_Ad08_Ac04_thetac0.70_thetad0.56_HetGrowth1_tauul0.181_util0_withtarget0_lgd0.png}
	\end{minipage}
\begin{minipage}[]{0.32\textwidth}
\centering{\footnotesize{(g) Output ratio $y_d/y_c$}}
%	\captionsetup{width=.45\linewidth}
\includegraphics[width=1\textwidth]{../codding_model/Own/figures/Rep_agent/staticRam_LF_separate_ydyc_periods59_eppsilon0.40_zeta1.40_Ad08_Ac04_thetac0.70_thetad0.56_HetGrowth1_tauul0.181_util0_withtarget0_lgd0.png}
\end{minipage}
\begin{minipage}[]{0.32\textwidth}
	\centering{\footnotesize{(g) Welfare}}
	%	\captionsetup{width=.45\linewidth}
	\includegraphics[width=1\textwidth]{../codding_model/Own/figures/Rep_agent/staticRam_LF_separate_welfare_periods59_eppsilon0.40_zeta1.40_Ad08_Ac04_thetac0.70_thetad0.56_HetGrowth1_tauul0.181_util0_withtarget0_lgd0.png}
\end{minipage}
\end{figure}

\begin{figure}[h!!]
	\centering
	\caption{Optimal versus laissez-faire allocation: With emission target, complements }\label{fig:optallo_comp_target}
	\begin{minipage}[]{0.32\textwidth}
		\centering{\footnotesize{(a) Consumption }}
		%	\captionsetup{width=.45\linewidth}
		\includegraphics[width=1\textwidth]{../codding_model/Own/figures/Rep_agent/staticRam_LF_separate_c_periods59_eppsilon0.40_zeta1.40_Ad08_Ac04_thetac0.70_thetad0.56_HetGrowth1_tauul0.181_util0_withtarget1_lgd1.png}
	\end{minipage}
	\begin{minipage}[]{0.32\textwidth}
		\centering{\footnotesize{(b) High skill supply }}
		%	\captionsetup{width=.45\linewidth}
		\includegraphics[width=1\textwidth]{../codding_model/Own/figures/Rep_agent/staticRam_LF_separate_hh_periods59_eppsilon0.40_zeta1.40_Ad08_Ac04_thetac0.70_thetad0.56_HetGrowth1_tauul0.181_util0_withtarget1_lgd0.png}
	\end{minipage}
	\begin{minipage}[]{0.32\textwidth}
		\centering{\footnotesize{(c) Low skill supply}}
		%	\captionsetup{width=.45\linewidth}
		\includegraphics[width=1\textwidth]{../codding_model/Own/figures/Rep_agent/staticRam_LF_separate_hl_periods59_eppsilon0.40_zeta1.40_Ad08_Ac04_thetac0.70_thetad0.56_HetGrowth1_tauul0.181_util0_withtarget1_lgd0.png}
	\end{minipage}
	\begin{minipage}[]{0.32\textwidth}
		\centering{\footnotesize{(d) clean output }}
		%	\captionsetup{width=.45\linewidth}
		\includegraphics[width=1\textwidth]{../codding_model/Own/figures/Rep_agent/staticRam_LF_separate_yc_periods59_eppsilon0.40_zeta1.40_Ad08_Ac04_thetac0.70_thetad0.56_HetGrowth1_tauul0.181_util0_withtarget1_lgd0.png}
	\end{minipage}
	\begin{minipage}[]{0.32\textwidth}
		\centering{\footnotesize{(e) dirty output }}
		%	\captionsetup{width=.45\linewidth}
		\includegraphics[width=1\textwidth]{../codding_model/Own/figures/Rep_agent/staticRam_LF_separate_yd_periods59_eppsilon0.40_zeta1.40_Ad08_Ac04_thetac0.70_thetad0.56_HetGrowth1_tauul0.181_util0_withtarget1_lgd0.png}
	\end{minipage}
	\begin{minipage}[]{0.32\textwidth}
		\centering{\footnotesize{(f) machines dirty}}
		%	\captionsetup{width=.45\linewidth}
		\includegraphics[width=1\textwidth]{../codding_model/Own/figures/Rep_agent/staticRam_LF_separate_xd_periods59_eppsilon0.40_zeta1.40_Ad08_Ac04_thetac0.70_thetad0.56_HetGrowth1_tauul0.181_util0_withtarget1_lgd0.png}
	\end{minipage}
	\begin{minipage}[]{0.32\textwidth}
		\centering{\footnotesize{(f) machines clean}}
		%	\captionsetup{width=.45\linewidth}
		\includegraphics[width=1\textwidth]{../codding_model/Own/figures/Rep_agent/staticRam_LF_separate_xc_periods59_eppsilon0.40_zeta1.40_Ad08_Ac04_thetac0.70_thetad0.56_HetGrowth1_tauul0.181_util0_withtarget1_lgd0.png}
	\end{minipage}
	\begin{minipage}[]{0.32\textwidth}
		\centering{\footnotesize{(g) Output ratio $y_d/y_c$}}
		%	\captionsetup{width=.45\linewidth}
		\includegraphics[width=1\textwidth]{../codding_model/Own/figures/Rep_agent/staticRam_LF_separate_ydyc_periods59_eppsilon0.40_zeta1.40_Ad08_Ac04_thetac0.70_thetad0.56_HetGrowth1_tauul0.181_util0_withtarget1_lgd0.png}
	\end{minipage}
\end{figure}


% only ramsey
\begin{figure}[h!!]
	\centering
	\caption{Optimal allocation: No emission target, complements }\label{fig:optallo_comp_onlyR}
	\begin{minipage}[]{0.32\textwidth}
		\centering{\footnotesize{(a) Consumption }}
		%	\captionsetup{width=.45\linewidth}
		\includegraphics[width=1\textwidth]{../codding_model/Own/figures/Rep_agent/staticonlyRam_separate_c_periods59_eppsilon0.40_zeta1.40_Ad08_Ac04_thetac0.70_thetad0.56_HetGrowth1_tauul0.181_util0_withtarget0_lgd1.png}
	\end{minipage}
	\begin{minipage}[]{0.32\textwidth}
		\centering{\footnotesize{(b) High skill supply }}
		%	\captionsetup{width=.45\linewidth}
		\includegraphics[width=1\textwidth]{../codding_model/Own/figures/Rep_agent/staticonlyRam_separate_hh_periods59_eppsilon0.40_zeta1.40_Ad08_Ac04_thetac0.70_thetad0.56_HetGrowth1_tauul0.181_util0_withtarget0_lgd0.png}
	\end{minipage}
	\begin{minipage}[]{0.32\textwidth}
		\centering{\footnotesize{(c) Low skill supply}}
		%	\captionsetup{width=.45\linewidth}
		\includegraphics[width=1\textwidth]{../codding_model/Own/figures/Rep_agent/staticonlyRam_separate_hl_periods59_eppsilon0.40_zeta1.40_Ad08_Ac04_thetac0.70_thetad0.56_HetGrowth1_tauul0.181_util0_withtarget0_lgd0.png}
	\end{minipage}
	\begin{minipage}[]{0.32\textwidth}
		\centering{\footnotesize{(d) clean output }}
		%	\captionsetup{width=.45\linewidth}
		\includegraphics[width=1\textwidth]{../codding_model/Own/figures/Rep_agent/staticonlyRam_separate_yc_periods59_eppsilon0.40_zeta1.40_Ad08_Ac04_thetac0.70_thetad0.56_HetGrowth1_tauul0.181_util0_withtarget0_lgd0.png}
	\end{minipage}
	\begin{minipage}[]{0.32\textwidth}
		\centering{\footnotesize{(e) dirty output }}
		%	\captionsetup{width=.45\linewidth}
		\includegraphics[width=1\textwidth]{../codding_model/Own/figures/Rep_agent/staticonlyRam_separate_yd_periods59_eppsilon0.40_zeta1.40_Ad08_Ac04_thetac0.70_thetad0.56_HetGrowth1_tauul0.181_util0_withtarget0_lgd0.png}
	\end{minipage}
	\begin{minipage}[]{0.32\textwidth}
		\centering{\footnotesize{(f) machines dirty}}
		%	\captionsetup{width=.45\linewidth}
		\includegraphics[width=1\textwidth]{../codding_model/Own/figures/Rep_agent/staticonlyRam_separate_xd_periods59_eppsilon0.40_zeta1.40_Ad08_Ac04_thetac0.70_thetad0.56_HetGrowth1_tauul0.181_util0_withtarget0_lgd0.png}
	\end{minipage}
	\begin{minipage}[]{0.32\textwidth}
		\centering{\footnotesize{(f) machines clean}}
		%	\captionsetup{width=.45\linewidth}
		\includegraphics[width=1\textwidth]{../codding_model/Own/figures/Rep_agent/staticonlyRam_separate_xc_periods59_eppsilon0.40_zeta1.40_Ad08_Ac04_thetac0.70_thetad0.56_HetGrowth1_tauul0.181_util0_withtarget0_lgd0.png}
	\end{minipage}
	\begin{minipage}[]{0.32\textwidth}
		\centering{\footnotesize{(g) Output ratio $y_d/y_c$}}
		%	\captionsetup{width=.45\linewidth}
		\includegraphics[width=1\textwidth]{../codding_model/Own/figures/Rep_agent/staticonlyRam_separate_ydyc_periods59_eppsilon0.40_zeta1.40_Ad08_Ac04_thetac0.70_thetad0.56_HetGrowth1_tauul0.181_util0_withtarget0_lgd0.png}
	\end{minipage}
\end{figure}

\begin{figure}[h!!]
	\centering
	\caption{Optimal allocation: With emission target, complements }\label{fig:optallo_comp_onlyR_target}
	\begin{minipage}[]{0.32\textwidth}
		\centering{\footnotesize{(a) Consumption }}
		%	\captionsetup{width=.45\linewidth}
		\includegraphics[width=1\textwidth]{../codding_model/Own/figures/Rep_agent/staticonlyRam_separate_c_periods59_eppsilon0.40_zeta1.40_Ad08_Ac04_thetac0.70_thetad0.56_HetGrowth1_tauul0.181_util0_withtarget1_lgd1.png}
	\end{minipage}
	\begin{minipage}[]{0.32\textwidth}
		\centering{\footnotesize{(b) High skill supply }}
		%	\captionsetup{width=.45\linewidth}
		\includegraphics[width=1\textwidth]{../codding_model/Own/figures/Rep_agent/staticonlyRam_separate_hh_periods59_eppsilon0.40_zeta1.40_Ad08_Ac04_thetac0.70_thetad0.56_HetGrowth1_tauul0.181_util0_withtarget1_lgd0.png}
	\end{minipage}
	\begin{minipage}[]{0.32\textwidth}
		\centering{\footnotesize{(c) Low skill supply}}
		%	\captionsetup{width=.45\linewidth}
		\includegraphics[width=1\textwidth]{../codding_model/Own/figures/Rep_agent/staticonlyRam_separate_hl_periods59_eppsilon0.40_zeta1.40_Ad08_Ac04_thetac0.70_thetad0.56_HetGrowth1_tauul0.181_util0_withtarget1_lgd0.png}
	\end{minipage}
	\begin{minipage}[]{0.32\textwidth}
		\centering{\footnotesize{(d) clean output }}
		%	\captionsetup{width=.45\linewidth}
		\includegraphics[width=1\textwidth]{../codding_model/Own/figures/Rep_agent/staticonlyRam_separate_yc_periods59_eppsilon0.40_zeta1.40_Ad08_Ac04_thetac0.70_thetad0.56_HetGrowth1_tauul0.181_util0_withtarget1_lgd0.png}
	\end{minipage}
	\begin{minipage}[]{0.32\textwidth}
		\centering{\footnotesize{(e) dirty output }}
		%	\captionsetup{width=.45\linewidth}
		\includegraphics[width=1\textwidth]{../codding_model/Own/figures/Rep_agent/staticonlyRam_separate_yd_periods59_eppsilon0.40_zeta1.40_Ad08_Ac04_thetac0.70_thetad0.56_HetGrowth1_tauul0.181_util0_withtarget1_lgd0.png}
	\end{minipage}
	\begin{minipage}[]{0.32\textwidth}
		\centering{\footnotesize{(f) machines dirty}}
		%	\captionsetup{width=.45\linewidth}
		\includegraphics[width=1\textwidth]{../codding_model/Own/figures/Rep_agent/staticonlyRam_separate_xd_periods59_eppsilon0.40_zeta1.40_Ad08_Ac04_thetac0.70_thetad0.56_HetGrowth1_tauul0.181_util0_withtarget1_lgd0.png}
	\end{minipage}
	\begin{minipage}[]{0.32\textwidth}
		\centering{\footnotesize{(f) machines clean}}
		%	\captionsetup{width=.45\linewidth}
		\includegraphics[width=1\textwidth]{../codding_model/Own/figures/Rep_agent/staticonlyRam_separate_xc_periods59_eppsilon0.40_zeta1.40_Ad08_Ac04_thetac0.70_thetad0.56_HetGrowth1_tauul0.181_util0_withtarget1_lgd0.png}
	\end{minipage}
	\begin{minipage}[]{0.32\textwidth}
		\centering{\footnotesize{(g) Output ratio $y_d/y_c$}}
		%	\captionsetup{width=.45\linewidth}
		\includegraphics[width=1\textwidth]{../codding_model/Own/figures/Rep_agent/staticonlyRam_separate_ydyc_periods59_eppsilon0.40_zeta1.40_Ad08_Ac04_thetac0.70_thetad0.56_HetGrowth1_tauul0.181_util0_withtarget1_lgd0.png}
	\end{minipage}
\end{figure}

\begin{figure}[h!!]
	\centering
	\caption{Optimal allocation: With emission target, substitutes }\label{fig:optallo_subst_onlyR_target}
	\begin{minipage}[]{0.32\textwidth}
		\centering{\footnotesize{(a) Consumption }}
		%	\captionsetup{width=.45\linewidth}
		\includegraphics[width=1\textwidth]{../codding_model/Own/figures/Rep_agent/staticonlyRam_separate_c_periods59_eppsilon4.00_zeta1.40_Ad08_Ac04_thetac0.70_thetad0.56_HetGrowth1_tauul0.181_util0_withtarget1_lgd1.png}
	\end{minipage}
	\begin{minipage}[]{0.32\textwidth}
		\centering{\footnotesize{(b) High skill supply }}
		%	\captionsetup{width=.45\linewidth}
		\includegraphics[width=1\textwidth]{../codding_model/Own/figures/Rep_agent/staticonlyRam_separate_hh_periods59_eppsilon4.00_zeta1.40_Ad08_Ac04_thetac0.70_thetad0.56_HetGrowth1_tauul0.181_util0_withtarget1_lgd0.png}
	\end{minipage}
	\begin{minipage}[]{0.32\textwidth}
		\centering{\footnotesize{(c) Low skill supply}}
		%	\captionsetup{width=.45\linewidth}
		\includegraphics[width=1\textwidth]{../codding_model/Own/figures/Rep_agent/staticonlyRam_separate_hl_periods59_eppsilon4.00_zeta1.40_Ad08_Ac04_thetac0.70_thetad0.56_HetGrowth1_tauul0.181_util0_withtarget1_lgd0.png}
	\end{minipage}
	\begin{minipage}[]{0.32\textwidth}
		\centering{\footnotesize{(d) clean output }}
		%	\captionsetup{width=.45\linewidth}
		\includegraphics[width=1\textwidth]{../codding_model/Own/figures/Rep_agent/staticonlyRam_separate_yc_periods59_eppsilon4.00_zeta1.40_Ad08_Ac04_thetac0.70_thetad0.56_HetGrowth1_tauul0.181_util0_withtarget1_lgd0.png}
	\end{minipage}
	\begin{minipage}[]{0.32\textwidth}
		\centering{\footnotesize{(e) dirty output }}
		%	\captionsetup{width=.45\linewidth}
		\includegraphics[width=1\textwidth]{../codding_model/Own/figures/Rep_agent/staticonlyRam_separate_yd_periods59_eppsilon4.00_zeta1.40_Ad08_Ac04_thetac0.70_thetad0.56_HetGrowth1_tauul0.181_util0_withtarget1_lgd0.png}
	\end{minipage}
	\begin{minipage}[]{0.32\textwidth}
		\centering{\footnotesize{(f) machines dirty}}
		%	\captionsetup{width=.45\linewidth}
		\includegraphics[width=1\textwidth]{../codding_model/Own/figures/Rep_agent/staticonlyRam_separate_xd_periods59_eppsilon4.00_zeta1.40_Ad08_Ac04_thetac0.70_thetad0.56_HetGrowth1_tauul0.181_util0_withtarget1_lgd0.png}
	\end{minipage}
	\begin{minipage}[]{0.32\textwidth}
		\centering{\footnotesize{(f) machines clean}}
		%	\captionsetup{width=.45\linewidth}
		\includegraphics[width=1\textwidth]{../codding_model/Own/figures/Rep_agent/staticonlyRam_separate_xc_periods59_eppsilon4.00_zeta1.40_Ad08_Ac04_thetac0.70_thetad0.56_HetGrowth1_tauul0.181_util0_withtarget1_lgd0.png}
	\end{minipage}
	\begin{minipage}[]{0.32\textwidth}
		\centering{\footnotesize{(g) Output ratio $y_d/y_c$}}
		%	\captionsetup{width=.45\linewidth}
		\includegraphics[width=1\textwidth]{../codding_model/Own/figures/Rep_agent/staticonlyRam_separate_ydyc_periods59_eppsilon4.00_zeta1.40_Ad08_Ac04_thetac0.70_thetad0.56_HetGrowth1_tauul0.181_util0_withtarget1_lgd0.png}
	\end{minipage}
\end{figure}
\section{Conclusion}\label{sec:con}
Some scholars argue that  reductive policies are necessary to handle environmental limits \citep{Schor2005SustainableReductionb, VanVuuren2018AlternativeTechnologies, Bertram2018TargetedScenarios}, and the question has been raised whether consumption is too high \citep{Arrow2004AreMuch}. On the other hand, the focus of environmental policy discussions in economics rests on corrective environmental taxation. In the light of tightening environmental limits \citep{Rockstrom2009AHumanity, IPCC2022}, I study whether labor income taxes - as a reductive policy tool - can help mitigate externalities. 

In the analytical part of the paper, I show in a simple model that labor income taxes are  progressive as part of the optimal environmental policy. %The model does not feature inequality.
% Quantitative results
% baseline model
When environmental tax revenues are not redistributed lump sum, labor supply is inefficiently high. Then, income taxes serve to diminish hours worked closer to the efficient level. The result prevails absent income inequality.


% quantitative
In the second part of the paper, I analyze in a quantitative model with skill heterogeneity and endogenous growth whether the optimal labor income tax remains progressive. Again, there are no equity concerns, but workers are perfectly ensured against income differences. 
The optimal income tax is progressive to reduce inefficiently high hours worked. The quantitative model reveals that income taxes also serve as a substitute for corrective taxes. Knowledge spillovers from the non-energy sector render environmental taxes especially costly. 
Fossil taxes make energy relatively more expensive which directs research from non-energy to energy sectors. As the non-energy sector features the most research processes it is especially important for aggregate technology and knowledge spillovers. Using income taxes instead of fossil taxes to lower emissions allows to direct more research to the non-energy sector and to profit from knowledge spillovers.
In sum, however, the reduction in labor supply outweighs the positive effect on growth and consumption decreases compared to a scenario where no income tax is used. 

In the quantitative setting, the income tax affects the economic structure through two channels. First, because the fossil sector is comparably labor intense, a reduction in labor supply favors the green sector. This mechanism makes a higher tax progressivity optimal. However, the effect vanishes in equilibrium due to endogenous growth.
Second, a skill-recomposition channel makes green energy production more costly compared to fossil production. This effect arises from a skill bias in the green sector and high-skill labor being more responsive to income taxation. 
The second channel dominates the recomposing effect of  income tax progressivity in equilibrium. A market size effect amplifies the skill-recomposition channel directing research to the fossil sector. 

%Initially, the intention not to harm growth too much makes a lower progressivity optimal. As growth in the fossil sector accelerates due to the dynamic structure of endogenous growth too low progressive income taxes conflict with meeting the emission limit. As a result, optimal progressivity increases over time.
%The optimal path of income tax progressivity is decreasing, a feature mainly driven by endogenous growth. As a result, the optimal income tax progressivity and the optimal fossil tax seem to behave like substitutes in the quantitative model. 

%Skill heterogeneity depresses optimal tax progressivity due to the adverse recomposing effect of a lower high-to-low skill labor supply on the green-to-fossil energy ratio. A higher corrective tax is required to meet emission limits when there is only one skill type: with only one skill the supply of fossil-specific inputs increases thereby violating the emission limit.

%% lump-sum transfers
%When environmental tax revenues are redistributed lump-sum, the motive to use labor income taxes to deal with inefficiently high labor supply vanishes. Instead, income taxes serve to boost growth as long as this does not conflict with meeting emission limits. Therefore, they are regressive. 
%\tr{not true! it is rather that the more in research is not worth it given the dynamics! and decreasing utility gains}
%However, the regressivity decreases since more labor supply causes more emissions especially the more progressed the technology. With only a labor income tax as a tool to raise growth, accelerating technology growth is not feasible as it is concomitant with more production and emissions. 

% extensions
In an extension, I am planning to give the Ramsey planner the opportunity to limit working hours directly. The literature advocating a reduction in consumption levels \citep[e.g.,][]{Schor2005SustainableReductionb} proposes a restriction of hours worked as policy instrument to lower the consumption of resources.
Even though advocated in the literature, there is evidence for political difficulties in reducing working hours. In 2020, the French Citizens' Convention on Climate voted against reducing working hours as a measure to handle climate change. Potentially, ignorance about economic consequences is an explanation. The extension would serve to better understand economic consequences. 



\begin{comment}
\paragraph{Extension: What if the low skilled get a higher share \ar they reduce even less \ar more fossil input supply}

Redistribution to households with a higher marginal propensity to consume emissions counteracts the externality. This effect is amplified by a market size effect  of dirty goods. 

content...
\end{comment}

% I plan to discuss results under counterfactual parameter values to elicit the robustness of the main result: the preference of progressive labour taxation above higher fossil taxes. 
%First, the productivity gap between sectors might be driving the results. Second, I will abstract from endogenous growth to learn about the labour-supply-innovation channel as a driver of the optimal policy. Finally, I plan to study how results change as returns to research are increasing within sector. 
%Due to the endogeneity of technological growth in the model, the reduction in work effort fosters less research especially in the non-energy sector.  %However, more hours worked in the Ramsey model fostering research would violate the emission target. As a result, growth in technology and in consumption is inefficiently low in order to meet the emission target. 

\begin{comment}
To shed more light on the main findings, I plan conduct several additional quantitative experiments. First, I want to reduce the size of the emission target, second, I allow for a longer time frame until net-zero emissions have to be reached. The IPCC report states that for a temperature target of 2°C net-zero emissions have to be reached by 2070 only. How does this laxer target affect the importance of labour income taxes. Given the wider time frame, the green sector might be able to catch up and growth could continue. Finally, how does a change in spillovers shape the result? % \textit{(Question: I guess that substitutability is key here! Growth in green implies growths in fossil when goods are no perfect substitutes! )}
content...

%Another central aspect of the paper is the importance of inequality for the optimal environmental policy. How does household heterogeneity in labour supply shape the optimal environmental policy? First, I hypothesise that the skill bias of the green sector makes a less progressive income tax optimal. 
One main result of the paper is reduction of consumption and work effort as an optimal policy. So far, I have assumed that households are passive and preferences are fixed; there is no trade-off between environmental quality and consumption from a household perspective.
In an extension to the baseline model, I plan to depart from the representative agent assumption and explicitly model household heterogeneity. This setting allows to capture a change in household behaviour: A share of households is willing to voluntarily reduce consumption. I provide evidence for such behaviour using a representative Dutch dataset. More than 50\% of households are willing to reduce consumption in order to help the economy. Importantly, these households have a higher likelihood to work in the green sector. How does such a change in behaviour affect the optimal policy? Given the additional reduction in green-specific labour supply, the planner might find it optimal to set a more regressive tax to booster green production and research.    

\end{comment}

%However, data suggests, that households do care, and they express a willingness to reduce consumption.\footnote{\ The data I have studied comes from the Liss Panel, a representative sample of Dutch households, more than 50\% of participants indicate a readiness to change their behaviour to help the environment.} I want to study the effect of such behavioural  change on the optimal policy. Interestingly, households in high-skill jobs are more likely to declare their willingness to reduce. This linkage may intensify the trade-off between reduction and green labour supply. 


%1) BN and inequality
%2) preferences for labour
\begin{comment}
Preferences and the trade-off between leisure and consumption determining household behaviour seem to be key to the results. As argued by \cite{Boppart2019labourPerspectiveb}, the intensive margin of hours worked have been falling steadily over the last 130 years. They argue for the consistency of preferences which feature a slightly higher income effect than substitution effect. In the current model with log-utility and representative family framework,  the substitution effects offset each other. With the preferences suggested in \cite{Boppart2019labourPerspectiveb}, growth would affect hours worked, assumably changing the optimal policy. It could, for instance, be the case, that growth has to be slowed down even more, to prevent too high work efforts and consumption levels. % high-income, high-skill households might increase their labour supply with growth. 

content...
\end{comment}



%Finally, endogenising growth constitutes another interesting trade-off when the impact of fiscal policy is skill specific. 
%As regards growth, it seems reasonable to consider growth as a change in the substitutability of dirty and clean goods in the final consumption good. As it stands now, growth in the dirty sector results in emission growth, ceteris paribus. Growth might instead be associated with a more efficient use of dirty energy sources, so that more output can be generated at lower emissions.
%
%Think about effects of government using revenues for other consumption. Then reducing demand will diminish demand for the final good. 
%Broadly speaking, there are two channels through which distortionary labour taxation affects emissions. First, by affecting households' labour supply decision (efficiency channel) and second in a mechanical way by changing households disposable income. The latter effect cancels out when tax revenues are used by the government to consume the final output good. Allowing the government to recycle revenues in a different way than for final good consumption uncloses another instrument to reduce emissions. 

%Further ideas for extensions: include behavioural aspects: a voluntary reduction in demand, and a lower disutility from working in the green sector.
\begin{comment}
\paragraph{Ways forward}
How to introduce compositional effects:
\begin{enumerate}
	\item 	Utility function: With substitution and income effect not canceling (u(c)=$\frac{c^{1-\gamma}}{1-\gamma},\ \gamma\neq 1$), the wage rate might play a role, depends on GE effects.
	\item endogenising skill supply (rep agent chooses how much skill to supply, but this he already does... / might need to introduce structure as in HSV)
	\item government revenues are not used for final good consumption. Instead,  disposed of/ used for sth useful (this could be an extension and contribute to benefits of progressivity) THINK THIS ONLY CHANGES THE LEVEL TOO!
\end{enumerate}
\paragraph{Point 1 above}
change the utility function in the code to see what happens, if $\frac{Y_d}{Y_c}$ is constant in particular 
\paragraph{Point 3 above}
\textcolor{blue}{2) Government consumption wasted}
Letting the government not consume the final output good may alter the result. 
Now, the aggregate price level is determined endogenously as the goods market does not clear by Walras' law. 

In the equilibrium equations, I drop $p_t=1$ and use goods market clearing instead\\ $Y=c+\psi (x_c+x_d)$.

Blödsinn, only changes level

content...
\end{comment}
\appendix
\section{Appendix}
\subsection{Growth and the Environment}
It is a vibrant debate whether technological process will result in a production technology that is perfectly clean in that it does not exert any environmental externality. 
\begin{itemize}
	%\item \underline{Extensions to technology in \cite{Acemoglu2012TheChange} }
	%\begin{itemize}
	\item \underline{externality of ``clean'' sector} \citep[see also][]{Dasgupta2021, Brock2005ChapterEmpirics}
	\begin{itemize}
		\item[-] renewable/ non-fossil fuels \ar externalities in production process are present e.g. production of solar panels uses toxic inputs \citep{Yue2014DomesticAnalysis}; non-fossil fuel nitrogen generation (e.g., biomass burning to clear land) important ($\approx$ 50\%) \citep{Song2021ImportantEmissions}; low but chronical levels of nitrogen cause species extinctions \citep{Clark2008LossGrasslands}
		\item[-] waste (after use) \ar depends on recycling technology %\ar recycling system for solar panels not profitable enough today
		%	\item[-] substitutability of nature in production (input sources eg. fossil vs. non-fossil fuels)
		%\end{itemize}
		%\item Irreversibilities already before thresholds are hit (e.g. species extinction)
		
	\end{itemize}
	%\item greenhouse gases: Carbon dioxide $CO_2$ (vast majority), Nitrous oxide $N_2O$, methane $CH_4$
	%\item stock of nature globally determined
	\item \underline{parallel positive trend in demand} (population growth, rebound effect) that outperforms increase in clean technology growth \small{(no long-run issue if perfectly clean technology exists)}
	\item \normalsize{\underline{objective function}:} \cite{Arrow2004AreMuch}(Journal of Economic Perspectives) \ar using a sustainability measure they provide evidence that consumption is too high (= not leaving enough natural resources for future generations)
	\item \underline{risk, ambiguity}
	\item if have to meet climate target in short run, might need to lower production to do so; or it might be better in terms of inequality?
\end{itemize}
\section{Guideline Computations}
\begin{enumerate}
	\item calibrate initial situation to data, using observed tax rates (this would be a competitive equilibrium with taxes as given)
	\item find BGP; BGP exists when $c^*_s+c_n^*\geq \bar{c}$; that is optimal allocation without penalty satisfies basic needs; from this point onwards there are no reallocations across sectors and sectors grow at a constant rate, this is equivalent to the solution of the problem without penalty term \ar \textbf{Need to solve Ramsey problem for BGP absent penalty term}
	Could also solve for the BGP in Ramsey model numerically: get model equations and set growth corrected variables to constant values\\
	Follow \cite{Jones1993OptimalGrowth}:
	\begin{enumerate}
		\item fix assumed SS tax rates and transfers relative to output
		\item calculate ss values of consumption/output, other variables relative to output (constant in ss)
		\item make end corrections to Ramsey problem which is explicitly solved up to period T given the values from point 1 and 2 above
		\item iterate until guess in 1 matches with solution for ss value 
	\end{enumerate}
end corrections are derived analytically.
	\item for competitive equilibrium follow \cite{Acemoglu2008CapitalGrowth}: 
	\begin{enumerate}
		\item analytically or numerically calculate BGP values
		\item initial values and parameters match to data (including tax rates and transfers)
		\item use shooting (or relaxation) algorithm to find solution, i.e. sequence of allocations that solve two boundary value problem: shooting algorithm finds initial conditions that s.t. ss values are matched. 
		\item proof uniqueness of transition path? 
	\end{enumerate}
	
	they write \begin{quote}
		The previous subsection demonstrated that there exists a unique CGP with  nonbalanced  sectoral  growth;  that  is,  there  is  aggregate  output growth at a constant rate together with differential sectoral growth and reallocation of factors of production across sectors. We now investigate whether the competitive equilibrium will approach the CGP. 
	\end{quote} 
\ar From where my economy starts today with calibrated ws, and taxes (a constant growth path), does it converge to a new constant growth path where ws is higher? 
\item in my model growth rates of sectoral consumption are not constant over time! Households reallocate shares as they get richer
\end{enumerate}

Simple version without growth
\begin{enumerate}
	\item economy is in SS today, as households income does not change their consumption is fixed, period t=0
	\item then in period t=1 ws rises (1)\ar what is the optimal policy when ws rises starting from calibrated tax rates; (2)\ar what is the new ss and how does the economy converge?
	\item[\ar] I know initial and end conditions, the shock system is also known a priori, then use shooting or relaxation algorithm to calculate transition
\end{enumerate}
%-------------------------------------
\clearpage
\bibliography{../../../bib_2_0}
\addcontentsline{toc}{section}{References}
\end{document}