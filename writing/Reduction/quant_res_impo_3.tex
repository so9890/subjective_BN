\section{Quantitative results}\label{sec:res}


In this section, I present and discuss the quantitative results.
Subsection \ref{subsec:mr} depicts the optimal policy under the benchmark policy regime: environmental tax revenues are consumed by the government and an income tax is available. Subsection \ref{subsec:dis} discusses the results focusing on the role of labor income taxes, endogenous growth, and skill heterogeneity. 
%I focus on analyzing the mechanisms and welfare benefits from integrating the income tax scheme into the environmental policy. I also discuss the costs of not using lump-sum transfers.


\subsection{Results}\label{subsec:mr}

%This section depicts results on the optimal policy followed by the implied allocation in the benchmark model where environmental tax revenues are redistributed via the income tax scheme. 

\begin{figure}[h!!]
	\centering
	\caption{Optimal Policy }\label{fig:optPol}
	\begin{minipage}[]{0.4\textwidth}
		\centering{\footnotesize{(a) Income tax progressivity, $\tau_{\iota t}$}}
		%	\captionsetup{width=.45\linewidth}
		\includegraphics[width=1\textwidth]{../../codding_model/own_basedOnFried/optimalPol_190722_tidiedUp/figures/all_10Aout22/Single_OPT_T_NoTaus_taul_regime3_spillover0_noskill0_sep1_xgrowth0_extern0_etaa0.79.png}
	\end{minipage}
	\begin{minipage}[]{0.1\textwidth}
		\
	\end{minipage}
	\begin{minipage}[]{0.4\textwidth}
		\centering{\footnotesize{(b) Environmental tax, $\tau_{Ft}$ }}
		%	\captionsetup{width=.45\linewidth}
		\includegraphics[width=1\textwidth]{../../codding_model/own_basedOnFried/optimalPol_190722_tidiedUp/figures/all_10Aout22/Single_OPT_T_NoTaus_tauf_regime3_spillover0_noskill0_sep1_xgrowth0_extern0_etaa0.79.png}
	\end{minipage}

%\begin{minipage}[]{0.4\textwidth}
%	\centering{\footnotesize{(a) Income tax progressivity, $\tau_{\iota t}$}}
%	%	\captionsetup{width=.45\linewidth}
%	\includegraphics[width=1\textwidth]{../../codding_model/own_basedOnFried/optimalPol_190722_tidiedUp/figures/all_10Aout22/Single_OPT_T_NoTaus_taul_regime3_spillover0_noskill0_sep1_xgrowth0_extern0_PV0_etaa0.79.png}
%\end{minipage}
%\begin{minipage}[]{0.1\textwidth}
%	\
%\end{minipage}
%\begin{minipage}[]{0.4\textwidth}
%	\centering{\footnotesize{(b) prog no target }}
%	%	\captionsetup{width=.45\linewidth}
%	\includegraphics[width=1\textwidth]{../../codding_model/own_basedOnFried/optimalPol_190722_tidiedUp/figures/all_10Aout22/Single_OPT_NOT_NoTaus_taul_regime3_spillover0_noskill0_sep1_xgrowth0_extern0_etaa0.79.png}
%\end{minipage}
\end{figure} 

%\paragraph{Optimal policy}
To meet the emission limits suggested by the IPCC, the optimal income tax is progressive for all periods; see panel (a) in figure \ref{fig:optPol}.  The x-axis indicates the first year of the 5 year period to which the variable value corresponds. 
% optimal taul over time:   
%    0.1043    0.0978    0.0926    0.0882    0.0845    0.0811    0.0707    0.0681    0.0659    0.0638    0.0619   -1.4313

The optimal income tax scheme is progressive, but the progressivity is decreasing over time. Starting from a value of 0.104 in 2020 it steadily decreases until the net-zero emission limit is implemented in 2050. Then, the progressivity parameter displays some discontinuity dropping from 0.08 to 0.071 from which it continues to decline to 0.062 in 2070.
Overall, the optimal tax progressivity is approximately  around half the size found for the US in \cite{Heathcote2017OptimalFramework}: $\tau_{l}=0.181$.

%   optimal tauf
%      0.8583    0.8663    0.8739    0.8812    0.8879    0.8943    0.9481    0.9501    0.9520    0.9538    0.9555    0.9664

Consider panel (b). The optimal fossil tax is increasing over the period considered and jumps to higher levels when the zero-net emission limit is introduced in 2050.
In 2020, the environmental tax equals 86\% of the fossil sector's revenues, from where it rises steadily to 90\% in 2030.  As the emission limit declines to net-zero in 2050, the tax rapidly surges to 95\% and gradually increases afterwards reaching 97\% in 2070. 

%\paragraph{Allocation}
Figure \ref{fig:optAll} depicts the optimal allocation. Limiting emissions in line with the Paris Agreement is concomitant with both a reduction and recomposition of consumption and production over time. 
Panel (a) shows consumption which increases over time; yet, its increase is slowed when the net-emission limit become active in  2050. Labor effort of both skill types increase. The rise is more pronounced when 2050 as the net-zero limit becomes binding; panel (b). The rise in hours can be explained by the drop in income tax progressivity. In comparison to hours supplied by low-skill workers, high-skill workers augment hours more as the tax progressivity declines; compare panel (c) which shows the ratio of hours worked by high to low skill workers. 

The rise in consumption after each reduction is driven by technological progress in all sectors; compare panel (d) which shows 5-year growth rates by sector and as aggregate in per cent. 
The green sector sees a rise in technological progress, the dashed black line, while growth in the fossil and the non-energy sector is positive, yet diminishing over time. Overall, aggregate growth is positive and increasing; compare the gray dashed graph. 

Research efforts, shown in panel (e) decrease over time; compare the gray graph which depicts the sum of researchers across sectors. When the net-zero limit is binding, there is a recomposition towards the green sector: while research in the non-energy and the fossil sector decrease over time, green research effort rises. 
Finally, the fossil sector has a higher labor input than the green sector. 

\begin{figure}[h!!]
	\centering
	\caption{Optimal Allocation }\label{fig:optAll}
	
	
	\begin{minipage}[]{0.32\textwidth}
		\centering{\footnotesize{(a) Consumption}}
		%	\captionsetup{width=.45\linewidth}
		\includegraphics[width=1\textwidth]{../../codding_model/own_basedOnFried/optimalPol_190722_tidiedUp/figures/all_10Aout22/Single_OPT_T_NoTaus_C_regime3_spillover0_noskill0_sep1_xgrowth0_extern0_etaa0.79.png}
	\end{minipage}
	\begin{minipage}[]{0.32\textwidth}
		\centering{\footnotesize{(b) Hours worked }}
		%	\captionsetup{width=.45\linewidth}
		\includegraphics[width=1\textwidth]{../../codding_model/own_basedOnFried/optimalPol_190722_tidiedUp/figures/all_10Aout22/SingleJointTOT_regime3_OPT_T_NoTaus_Labour_spillover0_noskill0_sep1_xgrowth0_extern0_PV1_etaa0.79_lgd1.png}
	\end{minipage}
	\begin{minipage}[]{0.32\textwidth}
		\centering{\footnotesize{(c) High-to-low-skill ratio}}
		%	\captionsetup{width=.45\linewidth}
		\includegraphics[width=1\textwidth]{../../codding_model/own_basedOnFried/optimalPol_190722_tidiedUp/figures/all_10Aout22/Single_OPT_T_NoTaus_hhhl_regime3_spillover0_noskill0_sep1_xgrowth0_extern0_etaa0.79.png}
	\end{minipage}
	\begin{minipage}[]{0.32\textwidth}
		\centering{\footnotesize{\ \\ (d) Technology growth}}
		%	\captionsetup{width=.45\linewidth}
		\includegraphics[width=1\textwidth]{../../codding_model/own_basedOnFried/optimalPol_190722_tidiedUp/figures/all_10Aout22/SingleJointTOT_regime3_OPT_T_NoTaus_Growth_spillover0_noskill0_sep1_xgrowth0_extern0_PV1_etaa0.79_lgd1.png}
	\end{minipage}
%\begin{minipage}[]{0.32\textwidth}
%	\centering{\footnotesize{\ \\ (d) Technology growth}}
%	%	\captionsetup{width=.45\linewidth}
%	\includegraphics[width=1\textwidth]{../../codding_model/own_basedOnFried/optimalPol_190722_tidiedUp/figures/all_July22/SingleJointTOT_regime0_OPT_T_NoTaus_Growth_spillover0_noskill0_sep1_xgrowth0_extern0_etaa0.79_lgd1.png}
%\end{minipage}
	\begin{minipage}[]{0.32\textwidth}
		\centering{\footnotesize{\ \\(e) Scientists }}
		%	\captionsetup{width=.45\linewidth}
		\includegraphics[width=1\textwidth]{../../codding_model/own_basedOnFried/optimalPol_190722_tidiedUp/figures/all_10Aout22/SingleJointTOT_regime3_OPT_T_NoTaus_Science_spillover0_noskill0_sep1_xgrowth0_extern0_PV1_etaa0.79_lgd1.png}
	\end{minipage}
	\begin{minipage}[]{0.32\textwidth}
		\centering{\footnotesize{\ \\(f) Labor input}}
		%	\captionsetup{width=.45\linewidth}
		\includegraphics[width=1\textwidth]{../../codding_model/own_basedOnFried/optimalPol_190722_tidiedUp/figures/all_10Aout22/SingleJointTOT_regime3_OPT_NOT_NoTaus_LabourInp_spillover0_noskill0_sep1_xgrowth0_extern0_PV1_etaa0.79_lgd1.png}
	\end{minipage}
\end{figure} 



%\subsubsection{Consumption equivalence}
%
%The importance of the income tax schedule amounts to 9.28\% of per period consumption. The bulk of this utility gain is driven by future periods. When limiting the measure of the consumption equivalence to the 55 years considered in the explicit optimization, the CEV reduces to 0.18\%.  

%\begin{figure}[h!!]
%	\centering
%	\caption{Change relative to first period }\label{fig:optAll_percEffOpt}
%	
%	
%	\begin{minipage}[]{0.32\textwidth}
%		\centering{\footnotesize{(a) Consumption}}
%		%	\captionsetup{width=.45\linewidth}
%		\includegraphics[width=1\textwidth]{../../codding_model/own_basedOnFried/optimalPol_190722_tidiedUp/figures/all_10Aout22/C_PercentageEffOptFirstPeriod_Target_regime3_spillover0_noskill0_sep1_xgrowth0_etaa0.79_lgd1.png}
%	\end{minipage}
%	\begin{minipage}[]{0.32\textwidth}
%		\centering{\footnotesize{(b) high-skill hours worked }}
%		%	\captionsetup{width=.45\linewidth}
%		\includegraphics[width=1\textwidth]{../../codding_model/own_basedOnFried/optimalPol_190722_tidiedUp/figures/all_10Aout22/hh_PercentageEffOptFirstPeriod_Target_regime3_spillover0_noskill0_sep1_xgrowth0_etaa0.79_lgd0.png}
%	\end{minipage}
%	\begin{minipage}[]{0.32\textwidth}
%		\centering{\footnotesize{(c) low-skill hours worked}}
%		%	\captionsetup{width=.45\linewidth}
%		\includegraphics[width=1\textwidth]{../../codding_model/own_basedOnFried/optimalPol_190722_tidiedUp/figures/all_10Aout22/hl_PercentageEffOptFirstPeriod_Target_regime3_spillover0_noskill0_sep1_xgrowth0_etaa0.79_lgd0.png}
%	\end{minipage}
%	\begin{minipage}[]{0.32\textwidth}
%		\centering{\footnotesize{\ \\ (d) Green-to-fossil technology ratio }}
%		%	\captionsetup{width=.45\linewidth}
%		\includegraphics[width=1\textwidth]{../../codding_model/own_basedOnFried/optimalPol_190722_tidiedUp/figures/all_10Aout22/AgAf_PercentageEffOptFirstPeriod_Target_regime3_spillover0_noskill0_sep1_xgrowth0_etaa0.79_lgd1.png}
%	\end{minipage}
%	%\begin{minipage}[]{0.32\textwidth}
%	%	\centering{\footnotesize{\ \\ (d) Technology growth}}
%	%	%	\captionsetup{width=.45\linewidth}
%	%	\includegraphics[width=1\textwidth]{../../codding_model/own_basedOnFried/optimalPol_190722_tidiedUp/figures/all_July22/SingleJointTOT_regime0_OPT_T_NoTaus_Growth_spillover0_noskill0_sep1_xgrowth0_extern0_etaa0.79_lgd1.png}
%	%\end{minipage}
%	\begin{minipage}[]{0.32\textwidth}
%		\centering{\footnotesize{\ \\(e) Green-to-fossil scientists ratio }}
%		%	\captionsetup{width=.45\linewidth}
%		\includegraphics[width=1\textwidth]{../../codding_model/own_basedOnFried/optimalPol_190722_tidiedUp/figures/all_10Aout22/sgsff_PercentageEffOptFirstPeriod_Target_regime3_spillover0_noskill0_sep1_xgrowth0_etaa0.79_lgd1.png}
%	\end{minipage}
%	\begin{minipage}[]{0.32\textwidth}
%		\centering{\footnotesize{\ \\(f) Green-to-fossil labor input}}
%		%	\captionsetup{width=.45\linewidth}
%		\includegraphics[width=1\textwidth]{../../codding_model/own_basedOnFried/optimalPol_190722_tidiedUp/figures/all_10Aout22/LgLf_PercentageEffOptFirstPeriod_Target_regime3_spillover0_noskill0_sep1_xgrowth0_etaa0.79_lgd1.png}
%	\end{minipage}
%\end{figure} 
%



%%%%%%%%%%%%%%%%%%%%%%%%%%%%%%%%%%%%%%%%%%%%%%%%%%%%%%%%%%%%%%%%%%%%%%%%%%%%%%%%%%%%
%% DISCUSSION 
%%%%%%%%%%%%%%%%%%%%%%%%%%%%%%%%%%%%%%%%%%%%%%%%%%%%%%%%%%%%%%%%%%%%%%%%%%%%%%%%%%%%

\subsection{Discussion}\label{subsec:dis}
%\tr{Questions}
%\begin{itemize}
%	\item why progressive tax? and why the drop in progressivity in 2050? (a means to boost high-skill supply and keeping low skill stable)
%	\item what are the costs of the progressive tax
%\end{itemize}

 What explains the optimal policy?
 In section \ref{subsec:notaul}, I contrast the optimal allocation under the benchmark policy regime to a scenario where no income tax is available. The impact of endogenous growth and skill heterogeneity on the optimal policy is analyzed in section \ref{subsec:xgrnsk}.
%Finally, in section \ref{subsec:comp_lumpsum}, I turn to analyze the optimal allocation under the alternative policy regimes: redistribution of environmental tax revenues via (1) lump-sum transfers and (2) the income tax scheme. 

%\begin{enumerate}
%	\item What is the goal of policy intervention? \ar social planner allocation
%	\item (Benefits) What is different when no integrated policy is run and instead revs consumed by government \ar Benefits of an integrated policy
%	\item double dividend literature: use of labor income tax when all env tax revenues are consumed by the government.
%	\item (Costs) What cannot be reached by integrated policy as compared to lump-sum transfers: is taul used for different purpose? without endogenous growth should be zero; eg. can use taul to boost growth as lump-sum transfers take care of labor supply 
%	\item What could be reached if there was no trade-off with heterogenous skills or growth? no heterogeneous skills, no endogenous growth \ar how does the optimal policy differ?
%\end{enumerate}

%\subsubsection{Comparison to other policy regimes}
%\tr{To be rewritten}
%How does the optimal allocation and especially its relation to the efficient allocation change under alternative policy scenarios?
%In this section, I discuss two policy alternations which have already been discussed in the analytical section. First, a version where environmental tax revenues are consumed by the government and no labor income tax scheme is available, henceforth referred to as \textit{separate policy}. The comparison of this scenario serves to assess the benefits of an integrated environmental-fiscal policy when no lump-sum transfers are available. 
%Second, I look at the optimal allocation 

\subsubsection{The role of income taxes}\label{subsec:notaul}



In figure \ref{fig:optAll_percLf_dyn}, I contrast the optimal allocation under the benchmark regime with income tax scheme, black solid graph, with the optimal allocation without labor income tax, the blue dashed graph. As a benchmark to the optimal policy, the figure depicts the social planner's allocation by the orange dotted graph.\footnote{\ I formulate the social planner's problem in appendix section \ref{app:sp_prob}.} 
The efficient allocation can be perceived as the allocation the Ramsey planner seeks to implement. However, she may not be able to achieve the efficient allocation due to the reliance on tax instruments.
All graphs depict percentage changes relative to the laissez-faire allocation of the same period.\footnote{\ Figure \ref{fig:LF} in appendix section \ref{app:quant_res} shows the laissez-faire allocation. } Except for panel (f) which compares the environmental tax in the model with and without income tax. 

\begin{figure}[h!!!]
	\centering
	\caption{Costs and benefits of progressive income taxes }\label{fig:optAll_percLf_dyn}
	\begin{minipage}[]{0.32\textwidth}
		\centering{\footnotesize{(a) Consumption\\ \ }}
		%	\captionsetup{width=.45\linewidth}
		\includegraphics[width=1\textwidth]{../../codding_model/own_basedOnFried/optimalPol_190722_tidiedUp/figures/all_10Aout22/C_PercentageLFDynNT_Target_regime3_spillover0_noskill0_sep1_xgrowth0_etaa0.79_lgd1.png}
	\end{minipage}
	\begin{minipage}[]{0.32\textwidth}
		\centering{\footnotesize{(b) High-skill hours worked\\ \  }}
		%	\captionsetup{width=.45\linewidth}
		\includegraphics[width=1\textwidth]{../../codding_model/own_basedOnFried/optimalPol_190722_tidiedUp/figures/all_10Aout22/hh_PercentageLfDynNT_Target_regime3_spillover0_noskill0_sep1_xgrowth0_PV1_etaa0.79_lgd0.png}
	\end{minipage}
	\begin{minipage}[]{0.32\textwidth}
		\centering{\footnotesize{(c) Low-skill hours worked\\ \ }}
		%	\captionsetup{width=.45\linewidth}
		\includegraphics[width=1\textwidth]{../../codding_model/own_basedOnFried/optimalPol_190722_tidiedUp/figures/all_10Aout22/hl_PercentageLfDynNT_Target_regime3_spillover0_noskill0_sep1_xgrowth0_PV1_etaa0.79_lgd0.png}
	\end{minipage}
	\begin{minipage}[]{0.32\textwidth}
		\centering{\footnotesize{\ \\(d) Aggregate technology growth\\ \ }}
		%	\captionsetup{width=.45\linewidth}
		\includegraphics[width=1\textwidth]{../../codding_model/own_basedOnFried/optimalPol_190722_tidiedUp/figures/all_10Aout22/gAagg_PercentageLfDynNT_noeff_Target_regime3_spillover0_noskill0_sep1_xgrowth0_PV1_etaa0.79_lgd0.png}
	\end{minipage}
	\begin{minipage}[]{0.32\textwidth}
		\centering{\footnotesize{\ \\(e) Environmental tax, $\tau_{Ft}$\\ \ }}
		%	\captionsetup{width=.45\linewidth}
		\includegraphics[width=1\textwidth]{../../codding_model/own_basedOnFried/optimalPol_190722_tidiedUp/figures/all_10Aout22/comp_benchregime3_notaul2_OPT_T_NoTaus_tauf_spillover0_noskill0_sep1_xgrowth0_PV1_etaa0.79_lgd0.png}
	\end{minipage}
	\begin{minipage}[]{0.32\textwidth}
		\centering{\footnotesize{\ \\(f) Aggregate research\\ \  }}
		%	\captionsetup{width=.45\linewidth}
		\includegraphics[width=1\textwidth]{../../codding_model/own_basedOnFried/optimalPol_190722_tidiedUp/figures/all_10Aout22/S_PercentageLfDynNT_noeff_Target_regime3_spillover0_noskill0_sep1_xgrowth0_PV1_etaa0.79_lgd0.png}
	\end{minipage}
	\begin{minipage}[]{0.32\textwidth}
		\centering{\footnotesize{\ \\(g) Green-to-fossil energy\\ \  }}
		%	\captionsetup{width=.45\linewidth}
		\includegraphics[width=1\textwidth]{../../codding_model/own_basedOnFried/optimalPol_190722_tidiedUp/figures/all_10Aout22/GFF_PercentageLfDynNT_Target_regime3_spillover0_noskill0_sep1_xgrowth0_PV1_etaa0.79_lgd0.png}
	\end{minipage}
	\begin{minipage}[]{0.32\textwidth}
		\centering{\footnotesize{\ \\(h) Green-to-fossil scientists ratio\\ \ \\ \ }}
		%	\captionsetup{width=.45\linewidth}
		\includegraphics[width=1\textwidth]{../../codding_model/own_basedOnFried/optimalPol_190722_tidiedUp/figures/all_10Aout22/sgsff_PercentageLfDynNT_Target_regime3_spillover0_noskill0_sep1_xgrowth0_PV1_etaa0.79_lgd0.png}
	\end{minipage}
	\begin{minipage}[]{0.32\textwidth}
		\centering{\footnotesize{\ \\(i) Green-to-fossil labor input\\ \ }}
		%	\captionsetup{width=.45\linewidth}
		\includegraphics[width=1\textwidth]{../../codding_model/own_basedOnFried/optimalPol_190722_tidiedUp/figures/all_10Aout22/LgLf_PercentageLfDynNT_Target_regime3_spillover0_noskill0_sep1_xgrowth0_PV1_etaa0.79_lgd0.png}
	\end{minipage}
\begin{minipage}[]{0.32\textwidth}
	\centering{\footnotesize{\ \\(j) Energy share\\ \ }}
	%	\captionsetup{width=.45\linewidth}
	\includegraphics[width=1\textwidth]{../../codding_model/own_basedOnFried/optimalPol_190722_tidiedUp/figures/all_10Aout22/EY_PercentageLfDynNT_Target_regime3_spillover0_noskill0_sep1_xgrowth0_PV1_etaa0.79_lgd0.png}
\end{minipage}
\begin{minipage}[]{0.32\textwidth}
\centering{\footnotesize{\ \\(k) Non-energy scientists share\\ \ }}
%	\captionsetup{width=.45\linewidth}
\includegraphics[width=1\textwidth]{../../codding_model/own_basedOnFried/optimalPol_190722_tidiedUp/figures/all_10Aout22/snS_PercentageLfDynNT_noeff_Target_regime3_spillover0_noskill0_sep1_xgrowth0_PV1_etaa0.79_lgd0.png}
\end{minipage}
\begin{minipage}[]{0.32\textwidth}
	\centering{\footnotesize{\ \\(l) Green-to-fossil technology\\ \ }}
	%	\captionsetup{width=.45\linewidth}
	\includegraphics[width=1\textwidth]{../../codding_model/own_basedOnFried/optimalPol_190722_tidiedUp/figures/all_10Aout22/AgAf_PercentageLfDynNT_Target_regime3_spillover0_noskill0_sep1_xgrowth0_PV1_etaa0.79_lgd0.png}
\end{minipage}
	\floatfoot{Notes: \footnotesize{ The figure shows the percentage deviation of the allocation resulting under the benchmark policy, that is, with income tax (the black solid graph), the allocation under the benchmark policy but the income tax is not available (the blue dashed graph), and the efficient allocation (the orange dotted graph), in relation to the laissez-faire allocation. 
			Panel (d) shows aggregate growth where the variable value in $t$ refers to the growth rate from  period $t$ to period $t+1$. Hence, from 2045 to 2050 growth reduces significantly, since in 2050 the net-emission limit has to be satisfied. Panel (f) shows the level of the environmental tax under the two policy regimes compared.
}}
\end{figure} 

\clearpage
%
% Labor supply
In comparison to a policy scenario without income tax, the availability of an income tax allows to more closely resemble the efficient levels of labor, panels (b) and (c). 
The social planner reduces hours worked for both the high- and the low-skill type by between 3 to 4 percent relative to the laissez-faire allocation.\footnote{\ Over time, the reduction in hours declines; the social planner chooses a higher work effort especially for the high-skill type. The reason is that more consumption becomes more valuable as the emission limit becomes stricter. This is the income effect of  externality mitigation on hours discussed in the analytical section. For the impact of the emission limit on the efficient allocation see figure \ref{fig:eff_with_notarget} in appendix section \ref{app:quant_res}.}


When no labor income tax is available, the reduction in  hours worked by the low type remains close to zero, the dashed graph. Under the same policy regime, hours of high-skill workers even increase slightly above laissez-faire level due to the strengthened importance of high-skill-intense green energy. When the government has income taxes available, it reduces hours worked of both types closer to the efficient allocation. While the efficient allocation sees a similar decline in working time of both types, the optimal policy allocation features a stronger decrease in working time by high-skill labor and too low a reduction of the low-skill ones. This result emerges from the higher wage elasticity of substitution for the high-skill type which works more hours in the laissez-faire allocation so that leisure is more valuable. The income channel of the wage rate is similar to both worker types because of perfect income insurance.

\begin{comment}
%- Labor income taxes have advantage in terms of growth
A second benefit of progressive income taxation in the quantitative model stems from endogenous growth and knowledge spillovers. 
In the periods before the net-zero emission limit (2020 to 2050), the optimal policy with progressive income tax achieves a higher aggregate growth rate; consider panel (d). The additional gains in terms of growth arise from substituting fossil taxes with income taxes; see the stronger reduction in fossil taxes in this period in panel (g).  The intuition goes as follows: by partly substituting fossil taxes with labor income taxation the economy can profit more from knowledge spillovers from the biggest research sector: the non-energy sector.\footnote{\ The higher growth rate, indeed, emerges from spillover effects and not a higher research effort; in fact, the amount of scientists is reduced more when an income tax is available, panel (f). However, the share of non-energy research increases (compare panel (c) in figure \ref{fig:optAll_percLf_dyn_app}). }
As the corrective tax reduces, the price for energy diminishes less, and energy becomes relatively less expensive. A price effect directs research to the more expensive non-energy sector. % the fossil tax is especially costly because it redirects research away from the non-energy sector which becomes relatively cheaper.

%Although the increase in growth rates is small - not a percentage point difference in growth rate reduction per period - the total effect on the future is substantial as highlighted by the consumption equivalence. 

 This mechanism underlines an advantage of reductive environmental policies as opposed to recomposing strategies. Here, the income tax and the fossil tax act as substitutes.  %\footnote{\ An alternative explanation for the advantage of income taxes above environmental taxes under the presented policy regime could be that labor taxes are redistributed to households, so there is no reduction in consumption via government consumption. To test this alternative explanation, I run a model version where both income and fossil taxes are redistributed through the income tax. And the results persist.}
Once the net-zero emission limit becomes binding in 2050, however, the gap between environmental taxes reduces and aggregate growth in the model without income taxes is minimally higher.   This suggests that the net-zero emission limit prevents to substitute fossil with labor income taxes to reap the gains from knowledge spillovers.\footnote{\ \cite{Acemoglu2012TheChange} demonstrate that consumption growth is hampered by a transition from dirty to green production when the dirty sector is more productive. However, their channel persists absent knowledge spillovers. Rather, final good production is slowed down more when dirty and fossil goods are easily substitutabel: then, the good with the higher technology is favored in production and technology improvements in the backward sector do not contribute to overall output growth.} 

content...
\end{comment}
% costs
The benefits of the progressive income tax come at the cost of less consumption, panel (a),  and a lower green-to-fossil energy mix, panel (g), and a higher energy share, panel (j). The social planner implements continues consumption growth and only reduces consumption below laissez-faire levels during the first periods. In contrast, the optimal allocation reduces consumption relative to the laissez-faire world in all periods. The reduction is increasing over time,\footnote{\ This observation speaks to the literature investigating limits to growth. When the Ramsey planner can only use fossil and income taxes, the emission limit is best satisfied by a continues reduction in growth relative to the laissez-faire allocation. Yet, the government cannot use research subsidies in the present setting. However, the efficient allocation with emission limit is characterized by a reduction in consumption and in consumption growth; consider figure \ref{fig:eff_with_notarget}.} and it is stronger in the model with income tax. As growth rates are higher, or only minimally lower in the model with income tax, the additional reduction in consumption is explained by lower work effort. The additional decrease in consumption as the net-zero limit is established, occurs despite more work effort (note that hours worked in the laissez-faire equilibrium are constant over time). The reduction in growth and a lower marginal product of labor as the fossil tax increases explain this result. 

The higher green-to-fossil energy ratio (panel (g)) in the efficient allocation is driven by a reallocation of input factors towards the green sector; see panel (i). In contrast, the ratio of scientists remains largely unchanged (panel (h)). This observation suggests that the social planner recomposes the economy by inputs and not through an adjustment in technology growth in order to further profit from knowledge spillovers. In fact, these spillovers in favor of the less advanced sector enable the social planner to implement a higher green to fossil technology ratio than in the laissez-faire economy (panel (l)). 

In the optimal allocation, irrespective of the policy regime, the increase in the green-to-fossil energy mix is inefficiently low. When an income tax is available, the optimal policy mix mutes the rise even further. The reduction is explained by both less green-to-fossil research (panel (h)) and labor input (panel(i)). 
The adverse recomposition, however, does not, as hypothesized,  arise from the higher income tax progressivity. Rather, the lower fossil tax explains this result. 
%This recomposing effect of the labor income tax arises from the higher responsiveness of high skill labor to the tax progressivity.
To back this claim, I only feed the optimal income tax into the model and compare the resulting allocation to the laissez-faire economy. Figure \ref{fig:LF_vs_onlytaul} in the appendix shows the results. The green-to-fossil energy mix only reduces slightly in response to the progressive income tax. The following three paragraphs serve to understand this result.

There are two mechanisms shaping the recomposing effect of the labor income tax on the economic structure. Abstract for now from endogenous growth.
The first asymmetry results from a higher labor share in the fossil sector compared to the green one; I will refer to this channel as \textit{labor-share} channel. Therefore, an overall reduction in labor supply has a stronger effect on the fossil sector. Via this mechanism, income tax progressivity boosts green production. Figure \ref{fig:LF_vs_onlytaul_xgrnsk} shows the effect of income tax progressivity in the model with neither endogenous growth nor skill heterogeneity. Therefore, the income tax affects the economic structure solely through the level of labor supply. The reduction in labor supply makes the fossil good relatively cheaper, demand for green energy increases, and the labor share employed in the green sector grows.  

However, the labor-share channel is muted by endogenous growth. Adding endogenous growth to the model with exogenous growth and one skill type (see figure \ref{fig:LF_vs_onlytaul_nsk}), the green-to-fossil energy ratio remains unchanged relative to the laissez-faire allocation. 
Therefore, in the benchmark model with endogenous growth and skill heterogeneity the \textit{skill-recomposition} channel dominates and the effect of the progressive income tax on the green-to-fossil energy share is negative. Nevertheless, it is relatively small as price adjustments absorb the effect of a relatively higher low-skill supply on the direction of innovation; consider again figure \ref{fig:LF_vs_onlytaul}.\footnote{\ A similar analysis to the one below applies to the share of energy and non-energy goods in final consumption. The non-energy good is more labor intense than the energy good and features a lower high-skill share. Again, a reduction in the energy-share dominates in the benchmark model. Research effort, again, is invariant to the changes in labor supply. }

Given the small recomposing effect of the labor income tax absent the fossil tax leads to the conclusion that it is the reduction of the fossil tax which drives the adverse effect on the energy mix once the government can use an income tax scheme. 
Because of the reductive effect of the progressive income tax, a higher fossil share does not conflict with meeting the emission limit. 

\begin{comment}
  This again transmits to research efforts, as machine producers' profits from research in the fossil sector are higher thereby amplifying the recomposing effect of income taxes. This finding is in line with the theoretic considerations in the literature: first,  since green and fossil energy are substitutes, a market size effect may dominate the price effect attracting research efforts in the sector with higher input supply. Second, complementarity of the non-energy and energy goods combined with a reliance of energy on the scarcer sector implies that research is directed towards the sector where input goods are scarcer; i.e., energy (compare figure \ref{fig:optAll_percLf_dyn_app} in the appendix). 

content...
\end{comment}
\begin{comment}
COMMENT ON WEAK DD

These results speak to the weak double-dividend literature. %When the government consumes environmental tax revenues, hours worked are inefficiently high. 
The weak double-dividend result posits that when environmental tax revenues suffice to cover all government funding requirements, it would be optimal to lower distortionary income taxes. The results presented herein, however, show that there is a lower bound. Lowering distortionary income taxes too much results in inefficiently high hours worked. Hence, even though there is no motive to fund government expenses  labor income taxation is not zero due to the environmental externality.
%Indeed, this reduces consumption further away from the efficient level, but, hours worked are aligned closer to the efficient level, panels (b) and (c). Next to consumption, the planner also forfeits an advantageous green-to-fossil energy ratio, panel (e). 

%The use of a progressive labor income tax contributes minimally to meeting the emission limit as can be seen by scrutinizing the optimal environmental tax, panel (b) in figure \ref{fig:comp_nored_pol}: when income taxes can be used, the environmental tax is lower. Still, the difference is minimal, supporting the thesis of complementarity of income and environmental taxes. 
%Environmental tax revenues are lower as the tax rate reduces, and income taxes reduce labor supply and hence the tax base of the environmental tax. 
%Even though labor income taxes have the advantage of being redistributed to households and lowering the externality, they are not used to substitute environmental tax revenues.\footnote{\ This might be a motive to prefer labor income taxes as an instrument to reduce emissions since labor income tax revenues are redistributed back to the household while environmental tax revenues are not in this setting. Nevertheless, the observation that the environmental tax only adjusts slightly once an income tax tool is available points to the advantage of environmental taxes in handling too high emissions.
%}
\end{comment}
\begin{comment}

\subsection{Knowledge spillovers}\label{subsec:noknow}
\begin{figure}[h!!]
	\centering
	\caption{The role of knowledge spillovers}\label{fig:comp_kn}
	\begin{minipage}[]{0.32\textwidth}
		\centering{\footnotesize{(a) Income tax progressivity, $\tau_{\iota t}$\\ \ }}
		%	\captionsetup{width=.45\linewidth}
		\includegraphics[width=1\textwidth]{../../codding_model/own_basedOnFried/optimalPol_190722_tidiedUp/figures/all_10Aout22/taul_KNCOUNT_FullMod_regime3_spillover0_noskill0_sep1_xgrowth0_PV1_etaa0.79_lgd1.png}
	\end{minipage}
	\begin{minipage}[]{0.32\textwidth}
		\centering{\footnotesize{(b) Environmental tax, $\tau_{Ft}$\\ \ }}
		%	\captionsetup{width=.45\linewidth}
		\includegraphics[width=1\textwidth]{../../codding_model/own_basedOnFried/optimalPol_190722_tidiedUp/figures/all_10Aout22/tauf_KNCOUNT_FullMod_regime3_spillover0_noskill0_sep1_xgrowth0_PV1_etaa0.79_lgd0.png}
	\end{minipage}	
\begin{minipage}[]{0.32\textwidth}
	\centering{\footnotesize{(c) Consumption\\ \ }}
	%	\captionsetup{width=.45\linewidth}
	\includegraphics[width=1\textwidth]{../../codding_model/own_basedOnFried/optimalPol_190722_tidiedUp/figures/all_10Aout22/C_KNCOUNT_FullMod_regime3_spillover0_noskill0_sep1_xgrowth0_PV1_etaa0.79_lgd0.png}
\end{minipage}
\begin{minipage}[]{0.32\textwidth}
\centering{\footnotesize{(d) Aggregate technology growth\\ \ }}
%	\captionsetup{width=.45\linewidth}
\includegraphics[width=1\textwidth]{../../codding_model/own_basedOnFried/optimalPol_190722_tidiedUp/figures/all_10Aout22/gAagg_KNCOUNT_FullMod_regime3_spillover0_noskill0_sep1_xgrowth0_PV1_etaa0.79_lgd0.png}
\end{minipage}
\begin{minipage}[]{0.32\textwidth}
\centering{\footnotesize{(e) High-skill hours worked\\ \ }}
%	\captionsetup{width=.45\linewidth}
\includegraphics[width=1\textwidth]{../../codding_model/own_basedOnFried/optimalPol_190722_tidiedUp/figures/all_10Aout22/hh_KNCOUNT_FullMod_regime3_spillover0_noskill0_sep1_xgrowth0_PV1_etaa0.79_lgd0.png}
\end{minipage}
\begin{minipage}[]{0.32\textwidth}
	\centering{\footnotesize{(f) Low-skill hours worked\\ \ }}
	%	\captionsetup{width=.45\linewidth}
	\includegraphics[width=1\textwidth]{../../codding_model/own_basedOnFried/optimalPol_190722_tidiedUp/figures/all_10Aout22/hl_KNCOUNT_FullMod_regime3_spillover0_noskill0_sep1_xgrowth0_PV1_etaa0.79_lgd0.png}
\end{minipage}
\begin{minipage}[]{0.32\textwidth}
\centering{\footnotesize{(d) Non-energy technology growth\\ \ }}
%	\captionsetup{width=.45\linewidth}
\includegraphics[width=1\textwidth]{../../codding_model/own_basedOnFried/optimalPol_190722_tidiedUp/figures/all_10Aout22/gAn_KNCOUNT_FullMod_regime3_spillover0_noskill0_sep1_xgrowth0_PV1_etaa0.79_lgd0.png}
\end{minipage}
\begin{minipage}[]{0.32\textwidth}
\centering{\footnotesize{(d) Non-energy scientist share\\ \ }}
%	\captionsetup{width=.45\linewidth}
\includegraphics[width=1\textwidth]{../../codding_model/own_basedOnFried/optimalPol_190722_tidiedUp/figures/all_10Aout22/snS_KNCOUNT_FullMod_regime3_spillover0_noskill0_sep1_xgrowth0_PV1_etaa0.79_lgd0.png}
\end{minipage}
\begin{minipage}[]{0.32\textwidth}
\centering{\footnotesize{(d) Non-energy scientist wage\\ \ }}
%	\captionsetup{width=.45\linewidth}
\includegraphics[width=1\textwidth]{../../codding_model/own_basedOnFried/optimalPol_190722_tidiedUp/figures/all_10Aout22/wsn_KNCOUNT_FullMod_regime3_spillover0_noskill0_sep1_xgrowth0_PV1_etaa0.79_lgd0.png}
\end{minipage}
\begin{minipage}[]{0.32\textwidth}
\centering{\footnotesize{(e) Green scientists\\ \ }}
%	\captionsetup{width=.45\linewidth}
\includegraphics[width=1\textwidth]{../../codding_model/own_basedOnFried/optimalPol_190722_tidiedUp/figures/all_10Aout22/sg_KNCOUNT_FullMod_regime3_spillover0_noskill0_sep1_xgrowth0_PV1_etaa0.79_lgd0.png}
\end{minipage}
\begin{minipage}[]{0.32\textwidth}
\centering{\footnotesize{(f) fossil scientists\\ \ }}
%	\captionsetup{width=.45\linewidth}
\includegraphics[width=1\textwidth]{../../codding_model/own_basedOnFried/optimalPol_190722_tidiedUp/figures/all_10Aout22/sff_KNCOUNT_FullMod_regime3_spillover0_noskill0_sep1_xgrowth0_PV1_etaa0.79_lgd0.png}
\end{minipage}
\begin{minipage}[]{0.32\textwidth}
\centering{\footnotesize{(g) green technology\\ \ }}
%	\captionsetup{width=.45\linewidth}
\includegraphics[width=1\textwidth]{../../codding_model/own_basedOnFried/optimalPol_190722_tidiedUp/figures/all_10Aout22/Ag_KNCOUNT_FullMod_regime3_spillover0_noskill0_sep1_xgrowth0_PV1_etaa0.79_lgd0.png}
\end{minipage}
\begin{minipage}[]{0.32\textwidth}
\centering{\footnotesize{(h) Energy\\ \ }}
%	\captionsetup{width=.45\linewidth}
\includegraphics[width=1\textwidth]{../../codding_model/own_basedOnFried/optimalPol_190722_tidiedUp/figures/all_10Aout22/E_KNCOUNT_FullMod_regime3_spillover0_noskill0_sep1_xgrowth0_PV1_etaa0.79_lgd0.png}
\end{minipage}
\begin{minipage}[]{0.32\textwidth}
\centering{\footnotesize{(i) Fossil output\\ \ }}
%	\captionsetup{width=.45\linewidth}
\includegraphics[width=1\textwidth]{../../codding_model/own_basedOnFried/optimalPol_190722_tidiedUp/figures/all_10Aout22/F_KNCOUNT_FullMod_regime3_spillover0_noskill0_sep1_xgrowth0_PV1_etaa0.79_lgd0.png}
\end{minipage}
\begin{minipage}[]{0.32\textwidth}
\centering{\footnotesize{(j) Fossil technology\\ \ }}
%	\captionsetup{width=.45\linewidth}
\includegraphics[width=1\textwidth]{../../codding_model/own_basedOnFried/optimalPol_190722_tidiedUp/figures/all_10Aout22/Af_KNCOUNT_FullMod_regime3_spillover0_noskill0_sep1_xgrowth0_PV1_etaa0.79_lgd0.png}
\end{minipage}
\end{figure}
This section highlights how knowledge spillovers shape the optimal policy. When there are no knowledge spillovers, i.e., $\phi=0$, the optimal policy is characterized by a higher environmental tax and a lower income tax progressivity parameter. This comparison stresses the advantage of avoiding fossil taxation but relying on labor income taxes to lower emissions. 

In the counterfactual exercise, I show how the benchmark economy would evolve if the optimal policy tuple resulting in the model without knowledge spillovers was implemented. 
Clearly, the economy would meet the emission limit at lower consumption growth but more labor effort. 

SHOW: policy comparison, consumption, technology growth, and labor supply 

	content...
\end{comment}
\subsubsection{Endogenous growth and skill heterogeneity}\label{subsec:xgrnsk}

I now address the impact of certain model features on the optimal policy. 
Figure \ref{fig:comp_mod} presents the optimal policy in a model with exogenous growth, the blue dashed graphs, and without skill heterogeneity, the orange dotted graphs. The black solid line indicates the results in the benchmark model. 

\paragraph{Endogenous growth}
When growth is exogenous, the optimal income tax progressivity is higher over the whole time period studied (panel (a)); the elasticity of after- to pre-tax income declines. Furthermore, the reduction in tax-progressivity over time is muted: the lowest level of tax progressivity in 2070 is slightly below $0.09$ compared to roughly above $0.06$ in the benchmark model. 

%The reduction in consumption below the respective efficient allocation is only slightly more than 10\% at maximum whereas consumption falls short of the efficient level by up to 55\% in the benchmark model; figure \ref{fig:comp_mod_allo_dev} in the appendix shows how optimal allocations deviate from the efficient one for the three models studied.

The rationale behind the higher tax progressivity and its attenuated decline in the exogenous growth model is that the labor-share channel dominates the skill-recomposition effect of income tax progressivity.\footnote{\ Compare figures \ref{fig:LF_vs_onlytaul_xgrnsk} and \ref{fig:LF_vs_onlytaul_xgr} which show models with exogenous growth but without and with skill heterogeneity, respectively.}
When growth is endogenous, the skill-recomposition channel dominates making income taxes less advantageous from an environmental policy perspective.
 %Since the fossil sector relies more on labor than the green sector, the reduction in labor makes fossil production relatively more expensive. This increases the green-to-fossil energy mix. Same holds true for the composition of the final output good which is recomposed towards the less labor intense energy good. However, with endogenous growth, this channel is muted and the adverse recomposing effect of labor income tax progressivity on the green energy share dominates. 
%Overall, the beneficial recomposing effect in the exogenous growth model explains the higher labor tax progressivity and the muted decline over time. 

%Figure \ref{fig:count_taul_xgr} shows how the economies evolve when the optimal income tax from the benchmark model is fed into (1) the exogenous growth model, the black graph, and (2) the exogenous growth model, the blue dashed graph. The respective grey graphs show the allocation in the laissez-faire economy.
%
%
%The reason for the higher tax progressivity in the exogenous growth model relative to the benchmark model is not driven by the effect of income tax progressivity on growth. Growth seems largely unaffected by the income tax; compare panel (e) for the level of growth and panel (f) for the ratio of green to fossil scientists. The effect on innovation is absorbed by wages which increase with the labour income tax; see panels (l) and (m).  
%
%Rather, the higher tax progressivity in the model with exogenous growth arises from the higher environmental tax and the complementarity of the two instruments. This narrative becomes clear when comparing how the social planner adjusts labor supply differently in the two models, see figure \ref{fig:eff_model}. The efficient level of hours worked is lower in the exogenous growth model. Not because it is less costly to reduce hours worked, but because the environmental tax is more aggressive. The channel analyzed in the analytical section.  




%\tr{is it the recomposiing effect or the effect on growth in general? \ar both not present---Cant tell }
The environmental tax is higher in all periods when growth is exogenous (panel (b)). This finding is in line with \cite{Fried2018ClimateAnalysis} who argues that directed technical change amplifies the recomposing effects of the fossil tax. The higher price for fossil energy increases demand for green energy. Since innovation responds to this shift in demand, the slowdown in fossil production is amplified. 


\begin{figure}[h!!]
	\centering
	\caption{Optimal policy by model}\label{fig:comp_mod}
	
	\begin{minipage}[]{0.4\textwidth}
		\centering{\footnotesize{(a) Income tax progressivity, $\tau_{\iota t}$\\ \ }}
		%	\captionsetup{width=.45\linewidth}
		\includegraphics[width=1\textwidth]{../../codding_model/own_basedOnFried/optimalPol_190722_tidiedUp/figures/all_10Aout22/CompMod1_OPT_T_NoTaus_taul_regime3_spillover0_noskill0_sep1_xgrowth0_extern0_PV1_etaa0.79_lgd1.png}
	\end{minipage}
	\begin{minipage}[]{0.1\textwidth}
		\
	\end{minipage}
	\begin{minipage}[]{0.4\textwidth}
		\centering{\footnotesize{(b) Environmental tax, $\tau_{Ft}$\\ \ }}
		%	\captionsetup{width=.45\linewidth}
		\includegraphics[width=1\textwidth]{../../codding_model/own_basedOnFried/optimalPol_190722_tidiedUp/figures/all_10Aout22/CompMod1_OPT_T_NoTaus_tauf_regime3_spillover0_noskill0_sep1_xgrowth0_extern0_PV1_etaa0.79_lgd0.png}
	\end{minipage}
\end{figure}
\begin{comment}

Despite the more aggressive intervention, consumption in the Ramsey allocation only deviates by -10\% from the efficient allocation over the whole time period considered (panel (a) in figure \ref{fig:comp_mod_allo}). In contrast, in the benchmark model, the deviation of consumption aggravates over time reaching -55\% relative to the respective efficient allocation. 
Interestingly, the recomposing effect of tax progressivity through its impact on the skill ratio is negligible in the exogenous growth model. Indeed, the skill ratio diverges more from the efficient ratio due to the higher tax progressivity (panel (e)). However, since the direction of growth does not respond to the supply of skills, the green-to-fossil energy mix is roughly similar to the efficient one (panel (d)). 

content...
\end{comment}

%- homogenous skill
\paragraph{Skill heterogeneity}
When there is only one type of skill, the optimal income tax progressivity is higher, as well, than in the benchmark model. Yet, the difference is less pronounced compared to the model with exogenous growth. Besides, optimal tax progressivity converges to the one in the benchmark model over time. 
As argued above, endogenous growth masks the advantageous recomposing effect of tax progressivity on the energy mix through the labor-share channel. 

Compared to the benchmark model, the planner does not face a trade-off between too low high-skill supply and too high low-skill supply and can implement hours close to the efficient level.\footnote{ Panel (b) in figure \ref{fig:comp_mod_allo_dev} in the appendix shows deviations from the respective efficient allocation by model. When skill heterogeneity is switched off, labor supply only deviates at maximum by -0.6\% from the efficient allocation. In the benchmark model the same number rises to -3.5\% and +1\% for the high- and the low-skill type. } This explains the higher tax progressivity. 
Furthermore, there is no increase in fossil research as is the case in the benchmark model relative to the laissez-faire allocation.\footnote{\ See figures \ref{fig:LF_vs_onlytaul} and \ref{fig:LF_vs_onlytaul_nsk} which show the effect of the respective optimal  tax progressivity relative to the laissez-faire allocation in the benchmark model and the one with skill homogeneity, respectively.}
%Indeed, the deviation in consumption and growth from the efficient allocation is similar in the model with homogeneous skills to the benchmark model (panels (a) and (g)). 

The fossil tax is higher than in the benchmark alternative throughout, the orange dotted graph in panel (b).
On the contrary to income tax progressivity, the environmental tax diverges more from the benchmark model when there is no skill heterogeneity relative to the alternative with exogenous growth. The motive behind this result is the increased input market for fossil production when there is only one skill type. A smaller low-skill supply, therefore, contributes to lower emissions. I discuss the mechanism in more detail in appendix section \ref{app:count}.

 
 
\begin{comment}
\paragraph{Comparison integrated policy to separate policy}

Consider figure \ref{fig:bench_nored_notaul}. The figure presents the optimal allocation in the integrated policy scenario,  the orange-dashed graph, the optimal allocation under the separate policy, the blue-dotted graph, and the efficient allocation, the black-solid graph.

In comparison to a policy scenario where environmental tax revenues are not redistributed, the integrated policy closer resembles the efficient allocation in terms of consumption, panel (a) and of labor, panels (b) and (c). %In total, the utility level of the representative household is at least as close to the efficient level for all time periods considered. 
The benefits of an integrated-policy regime come at the cost of a lower green-to-fossil energy mix, panel (e), and a reduction in growth, panel (d). Nevertheless, if a planner could choose between the two regimes, it would select the integrated-policy regime. The gains from the integrated regime amount to xxx. \tr{Do CEV}

Interestingly, the optimal environmental tax is only negligibly smaller in the integrated-policy regime. This suggests, that environmental taxes and labor income taxes are complements in the optimal environmental policy to lower inefficiently high hours worked. Only in the period from 2030 to 2050 the environmental tax necessary to meet emission limits is slightly smaller which can be rationalized by a lower level of production.\footnote{\ Absent an emission limit before 2030, the optimal environmental tax is slightly negative to subsidize fossil research which again spills over to research in the other sectors. }  

\begin{figure}[h!!]
	\centering
	\caption{Comparison to separate policy scenario; \tr{drop efficient from tauf graph }}\label{fig:bench_nored_notaul}
	
	\begin{minipage}[]{0.32\textwidth}
		\centering{\footnotesize{(a) Consumption}}
		%	\captionsetup{width=.45\linewidth}
		\includegraphics[width=1\textwidth]{../../codding_model/own_basedOnFried/optimalPol_190722_tidiedUp/figures/all_July22/C_CompEffOPT_T_NoTaus_pol2_spillover0_noskill0_sep1_xgrowth0_etaa0.79_lgd1_lff0.png}
	\end{minipage}
	\begin{minipage}[]{0.32\textwidth}
		\centering{\footnotesize{(b) High skill hours worked}}
		%	\captionsetup{width=.45\linewidth}
		\includegraphics[width=1\textwidth]{../../codding_model/own_basedOnFried/optimalPol_190722_tidiedUp/figures/all_July22/hh_CompEffOPT_T_NoTaus_pol2_spillover0_noskill0_sep1_xgrowth0_etaa0.79_lgd0_lff0.png}
	\end{minipage}
	\begin{minipage}[]{0.32\textwidth}
		\centering{\footnotesize{(c) Low skill hours worked}}
		%	\captionsetup{width=.45\linewidth}
		\includegraphics[width=1\textwidth]{../../codding_model/own_basedOnFried/optimalPol_190722_tidiedUp/figures/all_July22/hl_CompEffOPT_T_NoTaus_pol2_spillover0_noskill0_sep1_xgrowth0_etaa0.79_lgd0_lff0.png}
	\end{minipage}
	\begin{minipage}[]{0.32\textwidth}
		\centering{\footnotesize{(d) Aggregate growth}}
		%	\captionsetup{width=.45\linewidth}
		\includegraphics[width=1\textwidth]{../../codding_model/own_basedOnFried/optimalPol_190722_tidiedUp/figures/all_July22/gAagg_CompEffOPT_T_NoTaus_pol2_spillover0_noskill0_sep1_xgrowth0_etaa0.79_lgd0_lff0.png}
	\end{minipage}
	\begin{minipage}[]{0.32\textwidth}
		\centering{\footnotesize{(e) Energy mix, $\frac{G}{F}$}}
		%	\captionsetup{width=.45\linewidth}
		\includegraphics[width=1\textwidth]{../../codding_model/own_basedOnFried/optimalPol_190722_tidiedUp/figures/all_July22/GFF_CompEffOPT_T_NoTaus_pol2_spillover0_noskill0_sep1_xgrowth0_etaa0.79_lgd0_lff0.png}
	\end{minipage}
	%	\begin{minipage}[]{0.32\textwidth}
	%	\centering{\footnotesize{(f) Utility}}
	%	%	\captionsetup{width=.45\linewidth}
	%	\includegraphics[width=1\textwidth]{../../codding_model/own_basedOnFried/optimalPol_190722_tidiedUp/figures/all_July22/SWF_CompEffOPT_T_NoTaus_pol2_spillover0_noskill0_sep1_xgrowth0_etaa0.79_lgd0_lff0.png}
	%\end{minipage}
	\begin{minipage}[]{0.32\textwidth}
		\centering{\footnotesize{(f) Environmental tax, $\tau_{Ft}$}}
		%	\captionsetup{width=.45\linewidth}
		\includegraphics[width=1\textwidth]{../../codding_model/own_basedOnFried/optimalPol_190722_tidiedUp/figures/all_July22/tauf_CompEffOPT_T_NoTaus_pol2_spillover0_noskill0_sep1_xgrowth0_etaa0.79_lgd0_lff0.png}
	\end{minipage}
\end{figure}


	content...
\end{comment}

\begin{comment}
\subsubsection{Optimal policy and allocation with lump-sum transfers}\label{subsec:comp_lumpsum}
\tr{Or better: other policy regimes}

How do lump-sum transfers change the role of labor income taxes?\footnote{\ In appendix section \tr{To be added}, I present the optimal allocation under a policy regime where environmental tax revenues are redistributed through the income tax scheme. This scenario is relevant when the government wants to redistribute environmental tax revenues but lump-sum transfers are not feasible.}
 According to the theory in section \ref{sec:mod_an}, the use of lump-sum taxes should (i)  allow to attain an allocation closer to the efficient one and (ii) deprive the income tax scheme of its use as reductive environmental policy tool. Indeed, the optimal allocation under lump-sum transfers is much closer to the efficient one. 
 Yet, the emission limit still shapes the optimal income tax due to endogenous growth. 
  %This is so despite the advantageous recomposing effect of regressive income taxes through skill supply.
  
% YES, one can speak of a separation of environmental and fiscal policies as the goal of income taxes is to boost or lower growth in the first place. We also dont speak of the environmental tax being targeted at 

\begin{figure}[h!!]
	\centering
	\caption{Comparison integrated regime and regime lump-sum transfers}\label{fig:bench_lumpsum}
	
	\begin{minipage}[]{0.32\textwidth}
		\centering{\footnotesize{(a) Consumption}}
		%	\captionsetup{width=.45\linewidth}
		\includegraphics[width=1\textwidth]{../../codding_model/own_basedOnFried/optimalPol_190722_tidiedUp/figures/all_July22/C_CompEffOPT_T_NoTaus_bb3_pol4_spillover0_noskill0_sep1_xgrowth0_etaa0.79_lgd1_lff0.png}
	\end{minipage}
	\begin{minipage}[]{0.32\textwidth}
		\centering{\footnotesize{(b) High skill hours worked}}
		%	\captionsetup{width=.45\linewidth}
		\includegraphics[width=1\textwidth]{../../codding_model/own_basedOnFried/optimalPol_190722_tidiedUp/figures/all_July22/hh_CompEffOPT_T_NoTaus_bb3_pol4_spillover0_noskill0_sep1_xgrowth0_etaa0.79_lgd0_lff0.png}
	\end{minipage}
	\begin{minipage}[]{0.32\textwidth}
		\centering{\footnotesize{(c) Low skill hours worked}}
		%	\captionsetup{width=.45\linewidth}
		\includegraphics[width=1\textwidth]{../../codding_model/own_basedOnFried/optimalPol_190722_tidiedUp/figures/all_July22/hl_CompEffOPT_T_NoTaus_bb3_pol4_spillover0_noskill0_sep1_xgrowth0_etaa0.79_lgd0_lff0.png}
	\end{minipage}
	\begin{minipage}[]{0.32\textwidth}
		\centering{\footnotesize{(d) Aggregate growth}}
		%	\captionsetup{width=.45\linewidth}
		\includegraphics[width=1\textwidth]{../../codding_model/own_basedOnFried/optimalPol_190722_tidiedUp/figures/all_July22/gAagg_CompEffOPT_T_NoTaus_bb3_pol4_spillover0_noskill0_sep1_xgrowth0_etaa0.79_lgd0_lff0.png}
	\end{minipage}
	\begin{minipage}[]{0.32\textwidth}
		\centering{\footnotesize{(e) Energy mix, $\frac{G}{F}$}}
		%	\captionsetup{width=.45\linewidth}
		\includegraphics[width=1\textwidth]{../../codding_model/own_basedOnFried/optimalPol_190722_tidiedUp/figures/all_July22/GFF_CompEffOPT_T_NoTaus_bb3_pol4_spillover0_noskill0_sep1_xgrowth0_etaa0.79_lgd0_lff0.png}
	\end{minipage}
	\begin{minipage}[]{0.32\textwidth}
		\centering{\footnotesize{(f) Utility}}
		%	\captionsetup{width=.45\linewidth}
		\includegraphics[width=1\textwidth]{../../codding_model/own_basedOnFried/optimalPol_190722_tidiedUp/figures/all_July22/SWF_CompEffOPT_T_NoTaus_bb3_pol4_spillover0_noskill0_sep1_xgrowth0_etaa0.79_lgd0_lff0.png}
	\end{minipage}
\end{figure}

 Figure \ref{fig:bench_lumpsum} contrasts the efficient allocation, the black solid graphs, the allocation under the benchmark policy, the orange-dashed graphs, and the optimal allocation when lump-sum transfers are available, the blue-dotted graphs. 
When lump-sum transfers of environmental tax revenues are in the policy set, the Ramsey planner can implement hours worked closer to the efficient allocation, panels (b) and (c). Consumption under the regime with lump-sum transfers, as well, mirrors the efficient level more closely see panel (a). 
Growth in the scenario with lump-sum transfers is at least as high as under the benchmark policy due to the use of regressive income taxes to accelerate growth. % I DONT KNOW WHY: However, the green-to-fossil energy ratio is slightly higher under the benchmark policy under the net-zero emission limit. The recomposing mechanism of the income tax via a relatively higher supply of high-skill labor contributes to this finding. 
Utility gains from the availability of lump-sum transfers overall seem sizable especially under the net-zero emission limit; compare panel (f).

\begin{figure}[h!!]
	\centering
	\caption{Optimal policy in integrated regime and with lump-sum transfers}\label{fig:bench_lumpsum_pol}
	
	\begin{minipage}[]{0.32\textwidth}
		\centering{\footnotesize{(a) Income tax progressivity, $\tau_{\iota t}$}}
		%	\captionsetup{width=.45\linewidth}
		\includegraphics[width=1\textwidth]{../../codding_model/own_basedOnFried/optimalPol_190722_tidiedUp/figures/all_July22/comp_bb3_notaul4_OPT_T_NoTaus_taul_spillover0_noskill0_sep1_xgrowth0_etaa0.79_lgd1.png}
	\end{minipage}
	\begin{minipage}[]{0.32\textwidth}
		\centering{\footnotesize{(b) Environmental tax, $\tau_{Ft}$}}
		%	\captionsetup{width=.45\linewidth}
		\includegraphics[width=1\textwidth]{../../codding_model/own_basedOnFried/optimalPol_190722_tidiedUp/figures/all_July22/comp_bb3_notaul4_OPT_T_NoTaus_tauf_spillover0_noskill0_sep1_xgrowth0_etaa0.79_lgd0.png}
	\end{minipage}
	\begin{minipage}[]{0.32\textwidth}
		\centering{\footnotesize{(c) Lump-sum transfers}}
		%	\captionsetup{width=.45\linewidth}
		\includegraphics[width=1\textwidth]{../../codding_model/own_basedOnFried/optimalPol_190722_tidiedUp/figures/all_July22/comp_bb3_notaul4_OPT_T_NoTaus_Tls_spillover0_noskill0_sep1_xgrowth0_etaa0.79_lgd0.png}
	\end{minipage}
\end{figure}


% finding 1) income tax to boost growth, 2) no use of recomposing effect of income tax
Figure \ref{fig:bench_lumpsum_pol} shows the optimal policy when lump-sum transfers are available. 
Now, the optimal income tax scheme is regressive. Since lump-sum transfers ensure a reduction in labor supply, the inefficiency in labor supply as a motive for progressive income taxes vanishes.
Instead, the motive to boost growth and consumption dominates: absent endogenous growth, the income tax remains untouched.\footnote{\ Compare results in the model with exogenous growth depicted in figure \ref{fig:lumpsum_xgr_vglNotaul} in appendix section \ref{app:lumps}.}  In fact, the allocation attained in the model without endogenous growth but lump-sum transfers is similar to the efficient one at a zero income tax progressivity; compare figure \ref{fig:lumpsum_xgr_vglNotaul}. Hence, the environmental policy does not use income tax regressivity to recompose production towards green energy by subsidizing high-skill supply.
 
 \tr{Has to be rewritten: it is rather, that the costs of more research seem to be too high in terms of disutility.}
Nevertheless, note, though, that regressivity of the optimal income tax reduces over time; compare the orange-dashed graph in panel (a) in figure \ref{fig:bench_lumpsum_pol}. This points to the income tax, again, being shaped by the environmental externality.
This is so even though consumption is inefficiently low and higher growth rates would be efficiency improving. I conclude from this observation that it is the environmental externality which makes it optimal to forfeit growth. 
The conflict between growth and emissions, however, does not seem to arise from growth itself, since the social planner satisfies the emission limit at higher growth rates. Instead, the issue stems from labor supply as a means to boost growth in the competitive economy. 
Then, the decreasing pattern of income tax regressivity can be rationalized as follows: as growth increases, more labor means more production and hence emissions. Therefore, boost growth becomes more costly in terms of emissions. Only growth which is concomitant with lower production is feasible in the market economy.\footnote{\ If the Ramsey planner had tools at hand to boost growth without necessarily boosting production, more growth would be optimal. \tr{\textit{Does a research subsidy imply more production?}}}

%It suggests itself that the conflict with growth does not stem from growth itself but rather that it is fostered in the competitive economy through a market size effect, that is, more production, since the social planner chooses higher technology levels. This would speak against progressive income taxes as growth accelerates. Thus, the environmental indeed prevents usage of the income tax to boost demand, but I conclude that it is not used with the goal to reduce emissions through the endogenous growth channel. Since growth as such does not pose the conflict to the emission limit. 
I conclude from this discussion that it is solely the environmental tax and not the income tax which addresses the environmental externality when lump-sum transfers are available.

content...
\end{comment}
