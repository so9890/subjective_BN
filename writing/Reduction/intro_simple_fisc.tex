



\paragraph{Quantitative model}
Recent work has shown, that  higher tax progressivity is amplified in lowering inequality through a compression of the wage rate. (there is a second effect...). On the other hand, high skill labour is used in a higher share in green sectors \cite{Consoli2016DoCapital}. Therefore, progressivity implies a shift to dirty innovations and a higher externality. These channels constitute a new trade-off between inequality and climate change mitigation. 

% motivation from consumption reduction proponents
Furthermore, a higher tax progressivity reduces consumption at higher income levels (only if not smoothed by savings), production and externalities. Then again, high skill labour may be missing for green production. The reduction and recomposition mechanism counteract each other once accounting for skill heterogeneity. 

%\begin{itemize}
%	\item What is the optimal policy to achieve emission targets by 2050?
%	\item role for fiscal policy due to time frame and skill supply?
%	\item inefficient low supply of high skill labour \ar regressive tax optimal?
%	\item demand side: to counteract the negative effect of redistribution through lower skill supply? \ar demand side effect
%	\item voluntary reduction in consumption \ar even lower skill supply? \ar who are these households? (rich/ high low skill?)
%	\item non-monetary motive for scientists? 
%\end{itemize}

\subsection*{Comment: 31/03/22}
It seems difficult to solve the problem when no balanced growth path exist since fossil output has to be constant under the optimal policy. (But that is a constant growth rate, just that output ratios are not constant) 
Therefore,
\begin{enumerate}
	\item solve under assumption of constant growth rates \ar that would be the limit and determines the continuation value of the economy
	\item assume a planner who only cares about the transition to the net-zero emission economy \ar most important to satisfy voters today
	\begin{itemize}
		\item the objective function is the sum of transition periods (2020 to 2080), with the length of a period =5 years \ar 30 periods.
		\item using numeric method as in Barrage should be solvable; no continuation value; (later adding continuation value not a big deal)
		\item no need to make it stationary
		\item how does the economy evolve afterwards?
	\end{itemize}

\end{enumerate}
\subsection*{Comment: 25/03/2022}
In the models on endogenous growth, emissions positively depend on innovations in the dirty sector. Technology in this sector has to be perceived as more goods being produced each of which exerts the same level of the externality. This seems sensible, when these goods need the same input of emissions generating factors but can be produced with the same number of machines. Also sensible when thinking about waste which is on product level. But then again could think of progress as needing less input goods to achieve the same number of outputs, then should measure emissions by input factors and not output. This is captured by growth in the green sector.

Now, assuming emissions are proportional to output, and introducing the exogenous limit on emissions s.t. net-emissions have to equal zero from mid-century onward, then the assumption that on the BGP all technology ratios are constant implies that all growth has to stop. (also assuming here that carbon capture cannot grow without end). This seems very restrictive. Could divert from assumption of constant technology ratios on BGP: instead assuming a generalised BGP which allows for transitions across sectors.  On this GBGP fossil output has to remain at the same level (as an upper bound assume the one prescribed by the IPCC). Then growth in the fossil sector is zero (compatible with BGP, but technology ratios are not constant.) In fact, a BGP. 

Due to spill overs there might be room for ever growing corrective taxes to counter market forces. 
As the green sector growth, green energy becomes relatively cheaper and cheaper compared to fossil energy. This market effect could redirect production and innovation to green energy. 
The labour income tax could support by changing relative skill supply. 

\subsection*{Comment: 19/03/2022}
\begin{itemize}

	\item \textbf{Question 1 a)}: \textbf{how does the presence of a progressive tax (as calibrated) (or a demand target) change the optimal environmental policy?}\\ \ \\
%	\\ Steps
%	\begin{enumerate}
%		\item max swf st emission constraints; save optimal policy
%		\item max swf st emission constraint and demand constraint
%		\item compare optimal policies
%	\end{enumerate}
\textbf{Motivation:} How tax progressivity affects the externality: lowering demand on the one hand reduces emissions as output decreases; on the other hand, labour supply incentives change, potentially more so for high skilled than for low skilled workers.\\
In the literature on optimal environmental policies, positive labour income taxes lower the optimal environmental tax below the Pigouvian rate, due to efficiency costs. In this setting, however, the optimal environmental tax might be higher to counteract the positive effect of income tax progressivity on the externality. \\
Starting from \cite{Fried2018ClimateAnalysis}; set up:  model with rep agent, max social welfare function plus target;  the planner can choose corrective taxes the income tax is given exogenously
\begin{enumerate}
	\item amend model to incorporate income taxes and skill heterogeneity
\item set up her model to find the optimal environmental tax as done in \cite{Barrage2019OptimalPolicy}
\item 
\end{enumerate}
\item[\ar] check model implications in the data: is there heterogeneity in the effect of tax progressvity on the externality across countries which differ in their wage-hour elasticity. 

	\item Question 1 b): (Optimal policy) Is there the potential for the income tax code to be targeted at the externality even if a corrective tax is present and there is no demand target?
Why? Rather, a further reduction of tax progressivity might be optimal to increase high-skill labour supply. 
\ar \textbf{another trade-off between inequality and the externality}. Not only via the efficiency channel but due to redistribution, innovations will be directed towards the polluting sector. \\
Add exogenous demand target as suggested by natural scientists. What is the optimal policy in this case? Could also find that the environemntal tax is used to lower aggregate output when goods are complements! 

\item \textbf{Adding inequality}
On the other hand, the environmental tax could be lower than absent inequality as it increases wage dispersion. 

	\item Question 2 (This refers to \cite{Loebbing2019NationalChange}): reducing consumption bears the potential of social unrest. \ar add inequality. How would a social planner choose to meet the targets if he searches to minimise social tension/ impact on the poor/ utilitarian swf? 
	\\
	set up: two household types, max swf and targets; additional motive to avoid inequality
	\\
	Steps
	\begin{enumerate}
	\item what are the distributional effects of the policy found for question 1?
	\item How would the optimal planner set the optimal policy now? 
	\end{enumerate}
\item why not look at good specific taxes? \ar because (1) hit the poor (regressivity), (2) corresponds to corrective tax! \ar but then no need for income tax
\end{itemize}

New motivation: \\
Natural scientists have identified a reduction in demand for energy and land-intense products as key to meeting climate targets and sustainability goals jointly, or to diminish the reliance on risky carbon dioxide reduction technologies. More broadly, \cite{Arrow2004AreMuch} argue for the efficiency gains of lowering aggregate consumption through the use of public policy instruments. 
However, a dynamic general equilibrium analysis is missing. 

\paragraph{How to get to this?}
\ar Amend \cite{Fried2018ClimateAnalysis}. 
\begin{enumerate}
\item add income tax and elastic labour supply
\end{enumerate}
\subsection*{Comment: 16/03/2022}
\begin{itemize}
	\item this version: the planner maximises social welfare but is constrained by an emission target; the planner only has labour income taxes at its disposal; the economy can be represented by a representative agent
	\item way forward (quantitatively):
	\begin{enumerate}
		\item model rep agent as only supplying one skill \ar $\zeta=1$
			\item depart from log utility of consumption; use preferences with a slightly higher income effect than substitution effect as suggested by \cite{Boppart2019labourPerspectiveb}.
		\item  add a target on demand to the Ramsey problem\ar the planner not only has to meet emission targets but also a target on demand which is motivated by the natural sciences debate on climate change: 
		\begin{quote}Reduction policies alleviate the pressure to meet other sustainability goals \citep{Bertram2018TargetedScenarios}, they reduce the necessity to rely on \textit{carbon dioxide removal} technologies which are not without risk as they rely on  underground CO2 storage and compete with land needed for food production and biodiversity protection \citep{VanVuuren2018AlternativeTechnologies}.
		\end{quote}
	then, the planner might choose income taxes to meet the emission target even if corrective taxes are available.

		\item  introduce heterogeneity  (skills) 
		\item  introduce directed technical change 
		\item what if households want to lower consumption absent any policy intervention? How does this change the analysis?
	\end{enumerate}
	
\end{itemize}

\section{Introduction}
% this intro refers to the following setup:
% The government can only use labour taxes and corrective taxes are not available
% e.g what can the government achieve with common 

% Structure Intro
% 1. Motivation: (2) setting real world, (3) Why is the question interesting? (Tradeoff)

% 2. What I do: Contribution and main finding

% 3. Model (several layers)

% 4. Calibration

% 5. Main quantitative experiment and results
% 
% To do: 
%\tr{ (i) Connect paragraphs,
% (ii) guide reader, 
% (iii) make smooth }

\begin{comment}
\textcolor{violet}{Still to do:
\begin{itemize}
	\item possibilities to model technical change: substitutability of goods, growth in sector, innovation on substitutability versus consumption growth
\end{itemize}
}

content...
\end{comment}

%\paragraph{Classical use of fiscal instruments}
An equity-efficiency trade-off is central to the discussion of optimal labour income taxation and tax progressivity in the public finance literature.  The benefits of labour taxes and progressivity arise, inter alia, from redistribution. %and from generating government revenues. 
With concave utility specifications full redistribution is efficient. However, the optimal tax system does not feature full redistribution when labour supply is endogenous. Instead, redistribution is traded off against aggregate output as individuals reduce their labour supply and skill investment in response to labour income taxation. 

%\paragraph{Environmental Externality}
Adding environmental externalities to the classical public finance framework changes the perception of efficiency costs. Instead of merely reducing welfare, direct benefits through a reduction of the externality arise by lowering output. 
In theory, corrective, environmental taxes can establish the efficient allocation in a representative agent economy. Absent inequality, such a tax instrument is optimally set to the social cost of an externality. Originators then internalise these costs in addition to their private ones. However, governments face political difficulties in implementing such policy instruments.\footnote{\ Compare, for instance, the Yellow Vest movement in France in 2018.} On the other hand, scientific research has emphasised the urgency to act and highlighted the advantages of lowering demand for land and energy.
For example, reduction policies alleviate the pressure to meet other sustainability goals \citep{Bertram2018TargetedScenarios}, they reduce the necessity to rely on \textit{carbon dioxide removal} technologies which are not without risk as they rely on  underground CO2 storage and compete with land needed for food production and biodiversity protection \citep{VanVuuren2018AlternativeTechnologies}.
 Therefore, this paper shifts the focus of optimal environmental policies  to fiscal tax instruments as tools to lower demand and meet emission targets. What can be achieved in terms of climate targets and what are the costs?

%\paragraph{Trade-off/ Mechanisms}
 Consumption reduction in affluent countries has been promoted as an environmental policy \citep{Schor2005SustainableReduction, Pullinger2014WorkingDesign, Arrow2004AreMuch}. But, the general equilibrium effects are less well understood.
While having an advantageous direct effect on the externality, counteracting indirect effects may exist in a general equilibrium framework. Proponents of a reduction policy especially focus on consumption by the rich which consume a higher amount of natural resources.\footnote{\ There is a bunch of research on the consumption of resources by income groupd; see for instance \cite{Sager2019IncomeCurves}.} %\footnote{\ Note Sonja: Abstracting from inequality, would it still be best to reduce consumption by the rich when the poor have a higher marginal propensity to consume dirty? \textit{Could be an important aspect in the model}. Not in the baseline, look at it in an extension...}
This concern could add to the benefits of tax progressivity.
In contrast, targeting rich households in particular for environmental reasons will lower the supply of high skilled labour.\footnote{\ The relation of labour income tax progressivity and skill investment has been studied by \cite{Heathcote2017OptimalFramework}.} Yet, these skills are essentially important in greener sectors of the economy \citep{Consoli2016DoCapital}. As a result, dirty production becomes relatively cheaper and the dirty share of production rises. 
% I want to add endogenous innovation later

%\paragraph{Model}
% this version: With Rep agent
I build a tractable model which incorporates the key aspects sketched above. There are two sectors one of which emits pollutants: the dirty sector. Both clean and dirty goods are necessary inputs to the final consumption good. Sectors produce with a sector-specific labour input good. The labour input good in the clean sector contains a higher share of high-skilled labour. 
The economy behaves as if there was a representative household which provides high and low-skilled labour. The former exerts a higher utility cost for the household generating a wage premium for high-skilled labour. 

The government maximises social welfare from a Ramsey planner's perspective. However, it is constrained by an exogenous limit on emissions. The advantage of this approach is that it suffers less from  model misspesifications due to  uncertainties about how emissions affect the environment. Furthermore, it is closely related to the current political debate.\footnote{\ Compare appendix section \ref{app:emission_climate_targets} for a more in depth discussion of this aspect. } 

%\paragraph{Calibration}
I inform the exogenous emission limit by the  targets proposed in the 2018 report of the Intergovernmental Panel on Climate Change (IPCC)\footnote{\ A body of the United Nations established to assess the science related to climate change.},  \cite{Rogelj2018MitigationDevelopment.}. These targets are designed for states to comply with the Paris Agreement: global net greenhouse-gas emissions in 2030 shall equal 25-30 GtCO2e per year and zero in 2050 (p.95 in \cite{Rogelj2018MitigationDevelopment.}).%Indeed,  the agreement  foresees a tight time frame for emission reductions: climate neutrality should be achieved by mid-century.
\footnote{\ Under this treaty, states have agreed on limiting temperature rise to well below 2°C, preferably to 1.5°C, and to achieve climate neutrality by mid-century \url{https://unfccc.int/process-and-meetings/the-paris-agreement/the-paris-agreement}. }
%Compared to integrated climate assessment models, (CHECK DEFINITION) this approach requires less assumptions concerning the relation of emissions and the climate. What

Another important calibration choice is the substitutability of clean and dirty production in the final consumption good. I make the cautious assumption that goods are no perfect substitutes. In other words, there is always at least a small amount of dirty production necessary to produce the final consumption good. \tr{\cite{Cohen2019AnnualSubstitutable} discuss and estimate the substitutability of natural capital in production with a focus on energy. }

%\paragraph{Quantitative Exercise and Results}
The paper is divided into two parts: an analytical part where I derive propositions concerning the role of fiscal policy and a quantitative part which discusses the optimal policy and transitions. 

The main theoretical result is that in the laissez-faire economy, emissions grow without bound. Irrespective of whether the clean and dirty good are substitutes or compliments.

In the quantitative exercise I let a planner choose the optimal policy by maximizing a utilitarian social welfare function but it faces an constraint on emissions. I solve explicitly for the optimal policy in each period making the optimal tax progressivity time dependent. 


\paragraph{Literature}

The paper is related to 3 strands of literature. 

First, to the public finance literature.  \cite{Heathcote2017OptimalFramework} study optimal labour tax progressivity on 



Second, to the literature on optimal environmental policy. 

Third, to the literature on directed technical change. 
\textbf{HEMOUS and Olsen} discuss an endogenous growth model with heterogeneous labour input:
\begin{itemize}
	\item the wage premium is not constant on a BGP which they specify as stable if innovation occurs in both sectors
	\item hence: a non stable BGP is one where innovation does not occur in both sectors at some point
	\item need to allow for this option< when solving the model
	\item quality ladder model: each scientist after having chosen a sector of production, there is no congestion (each scientist works on one machine) legitimate due to within-sector spillovers
	\item on a BGP with equal growth the wage premium may grow (Result in \cite{Acemoglu2002DirectedChange}) 
	\item in \cite{Acemoglu2012TheChange} 
	\begin{itemize}
		\item as emissions are proportional to dirty output implicit assumption of a Leontief production function if there was energy (and also this as only source of emissions); 
		 \item endogenous labour (no sector-specific labour supply) \ar the more productive sector attracts more labour (the MPL is higher at an equal ratio so that more labour ends up in the more productive sector to have equal wages)
		 \item for substitutes innovation might be stuck in the more advanced market as the price effect (which directs innovation to the less productive market) is muted
		 \item[\ar] in \cite{Fried2018ClimateAnalysis} fossil and green energy are substitutes \ar stuck in fossil innovation; but non-energy goods and energy are complements \ar price effect strong; equalising effect
		 \item in LF stuck with dirty innovation, government can redirect innovation towards the clean sector until it catches up \ar policy intervention needer for a " sufficient amount of time" \ar might be missing in today's world
		 \item with DTC need postponing intervention problematic!
		 \item subsidies and corrective taxes needed to implement first best 
	\end{itemize}
\item \cite{Acemoglu2016TransitionTechnology} incremental (sector-specific) and radical (building on the leading technology irrespective of sector) innovation \ar cross-sectoral spillovers which were absent in \cite{Acemoglu2012TheChange}
\end{itemize}

%Finally, the paper is meant to add to the discussion on reduction versus recomposition policies as tools to reduce human impact on the environment. 

\paragraph{Outline}
The paper is structured as follows. In the next section, I define a simple model. % to analyse the role of fiscal taxes on the environment. 
Section \ref{sec:theory} discusses theoretical optimal policy results. Section \ref{sec:calib} argues for the plausibility of chosen parameter values. In section \ref{sec:simul}, I show dynamics of the economy under the laissez-faire and the optimal policy regimes. 

