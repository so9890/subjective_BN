\documentclass[11pt,aspectratio=169]{beamer}
%\usepackage[noxcolor]{beamerarticle} % to get presentation as article ! if used also set documentclass to article!
\usetheme[outer/progressbar=foot,
%outer/numbering=none
]{metropolis}
\setbeamertemplate{caption}{\raggedright\insertcaption\par}
\setbeamercolor{frametitle}{bg={}, fg=black!80}
\setbeamercolor{alerted text}{bg={}, fg=cyan!100}
\setbeamercolor{block title}{bg=black!10, fg=black}
\setbeamercolor{block body}{bg=black!10, fg=black}
%\usecolortheme{seahorse}
\usepackage[utf8]{inputenc}
\usepackage[english]{babel}
%\usepackage[T1]{fontenc}
\newcommand{\tr}[1]{\textcolor{blue}{#1}}
\usepackage{amsmath}
\usepackage{amsfonts}
\usepackage{amssymb}
\usepackage{mathtools}
\usepackage{calc}
\usepackage{soul}
\setbeamercolor{headerCol}{fg=blue!30,bg=black!80}
\setbeamercolor{bodyCol}{fg=black}
\usepackage{graphicx}
\usepackage{xcolor}
\usepackage{appendix}
\usepackage{hyperref}
\usepackage{natbib}
\usepackage{comment}
\usepackage{setspace}
\renewcommand{\bibsection}{}
\bibliographystyle{apa} 
% have to run bibtex mydocument.aux after first run to generate bbl file. 
\usepackage{appendixnumberbeamer}
\usepackage{xcolor}


\usepackage[customcolors]{hf-tikz}
\definecolor{sonja}{cmyk}{1.5,0,0.9,0.3}
%\definecolor{blue}{cmyk}{0,1,0,0}
\hfsetfillcolor{black!10}
\hfsetbordercolor{black}

\usepackage{tikz}
\usetikzlibrary{tikzmark}
\usetikzlibrary{decorations.markings}
\usepackage{tikz-cd}
\usetikzlibrary{arrows,calc,fit}
\tikzset{mainbox/.style={draw=white, text=white, fill=gray, rectangle, rounded corners, thick, node distance=7em, text width=8em, text centered, minimum height=3.5em}}
\tikzset{dummybox/.style={draw=none, text=white , rectangle, rounded corners, thick, node distance=7em, text width=8em, text centered, minimum height=3.5em}}
\tikzset{box/.style={draw , rectangle, rounded corners, thick, node distance=7em, text width=8em, text centered, minimum height=3.5em}}
\tikzset{container/.style={draw, rectangle, dashed, inner sep=2em}}
\tikzset{line/.style={draw, very thick, -latex'}}
\tikzset{    pil/.style={
		->,
		thick,
		shorten <=2pt,
		shorten >=2pt,}}
\tikzstyle{vecArrow} = [thick, decoration={markings,mark=at position
	1 with {\arrow[semithick]{open triangle 60}}},
double distance=1.4pt, shorten >= 5.5pt,
preaction = {decorate},
postaction = {draw,line width=1.4pt, white,shorten >= 4.5pt}]



%TITLE
\author[Sonja Dobkowitz]{\small Sonja Dobkowitz}
\institute[University of Bonn]{University of Bonn}
\title{What next?: Reducing Consumption Levels}

\newcommand{\ar}{$\Rightarrow$ \ }

%\addtobeamertemplate{navigation symbols}{}{%
%    \usebeamerfont{footline}%
%    \usebeamercolor[fg]{footline}%
%    \hspace{1em}%
%   \insertframenumber/\inserttotalframenumber
%}

\institute{University of Bonn} 
\date{\today} 
%\subject{} 
\begin{document}
	
	{\setbeamertemplate{footline}{}
		\begin{frame}
		\titlepage
	\end{frame}
}
%\addtocounter{framenumber}{-1}

% {\setbeamertemplate{footline}{}
% \begin{frame}{Content}
% \vspace{4mm}
% \tableofcontents
% \end{frame}
% }
 %\addtocounter{framenumber}{-1}


%---------------------------------------
%            Intro
%---------------------------------------
%\section{ Reducing Consumption Levels}
\begin{frame}{Motivation}

\begin{itemize}[<+-| alert@+>]
	\setbeamercolor{alerted text}{fg=black} %change the font color
	\setbeamerfont{alerted text}{series=\bfseries} 
	
	\item<+-| alert@+> due to climate change, we need to reduce the consumption of resources
	\vspace{4mm}
	\item<+-| alert@+> macroeconomic research largely focuses on a green \textbf{\textcolor{cyan!100}{recomposition}} of consumption 
\vspace{4mm}
	\item<+-| alert@+>  but it is unclear whether recomposition alone is sufficient to fight climate change at today's high consumption levels% some non-green production methods remain....
	%	\item<+-| alert@+> green production can meet today's high consumption levels
	%\end{enumerate}
	% the two are basically the same: some non-green technologies would remain to meet todays high levels of consumption. Or consumption has to reduce to make the recomposition sufficient

	%ADD LANCET REPORT
\vspace{4mm}
	\item[\ar]<+-| alert@+>  this project focuses on a \textbf{\textcolor{cyan!100}{reduction}}  of consumption % you might believe that a recomposition is enough given market mechanisms to foster new innovations, however, it could be the case that it is not possible => better be prepared! 
\end{itemize}
%\textcolor{blue}{insert data on inequality}
\end{frame}



%\addtocounter{framenumber}{-1}
\begin{frame}{}
	\vspace{4mm}
		\begin{block}{Research Question}
			What are the effects of a reduction in consumption?% on societal acceptance? % on inequality and macro variables?% by consumers? What would policymakers choose?\\ or\\
%What is the socially optimal policy to achieve a sufficient reduction in aggregate consumption?
	\end{block}
%\textbf{With a focus on}:
%- political economy and societal acceptance (profits and labour income)\\
%- environmental externality \small{(indirect effects through innovations, unemployment, a recomposition of demand)}
%
\pause
\textbf{Model}
\pause
\vspace{-2mm}
	\begin{itemize}[<+-| alert@+>]
		%\item neoclassical growth model: output and consumption determined by factor supply
		\item \textbf{demand-determined production level} allowing for excess supply of labour \\
		\small{(building on models of economic slack, such as \cite{Auerbach2021InequalityEconomy})}
		
		\item sectors differ with respect to the degree of resource usage % calibrated so that only producing with the low sector does not meet initial consumption levels	
		\item \textbf{Inequality}: Households differ with respect to
		\begin{itemize}
	
\item the sector where they are employed in
\item the composition and environmental cost of their consumption bundle due to\\ \textbf{basic needs} 
\item firm ownership
		\end{itemize}
	\item \textbf{directed innovations}, cost of non-green capital stock, dynamic model
	\end{itemize}
\vspace{2mm}
\pause 
\begin{block}{Focus of this paper}
(1) indirect effects on climate externality  and (2) societal acceptance of such a policy
\end{block}
\end{frame}

\begin{frame}{How to model a reduction of consumption?}
	\begin{itemize}[<+-| alert@+>]
		\item intrinsic change in household preferences \textbf{?}
		\item policy 
		\begin{itemize}
				\item reduction of working time as suggested by \cite{GoughCANGREEN} (non-separable utility)
				\item foster and establish easy ways to share durable consumption
				\item economic-ecological education (long-run policy)
			%	\item tax on consumption beyond needs or high-emission consumption 
	\item empirically study/ draw from literature studying relation of ecological cost of consumption bundles and household characteristics/ events 
		
	\end{itemize}
	\end{itemize}
\pause
\textbf{Exercise and Goal}
\begin{itemize}
\item compare effects of different policies to reduce aggregate consumption as to limit global warming to 1.5 °C above pre-industrial levels (accounting for indirect effects)
%\item  study the  political economy of consumption reduction (implementability, societal acceptance)
	\item shed light on  societal acceptance/ optimal implementation of reduction 
\end{itemize}
\end{frame}


\begin{frame}{Why a quantitative model?}
	\begin{itemize}[<+-| alert@+>]
		\setbeamercolor{alerted text}{fg=black} %change the font color
	%	\setbeamerfont{alerted text}{series=\bfseries} 
		
		\item Does a reduction in demand slow down green innovation? Or does a triggered recomposition imply an acceleration?
		\item If the poor get poorer, do
		 they adjust their composition of consumption  in a way that emissions increase?
		\item Who profits, who loses?  Does such a policy increase inequality? For example, because the poor see a lower labour income, or do profits sink in a ways so that inequality decreases?
	%	\item various mechanisms shape the effect of a reduction in consumption. Characterised by different income sources, inequality will be affected 
%\item ambiguous effect on climate externality: e.g. reduction of demand for high-emission goods \ar high-emission sectors lay off workers \ar increase in inequality if low-income households work primarily in these sectors \ar with motivation to meet basic needs those households revert to consume more emission-high goods in case they are cheaper
%\item effect on directed innovation unclear: lower demand \ar less innovations? 
%\item rebound effect?
	\end{itemize}
\end{frame}

\begin{frame}{Data}
	\begin{itemize}[<+-| alert@+>]\setbeamercolor{alerted text}{fg=black} %change the font color
	%	\setbeamerfont{alerted text}{series=\bfseries} 
		
		\item \textbf{Quantity}: measure resource consumption by household using consumption bundle from CEX 
		\item \textbf{Quality}: proxy share of green goods consumed by Nielsen sustainable panel
		\item[\ar] How high is emission-conusmption by household type? What consumption to target by policy?
		\item[\ar] How do households differ by the share of green goods consume? 
		%\item What household characteristic/ events determine consumption levels?
		\item[\ar] Informs model on how a reduction of emission-consumption can be implemented
	\end{itemize}
\end{frame}


\begin{comment}
%---------------------------------------------------------------------
\section{The environmental costs of inequality}

\begin{frame}{Motivation}
	
	\begin{itemize}[<+-| alert@+>]
		\setbeamercolor{alerted text}{fg=black} %change the font color
		\setbeamerfont{alerted text}{series=\bfseries} 
		
		\item<+-| alert@+> due to climate change, we need to reduce the consumption of resources
		\vspace{4mm}
		\item<+-| alert@+> a rising willingness to spend for green products implies a \textit{recomposition} of consumption towards less resource-intense goods
		\vspace{4mm}
		\item<+-| alert@+>  but: \textbf{\textcolor{cyan!100}{subjective basic needs}} are high, preventing a demand-driven transition to green production
		\vspace{4mm}
		\item[\ar]<+-| alert@+> How important are subjective basic needs in hampering a transition?  What are the economic consequences of a reduction of subjective basic needs?
	\end{itemize}
	%\textcolor{blue}{insert data on inequality}
\end{frame}


\begin{frame}{Motivation}
	Make model in first project a quantitative model:\\ (1) heterogenous agents to capture distribution of income more accurately;  (2) estimate social responsibility by household and \textbf{subjective} basic needs \ar households do not want to reduce the level of consumption beyond what they perceive as needed (will be a function of income);  (3) introduce carbon cycle to account for dynamics in externality; (4) directed innovation \ar interaction with demand!  \\
	
	\begin{block}{Research Question}
How important an obstacle are subjective basic needs and inequality for a transition to sustainable production?
\end{block}
	\ar empirical research has shown that income determines the level of resource consumption; Empirical work on how income inequality and C02 emissions relate and why; But, income inequality is also a factor that impacts a transition to sustainable production: subjective basic needs are a positive function of income \ar social responsibility will never be effective even if everyone is rich. We need a transition to lower subjective basic needs. 
	
	
\end{frame}

content...
\end{comment}

\begin{frame}[shrink]{References}
	
	\bibliography{../../bib_2_0}
	\bibliographystyle{apa}
\end{frame}
\end{document}