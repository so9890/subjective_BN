\documentclass[11pt,aspectratio=169]{beamer}
%\usepackage[noxcolor]{beamerarticle} % to get presentation as article ! if used also set documentclass to article!
\usetheme[outer/progressbar=foot,
%outer/numbering=none
]{metropolis}
\setbeamertemplate{caption}{\raggedright\insertcaption\par}
\setbeamercolor{frametitle}{bg={}, fg=black!80}
\setbeamercolor{alerted text}{bg={}, fg=cyan!100}
\setbeamercolor{block title}{bg=black!10, fg=black}
\setbeamercolor{block body}{bg=black!10, fg=black}
%\usecolortheme{seahorse}
\usepackage[utf8]{inputenc}
\usepackage[english]{babel}
%\usepackage[T1]{fontenc}
\newcommand{\tr}[1]{\textcolor{blue}{#1}}
\usepackage{amsmath}
\usepackage{amsfonts}
\usepackage{amssymb}
\usepackage{mathtools}
\usepackage{calc}
\usepackage{soul}
\setbeamercolor{headerCol}{fg=blue!30,bg=black!80}
\setbeamercolor{bodyCol}{fg=black}
\usepackage{graphicx}
\usepackage{xcolor}
\usepackage{appendix}
\usepackage{hyperref}
\usepackage{natbib}
\usepackage{comment}
\usepackage{setspace}
\renewcommand{\bibsection}{}
\bibliographystyle{apa} 
% have to run bibtex mydocument.aux after first run to generate bbl file. 
\usepackage{appendixnumberbeamer}
\usepackage{xcolor}


\usepackage[customcolors]{hf-tikz}
\definecolor{sonja}{cmyk}{1.5,0,0.9,0.3}
%\definecolor{blue}{cmyk}{0,1,0,0}
\hfsetfillcolor{black!10}
\hfsetbordercolor{black}

\usepackage{tikz}
\usetikzlibrary{tikzmark}
\usetikzlibrary{decorations.markings}
\usepackage{tikz-cd}
\usetikzlibrary{arrows,calc,fit}
\tikzset{mainbox/.style={draw=white, text=white, fill=gray, rectangle, rounded corners, thick, node distance=7em, text width=8em, text centered, minimum height=3.5em}}
\tikzset{dummybox/.style={draw=none, text=white , rectangle, rounded corners, thick, node distance=7em, text width=8em, text centered, minimum height=3.5em}}
\tikzset{box/.style={draw , rectangle, rounded corners, thick, node distance=7em, text width=8em, text centered, minimum height=3.5em}}
\tikzset{container/.style={draw, rectangle, dashed, inner sep=2em}}
\tikzset{line/.style={draw, very thick, -latex'}}
\tikzset{    pil/.style={
		->,
		thick,
		shorten <=2pt,
		shorten >=2pt,}}
\tikzstyle{vecArrow} = [thick, decoration={markings,mark=at position
	1 with {\arrow[semithick]{open triangle 60}}},
double distance=1.4pt, shorten >= 5.5pt,
preaction = {decorate},
postaction = {draw,line width=1.4pt, white,shorten >= 4.5pt}]



%TITLE
\author[Sonja Dobkowitz]{\small Sonja Dobkowitz}
%\institute[University of Bonn]{University of Bonn}
\title{Growth, the Environment, and Inequality}

\newcommand{\ar}{$\Rightarrow$ \ }

%\addtobeamertemplate{navigation symbols}{}{%
%    \usebeamerfont{footline}%
%    \usebeamercolor[fg]{footline}%
%    \hspace{1em}%
%   \insertframenumber/\inserttotalframenumber
%}

\date{\footnotesize{University of Bonn}} 
\institute{December 14, 2021} 
%\subject{} 
\begin{document}
	
	{\setbeamertemplate{footline}{}
		\begin{frame}
		\titlepage
	\end{frame}
}
%\addtocounter{framenumber}{-1}

% {\setbeamertemplate{footline}{}
% \begin{frame}{Content}
% \vspace{4mm}
% \tableofcontents
% \end{frame}
% }
 %\addtocounter{framenumber}{-1}


%---------------------------------------
%            Intro
%---------------------------------------
\begin{frame}{Outline}

\begin{minipage}{0.32\textwidth} 
	\textbf{1. Plan A:}
\hspace{2mm}
\vspace{26mm}
\end{minipage} 
\begin{minipage}{0.32\textwidth} 
	\tableofcontents
\end{minipage}

\textbf{2. Plan B}

\textbf{3. Conclusion}

\end{frame}
\section{Introduction}
\begin{frame}{Motivation}

\begin{itemize}
\item two measures to reduce environmental externalities prominent in literature: 
\begin{enumerate}
\item a \alert{\textbf{recomposition}} of production
\item a \textbf{\alert{reduction}} of production (especially: working time reduction for political economy reasons)
\end{enumerate}
%(1) a \alert{\textbf{recomposition}} and (2) a \textbf{\alert{reduction}} of production through a working time reduction
\vspace{3mm} 
\item the two approaches might be complementary, related through general equilibrium effects, and shaped by \textbf{\alert{inequality}}
\vspace{3mm}
\begin{itemize}
\item[-] reduction policy might be inevitable (e.g., if no perfectly clean technology is available)
\item[-] but: a reduction in demand and labour supply could affect growth in clean innovations \ar ambiguous effect on externality
\item[-] even more so as rich households' labour supply is specific to the clean sector
\end{itemize}
\vspace{3mm}
\item however,  these policies have mainly been studied separately and absent inequality
\end{itemize}
\end{frame}

\begin{frame}
	\begin{block}{Therefore...}
I want to study recomposition and reduction policies jointly. 
\\
I integrate policy instruments to lower labour supply in a model of directed technical change with  inequality. \\
\vspace{2mm}
\ar \textbf{What is the optimal policy?}
	\end{block}

\end{frame}

\begin{frame}{This Paper: Model}
	\begin{itemize}
\item model with \textbf{\alert{directed technical change}}  à la \cite{Acemoglu2012TheChange}
%\begin{itemize}
%\item[-] corrective tax, subsidies
%\item[-] clean and dirty sector
%\end{itemize}
\end{itemize}
I add 
\begin{itemize}
\item Inequality
\begin{itemize}
\item two household types: \textbf{\alert{high- and low-skilled}}
\item \alert{\textbf{sectors differ in the share of high-skilled labour }}
\end{itemize}
\item working time reduction to policy set
\begin{itemize}
\item households face a labour supply decision
\item government can levy distortionary labour and consumption taxes (optimal policy); or an arguably more popular policy: lowering maximum of hours worked
\end{itemize}

\item %\tr{@Pavel. Potentially not sure about this now}
 \hyperlink{cleanSec}{\textbf{\alert{externality to clean sector}} }
\begin{itemize}
	\item clean technology also requires natural resources
	\item and exerts environmental externalities: in production and as waste
	\item[] \ar working time reduction might become necessary (depends on regeneration capacity of nature)
\end{itemize}
	\end{itemize}
\end{frame}

\begin{frame}{This Paper: Exercises}
	%\vspace{-3mm}
\begin{block}{Policy evaluation with}
\begin{enumerate}[a)]
	\item focus on externality
	\item focus on inequality/ political feasibility
\end{enumerate}
\end{block} 
\pause
	\textbf{a) Focus on externality}: Is a working time reduction part of the optimal environmental policy when accounting for (1) directed technical change and (2) inequality?
	\vspace{0mm}
	\begin{enumerate}[(i)]
	%	\item establish conditions under which a working time reduction is optimal (analytically)
		\item in a representative agent model, calculate optimal policy to abstract from inequality (Ramsey)
		\ar effect of directed technical change
		\item plug optimal policy into a model with inequality\\ 
		\ar How does inequality shape the effect of the policy on the externality?
		\item calculate optimal policy in full model \ar How does the optimal policy change once inequality is added?
	\end{enumerate}

\end{frame}

%\begin{frame}{Outline suggestion}
%\begin{enumerate}
%\item present model
%\item derive critical parameter values for which working time reduction becomes indeed optimal in model without inequality
%\item add inequality, how do things change?
%\item look at data to see what parameter values are plausible (the with or the without working time reduction cases)
%\item simulation of transition! 
%\end{enumerate}
%\end{frame}

\begin{frame}{This Paper: Exercises}
	\begin{block}{Policy evaluation with}
		\begin{enumerate}[a)]
			\item focus on externality
			\item focus on inequality/ political feasibility
		\end{enumerate}
	\end{block} 
\pause
		\textbf{b) Focus on inequality}: Environmental policies, especially reduction policies, are \textbf{\alert{politically delicate}} \ar restriction of working hours as a politically more acceptable policy in the literature (which might replicate optimal allocation)
	%\begin{itemize}
	%	\item create winners and losers in an unequal society
	%	\item policies traditionally evaluated with a focus on employment and growth 
	%\end{itemize}
	\begin{enumerate}[(i)]
		\item Can a restriction on working hours replicate the optimal allocation from point 3?
		\item What are the effects of a restriction of working hours on inequality? 
		\item calculate optimal policy in full model from an economic policy perspective \ar Is a working hour restriction still part of the optimal policy? 
	\end{enumerate}
\end{frame}

\section{Main Part}
\begin{frame}{Roadmap}
%In this part, I want to talk about
\begin{itemize}
\item Related literature % to identify gaps
\begin{enumerate}
\item Endogenous growth and the environment
\item Working time reductions
\item Distortionary taxation/ Political economy
\end{enumerate}
\item Working time reductions % this is from a real world/ political perspective more relevant than a higher tax on labour or consumption
\begin{enumerate}
	\item What is meant by a working time reduction? 
\item Why consider a working time reduction as policy choice? % (1) environmental externality of consumption (Provide evidence already here!), (2) political economy considerations (Yellow Vests in France)
\item Why a macro model is needed ( Hypothised effects of a working time reduction ).
\end{enumerate}
\item Sketch of a model %and empirical motivations
\end{itemize}
\end{frame}

\subsection{Literature}
\begin{frame}[allowframebreaks]{Related literature: Endogenous growth and the environment}
\begin{enumerate}
\item  \underline{directed technical change}
\begin{itemize}
	\item \cite{Acemoglu2012TheChange}(AER) \ar reduction of long-run growth part of optimal policy for some parameter values; rep agent, inelastic labour supply
	\vspace{4mm}
	%\item \cite{Acemoglu2016TransitionTechnology}(JPE) \ar ? different model?
	\item \cite{Eriksson2018PhasingChange}(Economic Modelling) \ar focuses on low elasticity of substitution between clean and dirty output; continued government intervention and long-run growth drag 
\end{itemize}
\vspace{4mm}
\item  \underline{limits to growth}
\begin{itemize}
	\item \cite{Stokey1998AreGrowth}(International Economic Review) \ar environmental constraints can imply a limit to endogenous growth (AK model: no distinction between knowledge and capital); policy instruments: direct regulation, voucher system, pollution tax; rep agent; no labour decision 
	\item \cite{Jones2016LifeGrowth}(JPE)\ar safety may be valued above consumption growth under standard preferences when life and death are taken into account \ar optimal consumption growth lower than what is feasible, could fall to zero %; rep agent; optimal policy; transition
	\vspace{4mm}
	\item \textbf{\cite{Brock2005ChapterEmpirics} (Handbook of Economic Growth)}\ar argue for non-existence of pollution-free technology. It follows that, when environmental boundaries have to be respected, there is a limit to growth; no optimal policy
\end{itemize}
\vspace{4mm}
\alert{
\item[\ar] limiting growth already established as potential optimal policy in endogenous growth literature due to environmental concerns}
\end{enumerate}
\end{frame}

\begin{frame}{Related literature: Working time reduction}
\begin{enumerate}
\item \underline{... and the environment} (empirical/ argumentative studies)
\begin{itemize}
\item \cite{Schor2005SustainableReduction}(Journal of Industrial Ecology) \ar reduction in consumption inevitable to respect environmental constraints;  empirical study on slow down in hours reduction, firm incentives to keep hours per worker high (fixed costs)
\item \cite{Pullinger2014WorkingDesign}(Ecological Economics)\ar voluntary working time reduction possibilities in Netherlands/ Belgium; happiness from leisure to  offset disutility from consumption reduction%; other measures: maximum working time
%\item \cite{Cieplinski2021EnvironmentalReductionb}(Ecological Economics) \ar not clear about mechanisms
\end{itemize}
\vspace{4mm}
\item \underline{... and overconsumption} (due to habits or social preferences)

\begin{itemize}
\item \cite{Alvarez-Cuadrado2007EnvyHours}(Canadian Journal of Economics) \ar envy; benevolent planner reduces consumption; \textbf{restrictions on working hours} gets economy close to optimal policy
\end{itemize}
\vspace{4mm}
%\item[\ar] working time reduction in macro model studied with focus on habits/ social preferences; suggested as environmental policy \\
\alert{\item[\ar] missing: working time reduction policy in GE model; especially with directed technical change and inequlity}
\end{enumerate}
\end{frame}


\begin{frame}{Related Literature }
\textbf{- Distortionary taxes and the environment in endogenous growth models}
\begin{itemize}
	\item \cite{Bovenberg1997EnvironmentalGrowth}(Journal of Public Economics) \ar pre-existing distortionary taxes
\end{itemize}

\textbf{ - Political economy and limits to growth}
\begin{itemize}
\item \cite{Alesina1994DistributiveGrowth}(QJE)
\ar relation of growth and inequality from a political economy perspective
%\item  \cite{Alesina2005WorkDifferent}\ar social prefernces for leisure \ar jointly reducing working time \ar higher value of leisure
\end{itemize}


%What is missing: 
%\begin{itemize}
%	\item inequality in directed technological growth model (Aghion paper on skill input)
%	\item working time reduction policy tools in endogenous growth model with environemntal boundaries
%\end{itemize}
\end{frame}

\subsection{Working time reduction}
\begin{frame}{1. What is meant by a working time reduction?}
	\begin{itemize}
		\item increase relative value of leisure to correct for too high consumption and labour\\ \ar distortionary labour tax or consumption tax
		\item but: \textbf{\alert{politically feasible?}} \\ \ar \alert{restrictions on working hours} might be preferred when government cares about social acceptability \citep{Alvarez-Cuadrado2007EnvyHours}
		%\item[(ii)] could be that even a Ramsey planner chooses working time reduction policy when skills are not observable
		
	%	\item minimum number of holidays per year
	%	\item rights to reduce working hours
	%	\item pre-retirement policies
	%	\item[+] coupled with financial support; job protection
	%	\item increased time rights for employees (rights to career breaks, to flexible and part time working hours)
	\item \cite{Pullinger2014WorkingDesign} mentions as complements in reality:
	\begin{itemize}
\item fiscal support (credits, benefit restructuring, paid leave rights)
\item  facilitation to re-enter labour market (protection from future career losses)
%\item pre-retirement policies
%\item financial instruments to decouple working time from income reception 
	\end{itemize}
	\item France as historic example \citep[based on][]{Alvarez-Cuadrado2007EnvyHours}
	\begin{itemize}
		\item 1841: 12-hour workday was introduced
		\item 1904: Millerand law introduced 10-hours day
		\item 1936: left-wing coalition introduced 40-hour week
		\item around 1998: movement towards 35-hour week begun
	\end{itemize}
	\end{itemize}
	
\end{frame}



\begin{frame}{2. Why consider working time reduction policies in model? i}
%	Why may a working time reduction be chosen by a government for environmental reasons? 
\begin{enumerate}
	\item in literature: \cite{Alvarez-Cuadrado2007EnvyHours} studies reduction policy as response to negative externalities of consumption due to social preferences
\item here: inefficiently high consumption levels due to environmental externality
\end{enumerate}
\end{frame}

\begin{frame}{2. Why consider working time reduction policies in model? ii}
\begin{itemize}
	\item  \cite{Acemoglu2012TheChange} show that a reduction in growth is inevitable to avoid an environmental catastrophe in some cases
	\begin{itemize}
\item[-] when there is a trade-off between pollution-reducing and output-increasing innovations (p.146)
\item[-] when clean and dirty goods are weak substitutes (p.142; reduction in growth)  or complements (p. 144; no growth)
\item[-] when the initial environmental quality is low (p.142) \ar  \cite{Rockstrom2009AHumanity} argue that some planetary boundaries have already been passed 
\item[-] dependence on regeneration rate of the environment
	\end{itemize}
\alert{\item[\ar] working time reduction already optimal in this setting? }
\end{itemize}
\end{frame}



\begin{frame}{2. Why consider working time reduction policies in model? iii}
\begin{itemize}
\item \underline{Extensions to technology in \cite{Acemoglu2012TheChange} }
\begin{itemize}
\item externality of ``clean'' sector \citep[see also][]{Dasgupta2021, Brock2005ChapterEmpirics}
\begin{itemize}
	\item[-] renewable/ non-fossil fuels \ar externalities in production process are present e.g. production of solar panels toxic inputs \citep{Yue2014DomesticAnalysis}; non-fossil fuel nitrogen generation (e.g., biomass burning to clear land) important \citep{Song2021ImportantEmissions} 
	\item[-] waste (after use) \ar depends on recycling technology %\ar recycling system for solar panels not profitable enough today
%	\item[-] substitutability of nature in production (input sources eg. fossil vs. non-fossil fuels)
\end{itemize}
%\item Irreversibilities already before thresholds are hit (e.g. species extinction)

\end{itemize}
%\item greenhouse gases: Carbon dioxide $CO_2$ (vast majority), Nitrous oxide $N_2O$, methane $CH_4$
%\item stock of nature globally determined
\item \underline{parallel positive trend in demand} (population growth, rebound effect) that outperforms increase in clean technology growth \small{(not long run if perfectly clean technology exists)}
\item \normalsize{\underline{objective function}:} \cite{Arrow2004AreMuch}(Journal of Economic Perspectives) \ar using a sustainability measure they provide evidence that consumption is too high (not leaving enough natural resources for future generations)
\end{itemize}
\end{frame}


%\begin{frame}{Optimal policy versus working hour restrictions}
%	
%	\begin{itemize}
%	\item optimal policy: increase relative utility from leisure to correct for too high consumption and labour \ar distortionary labour tax or consumption tax
%	
%\end{itemize}
%\end{frame}

%\begin{frame}{Why a macro model?}
%\begin{itemize}
%	\item[-] proponents of a reduction approach suggest a \textbf{\alert{working time reduction}} \citep{Schor2005SustainableReduction, Pullinger2014WorkingDesign}; but do not account for directed technical change or inequality
%	\vspace{3mm}
%	\item however, there are interactions with directed technical change and inequality which might counteract the effect of working time reduction policies on the externality; social acceptance could also be questioned 
%	%\item[-] Would a government choose a working time reduction policy in presence of %recomposition is studied in models of \textbf{\alert{directed technical change}} which generally abstract from distortionary labour taxes and assumes a externality-free technology %\\ (nevertheless, a reduction in growth may be part of the optimal environmental policy \citep[e.g.][]{Acemoglu2012TheChange})
%	\vspace{2mm}
%	\alert{	\item[\ar] even if we assume that no clean technology exists it is unclear if a reduction in working time is indeed optimal once we account for (1) directed technical change or (2) heterogeneous skills (GE effects of WTR on environment)}
%
%	\item[\ar] construct model such that working time reduction might in principle become optimal, i.e., add environmental externality to clean sector %(in a Ramsey or a political economy setting)
%	\item What policy is optimal absent directed technical change and inequality?
%	\item How does the optimal policy change when accounting for (1) directed technical change and (2) skill heterogeneity?
%%	\item[\ar] both may counteract the effect of a working time reduction on the environment
%\end{itemize}
%\end{frame}


\begin{frame}{3. Why a macro model is necessary: Trade-offs}
	%\tr{Under construction}
%	Interactions between reduction and recomposition arise which have not been studied neither in the literature on directed technical change nor the literature on working time reduction
\begin{enumerate}[a)]
\item \textbf{Externality}
\begin{enumerate}[1.] 
\item<+->  direct effect of a working time reduction: lower labour supply and demand \ar lower output \ar diminishes externality
\item<+-> lower demand/ labour supply \ar reduces profits\ar slows down innovations and recomposition \ar detrimental effect on externality?
\item<+-> with heterogeneous skills: if high-skilled labour supply more responsive to distortionary tax \ar reduce their labour supply more \ar production costs in clean sector rise relatively more \ar demand shifts to dirty sector \ar unclear effect on direction of innovations 
(market-size(-) and price(+) effect)
	\end{enumerate}
	%\begin{itemize}
	%\item optimal environmental policy can entail reduction in consumption growth  \citep[e.g.,][]{Acemoglu2012TheChange} in a model with directed technical change
	%\item relying on directed technical change as a recomposition policy
	%\item despite assuming the existence of a technology which does not exert any externality
	%\item opponents argue: such technology is not available \citep{Dasgupta2021}; or that other forces counteract decoupling tendency (demand/rebound effect)
	%\item[\ar] even more substantial reductions in consumption growth might be required for environmental reasons (planetary boundary)
	%\item[\ar] de- or postgrowth literature proposes a working time reduction as  
	%\end{itemize}

\item<+->	\textbf{Inequality/ social acceptance}
	\begin{itemize}
\item[-]<+-> unproportional change in labour supply \ar relative wage changes
	\end{itemize}
\end{enumerate}
\end{frame}

\subsection{Model}
\begin{frame}{Model: \textbf{Households}}
\begin{itemize}
	\item unit mass of households, 2 types
	\item differ wrt  skills, indicated by $e\in\{l,h\}$, which they take as given 
	\item share of high-skilled households $\lambda$
	\item problem:
	
	\begin{align*}
	\max_{c_{st}, c_{nt}, h_{et}}U&=\sum_{t=0}^{\infty}\beta^t u(C_t,\alert{\pmb{h_{et}}}; S_t)\\
	s.t.& P_t C_t=\alert{\pmb{(1-\tau_l)}}w_{et}h_{et}+\pmb{\alert{T}}\\
	& C_t= \left(\omega c_{st}^{\frac{\sigma-1}{\sigma}}+(1-\omega)c_{nt}^\frac{\sigma-1}{\sigma}\right)^\frac{\sigma}{\sigma-1}\\
	&h_{et}\leq \bar{H_t}
	\end{align*}
\end{itemize}

\end{frame}

\begin{frame}{Model: Production
\tiny{(follows \cite{Acemoglu2012TheChange})}}
\begin{itemize}
	\item \textbf{final good sectors (perfect competition)}
	 \begin{align*}
	 &Y_{nt}=R_{nt}^{\alpha_{2}} L_{nt}^{1-\alpha_n}\int_{0}^{1}A_{nit}^{1-\alpha_n}x_{nit}^{\alpha_n} di; \ 
	 &Y_{st}=\alert{\pmb{R_{st}^{\alpha_{3}}}} L_{st}^{1-\alpha_s}\int_{0}^{1}A_{sit}^{1-\alpha_s}x_{sit}^{\alpha_s} di
	 \end{align*}
	\item \textbf{machine producing firms (monopolistic competition)}
	\begin{align*}
	\underset{x_{ijt}}{max}\  p_{ijt}x_{ijt}-\psi x_{ijt} \ j\in\{s,n\}
	\end{align*}

	\item \textbf{research}: scientiests decide each period where to allocate their research; better technology with success probability $\eta\in(0,1)$
\begin{align*}
A_{jit+1}=\left\{
\begin{array}{ll}
(1+\gamma)A_{jit}& \hspace{2mm} \text{if} \hspace*{2mm}\text{successful},\\
A_{jit}  &\hspace{2mm}\text{if}\hspace{2mm} \text{not \ successful}.
\end{array}
\right.
\end{align*}
\end{itemize}
\end{frame}

\begin{frame}{Model: Labour sector and market clearing}
	\begin{itemize}
\item \alert{\pmb{Labour input producing firm}} (perfect competition)
\begin{align*}
L_s= l_{hs}^{\varepsilon_s}l_{ls}^{1-\varepsilon_s}\\
L_n=l_{hn}^{\varepsilon_n}l_{ln}^{1-\varepsilon_n}
\end{align*}
assuming $\varepsilon_s>\varepsilon_n$ following \cite{Consoli2016DoCapital}
\vspace{3mm}
\item \textbf{Market clearing:}\\
{Labour markets} \hspace{12mm}
$ h_{ht}= l_{hst}+l_{hnt}; \  \hspace{4mm}
  h_{lt}= l_{lst}+l_{lnt}$
\\ {Goods markets}: \hspace{11mm} 
$Y_{st}=\lambda c_{hst}+(1-\lambda)c_{lst}; \ \hspace{4mm} Y_{nt}=\lambda c_{hnt}+(1-\lambda)c_{lnt}$
\\
Market for scientist: \hspace{4mm} $s\geq s_{dt}+s_{ct}$
	\end{itemize}
\end{frame}

\begin{frame}{Model: Environment}
%\textbf{Disaster, Irreversibility, externality, Carbon cycle?}
\begin{itemize}
\item quality of nature $S_t$ and irreversibility
\begin{align*}
S_{t}= -\xi Y_{nt}\pmb{\alert{-\kappa \xi Y_{st}}}+(1+\delta)S_{t-1} & \hspace{3mm} \text{if}\  S_{t}\in[0,\bar{S}]\\
S_{t+s}=0 \ \forall s>0& \hspace{3mm}  \text{if} \ S_{t}<0
\end{align*}
\item Environmental disaster: $S_t<0$
\begin{align*}
\underset{S\rightarrow0}{lim} u(C,h;S)=-\infty; 
\hspace{5mm} 
\underset{S\rightarrow0}{lim}\frac{\partial u(C,h;S)}{\partial S}=\infty
\end{align*}
\item stock of natural resources
\begin{align*}
Q_{t+1}=Q_t-R_{nt}\pmb{\alert{-R_{st}}}
\end{align*}
\end{itemize}
\end{frame}

\begin{frame}{Model: Government (very preliminary)}
	\begin{itemize}
\item 
\cite{Acemoglu2012TheChange}: subsidy on clean innovation, $\tau_i$, and corrective tax on dirty production, $\tau_n$
\item here:

\begin{enumerate}[a)]
	\item focus on externality \ar Ramsey planner with (in addition) labour tax, $\tau_l$, and/or corrective tax on cleaner good $\tau_s$
	\vspace{-3mm}
	\begin{align*}
	&\underset{\tau_l, \tau_n, \tau_s, \tau_i}{\max} \lambda U_h+(1-\lambda)U_l\\
	s.t.\ & feasibility\\
	& behaviour\  of\  agents 
	\end{align*}
	\vspace{-9mm}
	\pause
	\item focus on political acceptance
	\begin{enumerate}[i)]
		\item set maximum number of hours worked  $\bar{H_t}$ to match optimal leisure in Ramsey model\\ observability of types?
		\item Rawlsian social welfare function to focus on effect on the poor (Yellow Vest movement): $\max(\min(U_h, U_l))$
		\alert{\item notion of political feasibility?}
	\end{enumerate}
	
\end{enumerate}

	\end{itemize}
\end{frame}
%\begin{frame}{Interactions}
%\begin{itemize}
%\item on the one hand, proponents of a working time reduction may overlook general equilibrium effects on clean innovations
%\item on the other hand, literature on directed technical change generally assumes existence of externality-free technology
%\item allowing for working time reduction could shape the optimal policy in \cite{Acemoglu2012TheChange}: can reduce externality while
%	\end{itemize}
%\end{frame}

%\begin{frame}[allowframebreaks]{Motivation}
%\begin{itemize}
%\item macroeconomic research on environmental externality generally assumes the existence of a technology that does not exert any externality (using nature as a sink for used products (waste) or during the production process)
%\item[\ar] long-run growth might be sustainable \citep{Acemoglu2012TheChange}
%\item What if there is no such ``clean'' alternative but rather a gradual differentiation between technologies (dirty and less dirty); i.e., a lower bound on technical  innovations to reduce pollution in a way that the regenerative capacity of the planet is not  sufficient to sustain growth
%\item[\ar] growth might have to cease eventually
%\item politically delicate topic
%\item what are the effects on inequality?
%\end{itemize}
%\end{frame}


\begin{frame}{Plan B}
\begin{enumerate}
	\item extend paper on \textit{Redistribution, Demand, and Sustainable Production}
	\begin{itemize}
\item introduce monopolistic competition, heterogeneity in labour supply, and/or directed technical change in sustainability model
\item examine data on sustainable versus unsustainable consumption across income distribution and over time \\
\alert{\ar Data sources?}
	\end{itemize}
\pause
\item new paper on voluntary reduction in consumption (empirical and modelling)
\begin{itemize}
	\item there are studies on a voluntary reduction in consumption by individuals \citep{Alexander2012TheContext}; but limited in scope. 
	\item How important is this movement for the macroeconomy?
\item look at data on durable consumption; or data on labour supply; add survey information on environmental behaviour \ar evidence for voluntary reduction in labour supply?
\alert{\ar Any ideas on data sources?}
\item if there is this tendency: model a reduction in consumption to study its effects on inequality and the externality \small{(directed technical change, heterogeneous skills)}
\end{itemize}
\end{enumerate}
\end{frame}

\begin{frame}{Conclusion}
	\vspace{-7mm}
\centering{	\textbf{Plan A}}
	\vspace{-2mm}
\begin{itemize}
\item on the one hand, proponents of a working time reduction seem not to take interaction with recomposition approaches into account, directed technical change in particular
\item on the other hand, reduction policies generally not in policy set in endogenous growth models; yet, potentially inevitable when no perfectly clean technology exists
\item inequality hypothesised to shape interaction of reduction and recomposition policies
\vspace{2mm}
\item[\ar] What is the optimal policy?
\begin{itemize}
\item Is a working time reduction optimal when directed technical change and inequality are accounted for?
\item Is a restriction of working hours politically feasible when taking general equilibrium effects into account?
\end{itemize}
\end{itemize}
\vspace{-2mm}
\textbf{Plan B}
\end{frame}
\begin{frame}[allowframebreaks]{References}
	
	\bibliography{../../bib_2_0}
	\bibliographystyle{apa}
\end{frame}

%\appendix
%\begin{frame}{Why might restrictions on hours worked be chosen as policy?}
%	\begin{itemize}
%		\item \cite{Alvarez-Cuadrado2007EnvyHours}
%		The optimal tax policy in a setting with representative agent to restore the efficient allocation is to increase the value of leisure relative to consumption.
%		\item[\ar] a combination of distortionary labour taxation or consumption taxation
%		\item social acceptance of such policies questionable! Consumption becomes either more expensive or more of labour income has to be paid as tax
%		\item[\ar] a restriction on working hours (minimum of vacation days, restrictions on hours worked per week and worker (historic evidence)) might gain a broader social acceptance
%		\item[\ar] political economy approach to see working time reduction being chosen
%	\end{itemize}
%\end{frame}

\end{document}