\documentclass[11pt,aspectratio=169]{beamer}
%\usepackage[noxcolor]{beamerarticle} % to get presentation as article ! if used also set documentclass to article!
\usetheme[outer/progressbar=foot,
%outer/numbering=none
]{metropolis}
\setbeamertemplate{caption}{\raggedright\insertcaption\par}
\setbeamercolor{frametitle}{bg={}, fg=black!80}
\setbeamercolor{alerted text}{bg={}, fg=cyan!100}
\setbeamercolor{block title}{bg=black!10, fg=black}
\setbeamercolor{block body}{bg=black!10, fg=black}
%\usecolortheme{seahorse}
\usepackage[utf8]{inputenc}
\usepackage[english]{babel}
%\usepackage[T1]{fontenc}
\newcommand{\tr}[1]{\textcolor{blue}{#1}}
\usepackage{amsmath}
\usepackage{amsfonts}
\usepackage{amssymb}
\usepackage{mathtools}
\usepackage{calc}
\usepackage{soul}
\setbeamercolor{headerCol}{fg=blue!30,bg=black!80}
\setbeamercolor{bodyCol}{fg=black}
\usepackage{graphicx}
\usepackage{xcolor}
\usepackage{appendix}
\usepackage{hyperref}
\usepackage{natbib}
\usepackage{comment}
\usepackage{setspace}
\renewcommand{\bibsection}{}
\bibliographystyle{apa} 
% have to run bibtex mydocument.aux after first run to generate bbl file. 
\usepackage{appendixnumberbeamer}
\usepackage{xcolor}


\usepackage[customcolors]{hf-tikz}
\definecolor{sonja}{cmyk}{1.5,0,0.9,0.3}
%\definecolor{blue}{cmyk}{0,1,0,0}
\hfsetfillcolor{black!10}
\hfsetbordercolor{black}

\usepackage{tikz}
\usetikzlibrary{tikzmark}
\usetikzlibrary{decorations.markings}
\usepackage{tikz-cd}
\usetikzlibrary{arrows,calc,fit}
\tikzset{mainbox/.style={draw=white, text=white, fill=gray, rectangle, rounded corners, thick, node distance=7em, text width=8em, text centered, minimum height=3.5em}}
\tikzset{dummybox/.style={draw=none, text=white , rectangle, rounded corners, thick, node distance=7em, text width=8em, text centered, minimum height=3.5em}}
\tikzset{box/.style={draw , rectangle, rounded corners, thick, node distance=7em, text width=8em, text centered, minimum height=3.5em}}
\tikzset{container/.style={draw, rectangle, dashed, inner sep=2em}}
\tikzset{line/.style={draw, very thick, -latex'}}
\tikzset{    pil/.style={
		->,
		thick,
		shorten <=2pt,
		shorten >=2pt,}}
\tikzstyle{vecArrow} = [thick, decoration={markings,mark=at position
	1 with {\arrow[semithick]{open triangle 60}}},
double distance=1.4pt, shorten >= 5.5pt,
preaction = {decorate},
postaction = {draw,line width=1.4pt, white,shorten >= 4.5pt}]



%TITLE
\author[Sonja Dobkowitz]{\small Sonja Dobkowitz*}
\institute[University of Bonn]{University of Bonn}
\title{Growth, the Environment, and Inequality}

\newcommand{\ar}{$\Rightarrow$ \ }

%\addtobeamertemplate{navigation symbols}{}{%
%    \usebeamerfont{footline}%
%    \usebeamercolor[fg]{footline}%
%    \hspace{1em}%
%   \insertframenumber/\inserttotalframenumber
%}

\institute{*University of Bonn} 
\date{December 14, 2021} 
%\subject{} 
\begin{document}
	
	{\setbeamertemplate{footline}{}
		\begin{frame}
		\titlepage
	\end{frame}
}
%\addtocounter{framenumber}{-1}

% {\setbeamertemplate{footline}{}
% \begin{frame}{Content}
% \vspace{4mm}
% \tableofcontents
% \end{frame}
% }
 %\addtocounter{framenumber}{-1}


%---------------------------------------
%            Intro
%---------------------------------------

\begin{frame}[allowframebreaks]{Motivation}
\begin{itemize}
\item macroeconomic research on environmental externality generally assumes the existence of a technology that does not exert any externality (using nature as a sink for used products (waste) or during the production process)
\item[\ar] long-run growth might be sustainable \citep{Acemoglu2012TheChange}
\item What if there is no such ``clean'' alternative but rather a gradual differentiation between technologies (dirty and less dirty); i.e., a lower bound on technical  innovations to reduce pollution in a way that the regenerative capacity of the planet is not  sufficient to sustain growth
\item[\ar] growth will have to cease eventually
\item politically delicate topic
\item what are the effects on inequality?

\item macroeconomic research focuses on environment with an representative agent
\item political acceptability important for environmental policy (compare, e.g., Yellow Vest movement in France 2018)
\item especially important if optimal environmental policy slows down growth; conflicting with typical measures of political performance (employment, GDP growth)
\item NOT SURE if inequality has been studied in such a setting...
\end{itemize}
\end{frame}

\begin{frame}{Why introduce environmental externality to ``clean'' production}
scientific evidence that there are environmental costs related to what is generally considered ``\textit{clean}''
\end{frame}

\begin{frame}{related literature}
\begin{itemize}
\item economic papers on directed technical change
\item criticism of economic research on environment
\item 
\end{itemize}
\end{frame}

\begin{frame}{Model}
	content...
\end{frame}

\begin{frame}[shrink]{References}
	
	\bibliography{../../bib_2_0}
	\bibliographystyle{apa}
\end{frame}
\end{document}