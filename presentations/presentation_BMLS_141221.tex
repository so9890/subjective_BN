\documentclass[11pt,aspectratio=169]{beamer}
%\usepackage[noxcolor]{beamerarticle} % to get presentation as article ! if used also set documentclass to article!
\usetheme[outer/progressbar=foot,
%outer/numbering=none
]{metropolis}
\setbeamertemplate{caption}{\raggedright\insertcaption\par}
\setbeamercolor{frametitle}{bg={}, fg=black!80}
\setbeamercolor{alerted text}{bg={}, fg=cyan!100}
\setbeamercolor{block title}{bg=black!10, fg=black}
\setbeamercolor{block body}{bg=black!10, fg=black}
%\usecolortheme{seahorse}
\usepackage[utf8]{inputenc}
\usepackage[english]{babel}
%\usepackage[T1]{fontenc}
\newcommand{\tr}[1]{\textcolor{blue}{#1}}
\usepackage{amsmath}
\usepackage{amsfonts}
\usepackage{amssymb}
\usepackage{mathtools}
\usepackage{calc}
\usepackage{soul}
\setbeamercolor{headerCol}{fg=blue!30,bg=black!80}
\setbeamercolor{bodyCol}{fg=black}
\usepackage{graphicx}
\usepackage{xcolor}
\usepackage{appendix}
\usepackage{hyperref}
\usepackage{natbib}
\usepackage{comment}
\usepackage{setspace}
\renewcommand{\bibsection}{}
\bibliographystyle{apa} 
% have to run bibtex mydocument.aux after first run to generate bbl file. 
\usepackage{appendixnumberbeamer}
\usepackage{xcolor}


\usepackage[customcolors]{hf-tikz}
\definecolor{sonja}{cmyk}{1.5,0,0.9,0.3}
%\definecolor{blue}{cmyk}{0,1,0,0}
\hfsetfillcolor{black!10}
\hfsetbordercolor{black}

\usepackage{tikz}
\usetikzlibrary{tikzmark}
\usetikzlibrary{decorations.markings}
\usepackage{tikz-cd}
\usetikzlibrary{arrows,calc,fit}
\tikzset{mainbox/.style={draw=white, text=white, fill=gray, rectangle, rounded corners, thick, node distance=7em, text width=8em, text centered, minimum height=3.5em}}
\tikzset{dummybox/.style={draw=none, text=white , rectangle, rounded corners, thick, node distance=7em, text width=8em, text centered, minimum height=3.5em}}
\tikzset{box/.style={draw , rectangle, rounded corners, thick, node distance=7em, text width=8em, text centered, minimum height=3.5em}}
\tikzset{container/.style={draw, rectangle, dashed, inner sep=2em}}
\tikzset{line/.style={draw, very thick, -latex'}}
\tikzset{    pil/.style={
		->,
		thick,
		shorten <=2pt,
		shorten >=2pt,}}
\tikzstyle{vecArrow} = [thick, decoration={markings,mark=at position
	1 with {\arrow[semithick]{open triangle 60}}},
double distance=1.4pt, shorten >= 5.5pt,
preaction = {decorate},
postaction = {draw,line width=1.4pt, white,shorten >= 4.5pt}]



%TITLE
\author[Sonja Dobkowitz]{\small Sonja Dobkowitz*}
\institute[University of Bonn]{University of Bonn}
\title{Growth, the Environment, and Inequality}

\newcommand{\ar}{$\Rightarrow$ \ }

%\addtobeamertemplate{navigation symbols}{}{%
%    \usebeamerfont{footline}%
%    \usebeamercolor[fg]{footline}%
%    \hspace{1em}%
%   \insertframenumber/\inserttotalframenumber
%}

\institute{*University of Bonn} 
\date{December 14, 2021} 
%\subject{} 
\begin{document}
	
	{\setbeamertemplate{footline}{}
		\begin{frame}
		\titlepage
	\end{frame}
}
%\addtocounter{framenumber}{-1}

% {\setbeamertemplate{footline}{}
% \begin{frame}{Content}
% \vspace{4mm}
% \tableofcontents
% \end{frame}
% }
 %\addtocounter{framenumber}{-1}


%---------------------------------------
%            Intro
%---------------------------------------

\begin{frame}{Motivation}

\begin{itemize}
\item two (complementary) measures to reduce environmental externalities exist: \\ (1) a \alert{\textbf{recomposition}} and (2) a \textbf{\alert{reduction}} of production 
\begin{itemize}
\item recomposition is intrinsic to models of \textbf{\alert{directed technical change}}; yet, a reduction in growth may be part of the optimal environmental policy \citep[e.g.][]{Acemoglu2012TheChange}
\item proponents of a reduction approach suggest a \textbf{\alert{working time reduction}} \citep{Schor2005SustainableReduction, Pullinger2014WorkingDesign}
\item[\ar] \textbf{a reduction policy might be necessary for environmental reasons}
\end{itemize}

\item however,
\begin{itemize}
	\item it may counteract recomposition policies
	\item it is politically delicate (inequality)
\end{itemize}
\item[\ar] What is the optimal/politically feasible policy?
\end{itemize}
\end{frame}

\begin{frame}{Hypothesis}
\begin{itemize}
	\item 
\end{itemize}
\end{frame}
\begin{frame}{Interactions}
\begin{itemize}
\item on the one hand, proponents of a working time reduction may overlook general equilibrium effects on clean innovations
\item on the other hand, literature on directed technical change generally assumes existence of externality-free technology
\item allowing for working time reduction could shape the optimal policy in \cite{Acemoglu2012TheChange}: can reduce externality while
	\end{itemize}
\end{frame}

\begin{frame}{Mechanisms}
 in an unequal economy, interactions between reduction and recomposition arise which have not been studied neither in the literature on directed technical change nor the literature on working time reduction
\begin{itemize}
	\item reducing labour supply through distortionary labour income taxes implies stronger reduction in labour supply by rich/ high-skilled workers
	\item high-skilled labour share higher in clean sector CITE
	\item[\ar] effect on direction of innovations through market-size and  price effect
\end{itemize}
\begin{itemize}
\item models of directed technical change in general assume the existence of a technology which does not exert any environmental externality
\item questioned by proponents of reduction approach (post-/degrowth) literature
\end{itemize}
\begin{itemize}
\item optimal environmental policy can entail reduction in consumption growth  \citep[e.g.,][]{Acemoglu2012TheChange} in a model with directed technical change
\item relying on directed technical change as a recomposition policy
\item despite assuming the existence of a technology which does not exert any externality
\item opponents argue: such technology is not available \citep{Dasgupta2021}; or that other forces counteract decoupling tendency (demand/rebound effect)
\item[\ar] even more substantial reductions in consumption growth might be required for environmental reasons (planetary boundary)
\item[\ar] de- or postgrowth literature proposes a working time reduction as  
\end{itemize}
\end{frame}

\begin{frame}[allowframebreaks]{Motivation}
\begin{itemize}
\item macroeconomic research on environmental externality generally assumes the existence of a technology that does not exert any externality (using nature as a sink for used products (waste) or during the production process)
\item[\ar] long-run growth might be sustainable \citep{Acemoglu2012TheChange}
\item What if there is no such ``clean'' alternative but rather a gradual differentiation between technologies (dirty and less dirty); i.e., a lower bound on technical  innovations to reduce pollution in a way that the regenerative capacity of the planet is not  sufficient to sustain growth
\item[\ar] growth might have to cease eventually
\item politically delicate topic
\item what are the effects on inequality?

\item macroeconomic research on environmental externality focuses on representative agent models \citep{Golosov2014OptimalEquilibrium, Barrage2019OptimalPolicy, Acemoglu2012TheChange}
\item political acceptability important for environmental policy (compare, e.g., Yellow Vest movement in France 2018)
\item especially important if optimal environmental policy slows down growth; conflicting with typical measures of political performance (employment, GDP growth)
\item NOT SURE if inequality has been studied in such a setting...
\end{itemize}
\end{frame}

\begin{frame}{Why introduce environmental externality to ``clean'' production}
scientific evidence that there are environmental costs related to what is generally considered ``\textit{clean}''
\end{frame}

\begin{frame}{related literature}
\begin{itemize}
\item economic papers on directed technical change
\item criticism of economic research on environment
\item 
\end{itemize}
\end{frame}

\begin{frame}{Model}
	content...
\end{frame}

\begin{frame}{Plan B}
\begin{enumerate}
	\item extend paper on \textit{Redistribution, Demand, and Sustainable Production}
	\begin{itemize}
\item introduce directed technical change in sustainability model \ar interaction between 
\item examine data on sustainable versus unsustainable consumption across income distribution 
	\end{itemize}
\item new paper on voluntary reduction in consumption (empirical and modeling)
\begin{itemize}
\item look at data on durable consumption; data on labour supply plus survey \ar evidence for voluntary reduction?
\item if so: model a reduction in consumption to study its effects on inequality and the externality

\end{itemize}
\end{enumerate}
\end{frame}
\begin{frame}[shrink]{References}
	
	\bibliography{../../bib_2_0}
	\bibliographystyle{apa}
\end{frame}
\end{document}