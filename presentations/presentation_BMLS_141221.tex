\documentclass[11pt,aspectratio=169]{beamer}
%\usepackage[noxcolor]{beamerarticle} % to get presentation as article ! if used also set documentclass to article!
\usetheme[outer/progressbar=foot,
%outer/numbering=none
]{metropolis}
\setbeamertemplate{caption}{\raggedright\insertcaption\par}
\setbeamercolor{frametitle}{bg={}, fg=black!80}
\setbeamercolor{alerted text}{bg={}, fg=cyan!100}
\setbeamercolor{block title}{bg=black!10, fg=black}
\setbeamercolor{block body}{bg=black!10, fg=black}
%\usecolortheme{seahorse}
\usepackage[utf8]{inputenc}
\usepackage[english]{babel}
%\usepackage[T1]{fontenc}
\newcommand{\tr}[1]{\textcolor{blue}{#1}}
\usepackage{amsmath}
\usepackage{amsfonts}
\usepackage{amssymb}
\usepackage{mathtools}
\usepackage{calc}
\usepackage{soul}
\setbeamercolor{headerCol}{fg=blue!30,bg=black!80}
\setbeamercolor{bodyCol}{fg=black}
\usepackage{graphicx}
\usepackage{xcolor}
\usepackage{appendix}
\usepackage{hyperref}
\usepackage{natbib}
\usepackage{comment}
\usepackage{setspace}
\renewcommand{\bibsection}{}
\bibliographystyle{apa} 
% have to run bibtex mydocument.aux after first run to generate bbl file. 
\usepackage{appendixnumberbeamer}
\usepackage{xcolor}


\usepackage[customcolors]{hf-tikz}
\definecolor{sonja}{cmyk}{1.5,0,0.9,0.3}
%\definecolor{blue}{cmyk}{0,1,0,0}
\hfsetfillcolor{black!10}
\hfsetbordercolor{black}

\usepackage{tikz}
\usetikzlibrary{tikzmark}
\usetikzlibrary{decorations.markings}
\usepackage{tikz-cd}
\usetikzlibrary{arrows,calc,fit}
\tikzset{mainbox/.style={draw=white, text=white, fill=gray, rectangle, rounded corners, thick, node distance=7em, text width=8em, text centered, minimum height=3.5em}}
\tikzset{dummybox/.style={draw=none, text=white , rectangle, rounded corners, thick, node distance=7em, text width=8em, text centered, minimum height=3.5em}}
\tikzset{box/.style={draw , rectangle, rounded corners, thick, node distance=7em, text width=8em, text centered, minimum height=3.5em}}
\tikzset{container/.style={draw, rectangle, dashed, inner sep=2em}}
\tikzset{line/.style={draw, very thick, -latex'}}
\tikzset{    pil/.style={
		->,
		thick,
		shorten <=2pt,
		shorten >=2pt,}}
\tikzstyle{vecArrow} = [thick, decoration={markings,mark=at position
	1 with {\arrow[semithick]{open triangle 60}}},
double distance=1.4pt, shorten >= 5.5pt,
preaction = {decorate},
postaction = {draw,line width=1.4pt, white,shorten >= 4.5pt}]



%TITLE
\author[Sonja Dobkowitz]{\small Sonja Dobkowitz}
%\institute[University of Bonn]{University of Bonn}
\title{Growth, the Environment, and Inequality}

\newcommand{\ar}{$\Rightarrow$ \ }

%\addtobeamertemplate{navigation symbols}{}{%
%    \usebeamerfont{footline}%
%    \usebeamercolor[fg]{footline}%
%    \hspace{1em}%
%   \insertframenumber/\inserttotalframenumber
%}

\date{\footnotesize{University of Bonn}} 
\institute{December 14, 2021} 
%\subject{} 
\begin{document}
	
	{\setbeamertemplate{footline}{}
		\begin{frame}
		\titlepage
	\end{frame}
}
%\addtocounter{framenumber}{-1}

% {\setbeamertemplate{footline}{}
% \begin{frame}{Content}
% \vspace{4mm}
% \tableofcontents
% \end{frame}
% }
 %\addtocounter{framenumber}{-1}


%---------------------------------------
%            Intro
%---------------------------------------

\begin{frame}{Motivation}

\begin{itemize}
\item two measures to reduce environmental externalities prominent in literature: 
\begin{enumerate}
\item a \alert{\textbf{recomposition}} of production
\item a \textbf{\alert{reduction}} of production (especially a working time reduction)
\end{enumerate}
%(1) a \alert{\textbf{recomposition}} and (2) a \textbf{\alert{reduction}} of production through a working time reduction
\vspace{3mm} 
\item the two approaches might be complementary, related through general equilibrium effects, and shaped by \textbf{\alert{inequality}}
\vspace{3mm}
\begin{itemize}
\item[-] reduction policy might be inevitable (e.g., if no perfectly clean technology is available)
\item[-] but: a reduction in demand and labour supply could affect growth in clean innovations \ar ambiguous effect on externality
\item[-] even more so as rich households' labour supply is specific to the clean sector
\end{itemize}
\vspace{3mm}
\item however,  these policies have mainly been studied separately and absent inequality
\end{itemize}
\end{frame}
%
%\addtocounter{framenumber}{-1}
%\begin{frame}{Motivation}
%	
%	\begin{itemize}
%		\item the literature discusses two measures to reduce environmental externalities: \\ (1) a \alert{\textbf{recomposition}} and (2) a \textbf{\alert{reduction}} of production
%		\vspace{3mm} 
%		\item the two approaches might be complementary, related through general equilibrium effects, and shaped by \textbf{\alert{inequality}}
%		\vspace{3mm}
%		\begin{itemize}
%			\item[-] reduction policy might be inevitable if no perfectly clean technology is available
%			\item[-] but: a reduction in demand and labour supply could affect growth in clean innovations
%			\item[-] even more so as rich households' supply labour is specific to the clean sector
%		\end{itemize}
%		
%		\item however,  these policies have mainly been studied separately and absent inequality: \vspace{3mm}
%		\begin{itemize}
%			\item[-] proponents of a reduction approach suggest a \textbf{\alert{working time reduction}} \citep{Schor2005SustainableReduction, Pullinger2014WorkingDesign}; but do not account for directed technical change
%			\vspace{3mm}
%			\item[-] recomposition is studied in models of \textbf{\alert{directed technical change}} which generally abstract from distortionary labour taxes\\ (nevertheless, a reduction in growth may be part of the optimal environmental policy \citep[e.g.][]{Acemoglu2012TheChange})
%		\end{itemize}
%		
%	\end{itemize}
%\end{frame}

\begin{frame}
	\begin{block}{Therefore...}
I study environmental policies in a model of directed technical change with distortionary labour taxes and inequality. \\
\vspace{2mm}
\ar \textbf{What is the optimal policy?}
	\end{block}

\end{frame}

\begin{frame}{This Paper: Model}
	\begin{itemize}
\item model with \textbf{\alert{endougenous directed technical change}}  à la \cite{Acemoglu2012TheChange}
%\begin{itemize}
%\item[-] corrective tax, subsidies
%\item[-] clean and dirty sector
%\end{itemize}
\end{itemize}
I add 
\begin{itemize}
\item Inequality
\begin{itemize}
\item two household types: \textbf{\alert{high- and low-skilled}}
\item \alert{\textbf{sectors differ in the share of high-skilled labour }}
\end{itemize}
\item working time reduction to policy set
\begin{itemize}
\item households face a labour supply decision
\item government can levy distortionary labour tax 
\end{itemize}

\item %\tr{@Pavel. Potentially not sure about this now}
 \hyperlink{cleanSec}{\textbf{\alert{externality to clean sector}} }
\begin{itemize}
	\item clean technology also requires natural resources
	\item environmental externalities: in production and as waste
	\item[\ar] working time reduction might become necessary (depends on regeneration capacity of nature)
\end{itemize}
	\end{itemize}
\end{frame}

\begin{frame}{This Paper: Exercises}
	%\tr{@Pavel. not 100\% sure about this; but ideally I would like to integrate a political economy perspective, maybe we can talk about this. Bcs it seems a bit too much, maybe?}
	a) When is a working time reduction optimal in a model with directed technical change?
	\vspace{0mm}
	\begin{enumerate}
		\item establish conditions under which a working time reduction is optimal (analytically)
		\item in a representative agent model calculate optimal policy (Ramsey)
		\item plug this policy into a model with inequality: \\ 
		\ar How does inequality shape the effect of the policy on the externality?; How is inequality affected?
		\item How does the optimal policy change once inequality is added?
	\end{enumerate}
	b) Environmental policies, especially reduction policies, are \textbf{\alert{politically delicate}}
	%\begin{itemize}
	%	\item create winners and losers in an unequal society
	%	\item policies traditionally evaluated with a focus on employment and growth 
	%\end{itemize}
	\begin{enumerate}
		\item[4.] Political economy perspective to calculate ``optimal'' policy
	\end{enumerate}
\end{frame}

\begin{frame}{Outline suggestion}
\begin{enumerate}
\item present model
\item derive critical parameter values for which working time reduction becomes indeed optimal in model without inequality
\item add inequality, how do things change?
\item look at data to see what parameter values are plausible (the with or the without working time reduction cases)
\item simulation of transition! 
\end{enumerate}
\end{frame}

\section{Main Part}
\begin{frame}{Roadmap}
In this part, I want to talk about
\begin{itemize}
\item Related literature
\item how working time reductions are expected to work
\item The model in more depth and empirical motivations
%\item argue for why I think a reduction policy could be optimal 
\end{itemize}
\end{frame}

\begin{frame}[allowframebreaks]{Related literature}
\begin{itemize}
\item endogenous growth and the environment
\begin{enumerate}
\item  directed technical change
\begin{itemize}
	\item \cite{Acemoglu2012TheChange}(AER) \ar reduction of long-run growth part of optimal policy for some parameter values
	\item \cite{Acemoglu2016TransitionTechnology}(JPE) \ar 
	\item \cite{Eriksson2018PhasingChange}(Economic Modelling) \ar continued government intervention and 
\end{itemize}
\item  limits to growth \tr{@Sonja: optimal policies here?}
\begin{itemize}
	\item \cite{Stokey1998AreGrowth}(International Economic Review) \ar environmental constraints can imply a limit to endogenous growth (no distinction between knowledge and capital); policy instruments: direct regulation, voucher system, pollution tax; rep agent; no labour decision 
	\item \cite{Jones2016LifeGrowth}(JPE)\ar safety may be valued above consumption growth under standard preferences when life and death are taken into account \ar optimal consumption growth lower than what is feasible, could fall to zero; rep agent; optimal policy; transition
	\item \textbf{\cite{Brock2005ChapterEmpirics} (Handbook of Economic Growth)}: argue for non-existence of pollution-free technology. It follows that, when environmental boundaries have to be respected, there is a limit to growth
\end{itemize}
\end{enumerate}
\item working time reduction...
\begin{enumerate}
\item and the environment
\begin{itemize}
\item \cite{Schor2005SustainableReduction}(Journal of Industrial Ecology) \ar empirical study on slow down in hours reduction, firm incentives to keep hours per worker high (fixed costs)
\item \cite{Pullinger2014WorkingDesign}(Ecological Economics)\ar voluntary working time reduction possibilities in Netherlands/ Belgium; happiness from leisure to  offset disutility from consumption reduction; other measures: maximum working time
\item \cite{Cieplinski2021EnvironmentalReductionb}(Ecological Economics) \ar not clear about mechanisms
\end{itemize}
\item and macro (in case of overconsumption due to habits or social preferences)
\begin{itemize}
\item \cite{Alvarez-Cuadrado2007EnvyHours}(Canadian Journal of Economics) \ar envy; benevolent planner reduces consumption; \textbf{restrictions on working hours} close to optimal policy (subsidy on leisure, a general consumption tax) 
\end{itemize}
\end{enumerate}
\item political economy and limits to growth
\begin{itemize}
\item \cite{Alesina1994DistributiveGrowth}(QJE)
\end{itemize}
\item distortionary taxation and endogenous growth models: (a) paper by Bovenberg, deMooji, (b)...
\end{itemize}
What is missing: 
\begin{itemize}
\item inequality in directed technological growth model (Aghion paper on skill input)
\item working time reduction policy tools in endogenous growth model with environemntal boundaries
\end{itemize}
\end{frame}

\begin{frame}{Model}

\end{frame}

\begin{frame}[allowframebreaks]{Why a reduction policy might be optimal}
\begin{itemize}
	\item \cite{Acemoglu2012TheChange} show that a reduction of long-run growth is inevitable to avoid an environmental catastrophe in some cases
	\begin{itemize}
\item[-] when there is a trade-off between pollution-reducing and output-increasing innovations (p.146)
\item[-] when clean and dirty goods are weak substitutes (p.142; reduction in growth)  or complements (p. 144; no growth)
\item[-] when the initial environmental quality is low (p.142) \ar some planetary boundaries already passed today \citep{Rockstrom2009AHumanity}
\item[-] dependence on regeneration rate of the environment
	\end{itemize}
\item externality of ``clean'' sector
\begin{itemize}
\item[-] renewable/ nonfossil fuels \ar externalities in production process are present e.g. production of solar panels toxic inputs; non-fossil fuel nitrogen generation (biomass burning, to clear land, microbial N cycles) important \citep{Song2021ImportantEmissions} 
\item[-] waste (after use) \ar recycling system for solar panels 
\end{itemize}
\item parallel positive trend in demand (population growth, rebound effect)
\item \cite{Arrow2004AreMuch}(Journal of Economic Perspectives) \ar using a sustainability measure they provide evidence that consumption is too high
\item Irreversibility already before thresholds are hit (e.g. species extinction)
%\item greenhouse gases: Carbon dioxide $CO_2$ (vast majority), Nitrous oxide $N_2O$, methane $CH_4$
%\item stock of nature globally determined
\end{itemize}
\end{frame}

\begin{frame}{Working time reduction policies}
\begin{enumerate}
\item How done? \citep{Pullinger2014WorkingDesign}
\begin{itemize}
\item limits to max. number of working hours per week
\item minimum number of holidays per year
\item rights to reduce working hours
\item pre-retirement policies
\item[+] coupled with financial support; job protection
\item increased time rights for employees (rights to career breaks, to flexible and part time working hours)
\item fiscal support (credits, benefit restructuring, paid leave rights)
\item  facilitation to re-enter labour market (protection from future career losses)
\item financial instruments to decouple working time from income reception
\item \citep{Alvarez-Cuadrado2007EnvyHours}: tax on consumption, subsidy to leisure
\end{itemize}
\item Effects
\end{enumerate}
\end{frame}
\begin{frame}{Model}
\begin{itemize}
\item \textbf{Carbon Cycle}
\item \textbf{Waste}
\item \textbf{Planetary boundaries: Close to disastrous tipping point }
\end{itemize}
\end{frame}

\begin{frame}{Related Literature}
\begin{itemize}
	\item[-] proponents of a reduction approach suggest a \textbf{\alert{working time reduction}} \citep{Schor2005SustainableReduction, Pullinger2014WorkingDesign}; but do not account for directed technical change
	\vspace{3mm}
	\item[-] recomposition is studied in models of \textbf{\alert{directed technical change}} which generally abstract from distortionary labour taxes\\ (nevertheless, a reduction in growth may be part of the optimal environmental policy \citep[e.g.][]{Acemoglu2012TheChange})
\end{itemize}
\end{frame}
\begin{frame}{Interactions}
\begin{itemize}
\item on the one hand, proponents of a working time reduction may overlook general equilibrium effects on clean innovations
\item on the other hand, literature on directed technical change generally assumes existence of externality-free technology
\item allowing for working time reduction could shape the optimal policy in \cite{Acemoglu2012TheChange}: can reduce externality while
	\end{itemize}
\end{frame}

\begin{frame}{Mechanisms}
 in an unequal economy, interactions between reduction and recomposition arise which have not been studied neither in the literature on directed technical change nor the literature on working time reduction
\begin{itemize}
	\item reducing labour supply through distortionary labour income taxes implies stronger reduction in labour supply by rich/ high-skilled workers
	\item high-skilled labour share higher in clean sector CITE
	\item[\ar] effect on direction of innovations through market-size and  price effect
\end{itemize}
\begin{itemize}
\item models of directed technical change in general assume the existence of a technology which does not exert any environmental externality
\item questioned by proponents of reduction approach (post-/degrowth) literature
\end{itemize}
\begin{itemize}
\item optimal environmental policy can entail reduction in consumption growth  \citep[e.g.,][]{Acemoglu2012TheChange} in a model with directed technical change
\item relying on directed technical change as a recomposition policy
\item despite assuming the existence of a technology which does not exert any externality
\item opponents argue: such technology is not available \citep{Dasgupta2021}; or that other forces counteract decoupling tendency (demand/rebound effect)
\item[\ar] even more substantial reductions in consumption growth might be required for environmental reasons (planetary boundary)
\item[\ar] de- or postgrowth literature proposes a working time reduction as  
\end{itemize}
\end{frame}

\begin{frame}[allowframebreaks]{Motivation}
\begin{itemize}
\item macroeconomic research on environmental externality generally assumes the existence of a technology that does not exert any externality (using nature as a sink for used products (waste) or during the production process)
\item[\ar] long-run growth might be sustainable \citep{Acemoglu2012TheChange}
\item What if there is no such ``clean'' alternative but rather a gradual differentiation between technologies (dirty and less dirty); i.e., a lower bound on technical  innovations to reduce pollution in a way that the regenerative capacity of the planet is not  sufficient to sustain growth
\item[\ar] growth might have to cease eventually
\item politically delicate topic
\item what are the effects on inequality?

\item macroeconomic research on environmental externality focuses on representative agent models \citep{Golosov2014OptimalEquilibrium, Barrage2019OptimalPolicy, Acemoglu2012TheChange}
\item political acceptability important for environmental policy (compare, e.g., Yellow Vest movement in France 2018)
\item especially important if optimal environmental policy slows down growth; conflicting with typical measures of political performance (employment, GDP growth)
\item NOT SURE if inequality has been studied in such a setting...
\end{itemize}
\end{frame}

\begin{frame}{Why introduce environmental externality to ``clean'' production}
scientific evidence that there are environmental costs related to what is generally considered ``\textit{clean}''
\end{frame}

\begin{frame}{related literature}
\begin{itemize}
\item economic papers on directed technical change
\item criticism of economic research on environment
\item 
\end{itemize}
\end{frame}

\begin{frame}{Model}
	content...
\end{frame}

\begin{frame}{Plan B}
\begin{enumerate}
	\item extend paper on \textit{Redistribution, Demand, and Sustainable Production}
	\begin{itemize}
\item introduce directed technical change in sustainability model \ar interaction ?
\item examine data on sustainable versus unsustainable consumption across income distribution 
	\end{itemize}
\item new paper on voluntary reduction in consumption (empirical and modeling)
\begin{itemize}
	\item there are studies on a voluntary reduction in consumption by some households \citep{Alexander2012TheContext}. 
	\item How important is this movement for the macroeconomy?
\item look at data on durable consumption; or data on labour supply plus survey \ar evidence for voluntary reduction in labour supply?
\item if so: model a reduction in consumption to study its effects on inequality and the externality
\end{itemize}
\end{enumerate}
\end{frame}
\begin{frame}[allowframebreaks]{References}
	
	\bibliography{../../bib_2_0}
	\bibliographystyle{apa}
\end{frame}
\end{document}