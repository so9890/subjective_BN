\section*{Research Agenda}
%\begin{frame}
%	\normalsize
%	\vspace{5mm}
%	\centering
%	\begin{tikzpicture}[auto,scale=.7, transform shape]
%		% QUESTION
%		%	 	  \node[] (A) at (-10,9) {Question:\ \ \ \ \ \ \ \ \ };
%		\node[] (A) at (0,9) {\huge \textbf{\alert{How to transition to $\textit{green}$ economies?}} }; 
%		
%		%	  \node[] (A) at (-10,6) {Dimensions:\ \ \ \ \ \ \ \ \ };
%		
%		% Perspectives
%		\node[draw=none] (B) at (-7,7) {\textbf{{\hyperlink{prodmod}{}}}};
%		\node[draw=none] (C) at (-2.5,7) {\textbf{\hyperlink{backhh}{{}}}};
%		\node[draw=none] (D) at (2.5,7) {\textbf{\hyperlink{backhh}{{}}} {\textbf{}}};
%				\node[draw=none] (E) at (8,7) {\textbf{\hyperlink{backhh}{{}}} {\textbf{}}};
%		% Dimensions
%		
%		%	  \node[] (A) at (-10,3) {Projects: \ \ \ \ \ \ \ \ \ };	
%		\node[draw=none] (F) at (-6.6,4.5) {{\hyperlink{prodmod}{}}};
%		
%		\node[draw=none] (G) at (-3,-1) {};
%		\node[draw=none] (H) at (0.5,2.5) {};
%		\node[draw=none] (I) at (6.4,0) {};
%		
%		%\node[elli] (J) at (4.5,0) {{Worker's Concerns}\\ and Structural Transformation};
%		
%		\node[draw=none] (B1) at (5,4.25) {};
%		\node[draw=none] (B2) at (5,3.5) {};
%		\node[draw=none] (BA1) at (-5,4.25) {};
%		\node[draw=none] (BA2) at (-5,3.5) {};
%		\node[draw=none] (D1) at (1.8,7.8) {};
%		
%		\node[draw=none] (B22) at (6.4,5.6) {};
%		\node[draw=none] (D2) at (2.3,8.5) {};
%		\node[draw=none] (B3) at (5.3,5.3) {};
%		\node[draw=none] (D3) at (-1.5,7) {};
%		\node[draw=none] (A1) at (-4.2,4.6) {};
%		\node[draw=none] (D4) at (-2,8.3) {};
%		\node[draw=none] (A4) at (-6,5.4) {};
%		%	 	\draw [->] (B22) to node[pos=0.5, swap]{{Tax on labor, $\pmb{\tau_{\iota}}$}} (D2);
%		%	 	\draw [->] (D1) to node[pos=0.5, swap]{Transfers} (B3);
%		%	 	
%		%			\draw [->] (B1) to node[pos=0.75, swap]{Workers and scientists} (BA1);
%		%			\draw [->] (BA2) to node[pos=0.75, swap]{Final good} (B2);
%		%	 	\draw [->] (A4) to node[pos=0.5]{{Tax on carbon, $\pmb{\tau_F}$}}   (D4);
%	\end{tikzpicture}
%	
%\end{frame}
%\addtocounter{framenumber}{-1}
%\begin{frame}
%	\vspace{5mm}
%	\centering
%	\begin{tikzpicture}[auto,scale=.7, transform shape]
%		% QUESTION
%		%	 	  \node[] (A) at (-10,9) {Question:\ \ \ \ \ \ \ \ \ };
%		\node[] (A) at (0,9) {\huge \textbf{\alert{How to transition to $\textit{green}$ economies?}} }; 
%		
%		%	  \node[] (A) at (-10,6) {Dimensions:\ \ \ \ \ \ \ \ \ };
%		
%		% Perspectives
%		\pause
%		\node[modus] (B) at (-7,7) {\textbf{{\hyperlink{prodmod}{Public Finance}}}};
%
%		% Dimensions
%		\node[draw=none] (C) at (-2.5,7) {\textbf{\hyperlink{backhh}{{}}}};
%		\node[draw=none] (D) at (2.5,7) {\textbf{\hyperlink{backhh}{{}}} {\textbf{}}};
%		\node[draw=none] (E) at (8,7) {\textbf{\hyperlink{backhh}{{}}} {\textbf{}}};
%		
%		%	  \node[] (A) at (-10,3) {Projects: \ \ \ \ \ \ \ \ \ };	
%
%\node[draw=none] (F) at (-6.6,4.5) {{\hyperlink{prodmod}{}}};
%
%\node[draw=none] (G) at (-3,-1) {};
%\node[draw=none] (H) at (0.5,2.5) {};
%\node[draw=none] (I) at (6.4,0) {};
%		
%	\end{tikzpicture}
%	
%\end{frame}

%\addtocounter{framenumber}{-1}
%\begin{frame}
%	\vspace{5mm}
%	\centering
%	\begin{tikzpicture}[auto,scale=.7, transform shape]
%		% QUESTION
%		%	 	  \node[] (A) at (-10,9) {Question:\ \ \ \ \ \ \ \ \ };
%		\node[] (A) at (0,9) {\huge \textbf{\alert{How to transition to $\textit{green}$ economies?}} }; 
%		
%		%	  \node[] (A) at (-10,6) {Dimensions:\ \ \ \ \ \ \ \ \ };
%		
%		% Perspectives
%		\pause
%		\node[modus] (B) at (-7,7) {\textbf{{\hyperlink{prodmod}{Public Finance}}}};
%		\pause
%		\node[modus] (E) at (8,7) {\textbf{\hyperlink{backhh}{{Political}}} {\textbf{Economy}}};
%		% Dimensions
%		\node[draw=none] (C) at (-2.5,7) {};
%		\node[draw=none] (D) at (2.5,7) {};
%		
%		
%	
%	\node[draw=none] (F) at (-6.6,4.5) {{\hyperlink{prodmod}{}}};
%	
%	\node[draw=none] (G) at (-3,-1) {};
%	\node[draw=none] (H) at (0.5,2.5) {};
%	\node[draw=none] (I) at (6.4,0) {};
%	
%	\end{tikzpicture}
%\end{frame}
%
%\addtocounter{framenumber}{-1}
%\begin{frame}
%	\vspace{5mm}
%	\centering
%	\begin{tikzpicture}[auto,scale=.7, transform shape]
%		% QUESTION
%		%	 	  \node[] (A) at (-10,9) {Question:\ \ \ \ \ \ \ \ \ };
%		\node[] (A) at (0,9) {\huge \textbf{\alert{How to transition to $\textit{green}$ economies?}} }; 
%		
%		%	  \node[] (A) at (-10,6) {Dimensions:\ \ \ \ \ \ \ \ \ };
%		
%		% Perspectives
%		\node[modus] (B) at (-7,7) {\textbf{{\hyperlink{prodmod}{Public Finance}}}};
%		\node[modus] (E) at (8,7) {\textbf{\hyperlink{backhh}{{Political}}} {\textbf{Economy}}};
%		% Dimensions
%		\node[modus] (C) at (-2.5,7) {\textbf{\hyperlink{backhh}{{Inequality}}}};
%		\node[modus] (D) at (2.5,7) {\textbf{\hyperlink{backhh}{{Behavioral}}} {\textbf{Changes}}};
%		
%		
%		%	  \node[] (A) at (-10,3) {Projects: \ \ \ \ \ \ \ \ \ };	
%		\node[draw=none] (F) at (-6.6,4.5) {};
%		\node[draw=none] (G) at (-3,-1) {};
%		\node[draw=none] (H) at (0.5,2.5) {};
%		\node[draw=none] (I) at (6.4,0) {};
%	\end{tikzpicture}
%\end{frame}
%
%
%%\addtocounter{framenumber}{-1}
%\begin{frame}
%	\vspace{5mm}
%	\centering
%	\begin{tikzpicture}[auto,scale=.7, transform shape]
%		% QUESTION
%		%	 	  \node[] (A) at (-10,9) {Question:\ \ \ \ \ \ \ \ \ };
%		\node[] (A) at (0,9) {\huge \textbf{\alert{How to transition to $\textit{green}$ economies?}} }; 
%		\pause
%		%	  \node[] (A) at (-10,6) {Dimensions:\ \ \ \ \ \ \ \ \ };
%		
%		% Perspectives
%		\pause
%		\node[modus] (B) at (-7,7) {\textbf{{\hyperlink{prodmod}{Public Finance}}}};
%		\pause
%		\node[modus] (E) at (8,7) {\textbf{\hyperlink{backhh}{{Political}}} {\textbf{Economy}}};
%		% Dimensions
%		\pause
%		\node[modus] (C) at (-2.5,7) {\textbf{\hyperlink{backhh}{{Inequality}}}};
%		\node[modus] (D) at (2.5,7) {\textbf{\hyperlink{backhh}{{Behavioral}}} {\textbf{Changes}}};
%		
%		\pause
%		%	  \node[] (A) at (-10,3) {Projects: \ \ \ \ \ \ \ \ \ };	
%		\node[elli] (F) at (-6.6,4.5) {{\hyperlink{prodmod}{Meeting Climate Targets:}}\\ {The Optimal}\\ {Fiscal Policy Mix}};
%		\node[draw=none] (G) at (-3,-1) {};
%		\node[draw=none] (H) at (0.5,2.5) {};
%		\node[draw=none] (I) at (6.4,0) {};
%		
%	\end{tikzpicture}
%	
%\end{frame}
%
%\addtocounter{framenumber}{-1}
\begin{frame}
	\vspace{5mm}
	\centering
	\begin{tikzpicture}[auto,scale=.7, transform shape]
		% QUESTION
		%	 	  \node[] (A) at (-10,9) {Question:\ \ \ \ \ \ \ \ \ };
		\node[] (A) at (0,9) {\huge \textbf{\alert{How to transition to $\textit{green}$ economies?}} }; 
		
		%	  \node[] (A) at (-10,6) {Dimensions:\ \ \ \ \ \ \ \ \ };
		\pause
		% Perspectives
		\node[modus] (B) at (-7,7) {\textbf{{\hyperlink{prodmod}{Public Finance}}}};
		\pause
		\node[modus] (E) at (8,7) {\textbf{\hyperlink{backhh}{{Political}}} {\textbf{Economy}}};
		% Dimensions
		\pause
		\node[modus] (C) at (-2.5,7) {\textbf{\hyperlink{backhh}{{Inequality}}}};
		\node[modus] (D) at (2.5,7) {\textbf{\hyperlink{backhh}{{Behavioral}}} {\textbf{Changes}}};
		
		
		%	  \node[] (A) at (-10,3) {Projects: \ \ \ \ \ \ \ \ \ };
		\pause	
		\node[elli] (F) at (-6.6,4.5) {{\hyperlink{prodmod}{Meeting Climate Targets:}}\\ {The Optimal}\\ {Fiscal Policy Mix}};
		\pause
		\node[elli] (G) at (-3,-1) {\alert{Labor Income Taxes,} \\ \alert{Social Responsibility, and Inequality}};
		\node[draw=none] (H) at (0.5,2.5) {};
		\node[draw=none] (I) at (6.4,0) {};
		
	\end{tikzpicture}
\end{frame}


\begin{frame}{Labor Income Taxes, Social Responsibility, and Inequality}
	\vspace{-26mm}
\footnotesize
	\alert{\textbf{How should the government react as the willingness to spend to avoid negative externalities increases?}}
	\pause
	\begin{itemize}[<+->]
	%	\setbeamercolor{alerted text}{} %change the font color
%		\setbeamerfont{alerted text}{}
		\item \textbf{Motivation}
		{\footnotesize
		\begin{itemize}			
			\item[-] Trivial: necessity of government intervention to cope with an externality diminishes
			\item[-] But: income inequality poses an obstacle if poor households cannot afford to subsist on \textit{sustainable} goods alone. A rise in \textit{social responsibility} aggravates inequality
			\item[-] Then again: redistribution emerges as an environmental policy instrument
		\end{itemize}}
	\end{itemize}
\end{frame}

\begin{frame}{US}
\vspace{-2mm}
\begin{minipage}[]{1\textwidth}
	\begin{figure}
		%		%	\caption{Distribution of per-capita disposable income in 2018 }\label{fig:poverty}	
		%			
		%			%	\captionsetup{width=.45\linewidth}
		\includegraphics[width=.7\textwidth]{../../cc_new_policy/codding_new/calibration/emp_results/Poster_histogramme_prices_estimatedpdf_PSID.png}	
	\end{figure}
	\centering
	\tiny{Sources: Disposable Income: PSID, TAXSIM; Basic Needs: Institute for Women's Policy Research, Prices: USDA}
	%		
\end{minipage}\end{frame}


\begin{frame}{Labor Income Taxes, Social Responsibility, and Inequality}
	
	\begin{itemize}[<+->]
		\item \textbf{This paper}
		\begin{itemize}
			\item[-] Quantitative model with inequality and {sustainable} and {non-sustainable} goods
			\item[-] Households seek to consume according to their desire for sustainable goods and to satisfy a minimum consumption level
			\item[-] Government maximizes welfare using corrective and labor income taxes
			\item[-] Exercise: exogenously change the desire to consume {sustainable} goods
		\end{itemize}
		\item \textbf{Results}	
		\begin{itemize}
			\item[-] Irrespective of social responsibility, labor income taxes are optimally used to target the externality due to inequality
			\item[-] As social responsibility rises, the optimal policy shifts away from corrective taxation to redistribution since inequality aggravates, and...
			\item[-] ... the government redistributes even more to target the externality
		\end{itemize}
				
		\end{itemize}
	\end{frame}

%\begin{frame}
%	\vspace{5mm}
%	\centering
%	\begin{tikzpicture}[auto,scale=.7, transform shape]
%		% QUESTION
%		%	 	  \node[] (A) at (-10,9) {Question:\ \ \ \ \ \ \ \ \ };
%		\node[] (A) at (0,9) {\huge \textbf{\alert{How to transition to $\textit{green}$ economies?}} }; 
%		
%		%	  \node[] (A) at (-10,6) {Dimensions:\ \ \ \ \ \ \ \ \ };
%		
%		% Perspectives
%		\node[modus] (B) at (-7,7) {\textbf{{\hyperlink{prodmod}{Public Finance}}}};
%		\node[modus] (E) at (8,7) {\textbf{\hyperlink{backhh}{{Political}}} {\textbf{Economy}}};
%		% Dimensions
%		\node[modus] (C) at (-2.5,7) {\textbf{\hyperlink{backhh}{{Inequality}}}};
%		\node[modus] (D) at (2.5,7) {\textbf{\hyperlink{backhh}{{Behavioral}}} {\textbf{Changes}}};
%		
%		
%		%	  \node[] (A) at (-10,3) {Projects: \ \ \ \ \ \ \ \ \ };	
%		\node[elli] (F) at (-6.6,4.5) {{\hyperlink{prodmod}{Meeting Climate Targets:}}\\ {The Optimal}\\ {Fiscal Policy Mix}};
%		\node[elli] (G) at (-3,-1) {{Labor Income Taxes,} \\ {Social Responsibility, and Inequality}};
%		\node[draw=none] (H) at (0.5,2.5) {};
%		\node[draw=none] (I) at (6.4,0) {};
%		
%	\end{tikzpicture}
%\end{frame}
%
%\addtocounter{framenumber}{-1}
\begin{frame}
	\vspace{5mm}
	\centering
	\begin{tikzpicture}[auto,scale=.7, transform shape]
		% QUESTION
		%	 	  \node[] (A) at (-10,9) {Question:\ \ \ \ \ \ \ \ \ };
		\node[] (A) at (0,9) {\huge \textbf{\alert{How to transition to $\textit{green}$ economies?}} }; 
		
		%	  \node[] (A) at (-10,6) {Dimensions:\ \ \ \ \ \ \ \ \ };
		
		% Perspectives
		\node[modus] (B) at (-7,7) {\textbf{{\hyperlink{prodmod}{Public Finance}}}};
		\node[modus] (E) at (8,7) {\textbf{\hyperlink{backhh}{{Political}}} {\textbf{Economy}}};
		% Dimensions
		\node[modus] (C) at (-2.5,7) {\textbf{\hyperlink{backhh}{{Inequality}}}};
		\node[modus] (D) at (2.5,7) {\textbf{\hyperlink{backhh}{{Behavioral}}} {\textbf{Changes}}};
		
		
		%	  \node[] (A) at (-10,3) {Projects: \ \ \ \ \ \ \ \ \ };	
		\node[elli] (F) at (-6.6,4.5) {{\hyperlink{prodmod}{Meeting Climate Targets:}}\\ {The Optimal}\\ {Fiscal Policy Mix}};
		\node[elli] (G) at (-3,-1) {{Labor Income Taxes,} \\ {Social Responsibility, and Inequality}};
		\pause
		\node[draw=none] (H) at (0.5,2.5) {};
		\node[elli] (I) at (6.4,0) {\alert{Green Demand,}\\ \alert{Clean Innovation, and Lobbying}};
		
	\end{tikzpicture}
\end{frame}


\begin{frame}{Green Demand, Clean Innovation, and Lobbying}
	\vspace{-5mm}
\hspace{-6mm}\footnotesize{joint with Olimpia Cutinelli-Rendina and Antoine Mayerowitz (PSE and Collège de France)}\\
\pause

\footnotesize
\vspace{2mm}
\textbf{\alert{How do firms react as households demand more green goods? Do they innovate cleaner or do they increase spending on anti-environmental lobbying?}}
\pause
\vspace{-1.5mm}
\begin{itemize}[<+->]
 \item \textbf{Motivation }
 \begin{itemize}
 	\item[-] Households may demand more green goods to avoid negative externalities on the environment \footnotesize{\citep{Benabou2010IndividualResponsibility, Bartling2015DoResponsibility}}
 	\item[-] One possible reaction of firms is to innovate cleaner technologies \footnotesize{\citep{Aghion2021EnvironmentalDirty}}
 	\item[-] But: anti-environmental lobbying may be an alternative response to protect profits
 \end{itemize}
\item \textbf{This paper}
\begin{itemize}
	\item[-] Empirical approach using US data on state level combining multiple datasets:
	\begin{itemize}
		\item[-] natural catastrophes to instrument shifts in green preferences
		\item[-] google trends data to proxy green preferences
		\item[-] sales data of the automobile industry \ar firms' exposure to green preferences
		\item[-] information on patents and lobbying expenses on firm level
	\end{itemize}
\item[-] Macro model to relate firm responses to environmental outcome considering general equilibrium effects % eg how clean innovation feeds through to demand by lowering prices \ar more green demand
\end{itemize}
\end{itemize}
\end{frame}

%\begin{frame}
%	\vspace{5mm}
%	\centering
%	\begin{tikzpicture}[auto,scale=.7, transform shape]
%		% QUESTION
%		%	 	  \node[] (A) at (-10,9) {Question:\ \ \ \ \ \ \ \ \ };
%		\node[] (A) at (0,9) {\huge \textbf{\alert{How to transition to $\textit{green}$ economies?}} }; 
%		
%		%	  \node[] (A) at (-10,6) {Dimensions:\ \ \ \ \ \ \ \ \ };
%		
%		% Perspectives
%		\node[modus] (B) at (-7,7) {\textbf{{\hyperlink{prodmod}{Public Finance}}}};
%		\node[modus] (E) at (8,7) {\textbf{\hyperlink{backhh}{{Political}}} {\textbf{Economy}}};
%		% Dimensions
%		\node[modus] (C) at (-2.5,7) {\textbf{\hyperlink{backhh}{{Inequality}}}};
%		\node[modus] (D) at (2.5,7) {\textbf{\hyperlink{backhh}{{Behavioral}}} {\textbf{Changes}}};
%		
%		
%		%	  \node[] (A) at (-10,3) {Projects: \ \ \ \ \ \ \ \ \ };	
%		\node[elli] (F) at (-6.6,4.5) {{\hyperlink{prodmod}{Meeting Climate Targets:}}\\ {The Optimal}\\ {Fiscal Policy Mix}};
%		\node[elli] (G) at (-3,-1) {{Labor Income Taxes,} \\ {Social Responsibility, and Inequality}};
%		\node[draw=none] (H) at (0.5,2.5) {};
%		\node[elli] (I) at (6.4,0) {{Green Demand,}\\ {Clean Innovation, and Lobbying}};
%		
%	\end{tikzpicture}
%\end{frame}
%
%\addtocounter{framenumber}{-1}
\begin{comment}
\begin{frame}
	\vspace{5mm}
	\centering
	\begin{tikzpicture}[auto,scale=.7, transform shape]
		% QUESTION
		%	 	  \node[] (A) at (-10,9) {Question:\ \ \ \ \ \ \ \ \ };
		\node[] (A) at (0,9) {\huge \textbf{\alert{How to transition to $\textit{green}$ economies?}} }; 
		
		%	  \node[] (A) at (-10,6) {Dimensions:\ \ \ \ \ \ \ \ \ };
		
		% Perspectives
		\node[modus] (B) at (-7,7) {\textbf{{\hyperlink{prodmod}{Public Finance}}}};
		\node[modus] (E) at (8,7) {\textbf{\hyperlink{backhh}{{Political}}} {\textbf{Economy}}};
		% Dimensions
		\node[modus] (C) at (-2.5,7) {\textbf{\hyperlink{backhh}{{Inequality}}}};
		\node[modus] (D) at (2.5,7) {\textbf{\hyperlink{backhh}{{Behavioral}}} {\textbf{Changes}}};
		
		
		%	  \node[] (A) at (-10,3) {Projects: \ \ \ \ \ \ \ \ \ };	
		\node[elli] (F) at (-6.6,4.5) {{\hyperlink{prodmod}{Meeting Climate Targets:}}\\ {The Optimal}\\ {Fiscal Policy Mix}};
		\node[elli] (G) at (-3,-1) {{Labor Income Taxes,} \\ {Social Responsibility, and Inequality}};
		\node[elli] (I) at (6.4,0) {{Green Demand,}\\ Clean Innovation, and Lobbying};
		\pause
		%\node[elli] (H) at (0.5,2.5) {\alert{Voluntary  Reduction, Green Skills,}\\ \alert{and the Environment}};
		
	\end{tikzpicture}
\end{frame}

\begin{frame}{Voluntary  Reduction, Green Skills, and the Environment}
		\vspace{-5mm}
	\hspace{-6mm}\footnotesize{joint with Moritz Mendel (University of Bonn)}\\
		%	Using a representative Dutch panel, we provide evidence for a voluntary reduction in household consumption. 	What are the effects of such a change in consumption patterns on the externality? 	We document that households which report a willingness to change their lifestyle have a higher likelihood to work in the green sector. 	A tractable, demand-determined model with satiated preferences is employed to answer the research question. We study the effect of an exogenous reduction in the satiation point analytically. 	While the overall reduction in consumption is beneficial to the environment, the decline in green-skilled labour supply raises the market share of the dirty sector.
	
	
	\vspace{3mm}\textbf{\alert{How important is \textit{voluntary reduction} of consumption/working time by households? What are its macroeconomic effects? }}
	\pause
\begin{itemize}[<+->]
\item \textbf{Motivation}
\begin{itemize}
\item[-] Phenomenon of \textit{voluntary simplicity/ reduction} gained in importance  \citep{Reboucas2021VoluntaryAgenda}%has been discussed
\item[-] We are lacking an understanding of its effect on green transitions
\item[-] On the one hand: lower levels of consumption reduce externalities. On the other hand: if it is especially high-skilled workers that reduce, their skills might be missing in a green transition \citep{Consoli2016DoCapital}
\end{itemize}
\item \textbf{This paper}
\begin{itemize}
	\item[-] Representative Dutch panel dataset to study the phenomenon of voluntary reduction and household characteristics
	\item[-] Use empirical findings to inform a model of the macro economy to investigate effects of voluntary reduction and optimal policy responses
\end{itemize}
\end{itemize}
\end{frame}

content...
\end{comment}
%\begin{frame}
%	\vspace{5mm}
%	\centering
%	\begin{tikzpicture}[auto,scale=.7, transform shape]
%		% QUESTION
%		%	 	  \node[] (A) at (-10,9) {Question:\ \ \ \ \ \ \ \ \ };
%		\node[] (A) at (0,9) {\huge \textbf{\alert{How to transition to $\textit{green}$ economies?}} }; 
%		
%		%	  \node[] (A) at (-10,6) {Dimensions:\ \ \ \ \ \ \ \ \ };
%		
%		% Perspectives
%		\node[modus] (B) at (-7,7) {\textbf{{\hyperlink{prodmod}{Public Finance}}}};
%		\node[modus] (E) at (8,7) {\textbf{\hyperlink{backhh}{{Political}}} {\textbf{Economy}}};
%		% Dimensions
%		\node[modus] (C) at (-2.5,7) {\textbf{\hyperlink{backhh}{{Inequality}}}};
%		\node[modus] (D) at (2.5,7) {\textbf{\hyperlink{backhh}{{Behavioral}}} {\textbf{Changes}}};
%		
%		
%		%	  \node[] (A) at (-10,3) {Projects: \ \ \ \ \ \ \ \ \ };	
%		\node[elli] (F) at (-6.6,4.5) {{\hyperlink{prodmod}{Meeting Climate Targets:}}\\ {The Optimal}\\ {Fiscal Policy Mix}};
%		\node[elli] (G) at (-3,-1) {{Labor Income Taxes,} \\ {Social Responsibility, and Inequality}};
%		\node[elli] (H) at (0.5,2.5) {{Voluntary  Reduction, Green Skills,}\\ {and the Environment}};
%		\node[elli] (I) at (6.4,0) {{Green Demand,}\\ Clean Innovation, and Lobbying};
%		
%	\end{tikzpicture}
%\end{frame}
%
%\addtocounter{framenumber}{-1}
\begin{frame}
	\vspace{5mm}
	\centering
	\begin{tikzpicture}[auto,scale=.7, transform shape]
		% QUESTION
		%	 	  \node[] (A) at (-10,9) {Question:\ \ \ \ \ \ \ \ \ };
		\node[] (A) at (0,9) {\huge \textbf{\alert{How to transition to $\textit{green}$ economies?}} }; 
		
		%	  \node[] (A) at (-10,6) {Dimensions:\ \ \ \ \ \ \ \ \ };
		
		% Perspectives
		\node[modus] (B) at (-7,7) {\textbf{{\hyperlink{prodmod}{Public Finance}}}};
		\node[modus] (E) at (8,7) {\textbf{\hyperlink{backhh}{{Political}}} {\textbf{Economy}}};
		% Dimensions
		\node[modus] (C) at (-2.5,7) {\textbf{\hyperlink{backhh}{{Inequality}}}};
		\node[modus] (D) at (2.5,7) {\textbf{\hyperlink{backhh}{{Behavioral}}} {\textbf{Changes}}};
		
		
		%	  \node[] (A) at (-10,3) {Projects: \ \ \ \ \ \ \ \ \ };	
		\node[elli] (F) at (-6.6,4.5) {{\hyperlink{prodmod}{Meeting Climate Targets:}}\\ {The Optimal}\\ {Fiscal Policy Mix}};
		\node[elli] (G) at (-3,-1) {{Labor Income Taxes,} \\ {Social Responsibility, and Inequality}};
		%\node[elli] (H) at (0.5,2.5) {{Voluntary  Reduction, Green Skills,}\\ {and the Environment}};
		\node[elli] (I) at (6.4,0) {{Green Demand,}\\ Clean Innovation, and Lobbying};
		\pause
		\node[elli] (I) at (4.5,4.5) {\alert{Workers' Concerns}\\ \alert{and Structural Transformation}};
		
	\end{tikzpicture}
\end{frame}

\begin{frame}{Workers' Concerns and Structural Transformation}
	\vspace{-5mm}
	\hspace{-6mm}\footnotesize{joint with Ximeng Fang (University of Oxford)}\\
	\pause
	
	\vspace{2mm}\textbf{\alert{How to politically facilitate structural transformation?}}
	\vspace{-1.6mm}
	\pause
	\begin{itemize}[<+->]
		\item \textbf{Motivation}
		\begin{itemize}
			\item[-] We are facing broad structural transformations of the economy (environment/ automation)
			\item[-] Structural transformation creates losers (at least in the short run)
			\item[-] Political opposition pushing political leaders to prevent structural transformation \\(US opt out of Paris Agreement 2017)
		%	\item[-] What political measures could be used to make affected workers support structural transformation/ climate policies? 
		%\item[-] Scarce literature on attitudes towards and heterogeneous effects of structural transformation
		\end{itemize}
		\item \textbf{This paper}
		\begin{itemize}
			\item[-] Compile historic data set of structural transformation \ar learn about effects on persons concerned %(a posteriori)
			\item[-] Survey affected individuals: What are their  beliefs about the future? What do they wish from the government in order to support climate change policies? %(a priori)
			\item[-] Behavioral experiments building on insights from survey and historic data set
			\item[-] Macro model to study effect of policy recommendations from behavioral experiment
		\end{itemize}
	\end{itemize}
\end{frame}

\begin{comment}
	
	
	
\begin{frame}
	\vspace{5mm}
	\centering
	\begin{tikzpicture}[auto,scale=.7, transform shape]
		% QUESTION
		%	 	  \node[] (A) at (-10,9) {Question:\ \ \ \ \ \ \ \ \ };
		\node[] (A) at (0,9) {\huge \textbf{\alert{How to transition to $\textit{green}$ economies?}} }; 
		
		%	  \node[] (A) at (-10,6) {Dimensions:\ \ \ \ \ \ \ \ \ };
		
		% Perspectives
		\node[modus] (B) at (-7,7) {\textbf{{\hyperlink{prodmod}{Public Finance}}}};
				\node[modus] (E) at (8,7) {\textbf{\hyperlink{backhh}{{Political}}} {\textbf{Economy}}};
		% Dimensions
		\node[modus] (C) at (-2.5,7) {\textbf{\hyperlink{backhh}{{Inequality}}}};
		\node[modus] (D) at (2.5,7) {\textbf{\hyperlink{backhh}{{Behavioral}}} {\textbf{Changes}}};

		
		%	  \node[] (A) at (-10,3) {Projects: \ \ \ \ \ \ \ \ \ };	
		\node[elli] (F) at (-6.6,4.5) {{\hyperlink{prodmod}{Meeting Climate Targets:}}\\ {The Optimal}\\ {Fiscal Policy Mix}};
		
		\node[elli] (G) at (-3,-1) {{Labor Income Taxes,} \\ {Social Responsibility, and Inequality}};
		\node[elli] (H) at (0.5,2.5) {Voluntary  Reduction, Green Skills,\\ and the Environment};
		\node[elli] (I) at (6.4,0) {{Green Demand,}\\ Clean Innovation, and Lobbying};
		
		%\node[elli] (J) at (4.5,0) {{Worker's Concerns}\\ and Structural Transformation};
		
		\node[draw=none] (B1) at (5,4.25) {};
		\node[draw=none] (B2) at (5,3.5) {};
		\node[draw=none] (BA1) at (-5,4.25) {};
		\node[draw=none] (BA2) at (-5,3.5) {};
		\node[draw=none] (D1) at (1.8,7.8) {};
		
		\node[draw=none] (B22) at (6.4,5.6) {};
		\node[draw=none] (D2) at (2.3,8.5) {};
		\node[draw=none] (B3) at (5.3,5.3) {};
		\node[draw=none] (D3) at (-1.5,7) {};
		\node[draw=none] (A1) at (-4.2,4.6) {};
		\node[draw=none] (D4) at (-2,8.3) {};
		\node[draw=none] (A4) at (-6,5.4) {};
		%	 	\draw [->] (B22) to node[pos=0.5, swap]{{Tax on labor, $\pmb{\tau_{\iota}}$}} (D2);
		%	 	\draw [->] (D1) to node[pos=0.5, swap]{Transfers} (B3);
		%	 	
		%			\draw [->] (B1) to node[pos=0.75, swap]{Workers and scientists} (BA1);
		%			\draw [->] (BA2) to node[pos=0.75, swap]{Final good} (B2);
		%	 	\draw [->] (A4) to node[pos=0.5]{{Tax on carbon, $\pmb{\tau_F}$}}   (D4);
	\end{tikzpicture}
	
\end{frame}

\begin{frame}{Worker's Concerns and Structural Transformation}
		\vspace{-5mm}
\hspace{-6mm}\footnotesize{joint with Ximeng Fang (University of Oxford)}\\

\textbf{\alert{How to widen political support for structural transformations?}}

\begin{itemize}
	\item Was ist die Frage? (broad research question): Workers' concerns about structural transformations. (Could ask about suggested policies) 
	\item Was ist das Problem? (in answering this question or in the literature) 
	\item Hauptresultat
	\item Mechanismus
\end{itemize}
\begin{itemize}
	\item This paper
	\begin{itemize}
		\item[-]
		\item[-] Conduct qualitative survey to elicit potential concerns
		\item[-] Behavioral Experiments
		\item[-] Macroeconomic model to study 
	\end{itemize}
\end{itemize}


\end{frame}

content...
\end{comment}