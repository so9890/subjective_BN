\documentclass[11pt,aspectratio=169]{beamer}
%\documentclass[11pt,aspectratio=169, handout]{beamer}
%\usepackage{handoutWithNotes}
\usetheme[outer/progressbar=foot,
%outer/numbering=none
]{metropolis}
\setbeamertemplate{caption}{\raggedright\insertcaption\par}
\setbeamercolor{frametitle}{bg={}, fg=black!80}
\definecolor{myorange}{rgb}{0.8500, 0.3250, 0.0980}
\setbeamercolor{alerted text}{bg={}, fg=myorange }
\setbeamercolor{block title}{bg=black!10, fg=black}
\setbeamercolor{block body}{bg=black!10, fg=black}
%\usecolortheme{seahorse}
\usepackage[utf8]{inputenc}
\usepackage[english]{babel}
%\usepackage[T1]{fontenc}
\newcommand{\tr}[1]{\textcolor{blue}{#1}}
\usepackage{amsmath}
\usepackage{amsfonts}
\usepackage{amssymb}
\usepackage{mathtools}
\usepackage{calc}
\usepackage{soul}
\setbeamercolor{headerCol}{fg=blue!30,bg=black!80}
\setbeamercolor{bodyCol}{fg=black}
\usepackage{graphicx}
\usepackage{xcolor}
\usepackage{appendix}
\usepackage{hyperref}
\usepackage{natbib}
\usepackage{comment}
\usepackage{setspace}
\renewcommand{\bibsection}{}
\bibliographystyle{apa} 
% have to run bibtex mydocument.aux after first run to generate bbl file. 
\usepackage{appendixnumberbeamer}
\usepackage{xcolor}

%table
\usepackage{makecell}
\usepackage{multirow}
\usepackage{bigdelim}

\usepackage[customcolors]{hf-tikz}
\definecolor{sonja}{cmyk}{1.5,0,0.9,0.3}
%\definecolor{blue}{cmyk}{0,1,0,0}
\hfsetfillcolor{black!10}
\hfsetbordercolor{black}

\usepackage{tikz}
\usetikzlibrary{tikzmark}
\usetikzlibrary{decorations.markings}
\usepackage{tikz-cd}
\usetikzlibrary{arrows,calc,fit}
\tikzset{mainbox/.style={draw=white, text=white, fill=gray, rectangle, rounded corners, thick, node distance=7em, text width=8em, text centered, minimum height=3.5em}}
\tikzset{dummybox/.style={draw=none, text=white , rectangle, rounded corners, thick, node distance=7em, text width=8em, text centered, minimum height=3.5em}}
\tikzset{box/.style={draw , rectangle, rounded corners, thick, node distance=7em, text width=8em, text centered, minimum height=3.5em}}
\tikzset{container/.style={draw, rectangle, dashed, inner sep=2em}}
\tikzset{line/.style={draw, very thick, -latex'}}
\tikzset{    pil/.style={
		->,
		thick,
		shorten <=2pt,
		shorten >=2pt,}}
\tikzstyle{vecArrow} = [thick, decoration={markings,mark=at position
	1 with {\arrow[semithick]{open triangle 60}}},
double distance=1.4pt, shorten >= 5.5pt,
preaction = {decorate},
postaction = {draw,line width=1.4pt, white,shorten >= 4.5pt}]



%TITLE
\author[Sonja Dobkowitz]{\small Sonja Dobkowitz\\ \footnotesize{University of Bonn, RTG 2281 The Macroeconomics of Inequality}\\ }
\institute[University of Bonn]{}
\title{Labour Income Taxes\\ and Social Responsibility in an Unequal World }

\newcommand{\ar}{$\Rightarrow$ \ }

%\addtobeamertemplate{navigation symbols}{}{%
%    \usebeamerfont{footline}%
%    \usebeamercolor[fg]{footline}%
%    \hspace{1em}%
%   \insertframenumber/\inserttotalframenumber
%}

%\institute{University of Bonn} 
\date{\small{EEA-ESEM Congress 2022\\ August 22, 2022 }} 
%\subject{} 
\begin{document}
	
	{\setbeamertemplate{footline}{}
		\begin{frame}
			\titlepage
		\end{frame}
	}
	\addtocounter{framenumber}{-1}
	
	% {\setbeamertemplate{footline}{}
	% \begin{frame}{Content}
	% \vspace{4mm}
	% \tableofcontents
	% \end{frame}
	% }
	% \addtocounter{framenumber}{-1}
	
	
	%---------------------------------------
	%            Intro
	%---------------------------------------
	
	
	
	\begin{frame}{Motivation}
		\vspace{4mm}
		\begin{itemize}[<+-| alert@+>]
			\setbeamercolor{alerted text}{fg=black} %change the font color
			\setbeamerfont{alerted text}{} 
			
			%	\item<1-> climate change raises the need for a transition to sustainable production
			%	\vspace{3mm}
			\item<+-|alert@+>households'  \textcolor{myorange}{\textbf{willingness to pay to avoid negative externalities}} \\ (``social responsibility'') is rising  %\citep{Bartling2015DoResponsibility, Kronthal-Sacco2020SustainableMessages}

			\begin{itemize}
				\vspace{4mm}
				\item<+->  \cite{Bartling2015DoResponsibility} provide experimental evidence for the existence of social responsibility in markets 
%				%				\vspace{2mm}
%				%				\item<+-> 
%				%				global share of households willing to pay a
%				%				premium rose from 50\% in 2013 to 66\% in 2015 \citep{NielsenSUSIMp} %(sample of 60 countries)
				\vspace{2mm}
				\item<+-> in the US, the market share of sustainable consumer-packaged goods rose from 14\% in 2013 to 16\% in 2018 despite a price premium \citep{Kronthal-Sacco2020SustainableMessages}%; faster growth in product sales with sustainability claim (March 2017-March2018) \citep{NielsenSUS2018}
%				\vspace{2mm} 
%				\item<+-> accepted price premium on average: 25\% %(multi-country)
%				\citep{SK2021}
			\end{itemize} % \ar \textbf{\textcolor{myorange}{social responsibility}}% \footnotesize{(Note:ADD link to NIELSEN DATA on willingness to pay/ identity/ benabou)}
			\vspace{5mm}
			\item[]<+-|alert@+>\ar {a demand-driven transition to sustainable production}
			\vspace{5mm}
			
			\item<+-| alert@+>  but: \textcolor{myorange}{\textbf{inequality}} renders sustainable goods unaffordable for poor households
		\end{itemize}
		
		% the role of demand and redistribution on an corrective externality is what has in large been overlooked by the macroeconomic literature...
		\vspace{5mm}
		\hfill	\hyperlink{indis}{\tiny{$\rightarrow$ graph,}}
		\hyperlink{rise2}{\tiny{$\rightarrow$ rise}}
	\end{frame}
	
	
	
	\begin{frame}{Therefore, ...}

		\vspace{-3mm}
		\textbf{\textcolor{myorange}{... rising social responsibility has opposed consequences for the optimal policy}}
		\pause
		\vspace{2mm}
		%\begin{minipage}[t!]{0.45\textwidth}
		%\begin{center}
		\begin{itemize}	\setbeamercolor{alerted text}{fg=black} %change the font color
			\setbeamerfont{alerted text}{series=\bfseries}
			
			\item<+-> \underline{On the one hand:}
			%\end{center} 
			\begin{itemize}	\setbeamercolor{alerted text}{fg=black} %change the font color
				\setbeamerfont{alerted text}{series=\bfseries}
				\vspace{2mm}
				\item[-]<+-| alert@+> demand-driven reduction of the externality
				\ar less government intervention; lower externality attained at higher equity/output level
				%		\vspace{1mm}
				%		\item[\ar]<+-| alert@+> less government intervention necessary
				%		\
				%, as households derive less utility from consuming the unsustainable good 
				%\item<+-> high elasticity of demand at relative price equal to 1
				%	\vspace{1mm}
				%	\item<+-| alert@+> the corrective tax can raise consumption utility of the poor %by making the sustainable good the cheaper alternative
				%				\vspace{1mm}
				%				\item[\ar]<+-| alert@+>
				%				higher corrective tax
			\end{itemize}
			\vspace{2mm}
			\item<+-> \underline{On the other hand:}
			%	\end{center} 
			\begin{itemize}	
				\setbeamercolor{alerted text}{fg=black} %change the font color
				\setbeamerfont{alerted text}{series=\bfseries}
				\vspace{2mm}
				\item[-]<+-| alert@+> basic needs mute shift in demand
				\vspace{1mm}
				%	\item<+-| alert@+> the efficiency level of the externality declines
				%%	
				%	\vspace{1mm}
				\item[-]<+-| alert@+> consumption inequality increases
				\vspace{2mm}
				\item[\ar]<+-| alert@+> more government intervention due to equity; high inequality and small reduction in externality %: the poor (i) consume less of the desired bundle and (ii) accept too low consumption levels
				
				%allowing poor households to consume more sustainable goods
				%	\vspace{1mm}
				%	\item[\ar]<+-| alert@+> more government intervention necessary
			\end{itemize}
			%\end{minipage}
			%\vspace{2mm}
			%	\item<+-> both motives compete due to efficiency costs of policy instruments
			% despite the demand driven reduction, the government might find it optimal to relinquish an efficient reduction of the externality; 
			%			alternatively: as ext reduces, more resources for redistribution \ar also able to achieve higher equity at lower externality
			\vspace{2mm}
			%			\item<+-> optimal policy depends on relative importance of each motive
			
			
			\item<+-> \underline{Then again:} redistribution may target both motives
			%\vspace{3mm}
			\vspace{2mm}
			\item[]<+-| alert@+> 
			\textcolor{myorange}{\textbf{\ar What is the optimal policy as social responsibility increases?}}
			% it could be that inequality rises so much that the gov. forfeits an efficient reduction in the externality
		\end{itemize}
	\hypertarget{backmov}{}
	\end{frame}
	\begin{comment}
	\begin{frame}{This Paper}
	\begin{itemize}
	\item<+-> two-sector model with demand-determined economic structure \vspace{2mm}
	
	\item<+-> unsustainable sector causes an \alert{\textbf{corrective externality}}%: $U(c_i, l_i; H_n)$ 
	\vspace{2mm}% relative sector size is a function of social responsibility and inequality
	\item<+-> two household types: income-rich and income-poor \vspace{2mm}
	\item<+-> households %\hyperlink{share}{\textbf{share}}
	tastes over sustainable and unsustainable goods are homogeneous
	%		$
	%		c_i =
	%		\left(\textcolor{myorange}{\omega}^{\frac{1}{\sigma}}c_{st}^{\frac{\sigma-1}{\sigma}}+(1-\textcolor{myorange}{\omega})^{\frac{1}{\sigma}}c_{nt}^{\frac{\sigma-1}{\sigma}}\right)^{\frac{\sigma}{\sigma-1}}.
	%		$
	
	\vspace{2mm}
	\item<+-> requirement to satisfy \alert{\textbf{basic needs}}%: $ U(c_i,l_i;H_{n})= u(c_i,l_i) \textcolor{myorange}{-\frac{1}{\phi}\exp(-\phi(c_{ni}+c_{si}-\bar{c}))}+g(H_{n}).$ 
	\vspace{2mm} % \pause \ar \alert{\textbf{income-dependent marginal propensities to consume unsustainably}} (MPCU) % role for redistribution as corrective policy
	\item<+-> Ramsey planner can use corrective and distortionary labour tax %to maximise social welfare
	\vspace{2mm}% focus on role of redistribution, trade-off with corrective tax which is classically used as corrective policy (Pigou) 
	\item<+-> study effect of an \textbf{\alert{exogenous rise in social responsibility}} on optimal policy %and allocations
	
	% either social responsibility absent inequality, and in optimal climate change macro litertaure: no inequality role for demand
	\end{itemize}
	\end{frame}
	
	content...
	\end{comment}
	

	\begin{frame}{Preview of results}
		\hypertarget{backCont}{}
		\vspace{1mm}
		\begin{itemize}
			\item[]<+->\textbf{\alert{Irrespective of social responsibility, ...}}
			\vspace{4mm}
			\begin{itemize}[<+-| alert@+>]
				\setbeamercolor{alerted text}{fg=black} %change the font color
				\setbeamerfont{alerted text}{series=\bfseries} 
				\item<+-| alert@+>  ... labour income taxes are optimally used to target the externality due to inequality.
								\vspace{3mm}
			\end{itemize}
			
			\item[]<+->\textbf{\alert{As social responsibility rises, ...}}
			\vspace{4mm}
			\begin{itemize}[<+-| alert@+>]
				\setbeamercolor{alerted text}{fg=black} %change the font color
				\setbeamerfont{alerted text}{series=\bfseries} 
				\item<+-| alert@+>  ... the optimal policy shifts away from corrective taxation to redistribution since inequality aggravates.
				\vspace{3mm}
				\item<+-| alert@+> ... the government redistributes even more to target the externality.
			%	\vspace{3mm} \item<+-| alert@+> ... redistribution becomes the central pillar of the corrective policy.  % accounting for up 93\% of the policy effect on the externality%if social responsibility is high%as social responsibility rises
				% less costly in terms of equity

			\end{itemize}
			
			
		\end{itemize}
		%		\hfill	\hyperlink{Contrib}{\tiny{$\rightarrow$ Contribution to the literature}}
	\end{frame}
	


	%-----------------Model-----------------------------
		\begin{comment}
	\begin{frame}{Roadmap}
		\tableofcontents
	\end{frame}
	\end{comment}

	\section{Model}
	
	
	\begin{frame}{Model}
		\textbf{Timing:} static model\\
		\vspace{1mm}
		\pause
		\textbf{Inequality:}
		\vspace{-2mm}
		\begin{itemize}
			\item[-] fraction $\lambda$ is rich with productivity $z_h$
			\item[-]  fraction $1-\lambda$ is poor with productivity $z_l<z_h$
		\end{itemize} 
	\pause
		\textbf{Production:}
		\vspace{-2mm}
		\begin{itemize}
			\item[-] two perfectly competitive sectors s and n 
			\item[-] production function: $Y_j=A_jH_j,\ \text{for} \ j\in\{s,n\}$
			\item[-] profits: $\pi_s=p_sY_s-wH_s$, \\ \hspace{12mm} $\pi_n=Y_n-w(1+\pmb{\textcolor{myorange}{\tau_n}})H_n$
		\end{itemize} 
	\pause
	\textbf{Government:} Ramsey planner maximises Utilitarian social welfare choosing $\tau_n$ and $\tau_l$\\
	\pause
	 		\vspace{1mm}
	\textbf{Markets:} for goods and labour clear
	\end{frame}
	
	\begin{frame}{Household problem} % \tr{here I want to motivate the penalty term}
		\vspace{2mm}
		\begin{minipage}[t!]{1\textwidth}
			\begin{align*}
			\tikzmarkin{first}(0.8,2.8)(-0.4,-2.6)
			%	\underset{c_{s,i},c_{n,i}, l_i}{\max} \ \hspace{2mm} U(c_{s,i}, c_{n,i}, l_i; h_n)= 
			\underset{c_{si},c_{ni}, l_i}{\max} U_i=\ \hspace{2mm} \underset{c_{si},c_{ni}, l_i}{\max} \log(c(c_{si},c_{ni}))-\chi\frac{l_i^{1+\frac{1}{\theta}}}{1+\frac{1}{\theta}}  \textcolor{black!10}{-\frac{1}{\phi}\exp(-\phi(c_{si}+c_{ni}-\bar{c})) -\psi H_n^\eta}\\
			s.t.\ \hspace{6mm}  \hspace{2mm }p_{s}c_{si}+c_{ni}\leq w(1-\pmb{\textcolor{myorange}{\tau_l}})z_il_i+T%; \ \hspace{6mm}(2) \ l_i\leq L
			\\	\hspace{-10mm} \textcolor{black!10}{where} \hspace{8mm}\textcolor{black!10}{c =
				\left(\textcolor{black!10}{\omega}^{\frac{1}{\sigma}}c_{s}^{\frac{\sigma-1}{\sigma}}+(1-\textcolor{black!10}{\omega})^{\frac{1}{\sigma}}c_{n}^{\frac{\sigma-1}{\sigma}}\right)^{\frac{\sigma}{\sigma-1}}}%& \hspace{2mm} \text{where} \hspace*{2mm} \sigma \neq 1
			\tikzmarkend{first}
			\end{align*}
		\end{minipage}
		
		%\begin{minipage}[t!]{0.4\textwidth}
		%
		%\begin{align*}
		%\hspace{4mm}c =
		%\left(\textcolor{blue}{\omega}^{\frac{1}{\sigma}}c_{s}^{\frac{\sigma-1}{\sigma}}+(1-\textcolor{blue}{\omega})^{\frac{1}{\sigma}}c_{n}^{\frac{\sigma-1}{\sigma}}\right)^{\frac{\sigma}{\sigma-1}}& \hspace{2mm} \text{where} \hspace*{2mm} \sigma \neq 1
		%\end{align*}
		%\end{minipage}
		
		\small
		\begin{minipage}[t!]{0.4\textwidth}
			\vspace{7mm}
			\begin{itemize}
				\item[] $c_s$: sustainable good \vspace{-2mm}
				\item[] $c_n$: unsustainable good\vspace{-2mm}
				\item[] $\tau_l$: linear labour tax
			\end{itemize}
		\end{minipage}
		\begin{minipage}[t!]{0.4\textwidth}
			\vspace{7mm}
			%\begin{itemize}	
			%	\item[] $\bar{c}$: basic needs
			%	\vspace{-2mm}
			%	\item[] $\phi$: importance penalty
			%\vspace{-2mm}	
			%		\item[] $H_n$: unsustainable labour input	
			%	\end{itemize}
		\end{minipage}
		
	\end{frame}
	
		\addtocounter{framenumber}{-1}
	\begin{frame}{Household problem} % \tr{here I want to motivate the penalty term}
		\hypertarget{backAtt}{}
		\vspace{2mm}
		\begin{minipage}[t!]{1\textwidth}
			\begin{align*}
			\tikzmarkin{first}(0.8,3)(-0.4,-2.6)
			%	\underset{c_{s,i},c_{n,i}, l_i}{\max} \ \hspace{2mm} U(c_{s,i}, c_{n,i}, l_i; h_n)= 
			\underset{c_{si},c_{ni}, l_i}{\max} U_i=\ \hspace{2mm} \underset{c_{si},c_{ni}, l_i}{\max} \log(c(c_{si},c_{ni}))-\chi\frac{l_i^{1+\frac{1}{\theta}}}{1+\frac{1}{\theta}}  \textcolor{black!10}{-\frac{1}{\phi}\exp(-\phi(c_{si}+c_{ni}-\bar{c})) -\psi H_n^\eta}\\
			s.t.\ \hspace{6mm}  \hspace{2mm }p_{s}c_{si}+c_{ni}\leq w(1-\textcolor{black}{\tau_l})z_il_i+T%; \ \hspace{6mm}(2) \ l_i\leq L
			\\
			where \hspace{8mm}c(c_{si},c_{ni}) =
			\left(\pmb{\textcolor{myorange}{\omega}}^{\frac{1}{\sigma}}c_{si}^{\frac{\sigma-1}{\sigma}}+(1-\pmb{\textcolor{myorange}{\omega}})^{\frac{1}{\sigma}}c_{ni}^{\frac{\sigma-1}{\sigma}}\right)^{\frac{\sigma}{\sigma-1}}%& \hspace{2mm} \text{where} \hspace*{2mm} \sigma \neq 1
			\tikzmarkend{first}
			\end{align*}
		\end{minipage}
		
		%\begin{minipage}[t!]{0.4\textwidth}
		%
		%\begin{align*}
		%\hspace{4mm}c =
		%\left(\textcolor{blue}{\omega}^{\frac{1}{\sigma}}c_{s}^{\frac{\sigma-1}{\sigma}}+(1-\textcolor{blue}{\omega})^{\frac{1}{\sigma}}c_{n}^{\frac{\sigma-1}{\sigma}}\right)^{\frac{\sigma}{\sigma-1}}& \hspace{2mm} \text{where} \hspace*{2mm} \sigma \neq 1
		%\end{align*}
		%\end{minipage}
		
		\small
		\begin{minipage}[t!]{0.32\textwidth}
			\vspace{7mm}
			\begin{itemize}
				\item[] $c_s$: sustainable good \vspace{-2mm}
				\item[] $c_n$: unsustainable good\vspace{-2mm}
				\item[] $\tau_l$: linear labour tax
				\item[]
				\item[]
			\end{itemize}
		\end{minipage}
		\begin{minipage}[t!]{0.35\textwidth}
			\vspace{8mm}
			\begin{itemize}
				\item[$\sigma$:]  governs price elasticity of substitution%; for $\sigma>1$ goods are substitutes
				\vspace{-2mm}
				\item[$\omega$:] \textit{social responsibility}; governs willingness to pay for sustainable goods % which is increasing in $\omega_t$: % in equilibrium wtp of the marginal buyer
				\item[]	
			\end{itemize}
		\end{minipage}
		\begin{minipage}[t!]{0.4\textwidth}
			\vspace{7mm}
			%\begin{itemize}	
			%	\item[] $\bar{c}$: basic needs
			%	\vspace{-2mm}
			%	\item[] $\phi$: importance penalty
			%\vspace{-2mm}	
			%		\item[] $H_n$: unsustainable labour input	
			%	\end{itemize}
		\end{minipage}
		
		\vspace{-7mm}
		\hfill	\hyperlink{atts}{\tiny{$\rightarrow$ Attitudes by income}}
	\end{frame}
	
	\begin{comment}
	\begin{frame}{Household problem} % \tr{here I want to motivate the penalty term}
	%	\hypertarget{backBN}{}
	%	\hypertarget{backengel}{}
	\vspace{2mm}
	\begin{minipage}[t!]{1\textwidth}
	\begin{align*}
	\tikzmarkin{first}(0.8,3)(-0.4,-2.6)
	%	\underset{c_{s,i},c_{n,i}, l_i}{\max} \ \hspace{2mm} U(c_{s,i}, c_{n,i}, l_i; h_n)= 
	\underset{c_{si},c_{ni}, l_i}{\max} U_i=\ \hspace{2mm} \underset{c_{si},c_{ni}, l_i}{\max} \log(c(c_{si},c_{ni}))-\chi\frac{l_i^{1+\frac{1}{\theta}}}{1+\frac{1}{\theta}}  \pmb{\textcolor{myorange}{-\frac{1}{\phi}\exp(-\phi(c_{si}+c_{ni}-\bar{c}))}}\textcolor{black!10}{ -\psi H_n^\eta}\\ 
	s.t.\ \hspace{6mm}  \hspace{2mm }p_{s}c_{si}+c_{ni}\leq w(1-\textcolor{black}{\tau_l})z_il_i+T%; \ \hspace{6mm}(2) \ l_i\leq L
	\\\hspace{-10mm} where \hspace{8mm}c_i =
	\left(\textcolor{black}{\omega}^{\frac{1}{\sigma}}c_{si}^{\frac{\sigma-1}{\sigma}}+(1-\textcolor{black}{\omega})^{\frac{1}{\sigma}}c_{ni}^{\frac{\sigma-1}{\sigma}}\right)^{\frac{\sigma}{\sigma-1}}%& \hspace{2mm} \text{where} \hspace*{2mm} \sigma \neq 1
	\tikzmarkend{first}
	\end{align*}
	\end{minipage}
	
	%\begin{minipage}[t!]{0.4\textwidth}
	%
	%\begin{align*}
	%\hspace{4mm}c =
	%\left(\textcolor{blue}{\omega}^{\frac{1}{\sigma}}c_{s}^{\frac{\sigma-1}{\sigma}}+(1-\textcolor{blue}{\omega})^{\frac{1}{\sigma}}c_{n}^{\frac{\sigma-1}{\sigma}}\right)^{\frac{\sigma}{\sigma-1}}& \hspace{2mm} \text{where} \hspace*{2mm} \sigma \neq 1
	%\end{align*}
	%\end{minipage}
	
	\small
	\begin{minipage}[t!]{0.32\textwidth}
	\vspace{7mm}
	\begin{itemize}
	\item[] $c_s$: sustainable good \vspace{-2mm}
	\item[] $c_n$: unsustainable good\vspace{-2mm}
	\item[] $\tau_l$: linear labour tax
	\item[]
	\item[]
	\end{itemize}
	\end{minipage}
	\begin{minipage}[t!]{0.35\textwidth}
	\vspace{8mm}
	\begin{itemize}
	\item[$\sigma$:]  governs price elasticity of substitution%; for $\sigma>1$ goods are substitutes
	\vspace{-2mm}
	\item[$\omega$:] \textit{social responsibility}; governs willingness to pay for sustainabile goods % which is increasing in $\omega_t$: % in equilibrium wtp of the marginal buyer
	\item[]	
	\end{itemize}
	\end{minipage}
	\begin{minipage}[t!]{0.3\textwidth}
	\vspace{7mm}
	\begin{itemize}	
	\item[] $\bar{c}$: basic needs
	\vspace{-2mm}
	\item[] $\phi$: importance penalty
	\vspace{-2mm}	
	\item[]% $H_n$: unsustainable labour input	
	\item[]
	\item[]
	
	\vspace{-2mm}
	%				\hfill	\hyperlink{wtp}{\tiny{$\rightarrow$ Willingness to pay}}\\ \vspace{-2mm} \hfill	\hyperlink{intensiveMargin}{\tiny{$\rightarrow$ Intensive margin}}\\ \vspace{-2mm} 
	%	\hfill \hyperlink{engel}{\tiny{$\rightarrow$ Engel curves}}
	\end{itemize}
	\end{minipage}
	\end{frame}
	
	content...
	\end{comment}
	
	\addtocounter{framenumber}{-1}
	\begin{frame}{Household problem} % \tr{here I want to motivate the penalty term}
		\hypertarget{backBN}{}
		\hypertarget{backengel}{}
		\hypertarget{wtp}{}
		\vspace{2mm}
		\begin{minipage}[t!]{1\textwidth}
			\begin{align*}
			\tikzmarkin{first}(0.8,3)(-0.4,-2.6)
			%	\underset{c_{s,i},c_{n,i}, l_i}{\max} \ \hspace{2mm} U(c_{s,i}, c_{n,i}, l_i; h_n)= 
			\underset{c_{si},c_{ni}, l_i}{\max} U_i=\ \hspace{2mm} \underset{c_{si},c_{ni}, l_i}{\max} \log(c(c_{si},c_{ni}))-\chi\frac{l_i^{1+\frac{1}{\theta}}}{1+\frac{1}{\theta}}  \pmb{\textcolor{myorange}{-\frac{1}{\phi}\exp(-\phi(c_{si}+c_{ni}-\bar{c}))}}\textcolor{black!10}{ -\psi H_n^\eta}\\ 
			s.t.\ \hspace{6mm}  \hspace{2mm }p_{s}c_{si}+c_{ni}\leq w(1-\textcolor{black}{\tau_l})z_il_i+T%; \ \hspace{6mm}(2) \ l_i\leq L
			\\\hspace{-10mm} where \hspace{8mm}c(c_{si},c_{ni}) =
			\left(\textcolor{black}{\omega}^{\frac{1}{\sigma}}c_{si}^{\frac{\sigma-1}{\sigma}}+(1-\textcolor{black}{\omega})^{\frac{1}{\sigma}}c_{ni}^{\frac{\sigma-1}{\sigma}}\right)^{\frac{\sigma}{\sigma-1}}%& \hspace{2mm} \text{where} \hspace*{2mm} \sigma \neq 1
			\tikzmarkend{first}
			\end{align*}
		\end{minipage}
		
		%\begin{minipage}[t!]{0.4\textwidth}
		%
		%\begin{align*}
		%\hspace{4mm}c =
		%\left(\textcolor{blue}{\omega}^{\frac{1}{\sigma}}c_{s}^{\frac{\sigma-1}{\sigma}}+(1-\textcolor{blue}{\omega})^{\frac{1}{\sigma}}c_{n}^{\frac{\sigma-1}{\sigma}}\right)^{\frac{\sigma}{\sigma-1}}& \hspace{2mm} \text{where} \hspace*{2mm} \sigma \neq 1
		%\end{align*}
		%\end{minipage}
		
		\small
		\begin{minipage}[t!]{0.32\textwidth}
			\vspace{7mm}
			\begin{itemize}
				\item[] $c_s$: sustainable good \vspace{-2mm}
				\item[] $c_n$: unsustainable good\vspace{-2mm}
				\item[] $\tau_l$: linear labour tax
				\item[]
				\item[]
			\end{itemize}
		\end{minipage}
		\begin{minipage}[t!]{0.35\textwidth}
			\vspace{8mm}
			\begin{itemize}
				\item[$\sigma$:]  governs price elasticity of substitution%; for $\sigma>1$ goods are substitutes
				\vspace{-2mm}
				\item[$\omega$:] \textit{social responsibility}; governs willingness to pay for sustainable goods % which is increasing in $\omega_t$: % in equilibrium wtp of the marginal buyer
				\item[]	
			\end{itemize}
		\end{minipage}
		\begin{minipage}[t!]{0.3\textwidth}
			\vspace{7mm}
			\begin{itemize}	
				\item[] $\bar{c}$: basic needs
				\vspace{-2mm}
				\item[] $\phi$: importance penalty
				\vspace{-2mm}	
				\item[]% $H_n$: unsustainable labour input	
				\item[]
				\item[]
				
				\vspace{-2mm}
				\hfill	\hyperlink{WTP}{\tiny{$\rightarrow$ Willingness to pay}}\\ \vspace{-2mm} %\hfill	\hyperlink{intensiveMargin}{\tiny{$\rightarrow$ Intensive margin}}\\ \vspace{-2mm} 
				\hfill \hyperlink{engel}{\tiny{$\rightarrow$ Engel curves}}
			\end{itemize}
		\end{minipage}
	\end{frame}
	
	
	\addtocounter{framenumber}{-1}
	\begin{frame}{Household problem} % \tr{here I want to motivate the penalty term}
		\vspace{2mm}
		\begin{minipage}[t!]{1\textwidth}
			\begin{align*}
			\tikzmarkin{first}(0.8,3)(-0.4,-2.6)
			%	\underset{c_{s,i},c_{n,i}, l_i}{\max} \ \hspace{2mm} U(c_{s,i}, c_{n,i}, l_i; h_n)= 
			\underset{c_{si},c_{ni}, l_i}{\max} U_i=\ \hspace{2mm} \underset{c_{si},c_{ni}, l_i}{\max} \log(c(c_{si},c_{ni}))-\chi\frac{l_i^{1+\frac{1}{\theta}}}{1+\frac{1}{\theta}}  \textcolor{black}{-\frac{1}{\phi}\exp(-\phi(c_{si}+c_{ni}-\bar{c}))}\pmb{\textcolor{myorange}{ -\psi H_n^\eta}}\\ 
			s.t.\ \hspace{6mm}  \hspace{2mm }p_{s}c_{si}+c_{ni}\leq w(1-\textcolor{black}{\tau_l})z_il_i+T%; \ \hspace{6mm}(2) \ l_i\leq L
			\\\hspace{-10mm} where \hspace{8mm}c(c_{si},c_{ni}) =
			\left(\textcolor{black}{\omega}^{\frac{1}{\sigma}}c_{si}^{\frac{\sigma-1}{\sigma}}+(1-\textcolor{black}{\omega})^{\frac{1}{\sigma}}c_{ni}^{\frac{\sigma-1}{\sigma}}\right)^{\frac{\sigma}{\sigma-1}}%& \hspace{2mm} \text{where} \hspace*{2mm} \sigma \neq 1
			\tikzmarkend{first}
			\end{align*}
		\end{minipage}
		
		%\begin{minipage}[t!]{0.4\textwidth}
		%
		%\begin{align*}
		%\hspace{4mm}c =
		%\left(\textcolor{blue}{\omega}^{\frac{1}{\sigma}}c_{s}^{\frac{\sigma-1}{\sigma}}+(1-\textcolor{blue}{\omega})^{\frac{1}{\sigma}}c_{n}^{\frac{\sigma-1}{\sigma}}\right)^{\frac{\sigma}{\sigma-1}}& \hspace{2mm} \text{where} \hspace*{2mm} \sigma \neq 1
		%\end{align*}
		%\end{minipage}
		
		\small
		\begin{minipage}[t!]{0.32\textwidth}
			\vspace{7mm}
			\begin{itemize}
				\item[] $c_s$: sustainable good \vspace{-2mm}
				\item[] $c_n$: unsustainable good\vspace{-2mm}
				\item[] $\tau_l$: linear labour tax
				\item[]
				\item[]
			\end{itemize}
		\end{minipage}
		\begin{minipage}[t!]{0.35\textwidth}
			\vspace{8mm}
			\begin{itemize}
				\item[$\sigma$:]  governs price elasticity of substitution%; for $\sigma>1$ goods are substitutes
				\vspace{-2mm}
				\item[$\omega$:] \textit{social responsibility}; governs willingness to pay for sustainable goods % which is increasing in $\omega_t$: % in equilibrium wtp of the marginal buyer
				% the higher omega the higher the utility derived from sustainable consumption
				\item[]	
			\end{itemize}
		\end{minipage}
		\begin{minipage}[t!]{0.3\textwidth}
			\vspace{9mm}
			\begin{itemize}	
				\item[] $\bar{c}$: basic needs
				\vspace{-2mm}
				\item[] $\phi$: importance penalty
				\vspace{-2mm}	
				\item[] $H_n$: unsustainable labour input	
				\item[]
				\item[]
			\end{itemize}
		\end{minipage}
	\\ 
	\vspace{-8mm}
	\hfill
		\hyperlink{red}{\tiny{$\rightarrow$ Calibration,}}		\hyperlink{backpa}{\tiny{$\rightarrow$ Ramsey}}
	\hypertarget{backmodel}{}
	\end{frame}
%	

	
	%-----------------results---------------------------
	\hypertarget{backres}{}
	\section{ Results}
	\hypertarget{backOmega}{}
	
	%-----------------------Experiment--------------
	
	%\begin{frame}{Quantitative Exercises and Results} 
	%	%\vspace{6mm}
	%	
	%	\begin{enumerate}[<+-| alert@+>]
	%		\setbeamercolor{alerted text}{fg=black} %change the font color
	%		\setbeamerfont{alerted text}{series=\bfseries} 
	%		%	\item<+-| alert@+> Effects of social responsibility: Engel Curves
	%		%			\vspace{4mm}		
	%		\item<+->  \textcolor{myorange}{{\textbf{Optimal policy and allocation}}} (exogenously changing social responsibility)
	%		\vspace{3mm}
	%		\item[]<+-> Interpretation:
	%		Comparison to efficient allocation
	%		\vspace{4mm}
	%		\item<+->\textcolor{myorange}{ \textbf{Income tax as an corrective policy instrument}}  
	%		
	%		(comparison to models without externality)
	%		\vspace{4mm}
	%		\item<+-> \textcolor{myorange}{\textbf{Importance of redistribution for the corrective policy}} 
	%		
	%		(timing assumption to decompose policy effects)
	%	\end{enumerate}
	%	
	%	
	%	
	%	%\vspace{30mm}
	%	%		\hfill	\hyperlink{intensiveMargin}{\tiny{$\rightarrow$ Intensive margin}}
	%\end{frame}
	
	
	
	%----------------Main results ------------------------
%	\section*{Optimal policy}
	%---------- optimal allocation
	\begin{frame}{ Optimal policy}
		\hypertarget{backmain}{}
		\vspace{4mm}
		\centering
		\begin{minipage}[]{0.32\textwidth}
			\centering{\footnotesize{Corrective tax, $\tau_n$\\ \ }}
			%	\captionsetup{width=.45\linewidth}
			\includegraphics[width=1\textwidth]{../codding_new/policies_solution/solutions_paper/fullmodel_pen1_targetext_2_tauun_withps0_lgd0_notauun0_pslow0_PSID1_matchGDP0_phii12.png}
		\end{minipage}
		\begin{minipage}[]{0.32\textwidth}
			\centering{\footnotesize{Income tax, $\tau_l$\\ \ }}
			%	\captionsetup{width=.45\linewidth}
			\includegraphics[width=1\textwidth]{../codding_new/policies_solution/solutions_paper/fullmodel_pen1_targetext_2_tauul_withps0_lgd0_notauun0_pslow0_PSID1_matchGDP0_phii12.png}
		\end{minipage}
		\begin{minipage}[]{0.32\textwidth}
			\centering{\footnotesize{Transfers, T\\ \ }}
			%	\captionsetup{width=.45\linewidth}
			\includegraphics[width=1\textwidth]{../codding_new/policies_solution/solutions_paper/fullmodel_pen1_targetext_2_T_withps0_lgd0_notauun0_pslow0_PSID1_matchGDP0_phii12.png}
		\end{minipage}
		\vspace{5mm}
		\begin{itemize}[<+-| alert@+>]
			\setbeamercolor{alerted text}{fg=black}
			\item shift in optimal policy mix away from corrective taxation to redistribution 			     
		\end{itemize}
		
			\vspace{11mm}
			\hfill
			\hyperlink{backallo}{\tiny{$\rightarrow$ allocation}}
			\hypertarget{backopt}{}
	\end{frame}
	
	
	%-------------optimal allocation
	%\begin{frame}{Optimal allocation}
	%	%\tr{this one and next slide to be moved to appendix (partially); this slide: comparison baseline model with basic needs to model without basic needs}
	%	\vspace{-5mm}
	%	\hypertarget{backallo}{}
	%	\begin{figure}			
	%		\begin{minipage}[]{0.32\textwidth}
	%			\centering{\footnotesize{Externality, $Y_n$\\ \ }}
	%			%	\captionsetup{width=.45\linewidth}
	%			\includegraphics[width=1\textwidth]{../codding/policies_solution/solutions_paper/fullmodel_pen1_targetext_2_yn_withps0_nolgd_notauun0_pslow0_PSID1_matchGDP0_phii15.png}
	%		\end{minipage}	
	%		\begin{minipage}[]{0.32\textwidth}
	%			\centering{\footnotesize{Output\\ \ }}
	%			%	\captionsetup{width=.45\linewidth}
	%			\includegraphics[width=1\textwidth]{../codding/policies_solution/solutions_paper/fullmodel_pen1_targetext_2_agg_output_withps0_nolgd_notauun0_pslow0_PSID1_matchGDP0_phii15.png}
	%		\end{minipage}
	%		\begin{minipage}[]{0.32\textwidth}
	%			\ 
	%%			\centering{\footnotesize{Gini of consumption\\ \ }}
	%%			%		%	\captionsetup{width=.45\linewidth}
	%%			\includegraphics[width=1\textwidth]{\ }
	%		\end{minipage}
	%	\end{figure}
	%	\begin{itemize}
	%		\item reduction in externality at higher output level
	%	%	\item<1-> inequality rises
	%		%	\item<2-> I will argue: higher output is a choice to mitigate inequality
	%	\end{itemize}
	%	%		\vspace{-1mm}
	%	%		\hfill
	%	%		\hyperlink{conc}{\tiny{$\rightarrow$ Conclusion}}
	%\end{frame}	
	%
	%\addtocounter{framenumber}{-1}
	
%	\section*{Mechanism}
	%\begin{frame}{Optimal policy without basic needs}
	%\begin{itemize}
	%\item no rise in inequality
	%\item redistribution does not change the externality
	%\end{itemize}
	%\end{frame}
	%\addtocounter{framenumber}{-1}
	
	%	\begin{frame}{Efficient allocation}
	%		Basic needs alter effectiveness of corrective tax\\  \ar unclear if policy difference driven by inequality or externality motive
	%	
	%\end{frame}
	%
	%\addtocounter{framenumber}{-1}
	
%	\begin{frame}{Efficient allocation}
%		
%		\vspace{4mm}
%		\begin{figure}
%			\begin{minipage}[]{0.32\textwidth}
%				\centering{\footnotesize{ Externality, $Y_n$ \\ \  }}
%				%	\captionsetup{width=.45\linewidth}
%				\includegraphics[width=1\textwidth]{../codding_new/policies_solution/solutions_paper/Compariosn_socialPlanner_pen1_targetext_2_yn_lgd1_notauun0_pslow0_PSID1_matchGDP0_phii12.png}
%			\end{minipage}
%			\begin{minipage}[]{0.32\textwidth}
%				\
%			%	\centering{\footnotesize{ Output ratio, $Y_n/Y_s$ \\ \  }}
%				%	\captionsetup{width=.45\linewidth}
%			%	\includegraphics[width=1\textwidth]{../codding_new/policies_solution/solutions_paper/Compariosn_socialPlanner_pen1_targetext_2_output_ratio_lgd0_notauun0_pslow0_PSID1_matchGDP0_phii12.png}
%			\end{minipage}
%			\begin{minipage}[]{0.32\textwidth}
%				\ 
%			%	\centering{\footnotesize{ Output ratio, $Y_n/Y_s$, \\ without basic needs}}
%				%	\captionsetup{width=.45\linewidth}
%			%	\includegraphics[width=1\textwidth]{../codding_new/policies_solution/solutions_paper/Compariosn_socialPlanner_pen0_targetext_2_output_ratio_lgd0_notauun0_pslow0_PSID1_matchGDP0_phii12.png}
%			\end{minipage}
%			
%			%\begin{minipage}[]{0.32\textwidth}
%			%	\centering{\footnotesize{ Unsustainable output, $y_n$, \\ baseline model}}
%			%	%	\captionsetup{width=.45\linewidth}
%			%	\includegraphics[width=1\textwidth]{../codding/policies_solution/solutions_paper/Compariosn_socialPlanner_pen1_targetext_2_yn_lgd0_notauun0_pslow0_PSID1_matchGDP0_phii15.png}
%			%\end{minipage}
%		\end{figure}
%		\vspace{4mm}
%		\begin{itemize}
%			%		\item<1-> efficient allocation same in both models
%			%		\item<2-> reduces as households prefer polluting good less
%			
%			\item<1-> the planner forfeits an efficient reduction of the externality due to inequality
%			\vspace{2mm}
%		%	\item<1->  basic needs prevent a reduction of the output ratio closer to the efficient level
%			%		\vspace{2mm}
%		\end{itemize}
%		\vspace{1mm}
%		\hfill		
%		\hyperlink{conc}{\tiny{$\rightarrow$ conclusion,}}
%		\hyperlink{effallo}{\tiny{$\rightarrow$ allocation,}} \hyperlink{backpol}{\tiny{$\rightarrow$ policy focus,}}
%		\hyperlink{skipallo}{\tiny{$\rightarrow$ standard model,}}
%		\hypertarget{backeff}{}
%		\hyperlink{moreaggtaun}{\tiny{$\rightarrow$ counterfactual}}
%	\end{frame}
%
%	\addtocounter{framenumber}{-1}

\begin{frame}{Efficient allocation}
	
%	\vspace{4mm}
	\begin{figure}
		\begin{minipage}[]{0.45\textwidth}
			\centering{\footnotesize{ Unsustainable output, $Y_n$ }}
			%	\captionsetup{width=.45\linewidth}
			\includegraphics[width=1\textwidth]{../codding_new/policies_solution/solutions_paper/Compariosn_socialPlanner_pen1_targetext_2_yn_lgd1_notauun0_pslow0_PSID1_matchGDP0_phii12.png}
		\end{minipage}
%		\begin{minipage}[]{0.32\textwidth}
%			\centering{\footnotesize{ Output ratio, $Y_n/Y_s$ \\ \  }}
%			%	\captionsetup{width=.45\linewidth}
%			\includegraphics[width=1\textwidth]{../codding_new/policies_solution/solutions_paper/Compariosn_socialPlanner_pen1_targetext_2_output_ratio_lgd0_notauun0_pslow0_PSID1_matchGDP0_phii12.png}
%		\end{minipage}
%		\begin{minipage}[]{0.32\textwidth}
%			\centering{\footnotesize{ Output ratio, $Y_n/Y_s$, \\ without basic needs}}
%			%	\captionsetup{width=.45\linewidth}
%			\includegraphics[width=1\textwidth]{../codding_new/policies_solution/solutions_paper/Compariosn_socialPlanner_pen0_targetext_2_output_ratio_lgd0_notauun0_pslow0_PSID1_matchGDP0_phii12.png}
%		\end{minipage}
%		
		%\begin{minipage}[]{0.32\textwidth}
		%	\centering{\footnotesize{ Unsustainable output, $y_n$, \\ baseline model}}
		%	%	\captionsetup{width=.45\linewidth}
		%	\includegraphics[width=1\textwidth]{../codding/policies_solution/solutions_paper/Compariosn_socialPlanner_pen1_targetext_2_yn_lgd0_notauun0_pslow0_PSID1_matchGDP0_phii15.png}
		%\end{minipage}
	\end{figure}
	\vspace{4mm}
	\begin{itemize}
		%		\item<1-> efficient allocation same in both models
		%		\item<2-> reduces as households prefer polluting good less
		
		\item<1-> the Ramsey planner forfeits an efficient reduction of the externality due to inequality and basic needs
		%\vspace{2mm}
		%\item<2-> basic needs prevent a reduction of the output ratio closer to the efficient level
		%		\vspace{2mm}
	\end{itemize}
	\vspace{-2mm}
	\hfill		
	\hyperlink{conc}{\tiny{$\rightarrow$ conclusion,}}
	\hyperlink{effallo}{\tiny{$\rightarrow$ allocation,}} \hyperlink{backpol}{\tiny{$\rightarrow$ policy focus,}}
	\hyperlink{skipallo}{\tiny{$\rightarrow$ standard model,}}
	\hypertarget{backeff}{}
	\hyperlink{moreaggtaun}{\tiny{$\rightarrow$ counterfactual}}
\end{frame}

	
%	\section*{Decomposing income taxes}
	\begin{frame}{Decomposing income taxes} 
		%	\tr{@Pavel: I will drop this slide and explain the experiment wh}
		\alert{\textbf{How much of the income tax is explained by equity and how much by the externality motive?}}
		\pause
		\vspace{3mm}
		\begin{itemize}
			\item<+-> \textbf{Problem:} corrective tax changes costs and benefits of redistribution %\\ \ar difference of income tax in full model to income tax in model without externality not solely due to externality!
			\vspace{3mm}
			\item[\ar]<+-> include optimal corrective tax as a parameter in the model without externality  and solve for the optimal income tax %\\ \ar size of labour tax only driven by equity concerns % due to externality and effect of corrective tax
			\vspace{3mm}
			\item<+-> difference to full model due to externality 
		\end{itemize}
	\end{frame}
	
	\begin{frame}{Optimal income tax: no externality}
		\hypertarget{backLF}{}
		%	\vspace{1mm}
		\begin{center}
			\begin{minipage}[]{0.45\textwidth}
				\centering{{Income tax, $\tau_l$}}
				%	\captionsetup{width=.45\linewidth}
				\includegraphics[width=1\textwidth]{../codding_new/policies_solution/solutions_paper/tauunfixed_pen1_targetext_2_tauul_onlynoext_pslow0_PSID1_matchGDP0_phii12_lgd1_presentation.png}
			\end{minipage}	
		\end{center}
		\vspace{3mm}
		\begin{itemize}
			\item more redistribution when social responsibility is high to avoid poverty and rising consumption inequality
		\end{itemize}
		%\begin{itemize}
		%	\item comparison to full model optimal income tax can be driven by both equity or corrective motive due to different corrective tax
		%	\item
		%\end{itemize}
		\vspace{-1mm}
		\hfill
		\hyperlink{LF}{\tiny{$\rightarrow$ Laissez-faire}}
	\end{frame}
	
	
	\addtocounter{framenumber}{-1}
	\begin{frame}{Optimal income tax: $\tau_n$ as parameter}
		%	\vspace{1mm}
		\begin{center}
			\begin{minipage}[]{0.45\textwidth}
				\centering{{Income tax, $\tau_l$}}
				%	\captionsetup{width=.45\linewidth}
				\includegraphics[width=1\textwidth]{../codding_new/policies_solution/solutions_paper/tauunfixed_pen1_targetext_2_tauul_notfullmodel_pslow0_PSID1_matchGDP0_phii12_lgd1_presentation.png}
			\end{minipage}
		\end{center}
		\vspace{3mm}
		\begin{itemize}
			\item corrective tax revenues used to lower income tax, when $\omega$ is low 
			\vspace{2mm}
			\item $\tau_n$ regressive, when the sustainable good is more expensive \ar higher labour tax
		\end{itemize}
	\end{frame}
	
	
	\addtocounter{framenumber}{-1}
	\begin{frame}{Optimal income tax: with externality}
		
		\begin{center}
			\begin{minipage}[]{0.45\textwidth}
				\centering{{Income tax, $\tau_l$}}
				%	\captionsetup{width=.45\linewidth}
				\includegraphics[width=1\textwidth]{../codding_new/policies_solution/solutions_paper/tauunfixed_pen1_targetext_2_tauul_withnoext_pslow0_PSID1_matchGDP0_phii12_lgd1_presentation.png}
			\end{minipage}
		\end{center}
		%\begin{minipage}[]{0.45\textwidth}
		%	\centering{{Income tax, $\tau_l$}}
		%	%	\captionsetup{width=.45\linewidth}
		%	\includegraphics[width=1\textwidth]{../codding_new/policies_solution/solutions_paper/tauunfixed_poorPoorer130_pen1_targetext_2_yn_withnoext_pslow0_PSID1_matchGDP0_phii12_lgd1_presentation.png}
		%\end{minipage}
		\vspace{2mm}
		\begin{itemize}
			\item<1-> income tax optimally used as a corrective policy tool for all levels of $\omega$
			\vspace{2mm}
			%\item externality motive adds between 30 and 2.5 percentage points
			\item<1-> use of income tax to target the externality persists without basic needs %because inequality makes a lower corrective tax optimal, the planner chooses a higher income tax to target the externality
		\end{itemize}
		\vspace{-2mm}
		\hfill
		\hyperlink{noBN}{\tiny{$\rightarrow$ standard model,}} 
		\hyperlink{3exp}{\tiny{$\rightarrow$ quantification,}} 
		\hyperlink{sensi}{\tiny{$\rightarrow$ sensitivity}}
		\hypertarget{backpe}{}
	\end{frame}
	
	
	
	%%	\addtocounter{framenumber}{-1}
	%\begin{frame}{Main result: Redistribution as corrective policy instrument}
	%	%\hypertarget{backpe}{}
	%	\vspace{2mm}
	%	\centering
	%	\begin{minipage}[]{0.45\textwidth}
	%		\centering{{Policy effects}}
	%		%	%	\captionsetup{width=.45\linewidth}
	%		\includegraphics[width=1\textwidth]{../codding/policies_solution/solutions_paper/contribs_sep_tauun_quant_redistribution_pen1_targetext_2_ext1_yn_pslow0_PSID1_matchGDP0_phii15_lgd1_presentation.png}
	%	\end{minipage}	
	%	\begin{minipage}[]{0.45\textwidth}
	%		\ 
	%		%		\centering{{Labour tax, $\tau_l$}}
	%		%		%	\captionsetup{width=.45\linewidth}
	%		%		\includegraphics[width=1\textwidth]{../codding/policies_solution/solutions_paper/tauunfixed_pen1_targetext_2_tauul_withouttauunfixed_notauun0_pslow0_PSID1_matchGDP0_presentation.png}
	%	\end{minipage}
	%	\vspace{4mm}
	%	%\begin{itemize} 
	%	%	\item redistribution becomes the central pillar of corrective policy
	%	%\item<2-> reliance on redistribution due to intensified severity of inequality
	%	%\end{itemize}
	%\end{frame}
	%
	%\addtocounter{framenumber}{-1}
	%\begin{frame}{Main result: Redistribution as corrective policy instrument}
	%	\hypertarget{backpe}{}
	%	\vspace{2mm}
	%	\centering
	%	\begin{minipage}[]{0.45\textwidth}
	%		\centering{{Policy effects}}
	%		%	%	\captionsetup{width=.45\linewidth}
	%		\includegraphics[width=1\textwidth]{../codding/policies_solution/solutions_paper/contribs_tauun_red_quant_redistribution_pen1_targetext_2_ext1_yn_pslow0_PSID1_matchGDP0_phii15_lgd1_presentation.png}
	%	\end{minipage}
	%	\begin{minipage}[]{0.45\textwidth}
	%		\ 
	%		%		\centering{{Labour tax, $\tau_l$}}
	%		%		%	\captionsetup{width=.45\linewidth}
	%		%		\includegraphics[width=1\textwidth]{../codding/policies_solution/solutions_paper/tauunfixed_pen1_targetext_2_tauul_withouttauunfixed_notauun0_pslow0_PSID1_matchGDP0_presentation.png}
	%	\end{minipage}
	%	\vspace{4mm}
	%	\begin{itemize} 
	%		\item redistribution becomes the central pillar of corrective policy
	%		%\item<2-> reliance on redistribution due to intensified severity of inequality
	%	\end{itemize}
	%	
	%	\vspace{2mm}
	%	\hfill
	%	%	\hyperlink{labt}{\tiny{$\rightarrow$ labour tax}}\\
	%	%	\vspace{-2mm}
	%	%		\hfill
	%	\hyperlink{eff}{\tiny{$\rightarrow$ efficiency channel}}\\
	%	\vspace{-2mm}
	%	\hfill
	%	\hyperlink{wtp}{\tiny{$\rightarrow$ policy shift}}	
	%\end{frame}
	%\addtocounter{framenumber}{-1}
	%\begin{frame}{Main result: Redistribution as corrective policy instrument}
	%	\hypertarget{backpe}{}
	%	\centering
	%	\vspace{2mm}
	%	\begin{minipage}[]{0.45\textwidth}
	%		\centering{{Policy effects}}
	%		%	%	\captionsetup{width=.45\linewidth}
	%		\includegraphics[width=1\textwidth]{../codding_new/policies_solution/solutions_paper/contribs_tauun_red_quant_redistribution_pen1_targetext_2_ext1_yn_pslow0_PSID1_matchGDP0_phii12_lgd1_presentation.png}
	%	\end{minipage}
	%	\begin{minipage}[]{0.45\textwidth}
	%		\centering{{Labour tax, $\tau_l$}}
	%		%	\captionsetup{width=.45\linewidth}
	%		\includegraphics[width=1\textwidth]{../codding/policies_solution/solutions_paper/tauunfixed_pen1_targetext_2_tauul_withouttauunfixed_notauun0_pslow0_PSID1_matchGDP0_presentation.png}
	%	\end{minipage}
	%	\vspace{4mm}
	%	\begin{itemize} 
	%		\item redistribution becomes the central pillar of corrective policy
	%		\vspace{2mm}
	%		\item labour tax chosen higher than absent externality
	%	\end{itemize}
	%	\vspace{-7mm}
	%	\hfill
	%	\hyperlink{labt}{\tiny{$\rightarrow$ labour tax}}\\
	%	\vspace{-2mm}
	%	\hfill
	%	\hyperlink{eff}{\tiny{$\rightarrow$ efficiency channel}}\\
	%	\vspace{-2mm}
	%	\hfill
	%	\hyperlink{wtp}{\tiny{$\rightarrow$ policy shift}}	
	%\end{frame}
	%
	%-----------Conclusion -----------------------
	\hypertarget{conc}{}
	\section{Conclusion}
	
	\begin{frame}{Conclusion}
		
		\begin{itemize}[<+-| alert@+>]
			\setbeamercolor{alerted text}{fg=black} %change the font color
			\setbeamerfont{alerted text}{series=\bfseries}
			\item How should the optimal policy react if we all became  more socially responsible?
			\vspace{4mm}
			%			\item Assume we all want to consume more socially responsibile, then, inequality prevents an efficient reduction of the externality under the optimal policy when the productivity gap is high. 
			\item The optimal policy shifts away from corrective taxes to redistribution. % since inequality aggravates. %as the sustainable good gets more preferred,
		%	and redistribution becomes the central pillar of the corrective policy.
			\vspace{4mm}
			\item Inequality aggravates with social responsibility. Therefore, the government forfeits an efficient reduction in the externality. % as social responsibility rises %associated with social responsibility
			\vspace{4mm}
			\item The income tax is used to lower the externality for all levels of social responsibility due to inequality.
			%			\item \textcolor{myorange}{\textbf{Timing of changes in social responsibility (generational, within generations) becomes relevant when endogenising growth}}
		\end{itemize}
	\end{frame}
	
	\appendix
	

	
	\begin{frame}[allowframebreaks]{References}
		
		\bibliography{../../bib_2_0}
		\bibliographystyle{apa}
	\end{frame}
	
	
	
	\begin{frame}{Social responsibility}
		\hypertarget{rise1}{}
		\begin{itemize}
			\vspace{4mm}
			\item<1->  \cite{Bartling2015DoResponsibility} provide experimental evidence for the existence of social responsibility in markets 
			%				\vspace{2mm}
			%				\item<+-> 
			%				global share of households willing to pay a
			%				premium rose from 50\% in 2013 to 66\% in 2015 \citep{NielsenSUSIMp} %(sample of 60 countries)
			\vspace{2mm}
			\item<1-> in the US, the market share of sustainable consumer-packaged goods rose from 14\% in 2013 to 16\% in 2018 despite a price premium \citep{Kronthal-Sacco2020SustainableMessages}%; faster growth in product sales with sustainability claim (March 2017-March2018) \citep{NielsenSUS2018}
			\vspace{2mm} 
			\item<1-> accepted price premium on average: 25\% %(multi-country)
			\citep{SK2021}
		\end{itemize} % \ar
		\hfill
		\hyperlink{backmov}{\tiny{$\rightarrow$ back}}
	\end{frame}

\begin{frame}{Income dependent support for costly policy}
	\hypertarget{rise2}{}
	\vspace{-4mm}
	\begin{figure}
		\includegraphics[width=.55\textwidth]{../codding/calibration/emp_results/reg_coal_emissions_inc_present.png}
	\end{figure}
	\vspace{-3mm}
	\tiny{Source: \cite{CCAM}; ``\textit{How much do you support or oppose the
			following policy?
			Set strict carbon dioxide emission limits on
			existing coal-fired power plants to reduce
			global warming and improve public health.
			Power plants would have to reduce their
			emissions and/or invest in renewable energy
			and energy efficiency. The cost of electricity to
			consumers and companies would likely
			increase}"}
	\vspace{-9mm}
	\hfill	\hyperlink{backmov}{\tiny{$\rightarrow$ back}}
\end{frame}
	\begin{frame}{In the US}
			\hypertarget{indis}{}
		\vspace{-5mm}
	\begin{minipage}[]{1\textwidth}
		\begin{figure}
			%		%	\caption{Distribution of per-capita disposable income in 2018 }\label{fig:poverty}	
			%			
			%			%	\captionsetup{width=.45\linewidth}
			\includegraphics[width=.7\textwidth]{../codding_new/calibration/emp_results/Poster_histogramme_prices_estimatedpdf_PSID.png}	
		\end{figure}
		\centering
		\tiny{Sources: Disposable Income: PSID, TAXSIM; Basic Needs: Institute for Women's Policy Research, Prices: USDA}
		%		
	\end{minipage}
	\vspace{-10mm}
	\hfill	\hyperlink{backmov}{\tiny{$\rightarrow$ back}}
\end{frame}

	\begin{frame}{}
		\hypertarget{Contrib}{}
		\vspace{4mm}
		\begin{block}{Contribution to the literature}
			impact of \textbf{\alert{social responsibility}} on  (1) \textbf{\alert{optimal policy}} in an (2) \textbf{\alert{unequal economy}}
		\end{block}
		\begin{itemize}
			\item<1-> \textbf{social responsibility} in behavioural economics \citep{Benabou2010IndividualResponsibility, Bartling2015DoResponsibility, Falk2021FightingValues}; a macro example: \cite{Aghion2021EnvironmentalDirty}\\ \ar basic needs %as an obstacle
			%			\vspace{2mm}
			%			\item<1-> macro models on \textbf{optimal climate policy} mainly focus on supply side \\ \citep[e.g.][]{Golosov2014OptimalEquilibrium, Acemoglu2016TransitionTechnology}%; \textit{no demand-side perspective, no role for redistribution}
			\vspace{2mm}
			\item<1-> optimal \textbf{corrective policy in distortionary fiscal setting}
			
			%(\textit{re-evaluation optimal-Pigouvian gap})% \ar add inequality and motive for lower env. tax and more redistribution%\ar env. tax falls short of Pigouvian rate if gov. has motive to generate funds % and cannot use transfers
			\begin{itemize}
				\vspace{1mm}
				\item<1->  with representative agent \citep[e.g.][]{ LansBovenberg1994EnvironmentalTaxation, Barrage2019OptimalPolicy} 
				\item<1-> \cite{Vona2011IncomeTechnologies, Jacobs2019RedistributionCurves} role of redistribution due to non-linear Engel curves  \ar non-linearity as a function of social responsibility
			\end{itemize}
			\item<1-> \textbf{structural transformation} \\ \citep{Herrendorf2014GrowthTransformation, Matsuyama2002TheSocieties, Foellmi2008StructuralGrowth, Boppart2014StructuralPreferences}
		\end{itemize}
		%\vspace{2mm}
		%		\hfill	\hyperlink{backCont}{\tiny{$\rightarrow$ back}}
	\end{frame}
	
		\section{Empirical Backup}
	
	\begin{frame}{Social Responsibility: homogeneous across income groups}
		\hypertarget{atts}{}
		\vspace{0mm}
		%		\begin{itemize}
		%			%	\item<1-> climate change raises the need for a transition to sustainable production
		%			%	\vspace{3mm}
		%			\item \textcolor{black}{\textbf{Claim: $\pmb{\omega}$ is }}
		%		\end{itemize}
		%\tr{env. attitudes as main driver of social responsibility\ar similar rise across income groups and intensive margin more relevant}
		%\pause
		\vspace{2mm}
		\centering
		\begin{minipage}[]{0.6\textwidth}
			\centering{\footnotesize{Share worried by climate change}}
			%	\captionsetup{width=.45\linewidth}
			\includegraphics[width=0.9\textwidth]{../codding/calibration/emp_results/worry_inc_present.png}
		\end{minipage}
		
		\tiny{Source: \cite{CCAM}}
		\vspace{-9mm}
		\hfill	\hyperlink{backAtt}{\tiny{$\rightarrow$ back}}
	\end{frame}
	
	%	\addtocounter{framenumber}{-1}
	
	
	%	\begin{frame}{Social Responsibility}
	%		\vspace{0mm}
	%		\begin{itemize}
	%			
	%			%	\item<1-> climate change raises the need for a transition to sustainable production
	%			%	\vspace{3mm}
	%			\item Claim 1: $\pmb{\omega}$ is homogeneous across income groups
	%			\item \textcolor{black}{\textbf{Claim 2: $\pmb{\omega}$ rises on the intensive margin}}
	%		\end{itemize}
	%		%\tr{env. attitudes as main driver of social responsibility\ar similar rise across income groups and intensive margin more relevant}
	%		\vspace{5mm}
	%		\centering
	%		%	\pause
	%		\begin{minipage}[]{0.45\textwidth}
	%			\centering{\footnotesize{Share strongly worried}}
	%			%	\captionsetup{width=.45\linewidth}
	%			\includegraphics[width=\textwidth]{../codding/calibration/emp_results/str_support_worry_inc_present.png}
	%		\end{minipage}
	%		\begin{minipage}[]{0.45\textwidth}
	%			\centering{\footnotesize{Share weakly worried}}
	%			%	\captionsetup{width=.45\linewidth}
	%			\includegraphics[width=\textwidth]{../codding/calibration/emp_results/sw_support_worry_inc_nolgd_present.png}
	%		\end{minipage}
	%		
	%		\tiny{Source: \cite{CCAM}}
	%		\vspace{-9mm}
	%		\hfill	\hyperlink{backAtt}{\tiny{$\rightarrow$ back}}
	%	\end{frame}
	
	
	\begin{frame}{Decomposition policy support}
		\vspace{4mm}
		\begin{minipage}[]{0.47\textwidth}
			\centering{\footnotesize{  \textbf{Strong} support}}
			\includegraphics[width=\textwidth]{../codding/calibration/emp_results/str_support_reg_coal_emissions_inc_present.png}
		\end{minipage}
		\begin{minipage}[]{0.47\textwidth}
			\centering{\footnotesize{  \textbf{Weak} support}}
			\includegraphics[width=\textwidth]{../codding/calibration/emp_results/sw_support_reg_coal_emissions_inc_nolgd_present.png}
		\end{minipage}
		
		\vspace{0mm}
		\tiny{Source: \cite{CCAM}}
		
		\vspace{8mm}
		\hfill	\hyperlink{wtp}{\tiny{$\rightarrow$ back }}
	\end{frame}
	
	\section{Model behaviour}
	\begin{frame}{Engel Curves}
		\hypertarget{engel}{}
		\begin{figure}[h!!]
			\centering
			\begin{minipage}[h!!]{0.39\textwidth}  
				%	\captionsetup{width=.45\linewidth}
				\centering\textbf{ $p_s=1.56; \ \omega=0.9$ }
				\includegraphics[width=1\textwidth]{../codding_new/calibration/calib_results/Demand_fullandstandard_PSID1_ws0.90_ps1.56_nozero_phii12_lgd0_presentation_labmatchedaverage.png}
			\end{minipage}
			\begin{minipage}[h!!]{0.39\textwidth}  
				%	\captionsetup{width=.45\linewidth}
				\centering\textbf{ $p_s=1.56;\ \omega=0.24$ }
				\includegraphics[width=1\textwidth]{../codding_new/calibration/calib_results/Demand_fullandstandard_PSID1_ws0.24_ps1.56_nozero_phii12_lgd0_presentation_labmatchedaverage.png}
			\end{minipage}
			\begin{minipage}[h!!]{0.03\textwidth}  
			\end{minipage}
			\begin{minipage}[h!!]{0.19\textwidth}  
				%	\captionsetup{width=.45\linewidth}
				%\centering\textbf{ $p_s=0.64;\ \omega_s=0.26$ }
				\includegraphics[width=1\textwidth]{../codding/calibration/calib_results/legend_EC_pre.png}
			\end{minipage}
		\end{figure}
		
		
		\hfill 
		\hyperlink{backengel}{\tiny{$\rightarrow$ back}}
	\end{frame}

%%%%
\section{Model}

\begin{frame}{Model: {Ramsey planner}}
	
	\vspace{8mm}
%	\textbf
	\begin{align*}
	\underset{\{\tau_n, \tau_l\}}{\max} \ \lambda U_{r}+(1-\lambda) U_{p}%-\psi h_n^\eta
	\end{align*}
	\vspace{-10mm}
	\begin{align*}
	s.t. \hspace{10mm}& (1)\  T=\tau_l wH+\tau_nwH_{n}\\
	&	(2)\  \text{behaviour of firms and households}\\
	&(3) \ \text{feasibility}\\
	& (4)\ H =\lambda z_h l_r+(1-\lambda)z_l l_p
	\end{align*}
	%\item[-] solved using primal approach
	%\item[-] markets for individual goods and labour clear
	
	%\vfill	\hfill	\hyperlink{pa}{\tiny{$\rightarrow$ primal approach}}
	%		\pause 
	%		\textbf{Markets for goods and labour clear}\\ 
	\vspace{6mm}
	\hfill
	\hyperlink{backmodel}{\tiny{$\rightarrow$ back}}
\end{frame}

	%%%CALIBRATION%%%
\section{Calibration}

\begin{frame}{Calibration overview}
	
	\begin{itemize}
		\item<+-> calibration to the US in 2018
		\item<+-> I proceed in three steps
		\vspace{4mm}
		\begin{enumerate}[<+-| alert@+>]\setbeamercolor{alerted text}{fg=black} %change the font color
			\item separately match $\lambda$, $\bar{c}$, $\phi$, $\omega$, $\theta$, $L$, $\tau_l$, and $\tau_n$
			\vspace{2mm}
			\item jointly calibrate $A_n$, $A_s$, $\chi$, $z_h$, $z_l$% so that equilibrium equations and target equations hold
			\vspace{2mm}
			\item calibrate parameters governing the externality, $\eta$ and $\psi$
		\end{enumerate}
	\end{itemize}
\end{frame}

\begin{frame}{Calibration I}
	\hypertarget{red}{}
	\vspace{-3mm}
	\begin{center}
		\resizebox{!}{.75\height}{
			%	\begin{table}[hh!!]
			
			%	\caption{ Calibration of the baseline model }
			\begin{tabular}{c|cll}
				%			\hline \hline
				%			\multicolumn{7}{c}{Calibration based on basic needs}\\
				\hline \hline
				Parameter& Calibrated value& Meaning&Target/Source\\ \hline
				\hline
				$\phi$ & 12 & importance of basic needs& - \\
				\hline
				$\sigma$&1.71& \makecell[l]{governs price elasticity\\ of substitution} &  \makecell[l]{price elasticities in US milk market \\ \cite{Chen2018OrganicMilk}} \\
				\hline
				\alert{$\pmb{\omega}$}&\alert{\textbf{0.24}}& \makecell[l]{governs \\ social responsibility}&  \makecell[l]{market share of sustainable goods (cpg)\\  \cite{Kronthal-Sacco2020SustainableMessages}}\\			
				\hline
				\alert{$\pmb{\bar{c}}$}&\alert{\textbf{1}}&basic needs, normalised& \makecell[l]{in US\$: 25,128\$\\  \cite{BESt}}  \\
				\hline
				\alert{$\pmb{\lambda}$}&\alert{\textbf{0.56}}& share of rich households& \makecell[l]{can cover basic needs with sustainable goods alone\\  prices from USDA,\\ food bundle from \cite{FoodCommission}\\ Income from PSID, TAXSIM}\\
				%\hline
				%$\beta$&\makecell{ annual nominal rate 3\%\\ and annual inflation rate of 2\%}& 0.9903& discount factor\\ 
				\hline
				L&1&\makecell[l]{annual time endowment,\\ normalised} &  14.5 hours per day, \cite{Jones1993OptimalGrowth}\\
				\hline
				$\theta$& 0.75&Frisch elasticity& \cite{Chetty2011AreMargins} \\
				\hline	
				$\tau_l$&0.24&labour income tax & \cite{Barrage2019OptimalPolicy}\\
				\hline
				$\tau_n$ &0&corrective tax& - \\
				\hline \hline
			\end{tabular}
			%	\end{table}
		}
	\end{center}
\end{frame}

\begin{frame}{Calibration II}
	\hypertarget{backcalib}{}
	\vspace{-5mm}	
	\begin{center}
		\resizebox{!}{0.75\height}{
			%	\begin{table}[hh!!]
			
			%	\caption{ Calibration of the baseline model }
			\begin{tabular}{c|cll}
				%			\hline \hline
				%			\multicolumn{7}{c}{Calibration based on basic needs}\\
				\hline \hline
				Parameter& Calibrated value& Meaning&Target/Source\\ \hline
				\alert{$\pmb{z_l}$}&\alert{\textbf{0.03}}&effective labour productivity poor& \makecell[l]{average  income poor (PSID): \\0.68 basic-needs bundles}\\
				\hline
				\alert{$\pmb{z_h}$}&\alert{\textbf{2.13}}& effective labour productivity rich& \makecell[l]{difference average income\\ poor and GDP p.c.: 4.00 basic-needs bundles}\\
				\hline	
				$\chi$&23.51&disutility from labour& \makecell[l]{average annual labour supply per \\ worker worked: 34.29 per week\\ \cite{OECDHoursworked}}\\
				\hline
				$A_n$&8.62&TFP unsustainable sector&\makecell[l]{GDP p.c.: 63,043\$;\\ 2.5 basic-needs bundles (OECD)} \\
				\hline
				$ A_s$&5.52&TFP sustainable sector& relative price of sustainable food bundle: 1.56\\ 
				&&& USDA, \cite{FoodCommission}\\ 
				\hline
				$\eta$&1.34&curvature externality& rich willing to give up 2\% of annual con-\\
				$\psi$ &9.98&weight on externality& sumption for 1\% reduction in $H_n$ at baseline\\
				\hline \hline
			\end{tabular}
			%	\end{table}
		}
	\end{center}
	\vspace{-1mm}
	\hfill
	\hyperlink{backmodel}{\tiny{$\rightarrow$ back}}
	
\end{frame}	
	
	\begin{frame}{}
		\hypertarget{tabIncome}{}
		\vspace{20mm}
		\centering
		\begin{tabular}{l|rrr}
			\multicolumn{4}{c}{\textbf{	Average annual income per capita in 2018}}\\
			\hline \hline
			Variable &poor&   rich&    total\\     
			\hline \hline 
			in US\$  &    17,249&  67,330 &   45,083\\
			\hline
			in basic needs&&&\\
			unsustainable prices  & 0.69     &     2.68   &  1.79\\
			\hline
			in basic needs&&&\\
			sustainable prices& 0.56        &    2.19  & 1.47\\
			\hline \hline
			\multicolumn{4}{l}{\tiny{Sources: PSID, TAXSIM}}
		\end{tabular}
		
		\vspace{13mm}
		\hfill
		\hyperlink{backmodel}{\tiny{$\rightarrow$ back}}
	\end{frame}
	
	
	
	%\begin{frame}{Willingness to pay and the role of redistribution}
	%	\hypertarget{wtp}{}
	%	\centering
	%	\begin{minipage}[]{0.55\textwidth}
	%		%	\captionsetup{width=.45\linewidth}
	%		\includegraphics[width=1\textwidth]{../codding/policies_solution/solutions_paper/fullmodel_WTP_pen1_targetext_2_withps1_lgd_notauun0_pslow0_PSID1_matchGDP0_phii15.png}
	%	\end{minipage}
	%\vspace{2mm}
	%	\begin{itemize}
	%		\item the shift in the corrective policy to redistribution is optimal when the willingness to pay equals 1
	%	\end{itemize}
	%	
	%	\vspace{0mm}
	%	\hfill
	%	\hyperlink{backpe}{\tiny{$\rightarrow$ back}}
	%\end{frame}
	
	\section{Additional results}
	%---- optimal allocation
		\begin{frame}{Optimal allocation}
		%\tr{this one and next slide to be moved to appendix (partially); this slide: comparison baseline model with basic needs to model without basic needs}
		\vspace{-5mm}
		\hypertarget{backallo}{}
		\begin{figure}			
			\begin{minipage}[]{0.32\textwidth}
				\centering{\footnotesize{Externality, $Y_n$\\ \ }}
				%	\captionsetup{width=.45\linewidth}
				\includegraphics[width=1\textwidth]{../codding_new/policies_solution/solutions_paper/fullmodel_pen1_targetext_2_yn_withps0_lgd0_notauun0_pslow0_PSID1_matchGDP0_phii12.png}
			\end{minipage}	
			\begin{minipage}[]{0.32\textwidth}
				\centering{\footnotesize{Output\\ \ }}
				%	\captionsetup{width=.45\linewidth}
				\includegraphics[width=1\textwidth]{../codding_new/policies_solution/solutions_paper/fullmodel_pen1_targetext_2_agg_output_withps0_lgd0_notauun0_pslow0_PSID1_matchGDP0_phii12.png}
			\end{minipage}
			\begin{minipage}[]{0.32\textwidth}
				\centering{\footnotesize{Gini of consumption\\ \ }}
				%		%	\captionsetup{width=.45\linewidth}
				\includegraphics[width=1\textwidth]{../codding_new/policies_solution/solutions_paper/fullmodel_pen1_targetext_2_Gini_consumption_withps1_lgd0_notauun0_pslow0_PSID1_matchGDP0_phii12.png}
			\end{minipage}
		\end{figure}
		\begin{itemize}
			\item reduced externality at higher output
			\item inequality rises
			%	\item<2-> I will argue: higher output is a choice to mitigate inequality
		\end{itemize}
		\vspace{-1mm}
		\hfill
		\hyperlink{backopt}{\tiny{$\rightarrow$ back}}
	\end{frame}	
	
	%--------Efficient allocation -----
	\begin{frame}{Efficient allocation}
		\hypertarget{effallo}{}
		\begin{minipage}[]{0.32\textwidth}
			\centering{\footnotesize{ Externality, $Y_n$\\ \ }}
			%	\captionsetup{width=.45\linewidth}
			\includegraphics[width=1\textwidth]{../codding_new/policies_solution/solutions_paper/socialPlanner_pen1_ext1_targetext_2_yn_nolgd_pslow0_PSID1_matchGDP0_phii12_labmatchedaverage.png}
		\end{minipage}
		\begin{minipage}[]{0.32\textwidth}
			\centering{\footnotesize{ Output\\ \ }}
			%	\captionsetup{width=.45\linewidth}
			\includegraphics[width=1\textwidth]{../codding_new/policies_solution/solutions_paper/socialPlanner_pen1_ext1_targetext_2_agg_out_nolgd_pslow0_PSID1_matchGDP0_phii12_labmatchedaverage.png}
		\end{minipage}
		%	\begin{minipage}[]{0.32\textwidth}
		%		\centering{(c) Output }
		%		%	\captionsetup{width=.45\linewidth}
		%		\includegraphics[width=1\textwidth]{../codding_new/policies_solution/solutions_paper/socialPlanner_pen1_ext1_targetext_2_agg_out_nolgd_pslow0_PSID1_matchGDP0_phii12_labmatchedaverage.png}
		%	\end{minipage}
		\begin{minipage}[]{0.32\textwidth}
			\centering{\footnotesize{ Composite consumption rich/ poor}}
			%	\captionsetup{width=.45\linewidth	
			\includegraphics[width=1\textwidth]{../codding_new/policies_solution/solutions_paper/socialPlanner_pen1_ext1_targetext_2_Cr_nolgd_pslow0_PSID1_matchGDP0_phii12_labmatchedaverage.png}
		\end{minipage}
	\vspace{4mm}
		\begin{itemize}
			\item trade-off between consumption and pollution loses intensity as social responsibility rises: \ar higher composite consumption and lower unsustainable production 
			\item disutility from labour exceeds utility from consumption when $\omega$ is very high
			\item no inequality
		\end{itemize}
		\vspace{-3mm}
		\hfill
		\hyperlink{backeff}{\tiny{$\rightarrow$ back}}
	\end{frame}	
	
	
	\begin{frame}{Policy effect}
		\hypertarget{backpol}{}
		\begin{figure}
			\begin{minipage}[]{0.32\textwidth}
				\centering{Externality, $Y_n$}
				%	\captionsetup{width=.45\linewidth}
				\includegraphics[width=1\textwidth]{../codding_new/policies_solution/solutions_paper/total_impact_modelcomp_targetext_2_ext1_yn_pslow0_PSID1_matchGDP0_phii15_lgd1.png}	
			\end{minipage}
			\begin{minipage}[]{0.32\textwidth}
				\centering{Output}
				%	\captionsetup{width=.45\linewidth}
				\includegraphics[width=1\textwidth]{../codding_new/policies_solution/solutions_paper/total_impact_modelcomp_targetext_2_ext1_agg_output_pslow0_PSID1_matchGDP0_phii15_lgd0.png}	
			\end{minipage}
			\begin{minipage}[]{0.32\textwidth}
				\centering{{ Gini of consumption}}
				%	\captionsetup{width=.45\linewidth}
				\includegraphics[width=1\textwidth]{../codding_new/policies_solution/solutions_paper/total_impact_modelcomp_targetext_2_ext1_Gini_consumption_pslow0_PSID1_matchGDP0_phii15_lgd0.png}
			\end{minipage}
			
		\end{figure}
		\vspace{4mm}
		\begin{itemize}
			\item with basic needs, the policy focus shifts away from the externality to inequality
			\vspace{2mm}
			\item inequality explains shift to redistribution
		\end{itemize}
		\vspace{9mm}
		\hfill
		\hyperlink{backeff}{\tiny{$\rightarrow$ back}}
	\end{frame}
	
	
	\begin{frame}{Counterfactual Policy: More aggressive corrective tax}
		%	\vspace{-5mm}
		\hypertarget{moreaggtaun}{}
		\begin{figure}	
			
			\begin{minipage}[]{0.32\textwidth}
				\centering{\footnotesize{ Output ratio, $y_n/y_s$}}
				%	\captionsetup{width=.45\linewidth}
				\includegraphics[width=1\textwidth]{../codding_new/policies_solution/solutions_paper/Compariosn_socialPlanner_counterfac_pen1_targetext_2_output_ratio_lgd1_notauun0_pslow0_PSID1_matchGDP0_phii12.png}
			\end{minipage}		
			\begin{minipage}[]{0.32\textwidth}
				\centering{\footnotesize{Externality, $Y_n$}}
				%	\captionsetup{width=.45\linewidth}
				\includegraphics[width=1\textwidth]{../codding_new/policies_solution/solutions_paper/counterfac_pen1_targetext_2_yn_pslow0_lgd1_PSID1_matchGDP0_phii12.png}
			\end{minipage}	
			\begin{minipage}[]{0.32\textwidth}
				\centering{\footnotesize{Output}}
				%	\captionsetup{width=.45\linewidth}
				\includegraphics[width=1\textwidth]{../codding_new/policies_solution/solutions_paper/counterfac_pen1_targetext_2_agg_output_pslow0_lgd0_PSID1_matchGDP0_phii12.png}
			\end{minipage}	\begin{minipage}[]{0.32\textwidth}
				\centering{\footnotesize{Transfers, T}}
				%		%	\captionsetup{width=.45\linewidth}
				\includegraphics[width=1\textwidth]{../codding_new/policies_solution/solutions_paper/counterfac_pen1_targetext_2_T_pslow0_lgd0_PSID1_matchGDP0_phii12.png}
			\end{minipage}
			\begin{minipage}[]{0.32\textwidth}
				\centering{\footnotesize{Gini of consumption}}
				%		%	\captionsetup{width=.45\linewidth}
				\includegraphics[width=1\textwidth]{../codding_new/policies_solution/solutions_paper/counterfac_pen1_targetext_2_Gini_consumption_pslow0_lgd0_PSID1_matchGDP0_phii12.png}
			\end{minipage}		
			\begin{minipage}[]{0.32\textwidth}
				\centering{\footnotesize{Penalty poor}}
				%		%	\captionsetup{width=.45\linewidth}
				\includegraphics[width=1\textwidth]{../codding_new/policies_solution/solutions_paper/counterfac_pen1_targetext_2_penalty poor_pslow0_lgd0_PSID1_matchGDP0_phii12.png}
			\end{minipage}
			%		\begin{minipage}[]{0.32\textwidth}
			%			\begin{itemize}
			%				\item higher corrective tax attains lower externality level
			%				\item inequality rises; poverty more severe
			%			\end{itemize}
			%		\end{minipage}
		\end{figure}
		\vspace{-3mm}
		\hfill
		\hspace{6mm}
		\hyperlink{backeff}{\tiny{$\rightarrow$ back}}
	\end{frame}
	
	\begin{frame}{Optimal policy without basic needs}
		% efficiency level of externality equal to baseline model
		% social resp does not affect inequality
		% redistribution no effect on externality
		\hypertarget{skipallo}{}
		\vspace{2mm}
		\begin{figure}
			\begin{minipage}[]{0.32\textwidth}
				\centering{\footnotesize{Corrective tax, $\tau_n$}}
				%	\captionsetup{width=.45\linewidth}
				\includegraphics[width=1\textwidth]{../codding_new/policies_solution/solutions_paper/comparison_targetext_2_tauun_penaltycomparison_nocounter_notauun0_lgd1_pslow0_PSID1_matchGDP0_phii12.png}
			\end{minipage}	
			\begin{minipage}[]{0.32\textwidth}
				\centering{\footnotesize{Income tax, $\tau_l$}}
				%	\captionsetup{width=.45\linewidth}
				\includegraphics[width=1\textwidth]{../codding_new/policies_solution/solutions_paper/comparison_targetext_2_tauul_penaltycomparison_nocounter_notauun0_lgd0_pslow0_PSID1_matchGDP0_phii12.png}
			\end{minipage}
			\begin{minipage}[]{0.32\textwidth}
				\centering{\footnotesize{Transfers, T}}
				%	\captionsetup{width=.45\linewidth}
				\includegraphics[width=1\textwidth]{../codding_new/policies_solution/solutions_paper/comparison_targetext_2_T_penaltycomparison_nocounter_notauun0_lgd0_pslow0_PSID1_matchGDP0_phii12.png}
			\end{minipage}
		\end{figure}
	\vspace{4mm}
		%	\underline{Without basic needs}
		\begin{itemize}
			\item optimal corrective tax decreases; rise in income tax to mitigate drop in revenues from corrective tax
			\vspace{3mm}
			\item no shift to redistribution! %\ar basic needs explain shift to redistribution
		\end{itemize}
		\hfill
		\hyperlink{backeff}{\tiny{$\rightarrow$ back}}
	\end{frame}
	
	
	\begin{frame}{Laissez-faire allocation}
		\hypertarget{LF}{}
		\vspace{6mm}
		\begin{center}
			\begin{minipage}[]{0.45\textwidth}
				\centering{{Gini of consumption}}
				%	\captionsetup{width=.45\linewidth}
				\includegraphics[width=1\textwidth]{../codding_new/policies_solution/solutions_paper/LF_vsRamsey_pen1ext1_targetext_2_Gini_consumption_pslow0_PSID1_matchagg0_phii12_lgd1.png}
			\end{minipage}	
			\begin{minipage}[]{0.45\textwidth}
				\centering{{Penalty poor}}
				%	\captionsetup{width=.45\linewidth}
				\includegraphics[width=1\textwidth]{../codding_new/policies_solution/solutions_paper/LF_vsRamsey_pen1ext1_targetext_2_penalty poor_pslow0_PSID1_matchagg0_phii12_lgd1.png}
			\end{minipage}	
		\end{center}
		\vspace{11mm}
		\hfill
		\hyperlink{backLF}{\tiny{$\rightarrow$ back}}
	\end{frame}
	
	
%	\begin{frame}{Decomposition standard model}
%		\vspace{3mm}
%		\hypertarget{noBN}{}
%		\centering
%		\begin{minipage}[]{0.45\textwidth}
%			\centering{{Income tax, $\tau_l$}}
%			%	\captionsetup{width=.45\linewidth}
%			\includegraphics[width=1\textwidth]{../codding_new/policies_solution/solutions_paper/tauunfixed_pen0_targetext_2_tauul_withnoext_pslow0_PSID1_matchGDP0_phii12_lgd1_presentation.png}
%		\end{minipage}
%		\vspace{7mm}
%		\begin{itemize} 
%			%\item redistribution becomes the central pillar of corrective policy
%			\item labour tax chosen higher to reduce the externality
%			%\item driven by intensified severity of inequality
%		\end{itemize}
%		\vspace{0mm}
%		\hfill
%		\hyperlink{backpe}{\tiny{$\rightarrow$ back}}
%	\end{frame}	
	
	\begin{frame}{Decomposition: no basic needs}
		\vspace{3mm}
		\hypertarget{noBN}{}
		\centering
		\begin{minipage}[]{0.45\textwidth}
			\centering{{Income tax, $\tau_l$}}
			%	\captionsetup{width=.45\linewidth}
			\includegraphics[width=1\textwidth]{../codding_new/policies_solution/solutions_paper/tauunfixed_pen0_targetext_2_tauul_withnoext_pslow0_PSID1_matchGDP0_phii12_lgd1_presentation.png}
		\end{minipage}
		\vspace{3mm}
		\begin{itemize} 
			%\item redistribution becomes the central pillar of corrective policy
			\item income tax also chosen higher to reduce the externality
			\item presence of corrective tax lowers income tax below optimal level without externality 
			%\item driven by intensified severity of inequality
		\end{itemize}
		\vspace{-7mm}
		\hfill	
		\hyperlink{backpe}{\tiny{$\rightarrow$ back}}		 
	\end{frame}	
	
	
	%%% policy effects
%	\section*{3rd: Decomposing policy effects}
	
	\begin{frame}{Effectiveness of policy instruments}
		\hypertarget{3exp}{}
		\alert{	\textbf{What is the role of the different policy channels on the externality?}}
		
		\vspace{3mm}
		\begin{itemize}
			\item \textbf{Problem:} policy effects are interrelated 
			\vspace{2mm}
			\item[\ar] additional timing assumption \\
			\vspace{2mm}
			\begin{enumerate}
				\item<2-> solve for the optimal policy tuple in full model
				\vspace{2mm}
				\item<3-> in laissez-faire allocation, impose optimal corrective tax \ar effect of corrective tax
				\vspace{2mm}
				\item<4-> next, add the optimal income tax but keep labour supply fixed \ar redistribution channel
				\vspace{2mm}
				\item<5-> allow labour supply to adjust \ar efficiency channel
			\end{enumerate}
		\end{itemize}
		
	\end{frame}
	
	\begin{frame}{Effectiveness of policy instruments}
		\hypertarget{backDecomp}{}
		\vspace{2mm}
		\centering
		\begin{minipage}[]{0.55\textwidth}
			\centering{{Policy effects}}
			%	%	\captionsetup{width=.45\linewidth}
			\includegraphics[width=1\textwidth]{../codding_new/policies_solution/solutions_paper/contribs_nototal_quant_redistribution_pen1_targetext_2_ext1_yn_pslow0_PSID1_matchGDP0_phii12_lgd1_presentation.png}
		\end{minipage}
		%	\vspace{4mm}
		%		\begin{itemize} 
		%			\item redistribution becomes the central pillar of corrective policy
		%			%\item<2-> reliance on redistribution due to intensified severity of inequality
		%		\end{itemize}
		
		\vspace{7mm}
		\hfill	
		\hyperlink{backpe}{\tiny{$\rightarrow$ back}}		 
		%		\hyperlink{sensi}{\tiny{$\rightarrow$ sensitivity}}
		
	\end{frame}

	\begin{frame}{Effectiveness of policy instruments: no basic needs}
		\hypertarget{effDecomp}{}
		\vspace{2mm}
		\centering
		\begin{minipage}[]{0.55\textwidth}
			\centering{{Policy effects}}
			%	%	\captionsetup{width=.45\linewidth}
			\includegraphics[width=1\textwidth]{../codding_new/policies_solution/solutions_paper/contribs_nototal_quant_redistribution_pen0_targetext_2_ext1_yn_pslow0_PSID1_matchGDP0_phii12_lgd1_presentation.png}
		\end{minipage}
		%	\vspace{4mm}
		%		\begin{itemize} 
		%			\item redistribution becomes the central pillar of corrective policy
		%			%\item<2-> reliance on redistribution due to intensified severity of inequality
		%		\end{itemize}
		
		\vspace{7mm}
		\hfill	
		\hyperlink{backpe}{\tiny{$\rightarrow$ back}}		 
		
	\end{frame}
	
	\section{Sensitivity}
	\begin{frame}{Less inequality}
		\hypertarget{sensi}{}
		\begin{figure}
			\begin{minipage}[]{0.32\textwidth}
				\centering{\footnotesize{ Corrective tax, $\tau_n$ \\ \ }}
				%	\captionsetup{width=.45\linewidth}
				\includegraphics[width=1\textwidth]{../codding_new/policies_solution/solutions_paper/higherInq130_pen1_targetext_2_tauun_PSID1_matchGDP0_phii12_lgd1.png}
			\end{minipage}
			\begin{minipage}[]{0.32\textwidth}
				\centering{\footnotesize{ Income tax, $\tau_l$\\ \ }}
				%	\captionsetup{width=.45\linewidth}
				\includegraphics[width=1\textwidth]{../codding_new/policies_solution/solutions_paper/higherInq130_pen1_targetext_2_tauul_PSID1_matchGDP0_phii12_lgd0.png}
			\end{minipage}
			\begin{minipage}[]{0.32\textwidth}
				\centering{\footnotesize{ Transfers\\ \ }}
				%	\captionsetup{width=.45\linewidth}
				\includegraphics[width=1\textwidth]{../codding_new/policies_solution/solutions_paper/higherInq130_pen1_targetext_2_T_PSID1_matchGDP0_phii12_lgd0.png}
			\end{minipage} 
		\end{figure}
		\begin{itemize}
			\item $z_h=2.14$, $z_l=0.14$ in contrast to $z_h=2.13$, $z_l=0.03$
			\item  even if the poor were 30\% richer, the shift to redistribution would remain optimal
		\end{itemize}
		%	\vspace{-1mm}
		%	\hfill
		%	\hyperlink{count}{\tiny{$\rightarrow$  counterfactual}},
		\vspace{-1mm}
		\hfill 
		\hyperlink{backpe}{\tiny{$\rightarrow$ back,}}
		\hyperlink{conc}{\tiny{$\rightarrow$ conclusion}}	
	\end{frame}
	
	
	\begin{frame}{Lower productivity gap: $\frac{A_n}{A_s}=1.26$}
		%\hypertarget{sensi}{}
		\begin{figure}
			%		\begin{minipage}[]{0.322\textwidth}
			%			\centering{\footnotesize{ Corrective tax, $\tau_n$}}
			%			%	\captionsetup{width=.45\linewidth}
			%			\includegraphics[width=1\textwidth]{../codding/policies_solution/solutions_paper/lowerProGap_pen1_targetext_2_tauun_PSID1_matchGDP0_phii15_lgd1.png}
			%		\end{minipage}
			\begin{minipage}[]{0.32\textwidth}
				\centering{\footnotesize{ Labour tax, $\tau_l$}}
				%	\captionsetup{width=.45\linewidth}
				\includegraphics[width=1\textwidth]{../codding_new/policies_solution/solutions_paper/lowerProGap_pen1_targetext_2_tauul_PSID1_matchGDP0_phii12_lgd1.png}
			\end{minipage}
			\begin{minipage}[]{0.32\textwidth}
				\centering{\footnotesize{Transfers, T}}
				%	\captionsetup{width=.45\linewidth}
				\includegraphics[width=1\textwidth]{../codding_new/policies_solution/solutions_paper/lowerProGap_pen1_targetext_2_T_PSID1_matchGDP0_phii12_lgd0.png}
			\end{minipage}  
			\begin{minipage}[]{0.32\textwidth}
				\centering{\footnotesize{ Output ratio, $y_n/y_s$}}
				%	\captionsetup{width=.45\linewidth}
				\includegraphics[width=1\textwidth]{../codding_new/policies_solution/solutions_paper/Compariosn_socialPlanner_pen1_targetext_2_output_ratio_lgd1_notauun0_pslow1_PSID1_matchGDP0_phii12.png}
			\end{minipage}  
		\end{figure}
		\begin{itemize}
			\item  redistribution is not used as an corrective policy instrument%, similar to model without basic needs%consumption inequality does not become too severe
			\item the output ratio approaches the efficient one
		\end{itemize}
		%	\vspace{-1mm}
		%	\hfill
		%	\hyperlink{count}{\tiny{$\rightarrow$  counterfactual}},
		\vspace{-1mm}
		\hfill 
		\hyperlink{backpe}{\tiny{$\rightarrow$ back,}}	
		\hyperlink{conc}{\tiny{$\rightarrow$ conclusion}}	
	\end{frame}
	
	
	\section{Data supplement}
	\begin{frame}
		\vspace{5mm}
		%\begin{minipage}[]{\textwidth}
		\centering{\textbf{ Weekly expenses for an organic and a conventional food bundle}}
		%	\captionsetup{width=.45\linewidth}
		\includegraphics[width=1\textwidth]{../codding/calibration/emp_results/expenses_joint_s.png}
		%\end{minipage}
		\small{The food bundle is determined by the \cite{FoodCommission}, which provides a food bundle in line with planetary and bodily health.}
	\end{frame}
	
	\begin{frame}{}
		\vspace{5mm}
		\begin{table}[h!!]
			\centering
			\caption{\textbf{Monthly basic expenses for a US single working adult in US\$ in 2018}}
			\begin{tabular}{l|rrr} 
				\hline \hline
				\text{ Category}	&\text{ (1) Unsustainable}& \text{(2) Sustainable} & \text{(3) Sustainable} \\ &&&\text{exists}\\
				\hline
				\text{ Housing \& Utilities}&785   & 785& {false}\\
				\text{ Food}&267   & 417.23& {true}\\
				\text{ Transportation}&476   & 476&{false}\\ 
				\text{ Personal \& Household items}&389   & 607.88& {true}\\
				\text{ Healthcare}&177  & 276.59 &{true}\\ 
				\hline 
				\text{Monthly basic needs (sum) }& 2,094& 2,562.70&\\
				
				\text{Annual basic needs }& 25,128& 30,752.38&\\
				\hline \hline
			\end{tabular} 

		\flushleft
		\footnotesize{Source: \cite{BESt}}
		\end{table}
	\end{frame}
\end{document}