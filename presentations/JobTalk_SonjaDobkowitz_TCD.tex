\documentclass[11pt,aspectratio=169]{beamer}
%\documentclass[handout]{beamer}
%\mode<handout>
%{
%		\usepackage{pgf}
%		\usepackage{pgfpages}
%		
%		\pgfpagesdeclarelayout{4 on 1 boxed}
%		{
%				\edef\pgfpageoptionheight{\the\paperheight} 
%				\edef\pgfpageoptionwidth{\the\paperwidth}
%				\edef\pgfpageoptionborder{0pt}
%			}
%		{
%				\pgfpagesphysicalpageoptions
%				{%
%						logical pages=4,%
%						physical height=\pgfpageoptionheight,%
%						physical width=\pgfpageoptionwidth%
%					}
%				\pgfpageslogicalpageoptions{1}
%				{%
%						border code=\pgfsetlinewidth{2pt}\pgfstroke,%
%						border shrink=\pgfpageoptionborder,%
%						resized width=.5\pgfphysicalwidth,%
%						resized height=.5\pgfphysicalheight,%
%						center=\pgfpoint{.25\pgfphysicalwidth}{.75\pgfphysicalheight}%
%					}%
%				\pgfpageslogicalpageoptions{2}
%				{%
%						border code=\pgfsetlinewidth{2pt}\pgfstroke,%
%						border shrink=\pgfpageoptionborder,%
%						resized width=.5\pgfphysicalwidth,%
%						resized height=.5\pgfphysicalheight,%
%						center=\pgfpoint{.75\pgfphysicalwidth}{.75\pgfphysicalheight}%
%					}%
%				\pgfpageslogicalpageoptions{3}
%				{%
%						border code=\pgfsetlinewidth{2pt}\pgfstroke,%
%						border shrink=\pgfpageoptionborder,%
%						resized width=.5\pgfphysicalwidth,%
%						resized height=.5\pgfphysicalheight,%
%						center=\pgfpoint{.25\pgfphysicalwidth}{.25\pgfphysicalheight}%
%					}%
%				\pgfpageslogicalpageoptions{4}
%				{%
%						border code=\pgfsetlinewidth{2pt}\pgfstroke,%
%						border shrink=\pgfpageoptionborder,%
%						resized width=.5\pgfphysicalwidth,%
%						resized height=.5\pgfphysicalheight,%
%						center=\pgfpoint{.75\pgfphysicalwidth}{.25\pgfphysicalheight}%
%					}%
%			}
%		
%		
%		\pgfpagesuselayout{4 on 1 boxed}[a4paper, border shrink=5mm, landscape]
%		\nofiles
%	}

\usefonttheme[onlymath]{serif}
\usetheme[outer/progressbar=foot,
%outer/numbering=none
]{metropolis}
\setbeamertemplate{caption}{\raggedright\insertcaption\par}
\setbeamercolor{background canvas}{bg=black!1}
\setbeamercolor{frametitle}{bg={}, fg=black!80}
\definecolor{myorange}{rgb}{0.8500, 0.3250, 0.0980}
\setbeamercolor{alerted text}{bg={}, fg=myorange }
\setbeamercolor{block title}{bg=black!10, fg=black}
\setbeamercolor{block body}{bg=black!10, fg=black}
\setbeamercolor{block frame}{bg=black, fg=black}
\setbeamertemplate{blocks}[rounded]
\setbeamertemplate{blocks}[framed]
%\usecolortheme{seahorse}
\usepackage[utf8]{inputenc}
\usepackage[english]{babel}
%\usepackage[T1]{fontenc}
\newcommand{\tr}[1]{\textcolor{blue}{#1}}
\usepackage{amsmath}
\usepackage{amsfonts}
\usepackage{amssymb}
\usepackage{mathtools}
\usepackage{calc}
\usepackage{soul}
\setbeamercolor{headerCol}{fg=blue!30,bg=black!80}
\setbeamercolor{bodyCol}{fg=black}
\usepackage{graphicx}
\usepackage{xcolor}
\usepackage{appendix}
\usepackage{hyperref}
\usepackage{natbib}
\usepackage{comment}
\usepackage{setspace}
\renewcommand{\bibsection}{}
\bibliographystyle{apa} 
% have to run bibtex mydocument.aux after first run to generate bbl file. 
\usepackage{appendixnumberbeamer}
\usepackage{xcolor}
\usepackage{subcaption}
%table
\usepackage{makecell}
\usepackage{multirow}
\usepackage{bigdelim}

\newif\ifabbreviation
\pretocmd{\thebibliography}{\abbreviationfalse}{}{}
\AtBeginDocument{\abbreviationtrue}
\DeclareRobustCommand\acroauthor[2]{%
	\ifabbreviation #2\else #1 (\mbox{#2})\fi}

\usepackage[customcolors]{hf-tikz}
\definecolor{sonja}{cmyk}{1.5,0,0.9,0.3}
%\definecolor{blue}{cmyk}{0,1,0,0}
\hfsetfillcolor{black!10}
\hfsetbordercolor{black}

\usepackage{tikz}
\usetikzlibrary{tikzmark}
\newcommand{\ImageWidth}{11cm}
\usetikzlibrary{decorations.pathreplacing,positioning, arrows.meta}
\usetikzlibrary{decorations.markings}
\usepackage{tikz-cd}
\usetikzlibrary{arrows,calc,fit}
\tikzset{mainbox/.style={draw=white, text=white, fill=gray, rectangle, rounded corners, thick, node distance=7em, text width=8em, text centered, minimum height=3.5em}}
\tikzset{dummybox/.style={draw=none, text=white , rectangle, rounded corners, thick, node distance=7em, text width=8em, text centered, minimum height=3.5em}}
\tikzset{box/.style={draw , rectangle, rounded corners, thick, node distance=7em, text width=8em, text centered, minimum height=3.5em}}
\tikzset{container/.style={draw, rectangle, dashed, inner sep=2em}}
\tikzset{line/.style={draw, very thick, -latex'}}
\tikzset{    pil/.style={
		->,
		thick,
		shorten <=2pt,
		shorten >=2pt,}}
\tikzstyle{vecArrow} = [thick, decoration={markings,mark=at position
	1 with {\arrow[semithick]{open triangle 60}}},
double distance=1.4pt, shorten >= 5.5pt,
preaction = {decorate},
postaction = {draw,line width=1.4pt, white,shorten >= 4.5pt}]
\usetikzlibrary{shapes}
\renewcommand{\figurename}{}

%TITLE
\author[Sonja Dobkowitz]{\small Sonja Dobkowitz\\ \footnotesize{University of Bonn%, RTG 2281 The Macroeconomics of Inequality}
}\\ }
%\institute[University of Bonn]{}
\title{Meeting Climate Targets: The Optimal Fiscal Policy Mix}

\newcommand{\ar}{$\Rightarrow$ \ }

%\addtobeamertemplate{navigation symbols}{}{%
%    \usebeamerfont{footline}%
%    \usebeamercolor[fg]{footline}%
%    \hspace{1em}%
%   \insertframenumber/\inserttotalframenumber
%}

%\institute{University of Bonn} 
\date{\small{Presentation at Trinity College Dublin\\ January 27, 2023 }} 
%\subject{} 
\begin{document}

\tikzstyle{modus}=[rectangle,inner sep=5mm,align=center, draw]
\tikzstyle{dialog}=[diamond, align=center, draw]
\tikzstyle{sphere}=[circle, align=center, dotted, minimum size=3cm, draw]
\tikzstyle{circll}=[circle, align=center, minimum size=3cm, draw]
\tikzstyle{circllsmall}=[circle, align=center, minimum size=2cm, draw]
{\setbeamertemplate{footline}{}
	\begin{frame}
		\titlepage
	\end{frame}
}
%\addtocounter{framenumber}{-1}

% {\setbeamertemplate{footline}{}
	% \begin{frame}{Content}
		% \vspace{4mm}
		% \tableofcontents
		% \end{frame}
	% }
%\addtocounter{framenumber}{-1}


%---------------------------------------
%            Intro
%---------------------------------------
\addtocounter{framenumber}{-1}
\begin{frame}{Motivation}
	\begin{itemize}[<+-| alert@+>]
		\setbeamercolor{alerted text}{} %change the font color
		\setbeamerfont{alerted text}{}
		\item How to best meet emission targets in line with climate goals? % Frage
		\vspace{3mm}
		\item Problem:
		\begin{itemize}
			\item[-] On the one hand,  reduce the use of fossil fuels,...
			\vspace{2mm}
			\item[-] ... on the other hand, %keep research productivity high 
			maintain some fossil research activity 
			\vspace{1mm}
			\begin{enumerate}
				\item[-] research in established sectors more productive
				%\item[-] higher risk of duplicating ideas if all research happens in one sector
				\item[-] non-green knowledge facilitates green innovation tomorrow %\tiny{\citep{Barbieri2021GreenPolicy}}
			\end{enumerate}% idea that researchers learn from past findings: building on the shoulder of giants mechanism
			%				and 
			\vspace{2mm}				
			\item[-] A \textbf{carbon tax} lowers the use of fossil fuels but also reduces research in established industries %state of the art
		\end{itemize} 
		\vspace{3mm}
		\item My contribution: introduce \textbf{labor income taxes} as a potential solution
		\vspace{3mm}
		\item[] \hspace{-4mm}\alert{{What is the optimal mix of carbon and labor income taxes to meet emission targets?}}
	\end{itemize}
\end{frame}


\begin{comment}
\addtocounter{framenumber}{-1}
\begin{frame}{Motivation}
	
	\begin{itemize}[<+-| alert@+>]
		\setbeamercolor{alerted text}{} %change the font color
		\setbeamerfont{alerted text}{}
		%	\item we are facing  environmental limits 
		%	\vspace{3mm}
		\item meeting climate targets requires a limit on emissions \citep{IPCC2022}
		\vspace{3mm}
		\item carbon taxes  direct  (i) demand towards emission-low alternatives\\ \hspace{23mm} \underline{and} (ii) research across sectors
		\vspace{3mm}
		\item labor income taxes affect the level of production 
		\vspace{3mm}
		\item \textbf{What is the optimal policy mix to meet the emission target?}
	\end{itemize}
\end{frame}


content...

\begin{frame}{This paper}
	\vspace{-2mm}
	\begin{itemize}
		\item<+-> quantitative model of \alert{directed technical change} building on \cite{Fried2018ClimateAnalysis}
		\vspace{2mm}
		\item<+->   the government   chooses the \alert{path of carbon and income taxes} to maximize welfare\vspace{2mm}
		\item<+-> an \alert{emission limit} constrains the government
	\end{itemize}
	\pause
	\begin{center}
		\begin{figure}
			\centering
			\textbf{US net CO$_2$ emission limit in Gt}\\
			\vspace{2mm}	\includegraphics[width=0.38\textwidth]{../codding_model/own_basedOnFried/optimalPol_010922_revision/figures/all_13Sept22_Tplus30/Emnet.png}
		\end{figure}
	\end{center}
\end{frame}
\end{comment}

\begin{frame}{This paper}
	\vspace{-2mm}
	\begin{itemize}
		\item<+-> Quantitative model building on \cite{Fried2018ClimateAnalysis} 
		\vspace{2mm}
		\begin{itemize}
			\item[-]<+-> \alert{technologies and research on technologies are sector specific}: green, fossil, and non-energy sector \footnotesize{(directed technical change) }
			\vspace{1mm}
%			\item[-]<+-> \alert{research} on technologies augments productivity within a sector \\ \footnotesize{(directed technical change) }
%			\vspace{1mm}
			\item[-]<+-> \alert{knowledge spillovers}: 
			\begin{enumerate}
				\item[a)]<+-> \alert{within-sector}: researchers learn from knowledge accumulated in their sector
				\item[b)]<+-> \alert{cross-sector}: knowledge generated in sector F stimulates innovation in sector G
			\end{enumerate}%\\ \footnotesize{(cross-sectoral knowledge spillovers)}
		\end{itemize}
		\vspace{2mm}
		\item<+->   The government   chooses the \alert{path of carbon and income taxes} to maximize welfare\vspace{2mm}
		\item<+-> An \alert{emission limit} constrains the government		\vspace{2mm}
		\item<+-> \alert{Main results:} 
		\begin{itemize}
			\item[-]<+-> a combination of carbon and labor income taxes is optimal to target the allocation of researchers 	\vspace{1mm} 
			\item[-]<+-> limited use of labor income tax \ar augment baseline model with research subsidies
		\end{itemize}
		%		\begin{itemize}
			%			\item[-]<+-> positive net emissions before 2050
			%			\item[-]<+-> net-zero emissions from 2050 onward
			%		\end{itemize}
	\end{itemize}

%	\pause
%	\vspace{0mm}
%	\centering
%	\resizebox{290pt}{70pt}{%
%		\begin{tikzpicture}
%			\draw[ultra thick, ->] (0,0) -- (10,0);
%			\foreach \x in {0}
%			\draw (\x cm,3pt) -- (\x cm,-3pt);
%			\foreach \x in {5}
%			\draw[dashed, gray, thick] (\x cm,10pt) -- (\x cm,-6pt);
%			% draw node
%			\draw[thick] (0,0) node[below=3pt,thick] {2020} node[above=3pt] {};
%			\draw[thick] (5,0) node[below=3pt,thick] {2050} node[above=3pt] {};
%			\draw [thick ,decorate,decoration={brace,amplitude=5pt}] (0,0.7)  -- +(5,0) 
%			node [black,midway,above=4pt, font=\scriptsize] {Some net emissions allowed};
%			\node[draw=none, black, above=4pt, font=\scriptsize] (B1) at (7.5,-.9) {Net-zero emissions};
%			
%		\end{tikzpicture}
%	}
%	%	\pause
%	%	\begin{center}
%		%		\begin{figure}
%			%	\caption{US CO$_2$ emission limit in Gt}
%			%	\vspace{-2mm}
%			%	\includegraphics[width=0.4\textwidth]{../codding_model/own_basedOnFried/optimalPol_010922_revision/figures/all_13Sept22_Tplus30/Emnet_goals_o0_lgd0.png}
%			%\end{figure}
%			%	\end{center}
	\end{frame}
	
	\begin{comment}
	\begin{frame}{Preview of results}
		\centering
		\vspace{-3mm}
		\begin{itemize}[<+-| alert@+>]
			\setbeamercolor{alerted text}{} %change the font color
			\setbeamerfont{alerted text}{}
			%begin{minipage}[]{1\textwidth}
			%\begin{itemize}
			\item A combination of carbon and labor income taxes is optimal to target the direction of research
			\vspace{3mm}
			\item Before the net-zero emission limit (ca. 2020-2050): 
			\begin{itemize}
				\item[-] lower carbon tax to maintain some fossil research %\tr{give an example of knowledge spillovers here}
				\item[-] a tax on labor reduces emissions
			\end{itemize}
			\vspace{3mm}
			\item Under the net-zero emission limit (from 2050 onward): 
			\begin{itemize}
				\item[-]  higher carbon tax to foster green research
				\item[-]  a smaller tax on labor boosts output
			\end{itemize}
			\vspace{3mm}
			\item Limited use of labor income tax \ar augment baseline model with research subsidies
			%		\begin{itemize}
				%			\item[-] income tax remains beneficial to correct labor market distortions arising from lack of lump-sum rebates of carbon tax revenues
				%			%\item[-] when only green research subsidies are available, welfare smaller than with lump-sum rebates			
				%		\end{itemize}
			%; \tr{e.g. a higher marginal value of leisure}	%	\end{minipage}
	\end{itemize}
\end{frame}

content...
\end{comment}

\begin{comment}
	\begin{frame}
		\begin{itemize}
			\item How to best implement emission targets in line with climate goals?
			\item 2 adjustment possibilities to reduce emissions: 
			\begin{enumerate}
				\item shift from fossil to green production technologies (e.g. carbon tax, green subsidies)
				\item lower the level of production overall (e.g. labor income taxes)
			\end{enumerate}
			\item Literature focuses on 1st point
		\end{itemize}
	\end{frame}
\end{comment}

\begin{frame}{Contribution to the literature}
	\begin{itemize}[<+->]
		\item \alert{\textbf{Standard to have inelastic labor supply in environmental policy discussion}}\\  \footnotesize{ \citep{Acemoglu2012TheChange, Golosov2014OptimalEquilibrium, Acemoglu2016TransitionTechnology, Fried2018ClimateAnalysis, Hart2019TheEconomists}}
		\\  \normalsize{\alert{\ar precludes studying combinations of policies targeted at the composition and\\ \hspace{5mm} the level of production }}
		\vspace{2mm}
		\item {Labor income taxes and environmental policies in the literature}
		\begin{itemize}
			\item[-]  government funding condition; labor income tax generally passive \\
			 \footnotesize{ \citep{ LansBovenberg1994EnvironmentalTaxation, Goulder1995EnvironmentalGuide, Barrage2019OptimalPolicy}} % Not an env. policy instrument
			\item[-] {income inequality} motivates taxing labor income as environmental policy tool \\ \footnotesize{\citep{Jacobs2019RedistributionCurves, Dobkowitz2022, Douenne2022OptimalHouseholds}}
			%	\item[-] This paper: novel motive for the use of labor income taxes within the optimal environmental policy: endogeneity of the allocation of researchers
		\end{itemize}		
		% Das klingt gut (differenzierung zwischen wirkungsweisen und nicht instrumenten! ): weil instrumente beide effekte haben können, dass die carbon tax auch das level beeinflusst wird ausgeklammert
		% I argue that we should study optimal env. policies allowing for level adjustments of production! I provide am example scenario where adjusting the level becomes part of the optimal policy
		\vspace{2mm}
		\item \alert{This paper}: motive for labor income taxation emerges from  environmental externality % alone %directly % is optimal due to  directed technical change % and (ii) lack of research subsidies
		%	\footnotesize{\citep{Acemoglu2012TheChange}}
		%\\ \normalsize{\alert{\ar role for labor income taxation}}
		% weak double dividend literature (advantage to use env tac revenues to lower existing distortions; deviation of optimal env tax from pigou)
		%			\begin{itemize}
			%			%	\item[-] Research subsidies essential to implement first best.
			%			%	Otherwise, carbon tax higher to bolster green research; costly in terms of output
			%			%	\item[-]  In this case, labor income taxes may help get closer to first best
			%				\item[-] I add a richer, quantitative framework \ar new qualitative insights % increasing carbon tax, potentially fossil subsidy optimal
			%				\item[-]  insights on importance of additional measures
			%			\end{itemize}
		%	With knowledge spillovers \ar (i) increasing policy intervention, (ii) potentially advantageous to maintain some fossil research
		%	\vspace{2mm}
		%	% theirs is an analytical model; qualitative results; I look at a quantitative framework
		%	\item Quantitative framework builds on \cite{Fried2018ClimateAnalysis} %\ar new qualitative insights
		%	\begin{itemize}
			%		\item[-] I add a \alert{dynamic optimal policy analysis} under an exogenous \alert{dynamic emission target} %eventually declining to \alert{net-zero emissions}
			%		\item[-] \alert{elastic labor supply}
			%	\end{itemize}
	\end{itemize}
\end{frame}


\section{Model and Mechanisms}

\begin{frame}{Model}
	\begin{figure}[h]
		\vspace{-4mm}
		\centering
		\begin{tikzpicture}[auto,scale=.7, transform shape]
			\node[circll] (A) at (-7,4) {\textbf{{\hyperlink{prodmod}{Production}}}\\ \textbf{\hyperlink{prodmod}{and Research}}};
			\node[circll] (B) at (7,4) {\textbf{\hyperlink{modhh}{{Representative}}}\\ \textbf{\hyperlink{modhh}{{Household}}}};
			\node[circll] (D) at (0,9) {\textbf{{Government}}}; 
			\node[draw=none] (B1) at (5,4.25) {};
			\node[draw=none] (B2) at (5,3.5) {};
			\node[draw=none] (BA1) at (-5,4.25) {};
			\node[draw=none] (BA2) at (-5,3.5) {};
			\node[draw=none] (D1) at (1.8,7.8) {};
			
			\node[draw=none] (B22) at (6.4,5.6) {};
			\node[draw=none] (D2) at (2.3,8.5) {};
			\node[draw=none] (B3) at (5.3,5.3) {};
			\node[draw=none] (D3) at (-1.5,7) {};
			\node[draw=none] (A1) at (-4.2,4.6) {};
			\node[draw=none] (D4) at (-2,8.3) {};
			\node[draw=none] (A4) at (-6,5.4) {};
			\draw [draw=none] (B22) to node[pos=0.65, swap]{\textcolor{black!1}{Tax on labor, $\pmb{\tau_{\iota}}$}} (D2);
			\draw [draw=none] (B22) to node[pos=0.45, swap]{\textcolor{black!1}{$\pmb{\tau_{\iota}}\uparrow$: labor supply $\downarrow$ \ar emissions $\downarrow$}} (D2);
			\draw [draw=none] (B22) to node[pos=0.3, swap]{\textcolor{black!1}{$\pmb{\tau_{\iota}}\downarrow$: labor supply $\uparrow$ \ar output $\uparrow$}} (D2);
			\draw [draw=none] (D1) to node[pos=0.5, swap]{\textcolor{black!1}{Transfers}} (B3);
			
			\draw [draw=none] (B1) to node[pos=0.75, swap]{\textcolor{black!1}{Workers and scientists}} (BA1);
			\draw [draw=none] (BA2) to node[pos=0.75, swap]{\textcolor{black!1}{Final good}} (B2);
			\draw [draw=none] (A4) to node[pos=0.5, swap]{\textcolor{black!1}{Tax on carbon, $\pmb{\tau_F}$}}   (D4);
		\end{tikzpicture}		
	\end{figure}
\hypertarget{backScheme}{}
\end{frame}


\addtocounter{framenumber}{-1}
\begin{frame}{Model}
	\begin{figure}[h]
		\vspace{-4mm}
		\centering
		\begin{tikzpicture}[auto,scale=.7, transform shape]
			\node[circll] (A) at (-7,4) {\textbf{{\hyperlink{prodmod}{Production}}}\\ \textbf{\hyperlink{prodmod}{and Research}}};
			\node[circll] (B) at (7,4) {\textbf{\hyperlink{modhh}{{Representative}}}\\ \textbf{\hyperlink{modhh}{{Household}}}};
			\node[circll] (D) at (0,9) {\textbf{{Government}}}; 
			\node[draw=none] (B1) at (5,4.25) {};
			\node[draw=none] (B2) at (5,3.5) {};
			\node[draw=none] (BA1) at (-5,4.25) {};
			\node[draw=none] (BA2) at (-5,3.5) {};
			\node[draw=none] (D1) at (1.8,7.8) {};
			
			\node[draw=none] (B22) at (6.4,5.6) {};
			\node[draw=none] (D2) at (2.3,8.5) {};
			\node[draw=none] (B3) at (5.3,5.3) {};
			\node[draw=none] (D3) at (-1.5,7) {};
			\node[draw=none] (A1) at (-4.2,4.6) {};
			\node[draw=none] (D4) at (-2,8.3) {};
			\node[draw=none] (A4) at (-6,5.4) {};
			\draw [draw=none] (B22) to node[pos=0.65, swap]{\textcolor{black!1}{Tax on labor, $\pmb{\tau_{\iota}}$}} (D2);
			\draw [draw=none] (B22) to node[pos=0.45, swap]{\textcolor{black!1}{$\pmb{\tau_{\iota}}\uparrow$: labor supply $\downarrow$ \ar emissions $\downarrow$}} (D2);
			\draw [draw=none] (B22) to node[pos=0.3, swap]{\textcolor{black!1}{$\pmb{\tau_{\iota}}\downarrow$: labor supply $\uparrow$ \ar output $\uparrow$}} (D2);
			\draw [draw=none] (D1) to node[pos=0.5, swap]{\textcolor{black!1}{Transfers}} (B3);
			
			\draw [->] (B1) to node[pos=0.75, swap]{Workers and scientists} (BA1);
			\draw [->] (BA2) to node[pos=0.75, swap]{Final good} (B2);
			\draw [draw=none] (A4) to node[pos=0.5, swap]{\textcolor{black!1}{Tax on carbon, $\pmb{\tau_F}$}}   (D4);
		\end{tikzpicture}		
	\end{figure}
\end{frame}


\addtocounter{framenumber}{-1}
\begin{frame}{Model}
	\begin{figure}[h]
		\vspace{-4mm}
		\centering
		\begin{tikzpicture}[auto,scale=.7, transform shape]
			\node[circll] (A) at (-7,4) {\textbf{{\hyperlink{prodmod}{Production}}}\\ \textbf{\hyperlink{prodmod}{and Research}}};
			\node[circll] (B) at (7,4) {\textbf{\hyperlink{modhh}{{Representative}}}\\ \textbf{\hyperlink{modhh}{{Household}}}};
			\node[circll] (D) at (0,9) {\textbf{\alert{Government}}\\ \textbf{max welfare} \\ \textbf{s.t. emission limit} }; 
			\node[draw=none] (B1) at (5,4.25) {};
			\node[draw=none] (B2) at (5,3.5) {};
			\node[draw=none] (BA1) at (-5,4.25) {};
			\node[draw=none] (BA2) at (-5,3.5) {};
			\node[draw=none] (D1) at (1.8,7.8) {};
			
			\node[draw=none] (B22) at (6.4,5.6) {};
			\node[draw=none] (D2) at (2.3,8.5) {};
			\node[draw=none] (B3) at (5.3,5.3) {};
			\node[draw=none] (D3) at (-1.5,7) {};
			\node[draw=none] (A1) at (-4.2,4.6) {};
			\node[draw=none] (D4) at (-2,8.3) {};
			\node[draw=none] (A4) at (-6,5.4) {};
			\draw [draw=none] (B22) to node[pos=0.65, swap]{\textcolor{black!1}{Tax on labor, $\pmb{\tau_{\iota}}$}} (D2);
\draw [draw=none] (B22) to node[pos=0.45, swap]{\textcolor{black!1}{$\pmb{\tau_{\iota}}\uparrow$: labor supply $\downarrow$ \ar emissions $\downarrow$}} (D2);
\draw [draw=none] (B22) to node[pos=0.3, swap]{\textcolor{black!1}{$\pmb{\tau_{\iota}}\downarrow$: labor supply $\uparrow$ \ar output $\uparrow$}} (D2);
\draw [draw=none] (D1) to node[pos=0.5, swap]{\textcolor{black!1}{Transfers}} (B3);
			
			\draw [->] (B1) to node[pos=0.75, swap]{Workers and scientists} (BA1);
			\draw [->] (BA2) to node[pos=0.75, swap]{Final good} (B2);
			\draw [draw=none] (A4) to node[pos=0.5, swap]{\textcolor{black!1}{Tax on carbon, $\pmb{\tau_F}$}}   (D4);
		\end{tikzpicture}		
	\end{figure}
\end{frame}

\begin{comment}
\addtocounter{framenumber}{-1}
\begin{frame}{Model}
	\begin{figure}[h]
		\vspace{-4mm}
		\centering
		\begin{tikzpicture}[auto,scale=.7, transform shape]
			\node[circll] (A) at (-7,4) {\textbf{{\hyperlink{prodmod}{Production}}}\\ \textbf{\hyperlink{prodmod}{and Research}}};
			\node[circll] (B) at (7,4) {\textbf{\hyperlink{modhh}{{Representative}}}\\ \textbf{\hyperlink{modhh}{{Household}}}};
			\node[circll] (D) at (0,9) {\textbf{\alert{Government}}\\ \textbf{max welfare} \\ \textbf{s.t. emission limit} }; 
			\node[draw=none] (B1) at (5,4.25) {};
			\node[draw=none] (B2) at (5,3.5) {};
			\node[draw=none] (BA1) at (-5,4.25) {};
			\node[draw=none] (BA2) at (-5,3.5) {};
			\node[draw=none] (D1) at (1.8,7.8) {};
			
			\node[draw=none] (B22) at (6.4,5.6) {};
			\node[draw=none] (D2) at (2.3,8.5) {};
			\node[draw=none] (B3) at (5.3,5.3) {};
			\node[draw=none] (D3) at (-1.5,7) {};
			\node[draw=none] (A1) at (-4.2,4.6) {};
			\node[draw=none] (D4) at (-2,8.3) {};
			\node[draw=none] (A4) at (-6,5.4) {};
			\draw [draw=none] (B22) to node[pos=0.65, swap]{\textcolor{black!1}{Tax on labor, $\pmb{\tau_{\iota}}$:}} (D2);
\draw [draw=none] (B22) to node[pos=0.45, swap]{\textcolor{black!1}{$\pmb{\tau_{\iota}}\uparrow$:  labor supply $\downarrow$ \ar emissions $\downarrow$}} (D2);
\draw [draw=none] (B22) to node[pos=0.3, swap]{\textcolor{black!1}{$\pmb{\tau_{\iota}}\downarrow$: labor supply $\uparrow$ \ar output $\uparrow$}} (D2);
\draw [draw=none] (D1) to node[pos=0.5, swap]{\textcolor{black!1}{Transfers}} (B3);

\draw [->] (B1) to node[pos=0.75, swap]{Workers and scientists} (BA1);
\draw [->] (BA2) to node[pos=0.75, swap]{Final good} (B2);
\draw [draw=none] (A4) to node[pos=0.5, swap]{\textcolor{black!1}{Tax on carbon, $\pmb{\tau_F}$}}   (D4);
		\end{tikzpicture}		
	\end{figure}
\end{frame}

content...
\end{comment}

\addtocounter{framenumber}{-1}
\begin{frame}{Model}
	\begin{figure}[h]
		\vspace{-4mm}
		\centering
		\begin{tikzpicture}[auto,scale=.7, transform shape]
			\node[circll] (A) at (-7,4) {\textbf{{\hyperlink{prodmod}{Production}}}\\ \textbf{\hyperlink{prodmod}{and Research}}};
			\node[circll] (B) at (7,4) {\textbf{\hyperlink{modhh}{{Representative}}}\\ \textbf{\hyperlink{modhh}{{Household}}}};
			\node[circll] (D) at (0,9) {\textbf{\alert{Government}}\\ \textbf{max welfare} \\ \textbf{s.t. emission limit} }; 
			\node[draw=none] (B1) at (5,4.25) {};
			\node[draw=none] (B2) at (5,3.5) {};
			\node[draw=none] (BA1) at (-5,4.25) {};
			\node[draw=none] (BA2) at (-5,3.5) {};
			\node[draw=none] (D1) at (1.8,7.8) {};
			
			\node[draw=none] (B22) at (6.4,5.6) {};
			\node[draw=none] (D2) at (2.3,8.5) {};
			\node[draw=none] (B3) at (5.3,5.3) {};
			\node[draw=none] (D3) at (-1.5,7) {};
			\node[draw=none] (A1) at (-4.2,4.6) {};
			\node[draw=none] (D4) at (-2,8.3) {};
			\node[draw=none] (A4) at (-6,5.4) {};
			\draw [->] (B22) to node[pos=0.65, swap]{\alert{Tax on labor, $\pmb{\tau_{\iota}}$:}} (D2);
			\draw [draw=none] (B22) to node[pos=0.45, swap]{\alert{$\pmb{\tau_{\iota}}\uparrow$:  labor supply $\downarrow$ \ar emissions $\downarrow$}} (D2);
			\draw [draw=none] (B22) to node[pos=0.3, swap]{\textcolor{black!1}{$\pmb{\tau_{\iota}}\downarrow$: labor supply $\uparrow$ \ar output $\uparrow$}} (D2);
\draw [draw=none] (D1) to node[pos=0.5, swap]{\textcolor{black!1}{Transfers}} (B3);

			
			\draw [->] (B1) to node[pos=0.75, swap]{Workers and scientists} (BA1);
			\draw [->] (BA2) to node[pos=0.75, swap]{Final good} (B2);
			%	\draw [->] (A4) to node[pos=0.5, swap]{\alert{Tax on carbon, $\pmb{\tau_F}$}}   (D4);
		\end{tikzpicture}
		
	\end{figure}
\end{frame}


%\addtocounter{framenumber}{-1}
%\begin{frame}{Model}
%	\begin{figure}[h]
%		\vspace{-4mm}
%		\centering
%		\begin{tikzpicture}[auto,scale=.7, transform shape]
%			\node[circll] (A) at (-7,4) {\textbf{{\hyperlink{prodmod}{Production}}}\\ \textbf{\hyperlink{prodmod}{and Research}}};
%			\node[circll] (B) at (7,4) {\textbf{\hyperlink{modhh}{{Representative}}}\\ \textbf{\hyperlink{modhh}{{Household}}}};
%			\node[circll] (D) at (0,9) {\textbf{\alert{Government}}\\ \textbf{max welfare} \\ \textbf{s.t. emission limit} }; 
%			\node[draw=none] (B1) at (5,4.25) {};
%			\node[draw=none] (B2) at (5,3.5) {};
%			\node[draw=none] (BA1) at (-5,4.25) {};
%			\node[draw=none] (BA2) at (-5,3.5) {};
%			\node[draw=none] (D1) at (1.8,7.8) {};
%			
%			\node[draw=none] (B22) at (6.4,5.6) {};
%			\node[draw=none] (D2) at (2.3,8.5) {};
%			\node[draw=none] (B3) at (5.3,5.3) {};
%			\node[draw=none] (D3) at (-1.5,7) {};
%			\node[draw=none] (A1) at (-4.2,4.6) {};
%			\node[draw=none] (D4) at (-2,8.3) {};
%			\node[draw=none] (A4) at (-6,5.4) {};
%			\draw [->] (B22) to node[pos=0.65, swap]{\alert{Tax on labor, $\pmb{\tau_{\iota}}$:}} (D2);
%			\draw [draw=none] (B22) to node[pos=0.45, swap]{\alert{$\pmb{\tau_{\iota}}\uparrow$:  labor supply $\downarrow$ \ar emissions $\downarrow$}} (D2);
%			\draw [draw=none] (B22) to node[pos=0.3, swap]{\alert{$\pmb{\tau_{\iota}}\downarrow$:    labor supply $\uparrow$ \ar output $\uparrow$}} (D2);
%			\draw [draw=none] (D1) to node[pos=0.5, swap]{\textcolor{black!1}{Transfers}} (B3);
%			
%			\draw [->] (B1) to node[pos=0.75, swap]{Workers and scientists} (BA1);
%			\draw [->] (BA2) to node[pos=0.75, swap]{Final good} (B2);		
%			%\draw [->] (A4) to node[pos=0.5, swap]{\alert{Tax on carbon, $\pmb{\tau_F}$}}   (D4);
%		\end{tikzpicture}
%		
%	\end{figure}
%\end{frame}

%\addtocounter{framenumber}{-1}
%\begin{frame}{Model}
%	\begin{figure}[h]
%		\vspace{-4mm}
%		\centering
%		\begin{tikzpicture}[auto,scale=.7, transform shape]
%			\node[circll] (A) at (-7,4) {\textbf{{\hyperlink{prodmod}{Production}}}\\ \textbf{\hyperlink{prodmod}{and Research}}};
%			\node[circll] (B) at (7,4) {\textbf{\hyperlink{modhh}{{Representative}}}\\ \textbf{\hyperlink{modhh}{{Household}}}};
%			\node[circll] (D) at (0,9) {\textbf{\alert{Government}}\\ \textbf{max welfare} \\ \textbf{s.t. emission limit} }; 
%			\node[draw=none] (B1) at (5,4.25) {};
%			\node[draw=none] (B2) at (5,3.5) {};
%			\node[draw=none] (BA1) at (-5,4.25) {};
%			\node[draw=none] (BA2) at (-5,3.5) {};
%			\node[draw=none] (D1) at (1.8,7.8) {};
%			
%			\node[draw=none] (B22) at (6.4,5.6) {};
%			\node[draw=none] (D2) at (2.3,8.5) {};
%			\node[draw=none] (B3) at (5.3,5.3) {};
%			\node[draw=none] (D3) at (-1.5,7) {};
%			\node[draw=none] (A1) at (-4.2,4.6) {};
%			\node[draw=none] (D4) at (-2,8.3) {};
%			\node[draw=none] (A4) at (-6,5.4) {};
%			\draw [->] (B22) to node[pos=0.65, swap]{\alert{Tax on labor, $\pmb{\tau_{\iota}}$:}} (D2);
%			\draw [draw=none] (B22) to node[pos=0.45, swap]{\alert{$\pmb{\tau_{\iota}}\uparrow$:  labor supply $\downarrow$ \ar emissions $\downarrow$}} (D2);
%			\draw [draw=none] (B22) to node[pos=0.3, swap]{\alert{$\pmb{\tau_{\iota}}\downarrow$:    labor supply $\uparrow$ \ar output $\uparrow$}} (D2);
%		%	\draw [->] (D1) to node[pos=0.5, swap]{{Transfers}} (B3);
%			
%			\draw [->] (B1) to node[pos=0.75, swap]{Workers and scientists} (BA1);
%			\draw [->] (BA2) to node[pos=0.75, swap]{Final good} (B2);		
%			%\draw [->] (A4) to node[pos=0.5, swap]{\alert{Tax on carbon, $\pmb{\tau_F}$}}   (D4);
%		\end{tikzpicture}
%		
%	\end{figure}
%\end{frame}


\addtocounter{framenumber}{-1}
\begin{frame}{Model}
	\begin{figure}[h]
		\vspace{-4mm}
		\centering
		\begin{tikzpicture}[auto,scale=.7, transform shape]
			\node[circll] (A) at (-7,4) {\textbf{{\hyperlink{prodmod}{Production}}}\\ \textbf{\hyperlink{prodmod}{and Research}}};
			\node[circll] (B) at (7,4) {\textbf{\hyperlink{modhh}{{Representative}}}\\ \textbf{\hyperlink{modhh}{{Household}}}};
			\node[circll] (D) at (0,9) {\hyperlink{govmod}{\textbf{\alert{Government}}}\\ \hyperlink{govmod}{\textbf{max welfare}} \\ \hyperlink{govmod}{\textbf{s.t. emission limit}} }; 
			\node[draw=none] (B1) at (5,4.25) {};
			\node[draw=none] (B2) at (5,3.5) {};
			\node[draw=none] (BA1) at (-5,4.25) {};
			\node[draw=none] (BA2) at (-5,3.5) {};
			\node[draw=none] (D1) at (1.8,7.8) {};
			
			\node[draw=none] (B22) at (6.4,5.6) {};
			\node[draw=none] (D2) at (2.3,8.5) {};
			\node[draw=none] (B3) at (5.3,5.3) {};
			\node[draw=none] (D3) at (-1.5,7) {};
			\node[draw=none] (A1) at (-4.2,4.6) {};
			\node[draw=none] (D4) at (-2,8.3) {};
			\node[draw=none] (A4) at (-6,5.4) {};
			\draw [->] (B22) to node[pos=0.65, swap]{{Tax on labor, $\pmb{\tau_{\iota}}$}} (D2);
			\draw [draw=none] (B22) to node[pos=0.45, swap]{\textcolor{black!1}{$\pmb{\tau_{\iota}}\uparrow$: labor supply $\downarrow$ \ar emissions $\downarrow$}} (D2);
			\draw [draw=none] (B22) to node[pos=0.3, swap]{\textcolor{black!1}{$\pmb{\tau_{\iota}}\downarrow$: labor supply $\uparrow$ \ar output $\uparrow$}} (D2);
		%	\draw [->] (D1) to node[pos=0.5, swap]{Transfers} (B3);
			
			\draw [->] (B1) to node[pos=0.75, swap]{Workers and scientists} (BA1);
			\draw [->] (BA2) to node[pos=0.75, swap]{Final good} (B2);
			\draw [->] (A4) to node[pos=0.5, swap]{\alert{Tax on carbon, $\pmb{\tau_F}$}}   (D4);
			%	\draw [->] (A4) to node[pos=0.5, swap]{\alert{Tax on carbon, $\pmb{\tau_F}$}}   (D4);
		\end{tikzpicture}
		
	\end{figure}

\hypertarget{backSchemeEnd}{}
\end{frame}

\begin{frame}{Production}
	\begin{figure}[h]
		\vspace{-10mm}
		\centering
		\begin{tikzpicture}[auto,scale=.7, transform shape]
			\node[circll] (A) at (0,17) {\textbf{Final}\textbf{ Good}}; 
			\node[circll] (B) at (-6,14) {\textbf{Energy}};
			\node[circll] (C) at (5,14) {\textbf{{Non-energy}}};
			\node[circll, draw=black!1] (D) at (-10,12) {\textbf{\textcolor{black!1}{Fossil}}};
			\node[circll, draw=black!1] (E) at (-2,12) {\textbf{\textcolor{black!1}{Green}}};
			\node[sphere, draw=black!1] (Ems) at (-11,16) {\textbf{\textcolor{black!1}{Emissions}}};
			
			
			
			\draw [->] (B) to node[pos=0.5, swap]{} (A);
			\draw [->] (C) to node[pos=0.5, swap]{} (A);
			
			\draw [draw=none] (E) to node[pos=0.5, swap]{} (B);
			\draw [draw=none] (D) to node[pos=0.5, swap]{} (B);
			
			
			\draw [draw=none] (D) to node[pos=0.5, swap]{} (Ems);
		\end{tikzpicture}
		
	\end{figure}
\end{frame}
\addtocounter{framenumber}{-1}
\begin{frame}{Production}
	\begin{figure}[h]
		\vspace{-10mm}
		\centering
		\begin{tikzpicture}[auto,scale=.7, transform shape]
			\node[circll] (A) at (0,17) {\textbf{Final}\textbf{ Good}}; 
			\node[circll] (B) at (-6,14) {\textbf{Energy}};
			\node[circll] (C) at (5,14) {\textbf{{Non-energy}}};
			\node[circll] (D) at (-10,12) {\textbf{{Fossil}}};
			\node[circll] (E) at (-2,12) {\textbf{{Green}}};
			\node[sphere] (Ems) at (-11,16) {\textbf{Emissions}};
			
			%		\node[circllsmall] (CM) at (4.6,8) {\textbf{{Machines}}};
			%		\node[circllsmall] (CL) at (7.4,8) {\textbf{{Labor}}};			\node[circllsmall] (EM) at (-3.4,8) {\textbf{{Machines}}};
			%	    \node[circllsmall] (EL) at (-0.6,8) {\textbf{{Labor}}};
			%	    \node[circllsmall] (DM) at (-11.4,8) {\textbf{{Machines}}};
			%	    \node[circllsmall] (DL) at (-8.6,8) {\textbf{{Labor}}};
			
			
			\draw [->] (B) to node[pos=0.5, swap]{} (A);
			\draw [->] (C) to node[pos=0.5, swap]{} (A);
			
			\draw [->] (E) to node[pos=0.5, swap]{} (B);
			\draw [->] (D) to node[pos=0.5, swap]{} (B);
			
			%		
			%		\draw [->] (EM) to node[pos=0.5, swap]{} (E);
			%		\draw [->] (EL) to node[pos=0.5, swap]{} (E);
			%		
			%		
			%		\draw [->] (DM) to node[pos=0.5, swap]{} (D);
			%		\draw [->] (DL) to node[pos=0.5, swap]{} (D);
			%		
			%		
			%		\draw [->] (CM) to node[pos=0.5, swap]{} (C);
			%		\draw [->] (CL) to node[pos=0.5, swap]{} (C);
			
			\draw [->] (D) to node[pos=0.5, swap]{} (Ems);
		\end{tikzpicture}
		
	\end{figure}
\end{frame}

\begin{frame}{1st: Carbon tax effect on \alert{fossil demand}}
	\vspace{13mm}
	
%			\text{Demand energy producers}
	\begin{align*}
		%		\tikzmarkin{first}(1.3,1.2)(-1,-0.8)
	%	\text{Final good}\hspace{4mm}&Y_t =\left(\delta_y^{\frac{1}{\varepsilon_y}}E_t^\frac{\varepsilon_y-1}{\varepsilon_y}+(1-\delta_y)^{\frac{1}{\varepsilon_y}}N_t^\frac{\varepsilon_y-1}{\varepsilon_y}\right)^\frac{\varepsilon_y}{\varepsilon_y-1} \\
	%	\ \\
	%	\text{Energy}\hspace{4mm}&E_t =\left({F}_t^\frac{\varepsilon_e-1}{\varepsilon_e}+G_t^\frac{\varepsilon_e-1}{\varepsilon_e}\right)^\frac{\varepsilon_e}{\varepsilon_e-1}\\
	%	\ \\
\hspace{4mm}&\frac{F_t}{G_t} = \left(\frac{p_{Gt}}{p_{Ft}+\alert{\pmb{\tau_{Ft}}}}\right)^{\varepsilon_e}
		%	\tikzmarkend{third}
	\end{align*}

	\small
%	\vfill
	\vspace{22mm}
	\hspace{-4mm}
	\begin{minipage}[t!]{0.24\textwidth}
		\vspace{0mm}
		\begin{itemize}	
			\item[]$F_t$: fossil energy
			\vspace{-2mm}	
			\item[]$G_t$: green energy
%			\vspace{-7mm}	
		%	\item[]$N_t$: non-energy
		\end{itemize}
	\end{minipage}
	\begin{minipage}[t!]{0.24\textwidth}
		\vspace{0mm}
		\begin{itemize}
			\item[] $p_{Gt}$: price green  \vspace{-2mm}
			\item[] $p_{Ft}$: price fossil
%			\vspace{-2mm}	
%			\item[] $\tau_{Ft}$: carbon tax
		\end{itemize}
	\end{minipage}
	\begin{minipage}[t!]{0.47\textwidth}
		\vspace{0mm}
		\begin{itemize}
%			\item[] $\delta_{y}$: weight on energy\vspace{-2mm}
%			\item[] $\varepsilon_y$: elasticity of substitution $E_t$ and $N_t$ \vspace{-2mm}
	\item[] $\tau_{Ft}$: carbon tax
	\vspace{-2mm}
			\item[] $\varepsilon_e$: elasticity of substitution $F_t$ and $G_t$
		\end{itemize}
	\end{minipage}
\end{frame}

\begin{comment}

\begin{frame}{1st: Effect of a carbon tax on \alert{fossil demand}}
	%	\begin{minipage}{0.4\textwidth}
		\begin{figure}[h]
			\vspace{4mm}
			\centering
			\begin{tikzpicture}[auto,scale=.7, transform shape]
				%			\node[circll] (A) at (0,16) {\textbf{Final}\textbf{ Good}}; 
				\node[circll] (B) at (-6,14) {\textbf{Energy}};
				%			\node[circll] (C) at (5,14) {\textbf{{Non-energy}}};
				\node[circll] (D) at (-10,12) {\textbf{{Fossil}}};
				\node[circll] (E) at (-2,12) {\textbf{{Green}}};
				\node[sphere] (Ems) at (-13,15) {\textbf{Emissions}};
				%		\node[modus] (DemF) at (-6, 9.2){
					%			\huge	${F_t} = \left(\frac{p_{Gt}}{p_{Ft}+\alert{\pmb{\tau_{Ft}}}}\right)^{\varepsilon_e}G_t$};
				
				
				\draw [->] (E) to node[pos=0.5, swap]{} (B);
				\draw [->] (D) to node[pos=0.5, swap]{} (B);
				\draw [->] (D) to node[pos=0.5, swap]{} (Ems);
			\end{tikzpicture}
		\end{figure}
		\begin{itemize}
			\item \alert{carbon tax lowers demand for fossil and raises demand for green energy}
			\item[]
		\end{itemize}
	\end{frame}
	
	content...
\end{comment}

		\begin{comment}
	\begin{frame}{2nd: Effect of a carbon tax on \alert{research activity}}
		%	\begin{minipage}{0.4\textwidth}
			\begin{figure}[h]
				\vspace{4mm}
				\centering
				\begin{tikzpicture}[auto,scale=.7, transform shape]
					\node[circll] (D) at (-10,10) {\textbf{Fossil}};
					\node[circll] (E) at (-4,10) {\textbf{Green}};
					
					\node[circll] (F) at (2,10) { \textbf{Non-energy}};
					
					\node[circll] (S) at (-4,6) {\textbf{{Scientists}}};
					
					\draw [->] (S) to node[pos=0.5, swap]{} (D);
					\draw [->] (S) to node[pos=0.5, swap]{} (E);
					\draw [->] (S) to node[pos=0.5, swap]{} (F);
				\end{tikzpicture}
			\end{figure}
			\pause
			\begin{itemize}
				\item \alert{carbon tax lowers revenues in fossil sector \ar returns to research in fossil sector $\downarrow$ \ar scientists shift from fossil to green sector}
			\end{itemize}

			\vspace{6mm}
			\hfill
			\hyperlink{effcarscie}{\tiny{$\rightarrow$ equations}}
		\end{frame}
	
	

			\begin{frame}{Machines and Labor}
				\begin{figure}[h]
					\vspace{-10mm}
					\centering
					\begin{tikzpicture}[auto,scale=.7, transform shape]
						\node[circll] (A) at (0,16) {\textbf{Final}\textbf{ Good}}; 
						\node[circll] (B) at (-6,14) {\textbf{Energy}};
						\node[circll] (C) at (5,14) {\textbf{{Non-energy}}};
						\node[circll] (D) at (-10,12) {\textbf{{Fossil}}};
						\node[circll] (E) at (-2,12) {\textbf{{Green}}};
						%			\node[sphere] (Ems) at (-11,16) {\textbf{CO$_2$}\\\textbf{Emissions}};
						
						\node[circllsmall] (CM) at (4.6,8) {\textbf{{Machines}}};
						\node[circllsmall] (CL) at (7.4,8) {\textbf{{Labor}}};			\node[circllsmall] (EM) at (-3.4,8) {\textbf{{Machines}}};
						\node[circllsmall] (EL) at (-0.6,8) {\textbf{{Labor}}};
						\node[circllsmall] (DM) at (-11.4,8) {\textbf{{Machines}}};
						\node[circllsmall] (DL) at (-8.6,8) {\textbf{{Labor}}};
						
						
						\draw [->] (B) to node[pos=0.5, swap]{} (A);
						\draw [->] (C) to node[pos=0.5, swap]{} (A);
						
						\draw [->] (E) to node[pos=0.5, swap]{} (B);
						\draw [->] (D) to node[pos=0.5, swap]{} (B);
						
						
						\draw [->] (EM) to node[pos=0.5, swap]{} (E);
						\draw [->] (EL) to node[pos=0.5, swap]{} (E);
						
						
						\draw [->] (DM) to node[pos=0.5, swap]{} (D);
						\draw [->] (DL) to node[pos=0.5, swap]{} (D);
						
						
						\draw [->] (CM) to node[pos=0.5, swap]{} (C);
						\draw [->] (CL) to node[pos=0.5, swap]{} (C);
						
						%			\draw [->] (D) to node[pos=0.5, swap]{} (Ems);
					\end{tikzpicture}
					
				\end{figure}
			\end{frame}
			
			\addtocounter{framenumber}{-1}
			
			\begin{frame}{Research}
				\begin{figure}[h]
					\vspace{-10mm}
					\centering
					\begin{tikzpicture}[auto,scale=.7, transform shape]
						\node[circllsmall] (E) at (-2,12) {\textbf{{Sector $J$}}};
						\node[circllsmall] (EM) at (-2,8) {\textbf{{\alert{Machines}}}};
						\node[circllsmall] (ES) at (-2,4) {\textbf{{\alert{Scientists}}}};
						
						\draw [->] (E) to node[pos=0.5, swap]{$\text{demand}(\underset{+}{A_J})$} (EM);
						\draw [->] (EM) to node[pos=0.5, swap]{$s_J^d$}(ES);
					\end{tikzpicture}
					
				\end{figure}
			\end{frame}
			
			
			content...
			\begin{frame}{Carbon tax: Effect on Research}
				\begin{figure}[h]
					\vspace{0mm}
					\centering
					\begin{tikzpicture}[auto,scale=.47, transform shape]
						%			\node[circll] (A) at (0,16) {\textbf{Final}\textbf{ Good}}; 
						%			\node[circll] (B) at (-6,14) {\textbf{Energy}};
						%			\node[circll] (C) at (5,14) {\textbf{{Non-energy}}};
						\node[circll] (D) at (-10,10) {\textbf{Fossil}};
						\node[circll] (E) at (-4,10) { \textbf{Green}};
						
						\node[circll] (F) at (2,10) {\ \textbf{Non-energy}};
						
						\node[circll] (S) at (-4,6) {\textbf{{Scientists}}};
						
						\draw [->] (S) to node[pos=0.5, swap]{} (D);
						\draw [->] (S) to node[pos=0.5, swap]{} (E);
						\draw [->] (S) to node[pos=0.5, swap]{} (F);
					\end{tikzpicture}
				\end{figure}
				\begin{itemize}
					\item $\tau_F \uparrow \Rightarrow p_F F \downarrow$ and  $p_G G \uparrow$
					\begin{align*}
						w_{sF}\left(\underbrace{p_FF}_{+},\underbrace{s_F}_{-}\right)\alert{\pmb{<}}	w_{sG}\left(\underbrace{p_GG}_{+},\underbrace{s_G}_{-}\right)
					\end{align*}
					\item[] % $s_F\downarrow$ and $s_G \uparrow$ in new equilibrium
				\end{itemize}
			\end{frame}
			\addtocounter{framenumber}{-1}
			\begin{frame}{Carbon tax: Effect on Research}
				\begin{figure}[h]
					\vspace{0mm}
					\centering
					\begin{tikzpicture}[auto,scale=.47, transform shape]
						%			\node[circll] (A) at (0,16) {\textbf{Final}\textbf{ Good}}; 
						%			\node[circll] (B) at (-6,14) {\textbf{Energy}};
						%			\node[circll] (C) at (5,14) {\textbf{{Non-energy}}};
						\node[circll] (D) at (-10,10) {\textbf{Machines }\\\textbf{Fossil}};
						\node[circll] (E) at (-4,10) {\textbf{Machines}\\ \textbf{Green}};
						
						\node[circll] (F) at (2,10) {\textbf{Machines}\\ \textbf{Non-energy}};
						
						\node[circll] (S) at (-4,6) {\textbf{{Scientists}}};
						
						\draw [->] (S) to node[pos=0.5, swap]{} (D);
						\draw [->] (S) to node[pos=0.5, swap]{} (E);
						\draw [->] (S) to node[pos=0.5, swap]{} (F);
					\end{tikzpicture}
				\end{figure}
				\begin{itemize}
					\item $\tau_F \uparrow \Rightarrow p_F F \downarrow$ and  $p_G G \uparrow$
					\begin{align*}
						w_{sF}\left(\underbrace{p_FF}_{+},\underbrace{s_F}_{-}\right)\alert{\pmb{<}}	w_{sG}\left(\underbrace{p_GG}_{+},\underbrace{s_G}_{-}\right)
					\end{align*}
					\item[\ar] $s_F\downarrow$ and $s_G \uparrow$ in new equilibrium
				\end{itemize}
			\end{frame}
			
			
			\begin{frame}{Carbon tax: 2. Effect on Research}
				\vspace{-7mm}
				In equilibrium: \large
				\begin{align*}
					\overbrace{{\psi_F} \underbrace{p_F{F}}_{\tau_F\uparrow\Rightarrow\downarrow}\frac{\partial A_{F}}{\partial s_{F}}}^{\text{wage fossil scientists}}=\overbrace{{\psi_G} \underbrace{p_G{G}}_{\tau_F\uparrow\Rightarrow\uparrow}\frac{\partial A_{G}}{\partial s_{G}}}^{\text{wage green scientists}}
				\end{align*}
				\normalsize
				\begin{itemize}
					\item carbon tax lowers returns to fossil research and raises returns to green research
					\item in equilibrium: scientists transition from fossil to green sector
				\end{itemize}
				\small
				\vspace{0mm}
				\hspace{-4mm}
				\begin{minipage}[t!]{0.3\textwidth}
					\vspace{0mm}
					\begin{itemize}
						\item[] $p_JJ$: revenues sector J
					\end{itemize}
				\end{minipage}
				\vspace{-5mm}
				\begin{minipage}[t!]{0.5\textwidth}
					\vspace{0mm}
					\begin{itemize}	
						\item[] $s_J$: scientists sector J
					\end{itemize}
				\end{minipage}
			\end{frame}
			
		\end{comment}
		
		%\begin{frame}
		%	Having seen how the carbon tax affects allocations, how does it serve to meet government goals?
		%\end{frame}
\begin{comment}
	content...		
		\begin{frame}{In a nutshell: Government trade-off and instruments}
			\pause
			\begin{itemize}[<+-| alert@+>]
				\setbeamercolor{alerted text}{} 
				\setbeamerfont{alerted text}{}
				\item 	Goal of government intervention
				\begin{enumerate}
					\item[a)] lower emissions
					\item[b)] keep productivity high
				\end{enumerate}
				\vspace{3mm}
				\item Carbon tax
				\begin{enumerate}
					\item[a)] reduces emissions by lowering fossil demand
					\item[b)] directs research across sectors
					\begin{itemize}
						\item[-] if want to foster green research
						\ar higher carbon tax \ar % but reduces returns to labor %\ar
						costly in terms of output % reduces share of fossil energy 
						\item[-] if want to foster fossil research \ar smaller carbon tax \ar but too high emissions
					\end{itemize}
				\end{enumerate}
				\item Labor income tax can be used to counter side effects of carbon tax 
			\end{itemize}
		\end{frame}

\end{comment}


\begin{frame}{\hyperlink{modma}{Innovation}}
	\hypertarget{backinnov}{}
	\vspace{-2mm}
	\begin{itemize}[<+->]
		\item[-] research raises productivity of machine $i$ in sector $J\in\{\text{Green},\text{Fossil},\text{Non-energy}\}$ \vspace{-1mm}
		\item[-] machine producers invest in research to increase profits \small{(monopolistic competition)}
		\vspace{-1mm}
		\normalsize
	%	\item[-] inefficiencies: one period patents
	\end{itemize}
\pause
	\vspace{-1.4mm}
	% talk about productivity of research bcs it determines optimal allocation of research
	\large
	\begin{align*}
		A_{Jit}={A_{Jt-1}}\left(1+\gamma\left(\frac{s_{Jit}}{\rho_J}\right)^\eta\left(\frac{A_{t-1}}{A_{Jt-1}}\right)^\phi\right)
	\end{align*}
	\normalsize
	\vspace{-1.9mm}
	\begin{enumerate}
		% \alert{within-sector knowledge spillovers} %: sector-specific knowledge renders scientists more productive} % \footnotesize{one-period patents} % due to patent structure not taken into account by machine producers when demanding research. 
	%		\begin{itemize}
		%	\item<+-> scientists in most advanced, fossil sector are more productive
		%			\item<+-> shift to green research early on to make green research more productive tomorrow
		%		\end{itemize}
	\item[] \  % knowledge from other sectors increases productivity of scientists
	\vspace{-1mm}
	\item[] \  % decreasing returns to research, $\eta<1$
	\vspace{-1mm}
	\item[] \
\end{enumerate}
\small
\vspace{4mm}
\hspace{-2mm}
\begin{minipage}[t!]{0.46\textwidth}
	\vspace{0mm}
	\begin{itemize}
		\item[] $A_{xt}$: technology/knowledge on level $x$
		\vspace{-7mm}		
		\item[] $s_{Jit}$: scientists firm $i$ sector $J$
		\vspace{-2mm}
		\item[] $\gamma$ : productivity of scientists
	\end{itemize}
\end{minipage}
\hspace{-5mm}
\vspace{-5mm}
\begin{minipage}[t!]{0.54\textwidth}
	\vspace{0mm}
	\begin{itemize}	
		\item[] {$\rho_J$: number of research processes in sector $J$}
		\vspace{-2mm}			
		\item[] $\eta$ : returns to research
		\vspace{-2mm}			
		\item[] $\phi$ : relative importance knowledge spillovers
	\end{itemize}
\end{minipage}
\end{frame}

		\addtocounter{framenumber}{-1}
\begin{frame}{\hyperlink{modma}{Innovation}}
	\vspace{-2mm}
	\begin{itemize}
		\item[-] research raises productivity of machine $i$ in sector $J\in\{\text{Green},\text{Fossil},\text{Non-energy}\}$ \vspace{-1mm}
		\item[-] machine producers invest in research to increase profits \small{(monopolistic competition)}
		\vspace{-1mm}
		\normalsize
	%	\item[-] inefficiencies: one period patents
	\end{itemize}
	\vspace{-1.4mm}
	% talk about productivity of research bcs it determines optimal allocation of research
	\large
	\begin{align*}
		A_{Jit}=\alert{A_{Jt-1}}\left(1+\gamma\left(\frac{s_{Jit}}{\rho_J}\right)^\eta\left(\frac{A_{t-1}}{A_{Jt-1}}\right)^\phi\right)
	\end{align*}
	\normalsize
	\vspace{-1.9mm}
	\begin{enumerate}
		\item[-] \alert{within-sector knowledge spillovers} \vspace{-1mm} %: sector-specific knowledge renders scientists more productive} % \footnotesize{one-period patents} % due to patent structure not taken into account by machine producers when demanding research. 
		%		\begin{itemize}
			%	\item<+-> scientists in most advanced, fossil sector are more productive
			%			\item<+-> shift to green research early on to make green research more productive tomorrow
			%		\end{itemize}
	%	\item[] \  % knowledge from other sectors increases productivity of scientists
		\item[] \  % decreasing returns to research, $\eta<1$
		\vspace{-1mm}
		\item[] \
	\end{enumerate}
	\small
\vspace{4mm}
\hspace{-2mm}
\begin{minipage}[t!]{0.46\textwidth}
	\vspace{0mm}
	\begin{itemize}
		\item[] $A_{xt}$: technology/knowledge on level $x$
		\vspace{-7mm}		
		\item[] $s_{Jit}$: scientists firm $i$ sector $J$
		\vspace{-2mm}
		\item[] $\gamma$ : productivity of scientists
	\end{itemize}
\end{minipage}
\hspace{-5mm}
\vspace{-5mm}
\begin{minipage}[t!]{0.54\textwidth}
	\vspace{0mm}
	\begin{itemize}	
		\item[] {$\rho_J$: number of research processes in sector $J$}
		\vspace{-2mm}			
		\item[] $\eta$ : returns to research
		\vspace{-2mm}			
		\item[] $\phi$ : relative importance knowledge spillovers
	\end{itemize}
\end{minipage}
\end{frame}


%\addtocounter{framenumber}{-1}
%\begin{frame}{\hyperlink{modma}{Innovation}}
%	\vspace{-2mm}
%\begin{itemize}
%	\item[-] machine producers invest in research to increase profits \small{(monopolistic competition)}
%	\vspace{-1mm}
%	\normalsize
%	\item[-] inefficiencies: one period patents
%\end{itemize}
%\vspace{-2.5mm}
%	\large
%	\begin{align*}
%		A_{Jit}={A_{Jt-1}}\left(1+\gamma\alert{\left(\frac{s_{Jit}}{\rho_J}\right)^{\eta}}{\left(\frac{A_{t-1}}{A_{Jt-1}}\right)^\phi}\right)
%	\end{align*}
%	\normalsize
%	\vspace{-1.4mm}
%	\begin{enumerate}
%		\item[-] within-sector knowledge spillovers %: sector-specific knowledge renders scientists more productive
%		\item[-] \alert{decreasing returns to research, $\eta<1$}
%		\item[]   % \alert{knowledge spillovers} 
%	\end{enumerate}
%	\small
%	\vspace{4mm}
%	\hspace{-2mm}
%	\begin{minipage}[t!]{0.43\textwidth}
%		\vspace{0mm}
%		\begin{itemize}
%			\item[] $A_{xt}$: firm/aggregate technology
%			\vspace{-2mm}		
%			\item[] $s_{Jit}$: scientists firm $i$ sector $J$
%			\vspace{-2mm}
%			\item[] $\gamma$ : productivity of scientists
%		\end{itemize}
%	\end{minipage}
%	\vspace{-5mm}
%	\begin{minipage}[t!]{0.55\textwidth}
%		\vspace{0mm}
%		\begin{itemize}	
%			\item[] \alert{$\rho_J$: number of research processes in sector $J$}
%			\vspace{-2mm}			
%			\item[] $\eta$ : returns to research
%			\vspace{-2mm}			
%			\item[] $\phi$ : relative importance knowledge spillovers
%		\end{itemize}
%	\end{minipage}
%\end{frame}


\addtocounter{framenumber}{-1}
\begin{frame}{\hyperlink{modma}{Innovation}}
	\vspace{-2mm}
	\begin{itemize}
		\item[-] research raises productivity of machine $i$ in sector $J\in\{\text{Green},\text{Fossil},\text{Non-energy}\}$ \vspace{-1mm}
		\item[-] machine producers invest in research to increase profits \small{(monopolistic competition)}
		\vspace{-1mm}
		\normalsize
	\end{itemize}
	\vspace{-1.4mm}
	\large
	\begin{align*}
		A_{Jit}={A_{Jt-1}}\left(1+\gamma{\left(\frac{s_{Jit}}{\rho_J}\right)^{\eta}}\alert{\left(\frac{A_{t-1}}{A_{Jt-1}}\right)^\phi}\right)
	\end{align*}
	\normalsize
	\vspace{-1.9mm}
	\begin{enumerate}
		\item[-] within-sector knowledge spillovers \vspace{-1mm}
		\item[-]  \alert{cross-sectoral knowledge spillovers} \vspace{-1mm}%: sector-specific
		\item[] \
	\end{enumerate}
	\small
\vspace{4mm}
\hspace{-2mm}
\begin{minipage}[t!]{0.46\textwidth}
	\vspace{0mm}
	\begin{itemize}
		\item[] $A_{xt}$: technology/knowledge on level $x$
		\vspace{-7mm}		
		\item[] $s_{Jit}$: scientists firm $i$ sector $J$
		\vspace{-2mm}
		\item[] $\gamma$ : productivity of scientists
	\end{itemize}
\end{minipage}
\hspace{-5mm}
\vspace{-5mm}
\begin{minipage}[t!]{0.54\textwidth}
	\vspace{0mm}
	\begin{itemize}	
		\item[] {$\rho_J$: number of research processes in sector $J$}
		\vspace{-2mm}			
		\item[] $\eta$ : returns to research
		\vspace{-2mm}			
		\item[] $\phi$ : relative importance knowledge spillovers
	\end{itemize}
\end{minipage}\end{frame}


\addtocounter{framenumber}{-1}
\begin{frame}{\hyperlink{modma}{Innovation}}
	\vspace{-2mm}
	\begin{itemize}
		\item[-] research raises productivity of machine $i$ in sector $J\in\{\text{Green},\text{Fossil},\text{Non-energy}\}$ \vspace{-1mm}
		\item[-] machine producers invest in research to increase profits \small{(monopolistic competition)}
		\vspace{-1mm}
		\normalsize
	\end{itemize}
	\vspace{-1.4mm}
	\large
	\begin{align*}
		A_{Jit}={A_{Jt-1}}\left(1+\gamma{\left(\frac{s_{Jit}}{\rho_J}\right)^{\eta}}\left(\frac{A_{t-1}}{A_{Jt-1}}\right)^\phi\right)
	\end{align*}
	\normalsize
	\vspace{-1.9mm}
	\begin{enumerate}
		\item[-] within-sector knowledge spillovers \vspace{-1mm}
		\item[-]  {cross-sectoral knowledge spillovers} \vspace{-1mm}
		\item[-] \alert{inefficiencies: one-period patents}
	\end{enumerate}
	\small
	\vspace{4mm}
	\hspace{-2mm}
\begin{minipage}[t!]{0.46\textwidth}
	\vspace{0mm}
	\begin{itemize}
		\item[] $A_{xt}$: technology/knowledge on level $x$
		\vspace{-7mm}		
		\item[] $s_{Jit}$: scientists firm $i$ sector $J$
		\vspace{-2mm}
		\item[] $\gamma$ : productivity of scientists
	\end{itemize}
\end{minipage}
\hspace{-5mm}
\vspace{-5mm}
\begin{minipage}[t!]{0.54\textwidth}
	\vspace{0mm}
	\begin{itemize}	
		\item[] {$\rho_J$: number of research processes in sector $J$}
		\vspace{-2mm}			
		\item[] $\eta$ : returns to research
		\vspace{-2mm}			
		\item[] $\phi$ : relative importance knowledge spillovers
	\end{itemize}
\end{minipage}
\end{frame}


\begin{frame}{2nd: Carbon tax effect on \alert{allocation of scientists}}
	\hypertarget{effcarscie}{}
	\begin{itemize}
		\item[-] free movement of scientists \ar in equilibrium equal wages
	\end{itemize}
\pause
	\vspace{0mm}
	\large
	\begin{align*}		
		\color{black}
		\overbrace{{\psi_F} \textcolor{black!1}{\underbrace{\textcolor{black}{p_{Ft}{F_t}}}_{{{\tau_{Ft}\uparrow\Rightarrow\downarrow}}}}\textcolor{black!1}{\underbrace{\textcolor{black}{\frac{\partial A_{Ft}}{\partial s_{Ft}}}}_{{{s_{Ft}\downarrow\Rightarrow\uparrow}}}}}^{\text{wage fossil scientists}}{\pmb{=}}\overbrace{{\psi_G} \textcolor{black!1}{\underbrace{\textcolor{black}{p_{Gt}G_t}}_{{{\tau_{Ft}\uparrow\Rightarrow\uparrow}}}}\textcolor{black!1}{\underbrace{\textcolor{black}{\frac{\partial A_{Gt}}{\partial s_{Gt}}}}_{	{{s_{Gt}\uparrow\Rightarrow\downarrow}}}}}^{\text{wage green scientists}}
		%		\overbrace{{\psi_F} p_F{F}\frac{\partial A_{F}}{\partial s_{F}}}^{\text{wage fossil scientists}}=\overbrace{{\psi_G} p_G{G}\frac{\partial A_{G}}{\partial s_{G}}}^{\text{wage green scientists}}
	\end{align*}
	\normalsize
		\vspace{-2mm}
	\begin{itemize}
		\item[] \ % carbon tax lowers returns to fossil research and raises returns to green research
		\vspace{2mm}
		\item[] \  % in equilibrium: scientists transition from fossil to green sector \small{(decreasing returns to research)}
	\end{itemize}
	\small
	\vspace{4mm}
	\hspace{-2mm}
	\begin{minipage}[t!]{0.4\textwidth}
		\vspace{0mm}
		\begin{itemize}
			\item[] $p_{Jt}J_t$: revenues sector $J$
			\vspace{-2mm}
			\item[] $\psi_{J}$ : sector-specific constant
		\end{itemize}
	\end{minipage}
	\vspace{-5mm}
	\begin{minipage}[t!]{0.5\textwidth}
		\vspace{0mm}
		\begin{itemize}	
			\item[] $A_{Jt}$: productivity sector $J$
			\vspace{-2mm}			
			\item[] $s_{Jt}$ : scientists sector $J$
		\end{itemize}
	\end{minipage}
	
	%	\vspace{8mm}
	%	\hfill
	%	\hyperlink{backinnov}{\tiny{$\rightarrow$ back}}
\end{frame}

\addtocounter{framenumber}{-1}

\begin{frame}{2nd: Carbon tax effect on \alert{allocation of scientists}}
		\begin{itemize}
		\item[-] free movement of scientists \ar in equilibrium equal wages
	\end{itemize}
	\vspace{0mm}
	\large
	\begin{align*}
		\color{black}
		\overbrace{{\psi_F} \underbrace{p_{Ft}{F_t}}_{{\alert{\tau_{Ft}\uparrow\Rightarrow\downarrow}}}\textcolor{black!1}{\underbrace{\textcolor{black}{\frac{\partial A_{Ft}}{\partial s_{Ft}}}}_{{{s_{Ft}\downarrow\Rightarrow\uparrow}}}}}^{\text{wage fossil scientists}}\alert{\pmb{<}}\overbrace{{\psi_G} \underbrace{p_{Gt}{G_t}}_{{\alert{\tau_{Ft}\uparrow\Rightarrow\uparrow}}}\textcolor{black!1}{\underbrace{\textcolor{black}{\frac{\partial A_{Gt}}{\partial s_{Gt}}}}_{	{{s_{Gt}\uparrow\Rightarrow\downarrow}}}}}^{\text{wage green scientists}}
		%		\overbrace{{\psi_F} \underbrace{p_F{F}}_{\alert{\pmb{\tau_F\uparrow\Rightarrow\downarrow}}}\frac{\partial A_{F}}{\partial s_{F}}}^{\text{wage fossil scientists}}\alert{\pmb{<}}\overbrace{{\psi_G} \underbrace{p_G{G}}_{\alert{\pmb{\tau_F\uparrow\Rightarrow\uparrow}}}\frac{\partial A_{G}}{\partial s_{G}}}^{\text{wage green scientists}}
	\end{align*}
	\normalsize
	\vspace{-2mm}
	\begin{itemize}
		\item[-] carbon tax lowers wages of fossil researchers and raises wages of green researchers
		\vspace{1mm}
		\item[] \  % in equilibrium: scientists transition from fossil to green sector \small{(decreasing returns to research)}
	\end{itemize}
	\small
	\vspace{4mm}
	\hspace{-2mm}
	\begin{minipage}[t!]{0.4\textwidth}
		\vspace{0mm}
		\begin{itemize}
			\item[] $p_{Jt}J_t$: revenues sector $J$
			\vspace{-2mm}
			\item[] $\psi_J$ : sector-specific constant
		\end{itemize}
	\end{minipage}
	\vspace{-5mm}
	\begin{minipage}[t!]{0.5\textwidth}
		\vspace{0mm}
		\begin{itemize}	
			\item[] $A_{Jt}$: productivity sector $J$
			\vspace{-2mm}			
			\item[] $s_{Jt}$ : scientists sector $J$
		\end{itemize}
	\end{minipage}
	
	%	\vspace{8mm}
	%	\hfill
	%	\hyperlink{backinnov}{\tiny{$\rightarrow$ back}}
\end{frame}

\addtocounter{framenumber}{-1}

\begin{frame}{2nd: Carbon tax effect on \alert{allocation of scientists}}
		\begin{itemize}
		\item[-] free movement of scientists \ar in equilibrium equal wages
	\end{itemize}
	\vspace{0mm}
	\large
	\begin{align*}
		\overbrace{{\psi_F} \underbrace{p_{Ft}{F_t}}_{\tau_{Ft}\uparrow\Rightarrow\downarrow}\underbrace{\frac{\partial A_{Ft}}{\partial s_{Ft}}}_{\alert{{s_{Ft}\downarrow\Rightarrow\uparrow}}}}^{\text{wage fossil scientists}}\alert{\pmb{=}}\overbrace{{\psi_G} \underbrace{p_{Gt}{G_t}}_{\tau_{Ft}\uparrow\Rightarrow\uparrow}\underbrace{\frac{\partial A_{Gt}}{\partial s_{Gt}}}_{\alert{{s_{Gt}\uparrow\Rightarrow\downarrow}}}}^{\text{wage green scientists}}
	\end{align*}
	\normalsize
	\vspace{-2mm}
	\begin{itemize}
		\item[-] carbon tax lowers wages of fossil researchers and raises wages of green researchers
		\vspace{1mm}
		\item[-] scientists transition from fossil to green sector \small{(decreasing returns to research)}
	\end{itemize}
	\small
	\vspace{4mm}
	\hspace{-2mm}
	\begin{minipage}[t!]{0.4\textwidth}
		\vspace{0mm}
		\begin{itemize}
			\item[] $p_{Jt}J_t$: revenues sector $J$
			\vspace{-2mm}
			\item[] $\psi_J$ : sector-specific constant
		\end{itemize}
	\end{minipage}
	\vspace{-5mm}
	\begin{minipage}[t!]{0.5\textwidth}
		\vspace{0mm}
		\begin{itemize}	
			\item[] $A_{Jt}$: productivity sector $J$
			\vspace{-2mm}			
			\item[] $s_{Jt}$ : scientists sector $J$
		\end{itemize}
	\end{minipage}
	
	%	\vspace{8mm}
	%	\hfill
	%	\hyperlink{backinnov}{\tiny{$\rightarrow$ back}}
\end{frame}



%%%%%%%%%%%%%%%%%%%%%%%%%%%%%%%%%%%%
%% Calibration 
%%%%%%%%%%%%%%%%%%%%%%%%%%%%%%%%%%%%
\section{Calibration to the US}
\begin{frame}{Emission limit}
%	\vspace{-1mm}
%	\begin{itemize}
%		\item<+->  \textbf{global} CO$_2$ emissions consistent with $1.5^\circ$C climate target \footnotesize{\citep{IPCC2022}}\normalsize:
%		\vspace{1mm}
%		\begin{itemize}
%			\item[-] before 2050: remaining carbon budget 510Gt
%			\item[-] from 2050 onward: net-zero  emissions
%		\end{itemize}
%		\vspace{0mm}
%		\item<+-> \textbf{equal} distribution of  reduction burden across countries %\footnotesize{\citep{RobiouDuPont2017EquitableGoals}}
%	\end{itemize}
%	\vspace{-2mm}
%	\pause
	\begin{center}
		\hspace{-20mm}
		\begin{minipage}{0.8\textwidth}
			\begin{figure}
				\caption{Net CO$_2$ emission limit in Gt}
				\includegraphics[width=0.7\textwidth]{../codding_model/own_basedOnFried/optimalPol_010922_revision/figures/all_13Sept22_Tplus30/Emnet_goals_o0_lgd0.png}
			\end{figure}
		\end{minipage}
		\hspace{-20mm}
		\begin{minipage}{0.3\textwidth}
			\begin{figure}
				\includegraphics[width=1.4\textwidth]{../codding_model/own_basedOnFried/optimalPol_010922_revision/figures/all_13Sept22_Tplus30/Emnet_goals_o0_lgd1_crop.png}
			\end{figure}
		\end{minipage}
	\end{center}

\vspace{-1.7mm}
\hfill	\hyperlink{emsall}{\tiny{$\rightarrow$ alternative emission targets,}} 		\hyperlink{calib}{\tiny{$\rightarrow$ parameters}}
\end{frame}

	%%%%%%%%%%%%%%%%%%%%%%%%%%%%%%%
	%% Results
	%%%%%%%%%%%%%%%%%%%%%%%%%%%%%%%
	\hypertarget{resback}{}
	\section{Results}
\begin{frame}{First-best and optimal allocation}
	\pause
	\begin{figure}[h!!]
		\centering
		\begin{subfigure}{0.45\textwidth}		
			\caption{{Green-to-fossil energy ratio}}
			%	\captionsetup{width=.45\linewidth}
			\includegraphics[width=1\textwidth]{../codding_model/own_basedOnFried/optimalPol_010922_revision/figures/all_13Sept22/NewCalib_eff_noBau_T_GFF_Sun2_emnet1_spillover0_knspil3_xgr0_nsk0_sep0_extern0_PV1_etaa0.79_lgd0.png}
		\end{subfigure}
		\begin{minipage}[]{0.05\textwidth}
			\
		\end{minipage}
		\begin{subfigure}{0.45\textwidth}		
			\caption{{Fossil-to-green scientists}}
			%	\captionsetup{width=.45\linewidth}
			\includegraphics[width=1\textwidth]{../codding_model/own_basedOnFried/optimalPol_010922_revision/figures/all_13Sept22/NewCalib_eff_noBau_T_sffsg_Sun2_emnet1_spillover0_knspil3_xgr0_nsk0_sep0_extern0_PV1_etaa0.79_lgd0.png}
		\end{subfigure}
	\end{figure}
		\vspace{7.7mm}
	%\begin{block}{}
	%	\begin{itemize}
		%		\item First-best: more fossil research and higher green-to-fossil energy ratio
		%		\item[] % \alert{Government dilemma:  more fossil research comes with higher fossil energy use}
		%	\end{itemize}
	%\end{block}	
\end{frame}

\addtocounter{framenumber}{-1}
\begin{frame}{First-best and optimal allocation}
	\begin{figure}[h!!]
		\centering
		\begin{subfigure}{0.45\textwidth}		
			\caption{Green-to-fossil energy ratio}
			%	\captionsetup{width=.45\linewidth}
			\includegraphics[width=1\textwidth]{../codding_model/own_basedOnFried/optimalPol_010922_revision/figures/all_13Sept22/NewCalib_effBauOpt_noBau_T_GFF_Sun2_emnet1_spillover0_knspil3_xgr0_nsk0_sep0_extern0_PV1_etaa0.79_lgd1.png}
		\end{subfigure}
		\begin{minipage}[]{0.05\textwidth}
			\
		\end{minipage}
		\begin{subfigure}{0.45\textwidth}		
			\caption{{Fossil-to-green scientists}}
			%	\captionsetup{width=.45\linewidth}
			\includegraphics[width=1\textwidth]{../codding_model/own_basedOnFried/optimalPol_010922_revision/figures/all_13Sept22/NewCalib_effBauOpt_noBau_T_sffsg_Sun2_emnet1_spillover0_knspil3_xgr0_nsk0_sep0_extern0_PV1_etaa0.79_lgd0.png}
		\end{subfigure}
	\end{figure}
	\vspace{1mm}
	\begin{block}{}
		\begin{itemize}
			%\item First-best: more fossil research and higher green-to-fossil energy ratio
			\item {Government dilemma: reducing fossil fuels entails too strong a decline in fossil research activity}
		\end{itemize}
	\end{block}	
\vspace{-5.7mm}
\hfill
\hyperlink{compfb}{\tiny{$\rightarrow$ BAU}}
\end{frame}


	\begin{frame}{Optimal Policy}
		\hypertarget{backop}{}
		\pause
		\vspace{-3mm}
		\begin{figure}[h!!]
			\begin{subfigure}{0.45\textwidth}		
				\caption{Tax per ton of carbon in US\$, $\tau_{Ft}$}
				%	\captionsetup{width=.45\linewidth}
				\includegraphics[width=1\textwidth]{../codding_model/own_basedOnFried/optimalPol_010922_revision/figures/all_13Sept22/Single_NC_T_Tauf_emnet1_Sun2_regime4_spillover0_knspil3_noskill0_sep0_xgrowth0_extern0_PV1_sizeequ0_GOV0_etaa0.79.png}
			\end{subfigure}	
			\begin{minipage}[]{0.05\textwidth}
				\ 
			\end{minipage}
			\begin{subfigure}{0.45\textwidth}		
				\caption{Marginal income tax rate in \%, $\tau_{\iota t}$}
				%	\captionsetup{width=.45\linewidth}
				\includegraphics[width=1\textwidth]{../codding_model/own_basedOnFried/optimalPol_010922_revision/figures/all_13Sept22/Single_NC_T_dTaulAv_emnet1_Sun2_regime4_spillover0_knspil3_noskill0_sep0_xgrowth0_extern0_PV1_sizeequ0_GOV0_etaa0.79.png}
			\end{subfigure}
		\end{figure}
		\vspace{3mm}
		\pause
		\begin{block}{}
			\begin{itemize}
				\item Labor income tax used to subsidize research
				\item \alert{How does the government use the labor income tax to meet emission targets?}
			\end{itemize}
		\end{block}	
		\vspace{-5.7mm}
		\hfill
		\hyperlink{Redis}{\tiny{$\rightarrow$ redistribution,}}
\hyperlink{compfb}{\tiny{$\rightarrow$ comparison first best}}
\hypertarget{backOPT}{}
	\end{frame}

	
\begin{comment}
	content...
\begin{frame}{Quantitative experiment}
	\alert{\textbf{How does the government use the labor income tax to meet emission targets?}}
	\pause 
	\begin{itemize}[<+->]
		\item Problem:
		\begin{itemize}
			\item[-] Optimal policy without emission target taxes labor in order to subsidize research
			\item[-] Comparing carbon-tax-only policy to a scenario with labor and carbon tax does not isolate motive to meet emission target %Mechanical reduction of required carbon tax not driven by motive to lower emissions \ar cannot compare carbon-tax-only policy to benchmark policy
			%			\item But: I want to study policy responses to the emission target
		\end{itemize}
		\item Solution: 
		\begin{itemize}
			\item[-] Compare optimal policy with labor tax fixed at optimal level without emission target (counterfactual) to optimal policy with flexible labor tax (integrated policy)
		\end{itemize}
	\end{itemize}
\end{frame}

\end{comment}
	

\begin{frame}{Integrating labor income taxes into the environmental policy}
	\pause
	\vspace{-3mm}
	\centering
	\begin{figure}[h!!]
		\centering
		
		\begin{subfigure}{0.45\textwidth}		
			\caption{{Marginal income tax rate in \%, $\tau_{\iota t}$}}
			%	\captionsetup{width=.45\linewidth}
			%	\includegraphics[width=1\textwidth]{../codding_model/own_basedOnFried/optimalPol_010922_revision/figures/all_13Sept22/NewCalib_polTaulFixed_T_dTaulAvS_Sun2_emnet1_spillover0_knspil3_xgr0_nsk0_sep0_extern0_PV1_etaa0.79_lgd1.png}
			\includegraphics[width=1\textwidth]{../codding_model/own_basedOnFried/optimalPol_010922_revision/figures/all_13Sept22/NewCalib_pol_TvsNoT_dTaulAv_base_emnet1_Sun2_spillover0_knspil3_xgr0_nsk0_sep0_extern0_PV1_etaa0.79_lgd1.png}
		\end{subfigure}
		\begin{minipage}[]{0.05\textwidth}
			\
		\end{minipage}
	\pause
		\begin{subfigure}{0.45\textwidth}		
			\caption{{Deviation carbon tax in \%, $\tau_{Ft}$}}
			%	\captionsetup{width=.45\linewidth}
			\includegraphics[width=1\textwidth]{../codding_model/own_basedOnFried/optimalPol_010922_revision/figures/all_13Sept22/NewCalib_polTaulFixedPer_T_Tauf_Sun2_emnet1_spillover0_knspil3_xgr0_nsk0_sep0_extern0_PV1_etaa0.79.png}
		\end{subfigure}
	\end{figure}
	\vspace{3mm}	
	\begin{block}{}
		\begin{itemize}
			%				\item integrated policy achieves more beneficial allocation of scientists \item at the costs of lower production in initial years
			\item In run-up to net-zero limit: policy adjustment to boost fossil research
			\item Under net-zero limit: policy adjustment to foster green research
		\end{itemize}
	\end{block}	
	\vspace{-5.7mm}
	\hfill
	\hyperlink{sensphi}{\tiny{$\rightarrow$ sensitivity,}} 
	\hyperlink{eff}{\tiny{$\rightarrow$ effect and decomposition}}
	\hypertarget{backmec}{}
\end{frame}

		
	\hypertarget{conc}{}
	\section{Conclusion}
	\begin{frame}{Conclusion}
		\begin{itemize}[<+-| alert@+>]
			\setbeamercolor{alerted text}{} %change the font color
			\setbeamerfont{alerted text}{}
			\item I study the optimal mix of taxes on carbon and income to meet emission targets
			\vspace{3mm}
			\item Labor income taxes complement carbon taxes to target the direction of research:
			\vspace{2mm}
			\begin{itemize}
				\item[-] Before the net-zero emission limit: 
				\begin{itemize}
					\item[-] lower carbon tax to maintain some fossil research
					\item[-] a higher tax on labor reduces emissions
				\end{itemize}
				\vspace{3mm}
				\item[-] Under the net-zero emission limit: 
				\begin{itemize}
					\item[-] high carbon tax to increase green research
					\item[-]  a smaller tax on labor stabilizes output
				\end{itemize}
			\end{itemize}
			\vspace{3mm}
			\item Small effect \ar introduce (green/fossil) research subsidies e.g. by using carbon tax revenues
			\vspace{2mm}
			
			\begin{itemize}
				\item[-] \textbf{Outlook}: role for labor income tax remains since carbon tax revenues are not redistributed lump-sum \ar labor market distortion
				%			\item[-] when only green research subsidies are available: green transition even more costly
			\end{itemize}	
			
			
		\end{itemize}
	\end{frame}
		
		\appendix
		
		\begin{frame}[shrink]{References}
			
			\bibliography{../../bib_2_0}
			\bibliographystyle{apa}
		\end{frame}
		
		\section*{Model}
		
	\begin{frame}{Representative household}
		\hypertarget{modhh}{}
		%\text{\textbf{Householdrt}}
		\vspace{2mm}
		\begin{minipage}[t!]{1\textwidth}
			\begin{align*}
				%	\tikzmarkin{first0}(1.5,2.7)(-1.2,-2.5)
				%	\underset{c_{s,i},c_{n,i}, l_i}{\max} \ \hspace{2mm} U(c_{s,i}, c_{n,i}, l_i; h_n)= 
				\max_{C_t, H_{t}, S_{t}} \log(C_t)-\chi\frac{H_{t}^{1+\sigma}}{1+\sigma}-\chi_s\frac{S_{t}^{1+\sigma}}{1+\sigma}
				\\
				\vspace{4mm}
				\\
				\text{s.t.}\ C_t=\lambda_t\left(w_{t}H_{t}\right)^{(\alert{\pmb{1-\tau_{\iota t}}})}+\lambda_t \left(w_{st}S_t\right)^{(\alert{\pmb{1-\tau_{\iota t}}})}+T_t%+Gov_t
				%\\
				%\hspace{2mm}\ H_{t}\leq \bar{H}; \hspace{4mm} S_{t}\leq \bar{H}
				%	\tikzmarkend{first0}
			\end{align*}
		\end{minipage}
		
		\small
		\vspace{4mm}
		\hspace{-8mm}
		\begin{minipage}[t!]{0.26\textwidth}
			\vspace{7mm}
			\begin{itemize}
				\item[] $C_{t}$: consumption\vspace{-2mm}
				\item[] $H_{t}$: hours workers\vspace{-2mm}
				\item[] $S_{t}$: hours scientists\vspace{-2mm}
				\item[] $w_{t}, w_{st}$: wages  %\vspace{-2mm}
			\end{itemize}
		\end{minipage}
		\begin{minipage}[t!]{0.37\textwidth}
			\vspace{8mm}
			\begin{itemize}
				\item[] $\lambda_{t}$: scale income tax scheme  \vspace{-2mm}
				\item[] $\tau_{\iota t}$: income tax progressivity
				\vspace{-2mm}	
				\item[] $T_{t}$: government transfers
				\vspace{-2mm}	
				\item[]%	$Gov_{t}$: government transfers
			\end{itemize}
		\end{minipage}
		\begin{minipage}[t!]{0.39\textwidth}
			\vspace{8mm}
			\begin{itemize}
				\item[] $\sigma$: curvature disutility of labor  \vspace{-2mm}
				\item[] $\chi$: disutility of work
				\vspace{-2mm}	
				\item[] $\chi_s$: disutility of research
				\vspace{-2mm}	
				\item[]%	$Gov_{t}$: government transfers
			\end{itemize}
		\end{minipage}
	
		\vspace{-2mm}
		\hfill \hyperlink{backScheme}{\tiny{$\rightarrow$ back}}
	\end{frame}

	\begin{frame}{Production: final and energy good}
		\vspace{-10mm}
		\hypertarget{prodmod}{}
		\begin{align*}
			%		\tikzmarkin{first}(1.3,1.2)(-1,-0.8)
			\text{Final good}\hspace{4mm}&Y_t =\left(\delta_y^{\frac{1}{\varepsilon_y}}E_t^\frac{\varepsilon_y-1}{\varepsilon_y}+(1-\delta_y)^{\frac{1}{\varepsilon_y}}N_t^\frac{\varepsilon_y-1}{\varepsilon_y}\right)^\frac{\varepsilon_y}{\varepsilon_y-1} \\
			\ \\
			\text{Energy}\hspace{4mm}&E_t =\left({F}_t^\frac{\varepsilon_e-1}{\varepsilon_e}+G_t^\frac{\varepsilon_e-1}{\varepsilon_e}\right)^\frac{\varepsilon_e}{\varepsilon_e-1}\\
			\ \\
			\text{Demand energy producers}\hspace{4mm}&\frac{F_t}{G_t} = \left(\frac{p_{Gt}}{p_{Ft}+\alert{\pmb{\tau_{Ft}}}}\right)^{\varepsilon_e}
			%	\tikzmarkend{third}
		\end{align*}
		
		\small
		\vspace{4mm}
		\hspace{-4mm}
		\begin{minipage}[t!]{0.24\textwidth}
			\vspace{0mm}
			\begin{itemize}	
				\item[]$F_t$: fossil energy
				\vspace{-2mm}	
				\item[]$G_t$: green energy
				\vspace{-7mm}	
				\item[]$N_t$: non-energy
			\end{itemize}
		\end{minipage}
		\begin{minipage}[t!]{0.24\textwidth}
			\vspace{0mm}
			\begin{itemize}
				\item[] $p_{Gt}$: price green  \vspace{-2mm}
				\item[] $p_{Ft}$: price fossil
				\vspace{-2mm}	
				\item[] $\tau_{Ft}$: carbon tax
			\end{itemize}
		\end{minipage}
		\begin{minipage}[t!]{0.47\textwidth}
			\vspace{0mm}
			\begin{itemize}
				\item[] $\delta_{y}$: weight on energy\vspace{-2mm}
				\item[] $\varepsilon_y$: elasticity of substitution $E_t$ and $N_t$ \vspace{-2mm}
				\item[] $\varepsilon_e$: elasticity of substitution $F_t$ and $G_t$
			\end{itemize}
		\end{minipage}

\vspace{-2mm}
\hfill \hyperlink{backScheme}{\tiny{$\rightarrow$ back}}
	\end{frame}
	
	%\addtocounter{framenumber}{-1}
	\begin{frame}{Production: intermediate goods $J\in \{N,F,G\}$ }
		\vspace{0mm}
		%Competitive producers
		\begin{align*}
			\underset{\{x_{Jit}\}_{i=0}^1, L_{Jt}}{\max}\ & p_{Jt}J_t-w_{t}L_{Jt}-\int_{0}^{1}p_{xJit}x_{Jit}di \\ \ \\
			\text{s.t.}\ & J_{t}=L_{Jt}^{1-\alpha_J}\int_{0}^{1}A^{1-\alpha_J}_{Jit}x_{Jit}^{\alpha_J}di
		\end{align*}
		
		\small
		\vspace{10mm}
		\hspace{-4mm}
		\begin{minipage}[t!]{0.3\textwidth}
			\vspace{0mm}
			\begin{itemize}	
				\item[]$L_{Jt}\ $: labor 
				\vspace{-2mm}	
				\item[]$x_{Jit}\ $: machines 
				\vspace{-2mm}	
				\item[]$p_{xJit}$: price machine 
			\end{itemize}
		\end{minipage}
		\begin{minipage}[t!]{0.5\textwidth}
			\vspace{0mm}
			\begin{itemize}
				\item[] $A_{Jit}$: productivity machine $i$ sector $J$ \vspace{-2mm}
				\item[] $J$\ \  : sector N(on-energy),F(ossil),G(reen)
				\vspace{-2mm}	
				\item[] $\alpha_J$\ : capital share 
			\end{itemize}
		\end{minipage}
	
	\hfill \hyperlink{backScheme}{\tiny{$\rightarrow$ back}}
	\end{frame}
	
	\begin{frame}{Production: machines and innovation}
		\hypertarget{modma}{}
		\vspace{0mm}
		\begin{align*}
			%	\tikzmarkin{sixth}(6.3,4)(-2.7,-3.8)
			\underset{p_{xJit}, s_{Jit}}{\max}\ & p_{xJit}(1+\zeta_{Jt})x_{Jit}-x_{Jit}-w_{st}s_{Jit}
			\\ 
			\text{s.t.}\ &(1)\ x_{Jit}=\left(\frac{\alpha_Jp_{Jt}}{p_{xJit}}\right)^{\frac{1}{1-\alpha_J}}L_{Jt}A_{Jit}\\ \ \\ %x_{ijt}= \left(\frac{p_{ft}(1-\tau_{jt})\alpha_j}{p_{xijt}}\right)^\frac{1}{1-\alpha_j}A_{ijt}L_{jt}\\
			& (2)\ A_{Jit}=f_{Jt}(s_{Jit})%A_{Jt-1}\left(1+\gamma\left(\frac{s_{Jit}}{\rho_J}\right)^\eta\left(\frac{A_{t-1}}{A_{Jt-1}}\right)^\phi\right)
			%	\tikzmarkend{sixth}
		\end{align*}
		
		\small
		%\vfill
		\vspace{6mm}
		\hspace{-4mm}	\begin{minipage}[t!]{0.32\textwidth}
			\vspace{0mm}
			\begin{itemize}
				\item[-] monopolistic competition 
				\vspace{-4mm}
				\item[-] one-period patents
			\end{itemize}	
		\end{minipage}
		\begin{minipage}[t!]{0.5\textwidth}
			\vspace{0mm}
			\begin{itemize}	
				\item[]$\zeta_{Jt}$: subsidy
				\vspace{-2mm}	
				\item[]$s_{Jit}$: scientists
				\vspace{-2mm}	
				\item[]$A_{Jit}$: productivity machine $i$ sector $J$
			\end{itemize}
		\end{minipage}
		%\begin{minipage}[t!]{0.32\textwidth}
		%	\vspace{0mm}
		%	\begin{itemize}
			%		\item[] $\eta$: returns to research  \vspace{-2mm}
			%		\item[] $\rho_j$: research processes
			%		\vspace{-2mm}	
			%		\item[] $\gamma$: productivity scientists
			%	\end{itemize}
		%\end{minipage}
		
		\vspace{0mm}
		\hfill \hyperlink{backinnov}{\tiny{$\rightarrow$ innovation, \hyperlink{backScheme}{\tiny{$\rightarrow$ model}}}}
	\end{frame}	
		
		
%		\begin{frame}{Production: returns to research}
%			\vspace{-2mm}
%			\begin{align*}
%				%	w_{st}&=\left(\frac{p_{Jt}\alpha_J}{p_{xJit}}\right)^{\frac{1}{1-\alpha_J}}L_{Jt}A_{Jt-1}\gamma \eta \left(\frac{A_{t-1}}{A_{Jt-1}}\right)^{\phi}\left(\frac{s_{Jit}}{\rho_J}\right)^{\eta-1}\\
%				w_{st}&=\underbrace{\left(\frac{\alert{p_{Jt}}\alpha_J}{p_{xJit}}\right)^{\frac{1}{1-\alpha_J}}\alert{L_{Jt}}}_{\frac{\partial x_{Jit}}{\partial A_{Jit}}}\gamma \eta A_{Jt-1} \left(\frac{A_{t-1}}{A_{Jt-1}}\right)^{\phi}\left(\frac{s_{Jit}}{\rho_J}\right)^{\eta-1}
%				%&	\frac{\eta \gamma \left(\frac{A_{t-1}}{A_{Jt-1}}\right)^\phi(1-\alpha_J)\alpha_Js_{Jt}^{\eta-1}p_{Jt}J_t}{\rho_J^\eta}	
%			\end{align*}
%			\begin{itemize}
%				\item \alert{carbon tax lowers demand for fossil machines and returns to fossil research}
%				\item scientists more productive in advanced sectors \ar path dependency
%				\item knowledge from other sectors increases productivity of scientists
%				\item decreasing returns to research, $\eta<1$
%			\end{itemize}
%			\small
%			\vspace{7mm}
%			\hspace{-4mm}
%			\begin{minipage}[t!]{0.3\textwidth}
%				\vspace{0mm}
%				\begin{itemize}
%					\item[] $\eta$: returns to research  \vspace{-2mm}
%					\item[] $\rho_j$: research processes
%				\end{itemize}
%			\end{minipage}
%			\vspace{-5mm}
%			\begin{minipage}[t!]{0.5\textwidth}
%				\vspace{0mm}
%				\begin{itemize}	
%					\item[] $\gamma$: productivity scientists
%					\vspace{-2mm}	
%					\item[] $\phi$: knowledge spillovers, $\phi\geq0$
%				\end{itemize}
%			\end{minipage}
%		\end{frame}
%		
%		\addtocounter{framenumber}{-1}
%		
%		\begin{frame}{Production: returns to research}
%			\vspace{-2mm}
%			\begin{align*}
%				%	w_{st}&=\left(\frac{p_{Jt}\alpha_J}{p_{xJit}}\right)^{\frac{1}{1-\alpha_J}}L_{Jt}A_{Jt-1}\gamma \eta \left(\frac{A_{t-1}}{A_{Jt-1}}\right)^{\phi}\left(\frac{s_{Jit}}{\rho_J}\right)^{\eta-1}\\
%				w_{st}&=\underbrace{\left(\frac{{p_{Jt}}\alpha_J}{p_{xJit}}\right)^{\frac{1}{1-\alpha_J}}{L_{Jt}}}_{\frac{\partial x_{Jit}}{\partial A_{Jit}}}\gamma \eta \alert{A_{Jt-1}} \left(\frac{A_{t-1}}{A_{Jt-1}}\right)^{\phi}\left(\frac{s_{Jit}}{\rho_J}\right)^{\eta-1}
%				%&	\frac{\eta \gamma \left(\frac{A_{t-1}}{A_{Jt-1}}\right)^\phi(1-\alpha_J)\alpha_Js_{Jt}^{\eta-1}p_{Jt}J_t}{\rho_J^\eta}	
%			\end{align*}
%			\begin{itemize}
%				\item carbon tax lowers demand for fossil machines and returns to fossil research
%				\item \alert{scientists more productive in advanced sectors \ar path dependency}
%				\item knowledge from other sectors increases productivity of scientists
%				\item decreasing returns to research, $\eta<1$
%			\end{itemize}
%			\small
%			\vspace{7mm}
%			\hspace{-4mm}
%			\begin{minipage}[t!]{0.3\textwidth}
%				\vspace{0mm}
%				\begin{itemize}
%					\item[] $\eta$: returns to research  \vspace{-2mm}
%					\item[] $\rho_j$: research processes
%				\end{itemize}
%			\end{minipage}
%			\vspace{-5mm}
%			\begin{minipage}[t!]{0.5\textwidth}
%				\vspace{0mm}
%				\begin{itemize}	
%					\item[] $\gamma$: productivity scientists
%					\vspace{-2mm}	
%					\item[] $\phi$: knowledge spillovers, $\phi\geq0$
%				\end{itemize}
%			\end{minipage}
%		\end{frame}
%		
%		
%		
%		\addtocounter{framenumber}{-1}
%		
%		\begin{frame}{Production: returns to research}
%			\vspace{-2mm}
%			\begin{align*}
%				%	w_{st}&=\left(\frac{p_{Jt}\alpha_J}{p_{xJit}}\right)^{\frac{1}{1-\alpha_J}}L_{Jt}A_{Jt-1}\gamma \eta \left(\frac{A_{t-1}}{A_{Jt-1}}\right)^{\phi}\left(\frac{s_{Jit}}{\rho_J}\right)^{\eta-1}\\
%				w_{st}&=\underbrace{\left(\frac{{p_{Jt}}\alpha_J}{p_{xJit}}\right)^{\frac{1}{1-\alpha_J}}{L_{Jt}}}_{\frac{\partial x_{Jit}}{\partial A_{Jit}}}\gamma \eta {A_{Jt-1}} \alert{\left(\frac{A_{t-1}}{A_{Jt-1}}\right)^{\phi}}\left(\frac{s_{Jit}}{\rho_J}\right)^{\eta-1}
%			\end{align*}
%			\begin{itemize}
%				\item carbon tax lowers demand for fossil machines and returns to fossil research
%				\item scientists more productive in advanced sectors \ar path dependency
%				\item \alert{knowledge from other sectors increases productivity of scientists}
%				\item decreasing returns to research, $\eta<1$ 
%			\end{itemize}
%			\small
%			\vspace{7mm}
%			\hspace{-4mm}
%			\begin{minipage}[t!]{0.3\textwidth}
%				\vspace{0mm}
%				\begin{itemize}
%					\item[] $\eta$: returns to research  \vspace{-2mm}
%					\item[] $\rho_j$: research processes
%				\end{itemize}
%			\end{minipage}
%			\vspace{-5mm}
%			\begin{minipage}[t!]{0.5\textwidth}
%				\vspace{0mm}
%				\begin{itemize}	
%					\item[] $\gamma$: productivity scientists
%					\vspace{-2mm}	
%					\item[] $\phi$: knowledge spillovers, $\phi\geq0$
%				\end{itemize}
%			\end{minipage}
%		\end{frame}
%		
%		\addtocounter{framenumber}{-1}
%		
%		\begin{frame}{Production: returns to research}
%			\vspace{-2mm}
%			\begin{align*}
%				%	w_{st}&=\left(\frac{p_{Jt}\alpha_J}{p_{xJit}}\right)^{\frac{1}{1-\alpha_J}}L_{Jt}A_{Jt-1}\gamma \eta \left(\frac{A_{t-1}}{A_{Jt-1}}\right)^{\phi}\left(\frac{s_{Jit}}{\rho_J}\right)^{\eta-1}\\
%				w_{st}&=\underbrace{\left(\frac{{p_{Jt}}\alpha_J}{p_{xJit}}\right)^{\frac{1}{1-\alpha_J}}{L_{Jt}}}_{\frac{\partial x_{Jit}}{\partial A_{Jit}}}\gamma \eta {A_{Jt-1}} \left(\frac{A_{t-1}}{A_{Jt-1}}\right)^{\phi}\alert{\left(\frac{s_{Jit}}{\rho_J}\right)^{\eta-1}}
%				%&	\frac{\eta \gamma \left(\frac{A_{t-1}}{A_{Jt-1}}\right)^\phi(1-\alpha_J)\alpha_Js_{Jt}^{\eta-1}p_{Jt}J_t}{\rho_J^\eta}	
%			\end{align*}
%			\begin{itemize}
%				\item carbon tax lowers demand for fossil machines and returns to fossil research
%				\item scientists more productive in advanced sectors \ar path dependency
%				\item knowledge from other sectors increases productivity of scientists
%				\item \alert{decreasing returns to research, $\eta<1$}
%			\end{itemize}
%			\small
%			\vspace{7mm}
%			\hspace{-4mm}
%			\begin{minipage}[t!]{0.3\textwidth}
%				\vspace{0mm}
%				\begin{itemize}
%					\item[] $\eta$: returns to research  \vspace{-2mm}
%					\item[] $\rho_j$: research processes
%				\end{itemize}
%			\end{minipage}
%			\vspace{-5mm}
%			\begin{minipage}[t!]{0.5\textwidth}
%				\vspace{0mm}
%				\begin{itemize}	
%					\item[] $\gamma$: productivity scientists
%					\vspace{-2mm}	
%					\item[] $\phi$: knowledge spillovers, $\phi\geq0$
%				\end{itemize}
%			\end{minipage}
%		\end{frame}
%		
%		
%	

	
\begin{frame}{Markets}
	\begin{minipage}[t!]{1\textwidth}
		\begin{align*}
			%	\tikzmarkin{8th}(3.6,2.4)(-4.5,-2.2)
			\text{Hours workers}&\hspace{6mm}		H_{t}=L_{Ft}+L_{Gt}+L_{Nt}\\
			\text{Hours scientists}&\hspace{6mm}	S_{t} = \int_{0}^{1}\left(s_{Fit}+s_{Git}+s_{Nit}\right)di\\
			\text{Final good}&\hspace{6mm}	Y_t =C_t+\int_{0}^{1}\left(x_{Fit}+x_{Git}+x_{Nit}\right)di
			%	\tikzmarkend{8th}
		\end{align*}
	\end{minipage}

\hfill \hyperlink{backScheme}{\tiny{$\rightarrow$ back}}
\end{frame}
		
		
		
		
%\begin{frame}{Government}
%	\hypertarget{gov}{}
%	\vspace{-4mm}
%	\centering
%	\begin{minipage}[t!]{1\textwidth}
%		\begin{align*}
%			\max_{\{\tau_{Ft}\}_{t=0}^{\infty}, \{\tau_{\iota t}\}_{t=0}^{\infty}}&\hspace{3mm} \sum_{t=0}^{\infty}\beta^t U\left(C_t,H_t, S_t\right)\\ \ \\
%			\text{s.t.} \hspace{4mm}
%			&{ (1)\ \ T_t={{\tau_{Ft}}}F_{t}}+T_{\pi t}\\
%			&{(2)\ \ T\left(w_{t}H_{t},w_{st}S_{t}; \lambda_t, \tau_{\iota t}\right)=0}\\
%			& {(3)\ \  \text{behavior of firms and households}}\\
%			& {(4)\ \ \text{resource constraints} }\\
%			& \textcolor{black!1}{(5)\ \  {\omega F_t-\delta \leq\Omega_t } \hspace{3mm} {\text{(dynamic emission target)}}}
%		\end{align*}
%	\end{minipage}
%	
%	\small
%	\vspace{-4mm}
%	\hspace{-10mm}
%	\begin{minipage}[t!]{0.5\textwidth}
%		\vspace{7mm}
%		\begin{itemize}
%			\item[] $\beta$\ \ : household discount factor\vspace{-2mm}
%			\item[] $T_\pi$: profits minus subsidies \\ \hspace{5.5mm} from machine producers \vspace{0mm}
%		\end{itemize}
%	\end{minipage}
%	\begin{minipage}[t!]{0.45\textwidth}
%		\vspace{8mm}
%		\begin{itemize}
%			\item[] \textcolor{black!1}{$\Omega_{t}$: net emission limit}
%			\vspace{-2mm}	
%			\item[] \textcolor{black!1}{$\omega$\ : emissions per unit of fossil} \vspace{-0.8mm}
%			\item[] \textcolor{black!1}{$\delta$\ \ : carbon sinks \tiny{\citep{VanVuuren2018AlternativeTechnologies}}}
%		\end{itemize}
%	\end{minipage}
%\end{frame}
%
%\addtocounter{framenumber}{-1}
\begin{frame}{ Government}
	\hypertarget{govmod}{}
	\vspace{-4mm}
	\centering
	\begin{minipage}[t!]{1\textwidth}
		\begin{align*}
			\max_{\{\tau_{Ft}\}_{t=0}^{\infty}, \{\tau_{\iota t}\}_{t=0}^{\infty}}&\hspace{3mm} \sum_{t=0}^{\infty}\beta^t U\left(C_t,H_t, S_t\right)\\ \ \\
			\text{s.t.} \hspace{4mm}
			&{ (1)\ \ T_t={{\tau_{Ft}}}F_{t}}+T_{\pi t}\\
			&{(2)\ \ T\left(w_{t}H_{t},w_{st}S_{t}; \lambda_t, \tau_{\iota t}\right)=0}\\
			& {(3)\ \  \text{behavior of firms and households}}\\
			& {(4)\ \ \text{resource constraints} }\\
			&{(5)\ \  \alert{\omega F_t-\delta \leq\Omega_t }} \hspace{3mm} \alert{\text{(dynamic emission target)}}
		\end{align*}
	\end{minipage}
	
	\small
	\vspace{-4mm}
	\hspace{-10mm}
	\begin{minipage}[t!]{0.5\textwidth}
		\vspace{7mm}
		\begin{itemize}
			\item[] $\beta$\ \ : household discount factor\vspace{-2mm}
			\item[] $T_\pi$: profits minus subsidies \\ \hspace{5.5mm} from machine producers \vspace{0mm}
		\end{itemize}
	\end{minipage}
	\begin{minipage}[t!]{0.45\textwidth}
		\vspace{8mm}
		\begin{itemize}
			\item[] $\Omega_{t}$: net emission limit
			\vspace{-2mm}	
			\item[] $\omega$\ : emissions per unit of fossil \vspace{-0.8mm}
			\item[] $\delta$\ \ : carbon sinks \tiny{\citep{VanVuuren2018AlternativeTechnologies}}
		\end{itemize}
	\end{minipage}
	
\hfill \hyperlink{backSchemeEnd}{\tiny{$\rightarrow$ back}}
\end{frame}

%%%%%%%%%%%%%%%%%%%%%%%%%%%%%%%%%
%% Calibration 
%%%%%%%%%%%%%%%%%%%%%%%%%%%%%%%%%


\section*{Calibration}
\begin{frame}{Emission limit}
	\hypertarget{emsall}{}
	\vspace{-1mm}
	\begin{itemize}
		\item  \textbf{global} CO$_2$ emissions consistent with $1.5^\circ$C climate targets \footnotesize{\citep{IPCC2022}}\normalsize :
		\vspace{1mm}
		\begin{itemize}
			\item[-] before 2050: remaining carbon budget 510Gt
			\item[-] from 2050 onward: net-zero  emissions
		\end{itemize}
		\vspace{0mm}
		\item \normalsize{\textbf{equal} distribution of  CO$_2$ emissions across countries }
	\end{itemize}
	\vspace{-2mm}
	
	\begin{center}
		\begin{minipage}{0.6\textwidth}
			\begin{figure}
				\caption{Net CO$_2$ emission limit in Gt}
				\includegraphics[width=0.7\textwidth]{../codding_model/own_basedOnFried/optimalPol_010922_revision/figures/all_13Sept22_Tplus30/Emnet_goals_o1_lgd0.png}
			\end{figure}
		\end{minipage}
		\hspace{-10mm}
		\begin{minipage}{0.3\textwidth}
			\begin{figure}
				\includegraphics[width=1.4\textwidth]{../codding_model/own_basedOnFried/optimalPol_010922_revision/figures/all_13Sept22_Tplus30/Emnet_goals_o1_lgd1.png}
			\end{figure}
		\end{minipage}
	\end{center}
	% equal per capita: -85\% reduction relative to 2019 levels (high because also corrects for emissions in US higher than population share!)
	% political goal -38\% relative to 2019
	% constant ratios/ carbon budget: -62\%
	
	\vspace{-5mm}
	\hfill	\hyperlink{resback}{\tiny{$\rightarrow$ back}}
\end{frame}

	\begin{frame}{Parameters}
		\hypertarget{calib}{}
		\vspace{-10mm}
		\begin{table}[h!]
			\begin{center}
				%		\captionsetup{width=0.3\textwidth}
				%		\caption{ Calibration}
				%		\label{tab:calib}
				\resizebox{4in}{!}{
					\begin{tabular}{l|ll}
						%			\hline \hline
						%			\multicolumn{7}{c}{Calibration based on basic needs}\\
						\hline \hline
						\textbf{Parameter}& \textbf{Value}& \makecell[l]{\textbf{Target}}\\ 
						\hline
						Household&\multicolumn{2}{c}{}\\
						\hline 
						($\sigma$, 	$\sigma_s$) & ($1.33$, $1.33$)&  \makecell[l]{\cite{Chetty2011AreMargins}}  \\
						%$z_h$& \makecell[l]{skill premium 2005-2016:\\ $w_h/w_l=1.9$\\ \citep{Slavik2020WagePremium}}\\	
						($\chi$, $\chi_s$)& (10.02, 0.48) &  \makecell[l]{average hours worked per\\ economic time endowment\\ by worker: 0.34 \citep{OECDHoursworked}} \\
						$\beta$ & 0.93 &  \makecell[l]{\cite{Barrage2019OptimalPolicy}}\\
						$\bar{H}$&1.00& \makecell[l]{14.5 hours per day \citep{Jones1993OptimalGrowth}} \\
						$\bar{S}$&0.50& \makecell[l]{upper bound hours scientists} \\
						\hline
						Research&\multicolumn{2}{c}{}
						\\
						\hline 
						${{\eta}}$ &0.79 & \rdelim\}{3}{5cm}[\normalfont\makecell{\cite{Fried2018ClimateAnalysis}}] \\
						($\rho_F$, $\rho_G$, $\rho_N$)& (0.01, 0.01, 1.00) &%\makecell[l]{\cite{Fried2018ClimateAnalysis}}   
						\\
						${{\phi}}$ &0.75&  \\
						$\gamma$ & 0.06 &\makecell[l]{maximum aggregate growth:\\4\% per annum \citep{OECDGDP}}\\
						\hline
						Production&\multicolumn{2}{c}{}\\
						\hline
						$({\varepsilon_y, \varepsilon_e})$&(0.05, 1.50)& \rdelim\}{2}{5cm}[\normalfont\makecell{\cite{Fried2018ClimateAnalysis}}] \\ %\cite{Fried2018ClimateAnalysis}\\	
						($\alpha_F$, $\alpha_G$, $\alpha_N$) &(0.72, 0.91, 0.36)&\\
						$\delta_y$&0.38&\makecell[l]{energy expenditure share  \citep{EIAEnergy}}\\
						\hline
						Initial TFP&\multicolumn{2}{c}{}\\
						\hline
						({${A_{F0}^{1-\alpha_f}}$, ${A_{G0}^{1-\alpha_g}}$, ${A_{N0}^{1-\alpha_n}}$})&(6.68, 1.50, 2.85) &fossil to green ratio, energy share to GDP \citep{EIAEnergy}  \\
						\hline 
						Emissions&\multicolumn{2}{c}{}\\
						\hline
						$\delta$&3.19& \makecell[l]{in GtCO$_2$ \citep{EPAems}}\\
						$\omega$&217.3& \cite{EPAems}\\
						\hline \hline
					\end{tabular}
				}
			\end{center}
		\end{table}
		
		\vspace{-6mm}
		\hfill
		\hyperlink{resback}{\tiny{$\rightarrow$ back}}
	\end{frame}
	
	%%%%%%%%%%%%%%%%%%%%%%%%%%%%%%%%%%%%%%%5
	%%%%%% first best
	%%%%%%%%%%%%%%%%%%%%%%%%%%%%%%%%%%%
	\section*{Results}
\begin{frame}{First-best and optimal allocation}
	\hypertarget{compfb}{}
	\vspace{-3mm}
	\begin{figure}[h!!]
		\centering
		\begin{subfigure}{0.45\textwidth}		
			\caption{Green-to-fossil energy ratio}
			%	\captionsetup{width=.45\linewidth}
			\includegraphics[width=1\textwidth]{../codding_model/own_basedOnFried/optimalPol_010922_revision/figures/all_13Sept22/NewCalib_effBauOpt_T_GFF_Sun2_emnet1_spillover0_knspil3_xgr0_nsk0_sep0_extern0_PV1_etaa0.79_lgd1.png}
		\end{subfigure}
		\begin{minipage}[]{0.05\textwidth}
			\
		\end{minipage}
		\begin{subfigure}{0.45\textwidth}		
			\caption{{Fossil-to-green scientists}}
			%	\captionsetup{width=.45\linewidth}
			\includegraphics[width=1\textwidth]{../codding_model/own_basedOnFried/optimalPol_010922_revision/figures/all_13Sept22/NewCalib_effBauOpt_T_sffsg_Sun2_emnet1_spillover0_knspil3_xgr0_nsk0_sep0_extern0_PV1_etaa0.79_lgd0.png}
		\end{subfigure}
	\end{figure}
	\vspace{1mm}
	\begin{block}{}
		\begin{itemize}
			%\item First-best: more fossil research and higher green-to-fossil energy ratio
			\item {Government dilemma: reducing fossil fuels entails too strong a decline in fossil research activity}
		\end{itemize}
	\end{block}	
	\vspace{-5.7mm}
	\hfill
	\hyperlink{backop}{\tiny{$\rightarrow$ back}}
\end{frame}	
	
	
	\begin{frame}{Marginal tax rates by type}
		\hypertarget{Redis}{}
		\vspace{-3mm}
		\begin{figure}[h!!]
			
			\begin{subfigure}{0.45\textwidth}		
				\caption{\footnotesize{Marginal income tax rate scientists, \%}}
				%	\captionsetup{width=.45\linewidth}
				\includegraphics[width=1\textwidth]{../codding_model/own_basedOnFried/optimalPol_010922_revision/figures/all_13Sept22/Single_NC_T_dTaulS_emnet1_Sun2_regime4_spillover0_knspil3_noskill0_sep0_xgrowth0_extern0_PV1_sizeequ0_GOV0_etaa0.79.png}
			\end{subfigure}	
			\begin{minipage}[]{0.05\textwidth}
				\ 
			\end{minipage}
			\begin{subfigure}{0.45\textwidth}		
				\caption{\footnotesize{Marginal income tax rate workers, \%}}
				%	\captionsetup{width=.45\linewidth}
				\includegraphics[width=1\textwidth]{../codding_model/own_basedOnFried/optimalPol_010922_revision/figures/all_13Sept22/Single_NC_T_dTaulAv_emnet1_Sun2_regime4_spillover0_knspil3_noskill0_sep0_xgrowth0_extern0_PV1_sizeequ0_GOV0_etaa0.79.png}
			\end{subfigure}
			
		\end{figure}
		
		\vspace{-4mm}
		\hfill
		\hyperlink{backOPT}{\tiny{$\rightarrow$ back}}

	\end{frame}
			
		\begin{frame}{Sensitivity to knowledge spillovers}
			\hypertarget{sensphi}{}
			\vspace{-3mm}
			\begin{figure}[h!!]
				
				\begin{subfigure}{0.45\textwidth}		
					\caption{Tax per ton of carbon,  US\$, $\tau_{Ft}$ }
					%	\captionsetup{width=.45\linewidth}
					\includegraphics[width=1\textwidth]{../codding_model/own_basedOnFried/optimalPol_010922_revision/figures/all_13Sept22/Phi_SensN_Tauf_spillover0_knspil0_xgr0_nsk0_sep0_extern0_PV1_etaa0.79_lgd1.png}
				\end{subfigure}	
				\begin{minipage}[]{0.05\textwidth}
					\ 
				\end{minipage}
				%	\begin{subfigure}{0.3\textwidth}		
					%		\caption{Marginal tax rate in \%, $\tau_{\iota t}$}
					%		%	\captionsetup{width=.45\linewidth}
					%		\includegraphics[width=1\textwidth]{../codding_model/own_basedOnFried/optimalPol_010922_revision/figures/all_13Sept22/Phi_SensN_dTaulAvS_spillover0_knspil0_xgr0_nsk0_sep0_extern0_PV1_etaa0.79_lgd1.png}
					%	\end{subfigure}
				\begin{subfigure}{0.42\textwidth}		
					\caption{ Deviation marginal income tax in \%}
					%	\captionsetup{width=.45\linewidth}
					\includegraphics[width=1\textwidth]{../codding_model/own_basedOnFried/optimalPol_010922_revision/figures/all_13Sept22/NewCalib_Sens_TvsNoT_dTaulAv_emnet1_Sun2_spillover0_knspil3_xgr0_nsk0_sep0_extern0_PV1_etaa0.79_lgd0.png}
				\end{subfigure}
			\end{figure}
			\vspace{2mm}
			\begin{block}{}
				\begin{itemize}
					\item The stronger knowledge spillovers, the higher the optimal carbon tax
					\item At the same time, can profit more from fossil knowledge generated in early periods
					%	\item with knowledge spillovers qualitatively similar results
				\end{itemize}
			\end{block}	
			
			\vspace{-4mm}
			\hfill
			\hyperlink{backmec}{\tiny{$\rightarrow$ back,}}	\hyperlink{conc}{\tiny{$\rightarrow$ conclusion}}
		\end{frame}
	
	
	
	\begin{frame}{Effect  of integrated policy}
		\hypertarget{eff}{}
		\vspace{-5mm}
		\centering
		\begin{figure}[h!!]
			\centering
			\begin{subfigure}{0.45\textwidth}		
				\caption{{Deviation fossil-to-green scientists in \%}}
				%	\captionsetup{width=.45\linewidth}
				\includegraphics[width=1\textwidth]{../codding_model/own_basedOnFried/optimalPol_010922_revision/figures/all_13Sept22/NewCalib_polTaulFixedPer_T_sffsg_Sun2_emnet1_spillover0_knspil3_xgr0_nsk0_sep0_extern0_PV1_etaa0.79.png}
			\end{subfigure}
			\begin{minipage}[]{0.05\textwidth}
				\
			\end{minipage}
			\begin{subfigure}{0.45\textwidth}		
				\caption{{Deviation output in \%}}
				%	\captionsetup{width=.45\linewidth}
				\includegraphics[width=1\textwidth]{../codding_model/own_basedOnFried/optimalPol_010922_revision/figures/all_13Sept22/NewCalib_polTaulFixedPer_T_Y_Sun2_emnet1_spillover0_knspil3_xgr0_nsk0_sep0_extern0_PV1_etaa0.79.png}
			\end{subfigure}
		\end{figure}
		\vspace{3mm}	
		\begin{block}{}
			\begin{itemize}
				%				\item integrated policy achieves more beneficial allocation of scientists \item at the costs of lower production in initial years
				\item In run-up to net-zero limit: Higher fossil-to-green ratio of scientists %. Reduction in output to meet emission limit
				\item Under net-zero limit: More green research to internalize within-sector spillovers % Allows for rise in overall production. 
			\end{itemize}
		\end{block}	
		
		\vspace{-5.5mm}
		\hfill
		\hyperlink{conc}{\tiny{$\rightarrow$ conclusion}}
	\end{frame}

\begin{frame}{Decomposing effect of integrated policy}
	\hypertarget{mec0}{}
	\vspace{-3mm}
	\centering
	\begin{figure}
		\begin{subfigure}{0.45\textwidth}
			\caption{{Deviation fossil-to-green scientists in \% }}
			%	\captionsetup{width=.45\linewidth}
			\includegraphics[width=1\textwidth]{../codding_model/own_basedOnFried/optimalPol_010922_revision/figures/all_13Sept22/NewCalib_polTaulFixedTaufJointPer_sffsg_Sun2_emnet1_spillover0_knspil3_xgr0_nsk0_sep0_extern0_PV1_etaa0.79_lgd1.png}
		\end{subfigure}
		\begin{minipage}[]{0.05\textwidth}
			\
		\end{minipage}
		\begin{subfigure}{0.45\textwidth}
			\caption{{Deviation output in \%}}
			%	\captionsetup{width=.45\linewidth}
			\includegraphics[width=1\textwidth]{../codding_model/own_basedOnFried/optimalPol_010922_revision/figures/all_13Sept22/NewCalib_polTaulFixedTaufJointPer_Y_Sun2_emnet1_spillover0_knspil3_xgr0_nsk0_sep0_extern0_PV1_etaa0.79_lgd0.png}
		\end{subfigure}
	\end{figure}
	\vspace{3mm}
	\begin{block}{}
		\begin{itemize}
			\item Adjustment in carbon tax directs research activity
			\item Changes in labor income tax adjust level of production %side effects on labor market
		\end{itemize}
	\end{block}	
	
	\vspace{-5.5mm}
	\hfill
	\hyperlink{conc}{\tiny{$\rightarrow$ conclusion}}
\end{frame}

	\end{document}