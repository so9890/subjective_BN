% 15 minutes teaching philosophy 

%he next 15 minutes are devoted to presenting your teaching philosophy, 
%including your experience in terms of courses and methods as well as
%intentions and general principles of your pedagogical concept. 

\section{Teaching Philosophy}


\begin{frame}{Experience: My teaching in the past}
	\begin{itemize}
		\item Undergraduate \textit{Introductory  Macroeconomics} Tutor, several times
		\vspace{2mm}
		\begin{itemize}
			\item[-] discussion of problem sets
		\end{itemize}
	\vspace{3mm}
		\item Undergraduate \textit{Monetary Theory and Policy} Teaching Assistance
		\vspace{2mm}		
		\begin{itemize}
			\item[-] offered room to ask questions
			\item[-] exam preparation 
			\item[-] during Pandemic for those with special needs
			\item[-] helped design final exam
		\end{itemize}
	\end{itemize}
\end{frame}

\begin{frame}{General Principles}
	\centering
\textbf{Education}
	\begin{itemize}
		\item  to develop personally 
		\item  to be successful in life % to know what they want
		\item to develop a sense for society
	\end{itemize}
\textbf{University education}
\begin{itemize}	
	\item critical, independent thinking
	\item discovering own (research) interests % Make a choice in what direction to develop
\end{itemize}
\textbf{Economic eduction}
\begin{itemize}
	\item enable students to understand and assess economic phenomena (undergraduate)
	\item to improve on existing methods (graduate/ doctoral students)
\end{itemize}
\end{frame}

\begin{frame}{How to achieve these goals?: \alert{Role of teacher}}
	
	\begin{enumerate}
		\item<+-> Provide room for development and stimulate interest
		\begin{enumerate}
	\item<+-> {make economics accessible: knowledge transfer} % creating the room
	\item<+-> {give opportunities to specialize based on own interests}
	%			%			\item provide opportunities to shape specialization content of classes
\end{enumerate}
		\item<+-> Invite multitudes of perspectives
%		\begin{itemize}
%			\item[] \textcolor{black!1}{classroom discussions}
%			\item[] \textcolor{black!1}{plays\\ \ }
%		\end{itemize}
		\item<+-> Active mentoring of minorities and those with special needs
	\end{enumerate}
\end{frame}


\begin{frame}{How to achieve these goals?: {Role of teacher} and \alert{methods}}
	
	\begin{enumerate}
		\item Provide room for development and  stimulate interest
				\begin{enumerate}
					\item {make economics accessible: knowledge transfer} % creating the room
						\item {provide opportunities to specialize based on own interests}
			%			%			\item provide opportunities to shape specialization content of classes
					\end{enumerate}
		\item Invite multitudes of perspectives
		%		\begin{itemize}
			%			\item[] \textcolor{black!1}{classroom discussions}
			%			\item[] \textcolor{black!1}{plays\\ \ }
			%		\end{itemize}
		\item Active mentoring of minorities and those with special needs
	\end{enumerate}
\end{frame}


\begin{frame}{How to achieve these goals?: Role of teacher and\alert{ methods}}
	\begin{enumerate}
		\item \alert{\textbf{Provide room for development and stimulate interest}}
		\begin{enumerate}
			\item make economics accessible: \alert{knowledge transfer} % being able to think critically about economics and economic research
			\begin{itemize}
				\item[-] lectures
				\item[-] real world examples
				\item[-] hands-on problem sets
			\end{itemize}
			\item provide \alert{opportunities to specialize} based on own interests %\\ %(individual work/ shape specialization content of classes)
			\begin{itemize}
				\item[-] opportunity to shape specialization content of classes
				\item[-] flipping the classroom; student presentations
				\item[-] opportunity to define content of essays/projects %hands on (data/programming) exercises
				\item[-] ask students to prepare questions prior to lectures
			\end{itemize}
			%			\item provide opportunities to shape specialization content of classes
		\end{enumerate}
		\item Invite various perspectives
%		\begin{itemize}
%			\item[-] classroom discussions 
%			\item[-] encourage group work
%			\item[-] plays
%			\item[-] interdisciplinary approach % e.g. moral philosophy when it comes to discussing welfare functions
%		\end{itemize}
		\item Active mentoring of minorities and those with special needs
%				\begin{itemize}
%			\item[-] availability of videos of lectures/ hybrid format
%			\item[-] monitor learning success to be able to support where necessary
%			\item[-] ensure easily approachable by students and actively approach students to learn about their needs
%			\item[-] use of inclusive language
%			\item[-] be available to discuss questions
%		\end{itemize}
	\end{enumerate}
\end{frame}



\begin{frame}{How to achieve these goals?: Role of teacher \alert{and methods}}
	\begin{enumerate}
		\item Provide room for development and stimulate interest
%		\begin{enumerate}
%			\item make economics accessible: knowledge transfer % creating the room
%			\item provide opportunities to specialize based on own interests\\ (individual work/ shape specialization content of classes)
%			\begin{itemize}
%				\item flipping the classroom; student presentations
%				\item essays/ hands on (data/programming) exercises
%				\item ask students to prepare questions prior to lecture
%			\end{itemize}
%			%			\item provide opportunities to shape specialization content of classes
%		\end{enumerate}
		\item \alert{\textbf{Invite various perspectives}}
		\begin{itemize}[<+->]
			\item[-] classroom discussions
			\begin{itemize}
				\item[-] start with provocative statement
				\item[-] explicitly  ask for counterarguments
				\item[-] assign roles
			\end{itemize}
			\item[-] plays (in small groups)
			\item[-] encourage group work in- and outside the classroom
			\item[-] interdisciplinary approach
			\begin{itemize}
				\item[-] highlight where other disciplines become relevant; e.g. natural sciences in environmental economics, philosophy for welfare/ utility functions
				\item[-] alternative approaches from other fields in- and outside of economics; e.g. behavioral econ and macroeconomics
			\end{itemize}
			 % e.g. moral philosophy when it comes to discussing welfare functions
		\end{itemize}
		\item Active mentoring of minorities and those with special needs
%		\begin{itemize}
%			\item[-] availability of videos of lectures/ hybrid format
%			\item[-] monitor learning success to be able to support where necessary
%			\item[-] ensure easily approachable by students and actively approach students to learn about their needs
%			\item[-] use of inclusive language
%			\item[-] be available to discuss questions
%		\end{itemize}
	\end{enumerate}
\end{frame}

\begin{frame}{How to achieve these goals?: Role of teacher \alert{and methods}}
	\begin{enumerate}
		\item Provide room for development and stimulate interest
		%		\begin{enumerate}
			%			\item make economics accessible: knowledge transfer % creating the room
			%			\item provide opportunities to specialize based on own interests\\ (individual work/ shape specialization content of classes)
			%			\begin{itemize}
				%				\item flipping the classroom; student presentations
				%				\item essays/ hands on (data/programming) exercises
				%				\item ask students to prepare questions prior to lecture
				%			\end{itemize}
			%			%			\item provide opportunities to shape specialization content of classes
			%		\end{enumerate}
		\item Invite various perspectives
%		\begin{itemize}
%			\item[-] classroom discussions 
%			\item[-] encourage group work
%			\item[-] plays
%			\item[-] interdisciplinary approach % e.g. moral philosophy when it comes to discussing welfare functions
%		\end{itemize}
		\item \alert{\textbf{Active mentoring of minorities and those with special needs}}
		\begin{itemize}[<+->]
			\item[-] availability of videos of lectures/ hybrid format
			\item[-] use of inclusive language
			\item[-] inviting participation in class; see discussions, preparing questions prior to lecture
			\item[-] monitor learning success to be able to support where necessary; e.g. online questionnaires during semester
			\item[-] ensure easily approachable by students and actively approach students to learn about their needs
			\item[-] be available to discuss questions
		\end{itemize}
	\end{enumerate}
\end{frame}


\begin{frame}{Intention: My teaching in the future}

\begin{itemize}
	\item Bachelor's
	\begin{itemize}
		\item[-] Macroeconomics
		\item[-] Monetary and Fiscal Policy
		\item[-] Environmental Economics 
	\end{itemize}
	\item Master's/ Ph.D.
	\begin{itemize}
		\item[-] Environmental Economics and Public Finance
		\item[-] Research and Reading Groups
	\end{itemize}
\end{itemize}
\end{frame}

