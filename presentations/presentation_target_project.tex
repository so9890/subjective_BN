\documentclass[11pt,aspectratio=169]{beamer}
%\usepackage[noxcolor]{beamerarticle} % to get presentation as article ! if used also set documentclass to article!
\usetheme[outer/progressbar=foot,
%outer/numbering=none
]{metropolis}
\setbeamertemplate{caption}{\raggedright\insertcaption\par}
\setbeamercolor{frametitle}{bg={}, fg=black!80}
\setbeamercolor{alerted text}{bg={}, fg=cyan!100}
\setbeamercolor{block title}{bg=black!10, fg=black}
\setbeamercolor{block body}{bg=black!10, fg=black}
%\usecolortheme{seahorse}
\usepackage[utf8]{inputenc}
\usepackage[english]{babel}
%\usepackage[T1]{fontenc}
\newcommand{\tr}[1]{\textcolor{blue}{#1}}
\usepackage{amsmath}
\usepackage{amsfonts}
\usepackage{amssymb}
\usepackage{mathtools}
\usepackage{calc}
\usepackage{soul}
\setbeamercolor{headerCol}{fg=blue!30,bg=black!80}
\setbeamercolor{bodyCol}{fg=black}
\usepackage{graphicx}
\usepackage{xcolor}
\usepackage{appendix}
\usepackage{hyperref}
\usepackage{natbib}
\usepackage{comment}
\usepackage{setspace}
\renewcommand{\bibsection}{}
\bibliographystyle{apa} 
% have to run bibtex mydocument.aux after first run to generate bbl file. 
\usepackage{appendixnumberbeamer}
\usepackage{xcolor}


\usepackage[customcolors]{hf-tikz}
\definecolor{sonja}{cmyk}{1.5,0,0.9,0.3}
%\definecolor{blue}{cmyk}{0,1,0,0}
\hfsetfillcolor{black!10}
\hfsetbordercolor{black}

\usepackage{tikz}
\usetikzlibrary{tikzmark}
\usetikzlibrary{decorations.markings}
\usepackage{tikz-cd}
\usetikzlibrary{arrows,calc,fit}
\tikzset{mainbox/.style={draw=white, text=white, fill=gray, rectangle, rounded corners, thick, node distance=7em, text width=8em, text centered, minimum height=3.5em}}
\tikzset{dummybox/.style={draw=none, text=white , rectangle, rounded corners, thick, node distance=7em, text width=8em, text centered, minimum height=3.5em}}
\tikzset{box/.style={draw , rectangle, rounded corners, thick, node distance=7em, text width=8em, text centered, minimum height=3.5em}}
\tikzset{container/.style={draw, rectangle, dashed, inner sep=2em}}
\tikzset{line/.style={draw, very thick, -latex'}}
\tikzset{    pil/.style={
		->,
		thick,
		shorten <=2pt,
		shorten >=2pt,}}
\tikzstyle{vecArrow} = [thick, decoration={markings,mark=at position
	1 with {\arrow[semithick]{open triangle 60}}},
double distance=1.4pt, shorten >= 5.5pt,
preaction = {decorate},
postaction = {draw,line width=1.4pt, white,shorten >= 4.5pt}]



%TITLE
\author[Sonja Dobkowitz]{\small Sonja Dobkowitz}
\institute[University of Bonn]{University of Bonn}
\title{The net-zero emission target and fiscal policies}

\newcommand{\ar}{$\Rightarrow$ \ }

%\addtobeamertemplate{navigation symbols}{}{%
%    \usebeamerfont{footline}%
%    \usebeamercolor[fg]{footline}%
%    \hspace{1em}%
%   \insertframenumber/\inserttotalframenumber
%}

\institute{University of Bonn} 
\date{\today} 
%\subject{} 
\begin{document}
	
	{\setbeamertemplate{footline}{}
		\begin{frame}
		\titlepage
	\end{frame}
}
%\addtocounter{framenumber}{-1}

% {\setbeamertemplate{footline}{}
% \begin{frame}{Content}
% \vspace{4mm}
% \tableofcontents
% \end{frame}
% }
 %\addtocounter{framenumber}{-1}


%---------------------------------------
%            Intro
%---------------------------------------
%\section{ Reducing Consumption Levels}
\begin{frame}{Motivation}

\begin{itemize}[<+-| alert@+>]
	\setbeamercolor{alerted text}{fg=black} %change the font color
	\setbeamerfont{alerted text}{series=\bfseries} 
\item natural scientists suggest net-emissions to be zero by mid-century \citep{Rogelj2018MitigationDevelopment.}
\item in economic models emissions arise from fossil energy utilisation
\item therefore, growth in fossil energy production has to stop (assuming no infinite carbon capture-storage technology)
\end{itemize}
\end{frame}

\begin{frame}{This paper}
\begin{itemize}
	\item the emission target determines fossil output starting from 2050
	\item model economy on BGP (with constant output ratios) until today then, under optimal policy, focus on non-balanced trajectory
	\item starting in 2020, the planner optimises over a finite horizon only
\end{itemize}
Working hypotheses

\begin{itemize}
	\item is there a role for labour income taxes given heterogeneity in skills? \ar regressive tax; 
	\item could also have motive to reduce demand due to short time frame until emissions have to be zero and a too slow innovation rate (look at model without skill heterogeneity)
	\item what if the government on top seeks to satisfy a demand target as shown to be important to meet sustainability goals \ar progressive tax \ar A higher corrective tax needed to meet emission targets!
	\item what if redistribution also affects the composition of demand? (need to include inequality) \ar progressive tax?
\end{itemize}
\end{frame}

\begin{frame}{Literature}
	\begin{itemize}
		\item public finance
		\item environmental policy
		\item directed technical change
	\end{itemize}
\end{frame}

\begin{frame}{Current State and way forward}
	\begin{enumerate}
		\item Block
		\begin{itemize}
			\item model economy \checkmark
			\item define planner's objective
			\item derive relevant equations for Barrage
			\item code in Barrage to find optimal policy ! use as little analytical solutions as possible to be able to quickly change equations
		\end{itemize}
		\ar What is the optimal tax progressivity? \ar hypothesis: regressive tax and recycling revenues as transfers to clean sector wages?
		\item Block: How does adding a demand target change the result?
		\begin{itemize}
			\item hypothesis: a higher tax progressivity could be optimal to reduce high consumption levels
		\end{itemize}
	\item Block: accounting for non-homotheticities in demand
	\begin{itemize}
		\item redistribution to affect the externality through a demand channel
	\end{itemize}
	\end{enumerate}
\end{frame}

\begin{frame}{Method: nonbalanced Growth Path}
	\begin{itemize}
		\item need to analytically derive a continuation value to derive optimal policy
		\item how done in \cite{Barrage2019OptimalPolicy}? She assumes a balanced growth path to exist to derive the continuation value (she does not have several sectors)
	\end{itemize}
\end{frame}

\begin{frame}{Questions}
\begin{itemize}
	\item papers which study BGP with unequal growth?
\end{itemize}
\end{frame}

\begin{frame}[shrink]{References}
	
	\bibliography{../../bib_2_0}
	\bibliographystyle{apa}
\end{frame}
\end{document}