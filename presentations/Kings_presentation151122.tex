\documentclass[11pt,aspectratio=169]{beamer}
%\documentclass[11pt,aspectratio=169, handout]{beamer}
%\usepackage{handoutWithNotes}
\usetheme[outer/progressbar=foot,
%outer/numbering=none
]{metropolis}
\setbeamertemplate{caption}{\raggedright\insertcaption\par}
\setbeamercolor{frametitle}{bg={}, fg=black!80}
\definecolor{myorange}{rgb}{0.8500, 0.3250, 0.0980}
\setbeamercolor{alerted text}{bg={}, fg=myorange }
\setbeamercolor{block title}{bg=black!10, fg=black}
\setbeamercolor{block body}{bg=black!10, fg=black}
\setbeamercolor{block frame}{bg=black, fg=black}
\setbeamertemplate{blocks}[rounded]
\setbeamertemplate{blocks}[framed]
%\usecolortheme{seahorse}
\usepackage[utf8]{inputenc}
\usepackage[english]{babel}
%\usepackage[T1]{fontenc}
\newcommand{\tr}[1]{\textcolor{blue}{#1}}
\usepackage{amsmath}
\usepackage{amsfonts}
\usepackage{amssymb}
\usepackage{mathtools}
\usepackage{calc}
\usepackage{soul}
\setbeamercolor{headerCol}{fg=blue!30,bg=black!80}
\setbeamercolor{bodyCol}{fg=black}
\usepackage{graphicx}
\usepackage{xcolor}
\usepackage{appendix}
\usepackage{hyperref}
\usepackage{natbib}
\usepackage{comment}
\usepackage{setspace}
\renewcommand{\bibsection}{}
\bibliographystyle{apa} 
% have to run bibtex mydocument.aux after first run to generate bbl file. 
\usepackage{appendixnumberbeamer}
\usepackage{xcolor}
\usepackage{subcaption}
%table
\usepackage{makecell}
\usepackage{multirow}
\usepackage{bigdelim}

\newif\ifabbreviation
\pretocmd{\thebibliography}{\abbreviationfalse}{}{}
\AtBeginDocument{\abbreviationtrue}
\DeclareRobustCommand\acroauthor[2]{%
	\ifabbreviation #2\else #1 (\mbox{#2})\fi}

\usepackage[customcolors]{hf-tikz}
\definecolor{sonja}{cmyk}{1.5,0,0.9,0.3}
%\definecolor{blue}{cmyk}{0,1,0,0}
\hfsetfillcolor{black!10}
\hfsetbordercolor{black}

\usepackage{tikz}
\usetikzlibrary{tikzmark}
\usetikzlibrary{decorations.markings}
\usepackage{tikz-cd}
\usetikzlibrary{arrows,calc,fit}
\tikzset{mainbox/.style={draw=white, text=white, fill=gray, rectangle, rounded corners, thick, node distance=7em, text width=8em, text centered, minimum height=3.5em}}
\tikzset{dummybox/.style={draw=none, text=white , rectangle, rounded corners, thick, node distance=7em, text width=8em, text centered, minimum height=3.5em}}
\tikzset{box/.style={draw , rectangle, rounded corners, thick, node distance=7em, text width=8em, text centered, minimum height=3.5em}}
\tikzset{container/.style={draw, rectangle, dashed, inner sep=2em}}
\tikzset{line/.style={draw, very thick, -latex'}}
\tikzset{    pil/.style={
		->,
		thick,
		shorten <=2pt,
		shorten >=2pt,}}
\tikzstyle{vecArrow} = [thick, decoration={markings,mark=at position
	1 with {\arrow[semithick]{open triangle 60}}},
double distance=1.4pt, shorten >= 5.5pt,
preaction = {decorate},
postaction = {draw,line width=1.4pt, white,shorten >= 4.5pt}]
\usetikzlibrary{shapes}
\renewcommand{\figurename}{}

%TITLE
\author[Sonja Dobkowitz]{\small Sonja Dobkowitz\\ \footnotesize{University of Bonn%, RTG 2281 The Macroeconomics of Inequality}
}\\ }
%\institute[University of Bonn]{}
\title{Meeting Climate Targets: The Optimal Fiscal Policy Mix}

\newcommand{\ar}{$\Rightarrow$ \ }

%\addtobeamertemplate{navigation symbols}{}{%
%    \usebeamerfont{footline}%
%    \usebeamercolor[fg]{footline}%
%    \hspace{1em}%
%   \insertframenumber/\inserttotalframenumber
%}

%\institute{University of Bonn} 
\date{\small{King's College and Collège de France Ph.D. Research Day\\ November 15, 2022 }} 
%\subject{} 
\begin{document}

\tikzstyle{modus}=[rectangle,inner sep=5mm,align=center, draw]
\tikzstyle{dialog}=[diamond, align=center, draw]
\tikzstyle{sphere}=[circle, align=center, dotted, minimum size=3cm, draw]
\tikzstyle{circll}=[circle, align=center, minimum size=3cm, draw]
\tikzstyle{circllsmall}=[circle, align=center, minimum size=2cm, draw]
{\setbeamertemplate{footline}{}
	\begin{frame}
		\titlepage
	\end{frame}
}
%\addtocounter{framenumber}{-1}

% {\setbeamertemplate{footline}{}
	% \begin{frame}{Content}
		% \vspace{4mm}
		% \tableofcontents
		% \end{frame}
	% }
%\addtocounter{framenumber}{-1}


%---------------------------------------
%            Intro
%---------------------------------------
%\section{ Reducing Consumption Levels}
\addtocounter{framenumber}{-1}
\begin{frame}{Motivation}
	
	\begin{itemize}[<+-| alert@+>]
		\setbeamercolor{alerted text}{} %change the font color
		\setbeamerfont{alerted text}{}
		%	\item we are facing  environmental limits 
		%	\vspace{3mm}
		\item meeting climate targets requires a limit on emissions \citep{IPCC2022}
		\vspace{3mm}
		\item carbon taxes  direct  (i) demand towards emission-low alternatives\\ \hspace{23mm} \underline{and} (ii) research across sectors
		\vspace{3mm}
		\item labor income taxes affect the level of production 
		\vspace{3mm}
		\item \textbf{What is the optimal policy mix to meet the emission target?}
	\end{itemize}
\end{frame}


\begin{frame}{This paper}
	\vspace{-2mm}
	\begin{itemize}
		\item<+-> quantitative model of \alert{directed technical change} building on \cite{Fried2018ClimateAnalysis}
		\vspace{2mm}
		\item<+->   the government   chooses the \alert{path of carbon and income taxes} to maximize welfare\vspace{2mm}
		\item<+-> an \alert{emission limit} constrains the government
	\end{itemize}
	\pause
	\begin{center}
		\begin{figure}
			\centering
			\textbf{US net CO$_2$ emission limit in Gt}\\
			\vspace{2mm}	\includegraphics[width=0.38\textwidth]{../codding_model/own_basedOnFried/optimalPol_010922_revision/figures/all_13Sept22_Tplus30/Emnet.png}
		\end{figure}
	\end{center}
\end{frame}

\section{Model and Mechanisms}
\begin{frame}{Model}
\begin{figure}[h]
	%	\vspace{-4mm}
	\centering
	\begin{tikzpicture}[auto,scale=.7, transform shape]
		
		\node[circll] (A) at (-7,4)  {\textbf{{\hyperlink{prodmod}{Production}}}\\ \textbf{{and Research}}};
		\node[circll] (B) at (7,4) {\textbf{\hyperlink{backhh}{{Representative}}}\\ \textbf{\hyperlink{backhh}{{Household}}}};
		\node[circll] (D) at (0,9) {\textbf{Government}};
	\end{tikzpicture}
\end{figure}
\end{frame}

\addtocounter{framenumber}{-1}
\begin{frame}{Model}
	\begin{figure}[h]
		\vspace{-4mm}
		\centering
		\begin{tikzpicture}[auto,scale=.7, transform shape]
			\node[circll] (A) at (-7,4) {\textbf{{\hyperlink{prodmod}{Production}}}\\ \textbf{{and Research}}};
			\node[circll] (B) at (7,4) {\textbf{\hyperlink{backhh}{{Representative}}}\\ \textbf{\hyperlink{backhh}{{Household}}}};
			\node[circll] (D) at (0,9) {\textbf{Government}}; 

			\node[draw=none] (B1) at (5,4.25) {};
			\node[draw=none] (B2) at (5,3.5) {};
			\node[draw=none] (BA1) at (-5,4.25) {};
			\node[draw=none] (BA2) at (-5,3.5) {};
			\node[draw=none] (D1) at (1.8,7.8) {};
			
			\node[draw=none] (B22) at (6.4,5.6) {};
			\node[draw=none] (D2) at (2.3,8.5) {};
			\node[draw=none] (B3) at (5.3,5.3) {};
			\node[draw=none] (D3) at (-1.5,7) {};
			\node[draw=none] (A1) at (-4.2,4.6) {};
			

			
			\draw [->] (B1) to node[pos=0.75, swap]{Workers and scientists} (BA1);
			\draw [->] (BA2) to node[pos=0.75, swap]{Final good} (B2);
		\end{tikzpicture}
		
	\end{figure}
\end{frame}
\addtocounter{framenumber}{-1}
\begin{frame}{Model}
	\begin{figure}[h]
		\vspace{-4mm}
		\centering
		\begin{tikzpicture}[auto,scale=.7, transform shape]
			\node[circll] (A) at (-7,4) {\textbf{{\hyperlink{prodmod}{Production}}}\\ \textbf{{and Research}}};
			\node[circll] (B) at (7,4) {\textbf{\hyperlink{backhh}{{Representative}}}\\ \textbf{\hyperlink{backhh}{{Household}}}};
			\node[circll] (D) at (0,9) {\textbf{\alert{Government}}\\ \textbf{max welfare} \\ \textbf{s.t. emission limit} }; 
			\node[draw=none] (B1) at (5,4.25) {};
			\node[draw=none] (B2) at (5,3.5) {};
			\node[draw=none] (BA1) at (-5,4.25) {};
			\node[draw=none] (BA2) at (-5,3.5) {};
			\node[draw=none] (D1) at (1.8,7.8) {};
			
			\node[draw=none] (B22) at (6.4,5.6) {};
			\node[draw=none] (D2) at (2.3,8.5) {};
			\node[draw=none] (B3) at (5.3,5.3) {};
			\node[draw=none] (D3) at (-1.5,7) {};
			\node[draw=none] (A1) at (-4.2,4.6) {};
			\node[draw=none] (D4) at (-2,8.3) {};
			\node[draw=none] (A4) at (-6,5.4) {};
			%\draw [->] (B22) to node[pos=0.5, swap]{\alert{Tax on labor, $\pmb{\tau_{\iota}}$}} (D2);
			%\draw [->] (D1) to node[pos=0.5, swap]{Transfers} (B3);
				
			\draw [->] (B1) to node[pos=0.75, swap]{Workers and scientists} (BA1);
			\draw [->] (BA2) to node[pos=0.75, swap]{Final good} (B2);
   		%	\draw [->] (A4) to node[pos=0.5, swap]{\alert{Tax on carbon, $\pmb{\tau_F}$}}   (D4);
		\end{tikzpicture}
		
	\end{figure}
\end{frame}

\addtocounter{framenumber}{-1}
\begin{frame}{Model}
	\begin{figure}[h]
		\vspace{-4mm}
		\centering
		\begin{tikzpicture}[auto,scale=.7, transform shape]
			\node[circll] (A) at (-7,4) {\textbf{{\hyperlink{prodmod}{Production}}}\\ \textbf{{and Research}}};
			\node[circll] (B) at (7,4) {\textbf{\hyperlink{backhh}{{Representative}}}\\ \textbf{\hyperlink{backhh}{{Household}}}};
			\node[circll] (D) at (0,9) {\textbf{\alert{Government}}\\ \textbf{max welfare} \\ \textbf{s.t. emission limit} }; 
			\node[draw=none] (B1) at (5,4.25) {};
			\node[draw=none] (B2) at (5,3.5) {};
			\node[draw=none] (BA1) at (-5,4.25) {};
			\node[draw=none] (BA2) at (-5,3.5) {};
			\node[draw=none] (D1) at (1.8,7.8) {};
			
			\node[draw=none] (B22) at (6.4,5.6) {};
			\node[draw=none] (D2) at (2.3,8.5) {};
			\node[draw=none] (B3) at (5.3,5.3) {};
			\node[draw=none] (D3) at (-1.5,7) {};
			\node[draw=none] (A1) at (-4.2,4.6) {};
			\node[draw=none] (D4) at (-2,8.3) {};
			\node[draw=none] (A4) at (-6,5.4) {};
			\draw [->] (B22) to node[pos=0.65, swap]{\alert{Tax on labor, $\pmb{\tau_{\iota}}$:}} (D2);
				\draw [->] (B22) to node[pos=0.45, swap]{\alert{$\pmb{\tau_{\iota}}>0$:  labor supply $\downarrow$ \ar emissions $\downarrow$}} (D2);
		%	\draw [->] (B22) to node[pos=0.3, swap]{\alert{$\pmb{\tau_{\iota}}<0$:    labor supply $\uparrow$ \ar output $\uparrow$}} (D2);
			\draw [->] (D1) to node[pos=0.5, swap]{Transfers} (B3);
			
			\draw [->] (B1) to node[pos=0.75, swap]{Workers and scientists} (BA1);
			\draw [->] (BA2) to node[pos=0.75, swap]{Final good} (B2);
		%	\draw [->] (A4) to node[pos=0.5, swap]{\alert{Tax on carbon, $\pmb{\tau_F}$}}   (D4);
		\end{tikzpicture}
		
	\end{figure}
\end{frame}

\addtocounter{framenumber}{-1}
\begin{frame}{Model}
	\begin{figure}[h]
		\vspace{-4mm}
		\centering
		\begin{tikzpicture}[auto,scale=.7, transform shape]
			\node[circll] (A) at (-7,4) {\textbf{{\hyperlink{prodmod}{Production}}}\\ \textbf{{and Research}}};
			\node[circll] (B) at (7,4) {\textbf{\hyperlink{backhh}{{Representative}}}\\ \textbf{\hyperlink{backhh}{{Household}}}};
			\node[circll] (D) at (0,9) {\textbf{\alert{Government}}\\ \textbf{max welfare} \\ \textbf{s.t. emission limit} }; 
			\node[draw=none] (B1) at (5,4.25) {};
			\node[draw=none] (B2) at (5,3.5) {};
			\node[draw=none] (BA1) at (-5,4.25) {};
			\node[draw=none] (BA2) at (-5,3.5) {};
			\node[draw=none] (D1) at (1.8,7.8) {};
			
			\node[draw=none] (B22) at (6.4,5.6) {};
			\node[draw=none] (D2) at (2.3,8.5) {};
			\node[draw=none] (B3) at (5.3,5.3) {};
			\node[draw=none] (D3) at (-1.5,7) {};
			\node[draw=none] (A1) at (-4.2,4.6) {};
			\node[draw=none] (D4) at (-2,8.3) {};
			\node[draw=none] (A4) at (-6,5.4) {};
			\draw [->] (B22) to node[pos=0.65, swap]{\alert{Tax on labor, $\pmb{\tau_{\iota}}$:}} (D2);
			\draw [->] (B22) to node[pos=0.45, swap]{\alert{$\pmb{\tau_{\iota}}>0$:  labor supply $\downarrow$ \ar emissions $\downarrow$}} (D2);
			\draw [->] (B22) to node[pos=0.3, swap]{\alert{$\pmb{\tau_{\iota}}<0$:    labor supply $\uparrow$ \ar output $\uparrow$}} (D2);
			\draw [->] (D1) to node[pos=0.5, swap]{Transfers} (B3);
			
			\draw [->] (B1) to node[pos=0.75, swap]{Workers and scientists} (BA1);
			\draw [->] (BA2) to node[pos=0.75, swap]{Final good} (B2);
			%	\draw [->] (A4) to node[pos=0.5, swap]{\alert{Tax on carbon, $\pmb{\tau_F}$}}   (D4);
		\end{tikzpicture}
		
	\end{figure}
\end{frame}

\addtocounter{framenumber}{-1}
\begin{frame}{Model}
	\begin{figure}[h]
		\vspace{-4mm}
		\centering
		\begin{tikzpicture}[auto,scale=.7, transform shape]
			\node[circll] (A) at (-7,4) {\textbf{{\hyperlink{prodmod}{Production}}}\\ \textbf{{and Research}}};
			\node[circll] (B) at (7,4) {\textbf{\hyperlink{backhh}{{Representative}}}\\ \textbf{\hyperlink{backhh}{{Household}}}};
			\node[circll] (D) at (0,9) {\textbf{\alert{Government}}\\ \textbf{max welfare} \\ \textbf{s.t. Emission limit} }; 
			\node[draw=none] (B1) at (5,4.25) {};
			\node[draw=none] (B2) at (5,3.5) {};
			\node[draw=none] (BA1) at (-5,4.25) {};
			\node[draw=none] (BA2) at (-5,3.5) {};
			\node[draw=none] (D1) at (1.8,7.8) {};
			
			\node[draw=none] (B22) at (6.4,5.6) {};
			\node[draw=none] (D2) at (2.3,8.5) {};
			\node[draw=none] (B3) at (5.3,5.3) {};
			\node[draw=none] (D3) at (-1.5,7) {};
			\node[draw=none] (A1) at (-4.2,4.6) {};
			\node[draw=none] (D4) at (-2,8.3) {};
			\node[draw=none] (A4) at (-6,5.4) {};
			\draw [->] (B22) to node[pos=0.5, swap]{{Tax on labor, $\pmb{\tau_{\iota}}$}} (D2);
			\draw [->] (D1) to node[pos=0.5, swap]{Transfers} (B3);
			
			\draw [->] (B1) to node[pos=0.75, swap]{Workers and scientists} (BA1);
			\draw [->] (BA2) to node[pos=0.75, swap]{Final good} (B2);
				\draw [->] (A4) to node[pos=0.5, swap]{\alert{Tax on carbon, $\pmb{\tau_F}$}}   (D4);
		\end{tikzpicture}
		
	\end{figure}
\end{frame}
\begin{frame}{Production}
	\begin{figure}[h]
		\vspace{-10mm}
		\centering
		\begin{tikzpicture}[auto,scale=.7, transform shape]
			\node[circll] (A) at (0,17) {\textbf{Final}\textbf{ Good}}; 
			\node[circll] (B) at (-6,14) {\textbf{Energy}};
			\node[circll] (C) at (5,14) {\textbf{{Non-energy}}};
			\node[circll] (D) at (-10,12) {\textbf{{Fossil}}};
			\node[circll] (E) at (-2,12) {\textbf{{Green}}};
				\node[sphere] (Ems) at (-11,16) {\textbf{Emissions}};
			
			%		\node[circllsmall] (CM) at (4.6,8) {\textbf{{Machines}}};
			%		\node[circllsmall] (CL) at (7.4,8) {\textbf{{Labor}}};			\node[circllsmall] (EM) at (-3.4,8) {\textbf{{Machines}}};
			%	    \node[circllsmall] (EL) at (-0.6,8) {\textbf{{Labor}}};
			%	    \node[circllsmall] (DM) at (-11.4,8) {\textbf{{Machines}}};
			%	    \node[circllsmall] (DL) at (-8.6,8) {\textbf{{Labor}}};
			
			
			\draw [->] (B) to node[pos=0.5, swap]{} (A);
			\draw [->] (C) to node[pos=0.5, swap]{} (A);
			
			\draw [->] (E) to node[pos=0.5, swap]{} (B);
			\draw [->] (D) to node[pos=0.5, swap]{} (B);
			
			%		
			%		\draw [->] (EM) to node[pos=0.5, swap]{} (E);
			%		\draw [->] (EL) to node[pos=0.5, swap]{} (E);
			%		
			%		
			%		\draw [->] (DM) to node[pos=0.5, swap]{} (D);
			%		\draw [->] (DL) to node[pos=0.5, swap]{} (D);
			%		
			%		
			%		\draw [->] (CM) to node[pos=0.5, swap]{} (C);
			%		\draw [->] (CL) to node[pos=0.5, swap]{} (C);
			
				\draw [->] (D) to node[pos=0.5, swap]{} (Ems);
		\end{tikzpicture}
		
	\end{figure}
\end{frame}

\begin{frame}{1st: Effect of a carbon tax on \alert{fossil demand}}
%	\begin{minipage}{0.4\textwidth}
	\begin{figure}[h]
	\vspace{-4mm}
	\centering
	\begin{tikzpicture}[auto,scale=.7, transform shape]
		%			\node[circll] (A) at (0,16) {\textbf{Final}\textbf{ Good}}; 
		\node[circll] (B) at (-6,14) {\textbf{Energy}};
		%			\node[circll] (C) at (5,14) {\textbf{{Non-energy}}};
		\node[circll] (D) at (-10,12) {\textbf{{Fossil}}};
		\node[circll] (E) at (-2,12) {\textbf{{Green}}};
		\node[sphere] (Ems) at (-13,15) {\textbf{Emissions}};
%		\node[modus] (DemF) at (-6, 9.2){
%			\huge	${F_t} = \left(\frac{p_{Gt}}{p_{Ft}+\alert{\pmb{\tau_{Ft}}}}\right)^{\varepsilon_e}G_t$};

		
		\draw [->] (E) to node[pos=0.5, swap]{} (B);
		\draw [->] (D) to node[pos=0.5, swap]{} (B);
		\draw [->] (D) to node[pos=0.5, swap]{} (Ems);
	\end{tikzpicture}
\end{figure}
\begin{itemize}
	\item \alert{carbon tax lowers demand for fossil and raises demand for green energy}
	\item[]
\end{itemize}
\end{frame}

\begin{frame}{2nd: Effect of a carbon tax on \alert{research activity}}
	%	\begin{minipage}{0.4\textwidth}
		\begin{figure}[h]
			\vspace{-4mm}
			\centering
			\begin{tikzpicture}[auto,scale=.7, transform shape]
						\node[circll] (D) at (-10,10) {\textbf{Fossil}};
			\node[circll] (E) at (-4,10) {\textbf{Green}};
			
			\node[circll] (F) at (2,10) { \textbf{Non-energy}};
			
			\node[circll] (S) at (-4,6) {\textbf{{Scientists}}};
			
			\draw [->] (S) to node[pos=0.5, swap]{} (D);
			\draw [->] (S) to node[pos=0.5, swap]{} (E);
			\draw [->] (S) to node[pos=0.5, swap]{} (F);
			\end{tikzpicture}
		\end{figure}
	\pause
		\begin{itemize}
			\item \alert{carbon tax lowers revenues in fossil sector \ar returns to research in fossil sector $\downarrow$ \ar scientists shift from fossil to green sector}
		\end{itemize}
		
	\end{frame}

\begin{comment}
\begin{frame}{Machines and Labor}
	\begin{figure}[h]
		\vspace{-10mm}
		\centering
		\begin{tikzpicture}[auto,scale=.7, transform shape]
			\node[circll] (A) at (0,16) {\textbf{Final}\textbf{ Good}}; 
			\node[circll] (B) at (-6,14) {\textbf{Energy}};
			\node[circll] (C) at (5,14) {\textbf{{Non-energy}}};
			\node[circll] (D) at (-10,12) {\textbf{{Fossil}}};
			\node[circll] (E) at (-2,12) {\textbf{{Green}}};
			%			\node[sphere] (Ems) at (-11,16) {\textbf{CO$_2$}\\\textbf{Emissions}};
			
			\node[circllsmall] (CM) at (4.6,8) {\textbf{{Machines}}};
			\node[circllsmall] (CL) at (7.4,8) {\textbf{{Labor}}};			\node[circllsmall] (EM) at (-3.4,8) {\textbf{{Machines}}};
			\node[circllsmall] (EL) at (-0.6,8) {\textbf{{Labor}}};
			\node[circllsmall] (DM) at (-11.4,8) {\textbf{{Machines}}};
			\node[circllsmall] (DL) at (-8.6,8) {\textbf{{Labor}}};
			
			
			\draw [->] (B) to node[pos=0.5, swap]{} (A);
			\draw [->] (C) to node[pos=0.5, swap]{} (A);
			
			\draw [->] (E) to node[pos=0.5, swap]{} (B);
			\draw [->] (D) to node[pos=0.5, swap]{} (B);
			
			
			\draw [->] (EM) to node[pos=0.5, swap]{} (E);
			\draw [->] (EL) to node[pos=0.5, swap]{} (E);
			
			
			\draw [->] (DM) to node[pos=0.5, swap]{} (D);
			\draw [->] (DL) to node[pos=0.5, swap]{} (D);
			
			
			\draw [->] (CM) to node[pos=0.5, swap]{} (C);
			\draw [->] (CL) to node[pos=0.5, swap]{} (C);
			
			%			\draw [->] (D) to node[pos=0.5, swap]{} (Ems);
		\end{tikzpicture}
		
	\end{figure}
\end{frame}

\addtocounter{framenumber}{-1}

\begin{frame}{Research}
	\begin{figure}[h]
		\vspace{-10mm}
		\centering
		\begin{tikzpicture}[auto,scale=.7, transform shape]
			\node[circllsmall] (E) at (-2,12) {\textbf{{Sector $J$}}};
			\node[circllsmall] (EM) at (-2,8) {\textbf{{\alert{Machines}}}};
			\node[circllsmall] (ES) at (-2,4) {\textbf{{\alert{Scientists}}}};
			
			\draw [->] (E) to node[pos=0.5, swap]{$\text{demand}(\underset{+}{A_J})$} (EM);
			\draw [->] (EM) to node[pos=0.5, swap]{$s_J^d$}(ES);
		\end{tikzpicture}
		
	\end{figure}
\end{frame}


content...
\begin{frame}{Carbon tax: Effect on Research}
	\begin{figure}[h]
		\vspace{0mm}
		\centering
		\begin{tikzpicture}[auto,scale=.47, transform shape]
			%			\node[circll] (A) at (0,16) {\textbf{Final}\textbf{ Good}}; 
			%			\node[circll] (B) at (-6,14) {\textbf{Energy}};
			%			\node[circll] (C) at (5,14) {\textbf{{Non-energy}}};
			\node[circll] (D) at (-10,10) {\textbf{Fossil}};
			\node[circll] (E) at (-4,10) { \textbf{Green}};
			
			\node[circll] (F) at (2,10) {\ \textbf{Non-energy}};
			
			\node[circll] (S) at (-4,6) {\textbf{{Scientists}}};
			
			\draw [->] (S) to node[pos=0.5, swap]{} (D);
			\draw [->] (S) to node[pos=0.5, swap]{} (E);
			\draw [->] (S) to node[pos=0.5, swap]{} (F);
		\end{tikzpicture}
	\end{figure}
	\begin{itemize}
		\item $\tau_F \uparrow \Rightarrow p_F F \downarrow$ and  $p_G G \uparrow$
		\begin{align*}
			w_{sF}\left(\underbrace{p_FF}_{+},\underbrace{s_F}_{-}\right)\alert{\pmb{<}}	w_{sG}\left(\underbrace{p_GG}_{+},\underbrace{s_G}_{-}\right)
		\end{align*}
		\item[] % $s_F\downarrow$ and $s_G \uparrow$ in new equilibrium
	\end{itemize}
\end{frame}
\addtocounter{framenumber}{-1}
\begin{frame}{Carbon tax: Effect on Research}
	\begin{figure}[h]
		\vspace{0mm}
		\centering
		\begin{tikzpicture}[auto,scale=.47, transform shape]
			%			\node[circll] (A) at (0,16) {\textbf{Final}\textbf{ Good}}; 
			%			\node[circll] (B) at (-6,14) {\textbf{Energy}};
			%			\node[circll] (C) at (5,14) {\textbf{{Non-energy}}};
			\node[circll] (D) at (-10,10) {\textbf{Machines }\\\textbf{Fossil}};
			\node[circll] (E) at (-4,10) {\textbf{Machines}\\ \textbf{Green}};
			
			\node[circll] (F) at (2,10) {\textbf{Machines}\\ \textbf{Non-energy}};
			
			\node[circll] (S) at (-4,6) {\textbf{{Scientists}}};
			
			\draw [->] (S) to node[pos=0.5, swap]{} (D);
			\draw [->] (S) to node[pos=0.5, swap]{} (E);
			\draw [->] (S) to node[pos=0.5, swap]{} (F);
		\end{tikzpicture}
	\end{figure}
	\begin{itemize}
		\item $\tau_F \uparrow \Rightarrow p_F F \downarrow$ and  $p_G G \uparrow$
		\begin{align*}
			w_{sF}\left(\underbrace{p_FF}_{+},\underbrace{s_F}_{-}\right)\alert{\pmb{<}}	w_{sG}\left(\underbrace{p_GG}_{+},\underbrace{s_G}_{-}\right)
		\end{align*}
		\item[\ar] $s_F\downarrow$ and $s_G \uparrow$ in new equilibrium
	\end{itemize}
\end{frame}

	
	\begin{frame}{Carbon tax: 2. Effect on Research}
		\vspace{-7mm}
		In equilibrium: \large
		\begin{align*}
			\overbrace{{\psi_F} \underbrace{p_F{F}}_{\tau_F\uparrow\Rightarrow\downarrow}\frac{\partial A_{F}}{\partial s_{F}}}^{\text{wage fossil scientists}}=\overbrace{{\psi_G} \underbrace{p_G{G}}_{\tau_F\uparrow\Rightarrow\uparrow}\frac{\partial A_{G}}{\partial s_{G}}}^{\text{wage green scientists}}
		\end{align*}
		\normalsize
		\begin{itemize}
			\item carbon tax lowers returns to fossil research and raises returns to green research
			\item in equilibrium: scientists transition from fossil to green sector
		\end{itemize}
		\small
		\vspace{0mm}
		\hspace{-4mm}
		\begin{minipage}[t!]{0.3\textwidth}
			\vspace{0mm}
			\begin{itemize}
				\item[] $p_JJ$: revenues sector J
			\end{itemize}
		\end{minipage}
		\vspace{-5mm}
		\begin{minipage}[t!]{0.5\textwidth}
			\vspace{0mm}
			\begin{itemize}	
				\item[] $s_J$: scientists sector J
			\end{itemize}
		\end{minipage}
	\end{frame}
	
\end{comment}

%\begin{frame}
%	Having seen how the carbon tax affects allocations, how does it serve to meet government goals?
%\end{frame}

\begin{frame}{In a nutshell: Government trade-off and instruments}
\pause
	\begin{itemize}[<+-| alert@+>]
		\setbeamercolor{alerted text}{} 
		\setbeamerfont{alerted text}{}
		\item 	Goal of government intervention
		\begin{enumerate}
			\item[a)] lower emissions
			\item[b)] keep productivity high
		\end{enumerate}
	\vspace{3mm}
	\item Carbon tax
	\begin{enumerate}
		\item[a)] reduces emissions by lowering fossil demand
		\item[b)] directs research across sectors
		\begin{itemize}
			\item[-] if want to foster green research
			\ar higher carbon tax \ar % but reduces returns to labor %\ar
			 costly in terms of output % reduces share of fossil energy 
			\item[-] if want to foster fossil research \ar smaller carbon tax \ar but too high emissions
		\end{itemize}
	\end{enumerate}
\item Labor income tax can be used to counter side effects of carbon tax 
	\end{itemize}
\end{frame}


%\begin{frame}
%	What determines the optimal allocation of research? 
%\end{frame}
% when to direct research to which sector
\begin{frame}{Productivity of research}
	\vspace{-10mm}
	What determines the optimal allocation of scientists?
	\vspace{4mm}
	\pause
	% talk about productivity of research bcs it determines optimal allocation of research
\large
\begin{align*}
		A_{Jt}=\alert{A_{Jt-1}}\left(1+\gamma\left(\frac{s_{Jt}}{\rho_J}\right)^\eta\left(\frac{A_{t-1}}{A_{Jt-1}}\right)^\phi\right)
\end{align*}
\normalsize
\begin{enumerate}
		\item \alert{path dependency: sector-specific knowledge renders scientists more productive}
%		\begin{itemize}
%	\item<+-> scientists in most advanced, fossil sector are more productive
%			\item<+-> shift to green research early on to make green research more productive tomorrow
%		\end{itemize}
	\item[] % knowledge from other sectors increases productivity of scientists
	\item[] % decreasing returns to research, $\eta<1$
\end{enumerate}
\end{frame}

\addtocounter{framenumber}{-1}
\begin{frame}{Productivity of research}
	\large
	\begin{align*}
		A_{Jt}={A_{Jt-1}}\left(1+\gamma\left(\frac{s_{Jt}}{\rho_J}\right)^\eta\alert{\left(\frac{A_{t-1}}{A_{Jt-1}}\right)^\phi}\right)
	\end{align*}
\normalsize
	\begin{enumerate}
		\item path dependency: sector-specific knowledge renders scientists more productive
%				\begin{itemize}
%			\item scientists in most advanced, fossil sector are more productive
%			\item shift to green early on to internalize green productivity increase tomorrow
%		\end{itemize}
%		\item decreasing returns to research, $\eta<1$
		\item \alert{knowledge spillovers} % knowledge from other sectors increases research productivity 
%		\pause
%		\begin{itemize}
%			\item fossil research increases green research productivity in future
%		\end{itemize}
	\item[] % knowledge spillovers may allow to profit from high research productivity in fossil sector while boosting green productivity tomorrow
	\end{enumerate}
\end{frame}

%\addtocounter{framenumber}{-1}
%\begin{frame}{Productivity of research}
%	\large
%	\begin{align*}
%		A_{Jit}={A_{Jt-1}}\left(1+\gamma\left(\frac{s_{Jit}}{\rho_J}\right)^\eta\alert{\left(\frac{A_{t-1}}{A_{Jt-1}}\right)^\phi}\right)
%	\end{align*}
%	\normalsize
%	\begin{enumerate}
%		\item path dependency: accumulated knowledge in sector renders scientists more productive
%		\begin{itemize}
%			\item scientists in most advanced, fossil sector are more productive
%			\item shift to green early on to internalize green productivity increase tomorrow
%		\end{itemize}
%		%		\item decreasing returns to research, $\eta<1$
%		\item {knowledge spillovers:} knowledge from other sectors increases research productivity
%%		\begin{itemize}
%%			\item fossil research increases green research productivity in future
%%		\end{itemize}
%%		\item[\ar] \alert{knowledge spillovers may allow to profit from high research productivity in fossil sector while boosting green productivity tomorrow}
%	\end{enumerate}
%\end{frame}
%
%\addtocounter{framenumber}{-1}
%\begin{frame}{Productivity of research}
%	\large
%	\begin{align*}
%		A_{Jit}={A_{Jt-1}}\left(1+\gamma\left(\frac{s_{Jit}}{\rho_J}\right)^\eta\alert{\left(\frac{A_{t-1}}{A_{Jt-1}}\right)^\phi}\right)
%	\end{align*}
%	\normalsize
%	\begin{enumerate}
%		\item scientists more productive in advanced sectors \ar path dependency \ar optimal to have more scientists in most advanced, fossil sector 
%		\item decreasing returns to research, $\eta<1$
%		\item {knowledge spillovers:} knowledge from other sectors increases productivity of scientists
%		\ar fossil research increases green technology level	
%	\end{enumerate}
%\alert{\textbf{\ar Could be optimal to maintain some fossil research}}
%\end{frame}

\section{Results}


\begin{frame}{Optimal Policy}
	\begin{figure}[h!!]
		
		\begin{subfigure}{0.4\textwidth}		
			\caption{Tax per ton of carbon,  US\$}
			%	\captionsetup{width=.45\linewidth}
			\includegraphics[width=1\textwidth]{../codding_model/own_basedOnFried/optimalPol_010922_revision/figures/all_13Sept22_Tplus30/Single_periods12_OPT_T_NoTaus_Tauf_regime4_spillover0_knspil0_noskill0_sep0_xgrowth0_extern0_PV1_sizeequ0_GOV0_etaa0.79.png}
		\end{subfigure}	
		\begin{minipage}[]{0.1\textwidth}
			\ 
		\end{minipage}
		\begin{subfigure}{0.4\textwidth}		
			\caption{Marginal tax rate}
			%	\captionsetup{width=.45\linewidth}
			\includegraphics[width=1\textwidth]{../codding_model/own_basedOnFried/optimalPol_010922_revision/figures/all_13Sept22_Tplus30/dTaulAv_OPT_T_NoTaus_COMPtaul_regime4_spillover0_knspil0_noskill0_sep0_xgrowth0_PV1_etaa0.79_lgd0.png}
		\end{subfigure}
	\end{figure}
	\pause
	\begin{itemize}
		\item In run-up to net-zero limit: labor income tax reduces emissions
		\item Under net-zero limit: subsidy on labor boosts production
		% the emission limit is too strict to maintain some fossil research bcs it comes with a higher fossil to green energy ratio. It is then optimal to at least internalize green dynamic spillovers
	\end{itemize}
\end{frame}

\begin{frame}{Mechanism: Deviation from carbon-tax only policy}
	\pause
	\begin{figure}[h!!]
		\begin{subfigure}{0.4\textwidth}
		\caption{{\% Deviation carbon tax}}
		%	\captionsetup{width=.45\linewidth}
		\includegraphics[width=1\textwidth]{../codding_model/own_basedOnFried/optimalPol_010922_revision/figures/all_13Sept22_Tplus30/Tauf_OPT_T_NoTaus_COMPtaulPer_regime4_spillover0_knspil0_noskill0_sep0_xgrowth0_PV1_etaa0.79.png}
	\end{subfigure}
\begin{minipage}[]{0.1\textwidth}
\ 
\end{minipage}
\begin{subfigure}{0.4\textwidth}
	\caption{{\% Deviation fossil-to-green scientists}}
	%	\captionsetup{width=.45\linewidth}
	\includegraphics[width=1\textwidth]{../codding_model/own_basedOnFried/optimalPol_010922_revision/figures/all_13Sept22_Tplus30/sffsg_OPT_T_NoTaus_COMPtaulPer_regime4_spillover0_knspil0_noskill0_sep0_xgrowth0_PV1_etaa0.79.png}
\end{subfigure}
	\end{figure}
\pause
\begin{itemize}
\item In run-up to net-zero limit: lower carbon tax to maintain some fossil research
\item Under net-zero limit: higher carbon tax to foster green research %, and a subsidy on labor to stabilize output
 % the emission limit is too strict to maintain some fossil research bcs it comes with a higher fossil to green energy ratio. It is then optimal to at least internalize green dynamic spillovers
\end{itemize}
\end{frame}

\section{Conclusion}
\begin{frame}{Conclusion}
	\begin{itemize}[<+-| alert@+>]
		\setbeamercolor{alerted text}{} %change the font color
		\setbeamerfont{alerted text}{}
		\item I study the optimal mix of taxes on carbon and labor to meet emission targets
			\vspace{3mm}
		\item Before the net-zero emission limit: 
		\begin{itemize}
			\item[-] lower carbon tax to maintain some fossil research
			\item[-] a tax on labor reduces emissions
		\end{itemize}
		\vspace{3mm}
	\item Under the net-zero emission limit: 
	\begin{itemize}
		\item[-]  high carbon tax to reduce fossil demand and boost green research
		\item[-]  a subsidy on labor stabilizes output
	\end{itemize}
\item a huge gap between the social planner and optimal allocation calls for alternative policy measures
	\end{itemize}
\end{frame}


%\begin{frame}{Mechanism}
%\begin{itemize}
%	\item in theory: optimal carbon tax set so that emitters internalize social costs of emissions
%	\vspace{2mm}
%	\item but, without research subsidy, carbon tax also used to target the direction of research
%	\vspace{2mm}
%	\item \normalsize{causes distortions on the labor market:}\\ households do not correctly internalize the social costs of labor
%\end{itemize}
%%\item<+->  \alert{sizable effect of skill heterogeneity}: \\ with only one skill, the optimal income tax increases social welfare by 0.85\%
%
%\end{frame}

\section{Model}



\addtocounter{framenumber}{-1}




\begin{frame}{Representative household}
\hypertarget{backhh}{}
%\text{\textbf{Householdrt}}
\vspace{2mm}
\begin{minipage}[t!]{1\textwidth}
	\begin{align*}
		%	\tikzmarkin{first0}(1.5,2.7)(-1.2,-2.5)
		%	\underset{c_{s,i},c_{n,i}, l_i}{\max} \ \hspace{2mm} U(c_{s,i}, c_{n,i}, l_i; h_n)= 
		\max_{C_t, H_{t}, S_{t}} \log(C_t)-\chi\frac{H_{t}^{1+\sigma}}{1+\sigma}-\chi_s\frac{S_{t}^{1+\sigma}}{1+\sigma}
		\\
		\vspace{4mm}
		\\
		s.t.\ C_t=(\alert{\pmb{1-\tau_{\iota t}}})w_{t}H_{t}+w_{st}S_t+T_t%+Gov_t
		%\\
		%\hspace{2mm}\ H_{t}\leq \bar{H}; \hspace{4mm} S_{t}\leq \bar{H}
		%	\tikzmarkend{first0}
	\end{align*}
\end{minipage}

\small
\vspace{4mm}
\hspace{-8mm}
\begin{minipage}[t!]{0.26\textwidth}
	\vspace{7mm}
	\begin{itemize}
		\item[] $C_{t}$: consumption\vspace{-2mm}
		\item[] $H_{t}$: hours worker\vspace{-2mm}
		\item[] $S_{t}$: hours scientists\vspace{-2mm}
	\end{itemize}
\end{minipage}
\begin{minipage}[t!]{0.37\textwidth}
	\vspace{8mm}
	\begin{itemize}
		\item[] $w_{t}, w_{st}$: wages  \vspace{-2mm}
		\item[] $\tau_{\iota t}$: marginal income tax rate 
		\vspace{-2mm}	
		\item[] $T_{t}$: government transfers
		%		\vspace{-2mm}	
		%		\item[]%	$Gov_{t}$: government transfers
	\end{itemize}
\end{minipage}
\begin{minipage}[t!]{0.39\textwidth}
	\vspace{8mm}
	\begin{itemize}
		\item[] $\sigma$: curvature disutility of labor  \vspace{-2mm}
		\item[] $\chi$: disutility of work
		\vspace{-2mm}	
		\item[] $\chi_s$: disutility of research
		%		\vspace{-2mm}	
		%		\item[]%	$Gov_{t}$: government transfers
	\end{itemize}
\end{minipage}

\vspace{12mm}
\hfill	\hyperlink{labsup}{\tiny{$\rightarrow$ labor supply}}
\hypertarget{hhopt}{}
\end{frame}




\begin{frame}{Production: final and energy good}
\vspace{-10mm}
\hypertarget{prodmod}{}
\begin{align*}
	%		\tikzmarkin{first}(1.3,1.2)(-1,-0.8)
	\text{Final good}\hspace{4mm}&Y_t =\left(\delta_y^{\frac{1}{\varepsilon_y}}E_t^\frac{\varepsilon_y-1}{\varepsilon_y}+(1-\delta_y)^{\frac{1}{\varepsilon_y}}N_t^\frac{\varepsilon_y-1}{\varepsilon_y}\right)^\frac{\varepsilon_y}{\varepsilon_y-1} \\
	\ \\
	\text{Energy}\hspace{4mm}&E_t =\left({F}_t^\frac{\varepsilon_e-1}{\varepsilon_e}+G_t^\frac{\varepsilon_e-1}{\varepsilon_e}\right)^\frac{\varepsilon_e}{\varepsilon_e-1}\\
	\ \\
	\text{Demand energy producers}\hspace{4mm}&\frac{F_t}{G_t} = \left(\frac{p_{Gt}}{p_{Ft}+\alert{\pmb{\tau_{Ft}}}}\right)^{\varepsilon_e}
	%	\tikzmarkend{third}
\end{align*}

\small
\vspace{4mm}
\hspace{-4mm}
\begin{minipage}[t!]{0.23\textwidth}
	\vspace{0mm}
	\begin{itemize}	
		\item[]$F$: fossil energy
		\vspace{-2mm}	
		\item[]$G$: green energy
		\vspace{-7mm}	
		\item[]$N$: non-energy
	\end{itemize}
\end{minipage}
\begin{minipage}[t!]{0.22\textwidth}
	\vspace{0mm}
	\begin{itemize}
		\item[] $p_G$: price green  \vspace{-7mm}
		\item[] $p_F$: price fossil
		\vspace{-2mm}	
		\item[] $\tau_F$: carbon tax
	\end{itemize}
\end{minipage}
\begin{minipage}[t!]{0.55\textwidth}
	\vspace{0mm}
	\begin{itemize}
		\item[] $\delta_{y}$: weight on energy\vspace{-2mm}
		\item[] $\varepsilon_y$: elasticity of substitution E and N \vspace{-2mm}
		\item[] $\varepsilon_e$: elasticity of substitution F and G
	\end{itemize}
\end{minipage}
\end{frame}

%\addtocounter{framenumber}{-1}
\begin{frame}{Production: intermediate goods $J\in \{N,F,G\}$ }
\vspace{0mm}
%Competitive producers
\begin{align*}
	\underset{\{x_{Jit}\}_{i=0}^1, L_{Jt}}{\max}\ & p_{Jt}J_t-w_{t}L_{Jt}-\int_{0}^{1}p_{xJit}x_{Jit}di \\ \ \\
	s.t.\ & J_{t}=L_{Jt}^{1-\alpha_J}\int_{0}^{1}A^{1-\alpha_J}_{Jit}x_{Jit}^{\alpha_J}di
\end{align*}

\small
\vspace{10mm}
\hspace{-4mm}
\begin{minipage}[t!]{0.3\textwidth}
	\vspace{0mm}
	\begin{itemize}	
		\item[]$L_J\ $: labor 
		\vspace{-2mm}	
		\item[]$x_{Ji}\ $: machines 
		\vspace{-2mm}	
		\item[]$p_{xJi}$: price machine 
	\end{itemize}
\end{minipage}
\begin{minipage}[t!]{0.47\textwidth}
	\vspace{0mm}
	\begin{itemize}
		\item[] $A_{Ji}$: productivity machine $i$ sector $J$ \vspace{-2mm}
		\item[] $J$\ \  : sector $N,F,G$
		\vspace{-2mm}	
		\item[] $\alpha_J$\ : capital share 
	\end{itemize}
\end{minipage}
\end{frame}

\begin{frame}{Production: machines and innovation}
\vspace{-8mm}
\begin{align*}
	%	\tikzmarkin{sixth}(6.3,4)(-2.7,-3.8)
	\underset{p_{xJit}, s_{Jit}}{\max}\ & p_{xJit}(1+\zeta_{Jt})x_{Jit}-x_{Jit}-w_{st}s_{Jit}
	\\ 
	s.t.\ &(1)\ x_{Jit}=\left(\frac{\alpha_Jp_{Jt}}{p_{xJit}}\right)^{\frac{1}{1-\alpha_J}}L_{Jt}A_{Jit}\\ \ \\ %x_{ijt}= \left(\frac{p_{ft}(1-\tau_{jt})\alpha_j}{p_{xijt}}\right)^\frac{1}{1-\alpha_j}A_{ijt}L_{jt}\\
	& (2)\ A_{Jit}=A_{Jt-1}\left(1+\gamma\left(\frac{s_{Jit}}{\rho_J}\right)^\eta\left(\frac{A_{t-1}}{A_{Jt-1}}\right)^\phi\right)
	%	\tikzmarkend{sixth}
\end{align*}

\small
\vspace{4mm}
\hspace{-4mm}	\begin{minipage}[t!]{0.32\textwidth}
	\vspace{0mm}
	\begin{itemize}
		\item[-] monopolistic competition 
		\vspace{-4mm}
		\item[-] one-period patents
	\end{itemize}	
\end{minipage}
\begin{minipage}[t!]{0.3\textwidth}
	\vspace{0mm}
	\begin{itemize}	
		\item[]$\zeta_{Jt}$: subsidy
		\vspace{-2mm}	
		\item[]$s_{Ji}$: scientists
		\vspace{-2mm}	
		\item[]$\phi$: knowledge spillovers
	\end{itemize}
\end{minipage}
\begin{minipage}[t!]{0.32\textwidth}
	\vspace{0mm}
	\begin{itemize}
		\item[] $\eta$: returns to research  \vspace{-2mm}
		\item[] $\rho_j$: research processes
		\vspace{-2mm}	
		\item[] $\gamma$: productivity scientists
	\end{itemize}
\end{minipage}
\end{frame}




\begin{frame}{Production: returns to research}
\vspace{-2mm}
\begin{align*}
	%	w_{st}&=\left(\frac{p_{Jt}\alpha_J}{p_{xJit}}\right)^{\frac{1}{1-\alpha_J}}L_{Jt}A_{Jt-1}\gamma \eta \left(\frac{A_{t-1}}{A_{Jt-1}}\right)^{\phi}\left(\frac{s_{Jit}}{\rho_J}\right)^{\eta-1}\\
	w_{st}&=\underbrace{\left(\frac{\alert{p_{Jt}}\alpha_J}{p_{xJit}}\right)^{\frac{1}{1-\alpha_J}}\alert{L_{Jt}}}_{\frac{\partial x_{Jit}}{\partial A_{Jit}}}\gamma \eta A_{Jt-1} \left(\frac{A_{t-1}}{A_{Jt-1}}\right)^{\phi}\left(\frac{s_{Jit}}{\rho_J}\right)^{\eta-1}
	%&	\frac{\eta \gamma \left(\frac{A_{t-1}}{A_{Jt-1}}\right)^\phi(1-\alpha_J)\alpha_Js_{Jt}^{\eta-1}p_{Jt}J_t}{\rho_J^\eta}	
\end{align*}
\begin{itemize}
	\item \alert{carbon tax lowers demand for fossil machines and returns to fossil research}
	\item scientists more productive in advanced sectors \ar path dependency
	\item knowledge from other sectors increases productivity of scientists
	\item decreasing returns to research, $\eta<1$
\end{itemize}
\small
\vspace{7mm}
\hspace{-4mm}
\begin{minipage}[t!]{0.3\textwidth}
	\vspace{0mm}
	\begin{itemize}
		\item[] $\eta$: returns to research  \vspace{-2mm}
		\item[] $\rho_j$: research processes
	\end{itemize}
\end{minipage}
\vspace{-5mm}
\begin{minipage}[t!]{0.5\textwidth}
	\vspace{0mm}
	\begin{itemize}	
		\item[] $\gamma$: productivity scientists
		\vspace{-2mm}	
		\item[] $\phi$: knowledge spillovers, $\phi\geq0$
	\end{itemize}
\end{minipage}
\end{frame}

\addtocounter{framenumber}{-1}

\begin{frame}{Production: returns to research}
\vspace{-2mm}
\begin{align*}
	%	w_{st}&=\left(\frac{p_{Jt}\alpha_J}{p_{xJit}}\right)^{\frac{1}{1-\alpha_J}}L_{Jt}A_{Jt-1}\gamma \eta \left(\frac{A_{t-1}}{A_{Jt-1}}\right)^{\phi}\left(\frac{s_{Jit}}{\rho_J}\right)^{\eta-1}\\
	w_{st}&=\underbrace{\left(\frac{{p_{Jt}}\alpha_J}{p_{xJit}}\right)^{\frac{1}{1-\alpha_J}}{L_{Jt}}}_{\frac{\partial x_{Jit}}{\partial A_{Jit}}}\gamma \eta \alert{A_{Jt-1}} \left(\frac{A_{t-1}}{A_{Jt-1}}\right)^{\phi}\left(\frac{s_{Jit}}{\rho_J}\right)^{\eta-1}
	%&	\frac{\eta \gamma \left(\frac{A_{t-1}}{A_{Jt-1}}\right)^\phi(1-\alpha_J)\alpha_Js_{Jt}^{\eta-1}p_{Jt}J_t}{\rho_J^\eta}	
\end{align*}
\begin{itemize}
	\item carbon tax lowers demand for fossil machines and returns to fossil research
	\item \alert{scientists more productive in advanced sectors \ar path dependency}
	\item knowledge from other sectors increases productivity of scientists
	\item decreasing returns to research, $\eta<1$
\end{itemize}
\small
\vspace{7mm}
\hspace{-4mm}
\begin{minipage}[t!]{0.3\textwidth}
	\vspace{0mm}
	\begin{itemize}
		\item[] $\eta$: returns to research  \vspace{-2mm}
		\item[] $\rho_j$: research processes
	\end{itemize}
\end{minipage}
\vspace{-5mm}
\begin{minipage}[t!]{0.5\textwidth}
	\vspace{0mm}
	\begin{itemize}	
		\item[] $\gamma$: productivity scientists
		\vspace{-2mm}	
		\item[] $\phi$: knowledge spillovers, $\phi\geq0$
	\end{itemize}
\end{minipage}
\end{frame}



\addtocounter{framenumber}{-1}

\begin{frame}{Production: returns to research}
\vspace{-2mm}
\begin{align*}
	%	w_{st}&=\left(\frac{p_{Jt}\alpha_J}{p_{xJit}}\right)^{\frac{1}{1-\alpha_J}}L_{Jt}A_{Jt-1}\gamma \eta \left(\frac{A_{t-1}}{A_{Jt-1}}\right)^{\phi}\left(\frac{s_{Jit}}{\rho_J}\right)^{\eta-1}\\
	w_{st}&=\underbrace{\left(\frac{{p_{Jt}}\alpha_J}{p_{xJit}}\right)^{\frac{1}{1-\alpha_J}}{L_{Jt}}}_{\frac{\partial x_{Jit}}{\partial A_{Jit}}}\gamma \eta {A_{Jt-1}} \alert{\left(\frac{A_{t-1}}{A_{Jt-1}}\right)^{\phi}}\left(\frac{s_{Jit}}{\rho_J}\right)^{\eta-1}
\end{align*}
\begin{itemize}
	\item carbon tax lowers demand for fossil machines and returns to fossil research
	\item scientists more productive in advanced sectors \ar path dependency
	\item \alert{knowledge from other sectors increases productivity of scientists}
	\item decreasing returns to research, $\eta<1$ 
\end{itemize}
\small
\vspace{7mm}
\hspace{-4mm}
\begin{minipage}[t!]{0.3\textwidth}
	\vspace{0mm}
	\begin{itemize}
		\item[] $\eta$: returns to research  \vspace{-2mm}
		\item[] $\rho_j$: research processes
	\end{itemize}
\end{minipage}
\vspace{-5mm}
\begin{minipage}[t!]{0.5\textwidth}
	\vspace{0mm}
	\begin{itemize}	
		\item[] $\gamma$: productivity scientists
		\vspace{-2mm}	
		\item[] $\phi$: knowledge spillovers, $\phi\geq0$
	\end{itemize}
\end{minipage}
\end{frame}

\addtocounter{framenumber}{-1}

\begin{frame}{Production: returns to research}
\vspace{-2mm}
\begin{align*}
	%	w_{st}&=\left(\frac{p_{Jt}\alpha_J}{p_{xJit}}\right)^{\frac{1}{1-\alpha_J}}L_{Jt}A_{Jt-1}\gamma \eta \left(\frac{A_{t-1}}{A_{Jt-1}}\right)^{\phi}\left(\frac{s_{Jit}}{\rho_J}\right)^{\eta-1}\\
	w_{st}&=\underbrace{\left(\frac{{p_{Jt}}\alpha_J}{p_{xJit}}\right)^{\frac{1}{1-\alpha_J}}{L_{Jt}}}_{\frac{\partial x_{Jit}}{\partial A_{Jit}}}\gamma \eta {A_{Jt-1}} \left(\frac{A_{t-1}}{A_{Jt-1}}\right)^{\phi}\alert{\left(\frac{s_{Jit}}{\rho_J}\right)^{\eta-1}}
	%&	\frac{\eta \gamma \left(\frac{A_{t-1}}{A_{Jt-1}}\right)^\phi(1-\alpha_J)\alpha_Js_{Jt}^{\eta-1}p_{Jt}J_t}{\rho_J^\eta}	
\end{align*}
\begin{itemize}
	\item carbon tax lowers demand for fossil machines and returns to fossil research
	\item scientists more productive in advanced sectors \ar path dependency
	\item knowledge from other sectors increases productivity of scientists
	\item \alert{decreasing returns to research, $\eta<1$}
\end{itemize}
\small
\vspace{7mm}
\hspace{-4mm}
\begin{minipage}[t!]{0.3\textwidth}
	\vspace{0mm}
	\begin{itemize}
		\item[] $\eta$: returns to research  \vspace{-2mm}
		\item[] $\rho_j$: research processes
	\end{itemize}
\end{minipage}
\vspace{-5mm}
\begin{minipage}[t!]{0.5\textwidth}
	\vspace{0mm}
	\begin{itemize}	
		\item[] $\gamma$: productivity scientists
		\vspace{-2mm}	
		\item[] $\phi$: knowledge spillovers, $\phi\geq0$
	\end{itemize}
\end{minipage}
\end{frame}



\begin{frame}{Markets}
\begin{minipage}[t!]{1\textwidth}
	\begin{align*}
		%	\tikzmarkin{8th}(3.6,2.4)(-4.5,-2.2)
		H_{t}&=L_{Ft}+L_{Gt}+L_{Nt}\\
		S_{t}& = \int_{0}^{1}\left(s_{Fit}+s_{Git}+s_{Nit}\right)di\\
		Y_t&=C_t+\int_{0}^{1}\left(x_{Fit}+x_{Git}+x_{Nit}\right)di
		%	\tikzmarkend{8th}
	\end{align*}
\end{minipage}
\end{frame}




\begin{frame}{ Government}
\hypertarget{gov}{}
\vspace{-4mm}
\centering
\begin{minipage}[t!]{1\textwidth}
	\begin{align*}
		\max_{\{\tau_{Ft}\}_{t=0}^{\infty}, \{\tau_{\iota t}\}_{t=0}^{\infty}}&\hspace{3mm} \sum_{t=0}^{\infty}\beta^t U\left(C_t,H_t, S_t\right)\\ \ \\
		{s.t.} \hspace{4mm}
		&{ (1)\ \ T_t=\tau_{\iota t}w_{t}H_{t}+{{\tau_{Ft}}}F_{t}}+T_{\pi t}\\
		& {(2)\ \  \text{behavior of firms and households}}\\
		& {(3)\ \ \text{feasibility} }\\
		& \ % {(4)\ \  \omega F_t-\delta \leq\Omega_t }
	\end{align*}
\end{minipage}

\small
\vspace{0mm}
\hspace{-10mm}
\begin{minipage}[t!]{0.5\textwidth}
	\vspace{7mm}
	\begin{itemize}
		\item[] $\beta$\ \ : household discount factor\vspace{-2mm}
		\item[] $T_\pi$: profits minus subsidies \\ \hspace{5.5mm} from machine producers \vspace{0mm}
	\end{itemize}
\end{minipage}
\begin{minipage}[t!]{0.4\textwidth}
	\vspace{8mm}
	\begin{itemize}
		\item[] % $\Omega_{t}$: net emission limit
		\vspace{-2mm}	
		\item[] %$\omega$\ : emissions per unit of fossil \vspace{-2mm}
		\item[] %$\delta$\ \ : carbon sinks
	\end{itemize}
\end{minipage}
\end{frame}

\addtocounter{framenumber}{-1}

\begin{frame}{ Government}
\hypertarget{gov}{}
\vspace{-4mm}
\centering
\begin{minipage}[t!]{1\textwidth}
	\begin{align*}
		\max_{\{\tau_{Ft}\}_{t=0}^{\infty}, \{\tau_{\iota t}\}_{t=0}^{\infty}}&\hspace{3mm} \sum_{t=0}^{\infty}\beta^t U\left(C_t,H_t, S_t\right)\\ \ \\
		{s.t.} \hspace{4mm}
		&{ (1)\ \ T_t=\tau_{\iota t}w_{t}H_{t}+{{\tau_{Ft}}}F_{t}}+T_{\pi t}\\
		& {(2)\ \  \text{behavior of firms and households}}\\
		& {(3)\ \ \text{feasibility} }\\
		&{(4)\ \  \alert{\omega F_t-\delta \leq\Omega_t }}
	\end{align*}
\end{minipage}

\small
\vspace{0mm}
\hspace{-10mm}
\begin{minipage}[t!]{0.5\textwidth}
	\vspace{7mm}
	\begin{itemize}
		\item[] $\beta$\ \ : household discount factor\vspace{-2mm}
		\item[] $T_\pi$: profits minus subsidies \\ \hspace{5.5mm} from machine producers \vspace{0mm}
	\end{itemize}
\end{minipage}
\begin{minipage}[t!]{0.4\textwidth}
	\vspace{8mm}
	\begin{itemize}
		\item[] $\Omega_{t}$: net emission limit
		\vspace{-2mm}	
		\item[] $\omega$\ : emissions per unit of fossil \vspace{-2mm}
		\item[] $\delta$\ \ : carbon sinks
	\end{itemize}
\end{minipage}

\end{frame}


%%%%%%%%%%%%%%%%%%%%%%%%%%%%%%%
%% Calibration
%%%%%%%%%%%%%%%%%%%%%%%%%%%%%%%
\hypertarget{calback}{}
\section{Calibration}
\begin{frame}{Calibration overview}
\begin{itemize}
	\item
	calibration to the US in 2015-2019
	\item emission limit
	\item model parameters
\end{itemize}
\end{frame}

\begin{frame}{Emission limit}
\vspace{-1mm}
\begin{itemize}
	\item<+-> reductions in global CO$_2$ emissions consistent with climate targets \citep{IPCC2022}:
	\vspace{1mm}
	\begin{itemize}
		\item[-] in 2030s\hspace{16mm}: -50\% relative to 2019 
		\item[-] from 2050s onward: net-zero emissions
		\item[-] remaining carbon budget
	\end{itemize}
	\vspace{2mm}
	\item<+-> \textit{equal-per-capita} distribution of  CO$_2$ emissions among countries
\end{itemize}
\end{frame}

\begin{frame}{Emission limit}
\vspace{-3mm}
\begin{center}
	\begin{figure}
		\centering
		\textbf{Net CO$_2$ emission limit in Gt}\\
		\vspace{2mm}	\includegraphics[width=0.38\textwidth]{../codding_model/own_basedOnFried/optimalPol_010922_revision/figures/all_13Sept22_Tplus30/Emnet.png}
	\end{figure}
\end{center}
\pause

\begin{itemize}
	\item<+-> from 2020 to 2050: ca. $-85\%$ relative to 2019 net CO$_2$ emissions in US
	\vspace{2mm}
	\begin{itemize}
		\item[-] US emissions in 2019 beyond equal-per-capita level
	\end{itemize}
	\vspace{2mm}
	\item<+-> political goal: $-38\%$ by 2030 relative to 2019
	%\\ \hspace{4mm} \small{\ar 5 times equal-per-capita share of remaining carbon budget}
\end{itemize}
\end{frame}

\begin{frame}{Parameters}
\pause
\vspace{-10mm}
\begin{table}[h!]
	\begin{center}
		%		\captionsetup{width=0.3\textwidth}
		%		\caption{ Calibration}
		%		\label{tab:calib}
		\resizebox{4in}{!}{
			\begin{tabular}{l|ll}
				%			\hline \hline
				%			\multicolumn{7}{c}{Calibration based on basic needs}\\
				\hline \hline
				Parameter& Value& \makecell[l]{Target}\\ 
				\hline
				Household&\multicolumn{2}{c}{}\\
				\hline 
				($\sigma$, 	$\sigma_s$) & ($1.33$, $1.33$)&  \makecell[l]{\cite{Chetty2011AreMargins}}  \\
				%$z_h$& \makecell[l]{skill premium 2005-2016:\\ $w_h/w_l=1.9$\\ \citep{Slavik2020WagePremium}}\\	
				($\chi$, $\chi_s$)& (10.02, 0.48) &  \makecell[l]{average hours worked per\\ economic time endowment\\ by worker: 0.34 \citep{OECDHoursworked}} \\
				$\beta$ & 0.93 &  \makecell[l]{\cite{Barrage2019OptimalPolicy}}\\
				$\bar{H}$&1.00& \makecell[l]{14.5 hours per day \citep{Jones1993OptimalGrowth}} \\
				\hline
				Research&\multicolumn{2}{c}{}
				\\
				\hline 
				${\alert{\eta}}$ &0.79 &  \\
				($\rho_F$, $\rho_G$, $\rho_N$)& (0.01, 0.01, 1.00) &\makecell[l]{\cite{Fried2018ClimateAnalysis}}   \\
				${{\phi}}$ &0.50&  \\
				$\gamma$ & 0.06 &\makecell[l]{maximum aggregate growth:\\4\% per annum \citep{OECDGDP}}\\
				\hline
				Production&\multicolumn{2}{c}{}\\
				\hline
				$(\alert{\varepsilon_y, \varepsilon_e})$&(0.05, 1.50)&\cite{Fried2018ClimateAnalysis}\\	
				($\alpha_F$, $\alpha_G$, $\alpha_N$) &(0.72, 0.91, 0.36)&\\
				$\delta_y$&0.38&\makecell[l]{energy expenditure share  \citep{EIAEnergy}}\\
				\hline
				Initial TFP&\multicolumn{2}{c}{}\\
				\hline
				(\alert{${A_{F0}^{1-alpha_f}}$, ${A_{G0}^{1-alpha_g}}$, ${A_{N0}^{1-alpha_n}}$})&(8.21, 1.27, 2.80) &-  \\
				\hline 
				Emissions&\multicolumn{2}{c}{}\\
				\hline
				$\delta$&3.19& \makecell[l]{in GtCO$_2$ \citep{EPAems}}\\
				$\omega$&217.3& \cite{EPAems}\\
				\hline \hline
			\end{tabular}
		}
	\end{center}
\end{table}
\end{frame}


%%%%%%%%%%%%%%%%%%%%%%%%%%%%%%%
%% Results
%%%%%%%%%%%%%%%%%%%%%%%%%%%%%%%
\hypertarget{resback}{}
\section{Results}

\begin{frame}{Optimal Policy}
\begin{figure}[h!!]
	
	\begin{subfigure}{0.4\textwidth}		
		\caption{Tax per ton of carbon,  US\$}
		%	\captionsetup{width=.45\linewidth}
		\includegraphics[width=1\textwidth]{../codding_model/own_basedOnFried/optimalPol_010922_revision/figures/all_13Sept22_Tplus30/Single_periods12_OPT_T_NoTaus_Tauf_regime4_spillover0_knspil0_noskill0_sep0_xgrowth0_extern0_PV1_sizeequ0_GOV0_etaa0.79.png}
	\end{subfigure}	
	\begin{minipage}[]{0.1\textwidth}
		\ 
	\end{minipage}
	\begin{subfigure}{0.4\textwidth}		
		\caption{Average marginal tax rate}
		%	\captionsetup{width=.45\linewidth}
		\includegraphics[width=1\textwidth]{../codding_model/own_basedOnFried/optimalPol_010922_revision/figures/all_13Sept22_Tplus30/dTaulAv_OPT_T_NoTaus_COMPtaul_regime4_spillover0_knspil0_noskill0_sep0_xgrowth0_PV1_etaa0.79_lgd0.png}
	\end{subfigure}
\end{figure}
\vspace{3mm}
\begin{block}{}
	The optimal fiscal mix is a combination of labor and carbon taxes throughout. 
\end{block}	
\end{frame}

\hypertarget{benf}{}
\section*{Mechanism}

\begin{frame}{What is the goal? Comparison to first-best}
\pause
\centering
\begin{figure}[h!!]
	\centering
		\begin{subfigure}{0.4\textwidth}		
		\caption{Green-to-fossil energy ratio}
		%	\captionsetup{width=.45\linewidth}
		\includegraphics[width=1\textwidth]{../codding_model/own_basedOnFried/optimalPol_010922_revision/figures/all_13Sept22_Tplus30/GFF_slides_CompEffOPT_T_NoTaus_regime4_opteff_knspil0_spillover0_noskill0_sep0_xgrowth0_countec0_PV1_etaa0.79_lgd1_lff1.png}
	\end{subfigure}	
	\begin{minipage}[]{0.1\textwidth}
		\ 
	\end{minipage}
	\begin{subfigure}{0.4\textwidth}		
		\caption{Fossil-to-green scientists}
		%	\captionsetup{width=.45\linewidth}
		\includegraphics[width=1\textwidth]{../codding_model/own_basedOnFried/optimalPol_010922_revision/figures/all_13Sept22_Tplus30/sffsg_slides_CompEffOPT_T_NoTaus_regime4_opteff_knspil0_spillover0_noskill0_sep0_xgrowth0_countec0_PV1_etaa0.79_lgd0_lff1.png}
	\end{subfigure}
\end{figure}
\pause
\vspace{3mm}
\begin{block}{}
	\begin{itemize}
		\item<+-> First-best: more fossil research and higher green-to-fossil energy ratio
		\item<+-> Optimal policy: trade-off between fossil research and fossil energy share
	\end{itemize}
\end{block}	
\end{frame}

\begin{frame}{Deviation from carbon-tax-only policy}
\pause
\centering
\begin{figure}
	\begin{subfigure}{0.4\textwidth}
		\caption{\normalsize{Carbon tax, in \%}}
		%	\captionsetup{width=.45\linewidth}
		\includegraphics[width=1\textwidth]{../codding_model/own_basedOnFried/optimalPol_010922_revision/figures/all_13Sept22_Tplus30/Tauf_OPT_T_NoTaus_COMPtaulPer_regime4_spillover0_knspil0_noskill0_sep0_xgrowth0_PV1_etaa0.79.png}
	\end{subfigure}
	\begin{minipage}[]{0.1\textwidth}
		\
	\end{minipage}
	\begin{subfigure}{0.4\textwidth}
		\caption{\normalsize{Fossil-to-green scientists,  in \% }}
		%	\captionsetup{width=.45\linewidth}
		\includegraphics[width=1\textwidth]{../codding_model/own_basedOnFried/optimalPol_010922_revision/figures/all_13Sept22_Tplus30/sffsg_OPT_T_NoTaus_COMPtaulPer_regime4_spillover0_knspil0_noskill0_sep0_xgrowth0_PV1_etaa0.79.png}
	\end{subfigure}
\end{figure}
\vspace{3mm}
\pause
\begin{block}{}
	\begin{itemize}
		\item<+-> In short run, can set a lower carbon tax to raise fossil research.
		\item<+-> In long run, increase carbon tax to boost green research.
	\end{itemize}
\end{block}	
\end{frame}

\begin{frame}{Decomposition}
\pause
\centering

\begin{figure}[h!!]
	\centering
	\begin{subfigure}{0.4\textwidth}		
		\caption{\normalsize{Green-to-fossil energy share}}
		%	\captionsetup{width=.45\linewidth}
		\includegraphics[width=1\textwidth]{../codding_model/own_basedOnFried/optimalPol_010922_revision/figures/all_13Sept22_Tplus30/CountTAUF_CTOPer_Opt_target_GFF_nsk0_xgr0_knspil0_regime4_spillover0_sep0_extern0_PV1_etaa0.79.png}
	\end{subfigure}
	\begin{minipage}[]{0.1\textwidth}
		\
	\end{minipage}
	\begin{subfigure}{0.4\textwidth}		
		\caption{\normalsize{Average hours worked} }
		%	\captionsetup{width=.45\linewidth}
		\includegraphics[width=1\textwidth]{../codding_model/own_basedOnFried/optimalPol_010922_revision/figures/all_13Sept22_Tplus30/CountTAUF_CTOPer_Opt_target_Hagg_nsk0_xgr0_knspil0_regime4_spillover0_sep0_extern0_PV1_etaa0.79.png}
	\end{subfigure}
\end{figure}
\vspace{3mm}
\begin{block}{}
	\begin{itemize}
		\item the carbon tax affects green-to-fossil energy ratio; hours unchanged
		\item[ ]\
	\end{itemize}
\end{block}	
\end{frame}

\addtocounter{framenumber}{-1}
\begin{frame}{Decomposition}
\centering

\begin{figure}[h!!]
	\centering
	\begin{subfigure}{0.4\textwidth}		
		\caption{\normalsize{Green-to-fossil energy share}}
		%	\captionsetup{width=.45\linewidth}
		\includegraphics[width=1\textwidth]{../codding_model/own_basedOnFried/optimalPol_010922_revision/figures/all_13Sept22_Tplus30/CountTAUF_Both_Opt_target_GFF_nsk0_xgr0_knspil0_regime4_spillover0_sep0_extern0_PV1_etaa0.79_lgd1.png}
	\end{subfigure}
	\begin{minipage}[]{0.1\textwidth}
		\
	\end{minipage}
	\begin{subfigure}{0.4\textwidth}		
		\caption{\normalsize{Average hours worked}}
		%	\captionsetup{width=.45\linewidth}
		\includegraphics[width=1\textwidth]{../codding_model/own_basedOnFried/optimalPol_010922_revision/figures/all_13Sept22_Tplus30/CountTAUF_Both_Opt_target_Hagg_nsk0_xgr0_knspil0_regime4_spillover0_sep0_extern0_PV1_etaa0.79_lgd0.png}
	\end{subfigure}
\end{figure}
\vspace{3mm}
\begin{block}{}
	\begin{itemize}
		\item a smaller carbon tax  affects green-to-fossil energy ratio; hours unchanged
		\item the labor tax adjusts hours to (i) reduce emissions or to (ii) raise output
	\end{itemize}
\end{block}	
\end{frame}

\hypertarget{conc}{}
\section{Conclusion}
\begin{frame}{Conclusion}
\begin{itemize}[<+-| alert@+>]
	\setbeamercolor{alerted text}{} %change the font color
	\setbeamerfont{alerted text}{}
	\item I study the optimal mix of taxes on carbon and labor to meet emission targets
	\vspace{3mm}
	\item Under the net-zero emission limit: 
	\begin{itemize}
		\item[-] high carbon tax to reduce fossil demand and research
		\item[-]  a subsidy on labor stabilizes output
	\end{itemize}
	\vspace{3mm}
	\item Before the net-zero emission limit: 
	\begin{itemize}
		\item[-] lower carbon tax to reduce fossil research less
		\item[-] a tax on labor reduces emissions
	\end{itemize}
\end{itemize}
\end{frame}
\begin{frame}[shrink]{References}

\bibliography{../../bib_2_0}
\bibliographystyle{apa}
\end{frame}



%----------------------------------------
%-- appendix
%----------------------------------------
\appendix

%-----------------------------------------
%- Model
%-----------------------------------------
\begin{frame}{Labor supply decision}
\hypertarget{labsup}{}
\begin{align*}
	H_t=\left(\frac{1-\tau_{\iota t}}{\chi}\right)^{\frac{1}{1+\sigma}}
\end{align*}

\vfill
\vspace{0mm}
\hfill 
\hyperlink{hhopt}{\tiny{$\rightarrow$ back}}
\end{frame}


%----------------------------------------
%- Calibration
%----------------------------------------



%----------------------------------------
%- Allocation 
%----------------------------------------

\end{document}